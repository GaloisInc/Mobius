\documentclass{acm_proc_article-sp}

\RequirePackage[T1]{fontenc}
\usepackage{pgf}
\usepackage{tikz}
\usepackage{algorithm}
\usepackage{algorithmic}
\author{Jedrzej Fulara, Krzysztof Jakubczyk}
\title{On Compiling JML Specifications into Bytecode}
\begin{document}
\maketitle


%\section{What should be in paper}
%\begin{itemize}
%	\item What is JML
%	\item What is BML
%	\item Why is translation needed??
% \item Why the tool needed?? (JVM -> VM) BML can be used to different languages, JML to one??
%	\item The tool isn't built on any existing comipler.
%	\item As input we get source file with JML annotations and compiled class file.
%	\item Therefore we can't used optimised bytecode (problem with loops, assertions etc.)
%	\item Optimized bytecode may be used for more general annotations eg. method invariants.
%	\item Detecting loops in bytecode
%	\item Description of matching source code with bytecode loops
%	\item Non-trivial example
%\end{itemize}

\section{Detecting Loops in Bytecode}
To be able to compile the JML loop invariants, one should detect in the bytecode the corresponding loop. The created BML annotation should be associated with the bytecode instruction that represents the loop condition. Note that the loop condition is translated into multiple bytecode instructions. We are interested in the last one (comparison). A loop can be translated in one of the following ways:
\begin{center}
\begin{tikzpicture}[shorten >=1pt,->]
\tikzstyle{vertex}=[circle,fill=blue!25,minimum size=17pt,inner sep=0pt]
\foreach \name/\text/\y in {s/.../1, a/a/2, b/b/3, body/.../4, c/c/5, d/d/6, e/.../7}
\node[vertex] (G-\name) at (0,-\y) {$\text$};
\foreach \from/\to in {s/a,b/body,body/c,c/d,d/e}
\draw (G-\from) -- (G-\to);
\draw (G-a) .. controls +(-30:2cm) and +(30:1cm) .. (G-c);
\draw (G-d) .. controls +(150:2cm) and +(-120:1cm) .. (G-b);

\foreach \name/\text/\y in {s/.../1, a/a/2, b/b/3, body/.../4, c/c/5, d/d/6, e/.../7}
\node[vertex] (Q-\name) at (4,-\y) {$\text$};
\foreach \from/\to in {s/a,a/b,b/body, body/c, d/e}
\draw (Q-\from) -- (Q-\to);
\draw (Q-b) .. controls +(-30:2cm) and +(30:1cm) .. (Q-d);
\draw (Q-c) .. controls +(150:1cm) and +(-120:2cm) .. (Q-a);
\end{tikzpicture}
\end{center}
In the first scenario, in the vertex \textit{a}, an unconditional jump (goto) to the vertex \textit{c} is done (vertex \textit{c} denotes loading the condition). In \textit{d} the condition is checked, and if it is fulfilled, we jump back to \textit{b}. Between \textit{b} and \textit{c} is the loop body. The annotation should be added to the vertex \textit{d}. In the second approach, the condition is tested at the beginning (\textit{a} puts the condition on the stack and \textit{b} checks it. If it is fulfilled, we enter the loop, otherwise we jump out). In \textit{c} an unconditional jump back to \textit{a} is done. The BML annotation should be associated with the instruction in the vertex \textit{a}.

\texttt{Do-while} loops and loops with always true condition (i.e. \texttt{while(true)\{...\}} or \texttt{for(;;)\{...\}} are usually compiled in another way:

\begin{center}
\begin{tikzpicture}[shorten >=1pt,->]
\tikzstyle{vertex}=[circle,fill=blue!25,minimum size=17pt,inner sep=0pt]
\foreach \name/\text/\x in {s/.../1, a/a/2, body1/.../3, b/b/4, body2/.../5, c/c/6, d/.../7}
\node[vertex] (G-\name) at (\x,0) {$\text$};
\foreach \from/\to in {s/a,a/body1, body1/b, b/body2, body2/c}
\draw (G-\from) -- (G-\to);
\draw (G-b) .. controls +(-30:2cm) and +(-150:1cm) .. (G-d);
\draw (G-c) .. controls +(150:2cm) and +(30:1cm) .. (G-a);
\end{tikzpicture}
\end{center}
Before entering the loop (between \textit{a} and \textit{c}), no condition is checked. There might be some \texttt{break} inside (vertex \textit{b}).
In this cases, the annotation should be added to \textit{a} (start of the loop). The Jml2Bml compiler covers all the cases described above. It tries to detect the first kind of loop. If it fails, tries to detect the second one. At the end checks the \texttt{do - while} case. In the first case:
\begin{itemize}
\item{assume that the tested instruction is in vertex \textit{c}. Consider all incoming edges that start in a vertex \textit{v}, which is before \textit{c}}
\item{if there are no such vertices, return null (tested instruction is not the \textit{c} vertex in the first kind loop)}
\item{else take this \textit{v} that has the longest jump to \textit{c} (other jumps come from some continue instructions inside the loop). This is \textit{a} from our graph,}
\item{look at the next instruction. This is our \textit{b}. Find the longest backward jump. It is our vertex \textit{d}}
\item{return d}
\end{itemize}
If no loop of the first kind was detected for an instruction, try to detect the second kind:
\begin{itemize}
\item{assume that the tested instruction is \textit{a}.}
\item{if the instruction has less than two incomming edges - return null}
\item{find \textit{v} that is after \textit{a} and has the longest jump to it. This is \textit{c} from our graph.}
\item{look at the next instruction \textit{d}.}
\item{find such \textit{u} that there exist an edge (\textit{u},\textit{d}) and \textit{u} is between \textit{a} and \textit{d} and there is no such \textit{u'} that \textit{u'} is between \textit{a} and \textit{u} and there is an edge (\textit{u'},\textit{d}). This is candidate for \textit{b}}
\item{if at \textit{u} is an unconditional jump (goto), then this is a break - this is the case of loop with always true condition. Return \textit{a}}
\item{else (at \textit{u} is a conditional jump) - \textit{u} is really our \textit{b}. Return it.}
\end{itemize}
If both cases described above fail, the algorithm tries to detect the \texttt{do - while} loop. We simply check if
\begin{itemize}
\item{there is a backward jump from the tested instruction to some \textit{a}}
\item{if yes, assuming that cases 1 and 2 failed, return \textit{a} as the beginning of the loop}
\end{itemize}

\section{Matching bytecode loops with source loops}
Let's take any source method that has loop with JML invariant and bytecode corresponding to the method. The compiler used to generate the bytecode may have used some optimizations. Unfortunatelly this causes some problems. Here are some example loop optimizations:
\begin{itemize}
	\item loop unwinding (loop unrolling)\\
In this case invariant should be put in every copy of the loop. But the invariant may need a change.
	\item loop interchange\\
If internal loop invariant depended on external loop variables - this invariant must also be translated.
	\item code-motion\\
Some part of code may be moved before the loop. What if invariant depended on it?
\end{itemize}
When we want to add BML specifications to optimized bytecode we have to know the optimizations that were used. There are two solutions to the problem:
\begin{itemize}
	\item Include the JML to BML compiler in existing Java compiler application.
	\item Use non optimizing compiler.
\end{itemize}

\section{Conclusion/Introduction??}
Java Virtual Machine can be used with languages different than Java. Few examples:
\begin{itemize}
	\item Jython - the Python Java implementation
	\item JRuby - the Ruby Java implementation
	\item Jacl - the Tcl Java implementation
	\item Rhino - the JavaScript Java implementation
\end{itemize}
At SugarCon 2008, Sun Microsystems CEO Jonathan Schwartz said that "we're just going to take the 'J' off the 'JVM' and just make it a 'VM'.". Therefore there will be a global trend with support of companies to use JVM with other languages. Java language has JML with proper tools that can be used to verify programs - check their correctness or find errors. Most of them operates on source code. Bytecode itself has a verification algorithm but the verification is done only to ensure that the loaded bytecode will not cause the crash of the JVM. Verification algorithm includes:
\begin{itemize}
	\item checking that all arguments on the operand stack are legal
	\item ensuring that all types of variables passed to methods are correct
	\item checking that all load and store operations have correct types
\end{itemize}
Unfortunatelly programming errors aren't checked. The BML language with proper verification tools can take over all qualities of source code verification using JML. The main advantage of such approach is that the verification itself won't be limited only to Java language - but can be used with any language that is compilled to JVM. There are mainly three sets of tools needed:
\begin{itemize}
	\item bytecode verification tools that use BML annotations
	\item modelling languages (such as JML for Java) for other programming languages
	\item compilers that compile programs to JVM bytecode along with annotation compilers
\end{itemize}
Our Jml2Bml fits in this scheme.

\end{document}
