\documentclass[a4paper]{llncs}

\title{The \modif~clause: semantics, verification and application}

\author{
  N\'estor Cata\~no and Marieke Huisman  \\
  \institute{
       \inria~Sophia-Antipolis, France \\
       \lemme~Project \\
       2004, route des Lucioles, BP 93 
  } 
  \email{\{Nestor.Catano, Marieke.Huisman\}@sophia.inria.fr}
}

\newcommand{\lemme}{\textsc{Lemme}}
\newcommand{\inria}{\textsc{Inria}}
\newcommand{\jml}{\textsc{Jml}}
\newcommand{\escj}{\textsc{Esc/Java}}
\newcommand{\jass}{\textsc{Jass}}
\newcommand{\jcontract}{\textsc{JContract}}
\newcommand{\loopp}{\textsc{Loop}}
\newcommand{\java}{\textsc{Java}}
\newcommand{\gplus}{\textsc{GemPlus}}
\newcommand{\eiff}{\textsc{Eiffel}}
\newcommand{\ctl}{\textsc{Ctl}}
\newcommand{\pltl}{\textsc{Pltl}}
\newcommand{\csrc}{\textsc{Compaq Src}}


\newcommand{\modif}{\textit{modifiable}}
\newcommand{\modifies}{\textit{modifies}}
\newcommand{\sem}[1]{\ensuremath{\mbox{[\![} {#1} \mbox{]\!]}\/}}

\newtheorem {df}{Definition}


\usepackage{amsmath} % it is needed by the \sem command
\message{<Paul Taylor's Proof Trees, 2 August 1996>}
%% Build proof tree for Natural Deduction, Sequent Calculus, etc.
%% WITH SHORTENING OF PROOF RULES!
%% Paul Taylor, begun 10 Oct 1989
%% *** THIS IS ONLY A PRELIMINARY VERSION AND THINGS MAY CHANGE! ***
%%
%% 2 Aug 1996: fixed \mscount and \proofdotnumber
%%
%%      \prooftree
%%              hyp1            produces:
%%              hyp2
%%              hyp3            hyp1    hyp2    hyp3
%%      \justifies              -------------------- rulename
%%              concl                   concl
%%      \thickness=0.08em
%%      \shiftright 2em
%%      \using
%%              rulename
%%      \endprooftree
%%
%% where the hypotheses may be similar structures or just formulae.
%%
%% To get a vertical string of dots instead of the proof rule, do
%%
%%      \prooftree                      which produces:
%%              [hyp]
%%      \using                                  [hyp]
%%              name                              .
%%      \proofdotseparation=1.2ex                 .name
%%      \proofdotnumber=4                         .
%%      \leadsto                                  .
%%              concl                           concl
%%      \endprooftree
%%
%% Within a prooftree, \[ and \] may be used instead of \prooftree and
%% \endprooftree; this is not permitted at the outer level because it
%% conflicts with LaTeX. Also,
%%      \Justifies
%% produces a double line. In LaTeX you can use \begin{prooftree} and
%% \end{prootree} at the outer level (however this will not work for the inner
%% levels, but in any case why would you want to be so verbose?).
%%
%% All of of the keywords except \prooftree and \endprooftree are optional
%% and may appear in any order. They may also be combined in \newcommand's
%% eg "\def\Cut{\using\sf cut\thickness.08em\justifies}" with the abbreviation
%% "\prooftree hyp1 hyp2 \Cut \concl \endprooftree". This is recommended and
%% some standard abbreviations will be found at the end of this file.
%%
%% \thickness specifies the breadth of the rule in any units, although
%% font-relative units such as "ex" or "em" are preferable.
%% It may optionally be followed by "=".
%% \proofrulebreadth=.08em or \setlength\proofrulebreadth{.08em} may also be
%% used either in place of \thickness or globally; the default is 0.04em.
%% \proofdotseparation and \proofdotnumber control the size of the
%% string of dots
%%
%% If proof trees and formulae are mixed, some explicit spacing is needed,
%% but don't put anything to the left of the left-most (or the right of
%% the right-most) hypothesis, or put it in braces, because this will cause
%% the indentation to be lost.
%%
%% By default the conclusion is centered wrt the left-most and right-most
%% immediate hypotheses (not their proofs); \shiftright or \shiftleft moves
%% it relative to this position. (Not sure about this specification or how
%% it should affect spreading of proof tree.)
%
% global assignments to dimensions seem to have the effect of stretching
% diagrams horizontally.
%
%%==========================================================================

\def\introrule{{\cal I}}\def\elimrule{{\cal E}}%%
\def\andintro{\using{\land}\introrule\justifies}%%
\def\impelim{\using{\Rightarrow}\elimrule\justifies}%%
\def\allintro{\using{\forall}\introrule\justifies}%%
\def\allelim{\using{\forall}\elimrule\justifies}%%
\def\falseelim{\using{\bot}\elimrule\justifies}%%
\def\existsintro{\using{\exists}\introrule\justifies}%%

%% #1 is meant to be 1 or 2 for the first or second formula
\def\andelim#1{\using{\land}#1\elimrule\justifies}%%
\def\orintro#1{\using{\lor}#1\introrule\justifies}%%

%% #1 is meant to be a label corresponding to the discharged hypothesis/es
\def\impintro#1{\using{\Rightarrow}\introrule_{#1}\justifies}%%
\def\orelim#1{\using{\lor}\elimrule_{#1}\justifies}%%
\def\existselim#1{\using{\exists}\elimrule_{#1}\justifies}

%%==========================================================================

\newdimen\proofrulebreadth \proofrulebreadth=.05em
\newdimen\proofdotseparation \proofdotseparation=1.25ex
\newdimen\proofrulebaseline \proofrulebaseline=2ex
\newcount\proofdotnumber \proofdotnumber=3
\let\then\relax
\def\hfi{\hskip0pt plus.0001fil}
\mathchardef\squigto="3A3B
%
% flag where we are
\newif\ifinsideprooftree\insideprooftreefalse
\newif\ifonleftofproofrule\onleftofproofrulefalse
\newif\ifproofdots\proofdotsfalse
\newif\ifdoubleproof\doubleprooffalse
\let\wereinproofbit\relax
%
% dimensions and boxes of bits
\newdimen\shortenproofleft
\newdimen\shortenproofright
\newdimen\proofbelowshift
\newbox\proofabove
\newbox\proofbelow
\newbox\proofrulename
%
% miscellaneous commands for setting values
\def\shiftproofbelow{\let\next\relax\afterassignment\setshiftproofbelow\dimen0 }
\def\shiftproofbelowneg{\def\next{\multiply\dimen0 by-1 }%
\afterassignment\setshiftproofbelow\dimen0 }
\def\setshiftproofbelow{\next\proofbelowshift=\dimen0 }
\def\setproofrulebreadth{\proofrulebreadth}

%=============================================================================
\def\prooftree{% NESTED ZERO (\ifonleftofproofrule)
%
% first find out whether we're at the left-hand end of a proof rule
\ifnum  \lastpenalty=1
\then   \unpenalty
\else   \onleftofproofrulefalse
\fi
%
% some space on left (except if we're on left, and no infinity for outermost)
\ifonleftofproofrule
\else   \ifinsideprooftree
        \then   \hskip.5em plus1fil
        \fi
\fi
%
% begin our proof tree environment
\bgroup% NESTED ONE (\proofbelow, \proofrulename, \proofabove,
%               \shortenproofleft, \shortenproofright, \proofrulebreadth)
\setbox\proofbelow=\hbox{}\setbox\proofrulename=\hbox{}%
\let\justifies\proofover\let\leadsto\proofoverdots\let\Justifies\proofoverdbl
\let\using\proofusing\let\[\prooftree
\ifinsideprooftree\let\]\endprooftree\fi
\proofdotsfalse\doubleprooffalse
\let\thickness\setproofrulebreadth
\let\shiftright\shiftproofbelow \let\shift\shiftproofbelow
\let\shiftleft\shiftproofbelowneg
\let\ifwasinsideprooftree\ifinsideprooftree
\insideprooftreetrue
%
% now begin to set the top of the rule (definitions local to it)
\setbox\proofabove=\hbox\bgroup$\displaystyle % NESTED TWO
\let\wereinproofbit\prooftree
%
% these local variables will be copied out:
\shortenproofleft=0pt \shortenproofright=0pt \proofbelowshift=0pt
%
% flags to enable inner proof tree to detect if on left:
\onleftofproofruletrue\penalty1
}

%=============================================================================
% end whatever box and copy crucial values out of it
\def\eproofbit{% NESTED TWO
%
% various hacks applicable to hypothesis list 
\ifx    \wereinproofbit\prooftree
\then   \ifcase \lastpenalty
        \then   \shortenproofright=0pt  % 0: some other object, no indentation
        \or     \unpenalty\hfil         % 1: empty hypotheses, just glue
        \or     \unpenalty\unskip       % 2: just had a tree, remove glue
        \else   \shortenproofright=0pt  % eh?
        \fi
\fi
%
% pass out crucial values from scope
\global\dimen0=\shortenproofleft
\global\dimen1=\shortenproofright
\global\dimen2=\proofrulebreadth
\global\dimen3=\proofbelowshift
\global\dimen4=\proofdotseparation
\global\count255=\proofdotnumber
%
% end the box
$\egroup  % NESTED ONE
%
% restore the values
\shortenproofleft=\dimen0
\shortenproofright=\dimen1
\proofrulebreadth=\dimen2
\proofbelowshift=\dimen3
\proofdotseparation=\dimen4
\proofdotnumber=\count255
}

%=============================================================================
\def\proofover{% NESTED TWO
\eproofbit % NESTED ONE
\setbox\proofbelow=\hbox\bgroup % NESTED TWO
\let\wereinproofbit\proofover
$\displaystyle
}%
%
%=============================================================================
\def\proofoverdbl{% NESTED TWO
\eproofbit % NESTED ONE
\doubleprooftrue
\setbox\proofbelow=\hbox\bgroup % NESTED TWO
\let\wereinproofbit\proofoverdbl
$\displaystyle
}%
%
%=============================================================================
\def\proofoverdots{% NESTED TWO
\eproofbit % NESTED ONE
\proofdotstrue
\setbox\proofbelow=\hbox\bgroup % NESTED TWO
\let\wereinproofbit\proofoverdots
$\displaystyle
}%
%
%=============================================================================
\def\proofusing{% NESTED TWO
\eproofbit % NESTED ONE
\setbox\proofrulename=\hbox\bgroup % NESTED TWO
\let\wereinproofbit\proofusing
\kern0.3em$
}

%=============================================================================
\def\endprooftree{% NESTED TWO
\eproofbit % NESTED ONE
% \dimen0 =     length of proof rule
% \dimen1 =     indentation of conclusion wrt rule
% \dimen2 =     new \shortenproofleft, ie indentation of conclusion
% \dimen3 =     new \shortenproofright, ie
%                space on right of conclusion to end of tree
% \dimen4 =     space on right of conclusion below rule
  \dimen5 =0pt% spread of hypotheses
% \dimen6, \dimen7 = height & depth of rule
%
% length of rule needed by proof above
\dimen0=\wd\proofabove \advance\dimen0-\shortenproofleft
\advance\dimen0-\shortenproofright
%
% amount of spare space below
\dimen1=.5\dimen0 \advance\dimen1-.5\wd\proofbelow
\dimen4=\dimen1
\advance\dimen1\proofbelowshift \advance\dimen4-\proofbelowshift
%
% conclusion sticks out to left of immediate hypotheses
\ifdim  \dimen1<0pt
\then   \advance\shortenproofleft\dimen1
        \advance\dimen0-\dimen1
        \dimen1=0pt
%       now it sticks out to left of tree!
        \ifdim  \shortenproofleft<0pt
        \then   \setbox\proofabove=\hbox{%
                        \kern-\shortenproofleft\unhbox\proofabove}%
                \shortenproofleft=0pt
        \fi
\fi
%
% and to the right
\ifdim  \dimen4<0pt
\then   \advance\shortenproofright\dimen4
        \advance\dimen0-\dimen4
        \dimen4=0pt
\fi
%
% make sure enough space for label
\ifdim  \shortenproofright<\wd\proofrulename
\then   \shortenproofright=\wd\proofrulename
\fi
%
% calculate new indentations
\dimen2=\shortenproofleft \advance\dimen2 by\dimen1
\dimen3=\shortenproofright\advance\dimen3 by\dimen4
%
% make the rule or dots, with name attached
\ifproofdots
\then
        \dimen6=\shortenproofleft \advance\dimen6 .5\dimen0
        \setbox1=\vbox to\proofdotseparation{\vss\hbox{$\cdot$}\vss}%
        \setbox0=\hbox{%
                \advance\dimen6-.5\wd1
                \kern\dimen6
                $\vcenter to\proofdotnumber\proofdotseparation
                        {\leaders\box1\vfill}$%
                \unhbox\proofrulename}%
\else   \dimen6=\fontdimen22\the\textfont2 % height of maths axis
        \dimen7=\dimen6
        \advance\dimen6by.5\proofrulebreadth
        \advance\dimen7by-.5\proofrulebreadth
        \setbox0=\hbox{%
                \kern\shortenproofleft
                \ifdoubleproof
                \then   \hbox to\dimen0{%
                        $\mathsurround0pt\mathord=\mkern-6mu%
                        \cleaders\hbox{$\mkern-2mu=\mkern-2mu$}\hfill
                        \mkern-6mu\mathord=$}%
                \else   \vrule height\dimen6 depth-\dimen7 width\dimen0
                \fi
                \unhbox\proofrulename}%
        \ht0=\dimen6 \dp0=-\dimen7
\fi
%
% set up to centre outermost tree only
\let\doll\relax
\ifwasinsideprooftree
\then   \let\VBOX\vbox
\else   \ifmmode\else$\let\doll=$\fi
        \let\VBOX\vcenter
\fi
% this \vbox or \vcenter is the actual output:
\VBOX   {\baselineskip\proofrulebaseline \lineskip.2ex
        \expandafter\lineskiplimit\ifproofdots0ex\else-0.6ex\fi
        \hbox   spread\dimen5   {\hfi\unhbox\proofabove\hfi}%
        \hbox{\box0}%
        \hbox   {\kern\dimen2 \box\proofbelow}}\doll%
%
% pass new indentations out of scope
\global\dimen2=\dimen2
\global\dimen3=\dimen3
\egroup % NESTED ZERO
\ifonleftofproofrule
\then   \shortenproofleft=\dimen2
\fi
\shortenproofright=\dimen3
%
% some space on right and flag we've just made a tree
\onleftofproofrulefalse
\ifinsideprooftree
\then   \hskip.5em plus 1fil \penalty2
\fi
}

%==========================================================================
% IDEAS
% 1.    Specification of \shiftright and how to spread trees.
% 2.    Spacing command \m which causes 1em+1fil spacing, over-riding
%       exisiting space on sides of trees and not affecting the
%       detection of being on the left or right.
% 3.    Hack using \@currenvir to detect LaTeX environment; have to
%       use \aftergroup to pass \shortenproofleft/right out.
% 4.    (Pie in the sky) detect how much trees can be "tucked in"
% 5.    Discharged hypotheses (diagonal lines).
 %showing demonstration


\begin{document}
\fussy
\maketitle

\begin{abstract}
not yet
%We have worked in \jml, which is a specification language
%tailored to Java. \jml~allows us to specify the functional behavior
%of methods$:$ preconditions and postconditions, and apart from that,
%class invariants. Unfortunately, using only the functional
%specification of a
%method is not enough to reason about arbitrary method
%invocations. Therefore, the so-called \modif~clauses are
%introduced. By asserting a
%method with a \modif~clause, it is stated this method may only modify
%the set of expressions represented by its \modif~clause. There are
%several projects studying \modif~clauses such as \jml~and \escj~(a
%subset of \jml). The \jml~specification language gives a definition
%for the \modif~clause, but when the \jml~tool is run on an asserted
%\java~program, it does not do any checking on its \modif~clause
%specification. In the \loopp~project the
%implementation of some \jml~assertions pragmas are defined, including
%the \modif~ one. This implementation is based on an underlying memory
%model
%and the logic representation of \java~and the \jml~pragmas. This
%implementation is not yet an easy and fast method for its verification
%. This paper presents a first solution to achieve this in a context
%without \textit{aliasing}. For this specification we have made
%an implementation, which extends the functionality of the
%\jml~parser. Our approach presents some advantages over the
%\loopp~project approach, namely$:$ $(i)$ it is faster, and $(ii)$ the
%verification carried out is closer to the specification
%language. Additionally, an electronic banking case study has been used
%to demonstrate the usability of our approach~\cite{CH02}.
\end{abstract}


\section{Introduction}
\label{sec-intro}

\paragraph{\bf Background.}
Some time ago the \emph{Formal methods} were considered as merely
theoretical issues no feasible for real problems. There were not
enough tools supporting them. However, some
events have changed this sight$:$ the specification languages
are now nearer programming languages than before and the techniques used
are more according to the actual programming paradigms (Object
oriented and distributed programming). Other reasons have also
contributed. When the systems increase their complexity and the
faults become more frequent, the need of having software zero-defaults 
become more latent. One example, during $1995$ is found a problem with 
the microprocessors Pentium of Intel. Even though Intel promised to
correct this error, there were already around two millions of processors 
delivered.

Currently, there are several techniques claiming to able to carry out
checking and
verification on software systems. Among these techniques we can find
\textit{testing}, \textit{model-checking}, \textit{theorem
proving} and \textit{runtime} and \textit{static checking}. Testing
is usually enough
for small systems, and it serves for checking some simple properties
before applying another more formal technique, but it does not give
us any certainty that a system provides some features which we are
interested in. The model-checking techniques comprise three
phases~\cite{BBLPS99}. $(i)$ Representing the program or the system
by means of automatons. 
%The whole system is achieved by composition
%and synchronization of each these parts. 
%Then, this whole automaton is
%simplified by using \textit{abstraction
%methods}~\cite{CGL94,DGG97}. 
$(ii)$ Declaring, in a formal way, the properties of this automaton we are
interested in verifying\footnote{Typically by using
\textit{Temporal logic}~\cite{Pnue77}}. 
%\ctl(\textit{Computation Tree
%Logic})~\cite{CE81,EH82} and \pltl(\textit{Proposition Linear
%Temporal Logic})~\cite{Pnue81} are special cases of temporal
%logics.
$(iii)$ Using a Model Checking algorithm for
verifying if the model constructed in $(i)$ satisfies the properties
declared in $(ii)$. Currently, there exist multiple model checkers
developed by universities and industrial teams. These tools provide certain
automatization to the model checking process. Model-checking
constitutes a strong technique
but it has limitations$:$ verification (even for simple systems) might
needs huge amounts of space and time. Theoretically, the
theorem proving techniques are able to solve any problem proposed in
practice, but its real implementation is often a
hard labor$:$ the current tools are only capable in assisting an
engineer in the proving process and they require a good understanding 
of the semantics underlying and the programming language. 
\textit{The runtime} and \textit{static checking} techniques try to
overcome
these problems. One can take a source program and
assert it by using a special specification language fitted to a
language in which source program is written. These assertions express
desirable conditions we would like that a program verifies, and
typically they are stated as
preconditions, postconditions and class invariants. After that, one
can run an assertion checker tool on this asserted program. It is
expected this tool issues a message about the feasibility of desirable
conditions we have asserted
\emph{w.r.t} the source program. Usually, the assertion language is 
very close to the programming language in which the source program is
written. The static checking techniques aim to explicitly check
desirable conditions from the source code. Runtime techniques
convert the asserted program to another one which will be executed. If 
this execution raise certain exception, this will indicate that the
original program does not fulfill the await conditions. 

The annotation language to \eiff~is the first example of such a
specification language, following the \emph{design by contract}
approach~\cite{Mey97}. Currently, several specification languages have
been proposed for \java~ aiming to carry out runtime
checking, such as \jml~\cite{LBR00}, \jcontract~\cite{JContractUrl} and
\jass~\cite{JassUrl}, and static checking such as
\escj~\cite{LeinoNS00}(a subset of \jml~\cite{EscJmlDiff}). 

By
asserting programs on can specify the functional
behavior of the methods, typically as preconditions and
postconditions. However, the functional behavior of the method is
often not enough to reason about arbitrary method
invocations. \escj~
provides an assertion pragma, \modifies, which allows us to
specify the set of expressions which a method is intended to
modify. Even though the \escj~specification language states it should
do static checking on every method body for checking
\modifies~specifications, its tool does not fulfill
this feature. Thus, some verification problems are
present. The \jass~tool come with the pragma \textit{changeonly},
which provides only a basic way to specify what expressions a method
intends to change, typically as variables joined by ``.''.

\paragraph{\bf This paper.}We
have implemented a static checker for \modif~clauses that is neither
sound nor complete (following the \escj~approach), but is useful
for detecting some typical \escj~specification errors such as
forgetting to mention some variable
as \modif, when a method implementation actually may modify it, or
declaring a method without side-effects when it modifies some
expression. We can not always try the most subtle errors because they 
require a detailed analysis, but by using our approach the most
simple errors can be detected quickly. We have defined certain
syntactical rules which define checking conditions for \java~programs
when their changeable behavior is specified by using
\modif~clauses. Moreover, we have done an implementation
of this approach by extending the \jml~parser and stating theses
condition as \java~code. The viability of our approach is
shown by presenting some gaps found in the process of expressing the
changeable behavior of a ``real'' case study~\cite{CH02} and checking
these specifications with our tool.


The rest of this paper is organized as follows$:$
Section~\ref{sec-esc-prg} presents the specification language \escj~
together with its most important assertion
pragmas. Section~\ref{sec-sem-mod-cla} shows the problem
with not to checking \modif~clauses. Section~\ref{sec-giv-sin-def}
presents our approach for solving this problem and the definition of
our checker for \modif~clauses. Section~\ref{sec-imp-iss} presents
some implementation aspects of our tool. Finally, Section~\ref{sec-con}
gives conclusions and presents future work.






\section{Checking with \escj}
\label{sec-esc-prg}

\escj~is a static checker for \java~programs developed
at \csrc~\cite{ESC}, enabling an user to find common programming errors like
\texttt{NullPointerException} and \texttt{ArrayIndexOutOfBoundException}. The basic idea is that
an user specifies the desired behavior of a class and its methods, by
using the specification language owned by \escj, and 
the \escj~tool will check whether the implementation satisfies the given
specification and it will issue a warning 
message if it can not establish the specification from the
implementation (the specification is assumed to be correct). Such a
warning does not necessarily mean that the program is
wrong, since the \escj~approach is neither sound nor complete. The
desirable behavior can be expressed as
\textit{preconditions}, \textit{postconditions} and \textit{class 
invariants} stated as \java~expressions, enriched by several
specification-specific \escj~pragmas. Here, we present
the main \escj~pragmas together with an example of their
use. Their full description can be found in~\cite{LeinoNS00}. 




\paragraph{\bf Pragmas for specifying methods} 
\begin{itemize} 
\item{\texttt{requires P}.} 
This pragma specifies a precondition {\tt P}. 
When \escj~checks the body of a
routine, it assumes that \texttt{P} holds initially and when checking
a method call, \escj~will issue a warning if 
it can not establish that \texttt{P} holds at the call point. 
 
\item{\texttt{ensures Q}.} 
This pragma specifies a postcondition \texttt{Q}. This postcondition
is checked for the body of routine and assumed for any method
invocation if the method terminates normally (without throwing an
exception).
 
\item{\texttt{exsures (E) R}.} 
This pragma specifies a exceptional condition. This condition is 
supposed to hold if the method finishes abruptly, \emph{i.e.}, if
the exception \texttt{e} that is thrown is a subclass of \texttt{E}. 
 
\item{\texttt{modifies L}.} 
This pragma specifies that the routine \emph{may} only modify any of 
the state components listed in \texttt{L}, the formal parameters of 
the method and the variables local to the method. 
\end{itemize}
 
 
 
\paragraph{\bf Specification expressions} 
\begin{itemize} 
\item{\texttt{==>}} is the logical implication. So, \texttt{P 
==> Q} is true if and only if \texttt{P} is false or \texttt{Q} is 
true, where \texttt{P} and \texttt{Q} are specification expressions of  
\texttt{boolean} type. Furthermore, \texttt{<==>} represents the 
equivalence relation and \texttt{<=!=>} specifies non-equivalence. 
 
\item {($\backslash$\texttt{forall T V; E)} and 
($\backslash$\texttt{exists T V; E})} are quantifier expressions (of 
type \texttt{boolean}).  The first denotes that \texttt{E} is true 
for all substitutions of values of type \texttt{T} for the bound 
variable \texttt{V}. The second denotes that \texttt{E} is true 
for a substitution of a value of type \texttt{T} for the bound 
variable \texttt{V}. 
 
\item{\texttt{$\backslash$old($E$)}} is used within a postcondition, where it
denotes the value of \texttt{E} in the pre-state of the method invocation. 
 
\item {\tt$\backslash$result} represents the value returned by 
a non-void method method. It can only be used within an 
\texttt{ensures} clause. 

\item{\texttt{$\backslash$fields\_of(x)}} denotes \texttt{x[*]} if
\texttt{x} is an array, otherwise it denotes the set of fields
belonging to \texttt{x}.

\item{\texttt{$\backslash$reach(x)}} denotes the minimal set
containing \texttt{x}, the fields of \texttt{x} and all object
reachable from the fields of \texttt{x}.
\end{itemize} 
Figure~\ref{fig-add-cur} shows a typical annotation example using 
\escj. This example was taken from the formal specification presented
in~\cite{CH01Url}. The \texttt{addCurrency}
method belongs
to a class \texttt{Currencies}. This class stores all currencies 
supported by the purse application. The method 
\texttt{addCurrency} adds a new currency to the list of valid 
currencies. This list is represented by the 
array \texttt{data}. The \texttt{modifies} clause declared in the 
method's header specifies that 
this method \emph{may} modify \texttt{nbData} and 
\texttt{data} in the position \texttt{nbData}\footnote{More precisely,
it specifies that the method only may modify these instance variables
and the local variables and formal parameters of the method.}. The
postcondition of the method \texttt{addCurrency} (written as 
\texttt{ensure} clauses) expresses that if \texttt{nbData} has not 
yet reached the threshold value \texttt{MAX$\_$DATA}, \texttt{nbData} 
will increase its value by one and the value of the formal parameter 
\texttt{cur} will be assigned to \texttt{data[\(\backslash\)old(nbData)]}.  
Inside the postcondition, the expression 
\texttt{$\backslash$old(nbData)} refers to the value of 
\texttt{nbData} before the method invocation. 
\begin{figure}[htb]
\rule{\linewidth}{0.25mm}
\begin{tabbing}
ddd\=ddd\=ddd\=ddd\= \kill 
\textsl{/*@} \\
\>\textsl{modifies nbData, data[nbData];}  \\
\>\textsl{ensures ($\backslash$old(nbData) $<$ MAX$\_$DATA) $?$}  \\
\>\>\>\textsl{(nbData $==$ $\backslash$old(nbData) + 1 $\&\&$ data$[\backslash$old(nbData)$]$ $==$ cur) $:$}  \\
\>\>\>\textsl{(nbData $==$ $\backslash$old(nbData));}  \\
\textsl{*/} \\
\texttt{void {\bf addCurrency}(byte cur)\{ } \\
\>\texttt{if(nbData $<$ MAX$\_$DATA)\{data[nbData] $=$ cur ; nbData$++$ ;\}}  \\
\texttt{\}}
\end{tabbing}
\caption{Example \escj~specification} 
\label{fig-add-cur} 
\rule{\linewidth}{0.25mm} 
\end{figure} 






\section{Specifying \modif~clauses}
\label{sec-sem-mod-cla}
The checking of \modif~clauses is specially important for not to
accepting wrong specifications. For instead, the
figure~\ref{fig-mod-spe} presents a typical example, where \escj
accepts an incorrect specification. The specification of the method
\texttt{m()} states after a method invocation the length of the
instance variable \texttt{arr} will be greater or equal than $4$. This
property is temporally ensured by the assignment occurred in the
method body, but for definitely ensuring this postcondition, it is
necessary to establish that method \texttt{n()} will not modify the
instance variable \texttt{arr}. What \escj~does, is to see the
modifiable specification of the method \texttt{n()} and to establish
if this method modifies this instance variable. When a method is
asserted without any \modif~clause \escj~takes by default a
\texttt{modifies $\backslash$nothing} specification, but it {\bf does
not} check it. Thus, and concerning this example, \escj~assumes the
method \texttt{n()} does not modify any variable, \emph{i.e.}, it
assumes \texttt{n()} as a method without side-effects. Consequently,
when using this specification in the method \texttt{m()},
\escj~definitely establishes that after a call to \texttt{m()}, its
postcondition will hold. Thus, does not reject the specification of
\texttt{m()}, even though it is incorrect.
\begin{figure}[htb]
\rule{\linewidth}{0.25mm}
\begin{tabbing}
ddd\=ddd\=ddd\=ddd\= \kill 
\texttt{public class C\{} \\
\>\texttt{int[] arr ;} \\
\>\textsl{/*@ modifies arr.length}  \\
\>\>\textsl{ ensures arr.length $>=$ 4 ;*/} \\
\>\texttt{public void {\bf m}()\{} \\
\>\>\texttt{arr $=$ new int[5] ; n() ;} \\
\>\texttt{\}} \\
\\
\>\texttt{public void n()\{arr $=$ new int [3] ;\}} \\
\texttt{\}}
\end{tabbing}
\caption{\modif~specifications} 
\label{fig-mod-spe} 
\rule{\linewidth}{0.25mm} 
\end{figure} 
This problem is caused because \escj~does not check any \modif~clause,
even though the specification of the \escj~language claims to do it. We
have specified and implemented a static checker for \modif~clauses,
which can detect this kind of inconsistencies in \java~asserted
programs. Section~\ref{sec-giv-sin-def} presents its specification
and the Section~\ref{sec-imp-iss} presents some implementation
issues together with the problems found in a ``real'' case study by
using the tool we have developed.


\section{Giving a syntactical definition}
\label{sec-giv-sin-def}
We present the definition of a checker of \modif~clauses. This
checker, like \escj~self, is neither sound nor complete, but is
particularly useful for detection specification errors such as
mentioned in the section~\ref{sec-sem-mod-cla}. This checker is an
efficient tool
(it is not spent time at all) for detecting some frequent
specification
problems in \java~programs before intending a more formal
verification. We have defined a syntactical specification for checking
\modif~clauses in \java~programs with assertions \jml. This
specification takes into account the grammar specification
given in~\cite{LBR00} together with the definition of \modif~clauses
which we present below.
\begin{df}[clause {\tt \bf modifiable}]
\label{def-mod}
A method may modify an expression provided that\footnote{We do not do
any dependency analyze as introduced by \emph{model} pragmas}$:$ 
\begin{itemize}
\item This expression is mentioned in the \modif~clause
\item The expression is a local variable to the method or in a
parameter\footnote{Particularly, it can not modify fields of any
parameter}
\end{itemize}
\end{df}
Basically, our rules check for all assignment expression \textup{e}$_1$\
$\oplus$\ \textup{e}$_2$, occurring inside of a
method, that the variables belonging to \textup{e}$_1$ are
according to definition~\ref{def-mod}, \emph{i.e.}, it checks the
variables in \textup{e}$_1$ are mentioned as modifiable expressions,
or they are local variables or parameters. Any other variable
will
be rejected. For defining these rules, we use the
following definition$:$
\begin{df}[mod]
\textup{S}\ \textit{mod}\ \textsc{Y} $=$ $\forall x \in LC.
x\not\in$ \textsc{Y}$\ \Rightarrow Pre(x) = Post(x)$ \\
where $LC$ is a global set of expressions, $Pre(x)$
refers to the state of $x$ before executing $S$,
$Post(x)$ refers to the state of $x$ after executing
$S$ and \textsc{Y} is the set of variables that can be mentioned
inside a \modif~pragma (see~\cite{LBR00}).
\end{df}
Consequently, a derivation rule such as \\
\begin{prooftree}
\rule[1ex]{0em}{1.5ex}
\textup{e}_1\ \textit{mod}\ \textsc{Y}, \cdots ,\textup{e}_n\ \textit{mod}\ \textsc{Y}
\justifies
\textup{f}_1\textup{(e}_1\textup{, f}_2\textup{(e}_2\textup{,}\cdots
\textup{(f}_n\textup{(e}_n\textup{))))}\ \textit{mod}\ \textsc{Y} 
\end{prooftree}\\ \\
states if some method \texttt{m} specifies it may modify the set
of expressions \textsc{Y}, and in its method body occur expressions or
declarations such as \textup{f}$_1$\textup{(e}$_1$\textup{,
f}$_2$\textup{(e}$_2$\textup{,}$\cdots$
\textup{(f}$_n$\textup{(e}$_n$\textup{))))}, their correctness will be
established in terms of the correctness of each subexpression
\textup{e}$_i$ \emph{w.r.t} \textsc{Y}. We present the
derivation rules for the most important constructs. The whole set of
rules can be found in~\cite{Cat01}. 

\textup{METH-DEC} (see figure~\ref{fig-meth-decl-inv-rul}) is the
general rule corresponding to a method declaration
\texttt{m(}$\overrightarrow{\texttt{o}}$\texttt{)} whose set of
modifiable declarations is \textsc{Y}. This rule states for this
method declaration is enough to
check every expression or declaration in its method body, \emph{i.e.}
\texttt{m(}$\overrightarrow{\texttt{o}}$\texttt{).body}, with respect
to \textsc{Y}, the set of parameters and the local variables to the
method. As it can seen, this rule essentially rephrases the
definition~\ref{def-mod}. Nevertheless, when we are trying with method
calls we have to do some special considerations. For this purpose we
present the following definitions$:$
\begin{df}[$\sqsubseteq$]
\label{def-subseq}
$X \sqsubseteq Y$ ssi $\forall x.\ x\underline{\in}X\Rightarrow
x\underline\in Y$\footnote{The concept $\underline{\in}$ will be
defined later, but intuitively it includes the membership standard
concept $\in$.}.
\end{df}

\begin{df}[$\overrightarrow{mod}$]
\label{def-mod-lis}
$X\ \overrightarrow{mod}\ Y$ ssi $\forall x.\ x\in X\Rightarrow x\ mod\ Y$
\end{df}
We have introduced some other \emph{mod} concepts, by profiting the
\java~syntax. It covers concepts such as \textit{modSTM},
\textit{modEXP}, \textit{modPE}, \textit{modPRM} and \textit{modPS},
which redefine the original concept \textit{mod} in terms of concepts
for \emph{statements}, \emph{expressions}, \emph{post-expressions},
\emph{primary expressions} and \emph{primary suffixes}. When an
instruction become a method invocation
\textup{e}\texttt{.m(}$\overrightarrow{\texttt{q}}$\texttt{)}, it 
must fulfill the \modif~clause specification corresponding to the
method declaration of
\texttt{m(}$\overrightarrow{\texttt{o}}$\texttt{)}, obtained by changing the
formal parameters
$\overrightarrow{\texttt{o}}$ by the real parameters
$\overrightarrow{\texttt{q}}$ and \emph{this} (if any) by the path
\textup{e}. This fact is showed below by the rule
\textup{METH-INV}. Additionally, one must check the formal parameters
of the method by using \emph{modExp} rules
($\overrightarrow{\texttt{q}}\ \overrightarrow{\textit{modEXP}}$) since,
according to the \java~syntax, they can become expressions. A similar
situation is occurred with the path \textup{e}.
\begin{figure}[hbt]
\rule{\linewidth}{0.25mm}
\begin{tabular}{ll}
METH-DEC\, & 
\begin{prooftree}
\rule[1ex]{0em}{1.5ex}
\texttt{m(}\overrightarrow{\texttt{o}}\texttt{).body}\
\overrightarrow{\textit{mod}}\
\textsc{Y}\cup \{\overrightarrow{\texttt{o}}\}\cup \{
\texttt{m(}\overrightarrow{\texttt{o}}\texttt{).locvars}\} 
\justifies
\texttt{m(}\overrightarrow{\texttt{o}}\texttt{)}\ \textit{mod}\ \textsc{Y}
\end{prooftree}
\\[3.0ex]
METH-INV\,\,\, &
\begin{prooftree}
\rule[1ex]{0em}{1.5ex}
\sem{\textup{e}\texttt{.m(}\overrightarrow{\texttt{o}})\texttt{.modifies}}
[\overrightarrow{\texttt{o}}\backslash \overrightarrow{\texttt{q}},
\texttt{this}\backslash \textup{e}]\sqsubseteq \textsc{Y},\
\textup{e}\ \textit{modPE}\ \textsc{Y},\
\overrightarrow{\texttt{q}}\ \overrightarrow{\textit{modEXP}}\
\textsc{Y}
\justifies
\textup{e}\texttt{.m(}\overrightarrow{\texttt{q}}\texttt{)}\ \textit{modPE}\ \textsc{Y}
\end{prooftree}
\\[3.0ex]
\end{tabular}
\caption{Method declaration and invocation rules}
\label{fig-meth-decl-inv-rul}
\rule{\linewidth}{0.25mm}
\end{figure}


According to the rule
\textup{ASSG}, when an assignment \textup{e}$_1$\ $\oplus$\
\textup{e}$_2$ is occurred inside of a method body, we must check
\textup{e}$_1$ belongs to set of expressions the method may modify
(\textup{e}$_1\underline{\in}$\ \textsc{Y}), since \textup{e}$_1$ is
target of an assignment. Additionally, and according to the
\java~syntax, we must check \textup{e}$_1$ and \textup{e}$_2$ by
using respectively the \textit{modPE} and \textit{modEXP} rules.

An instruction in a method body can become a local declaration. So, if
there exist a local declaration with an initial value, the rule
\textup{VAR-DECL-ASSG} states this variable has to belong to the set
of expressions modifiable, \emph{i.e.} \textup{x}$\underline{\in}$\
\textsc{Y}. Additionally, we have to keep checking \textup{e} by using
the rules for expressions. If the
variable is not assigned while is declared, we do not have to do any
membership checking. This is presented by the rule \textup{VAR-DECL}.

Similarly, \java~instructions can become pre and post expressions,
which can be seen as \emph{in place} affectations. As the last rule,
an similar affectation analyze can be done. The rule \textup{POST-PLUS}
presents the case for a post increment expression and it states the
expression
\textup{e}, which is the target of an assignment, has to belong to
set of expressions \textup{Y} modifiable by the method. Furthermore,
we must keep checking \textup{e} by using expression rules. The rest
of the figure~\ref{fig-exp-rul} presents the
rules for the \java~operators. For the arithmetic, relational and
logic operators is applicable the rule \textup{BINARY}. This rules
does not do any membership consideration to some set \textsc{Y},
since no expression is directly modified by these
operators\footnote{Although, eventually a subexpression can modify
some variable}. This analysis is also fitted for the unary,
\texttt{instanceof}, \texttt{()} (\emph{casting}) and $?$
\java~operators.
\begin{figure}[htb] %%%%%%%%%
\rule{\linewidth}{0.25mm}
\begin{tabular}{ll}
ASSG\, & 
\begin{prooftree} 
\rule[1ex]{0em}{1.5ex}
\textup{e}_1\underline{\in}\ \textsc{Y},\ \ \textup{e}_1\
\textit{modPE}\ \textsc{Y},\ \ \textup{e}_2\ \textit{modEXP}\
\textsc{Y}
\justifies
\textup{e}_1\ \oplus\ \textup{e}_2\ \textit{modEXP}\ \textsc{Y}
\using
\oplus \in \{ \texttt{=,+=,-=,*=,/=} \}
\end{prooftree}
\\[3.0ex]
VAR-DECL-ASSG\,\,\, & 
\begin{prooftree}
\rule[1ex]{0em}{1.5ex}
\textup{x}\underline{\in}\ \textsc{Y},\ \ \textup{e}\ \textit{modEXP}\ \textsc{Y}
\justifies
\texttt{T x =}\ \textup{e}\ \textit{modEXP}\ \textsc{Y}
\using
\end{prooftree}
\\[3.0ex]
VAR-DECL & 
\begin{prooftree}
\rule[1ex]{0em}{1.5ex}
\justifies
\texttt{T x}\ \textit{modEXP}\ \textsc{Y}
\using
\end{prooftree}
\\[3.0ex]
POST-PLUS &
\begin{prooftree}
\rule[1ex]{0em}{1.5ex}  
\textup{e}\underline{\in}\ \textsc{Y},\ \ \textup{e}\ \textit{modPE}\ \textsc{Y}
\justifies
\textup{e}\ \texttt{++}\ \textit{modEXP}\ \textsc{Y}
\end{prooftree}
\\[3.0ex]
BINARY & 
\begin{prooftree} 
\rule[1ex]{0em}{1.5ex}
\textup{e}_1\ \textit{modPE}\ \textsc{Y},\ \ \textup{e}_2\
\textit{modEXP}\ \textsc{Y}
\justifies
\textup{e}_1 \oplus \ \textup{e}_2\ \textit{modEXP}\ \textsc{Y}
\using
\oplus \in \{
	\begin{array}{l}
		\texttt{<,<=,>,>=,==}	\\
		\texttt{!=,||,\&\&,+,-}	\\
		\texttt{*,/,\,\&,\^\ ,|}
	\end{array}
	\}
\end{prooftree}
\\[3.0ex]
UNARY & 
\begin{prooftree} 
\rule[1ex]{0em}{1.5ex}
\textup{e}\ \textit{modEXP}\ \textsc{Y}
\justifies
\oplus \ \textup{e}\ \textit{modEXP}\ \textsc{Y}
\using
\oplus \in \{ \texttt{+,-,$\sim$,!}\}
\end{prooftree}
\\[3.0ex]
INSTANCE & 
\begin{prooftree} 
\rule[1ex]{0em}{1.5ex}
\textup{e}\ \textit{modEXP}\ \textsc{Y}
\justifies
\textup{e}\ \texttt{instanceof C}\ \textit{modEXP}\ \textsc{Y}
\end{prooftree}
\\[3.0ex]
CAST & 
\begin{prooftree} 
\rule[1ex]{0em}{1.5ex}
\textup{e}\ \textit{modEXP}\ \textsc{Y}
\justifies
\texttt{(T)}\textup{e}\ \textit{modEXP}\ \textsc{Y}
\end{prooftree}
\\[3.0ex]
CONDITIONAL & 
\begin{prooftree} 
\rule[1ex]{0em}{1.5ex}
\textup{e}_1\ \textit{modEXP}\ \textsc{Y},\ \ \textup{e}_2\
\textit{modEXP}\ \textsc{Y},\ \ \textup{e}3\ \textit{modEXP}\
\textsc{Y}
\justifies
\textup{e}_1?\textup{e}_2:\textup{e}3\ \textit{modEXP}\ \textsc{Y}
\end{prooftree}
\\[3.0ex]
\end{tabular}
\caption{Method declaration and expression rules}
\label{fig-exp-rul}
\rule{\linewidth}{0.25mm}
\end{figure} %%%%%%%%%%
\begin{figure}[htb]%%%%%%%%
\rule{\linewidth}{0.25mm}
\\[2.0ex]
\begin{prooftree}
   \[
     \[
       \[
	 \texttt{false}
         \justifies 
         \texttt{myArr}\ \in\ \phi
	 \using
         \textsc{AX}
       \]
       \justifies 
       \texttt{myArr}\ \underline\in\ \phi
       \using
       \textsc{IN-SIMP}
     \]
     \[
       \[
         \proofdotseparation=1.2ex 
         \proofdotnumber=3
         \leadsto 
       \]
       \justifies 
       \texttt{myArr}\ \textit{modPE}\ \phi
     \]
     \[
       \[
         \proofdotseparation=1.2ex 
         \proofdotnumber=3
         \leadsto 
       \]
       \justifies 
       \texttt{new int[3]}\ \textit{modEXP}\ \phi
     \]
     \justifies
     \texttt{myArr = new int[3]}\ \textit{mod}\ \phi
     \using
     \textsc{ASSG}
   \]
   \justifies
   \texttt{m}\ \textit{mod}\ \phi
   \using
   \textsc{METH-DEC}
\end{prooftree}
\caption{Detecting not modifiable definitions}
\label{fig-der-cor-fig-esc}
\rule{\linewidth}{0.25mm}
\end{figure} %%%%%%%%
\begin{figure}[hbt] %%%%%%%%%
\rule{\linewidth}{0.25mm}
\begin{tabular}{ll}
ID-FLD &
\begin{prooftree}
\rule[1ex]{0em}{1.5ex}
\justifies
\texttt{x}\ \textit{modPRM}\ \textsc{Y}
\end{prooftree}
\\[3.0ex]
ARR-FLD &
\begin{prooftree}
\rule[1ex]{0em}{1.5ex}
\textup{e}\ \textit{modEXP}\ \textsc{Y}
\justifies
\texttt{a[}\textup{e}\texttt{]}\ \textit{modPRM}\ \textsc{Y}
\end{prooftree}
\\[3.0ex]
SUPER & 
\begin{prooftree}
\rule[1ex]{0em}{1.5ex}
\justifies
\texttt{super}\ \textit{modPRM}\ \textsc{Y}
\end{prooftree}
\\[3.0ex]
THIS & 
\begin{prooftree}
\rule[1ex]{0em}{1.5ex}
\justifies
\texttt{this}\ \textit{modPRM}\ \textsc{Y}
\end{prooftree}
\\[3.0ex]
STATIC & 
\begin{prooftree}
\rule[1ex]{0em}{1.5ex}
\justifies
\texttt{A}\ \textit{modPRM}\ \textsc{Y}
\end{prooftree}
\\[3.0ex]
CONST & 
\begin{prooftree}
\rule[1ex]{0em}{1.5ex}
\justifies
\texttt{b}\ \textit{modPRM}\ \textsc{Y}
\using 
\texttt{b}\in \textsf{Constant} 
\end{prooftree}
\\[3.0ex]
METH &
\begin{prooftree}
\rule[1ex]{0em}{1.5ex}
\sem{\texttt{this.m(}\overrightarrow{\texttt{o}}\texttt{).modifies}}[\overrightarrow{\texttt{o}}\ \backslash
\overrightarrow{\texttt{q}}]\sqsubseteq \textsc{Y},\ \ \overrightarrow{\texttt{q}}\ \overrightarrow{\textit{modEXP}}\ \textsc{Y}
\justifies
\texttt{m(}\overrightarrow{\texttt{q}}\texttt{)}\ \textit{modPRM}\ \textsc{Y}
\end{prooftree}
\\[3.0ex]
NEW-EXP & 
\begin{prooftree}
\rule[1ex]{0em}{1.5ex}
\overrightarrow{\textup{e}}\ \overrightarrow{\textit{modEXP}}\ \textsc{Y}
\justifies
\texttt{new T(}\overrightarrow{\textup{e}}\texttt{)}\ \textit{modPRM}\ \textsc{Y}
\end{prooftree}
\\[3.0ex]
NEW-ARR & 
\begin{prooftree}
\rule[1ex]{0em}{1.5ex}
\textup{e}\ \textit{modEXP}\ \textsc{Y}
\justifies
\texttt{new T[}{\textup{e}}\texttt{]}\ \textit{modPRM}\ \textsc{Y}
\end{prooftree}
\\[3.0ex]
\end{tabular}
\caption{Primary expressions}
\label{fig-pri-exp}
\rule{\linewidth}{0.25mm}
\end{figure}
\begin{figure}[hbt]
\rule{\linewidth}{0.25mm}
\begin{tabular}{ll}
SEQ & 
\begin{prooftree}
\rule[1ex]{0em}{1.5ex}
\textsc{S}\ \textit{modSTM}\ \textsc{Y},\ \ \textsc{T}\ \textit{modSTM}\ \textsc{Y}
\justifies
\textsc{S}\texttt{;}\textsc{T}\ \textit{modSTM}\ \textsc{Y}
\end{prooftree} 
\\[3.0ex]
IF-THEN & 
\begin{prooftree}
\rule[1ex]{0em}{1.5ex}
\textsc{C}\ \textit{modEXP}\ \textsc{Y},\ \ \textsc{S}\ \textit{modSTM}\ \textsc{Y}
\justifies
\texttt{if(}\textsc{C}\texttt{) }\textsc{S}\ \textit{modSTM}\ \textsc{Y}
\end{prooftree}
\\[3.0ex]
IF-THEN-ELSE & 
\begin{prooftree}
\rule[1ex]{0em}{1.5ex}
\textsc{C}\ \textit{modEXP}\ \textsc{Y},\ \ \textsc{S}\ \textit{modSTM}\ \textsc{Y},\ \ \textsc{T}\ \textit{modSTM}\ \textsc{Y}
\justifies
\texttt{if(}\textsc{C}\texttt{)}\ \textsc{S}\ \texttt{else}\ \textsc{T}\ \textit{modSTM}\ Y
\end{prooftree}
\\[3.0ex]
WHILE & 
\begin{prooftree}
\rule[1ex]{0em}{1.5ex}
\textsc{C}\ \textit{modEXP}\ \textsc{Y},\ \ \textsc{B}\ \textit{modSTM}\ \textsc{Y}
\justifies
\texttt{while(}\textsc{C}\texttt{)}{\tt \{}\textsc{B}{\tt \}}
\end{prooftree}
\\[3.0ex]
SKIP &  
\begin{prooftree}
\rule[1ex]{0em}{1.5ex}
\justifies
\texttt{skip}\ \textit{modSTM}\ \textsc{Y}
\end{prooftree}
\\[3.0ex]
BREAK & 
\begin{prooftree}
\rule[1ex]{0em}{1.5ex}
\justifies
\texttt{break}\ \textit{modSTM}\ \textsc{Y}
\end{prooftree}
\\[3.0ex]
BREAK-LBL &
\begin{prooftree}
\rule[1ex]{0em}{1.5ex} 
\justifies
\texttt{break}\ \textup{label}\ \textit{modSTM}\ \textsc{Y}
\end{prooftree}
\\[3.0ex]
CONTINUE & 
\begin{prooftree} 
\rule[1ex]{0em}{1.5ex}
\justifies
\texttt{continue}\ \textit{modSTM}\ \textsc{Y}
\end{prooftree}
\\[3.0ex]
CONTINUE-LBL\,\,\, & 
\begin{prooftree} 
\rule[1ex]{0em}{1.5ex}
\justifies
\texttt{continue}\ \textup{label}\ \textit{modSTM}\ \textsc{Y}
\end{prooftree}
\\[3.0ex]
RETURN & 
\begin{prooftree} 
\rule[1ex]{0em}{1.5ex}
\justifies
\texttt{return}\ \textit{modSTM}\ \textsc{Y}
\end{prooftree}
\\[3.0ex]
RETURN-EXP & 
\begin{prooftree} 
\rule[1ex]{0em}{1.5ex}
\textup{e}\ \textit{modEXP}\ \textsc{Y}
\justifies
\texttt{return}\ \textup{e}\ \textit{modSTM}\ \textsc{Y}
\end{prooftree}
\\[3.0ex]
\end{tabular}
\caption{Rules concerning Statements}
\label{fig-rul-con-sta}
\rule{\linewidth}{0.25mm}
\end{figure}%%%%%%%

\begin{figure}[hbt]%%%%%%%%%%
\rule{\linewidth}{0.25mm}
\begin{tabular}{ll}
IN-SIMP &
\begin{prooftree}
\rule[1ex]{0em}{1.5ex}
\texttt{this.x}\in \textsc{Y}\vee \texttt{x}\in \textsc{Y}
\justifies
\texttt{this.x}\underline{\in} \textsc{Y}
\end{prooftree}
\\[3.0ex]
IN-THIS &
\begin{prooftree}
\rule[1ex]{0em}{1.5ex}
\backslash \texttt{fields}$\_$\texttt{of(}\texttt{this}\textsc{)}\in \textsc{Y}
\justifies
\texttt{this.x}\underline{\in} \textsc{Y}
\end{prooftree}
\\[3.0ex]
IN-REACH &
\begin{prooftree}
\rule[1ex]{0em}{1.5ex}
\backslash
\texttt{fields}$\_$\texttt{of(}\backslash\texttt{reach(}\textup{e}\texttt{))}\in
\textsc{Y},\ \ \texttt{this}\in \backslash
\texttt{reach(}\textup{e}\texttt{)}
\justifies
\texttt{this.x}\underline{\in} \textsc{Y}
\end{prooftree}
\\[3.0ex]
IN-ARR &
\begin{prooftree}
\rule[1ex]{0em}{1.5ex}
\texttt{this.a[}\textup{e}\texttt{]}\in Y\vee \texttt{a[}\textup{e}\texttt{]}\in \textsc{Y}
\justifies
\texttt{this.a[}\textup{e}\texttt{]}\underline{\in} \textsc{Y}
\end{prooftree}
\\[3.0ex]
IN-ARR-FLD &
\begin{prooftree}
\rule[1ex]{0em}{1.5ex}
\backslash \texttt{fields}\_\texttt{of(a)}\in \textsc{Y}
\justifies
\texttt{this.a[}\textup{e}\texttt{]}\underline{\in} \textsc{Y}
\end{prooftree}
\\[3.0ex]
IN-ARR-REACH &
\begin{prooftree}
\rule[1ex]{0em}{1.5ex}
\backslash \texttt{fields}\_\texttt{of(}\backslash
\texttt{reach(}\textup{e}_1\texttt{))}\in \textsc{Y},\ \
\texttt{this.a}\in \backslash \texttt{reach(}\textup{e}\texttt{)}
\justifies
\texttt{this.a[}\textup{e}\texttt{]}\underline{\in} Y
\end{prooftree}
\\[3.0ex]
IN-EXP &
\begin{prooftree}
\rule[1ex]{0em}{1.5ex}
\textup{e}\texttt{.x}\in \textsc{Y}
\justifies
\textup{e}\texttt{.x}\underline{\in} \textsc{Y}
\end{prooftree}
\\[3.0ex]
IN-EXP-FLD &
\begin{prooftree}
\rule[1ex]{0em}{1.5ex}
\backslash \texttt{fields}\_\texttt{of(}\textup{e}\texttt{)}\in \textsc{Y}
\justifies
\textup{e}\texttt{.x}\underline{\in} \textsc{Y}
\end{prooftree}
\\[3.0ex]
IN-EXP-REACH &
\begin{prooftree}
\rule[1ex]{0em}{1.5ex}
\backslash \texttt{fields}\_\texttt{of(}\backslash
\texttt{reach(}\textup{e}_1\textsc{))}\in Y,\ \
\textup{e}\texttt{.x}\in \backslash
\texttt{reach(}\textup{e}_1\texttt{)}
\justifies
\textup{e}\texttt{.x}\underline{\in} \textsc{Y}
\end{prooftree}
\\[3.0ex]
IN-EXP-ARR &
\begin{prooftree}
\rule[1ex]{0em}{1.5ex}
\textup{e}_2\texttt{.a[}\textup{e}_1\texttt{]}\in \textsc{Y}
\justifies
\textup{e}_2\texttt{.a[}\textup{e}_1\texttt{]}\underline{\in} \textsc{Y}
\end{prooftree}
\\[3.0ex]
IN-EXP-ARR-FLD &
\begin{prooftree}
\rule[1ex]{0em}{1.5ex}
\backslash \texttt{fields}\_\texttt{of(}\textup{e}_2\texttt{.a)}\in \textsc{Y}
\justifies
\textup{e}_2\texttt{.a[}\textup{e}_1\texttt{]}\underline{\in} \textsc{Y}
\end{prooftree}
\\[3.0ex]
IN-EXP-ARR-REACH\,\,\, &
\begin{prooftree}
\rule[1ex]{0em}{1.5ex}
\backslash \texttt{fields}\_\texttt{of(}\backslash
\texttt{reach(}\textup{e}\texttt{))}\in \textsc{Y},\ \
\textup{e}_2\texttt{.a}\in \backslash
\texttt{reach(}\textup{e}\texttt{)}
\justifies
\textup{e}_2\texttt{.a[}\textup{e}_1\texttt{]}\underline{\in} \textsc{Y}
\end{prooftree}
\\[3.0ex]
%INTERV-ARR & 
%\begin{prooftree}
%\rule[1ex]{0em}{1.5ex}
%\texttt{a[}i..j\texttt{]}\in \textsc{Y},\ i\leq \textup{e}\leq j,\ \
%\textup{e}\ \textit{modEXP}\ \textsc{Y}
%\justifies
%\texttt{this.a[}\textup{e}\texttt{]}\ \textit{modPRM}\ \textsc{Y}
%\end{prooftree}
%\\[3.0ex]
%TIMES-ARR &
%\begin{prooftree}
%\rule[1ex]{0em}{1.5ex}
%\texttt{a[*]}\in \textsc{Y},\ \ \textup{e}\ \textit{modEXP}\ \textsc{Y}
%\justifies
%\texttt{this.a[}\textup{e}\texttt{]}\ \textit{modPRM}\ \textsc{Y}
%\end{prooftree}
%\\[3.0ex]
\end{tabular}
\caption{Memeberhsip definition}
\label{fig-mem-def}
\rule{\linewidth}{0.25mm}
\end{figure}%%%%%%%%%
The rules defined until now enable us to detect the kind of problems
presented in the section~\ref{sec-sem-mod-cla} (see
figure~\ref{fig-mod-spe}). Thus, by applying the rule
\textup{METH-DEC}, the checking condition for the method \texttt{n()}
will become $\mathtt{\{myArr\ =\ new\ int[3]\}}$ \textit{mod}
$\phi$. Then,
by applying the rule \textup{ASSG} we would should be able to
deduce $myArr\underline\in\ \phi$. What is not possible. Thus, the
checking
of the method \texttt{n()} will fail. Figure~\ref{fig-der-cor-fig-esc}
presents the derivation corresponding this problem.

Figure~\ref{fig-pri-exp} shows rules corresponding to
primary and suffix \java~expressions. A primary \java~expression
represents the expression parameter of an postfix(or prefix)
expression
or the \emph{pre-path} in a sequence of expressions joined by the
symbol \texttt{``.''}, the 
last expression will be considered as a suffix expression. The rules
\textup{ID-FLD}, \textup{SUPER}, \textup{THIS}, \textup{STATIC},
\textup{CONST} does not do any membership consideration to some set,
since they do not change any variable. The rule \textup{ARR-FLD} have
to check the expression \textup{e} by using \textit{modEXP}
rules. The rule \textup{METH} expresses the fact a method invocation
can become a primary expression. This rule is similar to
\textup{METH-INV}, which was presented before. When creating a class
or an array (see rules \textup{NEW-EXP} and \textup{NEW-ARR}), the
expressions given as parameters must checked by using \textit{modEXP}
rules. Only the first fourth rules defined in this figure are applicable
to primary suffixes. %%%%%%%%

The \java~instructions can also become statements.  Their checking
rules are showed in the figure~\ref{fig-rul-con-sta}. The rule
\textup{SEQ} presents the composition of instructions. It states the
composition of instructions \textup{S} and \textup{T} may modify
\textsc{Y} if \textup{S} may modify \textsc{Y} and \textup{T} may
modify \textsc{Y}. The rules \textup{SKIP}, \textup{BREAK},
\textup{BREAK-LBL}, \textup{CONTINUE} and \textup{CONTINUE-LBL}
present the analyze for the \java~instructions \texttt{skip},
\texttt{break}, \texttt{break} \emph{lbl}, \texttt{continue} and
\texttt{continue} \emph{lbl} respectively. The rule for
\textup{IF-THEN}
states the guard of the conditional must be checked by using
\textit{modEXP} rules since, according to the \java~syntax, a guard is
an expression. Its body \textsc{S} can become a set of statements (an
expression can be seen as a statement), thus it must be checked by
using rules for statements. A similar analyze can be formulated for the
rules \textup{IF-THEN-ELSE} and \textup{WHILE}. %%%%%

According to the \jml~specification
\texttt{$\backslash$fields}$\_$\texttt{of} and
\texttt{$\backslash$reach} pragmas can occur inside of a
\modifies~pragma. Consequently, for establishing membership conditions
\textup{e}$_1$\texttt{.}\textup{e}$_2$$\underline\in$ \textsc{Y}, we
also need
to define rules taking into account the fact of these pragmas can
occur inside of \textsc{Y}. Figure~\ref{fig-mem-def} shows these
rules. Typically, if any of these pragmas occur in the modifiable
clause we fall on the standard definition of membership (see rules
\textup{IN-SIMP},  \textup{IN-ARR}, \textup{IN-EXP},
\textup{IN-EXP-ARR}). If an expression
like
$\backslash$\texttt{fields}$\_$\texttt{of(}\textup{e}\texttt{)} is
occured in \textsc{Y}, we
have to consider some cases according to the shape of \textup{e}. If
\textup{e} has the form \textup{e}$_1$\texttt{.}\textup{e}$_2$ and
\textup{e}$_2$ is not an array expression, it will be enough to have
the expression \texttt{fields$\_$of(}\textup{e}$_1$\texttt{)} occuring 
in \textsc{Y} for 
establishing the original condition of membership (see rules
\textup{IN-THIS} and \textup{IN-EXP-FLD}). Otherwise, if
\textup{e}$_2$ is an array expression
\texttt{a[}\textup{e}$_3$\texttt{]}, 
it will be enough to have
\texttt{fields$\_$of(}\textup{e}$_1$\texttt{.a}\texttt{)} occurring in
\textsc{Y} for establishing the original condition of membership
(see rules \textup{IN-ARR-FLD} and
\textup{IN-EXP-ARR-FLD}). Similarly, $\backslash$\texttt{reach}
expressions can occur in \modif~clause\footnote{According to the
\jml~syntax a
$\backslash$\texttt{reach} expression only can occur inside of a
$\backslash$\texttt{fields}$\_$\texttt{of} expression.}. So, if
$\backslash$\texttt{fields}$\_$\texttt{of($\backslash$reach(}\textup{e}$\_1$\texttt{))}
is found in \textsc{Y}, and we like to establish
\textup{e}\texttt{.x}$\underline\in$\textsc{Y}, it will be enough to
find an expression \textup{e}$_1$ from which
\textup{e}\texttt{.x}$\underline\in$\textsc{Y} is reachable. This
situation is
presented by the rule \textup{IN-EXP-REACH}. Similar situations are
happened when \texttt{x} is any other expression. This is presented by
the rules \textup{IN-REACH} and \textup{IN-ARR-REACH}.


\section{Implementation issues}
\label{sec-imp-iss}
dfdf
\section{Conclusion}
\label{sec-con}
dfdf
\bibliographystyle{alpha}
\bibliography{../bibliography}

\end{document}
