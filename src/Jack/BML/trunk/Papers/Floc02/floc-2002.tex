\documentclass[a4paper]{llncs}

\title{A static checker for JML's \emph{assignable} clause}

\author{
  N\'estor Cata\~no and Marieke Huisman  \\
  \institute{
       \inria\ Sophia-Antipolis, France \\
       \lemme~Project
  } 
  \email{\{Nestor.Catano, Marieke.Huisman\}@sophia.inria.fr}
}

\newcommand{\defn}[1]{\:\hat{#1}\:}

\newcommand{\lemme}{\textsc{Lemme}}
\newcommand{\inria}{\textsc{Inria}}
\newcommand{\jml}{\textsc{Jml}}
\newcommand{\escj}{\textsc{Esc/Java}}
\newcommand{\jass}{\textsc{Jass}}
\newcommand{\jcontract}{\textsc{JContract}}
\newcommand{\loopp}{\textsc{Loop}}
\newcommand{\java}{\textsc{Java}}
\newcommand{\cPP}{\texttt{C/C}\nolinebreak\hspace{-.05em}\raisebox{.4ex}{\tiny\bf
+}\nolinebreak\hspace{-.10em}\raisebox{.4ex}{\tiny\bf +}}
\newcommand{\gplus}{\textsc{GemPlus}}
\newcommand{\eiff}{\textsc{Eiffel}}
\newcommand{\ctl}{\textsc{Ctl}}
\newcommand{\pltl}{\textsc{Pltl}}
\newcommand{\csrc}{\textsc{Compaq Src}}
\newcommand{\insPP}{\textsc{Insure\nolinebreak\hspace{-.05em}\raisebox{.4ex}{\tiny\bf
+}\nolinebreak\hspace{-.10em}\raisebox{.4ex}{\tiny\bf +}}}

\newcommand{\modtool}{\texttt{assignable:tool}}

\newcommand{\sem}[1]{\ensuremath{\mbox{[\![} {#1} \mbox{]\!]}\/}}

\newtheorem {df}{Definition}

\newcommand{\FullExp}{\(<\)\textsf{Full-exp}\(>\)}
\newcommand{\SufExp}{\(<\)\textsf{Suf-exp}\(>\)}
\newcommand{\Exp}{\(<\)\textsf{Exp}\(>\)}
\newcommand{\Identifier}{\(<\)\textsf{Identifier}\(>\)}
\newcommand{\MethodCall}{\(<\)\textsf{Identifier}\(>\)(\Exp\ldots\Exp)}
\newcommand{\Static}{\(<\)\textsf{StaticClass}\(>\)}
\newcommand{\Type}{\(<\)\textsf{Type}\(>\)}
\newcommand{\option}{\(\:||\)}

\newcommand{\MOD}[2]{\ensuremath{\mathit{#1}\:\mathsf{mod}\:\ensuremath{\mathit{#2}}}}
\newcommand{\MODFE}[2]{\ensuremath{\mathit{#1}\:\mathsf{modFE}\:\mathit{#2}}}
\newcommand{\MODSuf}[2]{\ensuremath{\mathit{#1}\:\mathsf{modSuf}\:\mathit{#2}}}
\newcommand{\MODS}[2]{\ensuremath{\mathit{#1}\:\overrightarrow{\mathsf{mod}}\:\mathit{#2}}}

\newcommand{\method}[2]{\ensuremath{\mathtt{#1}\texttt{(}
                                    \overrightarrow{\mathtt{#2}}
                                    \texttt{)}}}
\newcommand{\methodbody}[2]{\ensuremath{\method{#1}{#2}\mathtt{.body}}}
\newcommand{\methodloc}[2]{\ensuremath{\method{#1}{#2}\mathtt{.locvars}}}
\newcommand{\methodassign}[2]{\ensuremath{\method{#1}{#2}\mathtt{.assignable}}}


\newcommand{\extmember}[2]{\ensuremath{#1\: \underline\in\: #2}}
\newcommand{\member}[2]{\ensuremath{#1\: \in\: #2}}
\newcommand{\extsubset}[2]{\ensuremath{#1 \sqsubseteq #2}}

\newcommand{\fieldsof}{\texttt{\(\backslash\)fields\_of}}
\newcommand{\fieldsofarg}[1]{\texttt{\(\backslash\)fields\_of(#1)}}
\newcommand{\reach}{\texttt{\(\backslash\)reach}}
\newcommand{\reacharg}[1]{\texttt{\(\backslash\)reach(#1)}}
\newcommand{\nothing}{\texttt{\(\backslash\)nothing}}
\newcommand{\everything}{\texttt{\(\backslash\)everything}}

\usepackage{lscape} % it's needed to turn a page
\usepackage{amsmath} % it is needed by the \sem command
\usepackage{alltt}
\message{<Paul Taylor's Proof Trees, 2 August 1996>}
%% Build proof tree for Natural Deduction, Sequent Calculus, etc.
%% WITH SHORTENING OF PROOF RULES!
%% Paul Taylor, begun 10 Oct 1989
%% *** THIS IS ONLY A PRELIMINARY VERSION AND THINGS MAY CHANGE! ***
%%
%% 2 Aug 1996: fixed \mscount and \proofdotnumber
%%
%%      \prooftree
%%              hyp1            produces:
%%              hyp2
%%              hyp3            hyp1    hyp2    hyp3
%%      \justifies              -------------------- rulename
%%              concl                   concl
%%      \thickness=0.08em
%%      \shiftright 2em
%%      \using
%%              rulename
%%      \endprooftree
%%
%% where the hypotheses may be similar structures or just formulae.
%%
%% To get a vertical string of dots instead of the proof rule, do
%%
%%      \prooftree                      which produces:
%%              [hyp]
%%      \using                                  [hyp]
%%              name                              .
%%      \proofdotseparation=1.2ex                 .name
%%      \proofdotnumber=4                         .
%%      \leadsto                                  .
%%              concl                           concl
%%      \endprooftree
%%
%% Within a prooftree, \[ and \] may be used instead of \prooftree and
%% \endprooftree; this is not permitted at the outer level because it
%% conflicts with LaTeX. Also,
%%      \Justifies
%% produces a double line. In LaTeX you can use \begin{prooftree} and
%% \end{prootree} at the outer level (however this will not work for the inner
%% levels, but in any case why would you want to be so verbose?).
%%
%% All of of the keywords except \prooftree and \endprooftree are optional
%% and may appear in any order. They may also be combined in \newcommand's
%% eg "\def\Cut{\using\sf cut\thickness.08em\justifies}" with the abbreviation
%% "\prooftree hyp1 hyp2 \Cut \concl \endprooftree". This is recommended and
%% some standard abbreviations will be found at the end of this file.
%%
%% \thickness specifies the breadth of the rule in any units, although
%% font-relative units such as "ex" or "em" are preferable.
%% It may optionally be followed by "=".
%% \proofrulebreadth=.08em or \setlength\proofrulebreadth{.08em} may also be
%% used either in place of \thickness or globally; the default is 0.04em.
%% \proofdotseparation and \proofdotnumber control the size of the
%% string of dots
%%
%% If proof trees and formulae are mixed, some explicit spacing is needed,
%% but don't put anything to the left of the left-most (or the right of
%% the right-most) hypothesis, or put it in braces, because this will cause
%% the indentation to be lost.
%%
%% By default the conclusion is centered wrt the left-most and right-most
%% immediate hypotheses (not their proofs); \shiftright or \shiftleft moves
%% it relative to this position. (Not sure about this specification or how
%% it should affect spreading of proof tree.)
%
% global assignments to dimensions seem to have the effect of stretching
% diagrams horizontally.
%
%%==========================================================================

\def\introrule{{\cal I}}\def\elimrule{{\cal E}}%%
\def\andintro{\using{\land}\introrule\justifies}%%
\def\impelim{\using{\Rightarrow}\elimrule\justifies}%%
\def\allintro{\using{\forall}\introrule\justifies}%%
\def\allelim{\using{\forall}\elimrule\justifies}%%
\def\falseelim{\using{\bot}\elimrule\justifies}%%
\def\existsintro{\using{\exists}\introrule\justifies}%%

%% #1 is meant to be 1 or 2 for the first or second formula
\def\andelim#1{\using{\land}#1\elimrule\justifies}%%
\def\orintro#1{\using{\lor}#1\introrule\justifies}%%

%% #1 is meant to be a label corresponding to the discharged hypothesis/es
\def\impintro#1{\using{\Rightarrow}\introrule_{#1}\justifies}%%
\def\orelim#1{\using{\lor}\elimrule_{#1}\justifies}%%
\def\existselim#1{\using{\exists}\elimrule_{#1}\justifies}

%%==========================================================================

\newdimen\proofrulebreadth \proofrulebreadth=.05em
\newdimen\proofdotseparation \proofdotseparation=1.25ex
\newdimen\proofrulebaseline \proofrulebaseline=2ex
\newcount\proofdotnumber \proofdotnumber=3
\let\then\relax
\def\hfi{\hskip0pt plus.0001fil}
\mathchardef\squigto="3A3B
%
% flag where we are
\newif\ifinsideprooftree\insideprooftreefalse
\newif\ifonleftofproofrule\onleftofproofrulefalse
\newif\ifproofdots\proofdotsfalse
\newif\ifdoubleproof\doubleprooffalse
\let\wereinproofbit\relax
%
% dimensions and boxes of bits
\newdimen\shortenproofleft
\newdimen\shortenproofright
\newdimen\proofbelowshift
\newbox\proofabove
\newbox\proofbelow
\newbox\proofrulename
%
% miscellaneous commands for setting values
\def\shiftproofbelow{\let\next\relax\afterassignment\setshiftproofbelow\dimen0 }
\def\shiftproofbelowneg{\def\next{\multiply\dimen0 by-1 }%
\afterassignment\setshiftproofbelow\dimen0 }
\def\setshiftproofbelow{\next\proofbelowshift=\dimen0 }
\def\setproofrulebreadth{\proofrulebreadth}

%=============================================================================
\def\prooftree{% NESTED ZERO (\ifonleftofproofrule)
%
% first find out whether we're at the left-hand end of a proof rule
\ifnum  \lastpenalty=1
\then   \unpenalty
\else   \onleftofproofrulefalse
\fi
%
% some space on left (except if we're on left, and no infinity for outermost)
\ifonleftofproofrule
\else   \ifinsideprooftree
        \then   \hskip.5em plus1fil
        \fi
\fi
%
% begin our proof tree environment
\bgroup% NESTED ONE (\proofbelow, \proofrulename, \proofabove,
%               \shortenproofleft, \shortenproofright, \proofrulebreadth)
\setbox\proofbelow=\hbox{}\setbox\proofrulename=\hbox{}%
\let\justifies\proofover\let\leadsto\proofoverdots\let\Justifies\proofoverdbl
\let\using\proofusing\let\[\prooftree
\ifinsideprooftree\let\]\endprooftree\fi
\proofdotsfalse\doubleprooffalse
\let\thickness\setproofrulebreadth
\let\shiftright\shiftproofbelow \let\shift\shiftproofbelow
\let\shiftleft\shiftproofbelowneg
\let\ifwasinsideprooftree\ifinsideprooftree
\insideprooftreetrue
%
% now begin to set the top of the rule (definitions local to it)
\setbox\proofabove=\hbox\bgroup$\displaystyle % NESTED TWO
\let\wereinproofbit\prooftree
%
% these local variables will be copied out:
\shortenproofleft=0pt \shortenproofright=0pt \proofbelowshift=0pt
%
% flags to enable inner proof tree to detect if on left:
\onleftofproofruletrue\penalty1
}

%=============================================================================
% end whatever box and copy crucial values out of it
\def\eproofbit{% NESTED TWO
%
% various hacks applicable to hypothesis list 
\ifx    \wereinproofbit\prooftree
\then   \ifcase \lastpenalty
        \then   \shortenproofright=0pt  % 0: some other object, no indentation
        \or     \unpenalty\hfil         % 1: empty hypotheses, just glue
        \or     \unpenalty\unskip       % 2: just had a tree, remove glue
        \else   \shortenproofright=0pt  % eh?
        \fi
\fi
%
% pass out crucial values from scope
\global\dimen0=\shortenproofleft
\global\dimen1=\shortenproofright
\global\dimen2=\proofrulebreadth
\global\dimen3=\proofbelowshift
\global\dimen4=\proofdotseparation
\global\count255=\proofdotnumber
%
% end the box
$\egroup  % NESTED ONE
%
% restore the values
\shortenproofleft=\dimen0
\shortenproofright=\dimen1
\proofrulebreadth=\dimen2
\proofbelowshift=\dimen3
\proofdotseparation=\dimen4
\proofdotnumber=\count255
}

%=============================================================================
\def\proofover{% NESTED TWO
\eproofbit % NESTED ONE
\setbox\proofbelow=\hbox\bgroup % NESTED TWO
\let\wereinproofbit\proofover
$\displaystyle
}%
%
%=============================================================================
\def\proofoverdbl{% NESTED TWO
\eproofbit % NESTED ONE
\doubleprooftrue
\setbox\proofbelow=\hbox\bgroup % NESTED TWO
\let\wereinproofbit\proofoverdbl
$\displaystyle
}%
%
%=============================================================================
\def\proofoverdots{% NESTED TWO
\eproofbit % NESTED ONE
\proofdotstrue
\setbox\proofbelow=\hbox\bgroup % NESTED TWO
\let\wereinproofbit\proofoverdots
$\displaystyle
}%
%
%=============================================================================
\def\proofusing{% NESTED TWO
\eproofbit % NESTED ONE
\setbox\proofrulename=\hbox\bgroup % NESTED TWO
\let\wereinproofbit\proofusing
\kern0.3em$
}

%=============================================================================
\def\endprooftree{% NESTED TWO
\eproofbit % NESTED ONE
% \dimen0 =     length of proof rule
% \dimen1 =     indentation of conclusion wrt rule
% \dimen2 =     new \shortenproofleft, ie indentation of conclusion
% \dimen3 =     new \shortenproofright, ie
%                space on right of conclusion to end of tree
% \dimen4 =     space on right of conclusion below rule
  \dimen5 =0pt% spread of hypotheses
% \dimen6, \dimen7 = height & depth of rule
%
% length of rule needed by proof above
\dimen0=\wd\proofabove \advance\dimen0-\shortenproofleft
\advance\dimen0-\shortenproofright
%
% amount of spare space below
\dimen1=.5\dimen0 \advance\dimen1-.5\wd\proofbelow
\dimen4=\dimen1
\advance\dimen1\proofbelowshift \advance\dimen4-\proofbelowshift
%
% conclusion sticks out to left of immediate hypotheses
\ifdim  \dimen1<0pt
\then   \advance\shortenproofleft\dimen1
        \advance\dimen0-\dimen1
        \dimen1=0pt
%       now it sticks out to left of tree!
        \ifdim  \shortenproofleft<0pt
        \then   \setbox\proofabove=\hbox{%
                        \kern-\shortenproofleft\unhbox\proofabove}%
                \shortenproofleft=0pt
        \fi
\fi
%
% and to the right
\ifdim  \dimen4<0pt
\then   \advance\shortenproofright\dimen4
        \advance\dimen0-\dimen4
        \dimen4=0pt
\fi
%
% make sure enough space for label
\ifdim  \shortenproofright<\wd\proofrulename
\then   \shortenproofright=\wd\proofrulename
\fi
%
% calculate new indentations
\dimen2=\shortenproofleft \advance\dimen2 by\dimen1
\dimen3=\shortenproofright\advance\dimen3 by\dimen4
%
% make the rule or dots, with name attached
\ifproofdots
\then
        \dimen6=\shortenproofleft \advance\dimen6 .5\dimen0
        \setbox1=\vbox to\proofdotseparation{\vss\hbox{$\cdot$}\vss}%
        \setbox0=\hbox{%
                \advance\dimen6-.5\wd1
                \kern\dimen6
                $\vcenter to\proofdotnumber\proofdotseparation
                        {\leaders\box1\vfill}$%
                \unhbox\proofrulename}%
\else   \dimen6=\fontdimen22\the\textfont2 % height of maths axis
        \dimen7=\dimen6
        \advance\dimen6by.5\proofrulebreadth
        \advance\dimen7by-.5\proofrulebreadth
        \setbox0=\hbox{%
                \kern\shortenproofleft
                \ifdoubleproof
                \then   \hbox to\dimen0{%
                        $\mathsurround0pt\mathord=\mkern-6mu%
                        \cleaders\hbox{$\mkern-2mu=\mkern-2mu$}\hfill
                        \mkern-6mu\mathord=$}%
                \else   \vrule height\dimen6 depth-\dimen7 width\dimen0
                \fi
                \unhbox\proofrulename}%
        \ht0=\dimen6 \dp0=-\dimen7
\fi
%
% set up to centre outermost tree only
\let\doll\relax
\ifwasinsideprooftree
\then   \let\VBOX\vbox
\else   \ifmmode\else$\let\doll=$\fi
        \let\VBOX\vcenter
\fi
% this \vbox or \vcenter is the actual output:
\VBOX   {\baselineskip\proofrulebaseline \lineskip.2ex
        \expandafter\lineskiplimit\ifproofdots0ex\else-0.6ex\fi
        \hbox   spread\dimen5   {\hfi\unhbox\proofabove\hfi}%
        \hbox{\box0}%
        \hbox   {\kern\dimen2 \box\proofbelow}}\doll%
%
% pass new indentations out of scope
\global\dimen2=\dimen2
\global\dimen3=\dimen3
\egroup % NESTED ZERO
\ifonleftofproofrule
\then   \shortenproofleft=\dimen2
\fi
\shortenproofright=\dimen3
%
% some space on right and flag we've just made a tree
\onleftofproofrulefalse
\ifinsideprooftree
\then   \hskip.5em plus 1fil \penalty2
\fi
}

%==========================================================================
% IDEAS
% 1.    Specification of \shiftright and how to spread trees.
% 2.    Spacing command \m which causes 1em+1fil spacing, over-riding
%       exisiting space on sides of trees and not affecting the
%       detection of being on the left or right.
% 3.    Hack using \@currenvir to detect LaTeX environment; have to
%       use \aftergroup to pass \shortenproofleft/right out.
% 4.    (Pie in the sky) detect how much trees can be "tucked in"
% 5.    Discharged hypotheses (diagonal lines).
 %showing demonstration

\begin{document}
\fussy
\maketitle
\pagestyle{plain}

\begin{abstract}
This paper proposes a syntactic method to check so-called
\emph{assignable} clauses in \java\ programs, annotated with \jml\
assertions. Assignable clauses describe which variables may be
modified by a method. Their correctness is crucial for reasoning about 
class specifications. 

The method that we propose in this paper is incomplete, as it only
does a syntactic check and it does not take aliasing or expression evaluation
into account, but it provides an efficient check that will find most
common specification errors in assignable clauses in practice. This is 
demonstrated by applying the method to the specification of an
industrial case study.


%The method is a
%syntactic one

%This paper presents a method for checking the specification of
%\emph{assignable} clauses in \java\ programs with assertions \jml. We
%propose syntactical rules for each type of \java\ instruction. These
%rules check for each \java\ instruction occurring in a method
%declaration that locations assigned by the instruction belong to set
%of variable locations the method has specified as
%\emph{assignable}. Hence, our specification does not concern with
%modification but the assignment. We show the feasibility of our
%approach presenting several specifications gaps found in the case
%study \emph{Formal Specification and Static Checking of \gplus's
%Electronic Purse Using \textsc{Esc/Java}}~\cite{CatanoH02a}.
\end{abstract}






\section{Introduction}
\label{sec-intro}

\paragraph{\bf Formal methods}
For a long time, formal methods have been considered as a merely
theoretical issue in computer science, which was not suitable to apply
to ``real'' problems: the complicated logical notations and the lack
of tool support made that formal methods could not compete with the
software development tools and methods used in industry. However,
recently this has started to change, for several reasons. First of
all, formal methods have become more applicable: specification
languages have been developed which resemble programming languages,
and together with these specification languages powerful tools have
been developed. At the same time, industry has started to feel the
need for certified software, in particular after the discovery of bugs such as
the one in the \textsc{Intel} \textsc{Pentium II} microprocessors,
which costed the company several billion dollars.


%Several decades ago, the \emph{formal methods} were considered as a merely
%theoretical issue in computer science, not suitable for real
%problems: \emph{complicated} notations taken from the logic and
%the lack of tools supporting formal methods could not compete with the
%tools and methods used in industry for software development. Recently, 
%this has been changing. There are several reasons for this
%change. Firstly, the specification languages are now closer to
%programming languages than ever before and the techniques used are
%more suitable to the actual programming paradigms (object oriented
%and distributed programming). Secondly, industry has started to feel
%the need for certified software. An
%example~\cite{Borger99programmer}: during $1995$,
%\textsc{Intel} found a problem with its microprocessors
%\textsc{Pentium}. This company was forced to replace around two
%million processors. Thus, it realized that the costs of formal
%verification would have been much lower than the costs required to
%fix this bug.

Traditionally, formal methods are concerned with the verification of
programs, using \textit{model checking} or 
\textit{theorem proving} techniques. Drawbacks of both these
techniques is that they require a good inside in the particular
verification technique and tool (\emph{e.g.}~to construct the model
that is input for the model checker or to formalise the programming
language in the logic of the theorem prover), and that it is often time
and memory consuming. By providing a front-end to the verification
tool, which translates the program (and possibly also its
specification) automatically into the input language for the tool,
this first drawback partially can be overcome. But, when the program
does not satisfy the specification, one still has to trace the problem 
in the language of the verification tool.


%Currently, with the interest of the industry and the academic world,
%the developments in formal methods have become
%increased. These developments are specially concerned with techniques
%such as
%\textit{model-checking} and \textit{theorem proving}.
%Model-checking is technique based on constructing a finite model of a
%problem and exhaustively checking the satisfaction of certain
%properties. Model-checking is
%a strong technique, as it is \emph{decidable}, but it presents some
%limitations: $(i)$\ it can validate a
%model of a certain system but not the system itself, and
%$(ii)$\ verification even for simple systems might need huge amounts
%of space and time. 

%Theoretically, interactive theorem proving techniques are able to
%solve any problem proposed in practice, but it presents some
%disadvantages: $(i)$\ it is \emph{undecidable}, $(ii)$\ the tools are
%only capable of assisting an engineer in the process of constructing a
%proof, and $(iii)$\ it requires expressing and understanding the
%semantics underlying the
%programming language in the logic of the theorem prover, in order to
%do program verification. 
%Testing checks whether a program produces
%correct output on certain input. This technique is important for
%checking some properties before applying a formal technique. A
%disadvantage of this approach is that one spends too much of time in the
%process of constructing all possible testing cases. Moreover,
%sometimes it is not possible to check all possible
%inputs. 

To make the use of formal methods more efficient, some more
lightweigth approaches such as
\textit{run-time checking} and
\textit{static checking} have been developed. These techniques allow 
to do quick checks on a program specification in order to find the
most common errors. Using these techniques, one does not get a full
correctness proof, but one can increase the trust in the correctness
of the implementation \emph{w.r.t.}~the specification.

Run-time checking is a technique where a program is transformed into a 
program which tests several safety conditions at run-time,
\emph{e.g.}~a program can be transformed so that every time a method 
is called, it is tested whether the precondition of this method is
satisfied at the calling point. Typically, if such a condition is
violated the transformed program will throw an exception, indicating
which condition was violated at which program point. The specification
language \jml\ that is in this paper ships with a run-time assertion
checker, which can perform such program transformations. Run-time
checking can be seen as extended testing: not only the input-output
behaviour is tested, but also the internal state of the program at
dedicated points. An important advantage of run-time checking is that
one does not have to change to another formalism, all the work is done
at the level of the programming language.

Other examples of run-time checkers are \insPP~\cite{InsurePP}, which
checks several memory operations for \cPP programs, and
\jass~\cite{bartetzko01assertions} which allows to check certain
assertions for \java\ programs. A special feature of \jass\ are the
so-called trace assertions, which allow to check that methods are
invoked in a prescribed order.

%. This checker keeps track of a \cPP\ program and
%complains whether a block of memory is liberated too many times, a memory
%block is used without being initialized or there exist memory leaks.

%Recently, several approaches have been
%proposed to overcome these kind of problems and although they can not
%actually replace formal verification techniques, they constitute
%important alternatives for checking the behavior of software
%systems. \emph{Run-time
%checking} is a kind of extended checking where, in addition to checking
%the input-output behavior, it is checked whether certain specific
%conditions are
%satisfied at particular points in the program. For checking these
%conditions, the original program is expressed into another
%one
%which can be then executed. If the execution throws an exception,
%this indicates that the original program does not fulfill the
%expected conditions.


Complementary to run-time checking is so-called static checking, where 
one tries to find problems in a program by using static analysis
techniques. Static checking can be considered as an extended form of
type checking, by using static checking many potential run-time
problems already can be found at compile-time.
Within this paper, we develop a static checking method to check
so-called assignable clauses, \emph{i.e.}~the part of the
specification which states which variables may be changed by a method.
Typically, static checking gives quick feedback on a program and can
find many common errors, so that more expensive formal verification
techniques can concentrate on the essential parts of the program.


% check the behavior of programs
%from its source code. This technique can be complemented with
%an automatic theorem prover which automatically can detect
%some specific problems. It is expected not to expend too much time in
%this process.

%When using static and run-time checking techniques on software systems,
%it is possible to state
%some specifications which we are interested to check. These
%specifications can be provided to the checking tools as part of the
%source code and written by using a suitable
%specific-specification language. 

\paragraph{\bf{Formal methods for \java}}
Recently, much work on (light-weight) formal methods has focused on
\java\ and \textsc{Javacard}. Several specification languages, such as
\jml~\cite{LeavensBR00}~and \jass~\cite{bartetzko01assertions},
have been developed which are especially tailored to
\java. \jml\ (\java\ modeling language) is developed by Leavens \emph{et 
al.}~at Iowa State University. As mentioned above, together with the
specification language a run-time assertion checker is
developed. \jml\ follows the \emph{Design by Contract}
approach~\cite{Meyer97} as advocated in \eiff. It allows to state pre
and post conditions of methods as well as class invariants. The
predicates are boolean \java\ expressions, with some
specification-specific constructs. \jml\ is designed to be easy to
understand for a \java\ programmer.


%proposed
%for \java\ and \cPP. \jml\ \ (\java\ Model
%Language)~\cite{LeavensBR00} is a tool developed at the Iowa State
%University
%doing run-time checking. Its specification language follows the
%\eiff's approach of \emph{Design by
%Contract}~\cite{Meyer97}, hence
%it is possible to state the functional behavior of methods as
%pre- and postconditions as well as class
%invariants. For
%constructing predicates, in addition to the \java\ boolean operators, 
%\jml\ relies on several logical operators such as the equivalence,
%\texttt{<==>}, and the implication \texttt{==>}, and several
%quantifiers as the existential,
%\texttt{$\backslash$exists}, and the universal, \texttt{$\backslash$forall}.
%Additionally, in \jml\  it is possible to use method
%invocations inside the predicates provide that
%those do not have \emph{side effects}.


%\jass\\  (\java\ With Assertions)~\cite{bartetzko01assertions} is a Design
%by Contract extension for \java\ which allows to annotate
%\java\ programs. The \jass\ tool~\cite{JassUrl} is developed at the
%University of Oldenburg by the group Semantics. \jass\ supports the
%concept of \emph{trace assertions} which are used to monitor the
%dynamic behavior of objects in run-time. This language provides a
%construct called \texttt{changeonly} to specify the list of
%attributes than \emph{may} be modified for a method invocation. This
%list of attributes just consist of identifiers separated by ``.''.

%\escj~\cite{ESCJavaUrl} is a specification language developed at
%\csrc, conceived as a subset of the \jml\ specification
%language~\cite{EscJmlDiff}. The \escj\ tool advocates for doing static
%checking of annotated \java\ programs. Additionally it relies on an
%automatic theorem prover for checking some specific problems. These
%problems cover \texttt{ArrayIndexOutOfBoundExceptions},
%\texttt{NullPointerExceptions} and \emph{Deadlock} and \emph{Race}
%problems in multi-thread \java\ programs.

\jml\ (and variations) is used as input language for several
tools. Apart from the \jml\ tool mentioned above, it is also used as
input for \emph{e.g.}~the \textsc{Loop}~\cite{LoopURL} tool, which translates
\jml-annotated \java\ programs into a series of input theories for the 
theorem prover \textsc{pvs}~\cite{?}, and for \escj~\cite{ESCJavaUrl}, 
which is a static checker for \java. \escj\ allows the user to
annotate the program and then tries to check the specification by
using an automatic, dedicated theorem prover. It is especially tuned
to find possible errors such as array index out of bounds errors, null 
pointer errors, deadlocks and race conditions, but it also can check
arbitrary specifications. If it cannot establish a certain
specification, it issues a warning. As \escj\ neither is sound nor
complete, a warning does not necessarily mean that the program is incorrect.



\paragraph{\bf{Static checking of assignable clauses}}
When checking method specifications for object-oriented programs,
verification should be done in a modular way. When one encounters a
method call in the method that one is checking, one does not know
which method actually will be called~--~due to dynamic
binding. Following the behavioural subtype approach~\cite{LiskovW?},
the method specification corresponding with the static type of the
receiver of the method call will be used. Additionally, for each
overriding method in a subclass, one has to show that it satisfies the 
specification of the superclass. 

When reasoning in such a way, pre and post conditions of methods alone 
are not sufficient, one also needs to know which variables may have
been changed by a method. Consider for example the following annotated 
code.

%<<<<<<< floc-2002.tex
%=======
%Recently, several approaches have been
%proposed to overcome these kind of problems and although they can not
%actually replace formal verification techniques, they constitute
%important alternatives for checking the behavior of software
%systems. \emph{Run-time
%checking} is a kind of extended checking where, in addition to checking
%the input-output behavior, it is checked whether certain specific
%conditions are
%satisfied at particular points in the program. For checking these
%conditions, the original program is expressed into another
%one
%which can be then executed. If the execution throws an exception,
%this indicates that the original program does not fulfill the
%expected conditions.

%\emph{Static checking} techniques check the behavior of programs
%from their source code. This technique can be complemented with
%an automatic theorem prover which automatically can detect
%some specific problems. It is expected not to expend too much time in
%this process.

%When using static and run-time checking techniques on software systems,
%it is possible to state
%some specifications which we are interested to check. These
%specifications can be provided to the checking tools as part of the
%source code and written by using a suitable
%specific-specification language. 

%Currently, several specification languages have been proposed
%for \java~and \cPP. \jml~\ (\java~Model
%Language)~\cite{LeavensBR00} is a tool developed at the Iowa State
%University
%doing run-time checking. Its specification language follows the
%\eiff's approach of \emph{Design by
%Contract}~\cite{Meyer97}, hence
%it is possible to state the functional behavior of methods as
%pre- and postconditions as well as class
%invariants. For
%constructing predicates, in addition to the \java~boolean operators, 
%\jml~relies on several logical operators such as the equivalence,
%\texttt{<==>}, and the implication \texttt{==>}, and several
%quantifiers as the existential,
%\texttt{$\backslash$exists}, and the universal, \texttt{$\backslash$forall}.
%Additionally, in \jml~ it is possible to use method
%invocations inside the predicates provide that
%those do not have \emph{side effects}.

%\insPP~is a automatic run-time error detector for
%\cPP~\cite{InsurePP}. This checker keeps track of a \cPP~program and
%complains whether a block of memory is liberated too many times, a memory
%block is used without being initialized or there exist memory leaks.

%\jass\~ (\java~With Assertions)~\cite{bartetzko01assertions} is a Design
%by Contract extension for \java~which allows to annotate
%\java~programs. The \jass~tool~\cite{JassUrl} is developed at the
%University of Oldenburg by the group Semantics. \jass~supports the
%concept of \emph{trace assertions} which are used to monitor the
%dynamic behavior of objects in run-time. This language provides a
%construct called \texttt{changeonly} to specify the list of
%attributes than \emph{may} be modified for a method invocation. This
%list of attributes just consist of identifiers separated by ``.''.

%\escj~\cite{ESCJavaUrl} is a specification language developed at
%\csrc, conceived as a subset of the \jml~specification
%language~\cite{EscJmlDiff}. The \escj~tool advocates for doing static
%checking of annotated \java~programs. Additionally it relies on an
%automatic theorem prover for checking some specific problems. These
%problems cover \texttt{ArrayIndexOutOfBoundExceptions},
%\texttt{NullPointerExceptions} and \emph{Deadlock} and \emph{Race}
%problems in multi-thread \java~programs. \escj~neither is sound nor
%complete, hence whether it issues a warning it does not necessary
%means that some problem occurs. \escj~ships with a specific construct,
%\texttt{modifies}, for specifying the \emph{assignable} behavior of a
%method, but unlike the \jass~tool the variable locations one can
%annotate as assignable can become intricate. Thus, it is possible,
%for instance, to state as assignable all elements of some array
%(\texttt{a[$*$]}) or the fields of some object
%expression $e$ (\texttt{fields\_of($e$)}). \escj~tool does not check
%the correct specification of the assignable behavior of a method,
%hence some verifications problems arise.
%%The annotation language for \eiff~is the first
%%example of such a specification language, following the \emph{design
%%by contract} approach~\cite{Meyer97}. 



%\paragraph{\bf The side-effects freeness problem.}
%The functional behavior as can be expressed by pre- and postconditions
%is not enough to reason about arbitrary method invocations. Suppose we 
%have the example presented below. 
%>>>>>>> 1.16
\begin{alltt}
public class C\verb!{!
  int[] arr;
  /*@
    ensures arr.length >= 4;
   */
  public void m()\verb!{!
    arr = new int[5]; n();
  \verb!}!
  /*@
    ensures true;
   */
  public void n()\verb!{}!
\verb!}!
\end{alltt}
To establish whether \texttt{m} will satisfy its postcondition, the
functional specification of \texttt{n} alone does not give enough
information, one also needs to know whether \texttt{n} might change
the variable \texttt{arr}. Therefore, \jml\ and \escj\ allow the user
to specify a so-called assignable clause,  using the
keyword \texttt{modifies}\footnote{\jml\ also allows the alternative
keywords \texttt{assignable} and \texttt{modifiable}}, specifying a
set of locations (typically variables) that may be modified by a
method. An assignable clause can contain variable names,
but also more  complicated expressions, \emph{e.g.}~to denote all the
elements in an array. \jml\ even allows more general expressions to
occur in the assignable clause.

However, as the information that is given in the modifies clause is
crucial in the verification of other methods, it is important that the
modifies clause is correct. An incorrect modifies clause can cause
incorrect method specifications to be accepted, which can then cause
other incorrect method specifications to be accepted, and so
on... \escj\ does not check assignable clauses, it only uses them when
checking method calls. But when specifying a real-life application it
is very easy to forget to mention that a certain variable may be
changed~--~see for example our experiences with the specification and
static checking of an industrial case study: Gemplus' electronic
purse~\cite{CatanoH02a}. Therefore, it is important to have a tool
which checks assignable clauses, even when it is only on a syntactic
basis, without taking aliasing into account. Such a tool is presented
in this paper.

The tool works completely on a syntactic basis. It analyses the
program code, and for every assignment (and syntactic variation of
this) it checks whether this does not possibly violate the specified
modifies clause. Similarly, for every method call it encounters, it
checks whether this method call does not violate the modifies clause
of the calling method. Currently, the tool does not work with model
variables (specification only variables), but these could be handled
in a similar way.

The tool is neither sound nor complete, as it does not handle
aliasing, and it does not keep track of variable updates. For example,
for a program fragment \texttt{i++; a[i] = 3;} the tool will accept a
modifies clause \texttt{modifies a[i];}. However, we feel that in
practice this is not such a heavy limitation. When specifying a method
in which variables are modifies in a complicated or tricky way, a
specifier will be very carefull when writing the modifies clause,
while for simple methods he is more likely to forget to mention some
variables. Moreover, one can always fall back on complete verification
techniques, such as advocated in the \textsc{Loop} project.

%The functional behavior of \texttt{n} is not enough to \emph{ensure}
%the post-condition of \texttt{m}, \emph{i.e.}, to ensure that the
%length of \texttt{arr} still will be greater or equal than $4$ after
%the invocation of \texttt{n}. Hence, it arises the need of specifying
%the \emph{changeable} behavior of methods,\emph{i.e}, of specifying what
%is set of variables locations a certain method modifies.



%\escj\ ships with a specific construct,
%\texttt{modifies}, 


%for specifying the \emph{assignable} behavior of a
%method, but unlike the \jass\ tool the variable locations one can
%annotate as assignable can become intricate. Thus, it is possible,
%for instance, to state as assignable all elements of some array
%(\texttt{a[$*$]}) or the fields of some object
%expression $e$ (\texttt{fields\_of($e$)}). \escj\ tool does not check
%the correct specification of the assignable behavior of a method,
%hence some verifications problems arise.


%\paragraph{\bf This paper.}We propose a static checker for the correct
%specification of the
%\emph{assignable} (as a subset of \emph{modifiable}) behavior of
%\jml\ annotated programs. For this purpose, we define a set
%of rules for each \java\ construct which state the variable locations
%these instructions \emph{may} modify. This checker carries out an
%syntactical analysis of the source program and checks whether its
%methods declarations fulfill their assignable specifications. By using
%these rules one can detect errors such as
%declaring a method without \emph{side-effects} when it actually modifies 
%some variable location. Our rules are neither sound nor complete
%(following the \escj\ approach), as we can not detect the most subtle
%errors since they require a detailed analysis, but using them the
%most simple errors can be detected quickly. We have developed the tool
%\modtool\ which implements these rules, by using the Abstract syntax
%tree generated by the parser of \jml. The feasibility of our approach is
%shown by presenting some gaps found in the process of checking the
%specification of the
%assignable behavior of a realistic case study~\cite{CatanoH02a}.

\paragraph{\bf{Related work}}


\paragraph{\bf{Organisation}}

The rest of this paper is organized as follows:
Section~\ref{sec-esc-prg} briefly presents the the most important
assertion constructs of of \jml\ and \escj. Then Section~\ref{sec-ass} 
discusses the meaning of the assignable clause in more detail.
Section~\ref{sec-syn-met-che-ass-cla}
presents the rules that we use to check assignable clauses, and
Section~\ref{sec-che-for-ass-cla} discusses how we have implemented
this on top of the \jml\ parser. Section~\ref{sec-example} then
discusses how we used the tool to improve the specification of the
electronic purse case study~\cite{CatanoH02a}. Finally,
Section~\ref{sec-con-and-fut-wor} gives conclusions and presents
future work.






\section{JML and ESC/Java specifications}
\label{sec-esc-prg}
%\escj\ is a static checker for \java\ programs developed
%at \csrc~\cite{ESCJavaUrl}, allowing a user to find common programming
%errors, such as nullpointer exceptions, array index out of bound
%exceptions, deadlocks and race conditions. It does this by generating
%appropriate proof obligations and sending these off to an automatic,
%dedicated theorem prover. A user can specify additional class and
%method specifications (\emph{e.g.}~class invariants, pre and post
%conditions), which the tool will also try to check.  If the underlying
%theorem prover cannot establish a certain proof obligation, the tool
%issues a warning. Such a warning does not necessarily mean that the
%program is wrong, since the \escj\ approach is neither sound nor
%complete. \escj\ expressions are boolean \java\ expressions, extended
%with several specification-specific constructs. 

%It relies on a dedicated theorem prover to
%find specific problems
%The basic idea is that
%a user specifies the behavior of a class and its methods, by
%using the specification language of \escj. This
%behavior can be expressed as \textit{preconditions},
%\textit{postconditions} and \textit{class invariants}. The \escj\ tool
%checks whether the implementation satisfies the given
%specification and 

As mentioned above, the \jml\ specification language is designed to be 
easy to use for \java\ programmers. In particular, \jml\ expressions
are side-effect free, boolean \java\ expressions, extended with some
specification-specific constructs. The specification language for
\escj\ used the same design principles. Initially there were some
differences between the two language, but to enable the use of
different tools on the same specification, effort has been put in
making the two languages converge~\cite{EscJmlDiff}. In this paper, we 
do not really distinguish between the two languages\footnote{In
particular, we do not consider \jml's model variables}, although we
take the more general format of \jml\ for expressions that can occur
within assignable clauses.

Below, we briefly present the main \jml\ and \escj\ constructs,
followed by a simple method specification example. A complete
description can be found in~\cite{LeavensBR00,LeinoNS00}.


\paragraph{\bf Constructs for specifying methods} 
\begin{itemize}
\item{\texttt{requires P}: precondition {\tt P}.}
%This construct specifies a precondition {\tt P}. 
%When \escj\ checks the body of a
%method, it assumes that \texttt{P} holds initially and when checking
%a method call, it will issue a warning if 
%it can not establish that \texttt{P} holds at the call point. 
 
\item{\texttt{ensures Q}: postcondition {\tt Q}.} 
%This construct specifies a postcondition \texttt{Q}. This postcondition
%is checked against the body of method and assumed for any method
%invocation if the method terminates normally (without throwing an
%exception).
 
\item{\texttt{exsures (E) R}: exceptional postcondition \texttt{R},
\emph{i.e.}~if the method terminates abruptly with exception
\texttt{E}, then \texttt{R} should be satisfied.}
%This construct specifies an exceptional condition \texttt{R}. This
%condition is
%supposed to hold if the method finishes abruptly and if
%the exception \texttt{e} that is thrown is a subclass of \texttt{E}. 

\item{\texttt{modifies L}: the set of variable locations that
a method may modify. (Section~\ref{sec-ass} below discuss this in more 
detail)}
 
\end{itemize}
 
 
 
\paragraph{\bf Specification expressions} 
\begin{itemize} 
\item{\texttt{==>}, \texttt{<==>} and \texttt{<=!=>}}: logical
implication, equivalence, and non-equivalence. 
%So, \texttt{P 
%==> Q} is true if and only if \texttt{P} is false or \texttt{Q} is 
%true, where \texttt{P} and \texttt{Q} are specification expressions of  
%\texttt{boolean} type. Furthermore, \texttt{<==>} represents the logical
%equivalence and \texttt{<=!=>} specifies non-equivalence. 
 
\item {($\backslash$\texttt{forall T V; E)} and 
($\backslash$\texttt{exists T V; E})}: universal and existential
quantification. 
% are quantifier expressions (of 
%type \texttt{boolean}).  The first expression denotes that \texttt{E}
%is true
%for all substitutions of values of type \texttt{T} for the bound 
%variable \texttt{V}. The second one denotes that \texttt{E} is true 
%for a substitution of a value of type \texttt{T} for the bound 
%variable \texttt{V}. 
 
\item{\texttt{$\backslash$old(E)}}: 
the value of the expression \texttt{E} in the pre-state of the method
(in postconditions only).
 
\item {\tt$\backslash$result}: method return value (postcondition
only).
%returned by 
%a non-void method. It can only be used within an
%\texttt{ensures} clause.
\end{itemize} 

As an example of a method specification, we give the
specification of the method \texttt{addCurrency} from the electronic
purse case study~\cite{CatanoH02a}. This specification has been
checked with \escj.
%A typical annotation example using 
%\escj. This example was taken from the formal specification of an
%\emph{electronic purse} as presented in~\cite{CatanoH02a}\footnote{The whole
%specification can be taken
%from~\cite{CatanoH01URL}.}. 
%%%%%%%
\begin{alltt}
/*@
  modifies nbData, data[nbData];
  ensures (\verb!\!old(nbData) < MAX_DATA) ?
      (nbData == \verb!\!old(nbData) + 1 && data[\verb!\!old(nbData)] == cur) :
      (nbData == \verb!\!old(nbData));
*/
void addCurrency(byte cur)\verb!{!
   if(nbData < MAX_DATA)\verb!{!data[nbData] = cur; nbData++;\verb!}!
\verb!}!
\end{alltt}
%%%%%%%%
The \texttt{addCurrency} method belongs
to the class \texttt{Currencies}, which stores all currencies 
supported by the purse application. The method 
\texttt{addCurrency} adds a new currency to the list of valid 
currencies (the array \texttt{data}). The \texttt{modifies} clause
specifies that this method may modify \texttt{nbData} and 
\texttt{data} in the position \texttt{nbData}. The
postcondition of the method \texttt{addCurrency} (written as 
\texttt{ensures} clauses) expresses that if \texttt{nbData} has not 
yet reached the threshold value \texttt{MAX$\_$DATA}, \texttt{nbData} 
will increase its value by one and the value of the formal parameter 
\texttt{cur} will be assigned to \texttt{data[\(\backslash\)old(nbData)]}.  
Inside the postcondition, the expression 
\texttt{$\backslash$old(nbData)} refers to the value of 
\texttt{nbData} before the method invocation. 



\section{JML's assignable clauses}
\label{sec-ass}

As mentioned above, an assignable clause specifies which variables may 
be modified by a method; all other variables \emph{should} remain
unchanged. Within an assignable clause, a list of so-called assignable 
expressions is specified which describe which memory locations may be
modified. The exact syntax for an assignable
expresssion is given in Figure~\ref{fig-syn-mod-spe}.
%The syntax of specification of a assignable clause in a
%\java\ annotated program
%is presented by the figure~\ref{fig-syn-mod-spe}. 
An assignable expression can be an identifier or a simple array
indexing expression denoting a single location, but it can also denote
a set of locations. For example,
\texttt{a[i\dots j]} denotes the locations
\texttt{a[i], a[i+1], \dots, a[j] }. Similarly,
\texttt{a[*]} denotes all locations
\texttt{a[0], \dots, a[a.length - 1]}. The assignable clause also can 
contain an expression
\fieldsofarg{x}, where \texttt{x} is some
object (possibly \texttt{this}) or it is the
\reach\ of some object. This means that all the
fields of the objects denoted by \texttt{x} may be modified. 
The expression \reacharg{y}, where \texttt{y} is some 
object, possibly \texttt{this}, denotes
the minimal set containing \texttt{y}, the fields of \texttt{y} and
all objects reachable from the fields of \texttt{y}.
Finally, there are special keywords \nothing\ and
\everything, denoting that a method may not modify 
any variables, or may modify all variables, respectively.

%collection of
%variable of reference type, and where
%\texttt{$\backslash$fields\_of(x)} denotes all fields of \texttt{
%denotes \texttt{x[*]} if \texttt{x}
%is an array, otherwise the set of fields of \texttt{x}. The construct
%\texttt{$\backslash$reach} only can occur inside a \texttt{fields\_of}
%construct. An specification like \texttt{$\backslash$reach(x)} denotes
%the minimal set containing \texttt{x}, the fields of \texttt{x} and
%all objects
%reachable from the fields of \texttt{x}. A specification like
%\texttt{modifies \!\!$\backslash$nothing} expresses the method as not
%having
%side-effects. Additionally, a specification like \texttt{modifies
%\!\!$\backslash$everything} expresses that the method may modify any
%variable location.
%%%%%%%%%%%%%%%%%%
\begin{figure}[hbt]
\begin{tabular}{lll}
%$<$\textsf{Assignable-spec}$>$ &::= &$<$\textsf{Assignable-clause}$>$ $<$\textsf{Assignable-exps}$>$ \\
%$<$\textsf{Assignable-exps}$>$ &::= &$<$\textsf{Assignable-exp}$>$\texttt{,}$<$\textsf{Assignable-exps}$>$ $|| $\\
%                           &    &$<$\textsf{Assignable-exp}$>$ \\
$<$\textsf{Assignable-exp}$>$  &::= &$<$\textsf{Identifier}$>$ $||$ \\
                           &    &$<$\textsf{Array-exp}$>$  $||$ \\
                           &  &$\backslash$\texttt{fields\_of(}$<$\textsf{Field-exp}$>$\texttt{)} $||$ \\
                           &  &$\backslash$\texttt{nothing} $||$\\
                           &  &$\backslash$\texttt{everything} \\
$<$\textsf{Array-exp}$>$      &::=  &$<$\textsf{Identifier}$>$\texttt{[}\textsf{Exp}\ldots \textsf{Exp}\texttt{]} $||$ \\
                           &  &$<$\textsf{Identifier}$>$\texttt{[}\textsf{Exp}\texttt{]} $||$ \\
                           &  &$<$\textsf{Identifier}$>$\texttt{[}*\texttt{]} \\
$<$\textsf{Field-exp}$>$      &::=  &$\backslash$\texttt{reach(}$<$\textsf{Point-identif}$>$\texttt{)} $||$ \\
                           &     &\texttt{reach(}\texttt{this}\texttt{)} $||$ \\
                           &     &$<$\textsf{Point-identif}$>$ $||$ \\
                           &     &\texttt{this} \\
$<$\textsf{Point-identif}$>$   &::=  &$<$\textsf{Identifier}$>$\texttt{.}$<$\textsf{Point-identif}$> $$||$ \\
                           &     &$<$\textsf{Identifier}$>$ \\
%$<$\textsf{Assignable-clause}$>$ &::= &\texttt{assignable} $||$  \\
%                             &    &\texttt{modifiable} $||$  \\
%                             &    &\texttt{modifies} \\
\end{tabular}
\label{fig-syn-mod-spe}
\caption{Grammar of the assignable clause}
\end{figure}
%%%%%%%%%%%%%%%%%%%%
%We present a set of syntactical rules which check the correct
%specification of assignable clauses in \java\ asserted programs. These
%rules have been defined taking into account both of the \jml\ grammar
%%specification and the definition of assignable clauses presented
%in~\cite{LeavensBR00}.

In \jml\ assingnable clauses are defined as
follows~\cite{LeavensBR00}\footnote{This definition differs
from~\cite{LeavensBR00} in that it does not consider dependencies, 
as we do not consider model variables yet.}
%We present the definition concerning the conditions about the assigment
%of expressions in \jml. This definition was taken directly
%from~\cite{LeavensBR00}.
\begin{definition}[Assignable clause]
\label{def-mod}
An assignable clause permits a method to modify a location \emph{loc}
provided that:
\begin{itemize}
\item \emph{loc} is mentioned in the method's \emph{assignable}
clause;
\item \emph{loc} was not allocated when the method started execution, or
\item \emph{loc} is local to the method (\emph{i.e.}, a local
variable or a formal parameter)
\end{itemize}
\end{definition}

Notice that this is a syntactic definition, thus an assignment
\texttt{x = x;} in a method \(m\) would only be correct if the 
variable \texttt{x} is mentioned in the assignable clause of the
method \(m\), or if \texttt{x} is local to the method.

%To ensure that a method implementation respects its assignable clause, 
%one should state proof obligations of the form \(s\:\mathsf{mod}\:Y\),
%for each instruction \(s\) in the method, where \textsf{mod} is
%defined as follows\footnote{Notice that 

%meaning \(s\) may only modify the variables
%denoted by \(Y\).

Below, in the next section, we present a syntactic method to derive
whether a modification of a location (\emph{i.e.} an assignment) is
correct. 

%We have adapted this definition in the presentation of our \textsf{mod}
%rules. Thereby the definition~\ref{df-mod}.
%%%%%%%%%%%%%%
%\begin{definition}[\textsf{mod}]
%\label{df-mod}
%\[s\:\mathsf{mod}\:Y \defn{=} \forall x \in LC.\,x \not\in Y \Rightarrow 
%\textit{Pre}_s(x) = \textit{Post}_s(x)\]
%%$s$\ \textsf{mod}~$Y$ $\Leftrightarrow$ $\forall x \in LC.\,
%%x\not\in$ \textsc{Y}$\ \Rightarrow Pre(x) = Post(x)$ \\
%where $LC$ is a set of variable locations, $s$ is some
%instruction, $\textit{Pre}_s(x)$
%refers to the value of $x$ before executing $s$,
%$\textit{Post}_s(x)$ refers to the value of $x$ after executing
%$s$ and $Y$ is the set of variables that may be modified by \(s\).
%\end{definition}



%This section 


%We present a set of syntactical rules which define the conditions that
%must be fulfilled for any instruction occurring in a
%\java\ annotated program. We have profited the syntax of \java\ for
%defining our rules. Thus, for instance, we rely on rules for
%expressions, statements, and so on. The basic idea is that for each
%sort of instruction occurring in a method
%declaration, these rules check that the variable locations they assign
%belongs to the set of variable locations that the
%method \emph{may} modify.


%The rest of this section is organized as it follows. Section
%\ref{sub-sec-ass-beh} present the syntax of specification of
%assignable clause and certain other definitions used in the
%specification of our rules. Sections \ref{sub-sec-rul-con-met},
%\ref{sub-sec-rul-con-pri-suf-and-pri-exp}, \ref{sub-sec-rul-con-ass},
%\ref{sub-sec-rul-con-ope}, \ref{sub-sec-rul-con-sta}, and
%\ref{sub-sec-tra-rul} present the rules for the different
%\java\ constructs. Section \ref{sub-sec-the-rel-mem} the definition of
%membership operator used in the specification of our rules.






%\subsection{Assignable Behavior}
%\label{sub-sec-ass-beh}


%%%%%%%%%%%%%%
%These \textsf{mod} rules have beeing reformulated in other more
%specific ones, considering the syntactical structure of
%\java. Consequently, we rely on \textsf{modEXP} for
%expressions, \textsf{modSTM} for statements and so on. 
%on so-called primary
%expressions (


\section{A syntactical method to check assignable clauses}
\label{sec-syn-met-che-ass-cla}

\subsection{Preliminaries}
As explained above, for each language construct \(s\) we define a rule
which allows us to derive the correctness of statements
\MOD{s}{Y}, with the intuitive meaning: \(s\) only assigns
to variables which are mentioned in \(Y\). Notice that \textsf{mod} is
a syntactic notion, and we cannot give a semantic definition for
it\footnote{In particular, a semantic definition would never be able
to find that \texttt{x = x;} could violate an assignable clause.}, but 
the rules presented below together define \textsf{mod}.
Below, we will also use the notions \textsf{modFE} and
\textsf{modSuf}, which are defined only on so- called \emph{full
expressions} and \emph{suffix expression}, respectively. Full
expressions and suffix expressions are defined according to the
grammar in Figure~\ref{FigExpGrammar}. Thus, a full-expression is an
array indexing expression, a method call, a quantified expression or a 
suffix-expression. A suffix-expresion can be for example an
identifier, a reference to super or this, a literal or an
initialisation of an object or an array.

This grammar is derived from
the \jml\ grammar~\cite{LeavensBR00} and the \java\
grammar~\cite{GoslingS?}, but with some simplifications, because we
assume that the program is accepted by the \java\ compiler and/or the
\jml\ tool. Thus, we can rely on the fact that the syntax will obey
this grammar.

The notions
\textsf{modFE} and \textsf{modSuf} have the similar intuitive meaning as
\textsf{mod}. We assume that we have predicates \textsf{full?} and
\textsf{suffix?} which can decide whether an expression is a full
expression, or an expression suffix, respectively.

\begin{figure}[t]
\begin{tabular}{rcl}
\FullExp & ::= & \FullExp[\Exp\(^+\)] \option \\
         &     & \FullExp(\Exp\(^*\)) \option \\
         &     & \FullExp.\SufExp \option \\
         &     & \SufExp\\
\\
\SufExp  & ::= & \Identifier \option \\
         &     & \texttt{this} \option \\
         &     & \texttt{super} \option \\
         &     & \texttt{new} \Type [\Exp]\\
         &     & \texttt{new} \Type (\Exp\(^*\))\\
         &     & \textsf{Literal}
%\FullExp & ::= & \FullExp.\SufExp \option \\
%              && \textsf{Literal} \option \\ && \texttt{new}
%              \Type(\Exp) \option \\ && \texttt{new} \Type[\Exp]
%              \option \\ && \FullExp.\MethodCall \option \\ &&
%              \MethodCall \option\\ && \SufExp \\
%\\
%\SufExp & ::= & \Identifier \option \texttt{super} \option
%                \texttt{this} \option \Static \option
%                \Identifier[\Exp]\\
\end{tabular}
\caption{Grammar of full expressions and expression suffices}
\label{FigExpGrammar}
\end{figure}
In some cases we need to lift the notion \textsf{mod} over a set of
instructions. Therefore, we define the following.

\begin{definition}[$\overrightarrow{\textsf{mod}}$]
\label{def-mod-lis}
\[\MODS{S}{Y} \defn{=}\forall s.\,s\in S\Rightarrow \MOD{s}{Y}\]
%$S\ \overrightarrow{\textsf{mod}}~Y$ iff $\forall s.\ s\in
%S\Rightarrow s\ \textsf{mod}~Y$
\end{definition}

As explained above, \(s\:\mathsf{mod}\:Y\) is correct if each variable 
assigned to in \(s\) is mentioned in \(Y\). As mentioned in the
previous section, the assignable clause can contain expressions as
\fieldsofarg{this}, thus testing whether a
variable occurs in \(Y\) cannot be done by simply using set
membership (\(\in\)). Therefore, we use an extended notion of set
membership, subsuming \(\in\), denoted as \(\underline\in\). Below, in 
Subsection~\ref{sub-sec-the-rel-mem} this notion is defined precisely.
Based on the notion of extended set membership, we can also define the 
notion of an \emph{extended membership subset}.
%Each rule
%checks that the set of locations modified by the corresponding
%instruction \emph{belongs} to a set of locations specifined as modified for
%the method. This concept $\underline\in$ of membership must be
%formulated considering
%the syntaxis of the assignable declarations (see
%Figure~\ref{fig-syn-mod-spe}), but intituivaly subsumes the standard
%concept $\in$. Section~\ref{sub-sec-the-rel-mem} shows the definition
%of this concept. 

%Once we count on the definition of membership for assignable
%locations, we should give the definition for the
%inclusion. Figure~\ref{def-subseq} presents the definition of inclusion of
%assignable locations.
%%%%%%%%%%
\begin{definition}[Extended membership subset $\sqsubseteq$]
\label{def-subseq}
\[\extsubset{Y_1}{Y_2} \defn{=} \forall v.\ \extmember{v}{Y_1}
\Rightarrow \extmember{v}{Y_2}\]
\end{definition} 
%%%%%%%%%%
%Definition~\ref{def-mod-lis} generalize the concept introduce by
%Definition~\ref{df-mod}, by extending it for set of instructions.
%%%%%%%%%
%%%%%%%%%%

%We have also introduced some other \emph{mod}
%constructs by exploiting the \java\ syntax. These \emph{mod}
%constructs cover concepts such as \textit{modSTM}, \textit{modEXP},
%\textit{modPE}, \textit{modPRM} and \textit{modPS}, which redefine the
%original concept \textit{mod} in terms of concepts for
%\emph{statements}, \emph{expressions}, \emph{post-expressions},
%\emph{primary} and \emph{suffixes} expressions respectively. A primary
%\java\ expression can be seen as the first expression in a
%sequence of \java\ expressions joined by the symbol \texttt{``.''}. All 
%of the other expressions make up the suffix of the original
%expression.
%
%expresses that if the set of assignable variable locations of certain
%method is represented by locations $Y$, then this method may modify the set
%of expressions \textsc{Y}, and in its method body occur expressions or
%declarations such as \textup{f}$_1$\textup{(e}$_1$\textup{,
%f}$_2$\textup{(e}$_2$\textup{,}$\cdots$
%\textup{(f}$_n$\textup{(e}$_n$\textup{))))}, then their correctness
%will be established in terms of the correctness of each subexpression
%\textup{e}$_i$ \emph{w.r.t} \textsc{Y}. 




\subsection{The rules for \textsf{mod}}

\subsubsection{Method rules}
\label{sub-sec-rul-con-met}
%We have defined certain syntactical rules for every \java\ construct,
%stating which variables may be modified. Our rules check for
%all \java\ constructs that the expressions modified by instructions
%inside all method bodies are according to
%their assignable specification. We present the derivation rules for
%the most important \java\ constructs. The whole set of rules can be found
%in~\cite{CatanoMasterThesis01}.
The first rules that we define are related with methods.
Checking of assignable clauses is always done on a method by method
basis. The first rule therefore deals with method
declarations. Following Definition~\ref{def-mod}, it is sufficient for 
the body of the method to check that it only modifies the variables in 
the assignable clause of the method (denoted as \(Y\)) or the local
variables or formal parameters of the method.


%\textsf{Meth-Dec} is the rule corresponding to a method declaration
%\texttt{m(}$\overrightarrow{\texttt{o}}$\texttt{)} that has
%specified may modify the set of variable locations $Y$. This rule
%states that for this method declaration, it is enough to
%check every instruction in its method body
%(\texttt{m(}$\overrightarrow{\texttt{o}}$\texttt{).body}) with respect
%to \textsc{Y}, the set of parameters and the local variables to the
%method. As can be seen, this rule essentially formalizes
%Definition~\ref{def-mod}. 
%%%%%
\[
\begin{tabular}{ll}
\textsf{(Meth-Dec)}\,\,\,&
\begin{prooftree} 
\MOD{\methodbody{m}{o}}{(Y \cup \{\overrightarrow{\texttt{o}}\}
                           \cup \{\methodloc{m}{o}\}) }
\justifies
\MOD{\method{m}{o}}{Y}
\end{prooftree}
\end{tabular}
\]
%%%%%

When checking a method body, if one encounters a method invocation the
assignable clause of this method is compared with the assignable
clause of the method body that is currently checked. The assignable
clause of the invoked method should be a subset of the assignable
clause of the current method~--~with appropriate substitutions~--~to
ensure that no other variables than the one mentioned in the
assignable clause can be modified. According to the grammar, method
invocations are always full expressions, thus the method invocation
will be of the form \(\method{\textit{fe}}{o}\) where \textit{fe} is a
full expression. The last element in \textit{fe} will be a method
name, \texttt{this} or \texttt{super}. In all cases, the appropriate
assignable clause can be found by looking at the static type of the
one but last element of \textit{fe}, if it exists, or otherwise by
taking the default value \texttt{this}.  This is sufficient, because
behavioural subtyping ensures that overriding methods in subclasses do
not modify more variables~\cite{?}.

Additionally, it has to be checked that the full expression
\textit{fe} and the actual parameters \(\overrightarrow{\texttt{q}}\)
respect the assignable clause.

The rule for method invocations is only defined on \textsf{modFE}: the 
\java\ grammar ensures that a method call can only occur in this context 
(provided the \java\ program is accepted by the compiler).

%A method invocation
%$e\texttt{.m(}\overrightarrow{\texttt{q}}\texttt{)}$,
%must fulfill the assignable specification corresponding to its
%method declaration $\texttt{m(}\overrightarrow{\texttt{o}}\texttt{)}$
%previous replacing: $(i)$\ \texttt{this} by $e$ and
%$(ii)$\ $\overrightarrow{\texttt{o}}$ by
%$\overrightarrow{\texttt{q}}$. Additionally, the actual parameters
%$\overrightarrow{\texttt{q}}$ must be checked for
%side-effects. It similar to the expression $e$ (see rule
%\textsf{Meth-Inv}). As can be seen this rules use \textsf{modEXP}
%and \textsf{modPE} rules. \textsf{modEXP} stands for expressions and
%\textsf{modPE} for post-fix expressions. A \java\ post-fix
%expression represents a sort of \emph{primary} expressions linked by
%``.''. A typical primary \java\ expression is a \java\ identifier.
%%%%%%
\[
\begin{tabular}{ll}
\textsf{(Meth-Inv)}\,\, &
\begin{prooftree}
\extsubset{\methodassign{\textit{fe}}{o}
\:[\overrightarrow{\texttt{o}}/\overrightarrow{\texttt{q}},
\texttt{this}/ \textit{fe}]}{Y}
\quad
\MODFE{fe}{Y}
\quad
\MODS{q}{Y}
\justifies
\MODFE{\method{\textit{fe}}{q}}{Y}
\end{prooftree}
\end{tabular}
\]
%%%%%%


\subsubsection{Assignment rules}
\label{sub-sec-rul-con-ass}
When we encounter an assignment $e_1\:\texttt{=}\:e_2$ we must check
that $e_1$ belongs to set of locations that the method may modify
(\(\extmember{e_1}{Y}\)). Additionally we must check that $e_1$ and
$e_2$ do not have unwanted side-effect. Following the \java\ syntax,
we know that for \(e_1\) we have to check \textsf{modFE} only.
%%%%%%%%%
\[
\begin{tabular}{ll}
\textsf{(Assg)}\,\,\, & 
\begin{prooftree}
\extmember{e_1}{Y}
\quad
\MODFE{e_1}{Y}
\quad
\MOD{e_2}{Y}
\justifies
\MOD{e_1\:\texttt{=}\:e_2}{Y}
\end{prooftree}
\end{tabular}
\]
%%%%%%%%
This rule generalises to all variations on assignments, as provided by
\java (\texttt{+=}, \texttt{*=} \emph{etc.}).
For initialisations of local variables we use a similar rule, although 
by default we know that the local variable will occur in \(Y\)~--~as
this is guaranteed by the rule \textsf{Meth-Decl}.

%An assignment can also be occurred in the declaration of a local
%variable. Suppose we have the declaration of certain variable
%\texttt{x} which is initialized with the value corresponding to
%certain expression $e$. The location
%corresponding a this variable must belong to the set $Y$ of possible
%assignable locations. Rule \textsf{Var-Decl-Assg} states this
%fact. Additionally, one must check the expression $e$ by using
%expressions rules.
%%%%%%%%%%
%\[
%\begin{tabular}{ll}
%\textsf{(Var-Decl-Assg)}\,\, & 
%\begin{prooftree}
%\rule[1ex]{0em}{1.5ex}
%\texttt{x}\underline{\in}Y,\ \ e\ \textsf{modEXP}~Y
%\justifies
%\texttt{T x =}~e\ \textsf{modEXP}~Y
%\end{prooftree}
%\end{tabular}
%\]
%%%%%%%%

A special kind of assignments in \java\ are constructed by using the
pre- and postfix operators (\emph{e.g.}~\texttt{x++} and
\texttt{++x}). Such instructions can be considered as  \emph{in
place} assignments, thus the rules for pre- and postfix operators are
similar to the assignment rule. For example, the rule
\textsf{Post-Plus} states that the expression $e$, which is the
target of an assignment, has to belong to set of expressions $Y$ which
may modified by the method, and that additionally $e$ must be checked for
(unwanted) side-effects.
%%%%%%
\[
\begin{tabular}{ll}
\textsf{(Post-Plus)}\,\, &
\begin{prooftree}
\extmember{e}{Y}
\quad
\MODFE{e}{Y}
\justifies
\MOD{e\texttt{++}}{Y}
\end{prooftree}
\end{tabular}
\]
%%%%%%%%

\subsubsection{Statement rules}
\label{sub-sec-rul-con-sta}
For all  \java\ constructs used to build statements, such as
\(S\texttt{;}T\), 
\(\texttt{if(}c\texttt{)\{}S\texttt{\}}\) and
%\(\texttt{if(}c\texttt{)\{}S\texttt{\}else\{}T\texttt{\}}\) and
\(\texttt{while(}c\texttt{)\{}S\texttt{\}}\) the rules simply pass on
the check on the modifies clause to the components of the
statement. As an example, we present the rule \textsf{If-Then}.
%$\overrightarrow{s}$\texttt{\}}},
%\texttt{if(}c\texttt{)\{$\overrightarrow{s}$\texttt{\}}else\{$\overrightarrow{t}$\}},
%\texttt{while(c)\{$\overrightarrow{s}$\}}

%. For the first case, for
%instance, the guard $c$ must be checked using
%\textsf{modEXP} rules since a guard
%corresponds to an expression. The set of instructions occurring in its
%body $\overrightarrow{s}$ must be checked using the most general rules
%\textsf{mod}. This is showed by the rule \textsf{If-Then}. For the
%others \java\ statements a similar analysis can be done.
%%%%%%%%%%
\[
\begin{tabular}{ll}
\textsf{(If-Then)}\,\, & 
\begin{prooftree}
\MOD{c}{Y}
\quad
\MOD{s}{Y}
\justifies
\MOD{\texttt{if(}c\texttt{)\{}S\texttt{\}}}{Y}
\end{prooftree}
\end{tabular}
\]

\subsubsection{Expression rules}
\label{sub-sec-rul-con-ope}
For most expressions, the same applies as for statements: the rules
simply passes on the check for side-effects to the arguments of the
expressions. In this way, we define for example the rule
\textsf{BinOp}, which applies to all binary operators (arithmetic,
relational and logical). Similar rules are defined for \emph{e.g.}~the 
unary operators, the \texttt{instanceof} expression, the casting
operator, and the conditional expression.

%%%%%%%%%
\[
\begin{tabular}{ll}
\textsf{(BinOp)} & 
\begin{prooftree} 
\MOD{e_1}{Y}
\quad
\MOD{e_2}{Y}
\justifies
\MOD{e_1 \oplus e_2}{Y}
\using
\oplus \in \Big\{
	\begin{array}{l}
		\texttt{<},\texttt{<=},\texttt{>},
                \texttt{>=},\texttt{==},\texttt{!=},\texttt{||},	\\
		\texttt{\&\&},\texttt{+},\texttt{-},\texttt{*},
                \texttt{/},\mathtt{\backslash},\texttt{\&},
                \texttt{\^\,}, \texttt{|}
	\end{array}
	\Big\}
\end{prooftree}
%\\[3.0ex] 
%\textsf{(Unary)} & 
%\begin{prooftree} 
%e\ \textsf{modEXP}~Y
%\justifies
%\oplus \ e\ \textsf{modEXP}~Y
%\using
%\oplus \in \{\texttt{+,-,$\sim$,!}\}
%\end{prooftree}
%\\[3.0ex] 
%\textsf{(Instance)} & 
%\begin{prooftree} 
%e\ \textsf{modEXP}~Y
%\justifies
%e\ \texttt{instanceof C}~\textsf{modEXP}~Y
%\end{prooftree}
%\\[3.0ex] 
%\textsf{(Cast)} & 
%\begin{prooftree}
%e\ \textsf{modEXP}~Y
%\justifies
%\texttt{(T)}e\ \textsf{modEXP}~Y
%\end{prooftree}
%\\[3.0ex] 
%\textsf{(Conditional)}\,\, & 
%\begin{prooftree} 
%e_1\ \textsf{modEXP}~Y,\ \ e_2\ \textsf{modEXP}~Y,\ \ e_3\
%\textsf{modEXP}~Y
%\justifies
%e_1\texttt{?}e_2\texttt{:}e_3\ \textsf{modEXP}~Y
%\end{prooftree}
\end{tabular}
\]
%%%%%%%%%%


However, if none of these rules applies to an expression, then it
means that we have a so-called full expression (see
Figure~\ref{FigExpGrammar}) and we apply the \textsf{To-Fe}.
\[
\begin{tabular}{ll}
\textsf{(To-Fe)}\,\, & 
\begin{prooftree}
\MODFE{e}{Y}
\justifies
\MOD{e}{Y}
\using
\textsf{full?}(e)
\end{prooftree}
\end{tabular}
\]

\subsubsection{Full expression rules}
Following the grammar for full expressions, we define rules defining
\textsf{modFE}. The rule for method invocation \textsf{Meth-Inv} 
already has been presented above. If this rule does not apply, then
one of the following rules (for array, qualified or suffix
expressions, respectively should apply.
\[
\begin{tabular}{c}
\begin{tabular}{p{6cm}p{5cm}}

\begin{tabular}{ll}
\textsf{(Array)}\,\, & 
\begin{prooftree}
\MODFE{fe}{Y}
\quad 
\MODS{\overrightarrow{e}}{Y}
\justifies
\MODFE{fe\texttt{[}\overrightarrow{e}\texttt{]}}{Y}
\end{prooftree}
\end{tabular}
&
\begin{tabular}{ll}
\textsf{(Qualified)} &
\begin{prooftree}
\MODFE{fe}{Y}
\quad
\MODSuf{s}{Y}
\justifies
\MODFE{fe\texttt{.}s}{Y}
\end{prooftree}
\end{tabular}

\end{tabular}

\\[3.0ex]
\begin{tabular}{ll}
\textsf{(To-Suf)}\,\, & 
\begin{prooftree}
\MODSuf{fe}{Y}
\justifies
\MODFE{fe}{Y}
\using
\textsf{suffix?}(\textit{fe})
\end{prooftree}
\end{tabular}
\end{tabular}
\]

\subsubsection{Suffix expression rules}
Finally, we present some rules for suffix expressions. Most
suffix expressions are constant values (\emph{e.g.}~\texttt{this} and
all the literals) which do not modify anything, thus the rules for
these constructs are straightforward. Also the access to an identifier 
does not modify anything~--~and if the identifier is the target of an
assignment, the rule \textsf{Assg} takes care that it is mentioned in
the assignable clause. Thus, the only suffix expressions which are of
some interest are the \texttt{new} expressions. Here, the possible
arguments have to be checked for unwanted side-effects. The creation
of a new object or array does not violate the rules for the assignable 
clause (\emph{cf.}~Definition~\ref{def-mod}).

%\subsubsection{Transition rules}
%\label{sub-sec-tra-rul}
%In addition to define rules for each \java\ construct, we need also
%certain \emph{transition} rules which allow us to pass from more
%general rules to more particular ones, depending on the kind of the
%instruction. So, for example, we need transition rules from the most
%general case to expressions statements (\textsf{Mod-To-Exp} and
%\textsf{Mod-To-Stm}), from expressions to post-fix expressions
%(\textsf{Exp-To-Pe}), and from post-fix expressions to primary suffix
%and primary expressions.
%%%%%%%%%%%
%\[
%\begin{tabular}{ll}
%\textsf{(Mod-To-Exp)}\,\, & 
%\begin{prooftree}
%e\ \textsf{modEXP}~Y
%\justifies
%e\ \textsf{mod}~Y
%\using
%e\in Expression
%\end{prooftree}
%\\[3.0ex]
%\textsf{(Mod-To-Stm)} & 
%\begin{prooftree}
%e\ \textsf{modSTM}~Y
%\justifies
%e\ \textsf{mod}~Y
%\using
%e\in Statement
%\end{prooftree}
%\\[3.0ex]
%\\[3.0ex]
%\textsf{(Pe-To-Ps)} &
%\begin{prooftree}
%e_1\ \textsf{modPE}~Y,\ e_2\ \textsf{modPS}~Y
%\justifies
%e_1\ \texttt{.}~e_2\ \textsf{modPE}~Y
%\end{prooftree}
%\\[3.0ex]
%\textsf{(Pe-to-Prm)} &
%\begin{prooftree}
%e\ \textsf{modPRM}~Y
%\justifies
%e\ \textsf{modPE}~Y
%\using
%e\in \ \textsf{Primary}
%\end{prooftree}
%\end{tabular}
%\]
%%%%%%%%%%%%


%\subsubsection{Rules concerning primary suffix and primary expressions}
%\label{sub-sec-rul-con-pri-suf-and-pri-exp}
%\java\ \emph{post-fix} expressions can be seen as set of \emph{suffix}
%expressions linked by ``.''. The first expression of which must be a
%\emph{primary} expression. Typical primary expressions correspond to
%identifiers, constants, the special \java\ keywords \texttt{this},
%\texttt{static} and \texttt{super}, method invocations (as presented
%before) and \emph{new} expressions. 

%In the case of \emph{new} expressions (see rule \textsf{New-Exp}), the
%expressions passed as parameters must be checked for side-effects by
%using \textsf{modEXP} rules. The other cases of primary expression do
%not require an additional analyze.
%%%%%%%%
\[
\begin{tabular}{ll}
\textsf{(New-Exp)}\,\, & 
\begin{prooftree}
\MODS{\overrightarrow{e}}{Y}
\justifies
\MODSuf{\texttt{new T(}\overrightarrow{e}\texttt{)}}{Y}
\end{prooftree}
\end{tabular}
\]
%%%%%%%%%%





%%%%%%%%%%%





\subsection{The extended membership relation}
\label{sub-sec-the-rel-mem}
Finally, we have to give rules to define the extended membership
relation. Figure~\ref{fig-syn-mod-spe} shows which expressions can
occur inside an assignable clause. Again, we give syntactic rules for
all possible formats of an assignment targets. First we consider the
assignable expressions \nothing\ and
\everything. In the first case the assignable
clause is the emptyset, in the second case it is the global
universe. The rules for these constructs are obvious.
\[
\begin{tabular}{p{5cm}p{5cm}}
\begin{tabular}{ll}
\textsf{(In-Nothing)} &
\begin{prooftree}
\textsf{false}
\justifies
\extmember{e}{\emptyset}
\end{prooftree}
\end{tabular}
&
\begin{tabular}{ll}
\textsf{(In-Everything)} &
\begin{prooftree}
\textsf{true}
\justifies
\extmember{e}{\mathcal{U}}
\end{prooftree}
\end{tabular}
\end{tabular}
\]

%Another important aspect that must be analyzed is
%the different constructs that can appear in assignable
%specifications. According to the syntax of the assignable clause as
%presented by Figure~\ref{fig-syn-mod-spe},
%\texttt{$\backslash$fields\_of} and \texttt{$\backslash$reach}
%constructs can occur inside of such assignable specifications, the
%latter only occurring inside the former. Hence, when establishing
%membership conditions $e\underline\in Y$, we need to define suitable
%rules taking into account the fact that these constructs can occur
%inside $Y$.

For non-trivial assignable clauses, the most simple case is where a
variable is mentioned literally in the assignable clause. In this
case, we fall back directly on the standard definition of set
membership.
%%%%%%%
\[
\begin{tabular}{p{5cm}p{5cm}}
\begin{tabular}{ll}
\textsf{(In-Var)} &
\begin{prooftree}
\member{\texttt{x}}{Y}
\justifies
\extmember{\texttt{x}}{Y}
\end{prooftree}
\\[3.0ex]
\textsf{(In-Arr)} &
\begin{prooftree}
\member{\texttt{a[}e\texttt{]}}{Y}
\justifies
\extmember{\texttt{a[}e\texttt{]}}{Y}
\end{prooftree}
\end{tabular}
&
\begin{tabular}{ll}
\textsf{(In-Exp)} &
\begin{prooftree}
\member{e\texttt{.x}}{Y}
\justifies
\extmember{e\texttt{.x}}{Y}
\end{prooftree}
\\[3.0ex]
\textsf{(In-Exp-Arr)}\,\, &
\begin{prooftree}
\member{e\texttt{.a[}e_1\texttt{]}}{Y}
\justifies
\extmember{e\texttt{.a[}e_1\texttt{]}}{Y}
\end{prooftree}
\end{tabular}
\end{tabular}
\]
%%%%%%%%%%

When checking array index expressions as
\texttt{a[}$e$\texttt{]$\underline{\in} Y$}, it is also sufficient
to find an expression \texttt{a[*]} or \texttt{a[i..j]} in the
assignable clause\footnote{Assuming that \(i \leq e \leq j\) can be
checked syntactically, which will not always be the case.}.

% \jml\ construct in $Y$ specifying that the
%corresponding method may modify any element of \texttt{a}
%(\textsf{Times-Arr}) or a certain interval where the value of $e$ is
%found (rule \textsf{Interv-Arr}).
%%%%%%%%%%
\[
\begin{tabular}{p{5cm}p{5cm}}
\begin{tabular}{ll}
\textsf{(Global-Arr)}\,\, &
\begin{prooftree}
\member{{a[}*\texttt{]}}{Y}
\justifies
\extmember{{a[}e_1\texttt{]}}{Y}
\end{prooftree}
\end{tabular}
&%\\[3.0ex]
\begin{tabular}{ll}
\textsf{(Interv-Arr)}\,\, & 
\begin{prooftree}
\member{\texttt{a[}i..j\texttt{]}}{Y}
\quad\ i\leq e\leq j
\justifies
\extmember{\texttt{a[}e\texttt{]}}{Y}
\end{prooftree}
\end{tabular}
\end{tabular}
\]
%%%%%%%%%%%
Further, an expression \fieldsofarg{\(e\)} can
occur in an assignable clause. Here we distinguish two cases: \(e\) is 
a single object or array\footnote{For arrays,
\fieldsofarg{\(e\)} is
equivalent to \(e\)\texttt{[*]}.}, or it is a
\reach\ expression. In the first case, the rules
check for each qualified expression whether
\fieldsof\ occurs for the receiving object. For 
unqualified expressions, this is checked for \texttt{this}.


%If these rules do not apply, we
%have to consider some cases according to the shape of $e$. If
%$e$ has shape $e_1\texttt{.}e_2$ and $e_2$ is not an array expression,
%it will be enough to have \texttt{$\backslash$fields\_of($e_1$)}
%occurring in $Y$ to establish the original condition of membership
%(rule \textsf{In-Fld-Exp-Var}). Otherwise, if $e_2$ is an array
%expression \texttt{a[$e_3$]}, it will be enough to have
%\texttt{$\backslash$fields\_of($e_1$.a)} occurring in $Y$ (see rule
%\textsf{In-Fld-Exp-Arr}). We have similar rules when $e$ is an
%expression not linked by ``.'' (see rules \textsf{In-Fld-Var} and
%\textsf{In-Fld-Arr}).
%%%%%%%%%%%
\[
\begin{tabular}{p{6.5cm}p{5cm}}
\begin{tabular}{ll}
\textsf{(In-Fld-Exp-Var)}\,\, &
\begin{prooftree}
\member{\fieldsofarg{\(e\)}}{Y}
\justifies
\extmember{e\texttt{.x}}{Y}
\end{prooftree}
\\[3.0ex]
\textsf{(In-Fld-Exp-Arr)}\,\, &
\begin{prooftree}
\member{\fieldsofarg{\(e\)\texttt{.a)}}}{Y}
\justifies
\extmember{e\texttt{.a[}e'\texttt{]}}{Y}
\end{prooftree}
\end{tabular}
&
\begin{tabular}{ll}
\textsf{(In-Fld-Var)} &
\begin{prooftree}
\member{\fieldsofarg{this}}{Y}
\justifies
\extmember{\texttt{x}}{Y}
\end{prooftree}
\\[3.0ex]
\textsf{(In-Fld-Arr)}\,\, &
\begin{prooftree}
\member{\fieldsofarg{a}}{Y}
\justifies
\extmember{\texttt{a[}e'\texttt{]}}{Y}
\end{prooftree}
\end{tabular}
\end{tabular}
\]
%%%%%%%%%%%
For reach expressions, we check whether an expression
\fieldsofarg{\reacharg{\(e'\)}} occurs in the assignable clause, and if \(e
\in \reacharg{\(e'\)}\), \emph{i.e.}~\(e\) is in the reach of \(e'\).

%Furthermore, \texttt{$\backslash$reach} expressions can occur in
%assignable clauses. So, if an expression such as
%\texttt{$\backslash$fields\_of($\backslash$reach($e_1$))} occurs in
%$Y$ it is sufficient having $e$ belonging to set expressions reachable
%from $e_1$ for establish the condition \texttt{$e$.x$\underline\in
%Y$}. This situation is presented by the rule
%\textsf{In-Reach-Exp}. Something similar happens when \texttt{x} is an
%array expression. This is presented by rules
%\textsf{In-Reach-Exp-Arr}. The rules \textsf{In-Reach-Var} and
%\textsf{In-Reach-Arr} present the cases when we intend to establish
%membership conditions for expressions not linked by ``.''.
%%%%%%%%%
\[
\begin{tabular}{ll}
\textsf{(In-Reach-Var)} &
\begin{prooftree}
\member{\fieldsofarg{\reacharg{\(e\)}}}{Y}
\quad
\member{\texttt{this}}{\reacharg{\(e\)}}
\justifies
\extmember{\texttt{x}}{Y}
\end{prooftree}
\\[3.0ex]
\textsf{(In-Reach-Arr)} &
\begin{prooftree}
\member{\fieldsofarg{\reacharg{\(e'\)}}}{Y}
\quad
\member{\texttt{a}}{\reacharg{\(e'\)}}
\justifies
\extmember{\texttt{a[}e\texttt{]}}{Y}
\end{prooftree}
\\[3.0ex]
\textsf{(In-Reach-Exp)} &
\begin{prooftree}
\member{\fieldsofarg{\reacharg{\(e'\)}}}{Y}
\quad
\member{e}{\reacharg{\(e'\)}}
\justifies
\extmember{e\texttt{.x}}{Y}
\end{prooftree}
\\[3.0ex]
\textsf{(In-Reach-Exp-Arr)}\,\, &
\begin{prooftree}
\member{\fieldsofarg{\reacharg{\(e'\)}}}{Y}
\quad
\member{e\texttt{.a}}{\reacharg{\(e'\)}}
\justifies
\extmember{e\texttt{.a[}e''\texttt{]}}{Y}
\end{prooftree}
\end{tabular}
\]
The reach of an object is defined recursively: an element \(e\) is in
the reach of an object, if it is the object itself, or if it is in the 
reach of one of the fields of this object.
\[
\begin{tabular}{ll}
\textsf{(Reach-Base)} &
\begin{prooftree}
e = e'
\justifies
\member{e}{\reacharg{\(e'\)}}
\end{prooftree}
\\[3.0ex]
\textsf{(Reach-Rec)} &
\begin{prooftree}
\member{e''}{\fieldsofarg{\(e'\)}}
\quad
\member{e}{\reacharg{\(e''\)}}
\justifies
\member{e}{\reacharg{\(e'\)}}
\end{prooftree}
\end{tabular}
\]

%%%%%%%%%%%
\subsection{A deriviation example}
%%%%%%%%%%%
Do we have the space to put this????


\subsection{Limitations of our approach}
As our method works on a purely syntactical basis, it has some
limitations. In particular, our method does not work in the context of 
\emph{aliasing}. For example, suppose we have the following class
\texttt{C}.

%Our specification just does not concern with the modification but the
%assignment, hence in a context with \emph{Aliasing} we miss the
%variable modified. Thereby, suppose we have a class \texttt{C} as
%presented below. 
%%%%%%%%
\begin{alltt}
public class C\verb!{!
  O y = new O(), x = new O();    

  //@ modifies x, y.i;
  public void p()\verb!{!
     x = y;
     y.i = 7;
  \verb!}!    
\verb!}!
public class O\verb!{!
  public int i;
\verb!}!
\end{alltt}
%%%%%%%%%%
According to the specification method \texttt{p} may modify
\texttt{x} and the field \texttt{i} of the object \texttt{y}. Because
of the assignment \texttt{x = y;} the variables \texttt{x} and
\texttt{y} are aliases, so the assignment \texttt{y.i = 7;} implicitly 
also modifies \texttt{x.i}, which is an unwanted side-effect. However, 
our method will not reject this method.

%Although the expression
%\texttt{y.i = 7;}, in addition to modify the field \texttt{i} of the
%object \texttt{y}, modifies the field \texttt{i} of the object
%\texttt{x} (since the aliasing between \texttt{x} and \texttt{y} as
%consequence of the \emph{assignment} \texttt{x = y}), our rules
%accept as valid this specification.

Our method also does not take earlier assignments to variables into
account, \emph{e.g.}~a specificiation \texttt{//@ modifies a[i];} will 
be accepted for a body \texttt{i++; a[i] = 3;}.

However, we think that these limitations do not severely restrict the
usability of our method. Our method should be considered as a quick
check to get a reasonable trust in the correctness of the specified
assignable clauses. In our experience, when specifying the assignable
clause of a method, it is more likely to forget to mention a simple
variable then to overlook complicated modification structures. Finding 
these small specification mistakes before doing formal verification
significantly can help to improve the verification speed.



\section{A checker for assignable clauses}
\label{sec-che-for-ass-cla}
%%%---Finally, we briefly discuss the implementation of the \modtool, which
%%%---implements the method presented above.
%%%---%We present some implementation issues concerning \modtool: a
%%%---%modifiable checker implementing the approach and rules presented in
%%%---%Section~\ref{sec-syn-met-che-ass-cla}.
%%%---
%%%---For the implementation we reuse the parser of \jml.
%%%---%\jml\ comes with several classes for handling abstract
%%%---%syntax trees. 
%%%---The \jml\ parser returns an
%%%---\emph{abstract syntax tree}, in instance of class \texttt{AST}.
%%%---This class \texttt{AST} allow one to represent pieces of code as the
%%%---sequence of a node and a \emph{sibling}, where a node is a tuple of an
%%%---\texttt{int} and a \texttt{String}, and the sibling is another
%%%---\texttt{AST}.
%%%---Further, there is an utility class \texttt{ASTFactory} which
%%%---provides operations to create and duplicate trees, and a class
%%%---\texttt{ASTArray}, which maintains an array of abstract syntax trees.
%%%---
%%%---
%%%---% for recovering the
%%%---%structure of programs we are checking. 
%%%---We use these classes to first $(i)$\ determine the \emph{assignable
%%%---clause} of all method declarations and then $(ii)$\ to apply the rules
%%%---presented above. In order to do this we have defined a class
%%%---\texttt{ASTContext}, and a class \texttt{Assignable}. 
%%%---%recovering the structure of the program for
%%%---%applying our checking rules.
%%%---The last one implements our deriviation rules as static methods
%%%---parameterized by the assignable clause of an instruction.
%%%---%context under which the respective instruction is
%%%---%found.
%%%---
%%%---%Section~\ref{abo-jml-cla-use} gives a general overview of the class
%%%---%for handling Abstract syntax trees coming with
%%%---%\jml. Section~\ref{sub-sec-cla-ast-con} summarizes the implementation
%%%---%of the class \texttt{ASTContext} which is in charge of figuring out
%%%---%context of method declarations. Section~\ref{sub-sec-the-cla-ass}
%%%---%presents the class \texttt{Assignable}, which implement our
%%%---%syntactical rules. Finally, Section ~\ref{sub-sec-usi-the-too}
%%%---%summarizes our experience using the tool \modtool.
%%%---
%%%---
%%%---
%%%---
%%%---%\subsection{Representing Abstract syntax trees with \jml}
%%%---%\label{abo-jml-cla-use}
%%%---
%%%---
%%%---
%%%---
%%%---
%%%---
%%%---\subsection{The class \texttt{ASTContext}}
%%%---\label{sub-sec-cla-ast-con}
%%%---The class \texttt{ASTContext} represents the context of a method
%%%---declaration. A method context consists of the representation of its
%%%---assignable locations, its parameters and its local variables. This
%%%---representation is based on the definition~\ref{def-mod}. Additionally
%%%---we store the package, and the class in which the method
%%%---occurs and for each class we store which class it extends.
%%%---
%%%---These contexts are passed as parameter to the methods of the class
%%%---\texttt{Assignable}, which implements the rules presented in
%%%---Section~\ref{sec-che-for-ass-cla}.
%%%---
%%%---
%%%---
%%%---
%%%---\subsection{The class \texttt{Assignable}}
%%%---\label{sub-sec-the-cla-ass}
%%%---
%%%---The class \texttt{Assignable} implements the assignable rules as
%%%---static methods. To each static method one passes the expression
%%%---\texttt{e} currently analyzed, the context \texttt{cxt} of the method
%%%---where the expression \texttt{e} occurs and the contexts \texttt{cxts}
%%%---of all known classes.
%%%---
%%%---For example, when the expression \texttt{e} is an assignment
%%%---$e_1\:\texttt{=}\:e_2$, it has to be checked that $e_1$ belongs to set
%%%---of assignable expressions of the method
%%%---(\texttt{\_in\_PRIME($\cdots$)}) and that it obeys the
%%%---\textsf{modFE} rules on $e_1$ (\texttt{\_modFE($\cdots$)}) and the
%%%---\textsf{mod} rules on $e_2$ (\texttt{\_mod($\cdots$)}).
%%%---\begin{alltt}
%%%---public static boolean _mod(AST e, ASTContext cxt, Vector cxts)\verb!{!	
%%%---  boolean tempRes;	
%%%---  switch(e.getType())\verb!{!	
%%%---    case JavaTokenTypes.ASSIGN:{\it // =}
%%%---      tempRes = _in_PRIME(e.getFirstChild(),cxt,cxts) &&
%%%---       _mod(e.getFirstChild(),currentCxt,cxts) &&
%%%---       _mod(e.getFirstChild().getNextSibling(),cxt,cxts);
%%%---      return tempRes;
%%%---    \vdots 
%%%---  \verb!}!
%%%---\verb!}!
%%%---\end{alltt}
%%%---
%%%---
%%%---
%%%---
%%%---\subsection{Experiences using \modtool}
%%%---\label{sub-sec-usi-the-too}
%%%---To test the usability of our approach, we have applied \modtool\ to
%%%---the specifications that we wrote earlier for an electronic purse case
%%%---study~\cite{CatanoH02a}. This case study consists of .. classes,
%%%---containing .. methods all together. The specifications are written in
%%%---\escj, thus the assignable clauses are relatively simple and do not
%%%---contain constructs as \fieldsof\ and \reach. When writing and
%%%---improving the specificaitions, we found several times that we had
%%%---forgotten to mention a variable in an assignable clause, but even
%%%---after correcting this ....
Finally we present our experiences in the process of using \modtool
for checking the specification of the case study Gemplus' electronic
purse~\cite{CatanoH02a}. 




\section{Conclusion and future work}
\label{sec-con-and-fut-wor}
This paper presents a method to do an efficient check on assignable
clauses. In particular, this method can be used to do a quick check on 
side-effect freeness.
This method works on a syntactic basis: 
for each \java\ construct a rule is defined which checks that every
assignment only assigns to variables that are declared as 
\emph{assignable}. There are some limitations to our approach, in
particular it does not work in a context with aliasing. In such a
case, formal verification is necessary, but applying our method first
helps to find the very common simple mistakes, thus allowing to
concentrate on the essentials when doing formal verification.
%, our
%method can miss the variable modified. Several solutions have been
%proposed for solving this problem. In~\cite{Leino97} a solution with
%\emph{Data groups} is proposed, which enables modular checking.

%Our rules, and the implementation we are done, allow one to check for
%side-efects efficiently, as no too much is spend in it. 

\subsubsection{Future work}
Our method does not keep track of newly allocated memory. Therefore,
it will reject for example the following method specification:
\begin{verbatim}
class C {int x;}
class D {
  //@ modifies \nothing;
  m() {
   C c = new C();
   c.x = 3;
  }
}
\end{verbatim}
To overcome this limitation, future work is to extend our method to extend the
assignable clause with the (primitively typed) fields of newly
allocated method.

\jml\ provides so-called model variables, which allows to define
specification-only variables and relate them to concrete variables by
using \texttt{represents} or \texttt{depends} clauses,
following~\cite{Leino97}. These model variables also can be mentioned
in assignable clauses. It is future work to extend the tool to
appropriately handle this.

%On the other hand, in the process of specifying abstract classes, some
%times is useful to define fields
%which \emph{represents} or \emph{depends} of certain properties of those
%classes. Due to the nature of the abstract classes it becomes not
%convenient to define these fields in an implementation
%level. \jml\ provides the construct \texttt{model} to overcome this
%difficulty. This construct allow one to define fields in a
%specification level, hence they do not have to be implemented. Besides 
%this construct, \jml\ provide the constructs
%\texttt{depends} and \texttt{represents}. Our specification does not do any 
%analyses of this type as ``it is reasonable to wait until we clarify
%the precise semantics of dependencies''\footnote{Text taken from an
%email from Gary Leavens.}.

Finally, in the end we hope to be able to give better results in the
context of aliasing, for example by returning a warning if a possibly
aliased variable is changed.

\bibliographystyle{alpha}
\bibliography{../specification}

\end{document}
