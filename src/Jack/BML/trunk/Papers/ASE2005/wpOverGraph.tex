

\subsection{Bytecode Weakest Precondition}\label{wpGraph}

The verification being modular proof, obligations are generated for every method separately. The verification condition generation 
is trivially defined over the acyclic execution graph of a method. 
The \wpi \ function runs in a backwards  direction starting from the blocks that do not have predecessors upto reaching the entry point
 instruction.  Calculating the precondition of the sequential part of the block $\blockm{i}$ 
 which we denote with  $\blockSeq{i}$  is standard - the precondition of an instruction  is inferred from the precondition of the previous.
The  interesting part is how the postcondition  for $\blockSeq{i}$ , i.e. the precondition of the last block instructions is inferred 
(i.e. jumps, return, loop end ) see for the definition of basic block section ~\ref{abstrCntrFlow}.  In the following we overload the function 
symbol \wpi applying to sequence of bytecode instructions.
The postcondition for the sequential part of a block  $\blockSeq{i}$ (without its last instruction )   is defined inductively 
 as follows :
 \begin{defn}[Block postcondition  $\blockPost{i}$]\label{post}
 If $\blockm{i}$ ends with instruction $i_s$ then :
 \begin{itemize} 
 \item  if  $i_s$ is a loop end instruction \wpi \footnote{those \ instructions \ are detected when building the acyclic  graph and the 
 backedge is "replaced"  by the corresponding \ invariant \ \invariant  } and whose invariant is \invariant: 
 $$ \blockPost{i} =  \wpi(  i_s, \invariant)  $$ 
 \item if  $i_s$ =  \instr{if\_cond n} \\
$$
\blockPost{i} =  \\
\left\{
\begin{array}{l}
cond( \stack{\counter}, \stack{\counter - 1} )   \Rightarrow \\
\phantom{ \counter }   \wpi(\blockSeq{n}, \blockPost{n} ,\excPost )[ \counter \leftarrow \counter -2 ]  \\
\wedge \\
 not ( cond( \stack{\counter}, \stack{\counter - 1} )  )   \Rightarrow \\
\phantom{ \counter}    \wpi(\blockSeq{i+1}, \blockPost{i+1},\excPost ) [ \counter \leftarrow \counter -2 ]      \\                             
\end{array}
\right.
$$
\item if  $i_s$ =  \instr{goto n} \\ 
$\blockPost{i} =  \wpi(\blockSeq{n}, \blockPost{n} , \excPost )$ \\

\item  if  $i_s$ =  \instr{athrow ExceptionType} \\ 
$$
\blockPost{i }  \\ 
 =   \left\{ \begin{array}{l} 
		\jmlKey{subtype}( \jmlKey{ typeof}(\stack{\counter} )) \\
		(\oplus (Exc_{i}^{thrown} \rightarrow \excPost(Exc_{i}^{thrown}) ) \\
		\phantom{\oplus Exc^{thrown} } [\counter \leftarrow  0] \\ 
		\phantom{ \oplus Exc^{thrown} } [EXC \leftarrow  \stack{\counter} ] )_{i=0}^{n}   \\
		(\oplus Exc_{i}^{handled(i)} \rightarrow \wpi(\blockm{i}, \blockPost{i}, \excPost )\\
		\phantom{\oplus Exc^{handled(i)} }  [\counter \leftarrow  0] \\ 
		\phantom{ \oplus Exc^{handled(i)}}   [EXC \leftarrow  \stack{\counter} ] )_{i=0}^{m} \\
		\end{array} \right.
$$
As we are doing partial evaluation the only thing we know about the exception thrown is that it is 
on the top of the stack $\stack{\counter}$. We keep track of all the possible postconditions upto
discovering the type of the thrown exception and then identifying the adequate predicate 
that must hold after the execution of the \instr{athrow} instruction.
\item  if  $i_s$ = \instr{return} \\
$$
   \blockPost{i } = \psi[\result \leftarrow \stack{\counter} ]    
$$
where $\psi $ is the specified method postcondition.
\item  else  
 $$ \blockPost{i} =  \wpi(  i_s , \wpi( \blockSeq{s+1} ,  \blockPost{ s  +1} ) )  $$ 
 \end{itemize}
\end{defn}

%\Paragraph{Subroutines}
Subroutines are treated by inlining them. As we stated in the beginning we assume that the bytecode has been certified by a java
bytecode verifier thus our analysis for identifying  subroutine is not a problem.  
\todo{example}

%\begin{center} \texttt{wp(instruction\_list;instruction, $\psi$)} = \texttt{wp(instruction\_list, wp(instruction, $\psi$))} \end{center}
%For example for the sequence of instructions of the block starting with $\tt{instr_{18}}$ from Figure~\ref{blockBC} we calculate
%\begin{center}
%\texttt{wp( iload\_2; ireturn,  $\ulcorner \backslash$\texttt{result} == $\backslash$ old(n)div 2 $\urcorner$ )}
%\\
%=
%\\
%\texttt{wp(iload\_2, wp( ireturn, $\ulcorner \backslash$\texttt{result} == $\backslash$ old(n)div 2 $\urcorner$ ))} \end{center}
%Thus the weakest precondition for the block starting at $\tt{instr_{18}}$ at figure~\ref{blockBC} is
%\texttt{local(2) $\ulcorner$==$\urcorner$ $\ulcorner$ $\backslash$ old(n)div 2 $\urcorner$}

