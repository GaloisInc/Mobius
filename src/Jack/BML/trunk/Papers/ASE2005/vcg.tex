\section{Verifying Java Bytecode Programs}\label{vcg}
Every method proof oblogations - pre ==> wpi 

The verification procedure has those features:
\begin{itemize}
\item modular, design by contract verification, in particular every method is verified separately method calls being translated to their specification 
\item it supports all Java sequential instructions except for floating point arithmetic instructions and 64 bit data (\java{long} and \java{double} types)
\item definition of the calculus for unstructured programs over the method's control flow graph
\item the specific features like exceptions and exception handling, objects, references and subroutines
\item the bytecode specification language discussed in the previous section \ref{bcSpecLg}. Every method has a specification written 
in BCL- pre- and postconditions, assertions at particular program point among 
which loop invariants (if there is nothing special specified the specification by default (preconditions, postconditions and invariants taken to true) is taken into account). The verification procedure assumes that the bytecode is specified enough, i.e. we do not try to infer specification, as we assume that they are compiled from the source program
\item the verification procedure does not trust neither the bytecode specification, nor the bytecode; both of the cases - wrong 
specification or incorrect implementation will result in verification conditions that are not provable 
\end{itemize}
