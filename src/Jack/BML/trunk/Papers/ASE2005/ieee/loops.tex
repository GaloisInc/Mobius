\subsection{Loops}
 We assume that a method's bytecode is provided with sufficient specification and in particular loop invariants. Under this assumption, we can ``cut'' the control flow graph at every program point where an invariant must hold and ``place'' at that point the invariant. 
The ``cut'' points correspond to the backedges in a method's bytecode which can be identified using standard techniques (\cite{ARUCom1986}).


     
\subsection{Exceptions and Subroutines}

Exceptions are treated by identifying the instruction at which they start. A normal Java compiler should supply for every method
a \textbf{Exception\_Table} which contains data structures describing the compilation of every implicit (in presence of subroutines)
 or explicite exception handler: the instruction at which the compiled exception handler starts, the protected region (its start and end instruction indexes), the exception type.
Subroutines are treated by inlining. This is not a problem under the assumption that the bytecode has been certified by a bytecode verifier.
