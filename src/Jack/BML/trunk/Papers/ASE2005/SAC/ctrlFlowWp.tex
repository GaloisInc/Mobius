\begin{itemize}
\item \textbf{Loops}
%\subsection{Loops}
Indentifying loops on bytecode and source programs is different because of their different nature --- 
the first one lacks while the second has structure. While on source level loops correspond to loop statements,  
on bytecode level we have to analyse the control flow graph in order to find them.
 The analysis consists in looking for the backedges in the control flow graph using standard techniques as described in \cite{ARUCom1986}. 
  
 We assume that a method's bytecode is provided with sufficient specification and in particular loop invariants.
 Under this assumption, we build an abstract control flow graph where the backedges are replaced by
 the corresponding invariant. We apply the \wpi \ function over the abstract version of the control flow graph which generates verification conditions for the 
preservation and initialization of every invariant in the abstraction garph. 


     
%\subsection{Exceptions and Subroutines}
\item \textbf{Exceptions and Subroutines}
Exception handlers are treated by identifying the instruction at which the handler compilation starts. The JVM specification mandates 
that a Java compiler must supply for every method an \textbf{Exception\_Table} attribute that contains data structures describing the compilation of
 every implicit (in the presence of subroutines) or explicit exception handler: the instruction at which the compiled exception handler starts,
 the protected region (its start and end instruction indexes), and the exception type the exception handler protects from. Thus, 
for every instruction \instr{ins} in method \method~ which may terminate exceptionally on exception \texttt{Exc} the exceptional function
 $\excPost$ (Fig.\ref{wpInv}) returns the \wpi \ predicate of the exceptional handler protecting \instr{ins} from \texttt{Exc} if such a handler exists.
Otherwise, $\excPost$ returns the specified exceptional postcondition for exception \texttt{Exc} as specified in the specification of
method \method.

Subroutines are treated by abstract inlining\footnote{NB: we do not transform the bytecode. It is rather the \wpi \
 function that treats subroutines as if the subroutines were inlined}. First, the instructions of every subroutine
 are identified. %To this end, we suppose that the bytecode has
% been certified by a bytecode verifier which guarantees
%that there are no recursive subroutines.
To this end, we assume that the code has passed the bytecode verification and that every subroutine terminates with a \instr{ret} 
instruction (usually, the compilation of subroutines ends with a \instr{ret} instruction but Java 
compilers can sometimes also generate a subroutine ending with a jump). Thus, by abstract inlining, we mean that
 whenever the \wpi~function is applied to an instruction \instr{jsr}  \texttt{ind}, a postcondition $\psi$ and an exceptional postcondition function $\excPost$, its precondition  $\wpi^{\instr{jsr} \ \tt{ind}}$ is calculated as follows:
the \wpi~ is applied to the bytecode instructions that represent the subroutine which starts at instruction \texttt{ind},
 the postcondition $\psi$ and the exceptional postcondition function  $\excPost$.
% This actually results in the weakest predicate 
%$\wpi^{\instr{jsr} \ \tt{ind}}$ of the subroutine starting at index \texttt{ind} and which guarantees that after its execution 
%$\psi$ will hold in the normal case, otherwise if the subroutine terminates on exception \texttt{Exc} then $\excPost(\texttt{Exc})$ will hold.
  
\end{itemize}
