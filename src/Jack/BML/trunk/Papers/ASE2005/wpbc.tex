
\section{Weakest Precondition Calculus For Java Bytecode}\label{verifCond}
As we mentioned before the research in program verification has been targeting for a long time to structured program languages (\cite{WPCDS},~\cite{DisDij}). 
The interest in low level program logic and verification is a fairly recent fashion, due to the wide spread use of mobile technology.
The presented bytecode logic is defined in terms of weakest precondition calculus. We assume 
that the bytecode program has passed the bytecode verification procedure, i.e. it is well typed 
(every instruction is applied to operands of the correct type) and well formed 
(e.g. no jumps to an un-existing bytecode index). The Java bytecode verification field is  well researched and 
for more information \cite{Ljbc}, for example, is a detailed overview of the state of the art in the domain. 
The logic considered here targets to express properties that are more complex than well typedness.

The logic has those features:
\begin{itemize}
\item modular, design by contract verification, in particular every method is verified separately method calls being translated to their specification 
\item it supports all Java sequential instructions except for floating point arithmetic instructions and 64 bit data (\java{long} and \java{double} types)
\item definition of the calculus for unstructured programs over the method's control flow graph
\item the specific features like exceptions and exception handling, objects, references and subroutines
\item the bytecode specification language discussed in the previous section \ref{bcSpecLg}. Every method has a specification written 
in BCL- pre- and postconditions, assertions at particular program point among 
which loop invariants (if there is nothing special specified the specification by default (preconditions, postconditions and invariants taken to true) is taken into account). The verification procedure assumes that the bytecode is specified enough, i.e. we do not try to infer specification, as we assume that they are compiled from the source program
\item the verification procedure does not trust neither the bytecode specification, nor the bytecode; both of the cases - wrong 
specification or incorrect implementation will result in verification conditions that are not provable 
\end{itemize}

Verification of unstructured programs has been first defined by Floyd ~\cite{F67amp}. The definition of the bytecode weakest 
precondition is straightforward.  It is defined over the control flow graph of the program and has two levels of definitions - 
 the first one is the set of rules for single Java bytecode instructions and the second one takes into account how control
 flows in the bytecode. A related problem is how the cycles in the control flow are treated. 
As we just mentioned we assume that every method is specified enough, i.e. if there are cycles the corresponding 
loop invariant is present. This gives us the right to ``cut'' the cycles in the graph at the program point where the invariant must hold. 
The ``cut'' generates an abstract control flow graph which is acyclic and over which the verification conditions are generated.  

In the rest of the section we describe the bytecode logic  and how the verification conditions are generated -
 the weakest precondition rules for single Java bytecode instructions in subsection \ref{wpInstr}, 
 the method abstract control flow graph is explained in subsection ~\ref{abstrCntrFlow}, 
 the definition of the weakest precondition over the abstract control flow graph is discussed in ~\ref{wpGraph} where how exception handling and subroutines
are described.



\input wpSingleInstr.tex
\input abstractCntrFlow.tex
\input wpOverGraph.tex

%\begin{center} \texttt{wp} : \texttt{STMT} $\longrightarrow$ \texttt{Predicate} $\longrightarrow$ \texttt{Predicate}\end{center}





