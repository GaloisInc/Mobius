\section{Purse Specification}
\label{sec-pur-spe}

\subsection{Some interesting aspects and problems}
This sections presents some interesting problems found in the process
of doing assertions for the purse application. These cases highlight
some commun programming problems and they show how an verification
tool can be help us to find them. 

\subsubsection{The method {\tt \bf isGreaterEqualThan}}

The \texttt{Decimal} class is a purse class and it allows us to
represent a floating point number as
composing by a decimal part and a integer part. These concepts are
represented by the variables \texttt{intPart} and \texttt{decPart}
respectively. This class is shown in the figure~\ref{fig-cla-dec}. The
method \texttt{isGreaterEqualThan} intends to decide when the decimal
represented by \texttt{this} is greater or equal than its parameter
\texttt{d}. Therefore, by mean of a \textsc{Esc/Java} assertion, we
have established a post-condition corresponding to this
specification. This work is done by an \texttt{ensures} comment, which
is shown on the line $8$. \\

After running the \textsc{Esc/Java} tool on this method, a warning
signal is activated. This warning suggest that this post-condition
will not be satisfied by the method. The problem is found on the line
$18$. This line must be replaced by \texttt{decPart >=
d.getDecPart()}. In this way, the condition asserted by the clause
\texttt{ensures} will be satisfied.


\begin{center}
\begin{figure}[hbt]
\rule{\linewidth}{0.3mm}
\rule{0em}{0.1ex}
\begin{tabbing}
pub\=lic\=cla\=ssD\=ecimale\=xtendsObject\kill
\emph{1.}\>$\mathtt{package com.gemplus.pacap.utils\ ;}$ \\
\\
\emph{2.}\>$\mathtt{public\ class\ {\bf Decimal}\ extends\ Object\{}$ \\
\>\>\vdots     \\
\emph{3.}\>\>{\it /*@ spec$\_$public */} $\mathtt{private\ short\ intPart\ =\
(short) 0\ ;}$ \\
\emph{4.}\>\>{\it /*@ spec$\_$public */} $\mathtt{private\ short\ decPart\ =\ (short)\ 0\ ;}$ \\
\>\>\vdots \\ 
\emph{5.}\>\>{\it /*@}  \\
\emph{6.}\>\>\>{\it //modifies $\backslash$nothing} \\
\emph{7.}\>\>\>{\it requires d $!=$ null;} \\
\emph{8.}\>\>\>{\it ensures $\backslash$ result $==$ (intPart $>$ d.intPart
$||$} \\
\emph{9.}\>\>\>\>{\it (intPart $==$
d.intPart $\&\&$ (decPart $==$ d.decPart $||$ decPart $>$ d.decPart))) ;} \\ 
\emph{10.}\>\>{\it */} \\
\emph{11.}\>\>$\mathtt{public\ boolean\ {\bf isGreaterEqualThan}(Decimal\ d)\{}$ \\
\emph{12.}\>\>\>$\mathtt{boolean\ resu\ =\ false\ ;}$ \\
\\
\emph{13.}\>\>\>$\mathtt{if(intPart\ >\ d.getIntPart())}$ \\
\emph{14.}\>\>\>\>$\mathtt{resu\ =\ true\ ;}$ \\
\emph{15.}\>\>\>$\mathtt{else\ if\ (intPart\ <\ d.getIntPart())}$ \\
\emph{16.}\>\>\>\>$\mathtt{resu\ =\ false\ ;}$ \\
\emph{17.}\>\>\>$\mathtt{else\ if(intPart\ ==\ d.getIntPart())\{} $        \\
\emph{18.}\>\>\>\>$\mathtt{if((decPart\ >\ d.getDecPart())\ ||\ (decPart\ >\
d.getDecPart()))}$ \\
\emph{19.}\>\>\>\>\>$\mathtt{resu\ =\ true\ ;}$ \\
\emph{20.}\>\>\>\>$\mathtt{else\ if\ (decPart\ <\ d.getDecPart())}$\\
\emph{21.}\>\>\>\>\>$\mathtt{resu = false;}$ \\
\emph{22.}\>\>\>$\mathtt{\}}$ \\
\emph{23.}\>\>\>$\mathtt{return\ resu\ ;}$ \\
\emph{24.}\>\>$\mathtt{\}}$ \\
\>\>\vdots \\
\emph{25.}\>$\mathtt{\}}$ 
\end{tabbing}
\caption{Piece of {\tt Decimal} class}
\label{fig-cla-dec}
\rule{\linewidth}{0.3mm}
\end{figure}
\end{center}






\subsubsection{The {\tt \bf final} variable}
The \texttt{Annee} class allows to represent a \textit{year} in the
\texttt{purse} application. Figure~\ref{fig-cla-ann} presents its
source code with \textsc{Esc/Java}assertions. This class declares two 
static variables called \texttt{MIN} and \texttt{MAX}, which represent
the minimum and maximum year allowed respectively. This class declares 
also an invariant which relates these values. The static method
\texttt{check} allow us to verify that its parameter is a valid
year. \\

The figure~\ref{fig-cla-dat} shows the \texttt{Date} class. This class 
allows the purse to represent a date$:$ \texttt{jour}(day),
\texttt{mois}(month), \texttt{annee}(year). We have established some
invariants for this class, which allow the purse application to place
each of them between valide intervals. The method \texttt{setDate}
allow us to assign some values to the internal variables as long as
these variables have valide values, i.e, these variables verify the
conditions expressed by the clauses ensures. In this way, if the
pre-condition is satisfied, the internal variables will be assigned
with the values of their parameters. Otherwise, the method will raise
an \texttt{DateException} exception. This abnormality condition is
declared by the \texttt{exsures} clause. \\

\textsc{Esc/Java} complains when it finds a declaration such as$:$ \\
\mbox{\tt date.setDate((byte)1, (byte)1, (byte)110);} (where date is a
\texttt{Date} instance). The warning message signals that its third
parameter does not verify the condition established by the clause
\texttt{requires} belonging to the method \texttt{setDate} (line
$5$). This warning is desployed in spite of this call respect the
requires condition, i.e, ($110>=99\ \&\&\ 110<= 127 $). \\

The problem is happend due to the wrong declaration of \textsc{MIN}
and \textsc{MAX} variables belonging to the \texttt{Annee}
class. These variables were not declared as final~\footnote{{\sc
Java} does not allow to change in runtime the value of a final {\tt
final} variable.}. Thus, their values could be changed in runtime by
mean of a direct assignation~\footnote{In fact, due to these variables
are declared are \texttt{public}} and the pre-condition would be not
satisfied. \\


The declaration variables as \texttt{static} (commun for all instance)
and \texttt{final} (which can not be changed) to prevent us of doing
any assumption that will be not satisfied by an application. This kind 
of control is carried out by \textsc{Esc/Java}. \\


\begin{center}
\begin{figure}[htb]
\rule{\linewidth}{0.3mm}
\\[2.0ex]
\begin{tabbing}
pub\=lic\=cla\=ssD\=ate  \kill
$\mathtt{package\ com.gemplus.pacap.utils\ ;}$ \\
 \\
$\mathtt{public\ abstract\ class\ {\bf Annee}\ extends\ Object\ \{}$ \\
\>{\it //@ invariant MIN $<=$ MAX} \\
 \\ 
\>$\mathtt{public\ static\ byte\ MIN\ =\ (byte)99\ ;} $\\
\>$\mathtt{public\ static\ byte\ MAX\ =\ (byte)127\ ;} $\\
\\
\\
\>{\it /*@} \\
\>\>{\it //modifies $\backslash$nothing ;} \\
\>\>{\it requires true ;} \\
\>\>{\it ensures $\backslash$result $==$ (j $>=$ MIN $\&\&$ j $<=$ MAX) ;} \\
\>\>{\it //exsures (RuntimeException)false ;} \\
\>*/ \\
\>$\mathtt{public\ static\ boolean\ {\bf check}(byte\ j)\ \{} $ \\
\>\>$\mathtt{return\ ((j\ >=\ Annee.MIN)\ \&\&\ (j\ <=\ Annee.MAX))\ ;}$  \\
\>$\mathtt{\}} $ \\
$\mathtt{\}} $ \\
\end{tabbing}
\caption{{\tt Annee} class}
\label{fig-cla-ann}
\rule{\linewidth}{0.3mm}
\end{figure}
\end{center}





\begin{center}
\begin{figure}[hbt]
\rule{\linewidth}{0.3mm}
\\[2.0ex]
\begin{tabbing}
pub\=lic\=cla\=ssD\=ate\=ext\=endsObject  \kill
$\mathtt{public\ class\ {\bf Date}\ extends\ Object\ \{}$ \\

\>{\it /*@} \\
\>\>{\it invariant jour $>=$ Jour.MIN $\&\&$ jour $<=$ Jour.MAX ;} \\	
\>\>{\it invariant mois $>=$ Mois.MIN $\&\&$ mois $<=$ Mois.MAX ;} \\
\>\>{\it invariant annee $>=$ Annee.MIN $\&\&$ annee $<=$ Annee.MAX ;} \\
\>{\it */}
\\
\\
\>{\it /*@ spec$\_$public */} $\mathtt{private\ byte\ jour\ =\ Jour.MIN\ ;}$ \\
\>{\it /*@ spec$\_$public */} $\mathtt{private\ byte\ mois\ =\ Mois.MIN\ ; }$ \\   
\>{\it /*@ spec$\_$public */} $\mathtt{private\ byte\ annee\ =\ Annee.MIN\ ;}$ \\
\\ 
\\
\emph{1. }\>{\it /*@} \\ 
\emph{2. }\>\>{\it modifies jour, mois, annee ;} \\
\emph{3. }\>\>{\it requires j $>=$ Jour.MIN  $\&\&$ j $<=$ Jour.MAX ;} \\
\emph{4. }\>\>{\it requires m $>=$ Mois.MIN  $\&\&$ m $<=$ Mois.MAX ;} \\
\emph{5. }\>\>{\it requires a $>=$ Annee.MIN $\&\&$ a $<=$ Annee.MAX ;} \\
\emph{6. }\>\>{\it ensures jour $==$ j $\&\&$ annee $==$ a $\&\&$ mois $==$ m ;} \\
\emph{7. }\>\>{\it exsures (DateException) false ;} \\
\emph{8. }\>{\it */} \\
\>$\mathtt{public\ void\ {\bf setDate}(byte\ j,\ byte\ m,\ byte\ a)\ throws\ DateException\{}$ \\
\>\>{\it // check the day } \\        
\>\>$\mathtt{if(!Jour.check(j)) \{}$ \\
\>\>\>$\mathtt{DateException.throwIt(DateException.ERREUR\_JOUR)\ ;}$ \\
\>\>$\mathtt{\}else\ \{}$ \\
\>\>\>{\it // check the month} \\
\>\>\>$\mathtt{if(!Mois.check(m))\ \{}$\\
\>\>\>\>$\mathtt{DateException.throwIt(DateException.ERREUR\_MOIS)\ ;}$ \\
\>\>\>$\mathtt{\}else\ \{}$ \\
\>\>\>\>{\it // check the year} \\
\>\>\>\>$\mathtt{if(!Annee.check(a)) \{}$ \\
\>\>\>\>\>$\mathtt{DateException.throwIt(DateException.ERREUR\_ANNEE)\ ;}$ \\
\>\>\>\>$\mathtt{\}else\ \{}$ \\
\>\>\>\>\>{\it $\slash \slash$ all is good} \\
\>\>\>\>\>$\mathtt{jour\ =\ j\ ;} $\\
\>\>\>\>\>$\mathtt{mois\ =\ m\ ;} $\\
\>\>\>\>\>$\mathtt{annee\ =\ a\ ;} $\\
\>\>\>\>$\mathtt{\}}$ \\
\>\>\>$\mathtt{\}} $\\
\>\>$\mathtt{\}} $\\
\>$\mathtt{\}} $\\
\>$\vdots$ \\
$\mathtt{\}}$ \\
\end{tabbing}
\caption{{\tt Date} class}
\label{fig-cla-dat}
\rule{\linewidth}{0.3mm}
\end{figure}
\end{center}


\subsubsection{The {\tt \bf AccessCondition} invariant}
The figure~\ref{fig-cla-acc} shows the class \texttt{AccessCondition}
class. This class models the access conditions for the purse
commands. These access conditions are represented by the variables
\texttt{FREE}, \texttt{LOCKED}, \texttt{SECRET$\_$CODE} et
\texttt{SECURE$\_$MESSAGING}. The internal variable \texttt{condition} 
may take one of these values. According to the purse
specification~\cite{BMGL00}, the \texttt{AccessCondition} class have
to verify the invariant$:$ \\

\begin{tabbing}
//@ invariant \=condition == FREE || condition == LOCKED \kill
{\it $\slash\slash$@ invariant condition $==$ FREE $||$} \\
\>{\it condition $==$ LOCKED $||$} \\
\>{\it condition $==$ SECRET$\_$CODE $||$} \\
\>{\it condition $==$ SECURE$\_$MESSAGING $||$} \\
\>{\it condition $==$ (SECURE$\_$MESSAGING $|$ SECRET$\_$CODE)} \\
\end{tabbing}

This invariant establishes that the variable \texttt{condition} only
takes a value declared in this class. This invariant is rejected by
the \textsc{Esc/Java} compiler. The problem is found on the line
$15$. There, its constructor initialises the variable
\texttt{condition} to a value which have not been considered by the
class \texttt{AccessCondition}. Apart from that, strangely this
variable have been initialised to $FREE$ on the line $7$ and to $0$
in the constructor.




\begin{center}
\begin{figure}[hbt]
\rule{\linewidth}{0.3mm}
\rule{0em}{0.1ex}
\begin{tabbing} 
pac\=kag\=eco\=m.gemplus.pacap.purse; \kill 
\emph{1.}\>$\mathtt{package com.gemplus.pacap.purse;}$\\
\\
\emph{2.}\>$\mathtt{public\ class\ {\bf AccessCondition}\ extends\ Object\{}$\\

\emph{3.}\>\>$\mathtt{public\ static\ final\ byte\ FREE\		=\ (byte)1\ ;}$\\
\emph{4.}\>\>$\mathtt{public\ static\ final\ byte\ LOCKED\		=\ (byte)2\ ;}$\\
\emph{5.}\>\>$\mathtt{public\ static\ final\ byte\ SECRET\_CODE\	=\ (byte)4\ ;}$\\
\emph{6.}\>\>$\mathtt{public\ static\ final\ byte\ SECURE\_MESSAGING\	=\
(byte)8\ ; }$\\
\\
\emph{7.}\>\>{\it /*@ spec$\_$public */} $\mathtt{private\ byte\ condition\ =\ FREE\ ;}$\\
\\
\emph{8.}\>\>{\it /*@ }\\
\emph{9.}\>\>\>{\it requires true ;}\\
\emph{10.}\>\>\>{\it ensures condition $==$ (byte)0 ;}\\
\emph{11.}\>\>\>{\it ensures $\backslash$fresh(this) ;}\\
\emph{12.}\>\>{\it */} \\
\emph{13.}\>\>$\mathtt{public\ {\bf AccessCondition}(\ ) \{}$\\
\emph{14.}\>\>\>$\mathtt{super()\ ;}$\\
\emph{15.}\>\>\>$\mathtt{setCondition((byte)0)\ ;}$\\
\emph{16.}\>\>$\mathtt{\}}$\\

\>\>\vdots \\

\emph{17.}\>\>{\it /*@ } \\
\emph{18.}\>\>\>{\it modifies condition ;} \\
\emph{19.}\>\>\>{\it requires true ;} \\
\emph{20.}\>\>\>{\it ensures condition $==$ c ;} \\
\emph{21.}\>\>{\it */} \\
\emph{22.}\>\>$\mathtt{public\ void\ {\bf setCondition}\ (byte\ c)\{}$\\
\emph{23.}\>\>\>$\mathtt{condition\ =\ c\ ;}$\\
\emph{24.}\>\>$\mathtt{\}}$\\

\>\>\vdots \\
\emph{25.}\>$\mathtt{\}}$
\end{tabbing}
\caption{Piece of {\tt AccessCondition} class}
\label{fig-cla-acc}
\rule{\linewidth}{0.3mm}
\end{figure}
\end{center}


