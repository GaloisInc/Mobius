%intro
The purpose of this chapter is to remind the reader the basic principles in the formal verification  
of Java programs. Because the concepts on Java source and bytecode verification are similar, we have decided   
to give here an overview of them on Java source. We hope that this will be a gentle
 introduction to the rest of the thesis especially for  those
which are not acquainted with Java bytecode.

 As we already stated in the introductory chapter, a formal program verification
relies on three elements: a specification language in which to express the requirements 
that a program must satisfy, a verification procedure for generating verification conditions
whose validity implies that the program respects its specification and finally,
 a system to decide the validity of the verification conditions.
In this chapter and in the rest of thesis, we shall focus on the first two components. 
In particular,in Section \ref{source}, we shall present a Java like programming language supporting the most important features of Java.
 We also present JML, the de facto Java specification language  tailored to Java. This will be done in Section \ref{BCSLprelim}. 
Section \ref{javaVerif:verifStyles} presents a discussion about  the different approaches for program verification using program logic.
In Section \ref{pog:wpSrc}, we shall define over it a  verification condition calculus.
