
%\section{Preliminaries} 
% \subsection{JACK}
%Jack \footnote{ http://www-sop.inria.fr/everest/soft/Jack/jack.html} is a plugin for the eclipse \todo{referenced - the eclipse site} integrated development environment for Java. Originally, Jack statically checks, using formal methods, source programs against their JML specification. The tool can interface with several theorem provers (AtelierB, 
%Simplify, Coq, PVS ) \todo{references}. We have upgraded the tool with implementations of the JML compiler and the bytecode verification condition generator. Jack has a user-friendly interface which makes it easy for use by developers that do not have a large background in formal methods.


\section{Overview of JML} \label{BCSLprelim}
% what is JML
JML~\cite{JMLRefMan} (short for Java Modeling Language) is a behavioral interface specification 
language tailored to Java applications which follows the design-by-contract approach (see~\cite{M97oos}).

% useful language
Over the last few years, JML has become the de facto specification language for
 Java source code programs. Several case studies have demonstrated that JML can be used to specify realistic
industrial examples, and that the different tools allow to find errors
in the implementations (see~e.g. \/~\cite{BreunesseCHJ04}). One
of the reasons for its success is that JML uses a Java-like
syntax.  Other important factors for the success of JML are its expressiveness and
flexibility.

% tool support 
JML is supported by several verification tools.
 Originally, it has been designed as a language of the runtime assertion checker~\cite{jmlrac} created by Yoonsik Cheon and G.T. Leavens. 
The JML runtime assertion checker compiles both the Java code and the JML specification into executable bytecode and thus, 
in this case, the verification consists in executing the resulting bytecode. Several static checkers based 
on formal logic exist which use JML as a specification language. Esc/java~\cite{escjava}  whose first version used a subset 
 of JML \footnote{the current version of the tool esc/java 2  supports almost all JML constructs} is among the first tools supporting JML.
Among the static checkers with JML  are
the Loop tool developed by the Formal group at the University of Nijmegen,
the Jack tool developed at Gemplus, the Krakatoa tool developed by the ProVal group at Inria, France.
The tool Daikon \cite{ECG01DDL}  tool uses a subset of JML for detecting loop invariants by run of programs.
 A  detailed overview of the tools which support JML can  be found in~\cite{BurdyCCEKLLP03}.
In the following, we shall proceed with the description of the basic features of JML which correspond basically to the
JML Level0 subset of JML. 


% technical details
\subsection{JML expressions}\label{javaVerif:JML:expression}   
    Expressions manipulated in JML  
    are side-effect free Java expressions, extended with
    specification-specific keywords. JML specifications are written as comments so they are not
    visible by Java compilers. The JML syntax is close to the Java syntax: JML extends the Java
    syntax with few keywords and operators. 
    Special JML operators are, for instance, \jmlKey{$\backslash$result} which stands for the value that a method returns if it is not
    void, the \jmlKey{$\backslash$old(expression)} operator 
    designates the value of \jmlKey{expression} in the prestate of a method and is usually
    used in the method's postcondition, \jmlKey{$\backslash$typeof(expression)} which stands for the dynamic type of  \jmlKey{expression}, etc.


\subsection{Method contracts and method intra specification}\label{javaVerif:JML:methodContracts}  
Method contracts consists of the conditions upon which relies  and the conditions that the method guarantees.
The conditions upon which a method relies are expressed in specifications as the method precondition, i.e. \ the method requires them to hold 
in its pre state. The conditions which a method guarantees are expressed in specifications as the method postcondition, i.e. 
 they express what the method guarantees to the methods that may invoke it. Method contracts 
 are visible by the other methods in the program as they inform the other methods what they must establish when calling them current  method
and what the current method guarantees them. Moreover, JML provides a third component in the method contract, the so called frame condition which 
introduces the locations that can be modified by the method. Frame conditions are actually desugared as part of the postcondition as we shall see later in subsection 
\ref{javaVerif:JML:frame}. For an insight information on method contracts, the reader may refer to \cite{M97oos}.

Specifying methods may involve writing specification which is not visible by ``the outside world'' (the other methods in the program) but which are useful 
for the program verification process. An example for such a specification are the loop invariants, which express a property which holds at the borders of a loop iteration.
Moreover, here we shall allow that a loop is also accompanied by a frame condition which as in the case of method frame conditions
 specifies which locations may be modified in a loop. Such an extension of JML is used in the tool Jack \cite{BRL-JACK}. We provide a more 
insight discussion about them later in subsection \ref{javaVerif:JML:frame}. 

For introducing method preconditions, postconditions and method frame conditions in JML the keywords \jmlKey{requires},
 \jmlKey{ensures} and \jmlKey{modifies} keywords are used  respectively. Also for loop invariants and loop frame conditions,
 the keywords \jmlKey{loop\_invariant} and \jmlKey{loop\_modifies} are used.
Fig.~\ref{replaceSrc} gives an example of a Java class that models a list stored in a private array field. 
The method \texttt{replace} will search in the array for the first occurrence of the object \texttt{obj1} passed as first argument
 and if found, it will be replaced with the object passed as second argument \texttt{obj2} and the method will return true; otherwise it 
returns false. Thus the method  specification between lines 5 and 9 which exposes the method contract states the following.
First the precondition (line 5 ) requires from any caller to assure that the instance variable \texttt{list} is not \texttt{null}.
The frame condition (line  6) states that the method may only
modify any of the elements in the instance field \texttt{list}. The method postcondition (lines 7---9) states the method will return 
\texttt{true} only if  the replacement has been done.
The method body contains a loop (lines 17---22) which is specified with a loop fame condition and a loop invariant (lines 13---16).
  The loop invariant (lines 14---16) says that all the elements of the list that are inspected up to now are 
 different from the parameter object \texttt{obj1} as well as the local variable \texttt{i}  is a valid index in the array \texttt{list}.
 The loop frame condition (line 13)  states that only the local variable \texttt{i} and any element of the array field \texttt{list} may be modified in the loop.

\begin{figure}[ht!]
\begin{lstlisting}[frame=trbl] 
public class ListArray {
  
  private Object[] list;
  
  //@ requires list != null;
  //@ modifies list[*];
  //@ ensures \result ==(\exists int i; 
  //@         0 <= i && i < list.length && 
  //@         \old(list[i]) == obj1 && list[i] == obj2);
  public boolean replace(Object obj1,Object obj2){
    int i = 0;
    
    //@ loop_modifies i, list[*];
    //@ loop_invariant i <= list.length && i >=0 
    //@ &&(\forall int k;0 <= k && k < i ==> 
    //@    list[k] != obj1 && list[k] == \old(list[k]));
    for (i = 0; i < list.length; i++ ){
      if ( list[i] == obj1){
        list[i] = obj2;
        return true;	
      }
    }
    return false;
  }
}
\end{lstlisting}
\caption{\sc class \mbox{\rm \lstinline!ListArray!} with JML annotations} 
\label{replaceSrc}
\end{figure}

\subsection{Ghost variables}\label{javaVerif:JML:ghost} 
 JML also allows the declaration of special JML variables, that are used only for specification purposes. 
 These variables are declared in comments with the \jmlKey{ghost} modificator and may be used only in specification clauses. Those variables 
 can also be assigned. Ghost variables are usually used  for expressing properties which can not be expressed with the program variables.

 We illustrate the utility of such variables through an example. 
 Let us have a class \texttt{Transaction}  which manages transactions in the program as shown in  Fig. \ref{bml:ghost}.  
 The class is provided with a method for opening transactions (\texttt{beginTransaction}) and 
 a method for closing transactions (\texttt{commitTransaction}). Suppose that we want to guarantee that in an application 
 there are never nested transactions. A possibility is to declare a method \texttt{getTransactionDepth} which counts the number
 of open transactions and use dynamic checks  for guaranteeing that there are no nested transactions. But this would require to modify the code which is not desirable. 
Another approach would be to use the method  \texttt{getTransactionDepth}  as far as its execution
 does not affect the program state (i.e. it is pure) in the specification 
 e.g. \ the precondition  of method \texttt{beginTransaction} will be \texttt{//@ requires getTransactionDepth() == 0}.
However,  the treatment of pure methods is  complicated and its semantics is still not well established.
 Ghost variables can be used for specifying the no nested transaction property in a simple way.
  The specification of the methods \texttt{beginTransaction } and \texttt{commitTransaction}
 models the property for no nested transactions via the ghost variable
 \texttt{TRANS} (declared on line 3) which keeps track if there is  a running transaction or not. In particular, 
 if the value of  \texttt{TRANS} is 0 then there is no running transaction and if it has value 1 then there is
 a running transaction.  Thus, when the method  \texttt{beginTransaction } is invoked the precondition (line 5) requires
 that there should be no running transaction and when the method is terminated the postcondition guarantees (line 6) that there is already a transaction running.
 We can also remark that the variable  \texttt{TRANS} is set to its new value (line 8) in the  body  \texttt{beginTransaction}.
% Note that this high level property is difficult to express without the presence of the ghost variable \texttt{TRANS}.  
Note that this is a simple way for specifying this property without modifying the code. 

\begin{figure}[ht!]
\begin{lstlisting}[frame=trbl] 
public class Transaction {

  //@ ghost static private int TRANS = 0; 
  
  //@ requires TRANS == 0;
  //@ ensures TRANS == 1;
  public void  beginTransaction() {
    //@ set TRANS = 1;
     ...
  }

  //@ requires TRANS == 1;
  //@ ensures TRANS == 0;
  public void  commitTransaction() {
    //@ set TRANS = 0; 
     ...
  }
}
\end{lstlisting}
\caption{\sc specifying No Nested Transaction property with ghost variable} 
\label{bml:ghost}
\end{figure}




\subsection{Light and heavy weight specification. Multiple specification cases}\label{javaVerif:JML:lightHeavy}
A useful feature of JML is that it allows two kinds of method specification, a \light  \ and \heavy \ weight specification. 
An example for a \light \  specification is the annotation of method \texttt{replace} (lines 5---9) in Fig. \ref{replaceSrc}. The specification in 
the example states what is the expected behavior of the method and under what conditions it might be called.
The user, however in JML, has also the possibility
to write detailed method specifications which allow 
to describe what are the different cases
on which a method might be called and what is the behavior of the method in
 every  case.
 This   specification style is called a \heavy{} weight specification.
 Heavy weight specification  are introduced by the JML keywords \jmlKey{normal\_behavior} and \jmlKey{exceptional\_behavior}. As the keywords
suggest every of them specifies a specific normal or exceptional behavior of a method.  (see \cite{PD06LBR}). 

The keyword  \jmlKey{normal\_behavior} introduces a precondition, frame condition and postcondition 
such that if the precondition holds in the prestate of the
 method then the method will terminate normally and the postcondition will hold in the poststate.

% example for heavy weight specification
An example for a \heavy{} weight specification is given in Fig. \ref{bml:heavySp}. In the figure, method \texttt{divide} has 
two behaviors, one in case the method terminates normally (lines 11---14) and the other (lines 17---20) in case the method
 terminates by throwing an object reference of \texttt{ArithmeticException}.
 In the normal behavior case, the exceptional postcondition is omitted  specification as by default if the precondition (line 12 )
 holds this assures that no exceptional
 termination is possible. Another observation over the example is that the exceptional behavior is introduced with the JML keyword \jmlKey{also}. 
 The keyword  \jmlKey{also} serves for introducing every new behavior of a method except the first one. Note that the keyword \jmlKey{also} 
 is used in case a  method overrides a method from the super class. In this case, the method specification (\heavy \ or \light \ weight) is preceded by the keyword \jmlKey{also} 
 to indicate that the method should respect also the specification of the super method.

\begin{figure}
\begin{lstlisting}[frame=trbl]
public class C {
    int a;
    
    //@ public instance invariant a > 0 ;
    
    //@ requires val > 0 ;
    public C(int val){
       a = val ;
    }
   
    //@ public normal_behavior
    //@ requires b != 0;
    //@ modifies a;
    //@ ensures  a == \old(a) / b;  
    //@
    //@ also 
    //@ public exceptional_behavior
    //@ requires b == 0;
    //@ modifies \nothing;
    //@ exsures (ArithmeticException) a == \old(a);
    public void divide(int b) {
        a = a / b;
    }
}
\end{lstlisting}
\caption{\sc An example for a method with a heavy weight specification in JML} \label{bml:heavySp}
\end{figure}
%desugaring
In a light weight specification style the specifier might omit some of the specification. For instance, in Fig. \ref{bml:heavySp},
the constructor \lstinline!C! is provided only with a precondition. On verification time, the missing part of an
 incomplete light weight specification is set to its default values.  The default value for omitted precondition and postcondition 
in a light weight specification is  \jmlKey{true} and for frame condition is \nothing\footnote{the default value for frame conditions
may depend on our choice. For instance, in Jack the developer can set the default value for frame conditions to \nothing{} or \everything  }.



A \heavy{} weight specification is a  syntactic sugar which is suitable for the specification process and  makes easier the understanding
 of the specified code. The verification of  a heavy weight specification involves appropriate desugaring of such specification corresponding to its 
semantics.  
The meaning of \jmlKey{normal\_behavior}  clause guarantees that the method will not terminate on an exception and thus the exceptional postcondition 
for any kind of exception, i.e.\ for the exception class  \texttt{Exception} is  \jmlKey{false}. Similarly, an
 \jmlKey{exceptional\_behavior} clause guarantees that if its precondition holds then the method terminates on exception and the specified 
exceptional postcondition will hold, i.e. \ its \ensures{} clause is always    \jmlKey{false}. 
In case a method has several specification cases, then one of the  preconditions of its specification cases must hold when the method starts execution. 
%That is why  if a method has several specification cases, then the preconditions of the specification cases must be disjoint.  
Moreover, if the precondition of a particular method specification case holds in the method prestate then the postcondition of this specification case must hold.

Fig.\ \ref{bml:Desugar}  shows  the desugared version of the method specification from  Fig. \ref{bml:heavySp}. As we can see, the specification of the constructor 
\lstinline!C! contains all the specification clauses, where the postcondition is set to the default postcondition \lstinline!true!.
The specification of the method \lstinline!divide! does not contain the syntactic sugar \jmlKey{normal\_behavior} and
 \jmlKey{exceptional\_behavior} but explicitly specifies the behavior of the method in the two cases. Particularly, it contains now 
two specification cases surrounded by the tags  \lstinline!{| |}!. The first one corresponds to the 
\jmlKey{normal\_behavior} case, i.e.\ the exceptional postcondition is set to \lstinline!false! and  the second specification case corresponds to the 
\jmlKey{exceptional\_behavior}, i.e.\ the normal postcondition is set to  \lstinline!false!.

\begin{figure}
\begin{lstlisting}[frame=trbl]
public class C {
    int a;
    
    //@ public instance invariant a > 0 ;
    
    //@ requires val > 0 ;
    //@ modifies \nothing;
    //@ ensures  true;
    public C(int val){
       a = val ;
    }
   
    //@ requires   b != 0 || b ==0;
    //@ {|
    //@ requires b != 0;
    //@ modifies a;
    //@ ensures  a == \old(a) / b;  
    //@ exsures (Exception) false
    //@|}
    //@ also 
    //@ {|
    //@ requires b == 0;
    //@ modifies \nothing;
    //@ ensures false;
    //@ exsures (ArithmeticException) a == \old(a);
    //@ |}
    public void divide(int b) {
        a = a / b;
    }
}
\end{lstlisting}
\caption{\sc Desugared specification of the example from Fig. \ref{bml:heavySp} } \label{bml:Desugar}
\end{figure}




\subsection{Frame conditions}\label{javaVerif:JML:frame}

As we can see in Fig. \ref {bml:heavySp} JML allows for specifying a modifies clause for methods. 
The \modifies{} clause or also the frame condition declares which are the locations that a method may modify.
The example in Fig. \ref{replaceSrc} shows that we also allow frame condition in the loop introduced by the keyword \loopMod.
The semantics of the method and loop frame conditions is almost the same except for the fact that a loop frame condition  may mention 
method parameters or local variables.  Also, in the case for method frame condition their semantics is part of the method postcondition 
while the semantics of  loop frame conditions can be encoded as part of the corresponding loop invariant.

In the following, let us look what is the meaning of the loop frame condition.
The semantics of a loop modifies list can be encoded as part of the invariant as a condition which states
that all the locations not mentioned in the  loop modifies list must have the same value at the beginning and at the end state of a loop iteration.
We illustrate this by an example. If a loop is specified with an invariant \invariant{} and the list of modified locations contains the expression \lstinline!a.f!:
\begin{lstlisting}[frame=trbl]
//@ loop_modifies a.f;
//@ loop_invariant I;
 while (e) {
(*@ \ldots @*)
} 
\end{lstlisting}
 then the augmented invariant resulting from the desugaring of the modifies clause is  : 
\begin{lstlisting}[frame=trbl]
//@ loop_invariant  I  &&
//@  \forall ref; ref !=  a ==> ref.f == ref.f(*@'@*) && 
//@  \forall g \forall ref; g != f ==>  ref.g == ref.g(*@'@*)
while (e) {
(*@ \ldots @*)
} 
\end{lstlisting}
The first conjunct of the new invariant is the specified invariant \lstinline!I!. The second conjunct expresses the fact the value in the field  \lstinline!f!
for every object which is different from the reference  \lstinline!a!  must have the same value in the beginning (\lstinline!ref.f!)and in the end of an iteration (\lstinline!ref.f!').
 Note that if an expression has the same value at every  beginning and end of a loop iteration then it has the same value at the beginning and end of the loop.   
The third conjunct expresses the fact that the value of any other field  \lstinline!g! different from  \lstinline!f! for any object 
 \lstinline!ref! changes its value in the beginning and end of a loop iteration.
This third part makes the verification of modifies frame conditions difficult as one has to enumerate explicitly all the fields of all the classes which are reachable.
We do not discuss further the desugaring of the method modifies clauses in the method's postcondition as it is the same as for loop modifies clauses. 

A very similar treatment of the  method and loop modifies clauses is done in Jack both on the Java source and Java bytecode verification condition generators.
For method modifies,  the difference  from the above formula is that the third conjunct is missing as we already said it is difficult to enumerate the 
whole heap. For loop modifies they are not verified but only assumed for the same reason of keeping the number of verification conditions reasonable. 


 In \cite{marche05tphols}, the authors report for a change in the formalization of the heap in the verification tool Krakatoa which allows
to treat also in a similar way as Jack the modifies clauses for methods. In this article, the authors also describe how to calculate read write effects of methods which
 allows  to construct a reasonable part of the  third conjunct above. 

May be here we may point out that not only the assertion of modifies clauses play a role in a verification condition but also their assumption.
Assuming  modifies clauses on method invocation  and over the verification conditions for a loop allow to an appropriate verification condition generator to
 initialize properly the variables or locations that are not modified by the loop or method. Consider the following example for a method
 \lstinline!m! which calculates the sum of the k natural numbers in a loop where k is a method parameter:


\begin{lstlisting}[frame=trbl]

class A { 
  int f;
  //@ requires k >= 0;
  //@ modifies this.f;
  //@ ensures \result == this.f *(this.f+1)/2;
  public int m(int k) {
    this.f = k;
    int sum = 0;
	    
    //@ loop_modifies sum, i;
    //@ loop_invariant sum == (i*(i+1))/2 && i<=k;
    for (int i = 0; i < k; i++) {
      sum = sum + i;
    }
    return sum;
  }
}
\end{lstlisting} 

 Before calculating the sum the instance variable \lstinline!a.f! is set to the value of the parameter  \lstinline!k!.
 Assuming then the modifies clause of the loop allows to initialize properly the value of \lstinline!a.f! and to  establish that
 the program  is correct w.r.t. the specification. Of course, the verification condition condition generator 
 must propagate the verification conditions up to the entry of the method body. 
 Such a verification  scheme based on a weakest precondition predicate transformers allows for writing smaller specifications. 
 Of course, here we could have used a stronger invariant (add in the loop invariant the fact that \lstinline!a.f == k! )
 but in real programs the number of variables is usually larger than in this small example and the specification burden of the programmer will be inadmissible.
 Actually, both verification condition generators on  Java source and bytecode in Jack propagate the verification conditions related to method correctness
 up to the method body entry point. 

\subsection{Class specification} \label{javaVerif:JML:classSpec}
JML can be used to specify not only methods but also properties of a class or interface. 
A Java class may be specified with an invariant or history constraints. An invariant of a class is a predicate which holds at all visible states of
  every object of this class ).
A visible state of an object is basically one of these cases:
\begin{enumerate}   
   \item at the poststate of a constructor where this object is a receiver
   \item at the prestate of a finalizer of that object 
   \item at the prestate and poststate of every method where this object is a receiver
   \item in a state when no method, constructor or finalizer of this object  is in progress 
\end{enumerate} 
For a detailed discussion on visible states, the reader may refer to the JML reference manual \cite{JMLRefMan}.
 An invariant may be either static (i.e. talks only about static fields) or instance (talks about instance fields). 
 The class \texttt{C}  in Fig.\ref{bml:heavySp} has also an instance invariant which states that the instance variable \texttt{a} is
 always greater than \texttt{0}. A Class history constraints is similar to a class invariant but it does not describe one visible
 state but rather relates  a visible state and any visible state that occurs later in the execution. It can be seen as a transitive relation between the 
 visible states of an object.

%\begin{description} 
 % \item [Class invariants] 
        Let us now see how instance class invariants   can be expressed as 
	method pre and postconditions (the treatment of static class invariants and history constraints has the same features thus, we do not discuss
       them further but the reader may refer to \cite{JMLRefMan}). From the first case of visible states described above, an instance invariant  must
	be established by the constructor of this class which means that it must be part of the constructor's postcondition. 
	From the  second case it follows that the invariant of an object must be part of the precondition of the finalizers of this object.
	From the third case it follows that every method (not a constructor neither a finalizer) which has as a receiver an object
	must have as part of its pre and postcondition the object's invariant. The fourth case of a visible state implies that every method which does not have
	as a receiver the object must preserve this object's invariant.
	This actually means that every constructor must preserve  (has as pre and postcondition)  the invariant of
	all objects except for the receiver object's invariant which must be only established
	in the poststate. Similarly, a finalizer of an object must preserve the invariants of all the other objects and include in its precondition 
	the object's invariant. Also every method must preserve
	the invariants of the objects of all classes.
	This actually makes the verification of class invariants a difficult task as the invariants of all objects in 
	the heap must be mentioned in every method's pre and postcondition (except for the particular cases of constructors and finalizers).  
	When applications are large, the number of verification conditions may be  practically unmanageable and thus the verification process may be very hard.
	Another problem is that the verification is non modular, i.e. extending an application with a new class requires to verify the whole application once again 
	in order to establish that the application does not violate the invariant of the objects of the newly added class.
	That is why verification tools may make some compromises.  
	In JACK, for instance, constructors in a class \texttt{C} preserve the invariants of all objects of class \texttt{C} and establish the invariant of the receiver
	object of the constructor. Non constructor and non finalizer methods declared in class \texttt{C} preserve the invariant of all objects of class \texttt{C}.   
	 
	A recent approach which relies on alias control type systems \cite{DietlMueller05,Naumann-vstteFAC} facilitates the verification of invariants and 
	makes it modular. This technique relies on a hierarchy  relation of ``owner and owned'' objects  in the heap where only owners of an object may modify its contents.
	The alias control is guaranteed through ownership type systems which check that only an owner of a reference can modify its contents. 
	Such technique  can be used also for the sound treatment of layer object structures \cite{Mueller-Poetzsch-Heffter-Leavens06}.
	
	
	

 % \item [Class invariants] 	A class invariant (\ClassInv)  is a property that must hold at every visible state of the class. This means that a
   %      class invariant must hold when a method is called and also must be established at the end of a method execution. 
 %	A class invariant must be established in the poststate 
 %	of the constructor of this class.
 %	Thus the semantics of
 %	a class invariant is part of the pre and  postcondition of every method and is a part of the postcondition of the constructor of
 %	the class to which it refers.
 %	
        
  %\item [History constraints] A class history constraint (\ClassHistoryConstr) gives a relation between the pre and poststate of every method in the class. 
   %     A class history constraint thus can be expressed as a postcondition of every method in the class.
  
%\end{description}     

 
%  \subsection{Behavioral subtyping for overriding methods}\label{javaVerif:JML:subtyping}
%         Method overriding appears in case when in a subclass a method is reimplemented.  For instance, in Fig. \ref{assertLang:lang:inherit},
% 	we give an example of a class \lstinline!B! which extends class   \lstinline!A! and overrides method  \lstinline!m!. 
% 	To deal with method overriding the virtual machine uses dinamically bounding, i.e. it is on execution time that the virtual machine decides which method to execute.
% 	If we go back to the figure, this means that the invokation of method   \lstinline!m! on the object reference of type \lstinline!A!e%         may stand for the method  \lstinline!m! declared in class \lstinline!B! or the one declared in \lstinline!A! because the dynamic type of the
% 	reference stored in the field  \lstinline!a! may be \lstinline!A! or \lstinline!B!. Because  method \lstinline!m! from
%         class \lstinline!B! can be called whenever method  \lstinline!m! from  \lstinline!A! may be called, we want that the former has the same 
% 	return and argument types as the latter. We also require that method \lstinline!m! from
% %         class \lstinline!B! behaves like method \lstinline!m! from \lstinline!A! or as we say that method is a behavioral subtype of \lstinline!A!. 
% 	
%         This is expressed by the two covariant conditions over their postconditions:
% 	\begin{itemize}
% 	   \item the precondition of the  overriden method must  imply
% 	         the precondition of the overriding method 
% 	   \item the postcondition of the overriding method must imply 
% 	         the postcondition of the overriden method.
% 	\end{itemize}
% 	We adopt the specification inheritance technique which allows to encode behavioral subtyping in the specification of the 
% 	overriding method. The methodology consists in expressing the specification of the overriden method 
% 	as part of the specification of the overriding method. The reader may have a look for a detailed description of
% 	specification inheritance  in  the article \cite{Dhara-Leavens95} of K.Dhara and G.T.Leavens. Here, we will 
% 	illustrate the specification inheritance through the example given in Fig. \ref{assertLang:lang:inherit}. We show in the second part of 
% 	the figure the specification of method  \lstinline!m!  from class \lstinline!B! which will be used in the verification of the method.
% 	Its precondition is the disjunction of the its specified postcondition and the precondition of the method it overrides. On the other hand, its new postcondition expresses
% 	that depending what held in the precondition of the  method then either its postcondition holds or the postcondition of the method it
% 	overrides. Note that the new specification of \lstinline!m!  declared in class \lstinline!B!  respects the two covariant condition described above.
% 	This approach allows to encode directly as preconditions and postconditions the subtyping relation. 
% 	
% 
% \begin{figure}[ht!]
% 
% \begin{lstlisting}[frame=trbl]
% class A {
%  //@ requires Pre1;
%  //@ ensures  Post1;
%  int m(){
%  }
%}
%
%class B extends A {
%  //@ requires Pre2;
%  //@ ensures  Post2;
%  int m(){
%  }
%}
%
%class C{
%  A a;
%  void n (){
%    ...
%    a.m()
%  }
%} 
%\end{lstlisting}
%
%\begin{lstlisting}[frame=trbl]
%%class B extends A {
%  //@ requires Pre1 || Pre2;
%  //@ ensures  (\old(Pre2) ==> Post2) && 
%               (\old(Pre1)==> Post1) ;
%  int m(){
%%  }
%}
%
%\end{lstlisting}
% 
% \caption{\sc An example for a specification inheritance} \label{assertLang:lang:inherit}
%\end{figure}



 % why do we discard the rest. 
% We abstract from the rest of the JML grammar, as it boils down to JML
% expressions and  predicates. 


