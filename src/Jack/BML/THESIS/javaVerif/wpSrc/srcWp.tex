


\section{Weakest precondition predicate transformer for  Java like source language } \label{pog:wpSrcGeneral}
We proceed here with the presentation of the weakest predicate transformer function. We would like first to remark
that our predicate transformer deals both with normal and exceptional termination of programs.
Among the first works on exceptional termination in weakest precondition predicate transformers is the work of R.Leino \cite{leino94semantics}.
We adopt here a similar approach as the above cited work.  Thus, the predicate transformer
takes as arguments a statement  $\stmt$, a normal postcondition $\normalPostSrc$ and an exceptional postcondition function  $\excPostSrc$.

The intended meaning of the predicate $WP$ that 
should be calculated by a weakest predicate transformer function against  $\stmt$,
$\normalPostSrc$ and  $\excPostSrc$. The predicate    $WP$ must be 
 such that if it holds in the pre state of $\stmt$ and   if $\stmt$ terminates normally then $\normalPostSrc$  holds in the poststate and
 if $\stmt$ terminates on exception \Exc{} then $\excPostSrc(\Exc)$ holds. 
The function $\excPostSrc$ \ returns the predicate $\excPostSrc (\mbox{\rm\texttt{Exc}}) $ that must hold in a particular program point if
 at this point an exception of type \mbox{\rm\texttt{Exc}} is thrown.

In the following, we will use function updates for the exceptional postcondition of the form
 $\update{\excPostSrc}{\Exc'}{ P } $ over $\excPostSrc $ which are defined in the usual way:
$$
\update{\excPostSrc}{\Exc'}{ P }( \Exc )  = 
       \left\{\begin{array}{ll} 
         P & if \ \Exc  <: \Exc'  \\
         \excPostSrc(\Exc ) & else 
     \end{array}\right.$$

Note that allowing for exceptional termination in the programming language,
 makes the definition of the \wpName{} more complicated  than standard definitions of the weakest precondition predicate. 
Usually, those assume that programs  terminate normally. 
Consider, for instance the standard rule of the weakest precondition predicate transformer over the conditional statement:
$$  \begin{array}{l}
\wpName((  \expressionSrcRel  )  \Mythen  \{ \stmt_1 \}   \Myelse \ \{ \stmt_2 \}, \normalPostSrc) = \\
  \expressionSrcRel \Rightarrow \wpName( \stmt_1, \normalPostSrc) \wedge
  \neg \expressionSrcRel \Rightarrow \wpName( \stmt_2, \normalPostSrc) \end{array}
$$
This rule is correct as far as the evaluation of the expression $\expressionSrcRel$ evaluates normally. But in our settings, this is not always the case.
For instance, if $ \expressionSrcRel$ is the relational expression $\mbox{\rm\texttt{a}}.\fieldd > 0$ and    \mbox{\rm\texttt{a}} evaluates to \Mynull{}
the above rule does not capture the semantics of the exceptional termination. As we shall see in the following, the definition presented here  will allow us to manage in  appropriate way
 exceptional termination\footnote{although here, we rule out these cases, such a definition also allows for dealing with side effects of expression evaluation like change
 the value of a variable or location during the evaluation as for instance is the case for Java expressions like \lstinline!i++!}.
 In particular, we extend the \wpName{} function also over expressions which makes the predicate calculus sensible to the exceptional and normal termination of the 
expression evaluation. A similar  weakest precondition predicate transformer underlines the verification condition of
 the Jack verification framework.


In the following, we shall limit ourselves to a simpler specification than described in the previous Section \ref{BCSLprelim}.
Particularly, we shall assume that  methods are provided only with one specification case for the sake of clarity.
 Extending the definitions here to multiple specification  cases should not present a major difficulty.

We shall also assume here that frame conditions are correct, i.e. we shall not desugar them  into the appropriate specification 
(postconditions in the  case of method frame conditions and loop invariants in the case of  loop frame conditions) as described in
subsection \ref{javaVerif:JML:frame}. 

In the following, we give a brief discussion on the substitution over expressions and afterwords, we concentrate
 on the definition of a weakest precondition predicate transformer function over the source expressions and statements.
