
\subsubsection{Statements}\label{pog:wpSrc:wpStmt}

In the following, we discuss  the weakest precondition predicate transformer for  control statements.
 We will not give an exhaustive overview of the \wpName{} rules for statements as they are standard 
but we will rather  concentrate on few which we consider illustrative. 

Fig. \ref{pog:wpSrc:wpStmt:withoutExc} gives  the rule for  field assignment and conditional control transfer
statements which do not deal with exceptions. They are defined in a standard way. 
The rule for field assignment shows that the statement may terminate normally or on an exception. 
In particular, if the  dereferenced object reference $\expressionSrc_1$ does not evaluate to \Mynull{} 
 then  the postcondition $\normalPostSrc{}$ must hold where the value of the field \fieldd{} for the 
evaluation $v_1$ of $\expressionSrc_1$ is changed to the value  $v_2$ of  $\expressionSrc_2$.
If  the  dereferenced object reference $\expressionSrc_1$  evaluates to \Mynull{} then the postcondition 
$ \excPostSrc(\NullPointerExc) $ must hold.
The rule for the conditional statement is also standard.

Let us now look  at the rule \textsf{while} which is slightly different from 
standard rules for \wpName{} for loops as it uses the list \modLoop{}  which stands for the locations that may be modified  in a loop iteration. 
%The difference is in the \modLoop{} framecondition which is a list of the locations that might be modified in a loop iteration. 
%Note that here we consider that this list is correct and thus, we do not verify its validity. 
We assume here that this list is correct, i.e. we assume that a location not mentioned in the list  \modLoop{} of a loop 
will have the same values at the borders of the loop iteration.
 Actually, if we want to verify   it, we can express it as part of the loop invariant. Remind that a loop invariant  is a predicate which must hold
whenever the loop entry is reached. The semantics of a \modLoop{} list can be encoded as part of the invariant as a condition which states
that all the locations not mentioned in the \modLoop{} must have the same value at the beginning and at the end state of a loop iteration.
As we can see in the figure, the conditions resulting from \wpName{} over a loop statement ( which stand for the loop preservation at loop borders 
and that the postcondition holds at the end of the loop execution) are quantified over the 
elements of the \modLoop{} list. This quantification allows to initialize correctly the variables that keep their values unchanged at the loop borders.
Of course, in order to initialize these variables the result of the \wpName{} function for a loop must be propagated up to the program entry point.

Although this complicates slightly the verification condition generator, the loop frame condition is useful as it can work 
on weaker specification than a verification condition generator which does not support it. This actually means that the burden of writing specification 
becomes definitely lighter, as the person which specifies an application must identify not  \textit{the strongest loop invariant }
but a \textit{loop invariant}. Of course, he should also identify  the locations which are modified by the loop (locations that may have different values 
at the end and at the beginning of a loop iteration). Modified locations in a loop is easier than identifying the variables not modified in the loop which 
must be done if we decide not use loop frame conditions. 

\begin{figure}[ht!]
\begin{frameit}
$${\scriptsize 
        \begin{array}{l} 
	   \wpSrcStmt{ \stmt_1;\stmt_2}{\normalPostSrc }{ \excPostSrc } =^{\mbox{\rm\textsf{seq}}} \\
	   \begin{array}{l} 
             \wpSrcStmt{\stmt_1 }{\wpSrcStmt{\stmt_2 }{\normalPostSrc }{ \excPostSrc}} { \excPostSrc} 
           \end{array} \\ \\ \\
    \wpSrcStmt{ \var = \expressionSrc_2}{\normalPostSrc }{ \excPostSrc } =^{\mbox{\rm\textsf{locVarAssign}}} \\
                  \begin{array}{l}   \wpSrcExpr{\expressionSrc_2 }{ 
				    \normalPostSrc \subst{\var}{v}   
				   }{ \excPostSrc}{v}  
		  \end{array} \\ \\ \\
    \wpSrcStmt{ \expressionSrc_1.\fieldd = \expressionSrc_2}{\normalPostSrc }{ \excPostSrc } = ^{\mbox{\rm\textsf{fieldAssign}}}\\
               \begin{array}{l} 
	       \wpSrcStmt{\expressionSrc_1}{  %\\
	        %\phantom{wpiSr}
		 \wpSrcExpr{\expressionSrc_2 }{ %\\
		%\phantom{wpiSr}\phantom{wpiSr}
		%\begin{array}{l}
		    v_1 \neq \Mynull  \Rightarrow % \\
	         %\phantom{wpiSr} 
		  \normalPostSrc \subst{\fieldd}{\update{\fieldd}{ v_1}{  v_2}}   %\\
		 \wedge %\\
		 v_1  = \Mynull  \Rightarrow  %\\
		   %\phantom{wpiSr}   
		   \excPostSrc(\NullPointerExc) 
		   %\end{array}
                   }{ %\\ \phantom{wpiSr}\phantom{wpiSr}  
		   \excPostSrc}{v_2} 
		   }{ \\ \phantom{\expressionSrc_1.\fieldd = \expressionSrc_2 }  
		   \excPostSrc}{v_1} 
		   \end{array} \\ \\ \\

    \wpSrcStmt{ \begin{array}{l} \Myif \  (  \expressionSrc_1 \ \rel \ \expressionSrc_2   )  \Mythen  \{ \stmt_1 \}   \Myelse \ \{ \stmt_2 \} \end{array}}{\normalPostSrc}{ \excPostSrc } 
     =^{\mbox{\rm\textsf{if}}}\\

         \begin{array}{l} 
	 \wpSrcExpr{ \expressionSrc_1 }{ 
	               \wpSrcExpr{ \expressionSrc_2 }{
		         %\begin{array}{l}  
		           v_1 \rel v_2  \Rightarrow \wpSrcStmt{\stmt_1 }{\normalPostSrc }{ \excPostSrc } %\\
			    \wedge \\
			  \phantom{wpiSr\expressionSrc_1wpiSr\expressionSrc_2} \neg v_1 \rel v_2   \Rightarrow \wpSrcStmt{\stmt_2 }{\normalPostSrc }{ \excPostSrc } %\\
	               %\end{array}
		       }{\excPostSrc }{v_2} 
	 } {\\ \phantom{wpiSr\expressionSrc_1}   \excPostSrc }{v_1} 
     \end{array} \\ \\ \\

     \wpSrcStmt{ \while \ (\expressionSrcRel) \ \lbrack \invariant, \modLoop \rbrack \  \do \ \{ \stmt \}}{ \normalPostSrc}{\excPostSrc}=^{\mbox{\rm\textsf{while}}} \\
	      \begin{array}{l} 
	       \invariant \ \wedge\\
	       \forall \  m, m \in \modLoop , \\
	       \invariant \Rightarrow %\\
	 	     %\Myspace    \Myspace 
		     \wpSrcExpr{\expressionSrcRel}{%\\\phantom{wp^{src}} 
		     %\begin{array}{l}  		
		           v = \Mytrue  \Rightarrow \  \wpSrcStmt{ \stmt }{\invariant} {\excPostSrc} \wedge % \\ 
		           v = \Myfalse  \Rightarrow  \normalPostSrc
		     %\end{array}
	      }{ %\\ \Myspace \phantom{wp^{src}} 
	      \excPostSrc}{v}  \end{array} \\ \\ \\

     \wpSrcStmt{ \returnSrc \ \expressionSrc }{ \normalPostSrc}{\excPostSrc} = \\
             \begin{array}{l}   \wpSrcExpr{ \expressionSrc}{ \normalPostSrc \subst{\result }{ v }} { \excPostSrc}{v}  \end{array}\\ \\ \\ 

 \wpSrcStmt{ \throw \ \expressionSrc }{ \normalPostSrc}{\excPostSrc} =^{\mbox{\rm\textsf{throw}}}\\
	       \begin{array}{l} 
	      \wpSrcExpr{\expressionSrc}{%\\ \phantom{wpiSr} 
                  %\begin{array}{l}
		       
	                v = \Mynull \Rightarrow \excPostSrc( \NullPointerExc  )  %\\
			\wedge %\\
			 v \neq \Mynull \Rightarrow%\\
			 %\Myspace
			  \left(\begin{array}{l}  
				 \forall \mbox{\rm\texttt{Exc} } , \\
				 \typeof{ v } <: \mbox{\rm\texttt{Exc} }   \Rightarrow \\
				 \Myspace   \methodd.\excPostSrc( \mbox{\rm\texttt{Exc} }    ) \subst{\EXC}{v}
			  \end{array}\right) 

			  
	          %\end{array}
		  }   
		     { %\\ \phantom{wpiSr}  
		     \excPostSrc}{v} 
	   \end{array}     \\ \\ \\

 \wpSrcStmt{ \try \ \{ \stmt_1 \} \ \catch (  \mbox{\rm\texttt{Exc}} \ c )\ \{ \stmt_2 \} } { \normalPostSrc}{\excPostSrc} =^{\mbox{\rm\textsf{try catch}}}\\
              \begin{array}{l}
	      \wpSrcStmt{ \stmt_1}{ %\\ 
	                 %\phantom{wpiSr} 
			 \normalPostSrc}{ %\\ 
			 %\phantom{wpiSr}
			  \update{\excPostSrc}{ \mbox{\rm\texttt{Exc}} }{\wpSrcStmt{\stmt_2}{ \normalPostSrc}{\excPostSrc} }} 
	      \end{array} \\ \\ \\

\wpSrcStmt{ \try \ \{ \stmt_1 \} \ \finally \ \{ \stmt_2 \} } { \normalPostSrc}{\excPostSrc} =^{\mbox{\rm\textsf{try finally}}}\\
       \begin{array}{l} 
	 \wpSrcStmt{ \stmt_1}{%\\ \phantom{wpiSr}
                     \wpSrcStmt{\stmt_2}{ \normalPostSrc} { \excPostSrc}}{ \\ \phantom{wpSrc\stmt_1}
		     \update{\excPostSrc}{\Exception}{ \wpSrcStmt{\stmt_2}{ 
		                                    \left(\begin{array}{l}
                                                        \EXC \neq \Mynull \Rightarrow \excPostSrc(\Exception) 
							\wedge\\
							\EXC = \Mynull \Rightarrow \excPostSrc(\NullPointerExc)
						     \end{array}\right) } { \excPostSrc }   } }
       \end{array} 

\end{array} } 
$$

\caption{\sc WP for source control statements}
\label{pog:wpSrc:wpStmt:withoutExc}
\end{frameit}
\end{figure}



The control statements related to the exception handling and throwing as well as the finally statements
 have a more particular definition. 
Let us look at the rule for \textsf{try catch} statements.
 %Actually, it is similar to the  rule  \textsf{seq} from  Fig. \ref{pog:wpSrc:wpStmt:withoutExc}, but dual in the way the postcondition modifications.
 The weakest predicate of a try catch statement $ \try \ \{ \stmt_1 \} \ $ 
$\catch (  \mbox{\rm\texttt{Exc}} \ c )\ \{ \stmt_2 \} $  w.r.t. a normal postcondition   $\normalPostSrc$  and exceptional postcondition function $ \excPostSrc$
is the weakest predicate of the try statement $\stmt_1$ w.r.t. the same normal postcondition  $\normalPostSrc$  and the updated exceptional function
$\update{\excPostSrc}{ \mbox{\rm\texttt{Exc}} }{\wpSrcStmt{\stmt_2}{ \normalPostSrc}{\excPostSrc} }$. We can see the rule for the 
\textsf{try catch} statement as  dual to the rule of the  compositional statement where in the latter we rather change the normal postcondition. 
 

The  \textsf{try finally} statement is more complicated. It calculates the precondition of the try statement $\stmt_1$ where it takes as normal postcondition
the precondition of the finally statement $\stmt_2$  calculated upon the initial normal postcondition $\normalPostSrc$ and exceptional postcondition function
 $\excPostSrc$. This expresses the fact that after the try statement terminates normally the finally statement must be executed and the whole statement will terminate 
as the finally statement. 
On the other hand, the exceptional postcondition function passed to the try statement $\stmt_1$ is basically the initial exceptional postcondition function $\excPostSrc$ but updated for the exception
type \Exception{}. It is updated 
 with the precondition of the finally statement $\stmt_2$ calculated of the $\wpNameSrcExpr$ which takes as normal postcondition a predicate which we explain in the following.
The postcondition which must hold in the poststate of  $\stmt_2$  states that   $\excPostSrc(\Exception)$ holds in the normal poststate of $\stmt_2$ if the variable
\EXC{} which stands for the thrown exception object is not \Mynull. It also states that the if \EXC{} is not \Mynull then the exceptional
postcondition  $\excPostSrc(\NullPointerExc)$ in case of a thrown \NullPointerExc{} is thrown.

 The exception type \Exception{} is the super class of all exception types and thus, 
an update of the exceptional postcondition function 
means that for any exception thrown by $\stmt_1$ the exceptional postcondition is actually the precondition of the finally statement $\stmt_2$.
 This also 
corresponds to the semantics of \textsf{try finally}  described earlier. In particular, it says that if the try statement terminates on exception \texttt{E}
the finally statement must be executed. If the finally statement terminates normally, then the  whole statement terminates on exception \texttt{E} and if the finally statement
terminates on exception  \texttt{E'} then the  whole statement terminates on exception \texttt{E'}.
