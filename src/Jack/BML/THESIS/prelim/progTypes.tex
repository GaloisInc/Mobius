\index{defaultValue}
\section{Program types and values}\label{types}
 The types supported by our language are a simplified version
 of the types supported by the JVM.
 Thus, we have a unique simple type : the integer data type \Myint.
 The reference type (\AllRefs) stands for the simple reference types (\reff)
 and array reference types (\reffArr).
 As we said in the beginning of this chapter, the language does not support interface types.

 
$$ \begin{array}{ll}
          \JavaType & ::= \AllRefs \mid \Myint  \\
          \AllRefs  & ::= \reff \mid \reffArr \\
	  \reff     & ::= \ClassSet \\
	  \reffArr  & ::= \JavaType[]	  
   \end{array}  $$


Our language supports two kinds of values : values of the basic type \Myint  \ and reference values   \RefValues. \RefValues{} may be references to
class objects, references to array objects or the special null value which denotes the reference pointing nowhere.
 The set of references of class objects is denoted with \Ref{}, the set of references to array  objects is represented with \RefValuesArr{}
and the null reference value is denoted with \Mynull. The following  definition gives the  formal grammar for values:
 
$$\begin{array}{ll}
             \Values &       ::=  \RefValues \mid i, i : \Myint \\
	     \RefValues &    ::= \Ref \mid \RefValuesArr  \mid   \Mynull \\  %: C , C \in \reff \mid \RefValuesArr  \mid   \Mynull \\
	     %\RefValuesArr & ::= \{  ra_1, \ldots,\}   %: C\lbrack \rbrack,  C\lbrack \rbrack \in \reffArr
 \end{array}$$



Every type has an associated default value which can be accessed via
the function  \defaultValueOnly.  Particularly, for reference types (\AllRefs) the default value is 
\Mynull{} and the default value of \Myint{} type is $0$.
Thus, the definition of the  function \defaultValueOnly{} is as follows:
$$\mbox{\rm\textsf{defVal}} :   \JavaType    \rightarrow   \Values $$
$$ \defaultValue{\anyType} = 
           \left\{\begin{array}{ll}
	      \Mynull & \anyType \in \AllRefs  \\
	       0 &  \anyType = \Myint
	    \end{array}\right. $$

We define also a subtyping relation as follows:

$$\begin{array}{ll}
  \frac{}{ \subtype{C}{C}} &  
  \frac{   C2 = C1.\superClass  }{\subtype{C1}{C2}} \\  
  & \\
  \frac{ C3 = C1.\superClass \   \subtype{C3}{C2} }{ \subtype{C1}{C2} } &
  \frac{}{  \subtype{C1}{\Object}} \\
  & \\
  \frac{   }{\subtype{C[]}{\Object}} &  
  \frac{ \subtype{C1}{C2}}{\subtype{C1[]}{C2[]}} \\
  \end{array}$$
Note that the subtyping relation that we use here is a subset of the Java subtyping relation. However, it is a subset of
the Java's subtyping relation which includes as well interface and abstract class types.
