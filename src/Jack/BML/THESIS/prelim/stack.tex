\subsection{The operand stack}

Like the JVM language, our bytecode language is stack based. This means that every method is supplied with a Last In First Out
 stack which is used for the method execution to store intermediate results.
The method stack is modeled by the partial function \stackOnly \ and the variable
\counterOnly keeps track of the number of the elements in the operand stack. 
 \stackOnly \ is defined for any integer \texttt{ind} smaller than the operand stack counter \counterOnly 
 and returns the value \stackOnly(\texttt{ind})  stored in the operand stack at \texttt{ind}
 positions of the bottom of the stack. When a method starts execution its operand stack is empty and we denote the empty stack
 with \newStack. Like in the JVM our language supports instructions to load values stored in registers or object fields and viceversa.
 There are also instructions that take their arguments from the operand stack \stackOnly, operate on them and push the result on the operand
 stack. The operand stack is also used to prepare parameters to be passed to methods and to receive method results.   
