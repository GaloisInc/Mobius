
 \section{Notation}\label{notation}
 Here we give the semantics of several notations used in the rest of this chapter.
 If we have a function $f$ with  domain type  $A$ and range type $B$
 we note it with $f : A \rightarrow B$. If the function receives n arguments of type $A_1 \ \ldots \ A_n$ respectively
 and maps them to elements of type $B$
 we note the function signature with $f : A_1 * ...*A_n \rightarrow B$.   
 Function updates of function $f$ with n arguments is denoted with $ \update{f}{x_1 \ldots x_n}{y} $ and the definition of such function is :
 $$  \update{f}{x_1 \ldots x_n }{y}(z_1 \ldots z_n) = 
    \left\{\begin{array}{ll}
                  y & if \ x_1 = z_1 \wedge ...\wedge x_n = z_n \\
		  f(z_1 \ldots z_n ) & else
           \end{array}\right.$$
 
 The function \isInListOnly  takes as arguments any list and an object and  returns \textit{true} if the object is in the 
 list and \textit{false} otherwise:
 $$ \isInListOnly : list \ A * A \rightarrow \mbox{ \rm \textit{bool}}$$
 

The empty list is denoted with $\lbrack \ \rbrack$. For any type $A$, the function \addInListOnly takes as argument any list $l: list \ A$ and an object 
$o: A$ and returns a list $l1 $ such that $l1.head = o \wedge l1.tail = l$: 
 
$$ \addInListOnly : list \ A * A \rightarrow  list \ A $$

The function $\textit{inDom}(f, e )$ determines if the element $e$ is in the domain of the function $f$.
