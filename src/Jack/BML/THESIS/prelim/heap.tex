 \newcommand{\HeapSet}{\mbox{\rm \textsf{HeapType}}}
 \newcommand{\heapFields}{\mbox{\rm \textsf{hFld}}}
 \newcommand{\heapArrays}{\mbox{\rm \textsf{hArr}}}
 \newcommand{\heapLocs}{\mbox{\rm \textsf{hLoc}}}

 \subsection{Modeling the Object Heap} \label{heap}
 An important issue for the modelization of an object oriented programming language and its operational semantics
 is the garbage collected memory heap. As the Java Virtual Machine (JVM) specification states, the heap is the
 runtime data area from which memory  for all class instances and arrays is allocated. Whenever a new instance
 is allocated, the JVM returns a reference value that points to the newly created object. 
 We introduce a record type \HeapSet \ which models the memory heap.
 It consists of three components:
 \begin{itemize}
       \item a component  named \heapFields \ which is a partial function that maps field
             structures (of type \FieldSet \ introduced in subsection \ref{clazz} ) into functions from references (\AllRefs)
	     into values (\Values).  
 

       \item  a component \heapArrays \ which maps the components of arrays  into their values

        \item  a component  \heapLocs  \ which stands for the  list of references that the heap has 
 \end{itemize}

 More formally, the data type \HeapSet \ has the following structure:



  $$ \begin{array}{l}
         \forall  \heap : \HeapSet , \\
         \heap = \left\{\begin{array}{ll}  \heapFields &  : \FieldSet \rightarrow (\AllRefs \rightarrow \Values) \\
                                           \heapArrays &  : \reffArr * nat \rightarrow \Values \\
					   \heapLocs   &  : list \ \AllRefs
                    \end{array} \right\}
   \end{array} $$




 A heap object \heap \ must assure that the value of the components \heap.\heapFields \ and \heap.\heapArrays \
 are functions which are defined only for references from the proper
 type and which are in the list of references of the heap \heap.\heapLocs:
 $$\begin{array}{ll}
          \forall  \fieldd : \FieldSet, \forall \Ref{\clazz} : \AllRefs, &  \InDom{ \heap.\heapFields(\fieldd)}{\Ref{\clazz} } \Rightarrow \\
	  & \isInList{\Ref{\clazz} }{ \getLocations{\heap}} \wedge \\
	  & \fieldd.\fieldType = \clazz\\
	  \wedge &  \\
	  \forall \Ref{\anyType\lbrack\rbrack} : \reffArr, &  \InDom{ \heap.\heapArrays}{(\Ref{\anyType\lbrack\rbrack}, i )} \Rightarrow \\
	  & \isInList{\heap.\heapLocs}{  \Ref{\anyType\lbrack\rbrack} }\wedge \\
	  & 0 \leq i < \heap.\heapFields( \length)(\Ref{\anyType\lbrack\rbrack}  )
	 
   \end{array}
  $$



 
% The function $\intersectOnly$ takes two lists $l1$ and $l2$ and returns a list of
% their common elements:
% $$\intersectOnly : list \ \AllRefs *  list \ \AllRefs \rightarrow  list \ \AllRefs   $$


 %The function \getFreshRefOnly returns a reference of type \texttt{C} which is not in the list $l$
 %\ 
 %$$\getFreshRefOnly: list \ \AllRefs  * \JavaClass \rightarrow \AllRefs $$

 %and which has the following property for every list of references $l$ and class type $C$:

 %$$ \getFreshRef{l}{C} = \Ref{C} \Rightarrow \isInList{l}{\Ref{C}} = \mbox{ \rm \textit{false}} $$

 We define an operation \addNewLocationOnly \  which adds a new reference to the list of references in a heap.
 The only change that the operation will cause to the heap \heap \ is to add
 a new reference $\referenceOnly$ to the list of references of the heap \heap.\heapLocs:   

 $$ \addNewLocationOnly : \heap *  \AllRefs   \rightarrow \heap $$

 Formally, the operation is defined as follows:
 $$ \begin{array}{l}
           \addNewLocation{\heap}{\referenceOnly} = \heap' \iff^{def} \\
	      \\
              \heap.\heapLocs = l \wedge \\
   	      \isInList{l}{\referenceOnly} = \mbox{ \rm \textit{false}} \wedge \\
	      \heap'.\heapLocs = \addInList{l}{\referenceOnly}  \wedge \\ 
	      \begin{array}{ll}
                  \forall \fieldd : \FieldSet, \ \forall \referenceOnly : \AllRefs,  & \isInList{l}{\referenceOnly} = \mbox{ \rm \textit{true}}  \wedge \\
                                                                        & \fieldd.\declaredIn =  typeof(\referenceOnly) \Rightarrow \\
									& (\heap.\heapFields ( \fieldd  ))(\referenceOnly) = (\heap'.\heapFields( \fieldd ))(\referenceOnly) \\
                    \wedge & \\
	            \forall \referenceOnly : \reffArr,  & \isInList{l}{\referenceOnly} = \mbox{ \rm \textit{true}}  \wedge \\
                          				& \forall i, 0 \leq i  < \heap.\heapFields(\length)(\referenceOnly) \Rightarrow \\
                                                        & \heap.\heapArrays( \referenceOnly, i  ) = \heap'.\heapArrays(\referenceOnly, i  )
               \end{array} 
    \end{array}$$ 

 Whenever the  field $\fieldd$ for the object pointed by reference
 $\referenceOnly$ is updated  with the value \textit{val},
 the component \heap.\heapFields \ is updated:
 $$ \heap.\heapFields := \update{\heap.\heapFields}{\fieldd}{\update{\heap.\heapFields(\fieldd)}{\referenceOnly}{val}} $$  
 In the following for the sake of clarity, we will use another lighter notation for a field update which do not imply any ambiguities:
 $$ 
  \update{\heap}{\fieldd}{\update{\fieldd}{\referenceOnly}{val}} 
 $$ 



 If in the heap \heap \ 
 the $i^{th}$ component in the array referenced by $\Ref{\anyType\lbrack \rbrack}$ is updated with the new value \textit{val},
 this results in assigning a new value of the component \heap.\heapArrays:
 $$\heap.\heapArrays := \update{\heap.\heapArrays}{(\Ref{\anyType\lbrack \rbrack} , i)}{val} $$ 
 In the following for the sake of clarity, we will use another lighter notation for an update of an array component
 which do not imply any ambiguities:
 $$ 
  \update{\heap}{(\Ref{\anyType\lbrack \rbrack} , i) }{val} 
 $$ 

 If a new object of class $\clazz$ is created in the memory,
 a fresh reference $\Ref{\clazz}$  which points to the newly created object is added in the heap \heap \ 
 and all the values of the field functions that correspond to the fields in class $\clazz$ 
 are updated for the new reference with the default values for their corresponding types.
 We denote the new heap with $\newRef{\heap}{C}$.  
 The formalization of the resulting heap and the new reference is the following:



 $$  \begin{array}{l}
            \newRef{\heap}{\clazz} = (\heap', \Ref{\clazz})     \iff^{def} \\
	    \\
	    \addNewLocation{\heap}{\Ref{\clazz}} = \heap' \wedge \\
	    
            \left\{\begin{array}{ll}
	           \forall  \fieldd : \FieldSet, & \fieldd.\declaredIn = \clazz \Rightarrow \\
                                                 & \heap'.\heapFields := 
			\update{\heap'.\heapFields}{\fieldd}{\update{\heap'.\heapFields(\fieldd)}{\referenceOnly}{ \defaultValue{\fieldd.\fieldType }}}\\                                       \end{array}\right\}
	  
     \end{array} $$





Identically, when allocating a new object of array type whose elements are of type \anyType and length is $l$, we obtain 
a new heap object  $\newArrRef{\heap}{\anyType \lbrack \ \rbrack  }{l} $ which is defined similarly to the previous case: 

 $$  \begin{array}{l}
            \newArrRef{\heap}{\anyType\lbrack \ \rbrack}{l} = (\heap', \Ref{\anyType\lbrack \ \rbrack})     \iff^{def} \\
	    \\
	    \addNewLocation{\heap}{\Ref{\anyType\lbrack \ \rbrack}} = \heap' \wedge \\
            \heap'.\heapFields :=  \update{\heap'.\heapFields}{\length}{\update{\heap'.\heapFields(\length)}{ \Ref{\anyType\lbrack \ \rbrack}}{ l }} \wedge  \\
	    \forall i, 0 \geq i < l  \Rightarrow   \heap'.\heapArrays :=
            \update{\heap'.\heapArrays}{(\Ref{\anyType\lbrack \rbrack} , i)}{ \defaultValue{\anyType }  }
     \end{array} $$


% The list of references of the resulting heap $\newRef{\heap}{\clazz}$ and the heap \heap
% has the following property:

% $$  \intersect{\getLocations{\heap}}{\newRef{\heap}{\clazz}} = \lbrack \Ref{\clazz} \rbrack $$



 
  

 
 

