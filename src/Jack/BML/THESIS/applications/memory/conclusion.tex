% conclusion 

Program logics have traditionally been used to verify functional
properties of applications, but we have shown that such logics 
are also appropriate to enforce security properties including
memory consumption policies. We have shown that program logics
complement nicely existing methods to verify memory consumption,
over which they are superior in terms of the precision of the
analysis (and inferior in terms of automation).


We intend to pursue our work in three directions. Firstly, we would
like to extend our approach to arrays. In principle, it should be
reasonably easy to extend the verification method to arrays; however,
it seems more complicated to extend our inference algorithm to arrays.
The main difficulty here is to provide an estimate of the size of an
array, as it is given by the top value on the operand stack at the
time of its creation. Our intuition is that this can be done using
an abstract interpretation or a symbolic evaluation of the program. 
If we look at the example code below (in source code): 
\begin{lstlisting}[frame=trbl] 
void m(int s){
 int len = s;
 int[] i = new int[len]
} 
\end{lstlisting}
where \verb!len! is a local variable to the method, one can infer by
symbolic computation that its value is the value of the method
parameter. Thus the method can be given the precondition 
\lstinline{Mem + 4*s <= \Max}\footnote{this because an integer in Java is encoded in 4 bytes}.
 In a similar line of work, we would like to extend our
results to concurrency using recent advances in program logics for
multi-threaded Java programs \cite{RDFHL05ess}. Providing an appropriate
treatment of arrays and multi-threading is an important step towards
applying our results to mobile phone applications.



Secondly, we would like to adapt our approach to account for explicit
memory management. More precisely, we would like to consider an
extended language with a special instruction \texttt{free(o)} that
deallocates the object \texttt{o}, and establish the correctness of
our method under the assumption that deallocation is correct, i.e.
that the object \texttt{o} is not reachable from the program point
where \texttt{free(o)} is inserted. By combining our approach with
existing compile-time analysis that infers for each program point
which objects are not reachable, we should be able to provide more
precise estimates of memory consumption.
%\alarm{ref needed}




Thirdly, we intend to apply our technique to other resources such as
communication channels, bandwidth, and power consumption, as well as
to more refined analyses that distinguish between different kinds of
memory, such as RAM or non-volatile EEPROM.  As suggested by the MRG
project~\cite{asp+04:cassis}, it seems also interesting to consider
policies that enforce limits on the interaction between the program
and its environment, for example w.r.t. the number of system calls or
the bounds on parameters passed to them.



