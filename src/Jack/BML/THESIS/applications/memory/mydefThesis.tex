\newcommand{\annotation}{BML}
\newcommand{\variant}{\texttt{variant}}

\newcommand{\fram}[1]{{\left\langle #1 \right\rangle}}
\newcommand{\stf}{{\mathit{sf}}}

\newcommand{\configMem}[2]{{\langle\!\langle #1, #2 \rangle\!\rangle}}
\newcommand{\instrAt}[1]{i_{#1}}


\newcommand{\Mem}{\texttt{MemUsed}}
\newcommand{\Max}{\texttt{Max}}

%%%%%%%%%%%% allocation function
\newcommand{\visited}{\texttt{visited}}
\newcommand{\allocated}[1]{\mbox{\rm\texttt{allocPath}}(#1)}
\newcommand{\allocatedOnly}{\mbox{\rm\texttt{allocPath}}}

%\newcommand{\instanceOfAlloc}[1]{instanceOfAllocates( #1 )}

\newcommand{\allocInstance}[1]{\mbox{\rm\texttt{allocInstance}}(#1)}
\newcommand{\allocInstanceOnly}{\mbox{\rm\texttt{allocInstance}}}
\newcommand{\allocLoop}[1]{\mbox{\rm\texttt{loopConsumption}}(#1)} % function that returns directly the allocations done by a loop : multiplied by the max iterations it can do
\newcommand{\allocMethod}[1]{\mbox{\rm\texttt{methodConsumption}}(#1)} % returns the allocations done in a method
\newcommand{\allocLoopWithEnd}[2]{\mbox{\rm\texttt{allocLoopPath}}(#1 , #2)} % returns the allocations done in a loop for a particular path that starts at the start instruction of a loop and that % ends with an insstruction that leads back to the start instructions

\newcommand{\allocIns}[1]{alloc\_instr(#1)} % function that returns that the allocation by the argument

\newcommand{\numLoop}[1]{\textit{numberLoop}(#1)}
\newcommand{\loopEndsSet}[1]{loopEndSet(#1)}
\newcommand{\loopSet}[1]{loopSet(#1)}
\newcommand{\loopEntry}[1]{entry(#1)} % predicate that says that the instruction is an entry to a loop

\newcommand{\backedge}[2]{backedge(#1,#2)} % the start of the backedge  and the end of the backedge


\newcommand{\atState}[2]{#1^{#2} }





\newcommand{\ensemble}[2]{#1 .. #2}
\newcommand{\maxIter}[1]{ iter^#1 }
\newcommand{\progLoop}[1]{\textit{#1}}
\newcommand{\srcCode}[1]{\texttt{#1}}
