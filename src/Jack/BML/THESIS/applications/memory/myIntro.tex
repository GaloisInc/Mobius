
Memory consumption policies provide a means to control
resource usage on constrained devices, and play an important role
in ensuring the overall quality of software systems, and in
particular resistance against resource exhaustion attacks. Such
memory consumption policies have been previously enforced through
static analysis, which yield automatic bounds at the
cost of precision, or run-time analysis, which incur an overhead
that is not acceptable for constrained devices.

Several approaches have been suggested to date to enforce memory
consumption policies for programs; all approaches are automatic,
but none of them is ideally suited for TPDs, either for their
lack of precision, or for the runtime penalty they impose on
programs:
\begin{itemize}
\item \emph{Static analysis and abstract interpretations:} in such an
approach, one performs an abstract execution of an approximation of
the program. The approximation is chosen to be coarse enough to be
computable, as a result of which it yields automatically bounds on
memory consumption, but at the cost of precision. Such methods are not
very accurate for recursive methods and loops, and often fail to
provide bounds for programs that contain dynamic object creation
within a loop or a recursive method;


\item \emph{Proof-carrying code:} here the program
comes equipped with a specification of its memory consumption, in the
form of statements expressed in an appropriate program logic, and a
certificate that establishes that the program verifies the memory
consumption specification attached to it. The approach potentially
allows for precise specifications. However, existing works on proof
carrying code for resource usage sacrifice the possibility of
enforcing accurate policies in favor of the possibility of generating
automatically the specification and the certificate, in line with
earlier work on certifying compilation;

\item \emph{Run-time monitoring:} here the program also comes equipped
with a specification of its memory consumption, but the verification
is performed at run-time, and interrupted if the memory consumption
policy is violated.
 


Such an approach is both precise and automatic, but incurs a runtime
overhead which makes it unsuitable for TPDs.
\end{itemize}
In this chapter, we study the use of logical methods to specify and
verify statically precise memory consumption policies for Java
bytecode programs. We demonstrate how the Bytecode Modeling
Language BML  can be used to specify precise memory consumption policies
for (sequential) Java, and how verification tools can
be used to enforce such memory consumption policies. 


Our broad conclusion is that logical methods provide a suitable means
to specify and verify expressive memory consumption policies, with a
minimal overhead.

The remainder of the chapter is organized as follows. 
Section \ref{sec:motivation} gives several motivating examples.
In section \ref{sec:principles},  we begin by describing the principles of our
approach. Section \ref{sec:examples} illustrates the approach with several examples on recursive methods, exception, inheritance.
%Section \ref{sec:infer} presents an algorithm for inferring method specification for constraint memory consumption policies 
%in terms of preconditions, postconditions and loop invariants. 
Finally, section \ref{sec:rel} presents an overview of related works.
