\section{Ahead-of-time \& just-in-time compilation}%{Java and Ahead-of-Time Compilation}
\label{sec:sota}

Compiling Java into native code common on embedded devices. This section gives an overview of the different compilation techniques of Java programs,
 and points out the issue of runtime exceptions. %We are then looking at how existing solutions address this issue.

%\subsection{Ahead-of-Time \& Just-in-Time Compilation}

% JIT & AOT compilations
Ahead-of-Time (AOT) compilation is a common way to improve the efficiency of Java programs. It is related to Just-in-Time (JIT) compilation by the fact that both processes take Java bytecode as input and produce native code that the architecture running the virtual machine can directly execute. AOT and JIT compilation differ by the time at which the compilation occurs. JIT compilation is done, as its name states, just-in-time by the virtual machine, and must therefore be performed within a short period of time which leaves little room for optimizations. The output of JIT compilation is machine-language. On the contrary, AOT compilation compiles the Java bytecode way before the program is run, and links the native code with the virtual machine. In other words, it translates non-native methods into native methods (usually C code) prior to the whole system execution. AOT compilers either compile the Java program entirely, resulting in a 100\% native program without a Java interpreter, or can just compile a few important methods. In the latter case, the native code is usually linked with the virtual machine. AOT compilation have no or few time constraints, and can generate optimized code. Moreover, the generated code can take advantage of the C compiler's own optimizations.

% Why JIT is not applicable to embedded devices
JIT compilation in interesting by several points. For instance, there is no prior choice about which methods must be compiled: the virtual machine compiles a method when it appears that doing so is beneficial, e.g. because the method is called often. However, JIT compilation requires embedding a compiler within the virtual machine, which needs resources to work and writable memory to store the compiled methods. Moreover, the compiled methods are present twice in memory: once in bytecode form, and another time in compiled form. While this scheme is efficient for decently-powerful embedded devices such as PDAs, it is inapplicable to very restrained devices like smartcards or sensors. For them, ahead-of-time compilation is usually preferred because it does not require a particular support from the embedded virtual machine outside of the ability to run native methods, and avoids method duplication. AOT compilation has some constraints, too: the compiled methods must be known in advance, and dynamically-loading new native methods is forbidden, or at least very unsafe.

% Runtime exceptions
Both JIT and AOT compilers must produce code that exactly mimics the behavior of the Java virtual machine. In particular, the safety checks performed on some bytecode must also be performed in the generated code.
