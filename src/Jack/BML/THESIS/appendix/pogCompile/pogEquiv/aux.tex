\chapter{Proofs from Section \ref{pogEq:aux}}




\begin{exprSrcBcWp1}[Wp of a compiled expression ] \label{exprValueOnStack}
 For any expression $\expressionSrc$ from our source language, for any formula $\psi: \formulaSrc$  
of the source assertion language and any formula $\phi: \formulaBc$ such that $\phi$ may only 
contain stack expressions of the form     $\stack{\counter - k}, k \ge 0$, the following holds



  There  exist $ Q, R : \formulaSrc$  such that 
$$ \begin{array}{l}
      \wpSrcExpr{\expressionSrc }{ \psi }{ \excPost}{v} \  = \\ 
          Q \Rightarrow \psi \\
          \wedge \\ 
	  R  \\
  \\
\Longrightarrow \\
 \\

  
	    \wpBcSeq{ \compileLabel{s}{\expressionSrc}{e} }{ \phi }{\excPost} \  = \\ 
              Q \Rightarrow \phi \begin{array}{l}
                                       \subst{\counter}{\counter + 1}\\
			               \subst{\stack{\counter +1}}{ v }
                         \end{array} \\ \\
          \wedge \\ 
	  R 
  \end{array}$$

\end{exprSrcBcWp1}


We  show several cases of the proof, which is done by induction over the structure of the formula. The rest of the cases
proceed in  a similar way.

Proof : \\
 


\begin{enumerate} 
		\item    $\expressionSrc = const , const \in \ \constantInt, \Mytrue, \Myfalse $
                     
		     $$ \begin{array}{l}  
		               \mbox{\rm\comment{source case }} \\
		              \numConclusion{1} \wpSrcExpr{ const}{\psi}{ \excPost}{const}  \\
		    	       \mbox{\rm\comment{following the definition of the wp function for source expressions in subsection  \ref{pog:wpSrc:wpExpr} }} \\
			       = \psi \\
			       \\
			       \mbox{\rm\comment{bytecode case }} \\
		               \numConclusion{2} \wpBcSeq{\compileLabel{s}{const}{s} }{\phi}{ \excPost}  \\
			
		    	       \mbox{\rm\comment{following the definition   of the compiler function in subsection \ref{pogEq:compile:compExpr} }} \\
			       = \wpExpl{s \ \push const }{\phi}{ \compile{\excPost}}  \\
			       \mbox{\rm\comment{following the definition of the wp function for bytecode in subsection \ref{pog:wpSrc:wpExpr}  }} \\
			       = \phi \begin{array}{l}
                                                \subst{ \counter }{ \counter + 1}\\
						\subst{\stack{ \counter +1} }{const  }
					    \end{array} \\
			        \\
				\\
				 \mbox{\rm\comment{from \numConclusion{1} and   \numConclusion{2} and $Q,R = \true $ this case holds   }} \\
					 
				    
			\end{array}$$
                        
			
			
                \item  $\expressionSrc = \expressionSrc.f $
                         $$ \begin{array}{l}  
		               \mbox{\rm\comment{source case }} \\
		               \numConclusion{1} \ \wpSrcExpr{ \expressionSrc.f  }{\psi}{ \excPost}{v.f}  \\
		    	       \mbox{\rm\textit{\{following the definition of the wp function  }} \\
			        \mbox{\rm\textit{ for source expressions in subsection  \ref{pog:wpSrc:wpExpr}  \} }} \\
			       = \wpSrcExpr{\expressionSrc }{ \begin{array}{l}
                                                                 v \neq \Mynull \Rightarrow \psi\\
								 \wedge \\
								 v = \Mynull \Rightarrow\methodd.\excPost(\NullPointerExc)
								 \end{array} }{ \excPost  }{v}  \\
			       \\
			       \\ 
			        \mbox{\rm\comment{bytecode case }} \\
		               \numConclusion{2} \  \wpBcSeq{ \compileLabel{s}{\expressionSrc.f}{e}}{\phi}{\compile{ \excPost}}   \\
			       \mbox{\rm\comment{following the definition of the compiler function in subsection \ref{pogEq:compile:compExpr} }} \\
			       = \wpBcSeq{ \begin{array}{l}
			                          \compileLabel{s}{\expressionSrc}{e-1} \\
						   e \ \getfield  \ f
						 \end{array} }{\phi}{ \excPost}  \\
			       \mbox{\rm\textit{\{following the definition of the wp function for bytecode    }} \\
			        \mbox{\rm\textit{ in subsection \ref{pog:wpBcGeneral}  \} }} \\
			        = \wpBcSeq{ \compileLabel{s}{\expressionSrc}{e-1}}{\\
                                                                          \phantom{wp^{bc}} \begin{array}{l} 
				                                                 \stack{\counter} \neq \Mynull \Rightarrow  \\
										    \phi \subst{\stack{ \counter} }{ \fieldAccess {\stack{\counter}}{\fieldd} } \\
										    \wedge \\
										    \stack{\counter} = \Mynull   \Rightarrow  \methodd.\excPost (\NullPointerExc)  
										 
				                                            \end{array} }{\\ \phantom{wp^{bc}} \compile{ \excPost}} \\
				\end{array}$$

			 $$ \begin{array}{l} 

	\mbox{\rm\comment{From \numConclusion{1} and \numConclusion{2} we apply the induction hypothesis  }} \\		
				\exists Q', R' : \formulaSrc , \\ 
			       \numConclusion{3} \
						\wpSrcExpr{\expressionSrc }{ \begin{array}{l}
                                                                 v \neq \Mynull \Rightarrow \psi\\
								 \wedge \\
								  v = \Mynull \Rightarrow\methodd.\excPost(\NullPointerExc)
								 \end{array} }{ \excPost  }{v}  \\
				=  \\
				Q'  \Rightarrow \begin{array}{l}
				                     v \neq \Mynull \Rightarrow \psi\\
						     \wedge \\
						     v \neq \Mynull \Rightarrow\methodd.\excPost(\NullPointerExc)
				               \end{array} \\
					      \wedge \\
					       R' \\\\
			  \Longrightarrow \\	   
			        \wpBcSeq{ \compile{\expressionSrc}}{\\
                                                                          \phantom{wp^{bc}} \begin{array}{l} 
				                                                 \stack{\counter} \neq \Mynull \Rightarrow  \\
										    \phi \subst{\stack{ \counter} }{ \fieldAccess {\stack{\counter}}{\fieldd} }\\
										    \wedge \\
										    \stack{\counter} = \Mynull   \Rightarrow  \methodd.\excPost (\NullPointerExc)  
										 
				                                            \end{array} }{\\ \phantom{wp^{bc}} \compile{ \excPost}} \\
				= \\

\end{array}$$
$$\begin{array}{l}

				Q'  \Rightarrow \begin{array}{ll}
				                                \begin{array}{l} \stack{\counter} \neq \Mynull \Rightarrow \phi\subst{\stack{ \counter}}{\fieldAccess {\stack{\counter}}{\fieldd}}\\
								\wedge \\
								\stack{\counter} = \Mynull \Rightarrow\methodd.\excPost(\NullPointerExc) 
							  \end{array} &
							  \begin{array}{l}   
						                \subst{\counter}{\counter + 1} \\
								 \subst{\stack{\counter + 1}}{v }
						          \end{array}
				               \end{array} \\	
			        \wedge \\
			       R'
			       \\
			       	=  \\
			
				 \mbox{\rm\comment{$\phi$ contains only stack expressions $\stack{\counter - k } , k \ge 0$ and properties of substitution }} \\
 				 Q'  \Rightarrow \begin{array}{l}
				                                 v \neq \Mynull \Rightarrow \phi
                                                                 \begin{array}{l}
								   \subst{\counter}{\counter + 1} \\
								   \subst{\stack{\counter + 1}}{\fieldAccess{v}{\fieldd} }
								   \end{array}  \\
								\wedge \\
								v = \Mynull \Rightarrow\methodd.\excPost(\NullPointerExc) 
							  
							  
				               \end{array} \\	
			        \wedge \\
			       R'		\\
			       \\
			       \mbox{\rm\comment{from \numConclusion{3}  this case holds  }}\\
			       		 
		      	\end{array}$$

\end{enumerate}


\begin{exprSrcBcWp2}[Wp of a compiled expression ] \label{pogEq:aux:exprSrcBcWp}
\todo{actually, here need an assumption that the modifies clauses cannot contain stack expressions }
 For any expression $\expressionSrc$ from our source language, for any formula $\psi: \formulaSrc$  
of the source assertion language and any formula $\phi: \formulaBc$ such that $\phi$ may only 
contain stack expressions of the form     $\stack{\counter - k}, k \ge 0$ and if the modifies clauses of every
 method does not contain stack expressions, then the following holds : \\
there exist $ Q,R:\formulaSrc, v_1 \ldots v_k :\expressionSrc $
$$ \begin{array}{l} \wpSrcExpr{\expressionSrc }{ \psi }{ \excPost}{v} = \\
 \forall    \freshVar, \freshVar_1, \ldots  \freshVar_k   ,  Q \Rightarrow \\
 \Myspace \psi 
                                                                             \begin{array}{l}
									         \subst{v }{ \freshVar }  \\
									        \subst{v_1 }{ \freshVar_1 }  \\
										\ldots \\
										\subst{v_k}{ \freshVar_k } 
									     \end{array}\\
          \wedge \\ 
	  R  

      
  \\
 \Longrightarrow \\
 \\

  
	    \wpBcSeq{ \compileLabel{s}{\expressionSrc}{e} }{ \phi }{\excPost} \  = \\ 
            \forall \freshVar, \freshVar_1, \ldots  \freshVar_k ,   Q \Rightarrow \\
	                      \Myspace \phi \begin{array}{l}
                                       \subst{\counter}{\counter + 1}\\
			               \subst{\stack{\counter +1}}{ \freshVar }\\
				       \subst{v_1 }{ \freshVar_1 }  \\
				       \ldots \\
				       \subst{v_k}{ \freshVar_k } 
                         \end{array} \\ 
          \wedge \\ 
	  R 
  \end{array}$$
\end{exprSrcBcWp2}
This lemma refers to the cases for method invokation and instance creation. In the following, we scatch the case for 
instance creation but we ommit the part of the formulas concerning the exceptional termination of the evaluation of
 the expression. \\
Proof:

$$ \begin{array}{l}
       \comment{following the definition from Fig.\ref{pog:wpSrc:wpExpr:wpSrcInvoke} we get for the source case } \\\\
       \numConclusion{1}  \  \wpSrcExpr{ \newSrc \ \class  ( \expressionSrc  ) } {\psi }{ \excPostSrc  }{v} = \\
 	  
 	             \begin{array}{l}  \forall \freshVar, \\
 				               \neg \ \instances(\freshVar ) \wedge \\
 					       \freshVar  \neq \Mynull \\
					       \typeof{\freshVar} <: \class  \Rightarrow \\ 
		     \wpSrcExpr{\expressionSrc} { \\ \\
 		                \phantom{wpSr}\left\{ \begin{array}{l}
                                   
 				   \Myspace  \Myspace  \Constructor{\class}.\preSrc
 				                \begin{array}{l} 
                                                       \subst{ \this}{ \freshVar } \\
                                                       \subst{ arg}{ v' } \\
						        \subst{ \fieldd} { \update{\fieldd} { \freshVar}{\defaultValue{ \fieldd.  \fieldType } } }_{{ \subtype{\fieldd.\declaredIn}{  \class}} } 
  					     	\end{array} \\
  					\Myspace  \Myspace	       \wedge \\
  					\Myspace  \Myspace  \forall \ m \in \Constructor{\class}. \mod , \\
  					       \Myspace  \Myspace  \Myspace    \Constructor{\class}.\normalPostSrc       
  					             \begin{array}{l}   
  						          \subst{ \this}{ \freshVar  } \\
  						          \subst{ arg}{ v' } 
							   \subst{ \fieldd} { \update{\fieldd} { \freshVar}{\defaultValue{ \fieldd.  \fieldType } } }_{{ \subtype{\fieldd.\declaredIn}{  \class}} } \\
  						       	  %\subst{	\typeof{\freshVar}}{ \class} 
  						     \end{array}  \\
                                                   \Myspace  \Myspace   \Myspace  \Myspace    \Rightarrow   \psi \subst{ v }{ \freshVar  } \\ 
  					         
  		              \end{array}\right. } {  \\  \\ \phantom{wpSrc}   \excPostSrc }{v'}  \end{array} \\\\
\comment{ compiler definition from Fig. \ref{pogEq:compile:compExpr:defExpr} and definition of $\wpNameBcSeq$ } \\\\
\numConclusion{2}  \ \wpBcSeq{ \compileLabel{s}{ \newSrc \ \class  ( \expressionSrc  ) }{ e}}{ \phi  } { \excPost}  = \\\\
\wpBcSeq{ \begin{array}{l}
               s :    \new \ \class; \\ 
	       s + 1: \dup; \\
	       \compileLabel{s+2}{\expressionSrc}{e - 1 };\\
	       e : \invoke \ \Constructor{\class};   
          \end{array} 
	}
	{\\ \phantom{\wpName^{bc}_{seq}} \phi  } 
	{\\ \phantom{\wpName^{bc}_{seq}} \excPost } = \\\\
\comment{definition of  $\wpNameBcSeq$  }\\\\
\wpBcSeq{ \begin{array}{l}
               s :    \new \ \class; \\ 
	       s + 1: \dup; \\
	       \compileLabel{s+2}{\expressionSrc}{e - 1 };
	      % e : \invoke \ \Constructor{\class};   
          \end{array} 
	}
	{\\ \phantom{\wpName^{bc}_{seq}}   \wpExpl{e: \invoke \ \Constructor{\class} }{\phi }{  \excPost  }  } 
	{\\ \phantom{\wpName^{bc}_{seq}} \excPost } =  \\  \\ 

 \comment{definition of  $\wpNameExpl$ for \invoke  } \end{array}$$

$$\begin{array}{l} 
 \wpBcSeq{ \begin{array}{l}
               s :    \new \ \class; \\ 
	       s + 1: \dup; \\
	       
	      % e : \invoke \ \Constructor{\class};   
          \end{array} 
	}
	{\\ \phantom{\wpName^{bc}_{seq}}\wpBcSeq{ \compileLabel{s+2}{\expressionSrc}{e - 1 };} { \\ \phantom{\wpName^{bc}_{seq}} \phantom{\wpName^{bc}_{seq}}
	                                                                  \begin{array}{l} \Constructor{\class} .\pre\begin{array}{l}
								           \subst{ \locVar{0}}{\stack{\counter - 1  } }\\
									   \subst{ \locVar{1}}{\stack{\counter } }
									   \end{array} \\
	 				 			\wedge \\
								\begin{array}{l}
	 				 			       \forall  mod ,  ( mod \in \Constructor{\class}.\modif ) (  \\ 
								       \Myspace \Constructor{\class}.\normalPost 
								       \begin{array}{l}
								           \subst{ \locVar{0}}{\stack{\counter - 1  } }\\
									   \subst{ \locVar{1}}{\stack{\counter } }
									   \end{array}
										    \Rightarrow    \phi  \subst{\counter}{ \counter - 2 } )
									\end{array}					   																		\end{array}	}{ \\ \phantom{\wpName^{bc}_{seq} } \phantom{\wpName^{bc}_{seq}} \excPost   }				
	 				 			 } 
	{\\ \phantom{\wpName^{bc}_{seq}} \excPost }   \\\\

 
\comment{we assume  for $\expressionSrc$, $\psi$ and $\excPostSrc$ that the following holds   } \\ \\

\numConclusion{3} \  for \ some \  Q,R:\formulaSrc  ,\\
\wpSrcExpr{\expressionSrc} { \\ 
 		                \phantom{wpSr} \begin{array}{l}
                                   
 				    \Constructor{\class}.\preSrc
 				                \begin{array}{l} 
                                                       \subst{ \this}{ \freshVar } \\
                                                       \subst{ arg}{ v' } \\
						        \subst{ \fieldd} { \update{\fieldd} { \freshVar }{\defaultValue{ \fieldd.  \fieldType } } }_{{\small \forall \fieldd: \FieldSet. \subtype{\fieldd.\declaredIn}{  \class}} } 
  					     	\end{array} \\
  					
  					  \forall \ m \in \Constructor{\class}. \mod , \\
  					       \Myspace  \Myspace \Constructor{\class}.\normalPostSrc       
  					             \begin{array}{l}   
  						          \subst{ \this}{ \freshVar  } \\
  						          \subst{ arg}{ v' } \\
							   \subst{ \fieldd} { \update{\fieldd} { \freshVar }{\defaultValue{ \fieldd.  \fieldType } } }_{{\small \forall \fieldd: \FieldSet. \subtype{\fieldd.\declaredIn}{  \class}} } 
  						       	  %\subst{	\typeof{\freshVar}}{ \class} 
  						     \end{array}   \Rightarrow \\
						   \Myspace  \Myspace     \psi   \begin{array}{l} 
						                                          \subst{ v }{ \freshVar  }\\
											   \subst{ \fieldd} { \update{\fieldd} { \freshVar }{\defaultValue{ \fieldd.  \fieldType } } }_{{\small \forall \fieldd: \FieldSet. \subtype{\fieldd.\declaredIn}{  \class}} } 
											  \end{array} 
                        \end{array} } {  \\  \\ \phantom{wpSrc}   \excPostSrc }{v'} = \\ \\

  Q \Rightarrow \left\{\begin{array}{l}
                                   
 				    \Constructor{\class}.\preSrc
 				                \begin{array}{l} 
                                                       \subst{ \this}{ \freshVar } \\
                                                       \subst{ arg}{ v' } \\
						        \subst{ \fieldd} { \update{\fieldd} { \freshVar }{\defaultValue{ \fieldd.  \fieldType } } }_{{\small \forall \fieldd: \FieldSet. \subtype{\fieldd.\declaredIn}{  \class}} } 
  					     	\end{array} \\
  					
  					  \forall \ m \in \Constructor{\class}. \mod , \\
  					       \Myspace  \Myspace  \Myspace    \Constructor{\class}.\normalPostSrc       
  					             \begin{array}{l}   
  						          \subst{ \this}{ \freshVar  } \\
  						          \subst{ arg}{ v' } \\
							   \subst{ \fieldd} { \update{\fieldd} { \freshVar }{\defaultValue{ \fieldd.  \fieldType } } }_{{\small \forall \fieldd: \FieldSet. \subtype{\fieldd.\declaredIn}{  \class}} } 
  						       	  %\subst{	\typeof{\freshVar}}{ \class} 
  						     \end{array}   \Rightarrow   \psi \subst{ v }{ \freshVar  } 
                       \end{array}\right. \\
  \wedge  \\ 
  R 

\\\\  
\comment{which means that } \\\\
 \numConclusion{1} = \\
 \forall \freshVar, \\
 				               \neg \ \instances(\freshVar ) \wedge \\
 					       \freshVar  \neq \Mynull \\
					       \typeof{\freshVar} <: \class  \Rightarrow \\ 
  Q \Rightarrow \left\{\begin{array}{l}
                                   
 				    \Constructor{\class}.\preSrc
 				                \begin{array}{l} 
                                                       \subst{ \this}{ \freshVar } \\
                                                       \subst{ arg}{ v' } \\
						        \subst{ \fieldd} { \update{\fieldd} { \freshVar }{\defaultValue{ \fieldd.  \fieldType } } }_{{\small \forall \fieldd: \FieldSet. \subtype{\fieldd.\declaredIn}{  \class}} } 
  					     	\end{array} \\
  					
  					  \forall \ m \in \Constructor{\class}. \mod , \\
  					       \Myspace  \Myspace  \Myspace    \Constructor{\class}.\normalPostSrc       
  					             \begin{array}{l}   
  						          \subst{ \this}{ \freshVar  } \\
  						          \subst{ arg}{ v' } \\
							   \subst{ \fieldd} { \update{\fieldd} { \freshVar }{\defaultValue{ \fieldd.  \fieldType } } }_{{\small \forall \fieldd: \FieldSet. \subtype{\fieldd.\declaredIn}{  \class}} } 
  						       	  %\subst{	\typeof{\freshVar}}{ \class} 
  						     \end{array}   \Rightarrow   \psi \subst{ v }{ \freshVar  } 
                       \end{array}\right. \\
  \wedge  \\ 
  R \end{array}$$

$$\begin{array}{l}
\comment{ from  Lemma \ref{exprValueOnStack} then we get }
\\\\

\numConclusion{4}  \
\wpBcSeq{ \compileLabel{s+2}{\expressionSrc}{e - 1 };}{ \\ \phantom{\wpName^{bc}_{seq}} \phantom{\wpName^{bc}_{seq}}
	                                                                  \begin{array}{l} \Constructor{\class} .\pre\begin{array}{l}
								           \subst{ \locVar{0}}{\stack{\counter - 1  } }\\
									   \subst{ \locVar{1}}{\stack{\counter } }
									   \end{array} \\
	 				 			\wedge \\
								\begin{array}{l}
	 				 			       \forall  mod ,  ( mod \in \Constructor{\class}.\modif ) (  \\ 
								       \Myspace \Constructor{\class}.\normalPost 
								       \begin{array}{l}
								           \subst{ \locVar{0}}{\stack{\counter - 1  } }\\
									   \subst{ \locVar{1}}{\stack{\counter } }
									   \end{array}
										    \Rightarrow    \phi  \subst{\counter}{ \counter - 2 } )
									\end{array}					   																		\end{array} } { \\ \phantom{\wpName^{bc}_{seq} }
                                                                  \phantom{\wpName^{bc}_{seq}} \excPost   } = \\\\

  Q \Rightarrow \left\{\begin{array}{l} \Constructor{\class} .\pre \begin{array}{l}
								           \subst{ \locVar{0}}{\stack{\counter - 1  } }\\
									   \subst{ \locVar{1}}{\stack{\counter } }
								 \end{array} \\
	 				 			\wedge \\
								\begin{array}{l}
	 				 			       \forall  mod ,  ( mod \in \Constructor{\class}.\modif )   \\ 
								       \Myspace \Constructor{\class}.\normalPost 
								       \begin{array}{l}
								           \subst{ \locVar{0}}{\stack{\counter - 1  } }\\
									   \subst{ \locVar{1}}{\stack{\counter } }
									   \end{array}\\
									\Myspace \Rightarrow    \phi  \subst{\counter}{ \counter - 2 } 
								\end{array}		
			   	 \end{array}\right\}  
									\begin{array}{l} 
									\subst{\counter}{\counter + 1} \\
									\subst{\stack{\counter + 1}}{v'}
									\end{array}\\
                     
\wedge \\
R \\
=\\\\
\mbox{\rm\textit{ \{apply substitution and because \Constructor{\class}.\modif \ does not contain stack expressions and  }}\\
\mbox{\rm\textit{  because of the initial hypothesis about $ \phi$ obtain \}}} \\\\







 Q \Rightarrow \left\{\begin{array}{l} \Constructor{\class} .\pre\begin{array}{l}
								           \subst{ \locVar{0}}{\stack{\counter  } }\\
									   \subst{ \locVar{1}}{v' }
									   \end{array} \\
	 				 			\wedge \\
								\begin{array}{l}
	 				 			       \forall  mod ,  ( mod \in \Constructor{\class}.\modif ) (  \\ 
								       \Myspace \Constructor{\class}.\normalPost 
								       \begin{array}{l}
								           \subst{ \locVar{0}}{ \stack{\counter  } }\\
									   \subst{ \locVar{1}}{v' }
									   \end{array} \\
									\Myspace 	    \Rightarrow    \phi  \subst{\counter}{ \counter - 1 } )
									\end{array}					   																	\end{array} \right. \\
\wedge \\
R \\
\end{array}$$

$$\begin{array}{l}

\comment{from \numConclusion{2} and \numConclusion{4}  }
\\\\

\numConclusion{5} \ \wpBcSeq{ \compileLabel{s}{ \newSrc \ \class  ( \expressionSrc  ) }{ e}}{ \phi  } { \excPost} = \\
 \wpBcSeq{ \begin{array}{l}
               s :    \new \ \class; \\ 
	       s + 1: \dup; \\
	       
	 
          \end{array} 
	}{\\ \phantom{\wpName^{bc}_{seq}}
           Q \Rightarrow \left\{\begin{array}{l} \Constructor{\class} .\pre\begin{array}{l}
								           \subst{ \locVar{0}}{\stack{\counter  } }\\
									   \subst{ \locVar{1}}{v' }
									   \end{array} \\
	 				 			\wedge \\
								\begin{array}{l}
	 				 			       \forall  mod ,  ( mod \in \Constructor{\class}.\modif ) (  \\ 
								       \Myspace \Constructor{\class}.\normalPost 
								       \begin{array}{l}
								           \subst{ \locVar{0}}{ \stack{\counter  } }\\
									   \subst{ \locVar{1}}{v' }
									   \end{array} \\
									\Myspace 	    \Rightarrow    \phi  \subst{\counter}{ \counter - 1 } )
									\end{array}					   																	\end{array} \right.
         

  }{ \\ \phantom{\wpName^{bc}_{seq}}  \excPost } = \\\\

\comment{apply rule $\wpNameExpl$ \ for  \dup }\\\\

\wpBcSeq{ \compileLabel{s}{ \newSrc \ \class  ( \expressionSrc  ) }{ e}}{ \phi  } { \excPost} = \\
 \wpBcSeq{   s :    \new \ \class;\\ 
	    
	    
	}{\\ \phantom{\wpName^{bc}_{seq}}
           Q \Rightarrow \left\{\begin{array}{l} \Constructor{\class} .\pre\begin{array}{l}
								           \subst{ \locVar{0}}{\stack{\counter  } }\\
									   \subst{ \locVar{1}}{v' }
									   \end{array} \\
	 				 			\wedge \\
								\begin{array}{l}
	 				 			       \forall  mod ,  ( mod \in \Constructor{\class}.\modif )   \\ 
								       \Myspace \Constructor{\class}.\normalPost 
								       \begin{array}{l}
								           \subst{ \locVar{0}}{ \stack{\counter  } }\\
									   \subst{ \locVar{1}}{v' }
									   \end{array} \\
									\Myspace 	    \Rightarrow    \phi  %\subst{\counter}{ \counter - 1 } 
									\end{array}					   																	\end{array} \right.
         

  }{ \\ \phantom{\wpName^{bc}_{seq}}  \excPost } = \\\\
\comment{apply rule $\wpNameExpl$ \ for  \new }\\\\
  \forall \freshVar,\\
                      \Myspace    \neg \ \instances(\freshVar) \wedge \\
		      \Myspace \freshVar \neq \Mynull  \\
		        \Myspace \typeof{\freshVar} <: \class  \Rightarrow	\\
		      Q \Rightarrow \left\{\begin{array}{l} \Constructor{\class} .\pre\begin{array}{l}
								           \subst{ \locVar{0}}{\freshVar } \\
									   \subst{ \locVar{1}}{v' }\\
									    \subst{ \fieldd} { \update{\fieldd} { \freshVar }{\defaultValue{ \fieldd.  \fieldType } } }_{{\small \subtype{\fieldd.\declaredIn}{  \class}} } \\
									   \end{array} \\
	\wedge \\
								\begin{array}{l}
	 				 			       \forall  mod ,  ( mod \in \Constructor{\class}.\modif )   \\ 
								       \Myspace \Constructor{\class}.\normalPost 
								       \begin{array}{l}
								           \subst{ \locVar{0}}{ \freshVar   }\\
									   \subst{ \locVar{1}}{v' }\\
									   \subst{ \fieldd} { \update{\fieldd} { \freshVar }{\defaultValue{ \fieldd.  \fieldType } } }_{{
\small \subtype{\fieldd.\declaredIn}{  \class}} } 
									   \end{array} \\
									\Myspace 	    \Rightarrow    \phi  \\
									\begin{array}{l}
									\subst{\counter}{ \counter + 1 }\\
									 \subst{ \stack{\counter + 1}}{ \freshVar} \\
									 \subst{ \fieldd} { \update{\fieldd} { \freshVar }{\defaultValue{ \fieldd.  \fieldType } } }_{{
\small \subtype{\fieldd.\declaredIn}{  \class}} }
									
									\end{array} 
									\end{array}
				\end{array}\right. \\\\		               
	\comment{from the equalities \numConclusion{5} , \numConclusion{1},  \numConclusion{2} and assumption  \numConclusion{3} this case holds }					  
\end{array}$$
\Qed \\
