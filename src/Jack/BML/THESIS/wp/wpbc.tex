 


\section{Weakest precondition calculus} \label{wpRules}


%\subsection{Rules for single instruction}
The weakest precondition predicate transformer function which for any instruction of the Java sequential fragment
determines the predicate that must hold in the prestate of the instruction has the following signature:

$$ \fwpi :   (int,\bcIns)  \longrightarrow   \MethodSet   \longrightarrow \predWp $$
The function \fwpi \ takes two arguments : 
the second argument is the method \methodd \ to which the  instruction belongs
and  the first argument is  the instruction (for instance  \putfield) along with its position in \methodd.
%and finally the position of the instruction in the body of  \methodd.

Let us first see what is the desired meaning of \fwpi.
Particularly, we would like that the function \fwpi{}  returns a predicate $\wpi{ins}{\methodd}{pos}$
such that  if it holds in the prestate of the method \methodd \  and if the
\methodd{} terminates normally then the normal postcondition \methodd.\normalPost \ holds when 
\methodd{} terminates execution, otherwise if \methodd \ terminates on an exception
\mbox{ \rm \texttt{Exc}} the exceptional postcondition  $\methodd.\excPostSpec(\mbox{ \rm \texttt{Exc}})$ holds where the function \excPostSpec{} was
introduced in Section \ref{methExtend}. Thus, the \fwpi \ function takes into account both normal and exceptional
program termination. The truthfulness of the predicate returned by the \fwpi{} function
may only guarantee that the postcondition holds under the assumption that the program terminates.
 
 In the following, we will give an intuition to the way in which we have defined our verification condition generator.
 Consider the example in Fig. \ref{wp:example:sum} which 
 shows both the source code and the bytecode of a method which calculates the sum of all the natural numbers
 smaller or equal to the parameter \lstinline!k!. The source and bytecode are annotated, the first one in JML and the latter in BML.
 However, the bytecode annotations are actually stored separately from the bytecode instructions as we have described in Section \ref{methExtend}
 but we have put them explicitely in the bytecode at the point where they must hold for the sake of clarity. 
 Note that in what follows we will name the loop invariant $I$.
 We have also marked the instructions which are identified as loop start and end
 according to Def.\ref{defEdge} in Chapter \ref{opSem}, Section \ref{prelim:ctrFlow}. 

  
\begin{figure}[ht!]
\begin{center}
\begin{minipage}[c]{\linewidth} 
\begin{minipage}[c]{.63\linewidth}
\scriptsize{
\begin{lstlisting}[frame=trbl]
//@requires reg(1)>=0
0 const 0
1 store 2
2 const 0
3 store 3
4 goto 10
5 load 2
6 load 3
7 add
8 store 2
9 iinc 3 //LOOP END
//@loop_modifies reg(2),reg(3)
//@loop_invariant I := 
//@               reg(3)>=0 &&
//@               reg(3)<=reg(1)&&
//@               reg(2)==reg(3)*(reg(3)-1)/2
10 load 3 //LOOP ENTRY 
11 load 1
12 if_icmplt 5 
13 return
//@ensures \result==
           reg(1)*(reg(1)+1)/2
\end{lstlisting}} 
\end{minipage}


%\hfill
\begin{minipage}[c]{.63\linewidth} 
\scriptsize{
\begin{lstlisting}[frame=trbl]
//@requires k >= 0 ;
//@ensures \result == k*(k+1)/2;
public void m(int k){
  int sum = 0;
  //@loop_modifies sum, i;
  //@loop_invariant i >= 0 
  && i <=k && sum == i*(i-1)/2;
  for (int i = 0;i < k;i++){
    sum = sum + i;
  } 
}
\end{lstlisting} }
\end{minipage}
\end{minipage}
\end{center}
\caption{\sc  bytecode of method \lstinline!sum! and its specification }
\label{wp:example:sum}
\end{figure}


 It is worth first to note that because the bytecode is not structured we cannot define
 the weakest precondition in the same way in which a predicate transformer for structured
 languages is defined.  
 We will rather  define the predicate transformer for  instructions that may have one possible successor to depend on
 this successor:
$$\fwpi( j ) =  S_k(\fwpi( k )) , where  \ j \execRel k $$

where $S_k$ stands for a function which might be the identity function or a function which applies some substitution over its argument.
 We would do similarly for instructions that may branch - instructions which may jump (\goto{} and \ifCond{}) as well as instructions
 which may throw an exception (e.g. \putfield, \arrstore), but this time the predicate transformer for them depends on all of its successors:
$$\fwpi( j ) = \bigwedge_k  C_k \Rightarrow S_k(\fwpi( k )) , where  \ j \execRel k $$
The predicates $C_k$ stand for some condition to be filled in order that after the execution of instruction at index $j$ the instruction at index $k$ is executed.
Note that here, in this example, we are using a loose notation for \wpName{}  function, as we only depends here on one parameter, namely the index
of the instruction for which we calculate the weakest precondition.

Returning back to our example in Fig. \ref{wp:example:sum}, the weakest precondition for the instruction at index 12 (in the bytecode version) 
which is a conditional jump  of the program will be:
$$ \begin{array}{ll} \fwpi( 12 )  = & 
                      \stack{\counter - 1} < \stack{\counter} \Rightarrow   \fwpi(13) \subst{\counter}{\counter - 2} \\
                      &\wedge \\
		      &\stack{\counter - 1} >= \stack{\counter} \Rightarrow   \fwpi(5)\subst{\counter}{\counter - 2}, \\
		      & where \ 12 \execRel 13, 12 \execRel 5
   \end{array}
		      $$

Let us see now what  we would expect about the result of   the function \fwpi{} when applied to the instructions
that have as successor the loop entry instruction at index 10. For instance, we can look at the instruction
at index 9 which is marked in the figure as the end of the loop. 
As we said earlier we have inlined annotations in the bytecode at the places where they must hold. Thus after the execution
of the instruction at index 9  the loop invariant must hold. It follows then that for a loop end instruction 
we will rather require that the \fwpi{} function takes into account the corresponding loop invariant:
$$ \begin{array}{ll} \fwpi( 9 )  = &
                    I \subst{\locVar{3}}{\locVar{3} + 1}, \\
		     & where \ 9 \execRel^l 10 
		     \end{array}$$


The situation is  similar  for the instruction at index 4 which jumps to the loop entry instruction at index 10. 
The semantics of the invariant requires that  in the state after the execution of instruction at index 4 and before the execution 
of the instruction at index 10 the loop invariant must hold
and second,  whatever are the values
of the program variables that might be modified by the loop, the invariant should imply the precondition of the loop entry instruction at 10.
 Thus we would like that  the function \fwpi{} gives us something like: 
   $$\begin{array}{ll} \fwpi( 4 ) =& 
                    I \\
		    &\wedge \\
		    &\forall \locVar{2}, \locVar{3},  I  \Rightarrow \fwpi(10), \\
			&    where \ 4 \execRel 10 \ \wedge 10 \ is \ a \ loop \ entry   
      \end{array}
                 $$
% May be it is here the place to discuss   why we   quantify over the values of the
% modified variables in the loop. The  reason for this is that we can thus properly
%initialize the non modified variables in verification conditions. This makes the set of 
%%the provable verification conditions larger than in the case we would have not done this.  
%In the following, we shall assume that the modifies clauses in specifications
%are correct although we verify it in our verification condition generator.

The example shows that the function \fwpi{} depends  on the semantics of the instruction 
for which it calculates a precondition and also on the execution relation it has with its successors.
In order to define the function \fwpi{} we will use an intermediate function which shall decide 
what is the postcondition of an instruction  upon the execution relation with its successors. This function 
is introduced in the next subsection \ref{wp:interPred}.
 We will also see how the weakest precondition is defined in the presence of exceptions in subsection \ref{wp:Exc}.



