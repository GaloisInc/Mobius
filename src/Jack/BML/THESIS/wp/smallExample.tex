\section{Example}\label{wp:example}
In the following, we will consider an example of the application of the verification procedure 
with \fwpi.
In Fig. \ref{wp:example:square}, we give an example of a Java method which calculates the square of its input which is stated in its postcondition.
The calculation of the square of the parameter {\verb i} is done with an iteration which sums all the impair numbers  {\verb 2*s + 1, 0<=s <=i } in the local variable
{\verb sqr}. The invariant states that whenever the loop entry is reached the variable {\verb sqr} will contain the square of the local variable  {\verb s} and that 
  {\verb 0 <=s <=i}. 



\begin{figure}
\begin{verbatim}
//@ ensures \result == i*i; 
public int square( int i ) { 
  int sqr  = 0;
  if ( i < 0) {
    i = -i;
  }
  //@ loop_modifies s, sqr;
  //@ loop_invariant (0 <= s) && (s <= i) && sqr == s*s ;
  for (int s = 0 ; s < i; s++ ) {
    sqr = sqr + 2*s + 1;
  }
  return sqr;
}

\end{verbatim}
\caption{\sc Java method which calculates the square of its input  }
\label{wp:example:square}
\end{figure}
