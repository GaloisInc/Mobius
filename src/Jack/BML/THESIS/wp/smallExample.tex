\section{Example}\label{wp:example}
In the following, we will consider an example of the application of the verification procedure 
with \fwpi.
Consider Fig. \ref{wp:example:sqrSrc}, which gives an example of a Java method which calculates the square of its input which is stated in its postcondition.
The calculation of the square of the parameter  \lstinline!i!  is done with an iteration which sums all the impair numbers  \lstinline! 2*s + 1, 0<=s <=i! 
 in the local variable \lstinline!sqr!. The invariant states that whenever the loop entry is reached the variable 
  \lstinline!sqr! will contain the square of the local variable  \lstinline!s! and that \lstinline!0 <=s <=i!. 
In Fig.\ref{wp:example:sqrBc}, we show the bytecode of method \lstinline!square!. \todo{create the figure and then explain}


We next show the weakest preconditions of the basic block containing the 
\return \ instruction as well the block containing the loop end instruction. \todo{make the example}



\begin{figure}
\begin{lstlisting}[frame=trbl]
//@ ensures \result == i*i; 
public int square( int i ) { 
  int sqr  = 0;
  if ( i < 0) {
    i = -i;
  }
  //@ loop_modifies s, sqr;
  //@ loop_invariant (0 <= s) && (s <= i) && sqr == s*s ;
  for (int s = 0 ; s < i; s++ ) {
    sqr = sqr + 2*s + 1;
  }
  return sqr;
}

\end{lstlisting} 
\caption{\sc Java method which calculates the square of its input  }
\label{wp:example:sqrSrc}
\end{figure}

\begin{figure}
\begin{lstlisting}[frame=trbl]

\end{lstlisting} 
\caption{\sc  bytecode of method \lstinline!square!}
\label{wp:example:sqrBc}
\end{figure}
