
This section describes a Hoare style verification condition generator for bytecode based on a weakest precondition predicate transformer function.
 

%technical
The vcGen is tailored to the bytecode language introduced in Section \ref{opSem} and thus, it deals
with stack manipulation, object creation and manipulation, field access and update, as well as exception throwing and handling.

% discussion about the choice for the vcgen algorithm
Different ways of generating verification conditions exist. The verification condition
 generator presented  propagates the weakest precondition and exploits the information 
about the modified locations by methods and loops. 
In Section \ref{wp:discussionVC}, we discuss the existing approaches and motivate the choices done here.

%related work
Bytecode verification has become lately quite fashionable, thus several works exist on bytecode verification. Section \ref{relWorkWp}
is an overview of the existing work in the domain.


% definition of the wp
As we stated earlier, our verification condition generator is based on a weakest precondition (wp) calculus.
 However, a logic tailored to stack based bytecode should take into account 
particular bytecode features as for example the operand stack. Another particularity of our verification condition calculus (weakest precondition function)
is the propagation of verification conditions up to the program point similar to the definition of the weakest precondition calculus
for the Java-like  source language in Chapter \ref{javaVerif}. To do this, we define the  weakest precondition predicate transformer
in terms of two mutually recursive function. 
The first one calculates the precondition of instructions over intermediate predicates which should hold 
in between the current instruction and 
its successors. A predicate which must hold between an instruction and its successor 
depends on the precondition of the successor and the execution relation between them, if it is a loop edge or not.
The definition of the weakest predicate transformer  is presented in Section  \ref{wpRules}. 

%example
Finally, section \ref{wp:example} gives an example for how the verification condition generator works.

% proof of correctness
%An important question is the correctness of our methodology, i.e. what guarantees that if the verification procedure succeeds on a program then 
%the program respects  for real the property that we verified for. A skatch of the proof of correctness is presented in Section \ref{proof}
% w.r.t. the operational semantics of our bytecode language introduced in Section \ref{opSem}. 
 





 
