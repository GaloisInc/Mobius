\section{Related work} \label{relWorkWp}
 Floyd is among the first to work on program verification using logic methods for  program
 languages (see \cite{F67amp}). Following the Floyd's approach, T.Hoare gives a formal logic for program verification in \cite{Hoare69ABC} known
 today under the name Hoare logic. Dijkstra and Scholten \cite{WPCDS} proposes then an efficient way for applying Hoare logic in
 program verification, in particular they propose two predicate transformer calculus and give their formal semantics. 
 
% With the increasing popularity of Java in security domains like networking, mobile phones, smart cards the design of
% verification conditions tailored to this language has also increased
 
% Concerning bytecode validation, there exists several approaches depending on the kind of properties that one want to check for.
 
% Bytecode verification is concerned with establishing that a bytecode is well typed 
%(every instruction is applied to operands of the correct type) and well formed 
%(e.g. no jumps to an un-existing bytecode index), differently from the goals of the present
%work where program correctness is defined in terms of functional correctness. The JVM, for example, 
%is provided with a bytecode verifier. There is a lot of research work done in the domain 
%and for a detailed overview of the state of the art one can look at~\cite{Ljbc}.  
%%%%%%%%%%%%%%%%%%%%%%%%%%%%%%%%%%%%%%%%%%%%%%%%%%%%%%%%%%%%
In the following, we review briefly the existing work related to bytecode verification
 and more particularly program verification tailored to Java bytecode programs. 
Few works have been dedicated to the definition of a bytecode logic. Among the earliest work in the field of bytecode verification 
is the thesis of C.Quigley  \cite{Quigley03PLJ} in which Hoare logic rules are given for a bytecode like language. This work is limited 
to a subset of the Java virtual machine instructions and does not treat for example method calls,
 neither exceptional termination. The logic is defined by searching a structure in the bytecode control flow graph,
 which gives an issue to complex and weak rules.

The work by Nick Benton \cite{B04tlsj} gives a  typed logic for a bytecode language with stacks and jumps. 
The technique that he proposes checks that both types and specifications are respected.
The language is simple and supports basically stack and arithmetic operations. A proof of correctness
w.r.t. an operational semantics is given. Differently from this work, here we assume that program are well typed. 
This is a safe assumption as the JVM is supplied with a bytecode verifier \cite{Ljbc}. 
We consider that the separation of the concerns for well typedness and functional correctness between the bytecode verifier
and a verification condition generator is a good design decision.

Following the work of Nick Benton, Bannwart and Muller \cite{BannwartMueller05} give  a Hoare logic rules
for a bytecode language with objects and  exceptions. A compiler from source proofs into bytecode proofs is also defined. 
As in our work, they assume that the bytecode is well-typed. 
In particular, they define a Hoare-style bytecode logic which consists in building a derivation tree 
and the leaves of the derivation tree must be proven in a classical logic. This is different from our solution
where we generate directly  verification conditions using a weakest precondition calculus which are then proven in a logic.
 A main inconvenient of using  Hoare logic triples for proving program correctness is that this is  
  a complex process and needs even in simple cases a high level of user interaction and competence. Of course, our approach
 also requires user interaction as far as the generated verification conditions are hard to prove but automation 
 is possible as far as the verification conditions are simple.
  
 In ~\cite{WildmoserN-ESOP05}, M. Wildmoser and T. Nipkow describe a framework for verifying Jinja (a Java   bytecode subset) which features
 object manipulation, exceptions, method invocations. The verification framework is   based on a verification condition generator which uses
 weakest preconditions. The  framework is developped  in the interactive theorem prover Isabelle/HOL and proven sound and complete. 
 They show how the safety policy against an arithmetic overflow can be checked. As in our case, they also assume that the program is provided
with annotations (e.g. \ loop invariants). 

% In this work, we have implemented a 

 The Spec\# \cite{BLS04sp} programming system developed at Microsoft proposes a static verification framework where 
 the method and class contracts  (pre, post conditions, exceptional postconditions, class invariants) are inserted in the intermediate code . 
 Spec\# is a superset of the C\# programming language, with a built-in  specification language, 
 which proposes a verification framework (there is a choice to perform the checks either at runtime or statically). 
 The verification procedure \cite{leinoWPUP} includes several processing stages of the bytecode program -  
 elimination of irreducible loops, transformation into an acyclic control flow graph,
 translation of the bytecode into a guarded passive command language program. 
 These transformations of course, facilitate the verification procedure.
 Transforming the bytecode into an acyclic control flow graph (or simply identifying the loop entries)
 allows for an easy treatment of loop invariants. As we said earlier, using an intermediate language allows to treat a smaller
 language and thus the changes in the verification condition generator can be easily applied. 
 Moreover, supporting an intermediate language allows for using the same verification framework for different programming languages. 
 Passification avoids duplication of formulas which can be exponential in the number of the conditional branches in a program 
\cite{RL05EWP}. This method consists basically in converting a program into a single assignment form.
 Despite that  in our implementation we also
  transform  the control graph into an acyclic program, we consider that in a mobile code scenario
 one should limit the number of program transformations for the following reasons.
 A design of a verification condition generator should be as simple as possible especially.
 A verification framework which relies on several layers of transformations
 can be relatively complex.  Second,  a verification condition generator must be proven correct. In the case of several transformations this may be not trivial.


% Another topic related to the present work is PCC.
%  PCC and the certifying compiler were proposed by Necula (see \cite{Necula97,ComNec,DesNecLee98}). PCC is an architecture for establishing trust in untrusted code 
% in which the code producer supplies a proof for correctness with the code. 
% The initial idea for PCC  was that the producer automatically infers annotation for properties like well typedness, 
% correct read/writes and automatically generates the proof for their correctness using the certifying compiler. 
% However, such properties guarantee that a program executes correctly w.r.t. to the semantics of the 
% abstract machine, but cannot guarantee if a program executes correctly w.r.t to a functional specification.
% The verification condition generator presented in the following is tailored to deal with functional properties.


 


 
