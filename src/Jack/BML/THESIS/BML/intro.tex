
%\input BML/cmdBML.tex

\newcommand{\code}{\textit{code}}
\newcommand{\indexComp}{\textit{index}}





\section{Introduction} \label{bcsl}
This section presents a bytecode level specification language, called for short BML and a compiler from a
 subset of the high level Java specification language JML to BML. 

% motivation

 Before going further, we discuss what advocates the need of a low level specification language.
Traditionally, specification languages were tailored for high level languages.  
Source  specification allows to express complex functional or security properties about programs.
Thus, they are / can successfully be used 
for software audit and validation. Still, source specification in the context of mobile code does not help a lot for several reasons.


First, the executable / interpreted code  may not be accompanied by its specified  source. Second, it is more reasonable for the 
code receiver to check the executable code than its source code, especially if he is not willing to trust the compiler. 
Third, if the client has complex requirements and even if the code respects them, in order to establish them, 
the code should be specified. Of course, for properties like well typedness this specification can be inferred automatically,
but in the general case this problem is not decidable. 
Thus, for more sophisticated policies, an automatic inference will not work.

 It is in this perspective, that we propose to make the Java
bytecode benefit from the source specification by defining the BML language and a compiler from JML towards BML.    

% what does the language support?
 BML supports the most important features of JML. Thus, we can express functional properties of Java
 bytecode programs in the form of method pre and postconditions, class and object invariants, assertions
 for particular program points like loop invariants. To our knowledge BML does not have predecessors that are tailored 
 to Java bytecode.  

 In section \ref{BCSLprelim}, we give an overview of the main features of JML. A detailed overview of BML is given in section \ref{BCSLgrammar}.  
  As we stated before, we support also a compiler from the high level specification language JML into BML. The 
 compilation process from JML to BML is discussed in section  \ref{BCSLcompile}.
 The full specification of the new user defined Java attributes in which the JML specification is compiled is given in the appendix.




