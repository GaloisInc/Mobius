

% CHANGED - the example o compilation of loop specification

\newtheorem{bml}{Definition}[section]

\section{BML} \label{BCSLgrammar}


BML corresponds to a representative subset of JML and is expressive enough for most purposes including the description of non trivial functional and 
 security properties. The following Def. \ref{BCSL} gives the formal grammar of BML. The formal grammar of BML is given in the next definition.

\subsection{Notation convention}


\begin{itemize}
     \item Nonterminals are written with a \nonterminal font
     \item Terminals are written with a \terminal font
    % \item Keywords are writtem with a \keyWord font
     \item brackets  $\lbrack \ \rbrack $ surround optional text.
\end{itemize}






\subsection{BML Grammar}\label{BCSL}

%\begin{bml}[Formal grammar of BML]\label{BCSL} 
$$\begin{array}{ll}
     \Constants   & ::= \intLiteral  \mid \signedInt  \mid \Mynull \mid \ident  \\
     & \\

     \signedInt & ::= + \nonZeroDigit  \lbrack \digits \rbrack  \mid  - \nonZeroDigit  \lbrack \digits \rbrack \\

     & \\
     \intLiteral & ::= \digit \mid \nonZeroDigit  \lbrack \digits \rbrack \\

     & \\ 
     \digits & ::=  \digit  \lbrack \digits \rbrack \\

     & \\
     
     \digit & ::=  \mbox{\rm\textbf{0}} \mid \nonZeroDigit \\
     
     & \\

     \nonZeroDigit &::= \mbox{\rm\textbf{1}}  \mid \ldots \mid \mbox{\rm\textbf{9}}   \\
     
     & \\
     
     \ident & ::= \# \ \intLiteral \\    
     
     & \\
     
     \boundVar & ::= \ \bound\_\intLiteral \\ 

     & \\
    % \RefValuesSpec    & ::=  Ref \mid \Mynull \\
    & \\
    \expression      & ::= \Constants \\
                     &  \mid  \locVar{ \digits } \\ 
       	             &  \mid  \fieldAccess{\expression}{\ident} \\
		     &  \mid \ident \\
		  %  &  \mid  \update{\fieldd}{ \expression}{\expression}(\expression) \\
		     &  \mid  \arrayAccess{\expression} {\expression} \\	   
		 %   &  \mid \update{ \arrayAccessOnly}{ (\expression , \expression)}{ \expression} (\expression,\expression) \\	
		     &  \mid  \expression \ \op \ \expression   \\
		     &  \mid  \counter \\
		     &  \mid  \stack{ \expression} \\
                     &  \mid \old{ \expression  } \\
                     &  \mid \EXC    \\
		     &  \mid  \result \\
		     &  \mid  \boundVar \\
    & \\
    \typeExp         & :: =  \typeof{ \expression} \\
                     &  \mid \type{\ident} \\
                     &  \mid \elemtype{\expression  }\\
		     &  \mid \TYPE\\   
   & \\ 
                  
 \expressions        & :: = 	\expression \\
 	     	     & \mid \typeExp   
     
 \end{array}$$
%\end{bml}

$$\begin{array}{ll}
 \op & ::=  \plus \mid \minus \mid \mult \mid \divis \mid  \modulo \\
 

    & \\
 \predicates & ::=  \  = \mid  \neq \mid \leq \mid \le \mid \geq \mid > \mid \subtypeSpec    \\
  & \\
 \formulaBc & ::= 
            \expression \ \predicates \  \expression   \\
	  & \mid  \true \\
	  & \mid  \false  \\	
          & \mid not \ \formulaBc  \\
	  & \mid \formulaBc  \wedge  \formulaBc \\
	  & \mid \formulaBc \vee  \formulaBc \\
	  & \mid \formulaBc  \Rightarrow \formulaBc \\
          & \mid \formulaBc \iff  \formulaBc \\
	  & \mid \forall \ \boundVar , \formulaBc\\
	  & \mid \exists \ \boundVar  , \formulaBc	 \\
    & \\
  \ClassSpec & ::=  \ClassInv \ \formulaBc  \\
                 & \mid \ClassHistoryConstr  \ \formulaBc   \\
                 & \mid \declare \ \ghost \ \ident \ \ident \\ 

    & \\
  \intraMethodSpec & ::=  \begin{array}{l}  \atIndex \ nat; \\
		                            \intraSpec ; 
			  \end{array}\\
&\\
\intraSpec & ::=  \loopSpec \\ 
             & \mid \assert \ \formulaBc  \\                      
	     & \mid \set \ \expression \  \expression \\
& \\
\loopSpec  & ::=  \begin{array}{l}  
                        \loopInv \  \formulaBc; \\ 
			\loopMod \ list; \\ 
                        \loopDecreases \ \expression; 
	         \end{array}\\

    & \\ 
  \MethodSpec & ::=   \specCase \\     
                  & \mid  \specCase \ \also \ \MethodSpec \\

 & \\
  \specCase & ::=  \begin{array}{l}  
                                 \requires  \ \formulaBc; \\
                                 \modifies  \ list \   \modifiesLoc;  \\
				 \ensures \ \formulaBc; \\
				 \exsuresList 
		     \end{array} \\
 & \\ 
 \exsuresList & ::=   [ ] \mid  \exsures \ ( \ident )  \ \formulaBc  ; \exsuresList  \\
 & \\
  \modifiesLoc & ::=  \fieldAccess{\expression}{\ident} \\
                    & \mid \locVar{i} \\ 
                    & \mid \arrayAccessMod{\expression}{\specIndex}\\
		    & \mid \everything \\
		    & \mid \nothing \\
              
 & \\
 \specIndex & ::= \all \mid i_1 .. i_2 \mid i  
\end{array}                
$$

$$ 
\begin{array}{ll}                

 
       & \\
       \bmlKeyWords & ::= \requires \\
                       & \mid   \ensures \\
		       & \mid   \modifies \\
		       & \mid   \assert \\
		       & \mid   \set \\
		       & \mid   \exsures \\
		       & \mid   \also \\
		       & \mid  \ClassInv \\
		       & \mid \ClassHistoryConstr \\
		       & \mid \atIndex \\ 
		       & \mid \loopInv \\
		       & \mid \loopDecreases \\
		       & \mid \loopMod \\
		       & \mid \jmlKey{$\backslash$ typeof} \\
		       & \mid \jmlKey{$\backslash$ elemtype}\\
		       & \mid \TYPE\\
		       & \mid \result 
\end{array}
$$


\subsection{Informal semantics of BML}

In the following, we will discuss informally the interpretation of the syntax structures of BML.

\subsubsection{BML expressions}
Most of the expressions supported in BML have their counterpart in JML. As we will see hereafter, BML allows
to express field access, array access, method parameters and local variables, stack expressions etc.
The rule that produces the BML expression corresponds to the nonterminal   $\expressions$.
As we can see from the rule, we divide the expressions into two categories : $\expression$ \ and \ $\typeExp$.

Note that few of the expressions that are generated by the nonterminal $ \expression$ \  do not have analogs in JML.
 We first focus on the expressions produced by $\expression$ \  and particularly those that have a translation
in JML :


\begin{itemize}
      \item  $ \Constants$ represents the constants in BML. 
             A constant is either a signed or unsigned integer, or an identificator.  
             Integers are defined in a standard way. Identificators correspond 
	     to indexes in the constant pool of a Java class and are always prefixed by the symbol $\#$. 
      
      \item  $ \locVar{i}$ a local variable in the array of local variables of a method at index $i$. Note that the array of local variables of a method on bytecode level
             is the list of formal parameters of the variables declared locally in the method.
             This is slightly different from the Java language where difference is made between method parameters and variables declared locally to a method.	     
     
      \item  $\fieldAccess{ \expression}{\ident}$ stands for accessing the field which is at index $ \ident $  in the class constant pool.
             for the reference denoted by the expression
             $ \expression $. 
	     
	    

      \item  $\arrayAccess{\expression^1} {\expression^2} $ stands for an access to the element at index $ \expression^2$ 
             in the array denoted by the expression $ \expression^1$. This corresponds to the Java notation $\expression^1[\expression^2 ]  $ 

  

      \item  $\expression \ \op \ \expression$ stands for the usual arithmetic operations.
             $\op$ ranges over the standard  arithmetic operations $ + , - , * , div ,  rem $ 

   
      \item $\old{\expression}$  denotes the value of $\expression$ in the pre state of a method. This expression is usually used in the postcondition of a
            method and thus, allows that the postcondition predicate relate to the prestate

      \item $\EXC$ is a special specification identifier which denotes the thrown exception object in exceptional postconditions
 \end{itemize}

The expressions that are produced by the nonterminal $\expression$ and which cannot be translated in JML are related to
 the way in which the virtual machine works, i.e. we refer to the stack and the stack counter.
Because intermediate calculations are done by using the stack, often we will need stack expressions in order to characterise the states before and after an instruction
execution. Let's see how stack expressions are represented in BML:
\begin{itemize}
      \item  $\counter$ represents the stack counter.
  \item  $\stack{ \expression}$ stands for the element in the operand stack at position $\expression$
             Differently from the JML, our bytecode specification language has to take into 
             account the operand stack and its counter. Of course, those expressions may appear in predicates that refer to intermediate instructions
	     in the bytecode. 
	     For instance, the element below the stack top is represented with $\stack{\counter - 1}$ 
 \end{itemize}

Finally, the expressions generated by the nonterminal $\typeExp$ \ are:
\begin{itemize}
   \item  $ \typeof{\expression}$  denotes the dynamic type of the expression $\expression$ 

      \item $ \type{ \ident } $  denotes the class  described at index $\ident$ in the constant pool of the corresponding class file
 
      \item $\elemtype{\expression}$ denotes the type of the elements of the array $\expression$
      \item \TYPE \todo{explain what does it mean}

 \end{itemize}
\subsubsection{BML predicates}
 The properties that our bytecode language can express are from first order predicate logic. The formal grammar of the predicates is
 given by the nonterminal $\formulaBc$. From the formal syntax, we can notice that BML supports the standard logical connectors
 $\wedge, \vee, \Rightarrow $, existential $\exists$ and universal quantification $\forall$ as well as standard relation between
 the  expressions of our language like $\neq, = , \leq, \le \ldots$  
 

\subsubsection{Class Specification} 
Class specifications refer to properties that must hold in every visible state of a class. Thus, we have
two kind of properties concerning classes: 
\begin{itemize}
      \item \ClassInv. Class invariants are predicates that must hold in every visible state of a class.
                       This means that they must hold at the beginning and end of every method as well as whenever a method is called.
      \item \ClassHistoryConstr. Class history constraints is a property which states a relation between the pre state and poststate of every 
                                method in the corresponding class. 
      \item  $\declare \ \ghost \ \ident  \ \ident$ declares a special specification variable which we call ghost variable.
             These variables do not change the program behaviour although they might be assigned to as we shall see later in this section.
	     Ghost variables are used only for specification purposes and  are not ``seen'' by the Java Virtual Machine.   
\end{itemize}
 
We give in Fig.\ref{classSpec} an example of a class specification in Java source code.
 Note, that we give these examples on source code for the sake of clarity. The specification from the example declares one  invariant 
which states that the field {\verb a } must always be greater than  {\verb 0 }. This means for instance, that whenever the
 method  {\verb decrease } is called the invariant must hold and when the method terminates execution the invariant once again should hold. 
In the example we also have specified a history constraint which states that the  value of the instance variable  {\verb a } in the prestate 
of any method of class  {\verb C }  must be greater or equal to its value in the state when the method terminates execution. For instance, 
the history constraint is established by the method  {\verb decrease }
\begin{figure}
\begin{lstlisting}[frame=trbl]
class C {
   int a ;

   /@*
     * invariant a > 0; 
     * historyConstraint old(a) >= a;
     *@/
   public void decrease(int b) {
      ...
   }
}
\end{lstlisting}
\caption{\sc An example for class specification} \label{classSpec}
\end{figure}

\subsubsection{Inter --- method specification}
In this subsection, we will focus on the method specification which is visible by the other methods in the program.
We call this kind of method specification an inter method specification as it exports to the outside the method contracts.
In particular, a method exports a precondition, a normal postcondition, a list of exceptional postconditions for every possible exception that the method may throw
 and the list of locations that it may modify. Those four components is one specification case, i.e. they describe a particular behaviour of the method, i.e. that if in the
prestate of the method the specified precondition holds, then when the method terminates normally, the specified normal postcondition holds and if 
it terminates on an exception \texttt{E} then the specified exceptional postcondition for  \texttt{E} will hold in the poststate of the method.
  
We also allow that a method might have several specification cases. 
Note that the specification cases that BML supports is actually the desugared version 
of the different behaviours of a method as well as its inherited specification.
\todo{reference to JML desugaring}

\subsubsection{Method specification case}
A specification case \specCase \  consists of the following specification units:

 \begin{itemize}
      \item $\requires \ \formulaBc$ which represent the   precondition of the specification case. If such a clause is not explicitely written
            in the specification, then the default precondition  \Mytrue \ is implicite
      \item $\ensures \ \formulaBc$   which stands for the normal postcondition of the method in case the precondition held in the prestate.
            In case this clause is not written in the specification explicitely, then the default postcondition   \Mytrue \  must hold.
      \item $\modifies \ list \ \modifiesLoc $   which is the frame condition of the specification case and denotes the 
            the locations that may be modified by the method if the precondition of this specification case holds in the prestate.
	    This in particular means that a location that is not mentioned in the \modifies \  clause may be modified. 
	    If the modifies clause is omitted, then the default modifies specification is \modifies \ \everything
      \item \exsuresList \  is the list of the exceptional postconditions that should hold in this specification case. In particular,
            every element in the list of exceptional postconditions has the following structure 
	    $ \exsures \ ( \ident )  $ $\ \formulaBc$. 
	    Note that  at index $\ident$  there is a constant which stands for some exception class  $\mbox{\rm\texttt{Exc}}$.
	    The semantics of such a specification expression is that if the method 
	    containing the exceptional postcondition terminates on an exception of type   $\mbox{\rm\texttt{Exc}}$
	    then the predicate denoted by $\formulaBc$ must hold in the poststate. Note that the list of exceptional postcondition may be empty.
	    Also the list of exceptional postconditions might not be complete w.r.t. exceptions that may be thrown  by the method.
	    In both cases, for every
	    exception that might be thrown by the method for which no explicite exceptional postcondition is given,
	     we take the default exceptional postcondition
	    \Myfalse
\end{itemize}
If a method has only one specification case this means that the precondition of the method is always the precondition of the unique 
specification case. If a method has several specification cases then, when the method is invoked at least the precondition of one 
of the specification cases must hold. If the precondition of a particular specification case holds in the prestate this requires
that in the poststate the postcondition of the same specification case holds and only the locations mentioned in the frame condition
of this specification case may be modified during method execution.
For instance, the example in Fig. \ref{bcSpec} shows the method {\verb decrease } which is now specified with two specification cases.
The specification cases describe two different behaviours of the method. The first specification case states that if the method is invoked 
with parameter smaller than the instance variable   {\verb a } then the method will modify  {\verb a } by decreasing it with the value of the parameter
{\verb b }. The other specification case describes the behavior of the method in case the actual parameter of the method is greater than the instance
variable {\verb a }. 



\begin{figure}
\begin{lstlisting}[frame=trbl]
class C {
   int a ;
  
   /@* 
     * invariant a > 0; 
     * historyConstraint  old(a) >= a;
     *@/

   /@*  
     * requires a > b;
     * modifies a;
     * ensures  a == \old(a) - b;  
     * exsures (Exception) false;
     *
     * also
     * requires a <= b;
     * modifies nothing;
     * ensures  a == \old(a);
     * exsures (Exception) false;
     *@/   
   public void decrease(int b) {
     if ( a > b) {
       a = a - b;
   }
 }
}
\end{lstlisting}
\caption{\sc An example for an intra method specification } \label{bcSpec}
\end{figure}



\subsubsection{Intra --- method specification}
As we can see from the formal  grammar in subsection \ref{BCSL}, BML allows to specify a property that must hold at 
particular program point inside a method body. The nonterminal which describes the grammar of assertions is
 \intraMethodSpec. Let us see in detail what kind of specifications can be supported in BML:

\begin{itemize}
  \item $\atIndex \ nat$ specifies the index of the instruction which identifies the instruction
        to which the specification refers. We would like to note here that the style of specification in BML
        is slightly different from the JML style. First, JML specification is written directly in the source code in comments
	at the point in the program text where the specification must hold. Second, as the Java source language is structured, JML allows
	to specify a particular program structure. For instance, in Fig. \ref{replaceSrc} the reader may notice that the loop specification
	refers to the control structure which follows after the specification and which corresponds to the loop. However, on bytecode level we
	could not write ditrectly in the bytecode of a method body, as this will corrupt the performance of any standard Java Virtual Machine.
	That's why specification is written outside the bytecode text and contains also information about the instruction to which the specification
	refers. Then, as bytecode does not have control structures specification will always refer to a particular instruction in the bytecode. 
        For instance, loops on bytecode are identified by a unique loop entry instruction and thus, a loop invariant must hold basically every time
	the corresponding loop entry instruction is reached. 
  \item  $\intraSpec$ specifies the property that must hold in  every  state that reaches the instruction at the index  specified by $ \atIndex \ nat$.
         We allow the following local assertions:  
	 
\begin{itemize}
  \item  \loopSpec \ gives the specification of a loop. It has the following syntax: 
          \begin{itemize}
	     \item $\loopInv \  \formulaBc $  where $  \formulaBc $ is   the property that must hold  whenever the corresponding
	           loop entry instruction is reached during execution
             \item $\loopMod \ list \ loc$  is the list of locations modified in the loop. This means that at the borders
	           of every iteration (beginning and end), all the expressions not mentioned in the loop frame condition must have
		   the same value.  
             \item $ \loopDecreases \ \expression $ specifies the expression $\expression $ which guarantees loop termination. 
	           The values of   $\expression $ must be from a well founded set (usually from $\Myint$ type ) and the values
		    of   $\expression $ should decrease at every iteration 
            \end{itemize}

  \item  $  \assert \ \formulaBc $ specifies the predicate $\formulaBc $ that must hold at the corresponding position in the bytecode

  \item  $ \set \ \expression \  \expression $ is a special expression that allows to set the value of a specification ghost variable. This means
         that the first argument must denote a reference to a ghost variable, while the second expression is the new value that this 
	 ghost variable is assigned to.
\end{itemize}


\end{itemize}

\subsubsection{Frame conditions} 
As we already saw, method or loop specifications  might declare the locations that are modified 
by the method / loop. We use the same syntax in both of the cases where the modified expressions for methods or loops are
 specified with $  \modifies  \ list \   \modifiesLoc;$. The semantics of such a specification clause is that 
all the locations that are not mentioned in the $\modifies$ list must be unchanged.
The syntax of the expressions that might be modified by a method is determined by the nonterminal
$  \modifiesLoc$. We now look more closely what a modified expression can be:
\begin{itemize}
      \item  $ \fieldAccess{\expression}{\ident} $ states that the method / loop modifies the value of the field at index $\ident$ 
             in the constant pool for the reference denoted by  $\expression$ 
      \item $\locVar{i}$ states that the local variable may modified by a loop. Note that this kind of modified
            expression makes sense only for expressions modified in a loop.
	    However a modification of a local variable does not make sense for a method frame condition, as methods in Java are called by value, and
	    thus, a method can not cause a modification of a local variable that is observable by the rest of the program.
	    
      \item  $\arrayAccessMod{\expression}{\specIndex}$ states that the components at the indexes specified by $\specIndex$ in
            the array denoted by $\expression$ may be modified. The indexes of the array components that may be modified $\specIndex$
	    have the following syntax:
	    \begin{itemize}
	          \item $i$ is the index of the component at index $i$. For instance, \\
		        $\arrayAccessMod{\expression}{i}$ means that the array component at index $i$ might be modified. Of course, in order that
			such a specification make sense the following must hold:
			$ 0  \le i < \length(\expression )   $
	          \item \all \ specifies that all the components of the array may be modified, i.e. the expression 
		         $\arrayAccessMod{\expression}{\all}$ is a syntactic sugar for 
                       $$ \forall \ i ,   0  \le i < \length(\expression ) \Rightarrow \arrayAccessMod{\expression}{i }$$
		       
		  \item $ i_1 .. i_2$ specifies the interval of array components between the index $i_1$  and $i_2$.  
		        Thus, the modified expression  $\arrayAccessMod{\expression}{i_1 .. i_2}$ is a syntactic sugar for 
			  $$ \forall \ i ,  i_1 \leq i \wedge  i \leq i_2 \Rightarrow \arrayAccessMod{\expression}{i }$$
			  Here, once again the following conditions must hold, otherwise the expression does not
			  make sense :
			  $$ \begin{array}{l}
			           0  \le i_1 \\
				   i_2  < \length(\expression )
			     \end{array}
			  $$
	    \end{itemize}

      \item $\everything $ states that every location might be modified by the method / loop
      \item $\nothing$ states that no location might be modified by a method / loop
\end{itemize}
