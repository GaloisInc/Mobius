

\section{Related work}\label{bml:relWork}

The idea of introducing annotations in the bytecode is actually not so recent.  
For instance, the possibility to check a property at run-time, using the \texttt{assert}
 construct, has been long adopted in the C programming language and recently also in Java (Java
1.5, see \cite[\S 14.10]{JLS}). Checking runtime simple assertions,
 as for instance that a reference is not  null is certainly very useful as it
 allows for detecting bugs early in the development process.    
But  in the presence of mobile code, the need for more
 powerful specification mechanisms arises. In particular, because mobile code verification scheme
 relies on static verification schemes, such specification mechanisms must be suitable 
 for static verification. Moreover, bytecode specification language can be interesting for software audit
 which does not trust the compiler. Another reason to 
 consider a rich specification bytecode language  is the verification of programs which
 are written directly in bytecode. 

Although verification of bytecode programs has started to attract 
the scientific interest,  not too much work has been done in
the direction of bytecode specification. Several logics have been developed to reason about
bytecode but none of them discusses the issue about a formalism for expressing program properties.
For instance, in~\cite{BannwartMueller05} Bannwart \& M\"uller propose a general purpose Hoare style bytecode logic
 which is proven correct and complete. They also address the issue of compiling Hoare style derivations 
from the source language into Hoare style derivations over bytecode in order to cope with complex properties in a PCC scenarios.
 However, they do not discuss the issue of the formalism into which to
 express those program properties. Within the MRG project~\cite{AspinallEtAl:TPHOLs2004}, a resource aware logic is designed for Grail, a bytecode
language which combines functional and object oriented features.  However, there also the main focus is the development of a sound proof system.
% In both cases, the authors address the problem of building an infrastructure for PCC.
 %But in order to make such a framework work especially for nontrivial policies, the untrusted code should come  annotated.
 BML comes in reply to the need of  writing  and encoding understandable specifications for
bytecode.

The development of BML is clearly inspired by the development of the
JML specification language~\cite{JMLRefMan}. Both JML and
BML follow the Design by Contract principle introduced first in
Eiffel~\cite{Meyer97}. However, currently BML supports only a subset of JML which corresponds
basically to the so called Level 0 of JML~\cite{JMLRefMan}.
JML is a rich specification language which supports many specification constructs. 
The semantics of few of these specification constructs is still under discussion.
Despite this, JML is ``the most popular formal specification'' of Java and thus 
we consider that supporting a relation between a Java bytecode specification language and JML is important.  

% JVer is a tool to verify annotated
%bytecode~\cite{ChanderEILN05}. However, as specification language they
%use a subset of JML, \emph{i.e.}\ a source code level specification language.

The Extended Virtual Platform project\footnote{See \url{http://www.cs.usm.maine.edu/~mroyer/xvp/}.} also is very close to JML.
 This project aims at developing a framework that allows to compile JML annotations, to
allow run-time checking~\cite{AlagicXVP05}. However, in contrast to
our work, they do not intend to do static verification of bytecode
programs. Moreover, their platform takes JML-annotated source code
files as starting point, so it is not possible to annotate bytecode
applications directly.

The Spec\# programming system~\cite{BLS04sp} is probably the system which is closest to BML as  
introduces in similarly the Design by Contract principles into the C\#
programming language. %, both at source code level and for CIL, the .NETintermediate language.
The system consists of the following components:  a programming language Spec\#, a compiler from Spec\# to 
CIL,( the .NET intermediate language), a runtime checker and a static verification which is
 based on a intermediate language BoogiePL and the static verifier Boogie~\cite{BarnettCDJL05}.
 In particular, Spec\# is a superset of the programming language  C\#. Spec\# allows for expressing method (preconditions, postconditions and frameconditions )
 and class contracts as well as class contracts (class invariants). Moreover, the language extends the programming type system with non - null types.
 This in particular, means that one of the design goals of Spec\# is to alter the underlying programming language. This is a difference between JML and Spec \#
 and consequently, between BML and Spec\#. Both JML and BML are clearly separated from Java, i.e. they can not force a programmer 
 to change the programming code. The purpose of BML is to provide the means for verifying statically standard Java bytecode.
 Spec\# programs can be verified dynamically and statically. 
 Both verification styles work actually  on  CIL programs.
 Thus the Spec\# compiler produces not only CIL code but also metadata which contains 
 the specification. Moreover, Spec\# supports the C\# custom attributes in specifications which are compiled
 into metadata in the CIL format. This metadata  information is then used by the verification mechanisms on runtime
 or statically. For the moment,  BML is not connected to metadata which becomes official part of the newest
 Java 5 (Tiger). But note that  we have taken the design  decision to relate BML to JML via a compiler. For the moment we have not considered such possibilty 
 but we could exploit in the future the facilities of metadata via a compiler from Java metadata to BML. 
 



