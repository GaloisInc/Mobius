

\section{Related work}
The interest in specification and verification of bytecode
applications is quite recent, and not too much work has been done in
that direction. Several logics have been developed to reason about
bytecode, \emph{e.g.}~by Bannwart \& M\"uller~\cite{BM05plb}
and within the MRG project~\cite{AspinallEtAl:TPHOLs2004}. However,
in this work, no attention is given to how one can conveniently write
understandable specifications for bytecode.

The development of BML is clearly inspired by the development of the
JML specification language~\cite{december-jml}. Both JML and
BML follow the Design by Contract principle introduced first in
Eiffel~\cite{M97oos}. The Boogie project~\cite{leinoWPUP}
introduces in similarly the Design by Contract principles into the C\#
programming language, both at source code level and for CIL, the .NET
intermediate language.  The possibility to check a property at
run-time, using the \texttt{assert} construct, has been long 
adopted in the C programming language and recently also in Java (Java
1.5, see \cite[\S 14.10]{JLS}). 

Finally, we should mention the Extended Virtual Platform
project\footnote{See
\url{http://www.cs.usm.maine.edu/~mroyer/xvp/}.}. This project aims at
developing a framework that allows to compile JML annotations, to
allow run-time checking~\cite{AlagicXVP05}. However, in contrast to
our work, they do not intend to do static verification of bytecode
programs. Moreover, their platform takes JML-annotated source code
files as starting point, while with BML one is able to annotate
bytecode applications directly\footnote{Some security-critical
applications are written in bytecode directly, to avoid security
problems related with compilation. Thus, for such applications one
needs to be able to specify and verify them directly at this level.}.
