 %\section{Introduction}
 In the previous chapter, we defined a verification
 condition generator for a Java bytecode like language. We used a weakest precondition
 to build the verification conditions. In this section, we will show formally that the
 proposed verification condition generator is correct, or in other words that it is sufficient
 to prove the verification conditions generated over a method's body and its specification 
 for establishing that the method respects the specification. 
 In particular, we will prove the correctness of our methodology w.r.t. the operational semantics of our bytecode language
 given in chapter \ref{opSem}. 


 

% We now proceed with the proof of the partial correctness of the weakest precondition calculus.
% Note also that in the following we do not consider recursive methods.
 In the following, in Section \ref{proof:outline} we first start with an outline of the proof.   
 The second Section \ref{substProp} establishes the relation between syntactic substitution  and semantic evaluation.
 The latter will play a role in the correctness proof  of the verification condition generator
 in Section  \ref{proof}.
% Section \ref{proof} starts with a formal definition for method correctness. Then, we establish the correctness
% of a single instruction (lemma \ref{lemma0}). The next step of the proof  is to establish that if all the steps
% in an execution path establish the intermediate predicates then the execution can either proceed by establishing
% the next weakest precondition predicate or will terminate in a state which respects the adequate postcondition.
  



