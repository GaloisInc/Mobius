\section{Proof outline} \label{proof:outline}
\todo{NB: pour les stacks est-ce qu'ils peuvent etre modifies? Le probleme c'est qu'on assume que le bytecode verifier
a passe sur le programme, et donc pour chaque instruction le stack va avoir le meme type et profondeur avant l'execution de l'instruction.
Alors est-ce que ca c'est grave?}
 We would first like to remark that in the following, we shall establish the soundness of our verification condition generator
 w.r.t. method correctness. For instance, the proof does not deal with the soundness of our
 methodology w.r.t. class invariants, class history constraints, neither about ghost variables. The reason for
 this is that their verification is the same on source and bytecode level. Establishing the semantics and 
 verifying class invariants in JML in a modular way, which is our front end language, is a subject which  by itself
 is complex (see \cite{barnett03verification}). Thus,  we do not consider that these verification issues are particular
 to the bytecode verification  techniques and thus, we shall omit to discuss their soundness here. 

Also, we assume that there are no recursive methods in the program. 
Proving the correctness of a verification calculus in the presence 
of recursive methods would require a notion of recursive method call depth in the operational
semantics. This would complicate the proof without bringing any particular feature of the bytecode.
The reader interested in how recursive methods are manipulated in the soundness proof
of program logics might look at \cite{Nipkow-MOD2001}.  

Remember that we assume that the control flow graph of a method is reducible, or in other words there 
are no cycles in the graph with two entries. This means that if program have cycles then they should 
conform to Def. \ref{defLoop} from Section \ref{prelim:ctrFlow}.
As we shall in the following, control flow graph reducibility plays an important role in
 the proof of soundness of the verification condition generator. 
Note that this restriction (as we said earlier) is realistic as every non-optimizing Java compiler produces
reducible control flow graphs and even hand written code is usually reducible.


We would like to make a last but not least remark. The proof for soundness presented here would seem large w.r.t. 
the proof of soundness of Spec\# presented in  \cite{leinoWPUP}. Spec\# relies on several transformations over the 
bytecode. First, a transformation of the potentially irreducible control flow graph to a reducible one is done.
The second step  eliminates loops. The third step consists in translating programs into a single-assignment form.
The fourth step is converting the program into a  guarded command language and finally, the fifth and last 
step is passifying it which means changing assignments
to \texttt{assume} statements. The proof presented in  
\cite{leinoWPUP} is done for  programs written in passified guarded command language.
But what is the formal guarantee that the initial  bytecode program is also correct? 
A proof of soundness for Spec\# which takes into account only the last three stages can be 
already complex. 


Before entering in the technical details of the proof, we 
give here an informal description of the steps to be overtaken there.
We will describe our reasoning in the direction opposite to its formalization in the later sections.
 We start with our main objective and then start to ``zoom'' in the steps that must be made in order that it be achieved. 
Every time we discuss informally a step in the proof we point out the lemma in the formal proof to which it refers.

Let us first see what exactly we want to show. 
We would like to establish that if the verification conditions
generated for a method 
are valid and the precondition of the method holds in the initial state of the method 
then the postcondition of  the method holds in its final state. This is formally given in Definition \ref{defCorrect}, Section \ref{proof}
and the main theorem \ref{vcGenCorrect} formally establishes that the verification conditions generated with the help of \fwpi{} respect this definition. 

To do this, we first show that if the precondition calculated by the \fwpi{} for the entry instruction of the method
holds in the initial state of the method
then the postcondition of the method will hold in the final state of the method provided that the method terminates
(Lemma \ref{lemma2}). 

 For this, we have to establish that if the precondition calculated by
the \fwpi{} for the entry instruction of the method holds in the initial state of the method then the precondition calculated by \fwpi{}
for every instruction reachable from the method entry instruction also holds in the prestate of that instruction. This is done in Lemma \ref{lemma3}. 
We use here an induction over the number of execution steps made in the execution path.
The induction case uses Lemma \ref{lemma1}.

Lemma \ref{lemma1}  shows that if in an execution path
   the preconditions calculated by \fwpi{} for  the instructions in the path are such that the respective precondition holds in the prestate
of the respective instruction then either the respective normal or exceptional method postcondition holds or another execution step can be made and
the weakest precondition of the next instruction  holds. This is also known as  progress property.  
  The latter lemma is may be the most complicated one as it uses the argument of the reducibility 
 of the execution graph (Section \ref{prelim:ctrFlow}, Def. \ref{defLoop}).
   The lemma has three cases: 

\begin{itemize}
    \item the case where the next instruction is not a loop entry instruction. In this case the proof is standard and uses the single step soundness of \fwpi.
    \item the case when the next instruction is a loop entry instruction and the current instruction is not a loop end. 
          In this case,  we use the single step soundness of \fwpi{} as well the special form of the precondition of the current instruction. 
    \item the case where the next instruction is a loop entry and the current is a loop end instruction (see Def. \ref{defLoop}).
          In this case, we use  the reducibility of the control flow graph which gives us that the execution path has 
	  a prefix subpath which passes through the loop entry instruction but not through the current loop end instruction. 
	  This fact allows us to conclude that also in that case the Lemma \ref{lemma1} holds
\end{itemize}

 What we mean by single step soundness of the \fwpi{} function is  
that if the predicate calculated by \fwpi{} for an instruction holds in its prestate then the postcondition 
upon which the precondition is calculated holds in its poststate (Lemma \ref{lemma0}). The argument for
the single step soundness uses the relation between syntactic substitution and semantic evaluation which looks
like:


$$\evalExp{\substitution{\expression_1}{ \expression_2 }{ \expression_3 }}{ s_1 } =
 \evalExp{\expression_1}{ \update{s_1}{ \evalExp{\expression_2}{s_1} }{\evalExp{\expression_3}{s_1} } } $$ 

This  equivalence is standardly  used for establishing the soundness of predicate transformer function w.r.t. a program semantics
and means that it does not matter if we use a syntactic substitution over the expression and evaluate the resulting expression  
or update the state and evaluate the original expression. The next section is dedicated to the relation between substitution and evaluation.

 
 
