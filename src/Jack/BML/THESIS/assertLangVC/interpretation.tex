

\newtheorem{interpExpr}{Definition}[subsection]
\newtheorem{interpTypeExpr}[interpExpr]{Definition} 
\newtheorem{interpPred}[interpExpr]{Definition}


\section{Interpretation}\label{interpret}

In this section, we shall focus on the semantics of formulas and expressions w.r.t. a state configuration.
The part which deserves a more  careful treating is the question of partiality in our programming language.

%  brief presentation of the problem 
We first discuss  the impact of the partial function evaluation on the definition of the semantics. 
Partiality is   a  fact  in program languages and at the same time represents a
 controversal point in defining a program logic. 

 Several logics has been proposed for dealing with partial functions, 
every one bringing certain advantages as well as disadvantages (see \cite{gries95avoiding,schieder99adapting}).
Three valued logic, where expressions may be evaluated to a value or  undefined  in a state and
 formulas might be false, true or undefined in a state appear to lose certain nice properties
 which standard logic with equality have
as for instance associativity of equality, the excluded middle \cite{gries95avoiding} etc.
 In \cite{bijlsma90sqb}, Bijlsma comes up with another solution where the  properties of the logic are preserved but 
however compositionality of expression evaluation does not hold anymore.

In \cite{gries95avoiding}, Gries and Schnieder give another solution to the problem which consists in underspecification 
and thus  avoiding the special object of undefideness.
 More particularly, their approach considers all functions as total but for some value of the argument 
the function is not specified. 

In \cite{schieder99adapting},  B. Schieder and M. Broy notice that 
  an extension  of the calculus of boolean structures proposed by Dijkstra and Scholten (\cite{WPCDS}) with a  third boolean value 
undefined where  the logical connectives are extended in a nonmonotonic way preserves the properties of the classical logic
on the price  of having two sorts of equality. 

Our formalization is simple and escapes the theoretical problems of a three valued logic by keeping the evaluation function partial but
 allowing undefideness only for expressions.
Thus, an expression evaluation may return a value or may  be not defined. On the contrary, formulas are still either true or false.   


Thus, we first define a function for expression evaluation
 $\evalExpName$ which evaluates expressions in a state has the following signature:
$$
\evalExpName : \expression \rightarrow \SetConfigs  \rightarrow \SetConfigs  \rightharpoonup \Values \cup \JavaType
$$
Note that the evaluation function is partial and  takes as arguments an expression of the assertion language presented in the previuos Section 
\ref{assertLang:lang} and two states (see Section \ref{def})  and returns a value as defined in Section \ref{types}. In the following,
if the evaluation   of an expression \expression{} is defined for the states $s$ $s_0$, i.e. the evaluation function returns a value,
  we use the notation  $\defined{ \evalExp{\expression}{s}}$.  





%The function $eval$ associates a value to expressions from  \expression  \  if they have a value in the state.
%For instance, a reference which is not in the domain of the  



\begin{interpExpr}[Evaluation of expressions] \label{interpExpr} 
The evaluation in a state \\
$s = \config{\heap}{\counterOnly}{ \stackOnly }{\locVarOnly}{\pc }$  or $s = \configFinal{\heap}{\locVarOnly}{\Final }$  of an expression $\expression$
w.r.t. an initial state $ s_{0} = \config{\heap_{0}}{ 0 }{ \lbrack \ \rbrack  }{\locVarOnly}{ 0 } $ 
is denoted with $\evalExp{\expression}{s}$  and is defined inductively over the grammar of expressions $\expression$ \  as follows:
 
$$
\begin{array}{l}
\evalExp{ v  }{s } = v\\
  where \  v \in \  \Myint   \ \vee  \  v \in  \RefValues \\
\\
 \evalExp{\fieldd(\exprWp ) }{s} = \\
 = \heap(\fieldd) (\evalExp{ \exprWp}{ s } ) \\
\\

 \evalExp{\update{\fieldd}{\expression_1}{\expression_2}(\expression_3)}{ s } = \\
=  \update{\heap}{\fieldd }{\update{\fieldd}{  \evalExp{\expression_1}{s } }{ \evalExp{\expression_2}{s }  }} (\fieldd)
                                            (  \evalExp{\expression_3}{ s } ) \\
 \\


 \evalExp{\arrayAccess{\expression_1} {\expression_2}  }{ s } = \\

 = \heap (\evalExp{ \expression_1}{ s } ,\evalExp{ \expression_2}{ s } )   \\
\\

 \evalExp{ \update{ \arrayAccessOnly}{ (\expression_1 , \expression_2)}{ \expression_3} (\expression_4,\expression_5)  } { s } = \\
 = \update{\heap}{ ( \evalExp{\expression_1}{s } ,  \evalExp{\expression_2}{s } ) }  
                 { \evalExp{\expression_3}{s }}\\
                 \Myspace ( \evalExp{\expression_4}{s } ,  \evalExp{\expression_5}{ s  } ) \\
\\
 \evalExp{ \locVar{i} } { s } = \locVarOnly(i) \\
\\

 \evalExp{ \old{\exprWp} } { s } =\evalExp{ \exprWp }{  s_{0}} \\
\\ 
 \evalExp{\expression_1 \ \op \ \expression_2   } { s } =   \evalExp{\expression_1}{s} \op  \evalExp{\expression_2}{s}  \\

\\
\evalExp{ \typeof{\exprWp}}{s } = \\
\left\{\begin{array}{ll}
      \Myint                                           & \evalExp{\exprWp}{s} \in \Myint \\
      \heap. \heapTypeOf(\evalExp{\exprWp}{s} ) & else \\
	   
\end{array}\right. \\

\\ 
\evalExp{ \elemtype{\exprWp}}{s } = \\  
\left\{\begin{array}{ll}
            \mbox{ \rm \texttt{T} }  & if \   \heap. \heapTypeOf(\evalExp{\exprWp}{s} ) = \mbox{ \rm \texttt{T[ ] } } 
\end{array}\right. \\					    
\\

\evalExp{ \TYPE }{s } = \mbox{ \rm \texttt{java.lang.Class}} 
\end{array}
$$

The evaluation of stack expressions can be done only in intermediate state configurations $s = \config{\heap}{\counterOnly}{ \stackOnly }{\locVarOnly}{\pc }$ :
$$
\begin{array}{ll}
 \evalExp{ \counter   } { s } = \counterOnly \\
\\

 \evalExp{ \stack{ \exprWp}   } { s } = \stackOnly ( \evalExp{\exprWp}{s} ) \\
\\

\end{array}
$$
The evaluation of the following expressions can be done only in a final state $s = \configFinal{\heap}{\locVarOnly}{\Final }$:
$$
\begin{array}{ll}
\evalExp{ \result }{s } = \Res  & where \ s=  \configFinalNorm{\heap}{\locVarOnly}{\Res} \\
\evalExp{ \EXC }{s } = \Exc  & where \ s=  \configFinalExc{\heap}{\locVarOnly}{\Exc}
\end{array}
$$
  

\end{interpExpr}



 
The relation $\vDash$ that we define next, gives a meaning to the formulas from our
 assertion language $\predWp$.
%$$ \vDash :  \SetConfigs *  \predWp  $$
 
\begin{interpPred}[Interpretation of predicates] \label{interpPred} 
The interpretation $ s \vDash \predWp$ of a predicate $\predWp$ in a state configuration $s = \config{\heap}{\counterOnly}{ \stackOnly }{\locVarOnly}{\pc }$ 
w.r.t. an initial state  $ s_{0} = \config{\heap_{0}}{ 0 }{ \lbrack \ \rbrack  }{\locVarOnly}{ 0 } $ is defined inductively as follows:
$$
\begin{array}{l}
\interp{\true}{s} \  is \ true \ in \ any \ state \ s \\
\\
\interp{\false}{s} \ is \ false \ in \ any \ state \ s \\
\\
\interp{  \neg \ \predWp  }{s} \ if \ and \ only \ if  \  not \ \interp{\predWp}{s} \\ 
\\

\interp{\predWp_1  \wedge  \predWp_2 }{s} \ if \ and \ only \ if  \ \interp{\predWp_1}{s} \ and \ \interp{\predWp_2}{s}  \\
\\

\interp{\predWp_1  \vee  \predWp_2 }{s} \ if \ and \ only \ if  \ \interp{\predWp_1}{s} \ or \ \interp{\predWp_2}{s}  \\
\\
\interp{\predWp_1  \Rightarrow  \predWp_2 }{s} \ if \ and \ only \ if  \ if \ \interp{\predWp_1}{s} \ then \ \interp{\predWp_2}{s}  \\
\\
\interp{\predWp_1  \ if \ and \ only \ if  \predWp_2 }{s} \ if \ and \ only \ if  \  \interp{\predWp_1}{s} \ if \ and \ only \ if  \ \interp{\predWp_2}{s}  \\
\\
\interp{\forall x : T .  \predWp(x)   }{s} \ if \ and \ only \ if  \ forall \ value \ \mbox{ \rm \textbf{v}} \ of \  type \  T \ \interp{\predWp(\mbox{ \rm \textbf{v}})}{s}  \\
\\

\interp{\exists x : T .  \predWp(x)   }{s} \ if \ and \ only \ if  \ a \ value \ \mbox{ \rm \textbf{v}} \ of \  type \  T \ exists \ such \ that \ \interp{\predWp(\mbox{ \rm \textbf{v}})}{s}  \\
\\
\interp{\expression_1 \  \predicates \  \expression_2 }{s} \ if \ and \ only \ if  \begin{array}{l}
                                                              \defined{ \evalExp{\expression_1 }{s}}  \wedge \\
							      \defined{ \evalExp{\expression_2 }{s} } \wedge \\
                                                             \evalExp{\expression_1 }{s} \evalRel{\predicates }  \evalExp{\expression_2 }{s} \ is  \ true

							     \end{array} \\
\\
\\
\interp{\instances(\referenceOnly) }{s}, where \ \referenceOnly \in \RefValues \ if \ and \ only \ if  \    \isInList{\referenceOnly }{ \getLocations{\heap_{0}}}  
\end{array}
$$   
\end{interpPred}
Note that the upper definition interprets atomic formulas in  states $s$ and $s_0$, i.e. formulas which 
represent relations between expressions, as true only if the evaluation of these expressions is defined
in $s$ and $s_0$. However, the interpretation of the logical connectors is standard.
 
