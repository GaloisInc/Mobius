
%\newtheorem{wellDef}{Definition}[section]
\newtheorem{interpExpr}{Definition}[section]
\newtheorem{interpTypeExpr}[interpExpr]{Definition} 
\newtheorem{interpPred}[interpExpr]{Definition}


\section{Interpretation}\label{interpret}

In this section, we shall focus on the semantics of formulas and expressions w.r.t. a state configuration.
Note that we shall give a semantics which ignores partiality in the language. Of course, this is an 
important theoretical question which by itself have given rise to a lot of research and has received different answers.
%The part which deserves a more  careful attention is the question of partiality in the language.
What we mean by partiality is the existence of functions like division or dereferencing of a field, array indexing etc. 
To get a precise idea of the problem we can consider the following logical statements: 

$$
\begin{array}{l} 
\numConclusion{1} \  \fieldd(\exprWp ) == 3\\
\numConclusion{2} \ \arrayAccess{\expression_1} {\expression_2} == 5
\end{array}
$$   

Under certain conditions, these formulas may not have a meaning. In case 
\numConclusion{1}  the statement does not make sense if
$\exprWp$ is \Mynull. In case  \numConclusion{2}  the statement does not make sense if 
either $\expression_1 $ is \Mynull{} or $\expression_2$  is not in the bounds of $\expression_1 $.


Building a logic for partiality is not trivial. Different solutions exist. 
Gries and Schnieder \cite{gries95avoiding} give a solution to the problem which consists in function 
underspecification  and thus  avoid the problem of undefideness.
 More particularly, their approach considers all functions as total but for some value of the argument 
the function is not specified.  B. Schieder and M. Broy \cite{schieder99adapting} give 
  an extension  of the calculus of boolean structures proposed by Dijkstra and Scholten (\cite{WPCDS}) with a  third boolean value 
undefined where  the logical connectives are extended in a nonmonotonic way preserves the properties of the classical logic
on the price  of having two sorts of equality.
A naive three valued logic, where expressions may be evaluated to a value or  undefined  in a state and
 formulas might be false, true or undefined in a state appear to lose certain nice properties
 which standard logic with equality have
as for instance associativity of equality, the excluded middle \cite{gries95avoiding} etc. 
In \cite{burdy98treatment}, L. Burdy shows a three valued logic in which the above features of classical logic are preserved.
a well-definedness operator $\wellDefined{}: \expression \cup \predWp \rightarrow \predWp$ over expressions and formulas. Thus formulas can be either true, false or not defined and
expressions may evaluate to a value or be undefined. The operator  $\wellDefined{\expression}$ ($\wellDefined{\predWp}$) 
 gives the, necessary and sufficient
 conditions such that $\expression$(\predWp ) is defined.
 More particularly, the application of the operator   $\wellDefined{\expression}$  ($\wellDefined{\predWp}$) over $\expression$  (\predWp ) 
 holds in a state only if the expression $\expression$ has  a value in this state, otherwise it is false.

In the following we shall abstract from partiality of evaluation and we shall consider that our evaluation is a total function. 
Note that we could have used for instance the formalization in \cite{burdy98treatment}, i.e. 
a three valued interpretation for formulas and expressions with an operator for well-definedness over predicates and expressions 
whose interpretation to true would mean that the predicate / expression is defined. We however, consider that this only will complicate 
the present formalization without bringing any original feature of the present thesis. 
This means that for instance, for predicates which contain field dereference  we shall assume that the reference is not \Mynull,
that arrays are always accessed inside their bounds etc. 
%%%%%%%%%%%%%%%%%%%%%%%%%%%%%%%%%%%%%deprcated%%%%%%%%%%%%%%%%%%%%%%%%%%%%%%%%%%%%%%%%%%%%5
% In \cite{bijlsma90sqb}, Bijlsma comes up with another solution where the  properties of the logic are preserved but  however compositionality of expression evaluation does not hold anymore.

%In \cite{gries95avoiding}, Gries and Schnieder give another solution to the problem which consists in underspecification 
%and thus  avoiding the special object of undefideness.
% More particularly, their approach considers all functions as total but for some value of the argument 
%the function is not specified. 

%In \cite{schieder99adapting},  B. Schieder and M. Broy give 
%  an extension  of the calculus of boolean structures proposed by Dijkstra and Scholten (\cite{WPCDS}) with a  third boolean value 
%undefined where  the logical connectives are extended in a nonmonotonic way preserves the properties of the classical logic
%on the price  of having two sorts of equality. 
%%%%%%%%%%%%%%%%%%%%%%%%%%%%%%%%%%%%%deprcated%%%%%%%%%%%%%%%%%%%%%%%%%%%%%%%%%%%%%%%%%%%%5

%In \cite{} \todo{Owe, Lilian}, the authors propose a three valued logic for dealing with partiality by introducing
%a well-definedness operator $\wellDefined{}: \expression \cup \predWp \rightarrow \predWp$ over expressions and formulas. Thus formulas can be either true, false or not defined and
%expressions may evaluate to a value or be undefined. The operator  $\wellDefined{\expression}$ ($\wellDefined{\predWp}$) 
% gives the, necessary and sufficient
% conditions such that $\expression$(\predWp ) is defined.
% More particularly, the application of the operator   $\wellDefined{\expression}$  ($\wellDefined{\predWp}$) over $\expression$  (\predWp ) 
% holds in a state only if the expression $\expression$ has  a value in this state, otherwise it is false. To get a more precise idea of the operator $\wellDefined{}$
%we scatch it hereafter:

 
%\begin{wellDef}[Well defined expressions] \label{wellDef}

%$$
%\begin{array}{l}
%\wellDefined{v} = \Mytrue \\
%\\
%\wellDefined{\fieldd(\expression )} =\wellDefined{\expression} \wedge  \expression \neq \Mynull \\
%\\
%\wellDefined{ \arrayAccess{\expression_1} {\expression_2}   } = 
%           \begin{array}{l}
%             \expression_1 \neq \Mynull \wedge\\
%	     0 \le  \expression_2 < \arraylength(\expression_1)
%           \end{array} \\
%
%\wellDefined{\predWp_1 \wedge \predWp_2} =  \begin{array}{l} 
%                                                     \wellDefined{\predWp_1} \wedge  \wellDefined{\predWp_2} \wedge \\
%                                                     
%                                          
%                                       
%                                            \end{array} \\                                                
%\ldots 
%\end{array}
%$$ 

 


 
The relation $\vDash$ that we define next, gives a meaning to the formulas from our
 assertion language $\predWp$ w.r.t. two state configurations - a current state and an initial state.

% Note that the interpretation of 
%  $\wellDefined{\expression}$($\wellDefined{\predWp}$) cannot be undefined, it is either true or false.
%As we shall see, the relation  $\vDash$  interprets formulas to the $true$ value under the condition
%is well -defined in the sense of the operator $\wellDefined{}$
%$$ \vDash :  \SetConfigs *  \predWp  $$
 
\begin{interpPred}[Interpretation of predicates] \label{interpPred} 
The interpretation $ s \vDash \predWp$ of a predicate $\predWp$ in a state configuration $s = \config{\heap}{\counterOnly}{ \stackOnly }{\locVarOnly}{\pc }$ 
w.r.t. an initial state  $ s_{0} = \config{\heap_{0}}{ 0 }{ \lbrack \ \rbrack  }{\locVarOnly}{ 0 } $ is defined inductively as follows:
$$
\begin{array}{l}
\interp{\true}{s} \  is \ true \ in \ any \ state \ s \\
\\
\interp{\false}{s} \ is \ false \ in \ any \ state \ s \\
\\
\interp{  \neg \ \predWp  }{s} \ if \ and \ only \ if   \ \neg \ \interp{\predWp}{s} \\ 
\\

\interp{\predWp_1  \wedge  \predWp_2 }{s} \ if \ and \ only \ if   \ \interp{\predWp_1}{s} \ and \ \interp{\predWp_2}{s}  \\
\\

\interp{\predWp_1  \vee  \predWp_2 }{s} \ if \ and \ only \ if   \   \interp{\predWp_1}{s} \ or \ \interp{\predWp_2}{s}  \\
\\
\interp{\predWp_1  \Rightarrow  \predWp_2 }{s} \ if \ and \ only \ if    \  \interp{\predWp_1}{s} \ then \ \interp{\predWp_2}{s}  \\
\\

\\
\interp{\forall x : T .  \predWp(x)   }{s} \ if \ and \ only \ if \  \ forall \ value \ \mbox{ \rm \textbf{v}} \ of \  type \  T \ \interp{\predWp(\mbox{ \rm \textbf{v}})}{s}  \\
\\

\interp{\exists x : T .  \predWp(x)   }{s} \ if \ and \ only \ if  \ a \ value \ \mbox{ \rm \textbf{v}} \ of \  type \  T \ exists \ such \ that \ \interp{\predWp(\mbox{ \rm \textbf{v}})}{s}  \\
\\
\interp{\expression_1 \  \predicates \  \expression_2 }{s} \ if \ and \ only \ if  \
                                                             \evalExp{\expression_1 }{s} \evalRel{\predicates }  \evalExp{\expression_2 }{s} \ is  \ true

							    
\\
\\
\interp{\instances(\referenceOnly) }{s}, where \ \referenceOnly \in \RefValues \ if \ and \ only \ if  \    \referenceOnly \in \heap_{0}.\heapLocs
\end{array}
$$   
\end{interpPred}

% Note that the upper definition interprets atomic formulas in  states $s$ and $s_0$, i.e. formulas which 
%represent relations between expressions, as true only if the evaluation of these expressions is defined
%in $s$ and $s_0$. However, the interpretation of the logical connectors is standard.
 



Next, we define a function for expression evaluation
 $\evalExpName$ which evaluates expressions in a state has the following signature:
%\rightharpoonup
$$
\evalExpName : \expression \rightarrow \SetConfigs  \rightarrow \SetConfigs  \rightarrow  \Values \cup \JavaType
$$
Note that the evaluation function is total and  takes as arguments an expression of the assertion language presented in the previuos Section 
\ref{assertLang:lang} and two states (see Section \ref{def})  and returns a value as defined in Section \ref{types}.
% In the following,
%if the evaluation   of an expression \expression{} is defined for the states $s$ $s_0$, i.e. the evaluation function returns a value,
%  we use the notation  $\defined{ \evalExp{\expression}{s}}$.  





%The function $eval$ associates a value to expressions from  \expression  \  if they have a value in the state.
%For instance, a reference which is not in the domain of the  



\begin{interpExpr}[Evaluation of expressions] \label{interpExpr} 
The evaluation in a state \\
$s = \config{\heap}{\counterOnly}{ \stackOnly }{\locVarOnly}{\pc }$  or $s = \configFinal{\heap}{\locVarOnly}{\Final }$  of an expression $\expression$
w.r.t. an initial state $ s_{0} = \config{\heap_{0}}{ 0 }{ \lbrack \ \rbrack  }{\locVarOnly}{ 0 } $ 
is denoted with $\evalExp{\expression}{s}$  and is defined inductively over the grammar of expressions $\expression$ \  as follows:
 
$$
\begin{array}{l}
\evalExp{ v  }{s } = v\\
  where \  v \in \  \Myint   \ \vee  \  v \in  \RefValues \\
\\
 \evalExp{\fieldd(\expression ) }{s} = \\
 = \heap(\fieldd) (\evalExp{ \expression}{ s } ) \\
\\

 \evalExp{\update{\fieldd}{\expression_1}{\expression_2}(\expression_3)}{ s } = \\
=  \update{\heap}{\fieldd }{\update{\fieldd}{  \evalExp{\expression_1}{s } }{ \evalExp{\expression_2}{s }  }} (\fieldd)
                                            (  \evalExp{\expression_3}{ s } ) \\
 \\


 \evalExp{\arrayAccess{\expression_1} {\expression_2}  }{ s } = \\

 = \heap (\evalExp{ \expression_1}{ s } ,\evalExp{ \expression_2}{ s } )   \\
\\
 \evalExp{ \update{ \arrayAccessOnly}{ (\expression_1 , \expression_2)}{ \expression_3} (\expression_4,\expression_5)  } { s } = \\
 = \update{\heap}{ ( \evalExp{\expression_1}{s } ,  \evalExp{\expression_2}{s } ) }  
                 { \evalExp{\expression_3}{s }}\\
                 \Myspace ( \evalExp{\expression_4}{s } ,  \evalExp{\expression_5}{ s  } ) \\
\\
 \evalExp{ \locVar{i} } { s } = \locVarOnly(i) \\
\\

 \evalExp{ \old{\exprWp} } { s } =\evalExp{ \exprWp }{  s_{0}} \\
\\ 
 \evalExp{\expression_1 \ \op \ \expression_2   } { s } =   \evalExp{\expression_1}{s} \op  \evalExp{\expression_2}{s}  \\

\\
\evalExp{ \typeof{\exprWp}}{s } = \\
\left\{\begin{array}{ll}
      \Myint                                           & \evalExp{\exprWp}{s} \in \Myint \\
      \heap. \heapTypeOf(\evalExp{\exprWp}{s} ) & else \\
	   
\end{array}\right. \\

\\ 
\evalExp{ \elemtype{\exprWp}}{s } = \\  
\left\{\begin{array}{ll}
            \mbox{ \rm \texttt{T} }  & if \   \heap. \heapTypeOf(\evalExp{\exprWp}{s} ) = \mbox{ \rm \texttt{T[ ] } } 
\end{array}\right. \\					    
\\

\evalExp{ \TYPE }{s } = \mbox{ \rm \texttt{java.lang.Class}} 
\end{array}
$$

The evaluation of stack expressions can be done only in intermediate state configurations $s = \config{\heap}{\counterOnly}{ \stackOnly }{\locVarOnly}{\pc }$ :
$$
\begin{array}{ll}
 \evalExp{ \counter   } { s } = \counterOnly \\
\\

 \evalExp{ \stack{ \exprWp}   } { s } = \stackOnly ( \evalExp{\exprWp}{s} ) \\
\\

\end{array}
$$
The evaluation of the following expressions can be done only in a final state $s = \configFinal{\heap}{\locVarOnly}{\Final }$:
$$
\begin{array}{ll}
\evalExp{ \result }{s } = \Res  & where \ s=  \configFinalNorm{\heap}{\locVarOnly}{\Res} \\
\evalExp{ \EXC }{s } = \Exc  & where \ s=  \configFinalExc{\heap}{\locVarOnly}{\Exc}
\end{array}
$$
  

\end{interpExpr}

