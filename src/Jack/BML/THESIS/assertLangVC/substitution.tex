\section{Substitution} \label{subst}
In this section we focus on how substitution is defined in our assertion language. Basically, it is is defined inductively 
in a standard way over the expression and formula structure. Still, we
extend substitution to deal with field and array update as follows:

$$ \substitution{\expression}{ \fieldd }{ \update{\fieldd}{\expression}{\expression} } $$
This substitution does not affect any of the ground expressions,, i.e. it does not affect 
local variables ($\locVar{i}$), the constants of our language (\ConstantsWp),  the stack counter (\counter), the result expression
(\result), the thrown exception instance variable (\EXC). For instance, the following substitution does not change $\locVar{1}$: 
$$
 \substitution{\locVar{1}}{ \fieldd }{ \update{\fieldd}{\expression}{\expression} } = \locVar{1}
$$   

Field substitution affects only field objects as we see in the following: 



 
 %This is done by establishing the substitution rule for field access as follows:
 %$$  \substitution{\fieldd ( \expression^1)}{ \expression^2}{ \expression^3} =
 %\substitution{\fieldd}{ \expression^2}{ \expression^3} (  \substitution{ \expression^1  }{ \expression^2}{ \expression^3}) $$


% Let us see how we define  substitution over field objects:
$$  
\begin{array}{l}
\substitution{\fieldAccess{\expression}{\fieldd^1}}{\fieldd^2}{ \fieldd^2[ \oplus  \expression^1 \longrightarrow  \expression^2 ] } = \\\\
\left\{\begin{array}{ll}
       \fieldAccess{\expression}{\fieldd^1}    & if \ \fieldd^1 \neq \fieldd^2 \\
       & \\
       \fieldAccess{\expression} {\fieldd^2[ \oplus  \expression^1 \longrightarrow  \expression^2 ]} & else 
 \end{array}
\right. \\
\\
\\
\substitution {\update{\fieldAccess{\expression}{\fieldd^1}}{  \expression^1 }{  \expression^2}} {\fieldd^2}{  \update{\fieldd^2}{  \expression^3 }{  \expression^4}       } = \\\\
\left\{\begin{array}{ll}  

\update{\fieldd^1}{ \substitution { \expression^1}{\fieldd^2}{  \update{\fieldd^2}{  \expression^3 }{  \expression^4}}  }{ \substitution { \expression^2}{\fieldd^2}{  \update{\fieldd^2}{  \expression^3 }{  \expression^4}} } & if \ \fieldd^1 \neq \fieldd^2  \\

%\fieldd[ \oplus \mbox{ \rm \texttt{r} }\substitution{ \expression^1}{ \expression^2} \longrightarrow  \mbox{ \rm \texttt{v} }\substitution{ \expression^1}{ \expression^2} ] 

 \\ 
% else \   \fieldd^1 = \fieldd^2  \ then \\ 
\fieldd^1 \begin{array}{l}
             \lbrack \oplus  \substitution { \expression^1}{\fieldd^2}{  \update{\fieldd^2}{  \expression^3 }{  \expression^4}}  \longrightarrow  
                 \substitution { \expression^2}{\fieldd^2}{  \update{\fieldd^2}{  \expression^3 }{  \expression^4}}       \rbrack \\
	     \lbrack \oplus  \expression^3 \longrightarrow  \expression^4 \rbrack
	     \end{array} & else \\

               

\end{array}
\right. 
\end{array}$$ 


For example, consider the following  substitution expression:  
$$ \substitution{\fieldAccess{\locVar{1}}{ \fieldd}}{\fieldd}{\update{\fieldd}{ \locVar{2} }{  3 }} $$
This results in the new expression : 
$$\update{\fieldAccess{\locVar{1}}{ \fieldd}}{ \locVar{2}  }{ 3} $$ 


The same kind of substitution is allowed for array access expressions, where the array object \arrayAccessOnly  \ can be updated. 
