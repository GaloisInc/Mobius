\section{Extending method declarations with specification}\label{methExtend}

In the following, we propose an extension of the method formalization given in Section \ref{clazz}.
 The extension takes into account the method specification. The extended method structure is given below:

$$ \begin{array}{l} % \forall \methodd: \MethodSet, \\
                     \MethodSet  = \left\{\begin{array}{ll}  
                                                          \methodName & :\MethodName\\
						          \retType & :\JavaType\\
							  \args &  : list \ (name * \JavaType) \\
							  \numArgs & : nat \\
							  \body &  : list \  \bcIns \\
							  %\entryPoint  &  : \bcIns \\
							  \excHandlerTable & : list \ \ExcHandler  \\
							  \exceptions &  : list \ \ClassSet_{exc}\\
							  \pre & : \predWp \\
							  \modif & :   list \ x\expression \\
							  \excPostSpec & : \excType \rightharpoonup \predWp \\
							  \normalPost & : \predWp \\
                                                          \loopSpecTable & : list \ \LoopSpecSet 
							  
                                     \end{array}  \right\} 
     \end{array} $$

Let's see the meaning of the new elements in the method data structure.
\begin{itemize}
     \item \methodd.\pre \ gives the precondition of the method, i.e. the predicate that must hold
           whenever \methodd \  is called
     \item \methodd.\normalPost \ is the postcondition of the method in case \methodd terminates normally  
     
     \item \methodd.\modif \ is also called the method frame condition. It is a list of expressions that the method
            may modify during its execution    
     
     \item \methodd.\excPostSpec \ is a total function from exception types to formulas which returns the predicate
           \methodd.\excPostSpec(\texttt{Exc})  that must hold in the method's poststate 
	   if the method \methodd \ terminates on an exception of type \mbox{ \rm \texttt{Exc}}. 
	   Note that this function is constructed from the \exsures \ clause of a method introduced in  Chapter \ref{bcsl},
	   section \ref{BCSLgrammar}. For instance, if method \methodd \ has an exsures clause:
	   $$ \exsures \  ( \mbox{ \rm \texttt{Exc}}) \ \locVar{1} == \Mynull $$
	   then for every exception type $\mbox{ \rm \texttt{SExc}} $ such that \subtype{\texttt{SExc} }{\texttt{Exc}}
	    the result of the function \methodd.\excPostSpec \  for  \texttt{SExc} 
           is $\methodd.\excPostSpec(\texttt{SExc} ) = \locVar{1} == \Mynull$. If for an exception \texttt{Exc} there is not specified
	   explicitely an exsures clause then the function \excPostSpec \ returns the default exceptional postcondition predicate \Myfalse, i.e.
	   $\methodd.\excPostSpec(\texttt{Exc} ) = \Myfalse$
     \item \methodd.\loopSpecTable \ is a list of \LoopSpecSet \ data structures which give the specifcication information 
           for a particular loop in the bytecode         
\end{itemize}

 
The contents of a \LoopSpecSet \ data structure is given hereafter:
$$ \begin{array}{l}
      \LoopSpecSet = \left\{\begin{array}{ll}  
                                       \posL   & : nat \\
                                       \invL   & : \predWp \\                 
	                               \modifL   & :  list \ \expression  
                            \end{array}  \right\} 
     \end{array} $$ \todo{define modifies locations in the grammar }


For any method \methodd \ for any $ k $ such that $ 0 \leq k < \methodd.\loopSpecTable.length$ 
          \begin{itemize}
                \item the field $\methodd.\loopSpecTable[k].\posL$ is a valid index in the body of \methodd:\\
                      $ 0 \leq   \methodd.\loopSpecTable[k].\posL < \methodd.\body.length$ and is a loop entry
                      instruction in the sense of Def.\ref{defLoop}

	        \item $\methodd.\loopSpecTable[k].\invL$ is the predicate that must hold whenever the instruction 
		      $\methodd.\body[ \methodd.\loopSpecTable[k].\posL]$ 
                      is  reached in the execution of the method \methodd

                \item $\methodd.\loopSpecTable[k].\modifL$ are the locations such that for any two states
                      $state_1$, $state_2 $  in which the instruction $\methodd.\body[ \methodd.\loopSpecTable[k].\posL]$ 
		      executes agree on local variables and the heap modulo the locations that are in the list \modifL.
                      We denote the equality between  $state_1$, $ state_2 $   modulo the modifies locations like this 
                      $ state_1 =^{\modifL} state_2$
	  \end{itemize}
