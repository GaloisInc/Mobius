\section{Bytecode specification language}\label{intro:bml}


% motivation
% why does source specifications do not work any more in our context?
Let us see what advocates the need of a low level specification language.
Traditionally, specification languages were tailored for high level languages.  
Source  specification allows to express complex functional or security properties about programs.
Thus, they are successfully  used 
for software audit and validation. Still, source specification in the context of mobile code does not help a lot for several reasons.


First, the executable or interpreted code  may not be accompanied by its specified  source. Second, it is more reasonable for the 
code receiver to check the executable code than its source code, especially if he is not willing to trust the compiler.
Third, verifying a bytecode program against certain functional property needs a formalism into which the property will be encoded. 


 It is in this perspective, that we propose  a bytecode specification language which we call BML (short for Bytecode Modeling Language). 
BML is the bytecode encoding of an important subset of  JML (short for Java Modeling Language). The latter is a rich specification language tailored to Java source programs and
 allows to specify rich functional properties over Java source programs.


 

% what does the language support?
 BML supports the most important features of JML. Thus, we can express functional and security properties of Java
 bytecode programs in the form of method pre and postconditions, class and object invariants, assertions
 for particular program points like loop invariants. To our knowledge BML does not have predecessors that are tailored 
 to Java bytecode.  






