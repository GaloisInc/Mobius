\section{Related Work}\label{pog:relWork}

Several works dealing with the relation between 
source and its compilation verification condition generated.

Barthe, Rezk and Saabas in \ref{} also argue that proof obligations produced
over source code and
bytecode produced by a nonoptimizing compiler   are equivalent.
The source language supports method invokation, exception throwing and handling.They do not consider instructions that may throw
runtime exceptions. Note that because of this  However, in their work they do not discuss  what are the compiler assumptions
and properties  which will guarantee this equivalence. We claim that their proof for
 the verification condition preserving compilation holds only if their non optimizing compiler has the properties discussed here in Section
 \ref{compile}.

% \todo{because they do not expose those conditions, 
%they cannot use the induction hypothesis because they have not proven that the condition to apply the induction hold}

% Peter Mueller
 In \cite{FB04LBT}, F.Bannwart and P.Muller show how to transform a Hoare style logic derivation on source Java like
 program into a Hoare style logic derivation of a Java bytecode like program. This solution however has the shortcoming
that the certificate (in this case it is the Hoare style logic derivation) can be potentially large. In particular, this means that
applying this technique in scenarios like PCC could be hard because of the large size of the certificate. 
    


% see what are the others 
