\section{Related Work}\label{pog:relWork}


Several works dealing with the relation between the 
verification conditions over source and its compilation into a low  level  programming  language exist.

Barthe, Rezk and Saabas in \cite{gta05:fast} also argue that proof obligations produced
over source code and
bytecode produced by a nonoptimizing compiler   are equivalent.
The source language which they use supports method invokation, exception throwing and handling. 
They do not consider instructions that may throw
runtime exceptions. 
 However, in their work they do not discuss  what are the compiler assumptions
and properties  which will guarantee this equivalence. We claim that their proof for
 the verification condition preserving compilation holds only if the non optimizing 
 compiler has the properties discussed here in Section \ref{compile}.


In \cite{SU05CNS}, Saabas and Uustalu present a goto language which is provided with a compositional structure.
They give a Hoare logic rules for the language where they mention explicitely the program counter in the assertions. 
They show that if a source program has a Hoare logic derivation
 against a pre and postcondition then its compilation in the aforementioned goto language will also have a Hoare
 logic derivation in the aformentioned Hoare logic rules.\todo{to be continued. Actually they do not impose any special restrictions on the target code. Compositional rule strange}
 

% \todo{because they do not expose those conditions, 
%they cannot use the induction hypothesis because they have not proven that the condition to apply the induction hold}

% Peter Mueller
 In \cite{FB04LBT}, F.Bannwart and P.Muller show how to transform a Hoare style logic derivation on source Java like
 program into a Hoare style logic derivation of a Java bytecode like program. This solution however has the shortcoming
that the certificate (in this case it is the Hoare style logic derivation) can be potentially large. In particular, this means that
applying this technique in scenarios like PCC could be hardly applicable because of the large size of the certificate. 
    


% see what are the others 
