
\newtheorem{Expression}{Definition}[section]
\newtheorem{ExpressionRel}[Expression]{Definition}
\newtheorem{Statement}[Expression]{Definition}

\section{Source} \label{source}


We present a source Java-like programming language which supports the following features:
object manipulation and creation, method invokation, throwing and handling exceptions, subroutines etc.  
The first definition that we give hereafter presents all the constructs of our language which evaluate
to a value.

\begin{Expression}[Expression]\label{exprSrc}
    The grammar for source expressions  is defined as follows
    $$ \begin{array} {ll}    \expressionSrc ::=       
       		         	& \constantInt  \\
				& \mid \Mytrue \\ 
				& \mid \Myfalse \\
			%	& \mid \constantRef \\
				& \mid \expressionSrc \ op \ \expressionSrc \\  
				& \mid \expressionSrc.f \\
				& \mid \var \\
  			        & \mid (Class) \ \expressionSrc \\
                                & \mid \Mynull  \\
				& \mid \this \\
              	        	& \mid \expressionSrcRel \\
				& \mid \expressionSrc.m( \expressionSrc ) \\
				& \mid  \newSrc \ Class  ( \expressionSrc  ) \\ 
                                & \\
				& \\
        

     \expressionSrcRel ::=      & \expressionSrc \ \rel \ \expressionSrc \\
				& \mid \expressionSrc \ \instanceofSrc \ Class\\
				& \\   
				& \rel \in \{ \le, < ,  \ge, >, = , \neq \}                   
    \end{array} 
    $$
\end{Expression}


    
We now  a give an informal description of the meaning of the expressions of the above grammar:
\begin{itemize}
   
   \item $ \constantInt  $ is any integer literal 
   

   \item  $ \Mytrue$ and $\Myfalse$ are the unique boolean  constants 

   \item \constantRef \ is a reference to an object in the memory heap
   
   \item  $\expressionSrc \ op \ \expressionSrc$ which stands for an arithmetic expression with any of the arithmetic 
          operators $ + , - , div , rem, * $
	   
   
   
   \item $ \expressionSrc.f $ is a field access expression where the field with name $f$ is accessed

   \item the cast expression $(Class) \expressionSrc$  which is applied only to expressions from a reference type
   
   \item the expression \Mynull \ stands for the null reference which does not point to any location in the heap
  
   \item \this \ refers to the current object 
   \item $ \expressionSrc.m(\expressionSrc)  $ stands for a method invokation expression. Note that here we consider
         only methods with one argument which return a value
   \item $\newSrc \ Class(\expressionSrc )$ stands for an object creation expression of class $Class$. We consider only constructors which take
         only one argument for the sake of readability 
 

%   \item $\rel \in \{ \le, < ,  \ge, >, = , \neq \} $ 
   
\end{itemize}




The language is also provided with relational expressions, which evaluate to the boolean values: 

\begin{itemize}
   
   \item $\expressionSrc \ \rel \ \expressionSrc $ where $\rel  \in \{ \le, < ,  \ge, >, = , \neq \} $ stands for the relation between two expressions 
   
   \item  $\expressionSrc \ \instanceofSrc \ Class $ states that $\expressionSrc$ has as type the class $Class $ or one of its subclasses 
\end{itemize}






The expressions can be of object types or basic types. Formally the types are 

$$
\JavaType ::= Class, \ Class \in \ClassTypes \mid \Myint \mid \Mybool
$$ 

\todo{est-ce que je dois dire qu'on considere un sousensemble de Class qui represente les exceptions ?}

The next definition gives the control flow constructs of our language as well as the expressions that have a side 
effect  
\begin{Statement}[Statement]
    The grammar for expressions  is defined as follows :
\end{Statement}
    $$ \begin{array} {ll}    \stmt ::=  
                                & \stmt;\stmt \\
                                & \mid \Myif \ (\expressionSrcRel) \ \Mythen \ \{ \stmt \} \  \Myelse \ \{ \stmt \}  \\
			        & \mid \try  \ \{ \stmt \}  \ \catch \ ( Class  )\ \{ \stmt \} \\
		                & \mid \try  \ \{ \stmt \} \ \finally \ \{ \stmt \} \\
				& \mid \try  \ \{ \stmt \} \ \catch \ ( Class )\ \{ \stmt \} \ \finally \ \{ \stmt \} \\
				& \mid \throw \ \expressionSrc  \\
                                & \mid \while \ (\expressionSrcRel) \lbrack \invariant, \modLoop \rbrack \ \do \ \{ \stmt \} \\
				& \mid \returnSrc \  \expressionSrc \\
				& \mid \returnSrc \\
                   		& \mid \expressionSrc = \expressionSrc \\
				& \mid \expressionSrc \\
			
			 \end{array} 
    $$


% informal description 

From the definition we can see that the language supports also the following constructs :

\begin{itemize}
   \item $\stmt;\stmt$, i.e. statements that execute sequentially 
   \item $\Myif \ (\expressionSrcRel) \ \Mythen \ \{ \stmt \} \  \Myelse \ \{ \stmt \}  $ which stands for
         an if statement. The semantics of the construct is the standard one, i.e. if the relation expression $\expressionSrcRel$ 
	 evaluates to true then the statement in the $ \Mythen$ branch is executed, otherwise the statement in the
	 $ \Myelse$ branch is executed
   \item  \todo{give the complete explanation}
   \item 
   \item 
   \item 
   \item 
  
 
\end{itemize}
