\subsubsection{Expressions}\label{pog:wpSrc:wpExpr}
In the following, we shall give the definition of a weakest
predicate transformer function for expressions. 


As we shall see,  for some of the expressions   
the definition of the weakest precondition function is trivial (i.e. it is the identity function )
 as they do have  no side effects.
 However, this is not the case for expressions that might throw an exception or which may change values of program variables.
 The predicate returned by the weakest precondition predicate transformer for expressions which may throw an exception 
will basically acumulate the hypothesis under which the evaluation 
of the expression terminates normally and the  conditions under which it  terminates exceptionally.


The weakest precondition function for expressions has the following signature:
$$ \wpNameSrcExpr : \expressionSrc \rightarrow \formulaSrc \rightarrow ( \mbox{ \rm \texttt{Exc}} \rightarrow  \formulaSrc ) \rightarrow  
\MethodSet \rightarrow   (\expressionSpecSrc \cup \formulaSrc)  \rightarrow  \formulaSrc $$

For calculating the  $\wpNameSrcExpr$  \ predicate  of  an expression $\expressionSrc$ declared in method \methodd,
 the function $\wpNameSrcExpr$ \ takes as arguments  $\expressionSrc$, a postcondition $\normalPostSrc$, an exceptional postcondition 
function $\excPostSrc$  and  an $\expressionSpecSrc$ and returns the    formula 
$\wpSrcExpr{\expressionSrc}{\psi}{\excPostSrc}{ v }$ which is the \wpName \ precondition of $\expressionSrc$ 
if the its evaluation is represented by $v$.

In the following, we give the rules of $\wpNameSrcExpr$.
We provide comments at the places which we do not consider trivial.


\begin{itemize}

      \item integer and boolean constant access \\
	  ( $const \in \{  \constantInt , \Mytrue , \Myfalse, \constantRef  \}$ )
	$$ \wpSrcExpr{const}{\normalPostSrc }{ \excPostSrc }{const} = \normalPostSrc $$
	Constant expressions do not change the state and thus, if a predicate $ \normalPostSrc$ hold after its execution
	this means that it held in the prestate of the expession.
	
    \item field access expression
	         $$ \begin{array}{l} 
             \wpSrcExpr{\expressionSrc.f  }{\normalPostSrc }{ \excPostSrc }{v.f} = \\
	                        \wpSrcExpr{\expressionSrc }{\\
			                   \phantom{wpiSr} \begin{array}{l} 
						        v \neq \Mynull \Rightarrow \normalPostSrc\\
			                                \wedge \\
						        v = \Mynull \Rightarrow \excPostSrc(\NullPointerExc ) 
		                                   \end{array}}{\\ \phantom{wpiSrc} \excPostSrc }{v} 
                     		                    
	    \end{array} $$
	    The rule for the field access expression takes into account the two possible  outcomings of its evaluation. If the evaluation $v$ 
	    of the reference $\expressionSrc$ is different from $\Mynull$ then the evaluation terminates normally, otherwise the exceptional postcondition 
	    for \NullPointerExc \ must hold. 

	 \item arithmetic expressions 
             $$ \begin{array}{l}
	              \wpSrcExpr{ \expressionSrc_1 \ op \ \expressionSrc_2 }{\normalPostSrc }{ \excPostSrc }{v_1 + v_2} =  \\
		      \wpSrcExpr{ \expressionSrc_1  }{\wpSrcExpr{\expressionSrc_2 }{ \normalPostSrc }{ \excPostSrc }{v_2}  }{ \excPostSrc }{v_1}
		\end{array}$$

	\item method invokation

  $$ \begin{array}{l} \wpSrcExpr{ \expressionSrc.m()  }{\normalPostSrc }{ \excPostSrc }{v} = \\
	     \wpSrcExpr{\expressionSrc}{ \\
                          
                            \phantom{ wpisrc \expressionSrc_1 }  
			     \left\{ \begin{array}{l}
			      v'  \neq \mbox{ \rm \Mynull }  \Rightarrow     \\
			      
                                         \Myspace    \Myspace    m.\preSrc  %\begin{array}{l} 
					                         \subst{\this}{ v'} \\
                                                                 %\subst{\mbox{\rm arg}}{ v_2} \\
					                  %\end{array}  \\
                                          \Myspace   \Myspace \wedge \\
					  \Myspace   \Myspace \forall \freshVar, \ \forall \ m \in m. \mod \\ 
					  
                                          \Myspace   \Myspace  \left\{\begin{array}{l}  
					  \typeof{\freshVar} <: m.\returnType \wedge \\ 
					  m.\normalPostSrc
					                                \begin{array}{l} 
					                                    \subst{\result}{\freshVar } \\
									    \subst{ \this}{ v' } \\
									   % \subst{\mbox{\rm arg} }{ \expressionSrc_2}
					                                 \end{array}   \\
                                             \Myspace \Rightarrow \normalPostSrc  \subst{ v }{ \freshVar }  \\ 
					                       \end{array}\right.\\
					     \Myspace   \Myspace \wedge \\
					     \Myspace   \Myspace	 \forall \mbox{\rm \texttt{E}} \in m.\exceptionSrc, \\
					     \Myspace   \Myspace	 \forall \ m \in m. \mod\\
					     \Myspace   \Myspace  
					  \begin{array}{l}
					         m.\excPostSpecSrc( \mbox{\rm \texttt{E}} ) \Rightarrow \excPostSrc( \mbox{\rm \texttt{E}} )
					  \end{array} \\ 
			                \expressionSrc_1 = \mbox{ \rm \Mynull }  \Rightarrow    \excPostSrc( \NullPointerExc )
                           \end{array}\right. 
       } { \\  \phantom{ wpisrc \expressionSrc_1 } \excPostSrc  } 
	       {v'  }
 
            \end{array} $$	 



	  \item Cast expression \todo{may be give an example}
	       $$ \begin{array}{l} \wpSrcExpr{ ( \class ) \ \expressionSrc  }{\normalPostSrc }{ \excPostSrc }{v} = \\
	           \wpSrcExpr{ \expressionSrc } { \\ 
                      
                    \phantom{wpSr} \begin{array}{l}
                    \typeof{v } <:\class  \Rightarrow    \\
		      \phantom{wpSr} \phantom{wpSr}   \normalPostSrc  \\
		     \wedge \\
		    \neg \   \typeof{v } <:  \class \Rightarrow  \\ 
                     \phantom{wpSr} \phantom{wpSr}  \excPostSrc( \mbox{ \rm \ClassCastExc }  )
                    \end{array}  } { \\ \phantom{wpSrc}  \excPostSrc }{v}
	            \end{array} 
	          $$
				
                               
	   
            
	    \item Null expression

		    $$ \wpSrcExpr{\Mynull  }{\normalPostSrc }{ \excPostSrc }{\Mynull} = \normalPostSrc $$
           
	    \item this 

		    $$ \wpSrcExpr{\this  }{\normalPostSrc }{ \excPostSrc }{\this} = \normalPostSrc $$
           	       
	 \item instance creation

%\todo{may the precondition of a class constructor talk about this? I would say no } 

  $$ \begin{array}{l} 
    \wpSrcExpr{ \newSrc \ \class  ( \expressionSrc  ) } {\normalPostSrc }{ \excPostSrc  }{v} = \\
 	  
 	             \wpSrcExpr{\expressionSrc} { \\ \\
 		                \phantom{wpSr}\left\{ \begin{array}{l}
                                    \forall \freshVar, \\
 				   \Myspace  not \ \instances(\freshVar ) \wedge \\
 					       \Myspace  \freshVar  \neq \Mynull \Rightarrow \\ 
 				   \Myspace  \Myspace  \Constructor{\class}.\preSrc
 				                \begin{array}{l} 
                                                       \subst{ \this}{ \Ref{Class} } \\
                                                       \subst{ arg}{ v' } \\
  					     	\end{array} \\
  					\Myspace  \Myspace	       \wedge \\
  					\Myspace  \Myspace  \forall \ m \in \Constructor{Class}. \mod , \\
  					       \Myspace  \Myspace  \Myspace    \Constructor{Class}.\normalPostSrc       
  					             \begin{array}{l}   
  						          \subst{ \this}{ \freshVar  } \\
  						          \subst{ arg}{ v' } \\
  						       	  \subst{	\typeof{\freshVar}}{ Class} 
  						     \end{array}  \\
                                                   \Myspace  \Myspace   \Myspace      \Rightarrow   \normalPostSrc \subst{ v }{ \freshVar  } \\ 
  					          \Myspace  \Myspace \wedge \\
  						   \Myspace  \Myspace  \forall \mbox{\rm \texttt{Exc}} \in  \Constructor{Class}.\exceptionSrc, \\
  						    \Myspace \Myspace  \forall \ m \in \Constructor{Class}. \mod, \\
  						 \Myspace  \Myspace  \Myspace \Constructor{Class}.\excPostSpecSrc ( \mbox{\rm\texttt{Exc}}) \Rightarrow \excPostSrc( \mbox{\rm\texttt{Exc}})
  		              \end{array}\right. } {  \\  \\ \phantom{wpSrc}   \excPostSrc }{v'}  
  		     
              \end{array}  $$ 
  
  \end{itemize}


Let us see the relational expressions supported in the source programming language

\begin{itemize} 
     \item Instanceof expression
	        $$ \begin{array}{l} \wpSrcExpr{ \expressionSrc \ instanceof \ \class }{\normalPostSrc }{ \excPostSrc }{ \typeof{ v } <:  \class \wedge v \neq \Mynull} = \\
	                \wpSrcExpr{ \expressionSrc  }{ \normalPostSrc }{ \excPostSrc }{v}
	            \end{array} $$

     \item Binary relation over expressions
            $$ \begin{array}{l} \wpSrcExpr{ \expressionSrc_1 \ \rel \ \expressionSrc_2 }{\normalPostSrc }{ \excPostSrc }{v_1 \rel v_2} = \\
	                \wpSrcExpr{ \expressionSrc_1  }{  \wpSrcExpr{ \expressionSrc_2  }{ \normalPostSrc }{ \excPostSrc }{v_2}  }{ \excPostSrc }{v_1}
	            \end{array} $$
\end{itemize}
