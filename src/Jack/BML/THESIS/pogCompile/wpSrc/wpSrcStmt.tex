
\subsubsection{Statements}\label{pog:wpSrc:wpStmt}
In the following, we present the rules of the weakest precondition predicate transformer for 
control statements. In Fig. \ref{pog:wpSrc:wpStmt:withoutExc} we give the rules for control 
statements which do not deal with exceptions. They are defined in a standard way. 
For instance, the rule \textsf{seq} states that the weakest predicate for a sequence of statements  $\stmt_1;\stmt_2$
w.r.t. a normal postcondition $\normalPostSrc$ and exceptional postcondition function $ \excPostSrc$
is the weakest  predicate of $\stmt_1$ w.r.t. to the normal postcondition $ \wpSrcStmt{\stmt_2 }{\normalPostSrc }{ \excPostSrc} $
and the same exceptional postcondition function $ \excPostSrc$.
Let us look also at the rule \textsf{while}. The rule requires that the invariant 
holds. It also requires that if the invariant holds and the conditional expression evaluates to true then the
weakest precondition of the loop body w.r.t. a postcondition which is the loop invariant. Note that this is quantified over
the expressions which are in the modifies list \modLoop{} of the loop. This allows to properly initialize those variables which are not
modified in the loop. Actually, we use the same technique here as in Chapter \ref{wpGeneral}.

The control statements related to the exception handling and throwing as well as the finally statements
may be have a more particular definition. They are given in Fig. \ref{pog:wpSrc:wpStmt:withExc}. 
Let us look at the rule \textsf{try catch}. Actually, it is similar to the  rule  \textsf{seq} from 
Fig. \ref{pog:wpSrc:wpStmt:withoutExc}, but dual in the way the postcondition modifications. In particular, 
the weakest predicate of a try catch statement $ \try \ \{ \stmt_1 \} \ $ 
$\catch (  \mbox{\rm\texttt{Exc}} \ c )\ \{ \stmt_2 \} $  w.r.t. a normal postcondition   $\normalPostSrc$  and exceptional postcondition function $ \excPostSrc$
is the weakest predicate of the try statement $\stmt_1$ w.r.t. the normal postcondition  $\normalPostSrc$  and the updated exceptional function
$\update{\excPostSrc}{ \mbox{\rm\texttt{Exc}} }{\wpSrcStmt{\stmt_2}{ \normalPostSrc}{\excPostSrc} }$.

 
\begin{figure}[ht!]
\begin{frameit}
$${\scriptsize 
        \begin{array}{l} 
	   \wpSrcStmt{ \stmt_1;\stmt_2}{\normalPostSrc }{ \excPostSrc } =^{\mbox{\rm\textsf{seq}}} \\
	   \begin{array}{l} 
             \wpSrcStmt{\stmt_1 }{\wpSrcStmt{\stmt_2 }{\normalPostSrc }{ \excPostSrc}} { \excPostSrc} 
           \end{array} \\ \\ \\


	   \wpSrcStmt{ \expressionSrc_1 = \expressionSrc_2}{\normalPostSrc }{ \excPostSrc } =^{\mbox{\rm\textsf{locVarAssign}}} \\
                  \begin{array}{l}   \wpSrcExpr{\expressionSrc_2 }{ 
				    \normalPostSrc \subst{\expressionSrc_1}{v}   
				   }{ \excPostSrc}{v}  
		  \end{array}
	           \\ \\ \\

    \wpSrcStmt{ \expressionSrc_1.\fieldd = \expressionSrc_2}{\normalPostSrc }{ \excPostSrc } = ^{\mbox{\rm\textsf{fieldAssign}}}\\
               \begin{array}{l} 
	       \wpSrcStmt{\expressionSrc_1}{  \\
	        \phantom{wpiSr} \wpSrcExpr{\expressionSrc_2 }{ \\
		\phantom{wpiSr}\phantom{wpiSr}
		\begin{array}{l}
		    v_1 \neq \Mynull  \Rightarrow \\
	         \phantom{wpiSr}  \normalPostSrc \subst{\fieldd}{\update{\fieldd}{ v_1}{  v_2}}   \\
		 \wedge \\
		 v_1  = \Mynull  \Rightarrow  \\
		   \phantom{wpiSr}   \excPostSrc(\texttt{NullPointerExc}) 
		   \end{array}}{ \\ \phantom{wpiSr}\phantom{wpiSr}  \excPostSrc}{v_2} 
		   }{ \\ \phantom{wpiSr}  \excPostSrc}{v_1} 
		   \end{array} \\ \\ \\

    \wpSrcStmt{ \begin{array}{l} \Myif \  (\expressionSrcRel)\\  \Mythen  \{ \stmt_1 \} \\  \Myelse \ \{ \stmt_2 \} \end{array}}{\normalPostSrc}{ \excPostSrc } 
    =^{\mbox{\rm\textsf{if}}}\\

         \begin{array}{l} 
	 \wpSrcExpr{ \expressionSrcRel }{ \\
	              \phantom{wpiSr}  \begin{array}{l}  
		      v = \Mytrue  \Rightarrow \wpSrcStmt{\stmt_1 }{\normalPostSrc }{ \excPostSrc } \\
		      \wedge \\
		      v = \Myfalse    \Rightarrow \wpSrcStmt{\stmt_2 }{\normalPostSrc }{ \excPostSrc } \\
	 \end{array}
	 } {\\  \phantom{wpiSr}  \excPostSrc }{v} 
     \end{array} \\ \\ \\

      \wpSrcStmt{ \while \ (\expressionSrcRel) \ \lbrack \invariant, \modLoop \rbrack \  \do \ \{ \stmt \}}{ \normalPostSrc}{\excPostSrc}=^{\mbox{\rm\textsf{while}}} \\
	      \begin{array}{l} 
	       \invariant \ \wedge\\
	       \forall \  m, m \in \modLoop , \\
	       \Myspace \invariant \Rightarrow \\
	 	     \Myspace    \Myspace \wpSrcExpr{\expressionSrcRel}{\\\phantom{wp^{src}} 
		     \begin{array}{l}  		
		           \Myspace v = \Mytrue  \Rightarrow \  \wpSrcStmt{ \stmt }{\invariant} {\excPostSrc}  \\ 
		           \Myspace v = \Myfalse  \Rightarrow  \normalPostSrc
		     \end{array}
	      }{ \\ \Myspace \phantom{wp^{src}} \excPostSrc}{v}  \end{array} \\ \\ \\
\wpSrcStmt{ \returnSrc \ \expressionSrc }{ \normalPostSrc}{\excPostSrc} = \\
             \begin{array}{l}   \wpSrcExpr{ \expressionSrc}{ \normalPostSrc \subst{\result }{ v }} { \excPostSrc}{v}  \end{array}

\end{array} } $$

\caption{\sc WP for source control statements without exceptions }
\label{pog:wpSrc:wpStmt:withoutExc}
\end{frameit}
\end{figure}




\begin{figure}[ht!]
\begin{frameit}
$${\scriptsize 
        \begin{array}{l} 
 \wpSrcStmt{ \throw \ \expressionSrc }{ \normalPostSrc}{\excPostSrc} =^{\mbox{\rm\textsf{throw}}}\\
	       \begin{array}{l} 
	      \wpSrcExpr{\expressionSrc}{\\ \phantom{wpiSr} 
                  \begin{array}{l}
		       
	                v = \Mynull \Rightarrow \excPostSrc( \texttt{NullPointerExc} )  \\
			\wedge \\
			 v \neq \Mynull \Rightarrow\\
			 \Myspace
			  \begin{array}{l}  
				 \forall \mbox{\rm\texttt{Exc} } , \\
				 \typeof{ v } <: \mbox{\rm\texttt{Exc} }   \Rightarrow \\
				 \Myspace   \methodd.\excPostSrc( \mbox{\rm\texttt{Exc} }    ) \subst{\EXC}{v}
			  \end{array} 

			  
	          \end{array}}   
		     { \\ \phantom{wpiSr}  \excPostSrc}{v} 
	   \end{array}     \\ \\ \\

 \wpSrcStmt{ \try \ \{ \stmt_1 \} \ \catch (  \mbox{\rm\texttt{Exc}} \ c )\ \{ \stmt_2 \} } { \normalPostSrc}{\excPostSrc} =^{\mbox{\rm\textsf{try catch}}}\\
              \begin{array}{l}
	      \wpSrcStmt{ \stmt_1}{ \\ 
	                 \phantom{wpiSr} \normalPostSrc}{ \\ 
			 \phantom{wpiSr} \update{\excPostSrc}{ \mbox{\rm\texttt{Exc}} }{\wpSrcStmt{\stmt_2}{ \normalPostSrc}{\excPostSrc} }} 
	      \end{array} \\ \\ \\

\wpSrcStmt{ \try \ \{ \stmt_1 \} \ \finally \ \{ \stmt_2 \} } { \normalPostSrc}{\excPostSrc} =^{\mbox{\rm\textsf{try finally}}}\\
       \begin{array}{l} 
	 \wpSrcStmt{ \stmt_1}{\\ \phantom{wpiSr}
                     \wpSrcStmt{\stmt_2}{ \normalPostSrc} { \excPostSrc}}{ \\ \phantom{wpiSr}
		     \update{\excPostSrc}{\mbox{\rm\texttt{Exception}}}{ \wpSrcStmt{\stmt_2}{ \excPostSrc( \mbox{\rm\texttt{Exception}}  )}{\excPostSrc}  } }
       \end{array} 
       
\end{array} } $$

\caption{\sc WP for the control statements with exceptions }
\label{pog:wpSrc:wpStmt:withExc}
\end{frameit}
\end{figure}





