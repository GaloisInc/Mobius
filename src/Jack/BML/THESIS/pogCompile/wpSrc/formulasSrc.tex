%\newcommand{\predicate}[1]{\psi_{#1} }
%\newcommand{\predicates}{ \mathbb{P} }
%\newtheorem{Formula}{Formulas}
%\newtheorem{Predicate}{Predicates}

\subsection{Source assertion language} \label{formulasSrc}
The properties that our predicate calculus treats are from first order predicate logic.
In Fig. \ref{formulasSrc:grammar}, we give the formal definition of the assertion language into which 
the properties are encoded. Note that the language is almost the same as the bytecode assertion 
language presented earlier (see Chapter \ref{assertLang} ). However, the assertion language for source programs
 does not support  stack expressions ($\stack{\counter + - \ldots}$ and \counter) which make sense only for bytecode.  


 
\begin{figure}[ht!]
%\begin{Formula}[Definition]
\begin{frameit}
	$$
	\begin{array}{ll}
	\formulaSrc ::= & \expressionSpecSrc \predicates  \expressionSpecSrc \\
	  & \vert  \instances(\expressionSpecSrc)\\
	  & \vert \true \\
	  & \vert  \false \\	
	  & \vert \formulaSrc \wedge \formulaSrc \\
	  & \vert \formulaSrc \vee  \formulaSrc \\
	  & \vert \formulaSrc \Rightarrow \formulaSrc \\
	  & \vert \forall x  (  \formulaSrc(x) ) \\
	  & \vert \exists x  ( \formulaSrc (x))	 \\
      
          & \\
	  & \\
	  \predicates ::=  &   == \mid \neq \mid \leq \mid \le \mid \geq \mid > \mid  <: \\
	  & \\
	  & \\
	  \expressionSpecSrc  ::= & \constantInt  \\
	                          & \mid \Mytrue \\ 
	                          & \mid \Myfalse \\
	                          & \mid \freshVar \\
				  & \mid \EXC    \\
				  & \mid \expressionSpecSrc \ op \ \expressionSpecSrc \\  
				  & \mid \expressionSpecSrc.f \\
				  & \mid \var \\
  			          & \mid \Mynull  \\
				  & \mid \this \\
	                          & \mid \typeof{\expressionSpecSrc} \\
	                          & \mid \result
                                  
	\end{array}
	$$
\caption{\sc Source assertion grammar}
\label{formulasSrc:grammar}
\end{frameit}
\end{figure}
%\end{Formula}

% Note that the expressions in the assertion language are very similar to the expression  in the programming language presented in subsection \ref{source}.

%We define a function which maps expressions from the programming language into the expressions of the assertion language which 
%is denoted and is typed as follows: 

%$$\ulcorner . \urcorner^{\tiny{src2spec}} : \expressionSrc \rightarrow \expressionSpecSrc $$
%The function is defined as follows:

%$$ \begin{array}{lll}
%      \compileSrcSpec{\constantInt} & = &  \constantInt\\
%      \compileSrcSpec{\Mytrue} & = &  \Mytrue \\
%      \compileSrcSpec{\Myfalse} & = &  \Myfalse \\
%      \compileSrcSpec{\expressionSrc \ op \ \expressionSrc  } & = & \compileSrcSpec{\expressionSrc} \ op \ \compileSrcSpec{\expressionSrc  }  \\
%      \compileSrcSpec{(Class)\expressionSrc } & = & \compileSrcSpec{\expressionSrc } \\
%      \compileSrcSpec{\expressionSrc.m(\expressionSrc)} & = & \freshVar \\
%      \compileSrcSpec{\expressionSrc.f }  & = & \compileSrcSpec{\expressionSrc }.f \\
  %    \compileSrcSpec{\this}  & = & \this \\
%      \compileSrcSpec{  \newSrc \ Class  ( \expressionSrc  ) } & = &  \freshVar \\
%%      \compileSrcSpec{ \expressionSrc \ \instanceofSrc \ Class  } & = &  \typeof{\compileSrcSpec{ \expressionSrc} } <:  Class \wedge \compileSrcSpec{ \expressionSrc} \neq \Mynull  \\
%      \compileSrcSpec{ \expressionSrc \ \rel \ \expressionSrc  } & = & \compileSrcSpec{  \expressionSrc } \rel \compileSrcSpec{  \expressionSrc } \\
%      \end{array}
%  $$
