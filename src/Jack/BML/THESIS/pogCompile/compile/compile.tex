

\newtheorem{Compiler}{Definition}

\section{Compiler} \label{compile}

We now turn to specify a simple compiler from the source language presented in Section
\ref{source} into the bytecode language. The compiler does not perform any optimizations. 



% \subsection{Compilation of source values}
%In our modelisation, the source language presented in Section~\ref{source} keeps close to the Java 
%language semantics and the bytecode language defined in \ref{P05BSV} is close to the 
%Java Virtual Machine \cite{VMSpec}. This implies that our source language supports booleans while the bytecode
%language does not support them. In particular, values of type boolean on source level
% are represented as values of type integer. That's we formalise this fact by defining a compiler from source values 
%to bytecode values :
%
%$$
%\compile{\eval{\expressionSrc} } = 
%\left\{ \begin{array}{ll}
%1 &  if \expressionSrc == \Mytrue \\
%& \\
%0 &  if \expressionSrc == \Myfalse \\
%& \\
%\eval{\expressionSrc}  & else
%\end{array} \right.
%$$

\todo{the exception handler function}

