%\newtheorem{compProp0}{Property}[subsection]
\newtheorem{compProp1}{Property}[subsection] % control flow statement property 
\newtheorem{compProp2}[compProp1]{Property} % control flow statement property  
\newtheorem{compProp3}[compProp1]{Property} % expression property
\newtheorem{seqInstr}{Definition}[subsection]

 
\newcommand{\isStartExc}[1]{\mbox{\rm\textit{isExcHandlerStart}}(#1)}
\subsection{Properties of the compiler function}

In this last subsection, we will look at the properties of the
 bytecode  programs produced by the compiler.





\begin{compProp1}[Compilation of statements]\label{compProp1}
    For any expression $\stmt$, start label $s$ and end label $e$,
    the compiler will produce a sequence of bytecode instruction $\compileLabel{s}{\stmt}{e}$ such that: 
           $$ \begin{array}{l} 
                             \forall i, ( \isInList{\compileLabel{s}{\stmt}{e}}{\ins{i}} ) \wedge \\
			     \Myspace ( \ins{i} \execRel \ins{k}) \wedge \\
			     \Myspace \neg (  \isInList{\compileLabel{s}{\stmt}{e} }{\ins{k}} ) \\
			     \Myspace \neg \isStartExc{\ins{k}}     \Rightarrow   \\
                             \Myspace \Myspace k = e + 1
              \end{array}$$ 
\end{compProp1}
The following property states that  if there are instructions inside the compiled statetement 
 $\compileLabel{s}{\stmt}{e}$ which are in execution relation\footnote{see Def. \ref{defEdge}} ( but not with the start of an exception handler) with an instruction 
$\ins{k}$ outside it then the targetted instruction is  $\ins{e +1}$. 
 The conditions  $ \neg (  \isInList{\compileLabel{s}{\stmt}{e} }{\ins{k}} )$ and 
$\neg \isStartExc{\ins{k}}$  eliminate the case when the execution relation is between an instruction inside 
 $\compileLabel{s}{\stmt}{e}$ which may throw an exception and the start instruction of the proper handler exception handler.

 The proof is done by induction on the structure of the compiled statement. 

\textit{Proof:} 
$$
\begin{array}{l}
            \mbox{\rm\textit{We scatch the proof for the compilation of the if statement. }}\\
            \\
            \mbox{\rm\comment{by definition of the compiler function for if statements in section \ref{compCtrlStmt} }} \\
            \compileLabel{s}{ \Myif \ ( \expressionSrcRel ) \ \Mythen \ \{ \stmt_1 \} \  \Myelse \ \{ \stmt_2 \}   }{e} =  \\
        
	    \compileLabel{s}{\expressionSrcRel}{e'}; \\
            e' + 1: \ifCond \ e'' + 2;\\
            \compileLabel{e' + 2}{\stmt_2}{e''}\\
             e'' + 1: \goto \ e +  1; \\
            \compileLabel{e'' +  2}{\stmt_1}{ e  }; \\
	    %e: \goto   e + 1  \\
            \\
	    \mbox{\rm\comment{ induction hypothesis for $\expressionSrcRel$ and  $\stmt_1$ and $\stmt_2$ }} \\

	    \numConclusion{1} \  \forall i ,  s \ge i \le e' ,  \ \ins{i}   \execRel \ins{k}) \wedge \\
			     \Myspace \neg (  \isInList{\compileLabel{s}{\stmt}{e} }{\ins{k}} ) \\
			     \Myspace \neg \isStartExc{\ins{k}}   \Rightarrow   \\
                             \Myspace \Myspace k = e' + 1 \\ \\
    
	    \numConclusion{2} \  \forall i ,  e' + 2 \ge i \le e'' ,  \ \ins{i}   \execRel \ins{k}) \wedge \\
			     \Myspace \neg (  \isInList{\compileLabel{s}{\stmt}{e} }{\ins{k}} ) \\
			     \Myspace \neg \isStartExc{\ins{k}}   \Rightarrow   \\
                             \Myspace \Myspace k = e'' + 1\\\\
	
	    \numConclusion{3} \  \forall i ,  e'' + 2 \ge i \le e ,  \ \ins{i}   \execRel \ins{k}) \wedge \\
			     \Myspace \neg (  \isInList{\compileLabel{s}{\stmt}{e} }{\ins{k}} ) \\
			     \Myspace \neg \isStartExc{\ins{k}}   \Rightarrow   \\
                             \Myspace \Myspace k = e + 1	     \\\\
       
     \numConclusion{4} \  \mbox{\rm\textit{\{ from \numConclusion{1},\numConclusion{2}   and  \numConclusion{3} we get that  }} \\  
       \mbox{\rm\textit{ jumps from  $\expressionSrcRel$ and  $\stmt_1$ go inside the compilation of  }} \\
       \mbox{\rm\textit{ $ \Myif \ ( \expressionSrcRel ) \ \Mythen \ \{ \stmt_1 \} \  \Myelse \ \{ \stmt_2 \}  $  }} \\
       \mbox{\rm\textit{ as $ e' + 1$ and  $e'' + 1$ are labels in the compilation of the statement  and  }}	\\
       \mbox{\rm\textit{ and that jumps from $\stmt_2$ go to $e + 1$   \} }}\\

        \\ 
        \numConclusion{5} \  \mbox{\rm\textit{\{the instruction $   e' + 1: \ifCond \ e'' + 2; $ may jump}} \\
                               \mbox{\rm\textit{ to $e'' + 2$ which is inside the compilation of the if statement \}  }} \\\\
	  \numConclusion{6} \ \mbox{\rm\comment{the instruction $   e'' + 1: \goto \ e +  1; $ may jump to $e + 1$ }} \\\\
	 \mbox{from \numConclusion{4},  \numConclusion{5} and \numConclusion{6} the lemma holds in that case   }
\end{array}
$$

Before proceeding with the properties of the bytecode resulting from the expression compilation, we introduce the notion of block
of bytecode instructions.  


\begin{seqInstr}[Block of instructions] \label{seqInstr}
If the list of  instructions $l  = \lbrack \ins{i_1} \ldots \ins{i_k} \rbrack$ is such that 
 \begin{itemize}
      \item none of the instructions in the set is a jump instruction, i.e. $\forall m, m = 1..k \Rightarrow \neg( \ins{i_m} \in  \{\goto, \ifCond\} )  $ 
           
      \item   none of the instructions is a target of an instruction $\ins{i_j}$ which does not belong to $l$

 \end{itemize}

We denote such a list of instruction with $\ins{i_1};...;\ins{i_k}$

\end{seqInstr}


The next lemma establishes that the compilation of an expression $\expressionSrc$ results in
 a block of bytecode instructions.  
\begin{compProp3}[Compilation of expressions]\label{compProp3}
    For any expression $\expressionSrc$, starting label $s$ and end label $e$,
    the compiler will produce a block of bytecode instruction $\compileLabel{s}{\expressionSrc}{e}$ such that
    $\ins{s};...;\ins{e}$           
\end{compProp3}
The property states that there are no jump instructions  in the list of instructions representing the  compilation of an expression. 
The proof is done by induction over the structure of the expression.


\todo{give the definition of the function  $\isStartExc{\ins{k}}$   }
\todo{this is not sufficient. There must be established a property that the $\execRel^l$ is only in the compilation of while statements} 
 In the following, we  abstract
 from the labeling done by the compiler and we  will denote the compiler function with $ \lbrack \ \rbrack $.
 We assume that the compiler function $ \lbrack \ \rbrack $ does a labeling which has the above properties.
 Also, the notation for labeled instructions used here $ \ins{i} $ will be sometimes abbreviated just to the index of the instruction $ i $. 


%\begin{compProp2}[Compilation of statements ] \label{compProp2}
%  For any statement $\stmt$, the compiler will produce a sequence of bytecode instruction 
%  $\compileLabel{s}{\stmt}{e}$ such that
%  there may be no more than one instruction $ \ins{j} \in \compileLabel{s}{\stmt}{e}$, such that 
%  $ \ins{k} \execRel \ins{j} \wedge \neg ( \ins{k}  \ \in \  \compileLabel{s}{\stmt}{e} ) $
%\end{compProp2}
%The property states that 
%no more than one instruction in the compilation $\compileLabel{s}{\stmt}{e}$
%of the statement $\stmt$ may be targeted by an instruction 
%which does not belong to  $\compileLabel{s}{\stmt}{e}$.







%Another property of the compiler is that compilation of expressions result always 
%in a block of sequential instructions as defined in the above def. \ref{seqInstr}

%\begin{compProp3}[Compilation of expressions] \label{compProp3}
%  For any expression $\expressionSrc$,
%  starting label $s$ and end label $e$, the compiler
%  will produce a sequence of bytecode instructions
%  $ \compileLabel{s}{\expressionSrc}{e} =  \ins{s} ;...; \ins{e} $
%\end{compProp3}


%The property can be established by structural induction of the compilation $\compile{\expression}$

 
