

%\newtheorem{excProp}[interProp]{Lemme}
\newtheorem{pogEquiv}{Theorem}
\newtheorem{postProp}{Lemme} % states that the implicite postconditions for any compile(stmt ) does not contain stack expressions


% For keeping font inside a mathematical environment : \mbox{\rm \texttt{Exc}  }
\section{Compiling Proof Obligations} \label{proofPogEq}

We turn now to study the relationship between the proof obligations on source and bytecode level. 
We show that syntactically the proof obligations are the same modulo names and some types.





\subsection{ Proof obligation equivalence}



% We also need to establish that the exceptional postcondition function for any method \method \  
%  $\excPostSrc$ for every source expression(statement) $\texttt{exp}$ and exception \texttt{Exc} returns
%   the same formula (modulo names and types) as the exceptional postcondition function for the 
%  $\compile{\texttt{exp}}$ and the exception \texttt{Exc}. 

%  \begin{excProp} \label{excProp}
%  Let's have a source expression(statement) exp and let's the expression(statement) may terminate abruptly on an exception \mbox{\rm \texttt{Exc}}. 
%  The exceptional postcondition for the expression exp  has the following property:
%  $$
 % \excPostSrc(\mbox{\rm \texttt{Exc}},exp ) =^{mod \ names} \excPost(\mbox{\rm \texttt{Exc}},\compile{ exp})
%  $$
%  \end{excProp}
%
%  Proof. 
%  By definition\footnote{we take the semantics of exception handling as described in the Java Language Specification} it is the inner most exception handler that encloses\texttt{exp} that can handle \texttt{Exc} that will manage the thrown exception \texttt{Exc} from \texttt{exp}.
%  So, we take the inner most try catch statement such that :

%  $ \try \{ \ \ldots  \texttt{exp} \ldots \} \ \catch ( \texttt{Exc} \ )\ \{ \texttt{handler} \}$


%  \comment{by definition}

%  $$ \numConclusion{1} $$
%  $$
 % \begin{array}{l}
%  \wpiSrc{\try\{  \ \ldots \texttt{exp} \ldots \} \ \catch (\texttt{Exc} \ )\ \{ \texttt{handler} \}}{\psi}{\excPostSrc}\\
 %  =\\
%  \wpiSrc{\ldots \texttt{exp} \ldots }{\psi}
 %  {\excPostSrc \oplus [ \texttt{Exc}  \longleftarrow \wpiSrc{\texttt{handler} }{ \psi}{ \excPostSrc} ]}
%  \end{array}
%  $$ 
%
%  \comment{by definition}
 % $$\numConclusion{2} $$
%  $$
 % \excPostSrc \oplus [ \texttt{Exc}  \longleftarrow \wpiSrc{\texttt{handler} }{ \psi}{ \excPostSrc} ](\texttt{Exc}) = \wpiSrc{\texttt{handler} }{ \psi}{ \excPostSrc}
%  $$

  
%  \comment{by definition of the compiler function }
%  $$
%  \numConclusion{3}
 % $$
%  $$
 % \begin{array}{l}
%    \compile{\try \{ \ \ldots \texttt{exp} \ldots \} \ \catch ( \texttt{Exc} \ )\ \{ \texttt{handler} \} } \\
 %     = \\
%        \compile{\ldots \texttt{exp} \ldots  } ;\\
%      goto \ l; \\
   %  \compile{handler}; \\
%     goto \ l;\\
%     ... \\
%     l: \\ 
%  	\addExceptionTable{ \stR{\compile{\stmt_1} }}{ \enR{  \compile{ \stmt_1 }  }}{ \compile{\stmt_2} }{ Class} )
%    \end{array}
%  $$
%
 % \comment{by definition the exceptional postcondition for \mbox{ \rm  $ \compile{\texttt{exp}}$ } for exception \mbox{ \rm \texttt{Exc}} is}
%
%  $$
%  \numConclusion{4}
%  $$
%  $$
 % \begin{array}{l}
 %  \excPost(ind, \texttt{Exc}) = \wpi{\compile{handler}; goto \ l }{ \compile{\psi}}{\excPost} \\
 % where \  ind \in \compile{ \texttt{exp}}
 % \end{array}
%  $$
%  \comment{from lemme  and \numConclusion{4} }
%  
%  $$
%    \numConclusion{5}
 % $$
%  $$
 % \excPost(ind, \texttt{Exc}) = \wpi{\compile{handler} }{ \compile{\psi}}{\excPost}
%  $$




%\todo{say that $\psi$ and $\compile{\psi}$ do not contain stack expressions } 
\begin{pogEquiv}
 
  For every statement $\stmt $ from the source language, any formula $\psi \in \formulaSrc$   and any exceptional postcondition
function $\excPostSrc$ and $\excPost$ such that
 $\excPostSrc ( \mbox{ \rm \texttt{Exc}} , \expressionSrc ) \eqModNames  \excPost ( \mbox{ \rm \texttt{Exc}}  , \compile{\expressionSrc}  ) $ we have that
\begin{itemize}
     \item  $$\wpBcSeq{ \compile{ \stmt}} { \compile{\psi}} {\excPost}$$  does not contain subexpressions of the form 
     \mbox{ \rm ~\stack{ind} } and\mbox{ \rm  \counter } 

     \item  $$\wpiSrc{ \stmt } { \psi} {\excPostSrc} \eqModNames \wpBcSeq{ \compile{ \stmt }} { \compile{\psi}} {\excPost}  $$ 
  
\end{itemize}


Proof: \\
By structural induction over the structure of the source expressions and statements 

\textit{Note:} in the following, we are using the following property  of the \textrm{wp} predicate transformer, namely   
$$\rm{wp}( \stmt, \psi, \excPost) \wedge \rm{wp}( \stmt, \phi, \excPost ) = \rm{wp}( \stmt, \psi \wedge \phi, \excPost )    $$
which is easy to establish.
\begin{description}


\item \textbf{Expressions}

\begin{description}
  \item  \textbf{integer constant access}

  \comment{by definition }
  $$\wpiSrc{const}{\normalPostSrc }{ \excPostSrc } = \normalPostSrc $$

  \comment{from the definition of the compiler function and the weakest precondition over bytecode} 

\[
\begin{array}{l}
  \wpBcSeq{ \compile{ const}} { \compile{\normalPostSrc }}{ \excPost } \\
  \\
  =  \\
 \\
  \wpBcSeq{\instr{push }\ \compileSynt{const}}{ \compile{\normalPostSrc }}{ \excPost } \\
\end{array}
\]
  
\comment{from the definition of the weakest precondition function of \mbox{ \rm \instr{push}} } 

$$
\begin{array}{l}
  \wpBcSeq{\instr{push }\ \compile{ \eval{ const}}}{ \compile{\normalPostSrc }}{ \excPost } \\
= \\

 \compile{\normalPostSrc } \substitution{ \counter } { \counter + 1 }  \substitution{ \stack{\counter + 1  } } { const } 
\end{array}
$$
 
\comment{ $ \compile{\normalPostSrc }  $ does not contain stack and stack counter expressions from Property \ref{formulaCompProp2} 
on page \pageref{formulaCompProp2} }

$$\begin{array}{l}
 \compile{\normalPostSrc } \substitution{ \counter } { \counter + 1 }  \substitution{ \stack{\counter + 1  } } { const } \\
 = \\
 \compile{\normalPostSrc }
\end{array}
$$
 

\comment{from the compiler for formulas we know that $\forall \psi \ . \ \psi  \eqModNames \compile{\psi}$ }
$$\normalPostSrc \eqModNames \compile{\normalPostSrc }  $$

 \item \textbf{ method invocation }



 \item \textbf{ assignment expressions } 
 \begin{description}
    \item \textbf{ local variable assignment} \\
    \comment{by definition of the weakest precondition for assignment }
$$\numConclusion{0} $$
  $$ \begin{array}{l} 
                \wpiSrc{ \expressionSrc_1 = \expressionSrc_2}{\normalPostSrc }{ \excPostSrc } = \\
                \\
                \wpiSrc{\expressionSrc_2}{ \normalPostSrc\substitution{\expressionSrc_1 }{\compileSynt{\expressionSrc_2 }} }{\excPostSrc}
      \end{array}$$  

\comment{by defintion of the compiler } 

$$\numConclusion{1} $$
$$
\begin{array}{l}

 \wpBcSeq{ \compile{ \expressionSrc_1 = \expressionSrc_2 }}{ \compile{\normalPostSrc} }{ \excPost } \\
=\\
 \wpBcSeq{   \begin{array}{l}
                        \compileExp{\expressionSrc_2 }{\stack{\counter}};\\
			\instr{store} \ \compile{\expressionSrc_1};
             \end{array} }{ \compile{\normalPostSrc} }{ \excPost } \\
\end{array}
$$



   \comment{by defintion of the weakest precondition function for \mbox{ \rm \instr{store}} }  


$$\begin{array}{l}
 \wpBcSeq{   \begin{array}{l}
                      \compileExp{\expressionSrc_2 }{\stack{\counter}};\\
                     \instr{store} \ \compile{\expressionSrc_1};
             \end{array} }{ \compile{\normalPostSrc} }{ \excPost }\\

= \\

 \wpBcSeq{  \compileExp{\expressionSrc_2 } {\stack{\counter}} } {  \compile{\normalPostSrc}
                                           \begin{array}{l}
                                               \substitution{ \counter}{ \counter - 1}  \\
                                               \substitution{ \compileSynt{ \expressionSrc_1} }{ \stack{\counter} }    
                                           \end{array} } { \excPost }

 \end{array}
$$

\comment{ as \mbox{ \rm $ \compile{\normalPost} $} does not contain stack counter expressions  from Property \ref{formulaCompProp2} 
on page \pageref{formulaCompProp2}  }

$$\numConclusion{2}    $$
$$
\begin{array}{l} 

 \wpBcSeq{  \compileExp{\expressionSrc_2 } {\stack{\counter}} } {  \compile{\normalPostSrc}
                                           \begin{array}{l}
                                               \substitution{ \counter}{ \counter - 1}  \\
                                               \substitution{ \compileSynt{ \expressionSrc_1} }{ \stack{\counter } }    
                                           \end{array} } { \excPost } \\
=\\
 \wpBcSeq{  \compileExp{\expressionSrc_2 } {} }{  \compile{\normalPostSrc}
     \substitution{ \compileSynt{ \expressionSrc_1} }{  \stack{\counter + 1} }  } { \excPost }
 \end{array} $$





\comment{from Lemma \ref{exprValueOnStack} on page \pageref{exprValueOnStack}, \numConclusion{2} and \numConclusion{0}   }
$$\numConclusion{4} $$


$$ \begin{array}{l}
\exists P, Q : \formulaSrc, \\ 
\numConclusion{4.1} \\ 
\wpiSrc{ \expressionSrc_1 = \expressionSrc_2}{\normalPostSrc }{ \excPostSrc }  \\
                \\
  = \wpiSrc{\expressionSrc_2}{ \normalPostSrc\substitution{\expressionSrc_1 }{\compileSynt{\expressionSrc_2 }} }{\excPostSrc}  \\
 = P \Rightarrow \normalPostSrc\substitution{\expressionSrc_1 }{\compileSynt{\expressionSrc_2 }}  \wedge R 
\\
\\\\
\numConclusion{4.2} \\ 
\wpBcSeq{ \compile{ \expressionSrc_1 = \expressionSrc_2 }}{ \compile{\normalPostSrc} }{ \excPost } \\
  =\\
 \wpBcSeq{  \compileExp{\expressionSrc_2 } {} }{  \compile{\normalPostSrc}
     \substitution{ \compileSynt{ \expressionSrc_1} }{  }  } { \excPost } \\
  = \\
  \compile{P} \Rightarrow  \compileSynt{ \expressionSrc_1}   \begin{array}{l}
                                              
                                               \substitution{ \compileSynt{ \expressionSrc_1} }{ \stack{\counter } }    \\
					        \substitution{ \counter}{ \counter + 1}  \\
						\substitution{\stack{ \counter +1}}{ \compileSynt{\expressionSrc_2 } }  \\
                                           \end{array}   \\
					   \wedge \compile{R}\\

\mbox{\rm\textit{\{ as there are no stack expressions in $\compileSynt{ \expressionSrc_1}$ }}\\
\mbox{\rm\textit{ and applying  properties of substitution \} }} \\
=\\
\compile{P} \Rightarrow  \compileSynt{ \normalPostSrc } \substitution{ \compileSynt{ \expressionSrc_1} }{ \compileSynt{\expressionSrc_2 }    }    
\mbox{\rm\comment{ from \numConclusion{4.1} and  \numConclusion{4.2} this case holds }} \\

\end{array} 
$$





    \item \textbf{instance field assignment} \\ 

  
  \comment{by definition of the weakest precondition function for field assignment}
   
$$ \numConclusion{1}  $$

  $$ \begin{array}{l} \wpiSrc{ \expressionSrc_1.f = \expressionSrc_2}{\normalPostSrc }{ \excPostSrc } = \\
               \\
  \wpiSrc{\expressionSrc_1}{  
\begin{array}{l}
         \Mynull \neq \eval{\expressionSrc_1} \Rightarrow \normalPostSrc \substitution{f}{f \oplus [ \eval{\expressionSrc_1} \rightarrow \eval{\expressionSrc_2}]}   \\
          \wedge \\
        \Mynull = \eval{\expressionSrc_1} \Rightarrow  \excPostSrc(\mbox{ \rm  \texttt{NullPointerExc} } ,  \expressionSrc_1.f = \expressionSrc_2 ) 
  \end{array}
}{\excPostSrc} \\


          \end{array}
	$$  


\comment{by the definition of the compiler function }



$$\begin{array}{l}
\wpBcSeq{ \compile{\expressionSrc_1.f = \expressionSrc_2}}{\normalPostSrc }{ \excPostSrc } \\
= \\
\wpBcSeq{ \begin{array}{l}
                        \compileExp{\expressionSrc_1}{\stack{\counter}};\\
			\compileExp{\expressionSrc_2}{\stack{\counter - 1}};\\
			\instr{putfield} \ f;
             \end{array}  }{ \compile{\normalPostSrc} }{ \excPost } 
\end{array}
$$



\comment{by the definition of the weakest precondition for  \mbox{ \rm \instr{putfield} } } 

$$
\begin{array}{l}
= \\
\wpBcSeq{ \begin{array}{l}
                          \compileExp{\expressionSrc_1}{\stack{\counter}};\\
			\compileExp{\expressionSrc_2}{\stack{\counter - 1}};\\
			\end{array}}  { \\ 
    \spaceWpBc     \Mynull =  \stack{\counter -1}  \Rightarrow \compile{ \normalPostSrc} \begin{array}{l}
                    \substitution{\counter}{ \counter -2}  \\
                    \substitution{f}{ f\oplus [ \stack{\counter -1 } \rightarrow \stack{\counter}  ] } \end{array} \\
    \spaceWpBc    \wedge \\		    
    \spaceWpBc    \Mynull =  \stack{\counter -1}  \Rightarrow  \excPost( \mbox{ \rm  \texttt{NullPointerExc} } ,  \mbox{ \rm  \instr{putfield} \ f } ) 


 } {\\
      \spaceWpBc     \excPost } 
\end{array}
$$



\comment{ $\compile{\normalPostSrc}$ does not contain stack counter expressions from Property \ref{formulaCompProp2} 
on page \pageref{formulaCompProp2} }

$$
\begin{array}{l}
= \\
\wpBcSeq{ \begin{array}{l}
                          \compileExp{\expressionSrc_1}{\stack{\counter}};\\
			\compileExp{\expressionSrc_2}{\stack{\counter - 1}};\\
			\end{array}}  { \\ 
    \spaceWpBc   \Mynull =  \stack{\counter -1}  \Rightarrow \compile{ \normalPostSrc}
                 
                    \substitution{f}{ f\oplus [ \stack{\counter -1 } \rightarrow \stack{\counter}  ] }  \\
    \spaceWpBc    \wedge \\		    
    \spaceWpBc  \Mynull =  \stack{\counter -1}  \Rightarrow  \excPost( \mbox{ \rm  \texttt{NullPointerExc} } ,  \mbox{ \rm  \instr{putfield} \ f } ) 


 } {\\
      \spaceWpBc     \excPost } 
\end{array}
$$





\comment{by definition of the weakest precondition function for a sequence of bytecode instructions }
$$ \numConclusion{2}  $$
$$
\begin{array}{l}
 = \\
\wpBcSeq{  \compileExp{\expressionSrc_1}{\stack{\counter - 1}} }
	{ \\
          \spaceWpBc   \wpBcSeq{\compileExp{\expressionSrc_2}{\stack{\counter }} } { \\

    \spaceWpBc \spaceWpBc  \Mynull \neq \stack{\counter -1} \Rightarrow \compile{ \normalPostSrc}
                 
                    \substitution{f}{ f\oplus [ \stack{ \counter - 1 } } \rightarrow \stack{\counter}   ]   \\
    \spaceWpBc  \spaceWpBc   \wedge \\		    
    \spaceWpBc  \spaceWpBc   \Mynull = \stack{\counter -1} \Rightarrow  \excPost( \mbox{ \rm  \texttt{NullPointerExc} } ,  \mbox{ \rm  \instr{putfield} \ f } ) 
     } { \\  
                         \spaceWpBc \spaceWpBc  \excPost }} {  \\ 
        \spaceWpBc \excPost}

\end{array}
$$

\comment{applying twice the lemma \ref{exprValueOnStack} on page \pageref{exprValueOnStack}  }

$$\numConclusion{3}  $$

$$
\begin{array}{l}
 = \\
\wpBcSeq{  \compileExp{\expressionSrc_1}{\stack{\counter - 1}} }
	{ \\
          \spaceWpBc   \wpBcSeq{\compileExp{\expressionSrc_2}{\stack{\counter }} } { \\

    \spaceWpBc \spaceWpBc  \Mynull \neq  \stack{\counter -1} \Rightarrow \compile{ \normalPostSrc}
                 
                    \substitution{f}{ f\oplus[\stack{ \counter - 1 }}\rightarrow\stack{\counter}]  \\
    \spaceWpBc  \spaceWpBc   \wedge \\		    
    \spaceWpBc  \spaceWpBc   \Mynull =\stack{\counter-1}\Rightarrow\excPost( \mbox{ \rm  \texttt{NullPointerExc} } ,  \mbox{ \rm  \instr{putfield} \ f } ) \\
       \spaceWpBc  \spaceWpBc   \wedge \\	
       \spaceWpBc  \spaceWpBc \stack{\counter} = \compile{\eval{\expressionSrc_2}}   
     } { \\  
                         \spaceWpBc \spaceWpBc  \excPost } \\
     \spaceWpBc   \wedge \\
      \spaceWpBc  \stack{\counter} = \compile{\eval{\expressionSrc_1}} } {  \\ 
        \spaceWpBc \excPost}

\end{array}
$$


\comment{from lemma \ref{stackCounter} on page \pageref{stackCounter} }
$$\numConclusion{4}  $$
$$
\begin{array}{l}
 = \\
\wpBcSeq{  \compileExp{\expressionSrc_1}{\stack{\counter - 1}} }
	{ \\
          \spaceWpBc   \wpBcSeq{\compileExp{\expressionSrc_2}{\stack{\counter }} } { \\

    \spaceWpBc \spaceWpBc  \Mynull \neq  \stack{\counter -1} \Rightarrow \compile{ \normalPostSrc}
                 
                    \substitution{f}{ f\oplus[\stack{ \counter - 1 }}\rightarrow\stack{\counter}]  \\
    \spaceWpBc  \spaceWpBc   \wedge \\		    
    \spaceWpBc  \spaceWpBc   \Mynull =\stack{\counter-1}\Rightarrow\excPost( \mbox{ \rm  \texttt{NullPointerExc} } ,  \mbox{ \rm  \instr{putfield} \ f } ) \\
    \spaceWpBc  \spaceWpBc   \wedge \\	
    \spaceWpBc  \spaceWpBc \stack{\counter} = \compile{\eval{\expressionSrc_2}}   \\
    \spaceWpBc  \spaceWpBc     \wedge \\
    \spaceWpBc  \spaceWpBc  \stack{\counter - 1} = \compile{\eval{\expressionSrc_1}}    
     } { \\  
                         \spaceWpBc \spaceWpBc  \excPost } \\
       } {  \\ 
        \spaceWpBc \excPost}

\end{array}
$$

$$
\begin{array}{l}
= \\
\\
\wpBcSeq{  \compileExp{\expressionSrc_1}{\stack{\counter - 1}} }
	{ \\
          \spaceWpBc   \wpBcSeq{\compileExp{\expressionSrc_2}{\stack{\counter }} } { \\

    \spaceWpBc \spaceWpBc  \Mynull \neq \compile{\eval{\expressionSrc_1}}  \Rightarrow \compile{ \normalPostSrc}
                 
                    \substitution{f}{ f\oplus[\compile{\eval{\expressionSrc_1}} \rightarrow \compile{\eval{\expressionSrc_2}} ]}  \\
    \spaceWpBc  \spaceWpBc   \wedge \\		    
    \spaceWpBc  \spaceWpBc   \Mynull = \compile{\eval{\expressionSrc_1}} \Rightarrow\excPost( \mbox{ \rm  \texttt{NullPointerExc} } ,  \mbox{ \rm  \instr{putfield} \ f } ) \\
     } { \\  
        \spaceWpBc \spaceWpBc  \excPost } }
        {  \\ 
        \spaceWpBc \excPost}
\end{array}
$$

\comment{ from \numConclusion{1}  and \numConclusion{4} applying the induction hypothesis we obtain that the theorem holds in the case
of instance field assignment   }

        

\end{description}


 \item \textbf{field access}

 \item \textbf{arithmetic expressions }

\comment{by definition of the weakest precondition function for arithmetic expressions}
$$ \begin{array}{l}
	              \wpiSrc{ \expressionSrc_1 \ op \ \expressionSrc_2 }{\normalPostSrc }{ \excPostSrc } =  \\
		      \wpiSrc{ \expressionSrc_1  }{\wpiSrc{\expressionSrc_2 }{ \normalPostSrc }{ \excPostSrc }  }{ \excPostSrc }
		\end{array}
             $$
\comment{by defintition of the compiler in section \ref{compile} } 
$$ \numConclusion{0}$$
$$
\begin{array}{l}
\wpBcSeq{ \compile{\expressionSrc_1 \ op \ \expressionSrc_2 } }{ \compile{ \normalPostSrc } }{ \compile{ \excPostSrc} } \\
= \\ 
\wpBcSeq{ \begin{array}{l}
                       \compileExp{ \expressionSrc_1}{\stack{\counter - 1}}; \\
                       \compileExp{ \expressionSrc_2}{\stack{\counter}}; \\
                       \instr{op}
                 \end{array}         }{ \\
            \spaceWpBc       \compile{ \normalPostSrc } }{ \\
       \spaceWpBc    \excPost } 

\end{array}
$$


\comment{from the definition of the weakest precondition of the \mbox{ \rm \instr{op}} instruction }
$$ \numConclusion{1}$$
$$ 
\begin{array}{l}

= \\
 \wpBcSeq{ \begin{array}{l}
                        \compileExp{ \expressionSrc_1}{\stack{\counter - 1}}; \\
                       \compileExp{ \expressionSrc_2}{\stack{\counter}}; \\
                    
                      
            \end{array}         }{ \\
                      \spaceWpBc     \compile{ \normalPostSrc } \begin{array}{l}
								       \substitution{ \counter}{\counter -1} \\ 
								       \substitution{\stack{\counter - 1}  } { \stack{\counter } \ op \ \stack{\counter - 1 }  }
								   \end{array}}{ \\
                 \spaceWpBc     \compile{ \excPostSrc} }
\end{array}
 $$


%\comment{ by the definition of section \ref{compile}, before the execution of the instruction \mbox{ \rm $\instr{op}$} the first two stack top elements are  }
%$$ \numConclusion{2}$$
%$$
%\begin{array}{l}
%\stack{\counter} = \compile{\eval{ \expressionSrc_2}} \\
%\stack{\counter - 1 } = \compile{\eval{ \expressionSrc_1}}
%\end{array}
%$$

% \comment{ from  \numConclusion{2} and \numConclusion{1}  }
% $$ 
% \begin{array}{l}
%= \\
% \wpBcSeq{ \begin{array}{l}
%                       \compile{ \expressionSrc_1}; \\
%                       \compile{ \expressionSrc_2}; \\
%            \end{array}         }{  \compile{ \normalPostSrc } 
%                                   \begin{array}{l}
%				        \substitution{ \counter}{\counter -1} \\ 
%                                        \substitution{\stack{\counter - 1}  } { \compile{\eval{ \expressionSrc_2}}  \ op \  \compile{\eval{ \expressionSrc_1}} }  
%                                   \end{array} }  { \compile{ \excPostSrc} } 
%\end{array}
% $$

\comment{ the formula \mbox{ \rm $\compile{\psi}$} does not contain \mbox{ \rm \counter } and \mbox{ \rm \stack{ \counter} }   expressions from Property \ref{formulaCompProp2} 
on page \pageref{formulaCompProp2}  } 

$$ \numConclusion{2}$$

$$
\begin{array}{l}
= \\
 \wpBcSeq{ \begin{array}{l} 
                       \compileExp{ \expressionSrc_1}{\stack{\counter - 1}}; \\
                       \compileExp{ \expressionSrc_2}{\stack{\counter}}; \\
                                              
            \end{array}         }{ \\
              \spaceWpBc   \compile{ \normalPostSrc } }  { \\
              \spaceWpBc  \excPostSrc } 
\end{array}
$$

\comment{from \numConclusion{2}, as \mbox{ \rm $\compile{ \expressionSrc_1}  $} and \mbox{ \rm $\compile{ \expressionSrc_2}  $}  execute   sequentially} 

$$ \numConclusion{3}$$
$$
\begin{array}{l}
= \\
 \wpBcSeq{\compile{ \expressionSrc_1} }{ \wpBcSeq{\compile{ \expressionSrc_2} }{ \compile{ \normalPostSrc } }{ \excPost }  }  {  \excPost } 
\end{array}
$$

\comment{by induction hypothesis over the structure of the source statements }

$$ \numConclusion{4}$$

$$
\begin{array}{l}
\wpiSrc{\expressionSrc_2 }{ \normalPostSrc }{ \excPostSrc } \\
\eqModNames \\ 
\wpBcSeq{\compileExp{\expressionSrc_2}{\stack{\counter - 1}}}{\compile{\normalPostSrc} }{ \excPost}  \\
\end{array}
$$

\comment{by induction hypothesis over the structure of the source statements and \numConclusion{5}}

$$ \numConclusion{5}$$

$$
\begin{array}{l}
 \wpiSrc{ \expressionSrc_1  }{\wpiSrc{\expressionSrc_2 }{ \normalPostSrc }{ \excPostSrc }  }{ \excPostSrc } \\
\eqModNames \\ 
 \wpBcSeq{\compileExp{ \expressionSrc_1}{\stack{\counter}} }{ \\
     \spaceWpBc        \wpBcSeq{\compile{ \expressionSrc_2} }{ \\
                       \spaceWpBc    \spaceWpBc  \compile{ \normalPostSrc } }{ \\
     \spaceWpBc  \spaceWpBc    \excPost }  }  
{ \\
 \spaceWpBc \excPost } 
\end{array}
$$


\comment{ from \numConclusion{0}, \numConclusion{3} and  \numConclusion{5} the property holds in the case of arithmetic expression }

 \item \textbf{cast expressions}

\comment{by definition of the weakest precondition function for cast expressions}

 $$ \begin{array}{l} \wpiSrc{ ( \mbox { \rm \texttt{Class} } ) \ \expressionSrc  }{\normalPostSrc }{ \excPostSrc } = \\
	           \wpiSrc{ \expressionSrc } {  
                    \begin{array}{l}
                     \typeof{\expressionSrc }  <:  \texttt{Class}  \Rightarrow   \\
 \spaceWpBc     \spaceWpBc \wpiSrc{ \expressionSrc  }{ \normalPostSrc }{ \excPostSrc } \\
		     \wedge \\
		  \neg  \typeof{ \expressionSrc } <:   \texttt{Class}  \Rightarrow  \\ 
 \spaceWpBc     \spaceWpBc  \excPostSrc( \mbox{ \rm \tt{ClassCastException} }, \mbox{ \rm \tt{Class} }  \ \expressionSrc  )
                    \end{array}  } { \excPostSrc }  
\end{array} $$

\comment{by the definition of the compiler }



$$
\begin{array}{l}
\wpBcSeq{\compile{ ( \mbox{ \rm \texttt{Class} } ) \ \expressionSrc  } }{\compile{\normalPostSrc  } }{\excPost } \\
=\\
\wpBcSeq{  \begin{array}{l}  
                 \compileExp{\expressionSrc}{\stack{\counter}};\\
                   \instr{checkCast } \ \mbox{ \rm \texttt{Class }};
              \end{array}  }{ \\
        \spaceWpBc       \compile{\normalPostSrc  } }{\\
         \spaceWpBc   \excPost }\end{array}$$

\comment{from the definition of the weakest precondition function for \mbox{ \rm  \instr{checkCast} } }

$$\numConclusion{1}$$
$$ \begin{array}{l} \wpBcSeq{\compileExp{\expressionSrc}{\stack{\counter}} } 
           { \\
          \spaceWpBc  \begin{array}{l}
		    \typeof{\stack{\counter}} <: \mbox{ \rm \texttt{Class }} \Rightarrow \\
                        \spaceWpBc     \spaceWpBc    \compile{\normalPostSrc  }  \\
		    \wedge \\
                   \neg ( \typeof{\stack{\counter}} <: \mbox{ \rm \texttt{Class } } ) \Rightarrow \\ 
                      \spaceWpBc      \spaceWpBc    \excPost( \mbox{\rm \texttt{ClassCastException}}, 
\mbox{ \rm \instr{checkCast}})) 
             \end{array} }
           { \\
           \spaceWpBc   \excPost }
\end{array} $$

\comment{ from lemma \ref{exprValueOnStack} on page \pageref{exprValueOnStack} and \numConclusion{1} }
$$\numConclusion{2}$$

$$ \begin{array}{l} 

=\\
 \wpBcSeq{\compileExp{\expressionSrc}{\stack{\counter}} } 
           { \\
          \spaceWpBc  \begin{array}{l}
		    \typeof{\stack{\counter}} <: \mbox{ \rm \texttt{Class }} \Rightarrow \\
                        \spaceWpBc \spaceWpBc    \compile{\normalPostSrc  }  \\
		    \wedge \\
                   \neg ( \typeof{\stack{\counter}} <: \mbox{ \rm \texttt{Class } } ) \Rightarrow \\ 
                      \spaceWpBc  \spaceWpBc  \excPost( \mbox{\rm \texttt{ClassCastException}}, 
\mbox{ \rm \instr{checkCast}})) \\
                 \wedge \\ 
                  \stack{\counter} = \compile{\eval{\expressionSrc_1}}
             \end{array} }
           { \\
           \spaceWpBc   \excPost }
\end{array} $$

$$\numConclusion{3}$$
$$ \begin{array}{l} 
= \\
\wpBcSeq{\compileExp{\expressionSrc}{\stack{\counter}} } 
           { \\
          \spaceWpBc  \begin{array}{l}
		    \typeof{\compile{\eval{\expressionSrc_1}} } <: \mbox{ \rm \texttt{Class }} \Rightarrow \\
                        \spaceWpBc \spaceWpBc    \compile{\normalPostSrc  }  \\
		    \wedge \\
                   \neg ( \typeof{\compile{\eval{\expressionSrc_1}}  } <: \mbox{ \rm \texttt{Class } } ) \Rightarrow \\ 
                      \spaceWpBc  \spaceWpBc  \excPost( \mbox{\rm \texttt{ClassCastException}}, 
\mbox{ \rm \instr{checkCast}})) \\
                
                
             \end{array} }
           { \\
           \spaceWpBc   \excPost }
\end{array} $$

\comment{ from the induction hypothesis  }


$$
\begin{array}{l}
 \wpiSrc{ \expressionSrc } { \\ 
                    \begin{array}{l}
                    \typeof{\expressionSrc } <:\mbox { \rm \texttt{Class} }  \Rightarrow  \\
                \spaceWpBc     \spaceWpBc  \wpiSrc{ \expressionSrc  }{ \normalPostSrc }{ \excPostSrc } \\
		     \wedge \\
		    \neg \   \typeof{ \expressionSrc } <:  \mbox { \rm \texttt{Class} } \Rightarrow \\
                  \spaceWpBc     \spaceWpBc     \excPostSrc( \mbox{ \rm \texttt{ClassCastException}, \instr{checkCast}} )
                    \end{array}  } { \\ \excPostSrc } \\
	            
 
\eqModNames \\
\wpBcSeq{\compileExp{\expressionSrc}{\stack{\counter}}  } 
           { \\ 
       \spaceWpBc       \begin{array}{l}
		    \typeof{\stack{\counter}} <:  \texttt{Class } \Rightarrow  \\
                   \spaceWpBc     \spaceWpBc      \compile{\normalPostSrc  }  \\
		    \wedge \\
                   \neg ( \typeof{  \compile{\eval{\expressionSrc }} } <:\texttt{Class }  ) \Rightarrow  \\
 \spaceWpBc     \spaceWpBc \excPost( \mbox{ \rm \tt{ClassCastException} , \instr{checkCast} } ) 
             \end{array} }
           { \\         \spaceWpBc   \excPost }
\end{array}$$ 

\comment{ we can conclude that this case holds  }


\end{description}




\item \textbf{Statements}
     \begin{description}


           \item\textbf{  compositional statements}
	      $$\wpiSrc{ \stmt_1;\stmt_2} { \psi} {\excPostSrc} = $$
	      \comment{by defintition}
              $$ \wpiSrc{\stmt_1 }{\wpiSrc{\stmt_2 }{\psi}{ \excPostSrc}} { \excPostSrc} =  $$
	      \comment{applying the induction hypothesis}

	     \begin{itemize}
	     \item $ \wpBcSeq{\compile{\stmt_2}}{ \compile{\psi}} {\excPost} $ does not contain stack expressions
	      \item $$ \numConclusion{1} \  \wpiSrc{\stmt_2 }{\psi }{ \excPostSrc}\eqModNames 
\wpBcSeq{\compile{\stmt_2}}{ \compile{\psi}} {\excPost}  $$
              \end{itemize}
	      
	      \comment{applying the induction hypothesis and from \numConclusion{1} we conclude  }
	      $$ \numConclusion{2}$$
	      $$
	      \begin{array}{l}
		      \wpiSrc{\stmt_1 }{\wpiSrc{\stmt_2 }{\psi}{ \excPostSrc}} { \excPostSrc}\\  \eqModNames  \\
		       \wpBcSeq{\compile{\stmt_1} }{\wpBcSeq{\compile{\stmt_2} }{\compile{\psi}}{ \excPost}} {\compile{ \excPostSrc}}
                 \end{array}$$
	    




	    



	    \item \textbf{ conditional statement}

We suppose here that the condition of the statement is a boolean variable or constant (the case
 when the condition is a relation expression is similar) 
 $$ \wpiSrc{\Myif \ (\expressionSrc) \ \Mythen \ \{ \stmt_1 \} \  \Myelse \ \{ \stmt_2 \} } { \psi} {\excPostSrc} = $$
   \comment{by definition}
  $$ \numConclusion{0} \wpiSrc{ \expressionSrc }{ 
       \begin{array}{l}  
	 \eval{\expressionSrc } = \Mytrue \Rightarrow \wpiSrc{\stmt_1 }{\normalPostSrc }{ \excPost } \\
	 \wedge \\
	 \eval{\expressionSrc } = \Myfalse \Rightarrow \wpiSrc{\stmt_2 }{\normalPostSrc }{ \excPost } \\
	 \end{array}
     } {\excPostSrc }
     $$


\comment{by induction hypothesis}
$$\numConclusion{1} $$

\begin{itemize}
	     \item $ \wpBcSeq{\compile{\stmt_1}}{ \compile{\psi}} {\excPost} $ does not contain stack expressions
\item $$ 
 \begin{array}{l}
      \wpiSrc{\stmt_1 }{\normalPostSrc }{ \excPost }\\  
      \eqModNames  \\ 
      \wpBcSeq{\compile{\stmt_1} }{\compile{\normalPostSrc} }{
\compile{ \excPost} }
  \end{array}
$$
\end{itemize}
\comment{ from \numConclusion{1}  }

$$\numConclusion{1.1}$$
$$ 
 \begin{array}{l}
      \wpiSrc{\stmt_1 }{\normalPostSrc }{ \excPost }\\  
      \eqModNames  \\ 
      \wpBcSeq{\compile{\stmt_1} }{\compile{\normalPostSrc} }{
\compile{ \excPost} }\substitution{\counter}{ \counter - 1}
  \end{array}
$$

\comment{from \numConclusion{1.1}}
$$\numConclusion{1.2}$$
$$ 
  \begin{array}{l}
     \eval{\expressionSrc } = \Mytrue \Rightarrow \wpiSrc{\stmt_1 }{\normalPostSrc }{ \excPostSrc }\\  
      \eqModNames  \\ 
      \compile{\eval{\expressionSrc }} = \compile{\Mytrue} \Rightarrow \wpBcSeq{\compile{\stmt_1} }{\compile{\normalPostSrc} }{
 \excPost }\substitution{\counter}{ \counter - 1}
  \end{array}
$$



\comment{by induction hypothesis}
$$\numConclusion{2} $$
$$
\begin{array}{l}
       \wpiSrc{\stmt_2 }{\normalPostSrc }{ \excPostSrc }\\ 
       \eqModNames  \\ 
       \wpBcSeq{\compile{\stmt_2} }{\compile{\normalPostSrc} }{
 \excPost }
  \end{array} $$


\comment{from lemme \ref{interProp} on page \pageref{interProp}} 

$$\numConclusion{2.1} $$
$$
\begin{array}{l}
       \wpiSrc{\stmt_2 }{\normalPostSrc }{ \excPostSrc }\\ 
       =  \\ 
       \wpBcSeq{\compile{\stmt_2;}\instr{ goto } \ l }{\compile{\normalPostSrc} }{ \excPost }
  \end{array} $$

 

\comment{ from \numConclusion{2.1} as  \mbox{ \rm $ \wpBcSeq{\compile{\stmt_2};\instr{ goto } \ l }{\compile{\normalPostSrc} } $} does not contain the \mbox{ \rm $ \counter $} expression from Property \ref{formulaCompProp2} 
on page \pageref{formulaCompProp2}  }

$$\numConclusion{2.2}$$
$$\begin{array}{l}
       \wpiSrc{\stmt_2 }{\normalPostSrc }{ \excPostSrc }\\ 
       =  \\ 
       \wpBcSeq{\compile{\stmt_2;}\instr{goto} \ l }{\compile{\normalPostSrc} }{\excPost } \\
           \Myspace \Myspace  \Myspace \Myspace     \Myspace \Myspace  \substitution{\counter}{ \counter - 1}
  \end{array} $$







$$\numConclusion{2.3}$$
$$
\begin{array}{l}
     \eval{\expressionSrc } = \Myfalse \Rightarrow   \wpiSrc{\stmt_2 }{\normalPostSrc}{\excPostSrc } \\  
     \eqModNames  \\ 
      \compile{\eval{\expressionSrc }} = \compile{\Myfalse} \Rightarrow   \wpBcSeq{\compile{\stmt_2}\instr{ goto } \ l }{\compile{\normalPostSrc} }{
 \excPost } \\
\Myspace \Myspace \Myspace\Myspace \Myspace \Myspace \Myspace  \substitution{\counter}{ \counter - 1}
  \end{array} $$



\comment{from \numConclusion{2.3} and \numConclusion{1.2}  }
$$\numConclusion{3}$$
$$
  \begin{array}{l}
      \eval{\expressionSrc } = \Myfalse \Rightarrow   \wpiSrc{\stmt_2 }{\normalPostSrc }{
\excPostSrc } \\  \wedge   \\
   \eval{\expressionSrc } = \Mytrue \Rightarrow   \wpiSrc{\stmt_1}{\normalPostSrc }{ \excPost } \\
\\
     \eqModNames  \\ 
\\  
    \compile{\eval{\expressionSrc }} = \compile{\Myfalse} \Rightarrow   \wpBcSeq{\compile{\stmt_2}\instr{ goto } \ l }{\compile{\normalPostSrc} }{
\compile{ \excPost} } \\
\Myspace \Myspace \Myspace\Myspace \Myspace \Myspace \Myspace  \substitution{\counter}{ \counter - 1}  \\
\wedge \\  
   \compile{\eval{\expressionSrc }} = \compile{\Mytrue} \Rightarrow   
\wpBcSeq{\compile{\stmt_1}}{\compile{\normalPostSrc} }{\compile{ \excPost} } \\
\Myspace \Myspace \Myspace\Myspace \Myspace \Myspace \Myspace  \substitution{\counter}{ \counter - 1}

  \end{array} $$

\comment{from the definitions of the predicate transformer for the bytecode instruction \instr{ if\_cond}  }

$$ \numConclusion{4} $$
$$
\begin{array}{l}
  \begin{array}{l}  \wpBcSeq{
      \begin{array}{l}
          \compileExp{ \expressionSrc}{\stack{\counter}};\\
           \instr{ if\_cond \ l };\\
            \compile{\stmt_2};\\
	    goto \ l; \\
	    \compile{\stmt1}; \\
	    l: \ldots \\
      \end{array}}{\compile{\normalPostSrc} }{ \excPost } \\  


  \end{array}  \\
\\
= \\
\\
  \begin{array}{l}  \wpBcSeq{  \compileExp{ \expressionSrc}{\stack{\counter}}} {\\
     \spaceWpBc 
     \begin{array}{l}
     \compile{\stack{\counter}} = \compile{\Mytrue} \Rightarrow \wpBcSeq{\compile{\stmt_1} } { \compile{\normalPostSrc} }{
     \compile{ \excPost} } \\
     \Myspace \Myspace \Myspace\Myspace \Myspace \Myspace \Myspace  \substitution{\counter}{ \counter - 1} \\
     \wedge \\
     \compile{\stack{\counter}}  = \compile{\Myfalse} \Rightarrow \wpBcSeq{\compile{\stmt_2;} goto \ l }{\compile{\normalPostSrc} }{
     \compile{ \excPost} } \\
     \Myspace \Myspace \Myspace\Myspace \Myspace \Myspace \Myspace  \substitution{\counter}{ \counter - 1} 
     \end{array}   } {\\
     \spaceWpBc \excPost}
  
  \end{array}
\end{array}
$$


\comment{ from \numConclusion{4} and lemma \ref{exprValueOnStack}  on page \pageref{exprValueOnStack} }
$$ \numConclusion{5} $$
$$\begin{array}{l}
= \\
\\
  \begin{array}{l}  \wpBcSeq{  \compileExp{ \expressionSrc}{\stack{\counter}}} {\\
     \spaceWpBc 
     \begin{array}{l}
     \compile{\stack{\counter}} = \compile{\Mytrue} \Rightarrow \wpBcSeq{\compile{\stmt_1} } { \compile{\normalPostSrc} }{ \excPost} \\
     \Myspace \Myspace \Myspace\Myspace \Myspace \Myspace \Myspace  \substitution{\counter}{\counter - 1} \\
     \wedge \\
     \compile{\stack{\counter}}  = \compile{\Myfalse} \Rightarrow \wpBcSeq{\compile{\stmt_2;} goto \ l }{\compile{\normalPostSrc} }{\excPost } \\
     \Myspace \Myspace \Myspace\Myspace \Myspace \Myspace \Myspace  \substitution{\counter}{\counter - 1} \\
     \wedge \\
     \stack{\counter} = \compile{\eval{\expressionSrc}}
     \end{array}   } {\\
     \spaceWpBc \excPost}
  \end{array}
\end{array}
$$


$$ \numConclusion{6} $$

$$\begin{array}{l}
= \\
\\  \wpBcSeq{  \compileExp{ \expressionSrc}{ \stack{\counter}};}{ \\
\spaceWpBc
\begin{array}{l}
 \compile{\eval{\expressionSrc } } = \compile{\Mytrue} \Rightarrow \wpBcSeq{\compile{\stmt_1} } {\compile{\normalPostSrc} }{
\compile{ \excPost} } \\
\Myspace \Myspace \Myspace\Myspace \Myspace \Myspace \Myspace  \substitution{\counter}{ \counter - 1} \\
\wedge \\
\compile{\eval{\expressionSrc } }  = \compile{\Myfalse} \Rightarrow \wpBcSeq{\compile{\stmt_2;} goto \ l }{\compile{\normalPostSrc} }{
\compile{ \excPost} } \\
\Myspace \Myspace \Myspace\Myspace \Myspace \Myspace \Myspace  \substitution{\counter}{ \counter - 1} 
\end{array}} { \\
 \spaceWpBc \excPost }  
 \end{array}$$



\comment{ from \numConclusion{0}, \numConclusion{6}  and \numConclusion{3} applying the induction hypothesis}

$$\begin{array}{l}
\wpiSrc{\Myif \ (\expressionSrc) \ \Mythen \ \{ \stmt \} \  \Myelse \ \{ \stmt \}  }{\normalPostSrc }{\excPost } \\
\eqModNames \\ 
 \wpBcSeq{  \compile{ \Myif \ (\expressionSrc) \ \Mythen \ \{ \stmt \} \  \Myelse \ \{ \stmt \}    }      }{\compile{\normalPostSrc} }{\compile{ \excPost} } 
\end{array} $$

\item \textbf{ try catch statement}

   $$ \begin{array}{l}   
              \wpiSrc{ \try \ \{ \stmt_1 \} \ \catch (Class \ )\ \{ \stmt_2 \} } { \normalPostSrc}{\excPostSrc} = \\
              \\
	      \wpiSrc{ \stmt_1}{ \normalPostSrc}{ \excPostSrc \oplus [Class, \stmt_1   \longrightarrow \wpiSrc{\stmt_2}{ \normalPostSrc}{\excPostSrc}  ] }
       \end{array} 
   $$

\comment{We now show that for every exception the function \\
\mbox{ \rm $\excPostSrc \oplus [Class \longrightarrow \wpiSrc{\stmt_2}{ \normalPostSrc}{\excPostSrc}  ] (\stmt_1, \mbox{ \rm \texttt{Exc}})$}
 is syntactically equivalent to \mbox{ \rm  $\excPost (\compile{\stmt_1}, \mbox{ \rm \texttt{Exc}}) $ }.
By definition, if an exception of type \mbox{ \rm \texttt{Exc}} is thrown  during the execution of \mbox{ \rm $\stmt_1$}, we have the following:  }  

$$\numConclusion{1} \   $$
$$ \begin{array}{l}
\excPostSrc \oplus [\texttt{Class} \longrightarrow \wpiSrc{\stmt_2}{ \normalPostSrc}{\excPostSrc}  ]  (\stmt_1 , \mbox{ \rm \texttt{Exc}})  =\\
\left\{\begin{array}{ll} 
     \wpiSrc{\stmt_2}{ \normalPostSrc}{\excPostSrc} & if \ \texttt{Exc} <: \texttt{Class} \\
& \\
     \wpiSrc{ \mbox{ \rm handler } }{ \normalPostSrc}{\excPostSrc} & if \  \neg \mbox{ \rm (\texttt{Exc}} <: \texttt{Class} ) \ and \ \\
                                                    & \mbox{ \rm \texttt{Exc}} \ is \ handled \ by \  \mbox{ \rm handler} \\
& \\
      \excPostSpecSrc_{\tt{m}}(\texttt{Exc} )        & else  
\end{array} \right.
\end{array}  $$

\textit{\{ by definition, if an exception of type \mbox{\rm \texttt{Exc}} is thrown  during the execution of $\compile{\stmt_1}$ at instruction at index $ind$ as we know that
an exception handler $\compile{\stmt_2}$ for this exception exists \} }  
$$ \numConclusion{2} $$
$$\excPost ( \mbox{\rm \texttt{Exc}}, ind ) =
\left\{\begin{array}{ll} 
\wpi{\compile{\stmt_2}}{ \compile{\normalPostSrc}}{\excPost}  & if \ \mbox{\rm \texttt{Exc}}  <: \mbox{\rm \texttt{Class}} \\
& \\
\wpi{\compile{\mbox{ \rm handler} }}{ \compile{\normalPostSrc}}{\excPost}  & if \ \neg ( \mbox{\rm \texttt{Exc}} <: \mbox{\rm \texttt{Class}} ) \ and \ \\
                                                              &  \mbox{\rm \texttt{Exc}} \ is \ handled \ by \  \compile{\mbox{ \rm handler} }  \\
& \\
 \excPostSpec_{\tt{m}}( \mbox{\rm \texttt{Exc}}  )        & else  
\end{array}\right.   $$


\comment{by induction hypothesis } 
 $$\numConclusion{3}  $$
$$
\begin{array}{l}
 \wpiSrc{\stmt_2}{ \normalPostSrc}{\excPostSrc} \\
\eqModNames \\

\wpi{\compile{\stmt_2}}{ \compile{\normalPostSrc}}{\excPost} \\
\end{array}
$$


\comment{from  \numConclusion{1},\numConclusion{2} and \numConclusion{3} } 
 $$\numConclusion{4}  $$
$$
\begin{array}{l}
\forall  \mbox{\rm \texttt{Exc} } \\
      \excPostSrc \oplus [ \mbox{\rm \texttt{Class}} \longrightarrow \wpiSrc{\stmt_2}{ \normalPostSrc}{\excPostSrc}  ] 
(\mbox{\rm \texttt{Exc}}  , \stmt_1)\\
 \\
      =^{mod \ names} \\
\\   
  \excPost( \mbox{\rm \texttt{Exc} }, \compile{ \stmt_1}   )
\end{array}
 $$

\comment{from \numConclusion{4} and by induction hypothesis }

$$\numConclusion{5}   $$
$$\begin{array}{l} 

 \wpiSrc{ \stmt_1}{ \normalPostSrc}{ \excPostSrc \oplus [Class, \stmt_1 \longrightarrow \wpiSrc{\stmt_2}{ \normalPostSrc}{\excPostSrc}  ] } \\
  \eqModNames   \\
 \wpBcSeq{ \begin{array}{l}
                   \compile{\stmt_1} \\
		   goto \ l; \\
		       ...\\
		       l: 
           \end{array} } { \compile{\normalPostSrc}}{\excPost}
\end{array} $$

\comment{from \numConclusion{5} and the definition of the weakest precondition} 


$$\begin{array}{l} 
 \wpiSrc{ \try \{ \stmt_1 \} \catch (\mbox{ \rm \texttt{Class}} ) \{ \stmt_2 \} }{ \normalPostSrc}{ \excPostSrc  } \\
  \eqModNames   \\
 \wpBcSeq{ \begin{array}{l}
                   \compile{\stmt_1} \\
		   goto \ l; \\
                   \compile{\stmt_2} \\
                    goto \ l; \\ 
		    ...\\
		    l: 
           \end{array} } { \compile{\normalPostSrc}}{\excPost}
\end{array} $$



 \item \textbf{try finally statement}

\comment{by definition  }


$$
\begin{array}{l}
\wpiSrc{ \try \{ \stmt_1 \} \ \finally \ \{ \stmt_2 \} }{\normalPost}{ \excPostSrc} \\
\\ 
= \\
\\
 \wpiSrc{ \stmt_1}{ \wpiSrc{\stmt_2}{ \normalPostSrc} { \excPostSrc}}{ \\ \phantom{aaaaaaa} \excPostSrc \oplus [Exception \longrightarrow \wpiSrc{\stmt_2}{ \normalPostSrc}{\excPostSrc}  ] }
       \end{array}$$

\textit{Note}: $exc$ is the exception object thrown by $\stmt_2$

\comment{from the definition of the weakest precondition and the definition of the compiler function for\\
 \mbox{ \rm $\try \ \{ \stmt \} \ \finally \ \{ \stmt \} $ } statements  }

$$
\begin{array}{l}
\wpBcSeq{\compile{  \try \{ \stmt_1 \} \ \finally \ \{ \stmt_2 \}  } } {\compile{\normalPost}}{\excPost} \\
=\\
\wpBcSeq{ \begin{array}{l}
       \compile{\stmt_1}; \\
       \instr{jsr} \ s;\\
        \instr{goto } \ l ;\\
       \\ 
      
      h: \instr{astore} \ e; \\
       \instr{jsr} \ s; \\
       \instr{aload} \ e;\\
       \instr{athrow};\\

       \\      
       s: \instr{astore} \ k; \\
       \compile{\stmt_2}; \\ 
       \instr{ret} \ k \\
       l:

	%	\addExceptionTable{ \stR{\compile{\stmt_1} }}{ \enR{  \compile{ \stmt_1 }  }}{ h }{ Exception} )
   \end{array}
     } {\compile{\normalPost}}{\excPost} 
\end{array}$$

\comment{ from the definition of the weakest precondition \mbox{ \rm \instr{goto} } instruction }

$$
\begin{array}{l}

=\\
\wpBcSeq{ \begin{array}{l}
       \compile{\stmt_1}; \\
       \instr{jsr} \ s;\\
       
       \\      
       s: \instr{astore} \ k; \\
       \compile{\stmt_2}; \\ 
       \instr{ret} \ k \\
       

	%	\addExceptionTable{ \stR{\compile{\stmt_1} }}{ \enR{  \compile{ \stmt_1 }  }}{ h }{ Exception} )
   \end{array}
     } {\compile{\normalPost}}{\excPost} 
\end{array}$$

\comment{from the definition  of the weakest precondition \mbox{ \rm \instr{jsr} } instruction  }
$$
\begin{array}{l}
=\\
\wpBcSeq{  \compile{\stmt_1}} { \wpBcSeq{\begin{array}{l}   s: \instr{astore} \ k; \\
                                     \compile{\stmt_2}; \\ 
				     \instr{ret} \ k \\
	         \end{array}
                }{\compile{\normalPost} } {\excPost}  }{\excPost} 
\end{array}$$


\comment{from the definition  of the weakest precondition \mbox{ \rm \instr{ret} } instruction  }
$$
\begin{array}{l}
=\\
\wpBcSeq{  \compile{\stmt_1}} { \wpBcSeq{\begin{array}{l}   s: \instr{astore} \ k; \\
                                     \compile{\stmt_2}; \\ 
			                  \end{array}  }{\compile{\normalPost} } {\excPost}  }{\excPost} 
\end{array}$$

\comment{as the instructions \mbox{ \rm $ s: \instr{astore} \ k; \   \compile{\stmt_2} $} execute sequentially }

 
$$
\begin{array}{l}
=\\
\wpBcSeq{  \compile{\stmt_1}} { \\
    \spaceWpBc  \wpBcSeq{  s: \instr{astore} \ k }{ \\
   \spaceWpBc  \spaceWpBc  \wpBcSeq{\compile{\stmt_2} } { \\
   \spaceWpBc   \spaceWpBc \compile{\normalPost}}{ \\
        \spaceWpBc \spaceWpBc  \excPost} } {\\ 
        \spaceWpBc \excPost}  }{\\
        \excPost} 
\end{array}$$ 

\comment{from the definition  of the weakest precondition \mbox{ \rm \instr{astore} } instruction  }
$$
\begin{array}{l}
=\\
\wpBcSeq{  \compile{\stmt_1} }
        { 
              \wpBcSeq{\compile{\stmt_2} } { 
              \compile{\normalPost}}  { 
              \excPost} \begin{array}{l}
                           \substitution{\counter}{ \counter - 1}\\
		            \substitution{ \locVar{k}}{ \stack{ \counter } } 
			     \end{array} } 		    
			     {  \excPost} 
\end{array}$$ 

\comment{ by induction hypothesis there are no \mbox{ \rm \counter} expressions in \mbox{ \rm $ \wpBcSeq{\compile{\stmt_2} } { \compile{\normalPost}} {\excPost}  $}  }



$$
\begin{array}{l}
=\\
\wpBcSeq{  \compile{\stmt_1} } 
        { \wpBcSeq{\compile{\stmt_2} } { 
              \compile{\normalPost}}  { 
              \excPost} \substitution{ \locVar{k}}{ \stack{ \counter} }} 		    
        {  \excPost} 
\end{array}$$ 

\comment{there \mbox{ \rm $  \locVar{k}  $}  does not appear in the specification, neither is used in \mbox{ \rm $ \stmt_1$} by hypothesis, see Section \ref{compile} }
$$
\begin{array}{l}
=\\
\wpBcSeq{  \compile{\stmt_1} } 
        { \wpBcSeq{\compile{\stmt_2} } { 
              \compile{\normalPost}}  { \excPost}} 		    
        {  \excPost} 
\end{array}$$ 


\item \textbf{throw statement}

\comment{by definition } 

	$$ \begin{array}{l}   
              \wpiSrc{ \throw \ \expressionSrc }{ \normalPostSrc}{\excPostSrc} \\
              = \\
	      \\ \numConclusion{1} \\
	      \wpiSrc{\expressionSrc}{
                  \begin{array}{l}
		         \eval{\expressionSrc} \neq \Mynull \Rightarrow \excPostSrc( \typeof{ \eval{\expressionSrc} }, \expressionSrc ) \\
			 \wedge \\
	                   \eval{\expressionSrc} = \Mynull \Rightarrow \excPostSrc( \mbox{ \rm \texttt{NullPointerExc} }, \expressionSrc ) 
	          \end{array}}   
		     { \excPostSrc} 
	   \end{array}        $$

\comment{by definition of the compiler function }
	$$ \begin{array}{l}   
              \wpBcSeq{\compile{ \throw \ \expressionSrc } }{ \compile{\normalPostSrc}}{\excPost} \\
        = \\
	      
            \begin{array}{l}
	      \wpBcSeq{     \begin{array}{l}
	               \compileExp{\expressionSrc} {\stack{ \counter}} \\
		       \instr{athrow}
		       \end{array}}
				           { \\
	\spaceWpBc	   \compile{ \normalPost}}
		     { \\
        \spaceWpBc \excPost}
		     \end{array} 
	   \end{array}        $$
\comment{by definition of the function \mbox{ \rm $\textrm{wp}^{bc}$ } for  \mbox{ \rm \instr{athrow}}  }

      $$\numConclusion{2}$$
	$$  \begin{array}{l}   
             =  \\
	      \wpBcSeq{    
	               \compileExp{\expressionSrc} {\stack{ \counter}} 
		   
		      }   { \\
	\spaceWpBc \stack{\counter} \neq \Mynull \Rightarrow \\
	\spaceWpBc	\spaceWpBc \excPost(\typeof{\expressionSrc}, \compile{\expressionSrc} ) \\
	\spaceWpBc \wedge
		\spaceWpBc \stack{\counter} = \Mynull \Rightarrow \\
	\spaceWpBc	\spaceWpBc	 \excPost( \mbox{ \rm \texttt{NullPointerExc}} , \compile{\expressionSrc} ) \\
	
	} { \\
        \spaceWpBc \excPost}
		     \end{array}    $$
\comment{by initial hypothesis \mbox{ \rm $ \forall \texttt{Exc} , \expression . \excPostSrc(\texttt{Exc} , \expression ) \eqModNames \excPost(\texttt{Exc} , \compile{\expression} )  $} }

$$\numConclusion{3}$$
$$
\begin{array}{l}
    \begin{array}{l}
		         \eval{\expressionSrc} \neq \Mynull \Rightarrow \excPostSrc( \typeof{ \eval{\expressionSrc} }, \expressionSrc ) \\
			 \wedge
	                   \eval{\expressionSrc} = \Mynull \Rightarrow \excPostSrc( \mbox{ \rm \texttt{NullPointerExc} }, \expressionSrc ) 
	          \end{array} \\
    \eqModNames \\
     \begin{array}{l}
   	\stack{\counter} \neq \Mynull \Rightarrow 	 \excPost(\typeof{\expressionSrc}, \compile{\expressionSrc} ) \\
	 \wedge \\
		 \stack{\counter} = \Mynull \Rightarrow 		 \excPost( \mbox{ \rm \texttt{NullPointerExc}} , \compile{\expressionSrc} )
    \end{array}
\end{array}
$$

\comment{from \numConclusion{3} we can apply the induction hypothesis }


$$\numConclusion{1} \eqModNames \numConclusion{2}$$
\comment{and we can conclude that the hypothesis hold}


 \item  \textbf{ loop statement}\\
      \comment{we name the loop invariant   \invariant } 
   
      $$\wpiSrc{ \while \ (\expressionSrc) \ \do \ \{ \stmt \} } { \psi} {\excPostSrc}  $$
	      \comment{by defintition \ref{wpSrc} }
      $$ \numConclusion{0} $$
$$  
          \begin{array}{l}
	  = \\
	       \invariant \ \wedge\\
	       \forall \  m_i .  i = 1 .. k \\
	       \Myspace \invariant \Rightarrow \\
	 	     \Myspace    \Myspace \wpiSrc{\expressionSrc}{
		                 \begin{array}{l}  		
		                      \Myspace \eval{\expressionSrc } = \Mytrue \Rightarrow \\
				     	\phantom{aaaaa}    \wpiSrc{ \stmt }{\invariant} {\excPostSrc}  \\ 
				       \Myspace \eval{\expressionSrc } = \Myfalse  \Rightarrow  \normalPostSrc
			         \end{array}
	      }{ \excPostSrc} 
	      
        \end{array}
     $$     

	      \comment{by definition \ref{compile} of the compiler from source language to bytecode language  }

	     
	      $$ \numConclusion{1} $$
$$  
	      \begin{array}{l}
               \wpBcSeq{\compile{ \while \ (\expressionSrc) \ \do \ \{ \stmt \} } }{\compile{\psi}}{ \excPostSrc} \\
	      \\
               = \\
              \\
	      \wpBcSeq{ \begin{array}{l}
			   \instr{goto} \ loopEntry; \\
			   loopBody: \compile{\stmt}; \\
			   \compile{\invariant}; \\
			   loopEntry:\compileExp{\expressionSrc}{\stack{\counter}};\\
			   \instr{if\_cond} \ loopBody; 
		       \end{array} }{\compile{\psi}}{ \excPost} 
		\end{array}
		       $$
	      
	      \comment{ \todo{Say why is it like this ?} }
	      $$  \numConclusion{2}$$
	      $$ \begin{array}{l} 
	      = \\
                      \wpBcSeq{\begin{array}{l}
				         \instr{goto}\ loopEntry; \\
					  \compile{\invariant \ }; \\
					   loopEntry:\compileExp{\expressionSrc}{\stack{\counter}};\\
				   \end{array}  }{ \\
        \\
                   \spaceWpBc \begin{array}{l}
					   \stack{\counter} == \compile{ \Mytrue} \Rightarrow \wpBcSeq{ \compile{ \stmt}} { \compile{\invariant \ }}{ \excPost}       \\                                                                                   \wedge \\
                                           \stack{\counter} == \compile{ \Myfalse} \Rightarrow  \compile{\psi}\\
		              \end{array} } { \\
               \\
		 \spaceWpBc \excPost }
		 \end{array}
              $$

	      \comment{from lemma \ref{exprValueOnStack} on page \pageref{exprValueOnStack}  and from \numConclusion{2}}
	      $$\numConclusion{3} $$
$$\begin{array}{l}
= \\	  \wpBcSeq{\begin{array}{l}
				         \instr{goto} \ loopEntry; \\
					  \compile{\invariant \ }; \\
					   loopEntry:\compileExp{\expressionSrc}{\stack{ \counter}};\\
				   \end{array}  }{\\
	\spaceWpBc  \begin{array}{l}
					          \compile{\expressionSrc} == \compile{ \Mytrue} \Rightarrow \wpBcSeq{ \compile{ \stmt}} { \compile{\invariant \ }}{ \excPostSrc} \\
                                                  \wedge \\
                                                   \compile{\expressionSrc} == \compile{ \Myfalse} \Rightarrow  \compile{\psi} \\
						   \wedge 
						   \stack{\counter} = \compile{ \eval{ \expressionSrc}}
							        \end{array} } {\\
         	\spaceWpBc      \excPost }      
		\end{array}
	      $$
	    
	    \comment{by definition of the weakest precondition for loop entry instructions }
	    $$
\begin{array}{l}
= \\	  \wpBcSeq{ \instr{goto} \  loopEntry}{ \\
\spaceWpBc
	    \begin{array}{l}
	       \compile{ \invariant \ } \ \wedge\\
	       \forall \  m_i .  i = 1 .. k \\
	       \compile{  \invariant \ } \Rightarrow \\
	       \begin{array}{l}
	 	     \wpBcSeq{\compileExp{\expressionSrc}{\stack{\counter}}}{ \\
		     \spaceWpBc 
                                                 \begin{array}{l}
					            \compile{\expressionSrc} == \compile{ \Mytrue} \Rightarrow  \\
					  \spaceWpBc	   \wpBcSeq{ \compile{ \stmt}} { \compile{\invariant \ }}{ \excPost} \\
                                                    \wedge \\
                                                    \compile{\expressionSrc} == \compile{ \Myfalse} \Rightarrow \\
					\spaceWpBc	    \compile{\psi}\\
				                 \end{array} }{ \\
						 \spaceWpBc \excPost  }
		\end{array}
	    \end{array}}{\\
\spaceWpBc \excPost}
\end{array}
	    $$
	 \comment{by definition of the weakest precondition for \mbox{ \rm \instr{goto}} instructions  } 

	 

	 $$\numConclusion{4} $$
$$  =  \begin{array}{l}
	       \compile{ \invariant \ }  \ \wedge\\
	       \forall \  m_i .  i = 1 .. k \\
	       \compile{ \invariant \ } \Rightarrow \\
	       \begin{array}{l}
	 	     \wpBcSeq{\compileExp{\expressionSrc}{\stack{\counter}}} { \\ 
                               \spaceWpBc                   \begin{array}{l}
					            \compile{\expressionSrc} == \compile{ \Mytrue} \Rightarrow  \\
					 \spaceWpBc 	   \wpBcSeq{ \compile{ \stmt}} { \compile{\invariant \ }}{ \excPost} \\
                                                    \wedge \\
                                                    \compile{\expressionSrc} == \compile{ \Myfalse} \Rightarrow \\
						  \spaceWpBc    \compile{\psi}\\
				                 \end{array} }{ \\ \spaceWpBc \excPost  }
 \end{array} 
	    \end{array} $$

\comment{from \numConclusion{0} and  \numConclusion{4}  applying the structural induction hypothesis we can conclude }
$$
\begin{array}{l}
      \wpiSrc{ \while \ (\expressionSrc) \ \do \ \{ \stmt \} } { \psi} {\excPostSrc}  = \numConclusion{0} \\
       \\
       \eqModNames\\
       \\
            \wpBcSeq{\compile{  \while \ (\expressionSrc) \ \do \ \{ \stmt \} } }{\compile{\psi}}{ \excPost} = \numConclusion{4} 
\end{array}
$$


  %				& \mid \throw \ \expressionSrc  \\
    %                              & \mid \while \ (\expressionSrc) \ \do \ \{ \stmt \} \\
  %				& \mid \returnSrc \expressionSrc \\
  %				& \mid \returnSrc

  \item \textbf{Return statements}
We consider only the case of a non void return

  \comment{ by definition of the weakest precondition for \mbox{ \rm $ \returnSrc \ \expressionSrc $ }  }
$$ \numConclusion{0}$$
$$
\begin{array}{l}
  \wpiSrc{ \returnSrc \  \expressionSrc }{ \normalPostSrc}{\excPostSrc} \\
 = \\
  \wpiSrc{  \expressionSrc } {\normalPostSrc \substitution{\result }{ \eval{ \expressionSrc } } }  {\excPostSrc}
\end{array}
$$

\comment{by definition of the compiler in Section \ref{compile}}

$$\begin{array}{l}
\wpBcSeq{\compile{\returnSrc \  \expressionSrc }} {\compile{\psi}}{ \excPost } \\
\\
= \\
\\
\wpBcSeq{ \begin{array}{l}
          \compileExp{\expressionSrc }{\stack{\counter}};\\ 
	  \instr{return}
          \end{array}
        }  {\compile{\psi}}{ \excPost} 

\end{array}
$$


\comment{by definition of the weakest predicate transformer function for the \mbox{ \rm \instr{return}}  instruction}

$$
\begin{array}{l}
= \\
\\
\wpBcSeq{  \compileExp{\expressionSrc }{\stack{\counter}} } { \\
\spaceWpBc \compile{\psi}  \substitution{ \result}{\stack{\counter} }}{ \\
                          \spaceWpBc \excPost } 
\end{array}
$$

\comment{from lemma \ref{exprValueOnStack} on page \pageref{exprValueOnStack} }
$$ \numConclusion{1} $$
$$\begin{array}{l}
= \\
\\
\wpBcSeq{\compile{\expressionSrc } } { \begin{array}{l}
                                        \compile{\psi} \substitution{ \result}{\stack{\counter } } \\
					\wedge \\
					\stack{\counter} = \compile{ \eval{ \expressionSrc}}
					\end{array} }{ \excPost} 
\end{array}
$$





	        

$$ \numConclusion{2} $$

$$ \begin{array}{l}
       = \\
       \wpBcSeq{  \compileExp{\expressionSrc }{ \stack{\counter}} } { \\
        \spaceWpBc \compile{\psi}\substitution{ \result}{   \compile{\eval{ \expressionSrc} }  }  }{ \\
         \spaceWpBc \excPost} 
  \end{array}
 $$

\comment{as \mbox{\rm$\compile{\psi} \substitution{ \result}{   \compile{\eval{ \expressionSrc} }  }  $} does  not contain stack expressions, we can apply the induction hypothesis }
$$ \numConclusion{4} $$ 
$$ \begin{array}{l}
        \wpiSrc{  \expressionSrc } {\normalPostSrc \substitution{\result }{ \eval{ \expressionSrc} } }  {\excPostSrc} \\
 \\
       \eqModNames \\
\\   
\begin{array}{l}
    \wpBcSeq{\compileExp{\expressionSrc }{ \stack{\counter}}  } { \\
\spaceWpBc \compile{\psi} \begin{array}{l} 
                                                        \substitution{ \result}{   \compile{\eval{ \expressionSrc} }  } 
                                                    \end{array} }{ \\
\spaceWpBc \excPost}   
\end{array}
  \end{array}
 $$

\comment{from \numConclusion{0} and \numConclusion{3}  and \numConclusion{4} this case holds  }   
$$\begin{array}{l}
      \wpiSrc{ \returnSrc \  \expressionSrc }{ \normalPostSrc}{\excPostSrc} \\
 \\
        \eqModNames \\
\\  
   \wpBcSeq{\compile{ \returnSrc \  \expressionSrc }} {\compile{\psi}}{ \excPost} 
\end{array}
$$
 \item \textbf{Instance Creation expressions}
We will consider only the case when the constructor takes one argument for reasons of readability. The general case is straightforward.

\comment{from the definition of the weakest precondition for instance creation in the source language in section \ref{wpSrc}}
  
$$ \numConclusion{1} $$
$$ \begin{array}{l} 
    \wpiSrc{ \newSrc \ Class  ( \expressionSrc   ) } {\normalPostSrc }{ \excPostSrc  }\\
   = \\

	\wpiSrc{   \expressionSrc }{ \\
               \spaceWpSrc     \pre(ConsClass)  \left\{    
                                               \begin{array}{l} 
                                                      \substitution{ \this}{ \Ref{Class} } \\
                                                       \substitution{ arg_1}{ \eval{\expressionSrc}} 
					     	\end{array} \right. \\
                           \spaceWpSrc        \wedge \\
                           \spaceWpSrc   \forall \  mod_i .  i = 1 .. n  \\
                           \spaceWpSrc  \left\{  \begin{array}{l}
                        
                                \normalPostSrc(ConsClass)  \left\{    
                                          \begin{array}{l} 
					        \substitution{ \this}{ \Ref{Class}} \\
                                                \substitution{ arg_1}{ \eval{\expressionSrc}} \\
					    \end{array} \right.  \Rightarrow   \normalPostSrc \begin{array}{l} 
                             \end{array}  \\
                     
				   %\spaceWpSrc \spaceWpSrc \spaceWpSrc \phantom{\ldots } 	
                                   \wedge \\   
				   %\spaceWpSrc \spaceWpSrc \spaceWpSrc \phantom{\ldots } 
                       	\excPostSpec_{ConsClass}( \tt{Exc_1} ) \Rightarrow
                                    %\spaceWpSrc     \spaceWpSrc \spaceWpSrc \spaceWpSrc \phantom{\ldots }    
                              \excPostSrc(\tt{Exc_1}, new Class ( \expressionSrc   ))  \\
				   %\spaceWpSrc \spaceWpSrc \spaceWpSrc \phantom{\ldots } 
                                 	 \wedge \\ 
				 %  \spaceWpSrc \spaceWpSrc \spaceWpSrc \phantom{\ldots } 
                          	\ldots \\
				%   \spaceWpSrc \spaceWpSrc \spaceWpSrc \phantom{\ldots } 	
                                  \wedge \\ 
				  % \spaceWpSrc \spaceWpSrc \spaceWpSrc \phantom{\ldots } 	
                                \excPostSpec_{ConsClass}( \tt{Exc_s} ) \Rightarrow 
                            	%\spaceWpSrc \spaceWpSrc  	\spaceWpSrc   \spaceWpSrc \phantom{\ldots }     
                              \excPostSrc(\tt{Exc_s}, new Class( \expressionSrc ))   
                   \end{array} \right.
           } {\\
       \spaceWpSrc     \excPostSrc}   \substitution{\heap}{\heap \oplus[ \Ref{Class} \rightarrow \objCl{Class} ] }  
	            \end{array}
	          $$


\comment{by definition of the compiler function}

$$
\begin{array}{l}
  \wpBcSeq{ \compile{ \newSrc \ Class  ( \expressionSrc   )} } { \compile{ \normalPostSrc} }{ \excPost  }\\
 = \\
  \wpBcSeq{  \begin{array}{l}
                       \instr{new} \ Class; \\ 
		       \instr{dup}; \\
		       \compileExp{\expressionSrc}{\stack{\counter - k}};\\
		          \instr{invoke } ConsClass;     
	       \end{array} } { \compile{ \normalPostSrc} }{ \excPost  }
\end{array}
$$

\comment{apply the rule for method invocation}
$$
\begin{array}{l}
  
 = \\

  \wpBcSeq{  \begin{array}{l}
 \instr{new} \ Class; \\ 
		       \instr{dup}; \\
		       \compileExp{\expressionSrc}{\stack{\counter - k}};\\
	            
	       \end{array} } { \\       

        \spaceWpSrc    \compileSynt{ \pre(ConsClass)}  \left\{    
                                               \begin{array}{l} 
                                                      \substitution{ \compileSynt{\this}}{ \stack{\counter - 1} } \\
                                                       \substitution{\compileSynt{ arg_1}}{ \stack{\counter}} 
					     	\end{array} \right. \\
                           \spaceWpSrc        \wedge \\
                           \spaceWpSrc   \forall \  mod_i .  i = 1 .. n  \\
                           \spaceWpSrc  \left\{  \begin{array}{l}
                        
                       \compileSynt{ \normalPostSrc(ConsClass)}  \left\{    
                                          \begin{array}{l} 
                                                         \substitution{ \compileSynt{\this}}{ \stack{\counter - 1} } \\
                                                       \substitution{\compileSynt{ arg_1}}{ \stack{\counter}}   
					    \end{array} \right.  \Rightarrow   \normalPostSrc \begin{array}{l} 
                             \end{array}  \\
                     
				   %\spaceWpSrc \spaceWpSrc \spaceWpSrc \phantom{\ldots } 	
                                   \wedge \\   
				   %\spaceWpSrc \spaceWpSrc \spaceWpSrc \phantom{\ldots } 
                       	\excPostSpec_{ConsClass}( \tt{Exc_1} ) \Rightarrow
                                    %\spaceWpSrc     \spaceWpSrc \spaceWpSrc \spaceWpSrc \phantom{\ldots }    
                              \excPostSrc(\tt{Exc_1}, new Class ( \expressionSrc   ))  \\
				   %\spaceWpSrc \spaceWpSrc \spaceWpSrc \phantom{\ldots } 
                                 	 \wedge \\ 
				 %  \spaceWpSrc \spaceWpSrc \spaceWpSrc \phantom{\ldots } 
                          	\ldots \\
				%   \spaceWpSrc \spaceWpSrc \spaceWpSrc \phantom{\ldots } 	
                                  \wedge \\ 
				  % \spaceWpSrc \spaceWpSrc \spaceWpSrc \phantom{\ldots } 	
                                \excPostSpec_{ConsClass}( \tt{Exc_s} ) \Rightarrow 
                            	%\spaceWpSrc \spaceWpSrc  	\spaceWpSrc   \spaceWpSrc \phantom{\ldots }     
                              \excPostSrc(\tt{Exc_s}, new Class( \expressionSrc ))   
                   \end{array} \right. }{ \\
         \spaceWpSrc     \excPost  }\\
\end{array}
$$

\comment{applying the weakest precondition rule for a sequential list of instructions }

$$
\begin{array}{l}
  
 = \\

  \wpBcSeq{  \begin{array}{l}
                     \instr{new} \ Class; \\ 
		       \instr{dup}; \\
	     \end{array} } { \\       
     \spaceWpSrc  \wpBcSeq{ \compileExp{\expressionSrc}{\stack{\counter - k}}} { \\
          \spaceWpSrc   \spaceWpSrc    \compileSynt{ \pre(ConsClass)}  \left\{    
                                               \begin{array}{l} 
                                                      \substitution{ \compileSynt{\this}}{ \stack{\counter - 1} } \\
                                                       \substitution{\compileSynt{ arg_1}}{ \stack{\counter}} 
					     	\end{array} \right. \\
                            \spaceWpSrc     \spaceWpSrc        \wedge \\
                            \spaceWpSrc     \spaceWpSrc   \forall \  mod_i .  i = 1 .. n  \\
                             \spaceWpSrc   \spaceWpSrc  \left\{  \begin{array}{l}
                        
                               \compileSynt{ \normalPostSrc(ConsClass)}  \left\{    
                                          \begin{array}{l} 
                                                         \substitution{ \compileSynt{\this}}{ \stack{\counter - 1} } \\
                                                       \substitution{\compileSynt{ arg_1}}{ \stack{\counter}}   
					    \end{array} \right.  \Rightarrow   \normalPostSrc \begin{array}{l} 
                             \end{array}  \\
                     
				   %\spaceWpSrc \spaceWpSrc \spaceWpSrc \phantom{\ldots } 	
                                   \wedge \\   
				   %\spaceWpSrc \spaceWpSrc \spaceWpSrc \phantom{\ldots } 
                       	\excPostSpec_{ConsClass}( \tt{Exc_1} ) \Rightarrow
                                    %\spaceWpSrc     \spaceWpSrc \spaceWpSrc \spaceWpSrc \phantom{\ldots }    
                              \excPostSrc(\tt{Exc_1}, new Class ( \expressionSrc   ))  \\
				   %\spaceWpSrc \spaceWpSrc \spaceWpSrc \phantom{\ldots } 
                                 	 \wedge \\ 
				 %  \spaceWpSrc \spaceWpSrc \spaceWpSrc \phantom{\ldots } 
                          	\ldots \\
				%   \spaceWpSrc \spaceWpSrc \spaceWpSrc \phantom{\ldots } 	
                                  \wedge \\ 
				  % \spaceWpSrc \spaceWpSrc \spaceWpSrc \phantom{\ldots } 	
                                \excPostSpec_{ConsClass}( \tt{Exc_s} ) \Rightarrow 
                            	%\spaceWpSrc \spaceWpSrc  	\spaceWpSrc   \spaceWpSrc \phantom{\ldots }     
                              \excPostSrc(\tt{Exc_s}, new Class( \expressionSrc ))   
                   \end{array} \right. }{ \\    
                   \spaceWpSrc    \spaceWpSrc \excPost   }} { \\
         \spaceWpSrc     \excPost  }\\
\end{array}
$$

\comment{applying lemmas \ref{exprValueOnStack} on page \pageref{exprValueOnStack}, \ref{dupProp} on \pageref{dupProp}   }
$$
\begin{array}{l}
  
 = \\

  \wpBcSeq{  \begin{array}{l}
                     \instr{new} \ Class; \\ 
		       \instr{dup}; \\
	     \end{array} } { \\       
     \spaceWpSrc  \wpBcSeq{ \compileExp{\expressionSrc}{\stack{\counter - k}}}{ \\
          \spaceWpSrc   \spaceWpSrc       \compileSynt{ \pre(ConsClass)}  \left\{    
                                               \begin{array}{l} 
                                                      \substitution{ \compileSynt{\this}}{ \stack{\counter - 1} } \\
                                                       \substitution{\compileSynt{ arg_1}}{ \stack{\counter}} 
					     	\end{array} \right. \\
                            \spaceWpSrc     \spaceWpSrc        \wedge \\
                            \spaceWpSrc     \spaceWpSrc   \forall \  mod_i .  i = 1 .. n  \\
                             \spaceWpSrc   \spaceWpSrc  \left\{  \begin{array}{l}
                        
                                 \compileSynt{  \normalPostSrc(ConsClass)}  \left\{    
                                          \begin{array}{l} 
                                                         \substitution{ \compileSynt{\this}}{ \stack{\counter - 1} } \\
                                                       \substitution{\compileSynt{ arg_1}}{ \stack{\counter}}   
					    \end{array} \right.  \Rightarrow   \normalPostSrc \begin{array}{l} 
                             \end{array}  \\
                     
				   %\spaceWpSrc \spaceWpSrc \spaceWpSrc \phantom{\ldots } 	
                                   \wedge \\   
				   %\spaceWpSrc \spaceWpSrc \spaceWpSrc \phantom{\ldots } 
                       	\excPostSpec_{ConsClass}( \tt{Exc_1} ) \Rightarrow
                                    %\spaceWpSrc     \spaceWpSrc \spaceWpSrc \spaceWpSrc \phantom{\ldots }    
                              \excPostSrc(\tt{Exc_1}, new Class ( \expressionSrc   ))  \\
				   %\spaceWpSrc \spaceWpSrc \spaceWpSrc \phantom{\ldots } 
                                 	 \wedge \\ 
				 %  \spaceWpSrc \spaceWpSrc \spaceWpSrc \phantom{\ldots } 
                          	\ldots \\
				%   \spaceWpSrc \spaceWpSrc \spaceWpSrc \phantom{\ldots } 	
                                  \wedge \\ 
				  % \spaceWpSrc \spaceWpSrc \spaceWpSrc \phantom{\ldots } 	
                                \excPostSpec_{ConsClass}( \tt{Exc_s} ) \Rightarrow 
                            	%\spaceWpSrc \spaceWpSrc  	\spaceWpSrc   \spaceWpSrc \phantom{\ldots }     
                              \excPostSrc(\tt{Exc_s}, new Class( \expressionSrc ))   
                   \end{array} \right. \\
                   \spaceWpSrc   \spaceWpSrc \wedge \\
		     \spaceWpSrc   \spaceWpSrc \stack{\counter} = \compile{ \eval{ \expressionSrc}}     }{ \\    
                   \spaceWpSrc    \spaceWpSrc \excPost   } \\ 
		   \spaceWpSrc  \wedge\\
                   \spaceWpSrc \stack{\counter} = \stack{\counter-1} } { \\
         \spaceWpSrc     \excPost  }\\
\end{array}
$$

\comment{applying the lemma \ref{stackCounter} on page \pageref{stackCounter} }


$$
\begin{array}{l}
  
 = \\

  \wpBcSeq{  \begin{array}{l}
                     \instr{new} \ Class; \\ 
		       \instr{dup}; \\
	     \end{array} } { \\       
     \spaceWpSrc  \wpBcSeq{ \compileExp{\expressionSrc}{\stack{\counter - k}}}{ \\
          \spaceWpSrc   \spaceWpSrc       \compileSynt{ \pre(ConsClass)}  \left\{    
                                               \begin{array}{l} 
                                                      \substitution{ \compileSynt{\this}}{ \stack{\counter - 1} } \\
                                                       \substitution{\compileSynt{ arg_1}}{ \stack{\counter}} 
					     	\end{array} \right. \\
                            \spaceWpSrc     \spaceWpSrc        \wedge \\
                            \spaceWpSrc     \spaceWpSrc   \forall \  mod_i .  i = 1 .. n  \\
                             \spaceWpSrc   \spaceWpSrc  \left\{  \begin{array}{l}
                        
                                 \compileSynt{  \normalPostSrc(ConsClass)}  \left\{    
                                          \begin{array}{l} 
                                                       \substitution{ \compileSynt{\this}}{ \stack{\counter - 1} } \\
                                                       \substitution{\compileSynt{ arg_1}}{ \stack{\counter}}   
					    \end{array} \right.  \Rightarrow   \normalPostSrc \begin{array}{l} 
                             \end{array}  \\
                     
				   %\spaceWpSrc \spaceWpSrc \spaceWpSrc \phantom{\ldots } 	
                                   \wedge \\   
				   %\spaceWpSrc \spaceWpSrc \spaceWpSrc \phantom{\ldots } 
                       	\excPostSpec_{ConsClass}( \tt{Exc_1} ) \Rightarrow
                                    %\spaceWpSrc     \spaceWpSrc \spaceWpSrc \spaceWpSrc \phantom{\ldots }    
                              \excPostSrc(\tt{Exc_1}, new Class ( \expressionSrc   ))  \\
				   %\spaceWpSrc \spaceWpSrc \spaceWpSrc \phantom{\ldots } 
                                 	 \wedge \\ 
				 %  \spaceWpSrc \spaceWpSrc \spaceWpSrc \phantom{\ldots } 
                          	\ldots \\
				%   \spaceWpSrc \spaceWpSrc \spaceWpSrc \phantom{\ldots } 	
                                  \wedge \\ 
				  % \spaceWpSrc \spaceWpSrc \spaceWpSrc \phantom{\ldots } 	
                                \excPostSpec_{ConsClass}( \tt{Exc_s} ) \Rightarrow 
                            	%\spaceWpSrc \spaceWpSrc  	\spaceWpSrc   \spaceWpSrc \phantom{\ldots }     
                              \excPostSrc(\tt{Exc_s}, new Class( \expressionSrc ))   
                   \end{array} \right. \\
                   \spaceWpSrc   \spaceWpSrc \wedge \\
		     \spaceWpSrc   \spaceWpSrc \stack{\counter} = \compile{ \eval{ \expressionSrc}} \\
                       \spaceWpSrc  \spaceWpSrc  \wedge\\
                      \spaceWpSrc  \spaceWpSrc \stack{\counter-1} = \stack{\counter-2}     }{ \\    
                   \spaceWpSrc    \spaceWpSrc \excPost   } \\ 
		   } { \\
         \spaceWpSrc     \excPost  }\\
\end{array}
$$

\comment{apply the weakest precondition calculus for sequential instructions and lemma \ref{newProp} on page \pageref{newProp}}
 $$
\begin{array}{l}
   = \\
 \wpBcSeq{ \instr{new} \ Class  } { \\      
	    \spaceWpSrc       \wpBcSeq{\instr{dup}; } { \\ 
     \spaceWpSrc    \spaceWpSrc  \wpBcSeq{ \compileExp{\expressionSrc}{\stack{\counter - k}}} {  \\
      \spaceWpSrc        \spaceWpSrc   \spaceWpSrc       \compileSynt{ \pre(ConsClass)}  \left\{    
                                               \begin{array}{l} 
                                                      \substitution{ \compileSynt{\this}}{ \stack{\counter - 1} } \\
                                                       \substitution{\compileSynt{ arg_1}}{ \stack{\counter}} 
					     	\end{array} \right. \\
                             \spaceWpSrc   \spaceWpSrc     \spaceWpSrc        \wedge \\
                            \spaceWpSrc    \spaceWpSrc     \spaceWpSrc   \forall \  mod_i .  i = 1 .. n  \\
                            \spaceWpSrc     \spaceWpSrc   \spaceWpSrc  \left\{  \begin{array}{l}
                        
                                 \compileSynt{  \normalPostSrc(ConsClass)}  \left\{    
                                          \begin{array}{l} 
                                                       \substitution{ \compileSynt{\this}}{ \stack{\counter - 1} } \\
                                                       \substitution{\compileSynt{ arg_1}}{ \stack{\counter}}   
					    \end{array} \right.  \Rightarrow   \normalPostSrc \begin{array}{l} 
                             \end{array}  \\
                     
				   %\spaceWpSrc \spaceWpSrc \spaceWpSrc \phantom{\ldots } 	
                                   \wedge \\   
				   %\spaceWpSrc \spaceWpSrc \spaceWpSrc \phantom{\ldots } 
                       	\excPostSpec_{ConsClass}( \tt{Exc_1} ) \Rightarrow
                                    %\spaceWpSrc     \spaceWpSrc \spaceWpSrc \spaceWpSrc \phantom{\ldots }    
                              \excPostSrc(\tt{Exc_1}, new Class ( \expressionSrc   ))  \\
				   %\spaceWpSrc \spaceWpSrc \spaceWpSrc \phantom{\ldots } 
                                 	 \wedge \\ 
				 %  \spaceWpSrc \spaceWpSrc \spaceWpSrc \phantom{\ldots } 
                          	\ldots \\
				%   \spaceWpSrc \spaceWpSrc \spaceWpSrc \phantom{\ldots } 	
                                  \wedge \\ 
				  % \spaceWpSrc \spaceWpSrc \spaceWpSrc \phantom{\ldots } 	
                                \excPostSpec_{ConsClass}( \tt{Exc_s} ) \Rightarrow 
                            	%\spaceWpSrc \spaceWpSrc  	\spaceWpSrc   \spaceWpSrc \phantom{\ldots }     
                              \excPostSrc(\tt{Exc_s}, new Class( \expressionSrc ))   
                   \end{array} \right. \\
                    \spaceWpSrc   \spaceWpSrc   \spaceWpSrc \wedge \\
		      \spaceWpSrc   \spaceWpSrc   \spaceWpSrc \stack{\counter} = \compile{ \eval{ \expressionSrc}} \\
                      \spaceWpSrc     \spaceWpSrc  \spaceWpSrc  \wedge\\
               \spaceWpSrc           \spaceWpSrc  \spaceWpSrc \stack{\counter-1} = \stack{\counter-2}     }{ \\    
                  \spaceWpSrc     \spaceWpSrc    \spaceWpSrc \excPost   } \\ 
		   }  { \\
         \spaceWpSrc    \spaceWpSrc     \excPost  } \\ 
	  \spaceWpSrc   \wedge \\
          \spaceWpSrc    \stack{\counter} = \Ref{Class}  } { \\
          \spaceWpSrc     \excPost   } 
\end{array}$$

\comment{applying twice the lemma \ref{stackCounter} on page \pageref{stackCounter}}
$$
\begin{array}{l}
   = \\
 \wpBcSeq{ \instr{new} \ Class  } { \\      
	    \spaceWpSrc       \wpBcSeq{\instr{dup}; } { \\ 
     \spaceWpSrc    \spaceWpSrc  \wpBcSeq{ \compileExp{\expressionSrc}{\stack{\counter - k}}} {  \\
      \spaceWpSrc        \spaceWpSrc   \spaceWpSrc       \compileSynt{ \pre(ConsClass)}  \left\{    
                                               \begin{array}{l} 
                                                      \substitution{ \compileSynt{\this}}{ \stack{\counter - 1} } \\
                                                       \substitution{\compileSynt{ arg_1}}{ \stack{\counter}} 
					     	\end{array} \right. \\
                             \spaceWpSrc   \spaceWpSrc     \spaceWpSrc        \wedge \\
                            \spaceWpSrc    \spaceWpSrc     \spaceWpSrc   \forall \  mod_i .  i = 1 .. n  \\
                            \spaceWpSrc     \spaceWpSrc   \spaceWpSrc  \left\{  \begin{array}{l}
                        
                                 \compileSynt{  \normalPostSrc(ConsClass)}  \left\{    
                                          \begin{array}{l} 
                                                       \substitution{ \compileSynt{\this}}{ \stack{\counter - 1} } \\
                                                       \substitution{\compileSynt{ arg_1}}{ \stack{\counter}}   
					    \end{array} \right.  \Rightarrow   \normalPostSrc \begin{array}{l} 
                             \end{array}  \\
                     
				   %\spaceWpSrc \spaceWpSrc \spaceWpSrc \phantom{\ldots } 	
                                   \wedge \\   
				   %\spaceWpSrc \spaceWpSrc \spaceWpSrc \phantom{\ldots } 
                       	\excPostSpec_{ConsClass}( \tt{Exc_1} ) \Rightarrow
                                    %\spaceWpSrc     \spaceWpSrc \spaceWpSrc \spaceWpSrc \phantom{\ldots }    
                              \excPostSrc(\tt{Exc_1}, new Class ( \expressionSrc   ))  \\
				   %\spaceWpSrc \spaceWpSrc \spaceWpSrc \phantom{\ldots } 
                                 	 \wedge \\ 
				 %  \spaceWpSrc \spaceWpSrc \spaceWpSrc \phantom{\ldots } 
                          	\ldots \\
				%   \spaceWpSrc \spaceWpSrc \spaceWpSrc \phantom{\ldots } 	
                                  \wedge \\ 
				  % \spaceWpSrc \spaceWpSrc \spaceWpSrc \phantom{\ldots } 	
                                \excPostSpec_{ConsClass}( \tt{Exc_s} ) \Rightarrow 
                            	%\spaceWpSrc \spaceWpSrc  	\spaceWpSrc   \spaceWpSrc \phantom{\ldots }     
                              \excPostSrc(\tt{Exc_s}, new Class( \expressionSrc ))   
                   \end{array} \right. \\
                    \spaceWpSrc   \spaceWpSrc   \spaceWpSrc \wedge \\
		      \spaceWpSrc  \spaceWpSrc   \spaceWpSrc \stack{\counter} = \compile{ \eval{ \expressionSrc}} \\
                      \spaceWpSrc  \spaceWpSrc  \spaceWpSrc  \wedge\\
                      \spaceWpSrc   \spaceWpSrc  \spaceWpSrc \stack{\counter-1} = \stack{\counter-2}   \\
                 \spaceWpSrc   \spaceWpSrc  \spaceWpSrc   \wedge \\
           \spaceWpSrc   \spaceWpSrc  \spaceWpSrc    \stack{\counter - 2} = \Ref{Class}   }{ \\    
                  \spaceWpSrc     \spaceWpSrc    \spaceWpSrc \excPost   } \\ 
		   }  { \\
         \spaceWpSrc    \spaceWpSrc     \excPost  }  
	   }{ \\
          \spaceWpSrc     \excPost   } 
\end{array}
$$


$$ \numConclusion{2} $$
$$
\begin{array}{l} 
= \\
\\
\wpBcSeq{ \instr{new} \ Class  } { \\      
	    \spaceWpSrc       \wpBcSeq{\instr{dup}; } { \\ 
     \spaceWpSrc    \spaceWpSrc  \wpBcSeq{ \compileExp{\expressionSrc}{\stack{\counter - k}}} {  \\
      \spaceWpSrc        \spaceWpSrc   \spaceWpSrc       \compileSynt{ \pre(ConsClass)}  \left\{    
                                               \begin{array}{l} 
                                                      \substitution{ \compileSynt{\this}}{ \Ref{Class}  } \\
                                                       \substitution{\compileSynt{ arg_1}}{ \compile{ \eval{ \expressionSrc}} } 
					     	\end{array} \right. \\
                             \spaceWpSrc   \spaceWpSrc     \spaceWpSrc        \wedge \\
                            \spaceWpSrc    \spaceWpSrc     \spaceWpSrc   \forall \  mod_i .  i = 1 .. n  \\
                            \spaceWpSrc     \spaceWpSrc   \spaceWpSrc  \left\{  \begin{array}{l}
                        
                                 \compileSynt{  \normalPostSrc(ConsClass)}  \left\{    
                                          \begin{array}{l} 
                                                       \substitution{ \compileSynt{\this}}{ \Ref{Class}  } \\
                                                       \substitution{\compileSynt{ arg_1}}{\compile{ \eval{ \expressionSrc}}}   
					    \end{array} \right.  \Rightarrow   \normalPostSrc \begin{array}{l} 
                             \end{array}  \\
                     
				   %\spaceWpSrc \spaceWpSrc \spaceWpSrc \phantom{\ldots } 	
                                   \wedge \\   
				   %\spaceWpSrc \spaceWpSrc \spaceWpSrc \phantom{\ldots } 
                       	\excPostSpec_{ConsClass}( \tt{Exc_1} ) \Rightarrow
                                    %\spaceWpSrc     \spaceWpSrc \spaceWpSrc \spaceWpSrc \phantom{\ldots }    
                              \excPostSrc(\tt{Exc_1}, new Class ( \expressionSrc   ))  \\
				   %\spaceWpSrc \spaceWpSrc \spaceWpSrc \phantom{\ldots } 
                                 	 \wedge \\ 
				 %  \spaceWpSrc \spaceWpSrc \spaceWpSrc \phantom{\ldots } 
                          	\ldots \\
				%   \spaceWpSrc \spaceWpSrc \spaceWpSrc \phantom{\ldots } 	
                                  \wedge \\ 
				  % \spaceWpSrc \spaceWpSrc \spaceWpSrc \phantom{\ldots } 	
                                \excPostSpec_{ConsClass}( \tt{Exc_s} ) \Rightarrow 
                            	%\spaceWpSrc \spaceWpSrc  	\spaceWpSrc   \spaceWpSrc \phantom{\ldots }     
                              \excPostSrc(\tt{Exc_s}, new Class( \expressionSrc ))   
                   \end{array} \right. \\
                    }{ \\    
                  \spaceWpSrc     \spaceWpSrc    \spaceWpSrc \excPost   } \\ 
	         } { \\ \spaceWpSrc    \spaceWpSrc     \excPost  }  
	 }{ \\
         \spaceWpSrc     \excPost   }
\end{array}
$$

\comment{ applying the induction hypothesis we can conclude that the proposition holds for this case } 

 
 \end{description}
\end{description}

\end{pogEquiv}



The previous lemme gives us the relation between the precondition of a source statement $\stmt$ and its
compilation $\compile{\stmt} $ given that the precondition does not contain any stack expressions. Still, 
our initial definition of $\rm{wp^{bc}}$
 (see the discussion in Section \ref{bcwp}) works with implicite postconditions ( the function $inter$ defined in \cite{P05BSV})
 which depend on the instructions executed before.
In order to establish that the initial definition that we gave of the weakest predicate transformer preserves the proof obligations 
w.r.t. to the source language  we have to establish first the following property:


\todo{remind the definition of $inter$} 
\begin{postProp}

For any method ~\method \  with precondition Pre and postcondition Post,
 for any statement $\stmt in \ \body{\method}$
the function $\inter{ last{\compile{\stmt} } }{ \next{ last{\compile{\stmt}} }} \in \formulaBcNoStack$ 

\end{postProp}
The proof is by structural induction over the structure of a source statement.
 This lemme means that the postconditions determined by the function $inter$ for the compilation
 $\compile{\expressionSrc}$ of an expression $\expressionSrc$ is in $\formulaBcNoStack$.  
From property \ref{formulaCompProp2} on page \pageref{formulaCompProp2} we conclude that for any statement
$\expression$ there exists a formula
$\psi \in \formulaSrc$ such that  $\inter{ last{\compile{\stmt} } }{ \next{ last{\compile{\stmt}} }} = \compile{\psi} $.

Thus, we can conclude that:
$$ 
\begin{array}{l}
\forall \stmt . \exists \ \psi \ \in \formulaSrc \\
\inter{ last{\compile{\stmt} } }{ \next{ last{\compile{\stmt}} }} = \compile{\psi} \\
\wedge \\

\Myspace   \wpiSrc{\stmt}{ \psi }{\excPost} \\
\Myspace \eqModNames \\
\Myspace \wpBcSeq{\compile{ \stmt} } {\inter{ last{\compile{\stmt} } }{ \next{ last{\compile{\stmt}} }}  }{\excPost}
\end{array}$$



This establishes the equivalence of the preconditions modulo names over source and bytecode programs.
