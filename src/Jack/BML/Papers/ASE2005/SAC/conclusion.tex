\section{Conclusion and Future Work}\label{conclusion}
This article describes the bytecode specification language BCSL, a compiler from the JML language to BCSL and a bytecode \wpi \ calculus.
A proof obligation generator based on the wp calculus and a JML compiler to BCSL have been implemented and are part of the Jack 1.8 release\footnote{http://www-sop.inria.fr/everest/soft/Jack/jack.html}. Those components have been applied to Java to native code optimization of realistic examples \cite{CPG06LFN}.  

% The validation can be done at source or at bytecode level in a common environment: for instance, to prove lemmas ensuring bytecode correctness all the current and future provers plugged in Jack can be used.

 We conclude with future work directions. 
Currently, we have a framework for Java program verification which is the first step towards a PCC infrastructure.
 First, we aim to establish formally that non optimizing source compilation
preserves proof obligations modulo names and basic types. This equivalence can
 be used to complete the PCC framework where the proof certificates will be made interactively on source level. Second, we would like to perform more real case studies. 
Finally, we are interested in 
the extension of the framework applying previous research results in automated annotation generation for Java source programs
 (see~\cite{PBBHL}). The client thus will establish that the code respects his security policy %have the possibility to verify a security policy by
by first propagating automatically annotations in the loaded code and then verifying the resulting annotated code.

% \begin{itemize}
%   \item perform case studies
%   \item find an appropriate representation of code certificates for the PCC architecture described in Section \ref{architecture}
%   where the proofs are done interactively over the source code. We aim to establish formally that nonoptimizing source compilation
% preserves proof obligations modulo names and basic types as discussed in the previous Section \ref{pogEquiv}.
%    \item an extension of the framework applying previous research results in automated annotation generation for
%  Java bytecode (see~\cite{PBBHL}). The client thus will establish that the code respects his security policy %have the possibility to verify a security policy by
% by first propagating automatically annotations in the loaded code and then verifying the resulting annotated code.
% \end{itemize}




