
\section{Compiling JML specifications into BML specifications}
Since it is often easier and more intuitive to write specifications at
source coude level, we have defined a compiler from JML to BML
\JMLtoBML. BML is designed to be very close to JML, so that the
correspondence between the original and the compiled specification
remains relatively clear. Notice that in principle, the same can be
done for the proofs,
\emph{i.e.}\ a source code level proof can be transformed into a
bytecode level proof. It is future work to define this in full detail,
but some work in this direction has already been done, see
\emph{e.g.}~\cite{BartheRS05}. 

The compilation of the JML specification is separated from the
compilation of the Java source code. In fact, \JMLtoBML takes as input
an annotated Java source file \emph{and} the Java class file produced
by a non optimising compiler with the debug flag set. The debug
information that is generated helps us to compile the annotations
correctly. 

From the debug information, we use in particular the
\textbf{Line\_Number\_Table} and the \textbf{Local\_Variable\_Table}
attributes. The presence of these attribute is
optional~\cite{JVMspec}, but almost all standard non-optimising
compilers can generate this data. The \textbf{Line\_Number\_Table} is
computed as part of the compilation of a method; it links line numbers
in the Java source code and of the Java bytecode instructions.  The
\textbf{Local\_Variable\_Table} describes the local variables that
appear in a method.  

To be able to appropriately compile loop invariants, the control flow
graph corresponding to the list of bytecode instructions resulting
from the compilation of a method body must be a
\emph{reducible control flow graph}\footnote{The same restriction
applies if one wishes to define an efficient verification condition
generator for BML.}. This means basically that every cycle in the
graph must have exactly one entry point, and it is not possible to
jump into the middle of a cycle from outside the cycle
(see~\cite{AhoSU86} for the full definition of reducibility). Note
that this is not a serious restriction; all non-optomising Java
compilers produce reducible control flow graphs and in practice even
most hand-written bytecode is reducible.

The compilation from JML specifications into BML compilations is
defined in several steps. As mentioned above, we assume that the Java
source code has been compiled with the debug flag set, and that we
have access to the generated class file.

\begin{description}

\item[Compilation of ghost and model field declarations] This step
adds entries in the constant pool for all ghost and model variables
declared in the specification. The \textbf{Ghost\unsc Field \unsc
attribute} defined above is used for this. 

\item [Desugaring of the JML specification] This is an optional step,
to achieve more compact specifications directly. Here one would use
the standard JML procedure for desugaring~\cite{RaghavanL00}. This
desugaring can also be applied later on the BML specification directly.

\item[Linking and resolving of source data structures]
In this stage, the JML specification is transformed into an
intermediate format, where the identifiers are resolved to their
corresponding data structures in the class file.  The Java and JML
source identifiers are linked with their identifiers on bytecode
level, \emph{i.e.}\ the corresponding indexes either from the constant
pool or from the \textbf{Local\_Variable\_Table} attribute. This is
similar to the linking and resolving stage of the Java source code
compiler. 

\item[Locating instructions for annotation statements] 
Annotation statements, like loop specifications and asserts, specify
assertions that should hold at particular points in the program. The
compiler associates this with the appropriate point in the bytecode
program, using the \textbf{Line\_Number\_Table} attribute.

However, a problem is that a source line may correspond to more than
one instruction in the \textbf{Line\_Number\_Table}. This makes it
complicated to identify the exact loop entry instruction in the
bytecode, and thus to know to which instruction the compiled loop
specification should be associated. 
 
To solve this problem, we use the following heuristics: if the control
flow graph of the bytecode is reducible and if we search from an index
in the \textbf{Line\_Number\_Table} that corresponds to the first line
of a source loop, then the first loop entry instruction found will be
the loop entry corresponding to this source loop.  We do not have a
formal correctness proof for this algorithm, because it depends on the
particular implementation of the compiler.  However, our experiments
show that the heuristic works successfully for Sun's non-optimising
Java compiler.
 
\item[Compilation of JML predicates]
JML predicates are Java boolean expressions. However, the JVM does not
provide direct support for several integral types, such as byte,
short, char, nor for booleans. Instead, they are encoded as integers.
Thus, as mentioned above, the compiler wraps up the boolean
expressions in the JML specification are wrapped by a conditional
function, returning 1 if the predicate is true, 0 otherwise.

\item[Generation of user-specific class attributes]
Finally, the complete specification is compiled into appropriate
user-specific attributes, using the format defined in the previous
section. 
    
\end{description}

We have implemented a first prototype of the \JMLtoBML compiler. We
also have implementations of verification condition generators for
Java source code and bytecode (as part of the tool
JACK~\cite{BurdyRL03}). We have shown that if we compile an
application with its specification, there exists a correspondence
between the proof obligations generated at source and at bytecode
level, modulo differences in elimination of trivial goals, handling of
arithmetic expressions, and the naming convention of generated
variables. Moreover, when the proofs are done with the interactive
theorem prover Coq, the prover generates different names for
hypotheses at source code and bytecode level. It is future work to
clean up to compilation, so that there is a one-to-one correspondence.

