
\section{Compiling JML specifications into BML specifications}
Requirements on the class file format
\begin{itemize}
  \item \textbf{Debug Information} \\ A requirement to the class file
  format is that it must contain a debug information, more
  particularly the \textbf{Line\_Number\_Table} \\ and
  \textbf{Local\_Variable\_Table} attributes. The presence in the Java
  class file format of these attribute is optional \cite{VMSpec}, yet
  almost all standard non optimizing compilers can generate these
  data. The \textbf{Line\_Number\_Table} is part of the compilation of
  a method and describes the link between the Java source lines and
  the Java bytecode.  The \textbf{Local\_Variable\_Table} describes
  the local variables that appear in a method.  This debug information
  is necessary for the compiler from JML to BML, as we shall see later
  in Section \ref{BCSLcompile}.

\item  \textbf{Reducible control flow graph} \\ 
       The control flow graph corresponding to the list of bytecode
       instructions resulting from the compilation of a method body
       must be a reducible control flow graph. An intuition to the
       notion of reducibility is that every cycle in the graph must
       have exactly one entry point, or in other words a cycle can not
       be jumped from outside inside (see \cite{ARUCom1986} for the
       definition of reducibility). This condition is necessary for
       the compilation phase of the loop invariants as well as for the
       verification procedure (Section \ref{wpGeneral}).  Note, that
       this restriction is realistic as nonoptomizing Java compilers
       produce reducible control flow graphs and in practice even hand
       written code is in most cases reducible.
\end{itemize}

In this section, we turn to the \JMLtoBML \ compiler.  As we shall
see, the compilation consists of several phases, namely compiling the
Java source file, preprocessing of the JML specification, resolution
and linking of names, locating the position of intra --- method
specification, processing of boolean expressions and finally encoding
the BML specification in user defined class file attributes.  (their
structure is predefined by JVMS).  In the following, we look in
details at the phases of the compilation process:
\begin{enumerate}
\item Compilation of the Java source file \\
  This can be done by any Java compiler that supplies for every method
  in the generated class file the \textbf{Line\_Number\_Table} \\ and
  \textbf{Local\_Variable\_Table} attributes. % The presence in the
  Java class file format of % these attribute is optional
  \cite{VMSpec}, yet almost all standard non optimizing compilers can
  generate these data. % The \textbf{Line\_Number\_Table} describes
  the link between the source line and the bytecode of a method.  %
  The \textbf{Local\_Variable\_Table} describes the local variables
  that appear in a method. Those attributes are important for the next
  phases of the JML compilation.

\item Compilation of Ghost field declarations \\
      JML specification is invisible by the Java compilers. Thus Java
      compilers omit the compilation of ghost variables declaration.
      That is why it is the responsibility of the \JMLtoBML \ compiler
      to do this work. For instance, the compilation of the
      declaration of the ghost variable from Fig. \ref{bml:ghost} is
      given in Fig.\ref{bml:compiler:ghost} which shows the data
      structure \textbf{Ghost\_field\_Attribute} in which the
      information about the field \texttt{TRANS} is encoded in the
      class file format. Note that, the constant pool indexes
      \textbf{\#18} and \textbf{\#19} which contain its description
      were not in the constant pool table of the class file
      \texttt{Transaction.class} before running the \JMLtoBML \
      compiler on it.
\begin{figure}[t]
\textbf{     
\begin{tabbing}
 Gho\=st\_field\_Attribute \{\\
\> ...\\
\> \{\hspace{3 mm}\= access\_flag 10;\\
\> \> name\_ index = \#18; \\
\> \> descriptor\_index = \#19 \\
\> \} ghost[1];\\
\}
\end{tabbing}
}

\begin{itemize}
\item \textbf{access\_flag}: The kind of access that is allowed to the field

\item \textbf{name\_index}:  The index in the constant pool which contains information about the source name of the field

\item \textbf{descriptor\_index}: The index in the constant pool which contains information about the name of the field type  
\end{itemize}


\caption{\sc Compilation of ghost variable declaration}
\label{bml:compiler:ghost}
\end{figure}

\item Desugaring of the JML specification \\
      %BML supports less specification clauses than JML for the sake
      of keeping compact the class file format.  % In particular BML
      does not support heavy weight behaviour specification clauses or
      nested specification, neither an incomplete % method
      specification(see \cite{JMLRefMan}).  % Thus, a step in the
      compilation of JML specification into BML specification is the
      desugaring of the JML heavy weight The phase consists in
      converting the JML method heavy-weight behaviours and the light
      - weight non complete specification into BML specification
      cases.  It corresponds to part of the standard JML desugaring as
      described in \cite{RT03djml}.  For instance, the BML compiler
      will produce from the specification in Fig.\ref{bml:heavySp} the
      BML specification given in Fig.\ref{bml:heavySpBML}
      



\item Linking with source data structures \\
      When the JML specification is desugared, we are ready for the
      linking and resolving phases.  In this stage, the JML
      specification gets into an intermediate format in which the
      identifiers are resolved to their corresponding data structures
      in the class file.  The Java and JML source identifiers are
      linked with their identifiers on bytecode level, namely with the
      corresponding indexes either from the constant pool or the array
      of local variables described in the
      \textbf{Local\_Variable\_Table} attribute.

      For instance, consider once again the example in
      Fig. \ref{bml:heavySp} and more particularly the first
      specification case of method \texttt{divide} whose precondition
      \texttt{ b > 0 } contains the method parameter identifier
      \texttt{b}.  In the linking phase, the identifier \texttt{b} is
      resolved to the local variable $\locVar{1}$ in the array of
      local variables for the method \texttt{divide}.  We have a
      similar situation with the postcondition \texttt{ a == \old{a} /
      b } which mentions also the field \texttt{a} of the current
      object.  The field name \texttt{a} is compiled to the index in
      the class constant pool which describes the constant field
      reference.  The result of the linking process is in
      Fig.\ref{bml:heavySpBML}.

      % how ghost fields are compiled If, in the JML specification a
      field identifier appears for which no constant pool index
      exists, it is added in the constant pool and the identifier in
      question is compiled to the new constant pool index. This
      happens when declarations of JML ghost fields are compiled.
     

  





      

\item Locating the points for the intra ---method specification \\

      In this phase the specification parts like the loop invariants
      and the assertions which should hold at a certain point in the
      source program must be associated to the respective program
      point in the bytecode. For this, the
      \textbf{Line\_Number\_Table} attribute is used. The
      \textbf{Line\_Number\_Table} attribute describes the
      correspondence between the Java source line and the instructions
      of its respective bytecode.  In particular, for every line in
      the Java source code the \textbf{Line\_Number\_Table} specifies
      the index of the beginning of the basic block\footnote{a basic
      block is a sequence of instructions which does not contain jumps
      except may be for the last instruction and neither contains
      target of jumps except for the first instruction. This notion
      comes from the compiler community and more information on this
      one can find at \cite{ARUCom1986}} in the bytecode which
      corresponds to the source line. Note however, that a source line
      may correspond to more than one instruction in the
      \textbf{Line\_Number\_Table}.
     
      This poses problems for identifying loop entry instruction of a
      loop in the bytecode which corresponds to a particular loop in
      the source code. % which is important for the % compilation of
      the JML loop invariants (as we should know exactly where they
      must hold in the bytecode). For instance, for method
      \texttt{replace} in the Java source example in
      Fig. \ref{replaceSrc} the java compiler will produce two lines
      in the \textbf{Line\_Number\_Table} which correspond to the
      source line \textbf{17} as shown in
      Fig. \ref{bml:compiler:loopEntry}.  The problem is that none of
      the basic bloks determined by instructions \textbf{2} and
      \textbf{18} contain the loop entry instruction of the
      compilation of the loop at line \textbf{17} in
      Fig. \ref{replaceSrc}. Actually, the loop entry instruction in
      the bytecode in Fig. \ref{bml:loopBML} (remember that this is
      the compilation in bytecode of the Java source in
      Fig. \ref{replaceSrc}) which corresponds to the in the bytecode
      is at index \textbf{19}.
  
       Thus for identifying loop entry instruction corresponding to a
       particular loop in the source code, we use an heuristics.  It
       consists in looking for the first bytecode loop entry
       instruction starting from one of the \textbf{start\_pc} indexes
       (if there is more than one) corresponding to the start line of
       the source loop in the \textbf{Line\_Number\_Table}. The
       algorithm works under the assumption that the control flow
       graph of the method bytecode is reducible.  This assumption
       guarantees that the first loop entry instruction found starting
       the search from an index in the \textbf{Line\_Number\_Table}
       corresponding to the first line of a source loop will be the
       loop entry corresponding to this source loop.  However, we do
       not have a formal argumentation for this algorithm because it
       depends on the particular implementation of the compiler.  From
       our experiments, the heuristic works successfully for the Java
       Sun non optimizing compiler.
 
\begin{figure}[t]
\textbf{Line\_Number\_Table} 
\textbf{     
\begin{tabbing}
start\_pc \= line \\
\ldots \> \\
2 \> 17 \\ 18 \> 17 \\
\end{tabbing}
}

\caption{\textbf{Line\_Number\_Table}  {\sc for the method } \texttt{replace} {\sc in Fig.  \ref{replaceSrc}  } }
\label{bml:compiler:loopEntry}
\end{figure} \todo{une presentation tres laide }

      
      
\item Compilation of the JML boolean expressions into BML \\
      
     

Another important issue in this stage of the JML compilation is how
the type differences on source and bytecode level are treated. By type
differences we refer to the fact that the JVM (Java Virtual Machine)
does not provide direct support for integral types like byte, short,
char, neither for boolean. Those types are rather encoded as integers
in the bytecode. Concretely, this means that if a Java source variable
has a boolean type it will be compiled to a variable with an integer
type.


 For instance, in the example for the method
\texttt{replace} and its specification in Fig.\ref{replaceSrc} the postcondition states the equality between the JML expression  
\result \ and a predicate. This is correct as the method \texttt{replace} in the Java source is declared with return type boolean  and thus,
 the expression \result \ has type boolean. Still, the bytecode
 resulting from the compilation of the method \texttt{replace} returns
 a value of type integer. This means that the JML compiler has to
 ``make more effort'' than simply compiling the left and right side of
 the equality in the postcondition, otherwise its compilation will not
 make sense as it will not be well typed. Actually, if the JML
 specification contains program boolean expressions that the Java
 compiler will compile to bytecode expression with an integer type,
 the JML compiler will also compile them in integer expressions and
 will transform the specification condition in equivalent
 one\footnote{when generating proof obligations we add for every
 source boolean expression an assumption that it must be equal to 0 or
 1. A reasonable compiler would encode boolean values in a similar
 way}.

Finally, the compilation of the postcondition of method
\texttt{replace} is given in Fig. \ref{postCompile}. From the
postcondition compilation, one can see that the expression \result \
has integer type and the equality between the boolean expressions in
the postcondition in Fig.\ref{replaceSrc} is compiled into logical
equivalence.
% The example also 
% shows that local variables and  fields are respectively linked to the index of the register table for the method and to the corresponding 
% index of the constant pool table 
% (\#19 is the compilation of the field name \texttt{list} and $\locVar{1}$ stands for the method parameter \texttt{obj}). 

\begin{figure}[t]
 $$\begin{array}{l} \result = 1 \\ \\ \iff \\ \exists
 \bound\_{\mbox{\rm \textsf{0}}}, \biggl(\begin{array}{l} \ 0 \leq
 \bound\_{\mbox{\rm \textsf{0}}} \wedge\\ \bound\_{\mbox{\rm
 \textsf{0}}} < len(\#19(\locVar{0})) \wedge \\
 \arrayAccess{\#19(\locVar{0})}{\bound\_{\mbox{\rm \textsf{0}}} } =
 \locVar{1} \end{array} \biggr) \end{array} $$
\caption{\sc The compilation of the postcondition in Fig. \ref{replaceSrc}}
\label{postCompile}
\end{figure}





\item Encoding BML specification  into user defined class attributes\\
  The specification expression and predicates are compiled in binary
  form using tags in the standard way. The compilation of an
  expression is a tag followed by the compilation of its
  subexpressions.
    
 Method specifications, class invariants, loop invariants are newly
 defined attributes in the class file.  For example, the
 specifications of all the loops in a method are compiled to a unique
 method attribute whose syntax is given in
 Fig.~\ref{loopAttribute}. This attribute is an array of data
 structures each describing a single loop from the method source code.
 From the figure, we notice that every element describing the
 specification for a particular loop contains the index of the
 corresponding loop entry instruction \textbf{index}, the loop
 modifies clause (\textbf{modifies}), the loop invariant
 (\textbf{invariant}), an expression which guarantees termination
 (\textbf{decreases}).


\end{enumerate}

\begin{figure}[t]
\textbf{     
\begin{tabbing}
JML\=Loop\_specification\_attribute \{\\
\> ...\\
\> \{\hspace{3 mm}\= u2 index;\\
\> \> u2 modifies\_count;\\
\> \> formula modifies[modifies\_count];\\
\> \> formula invariant;\\
\> \> expression decreases;\\
\> \} loop[loop\_count];\\
\}
\end{tabbing}
}

\begin{itemize}
\item \textbf{index}: The index in the  \texttt{LineNumberTable } where the beginning of the corresponding loop is described

\item \textbf{modifies[]}: The array of locations that may be modified

\item \textbf{invariant }: The predicate that is the loop invariant. It is a compilation of the JML formula in the low level specification language

\item \textbf{decreases}: The expression which decreases at every loop iteration
\end{itemize}
\caption{\sc Structure of the Loop Attribute}
\label{loopAttribute}
%\end{frameit}
\end{figure}

% The JML compiler does not depend on any specific Java compiler, but it requires the presence of a debugging information,
%namely the presence of the \textbf{Line\_Number\_Table} attribute for the correct compilation of inter method
% specification, i.e. loops and assertions. We think that this is an acceptable restriction as few bytecode programs even handwritten are not reducible.
% The most problematic part of the compilation is to identify which source loop corresponds to which bytecode loop in the control flow
% graph.
%  To do this, we assume that the control flow graph is reducible (see~\cite{ARUCom1986}), i.e. there are no
% jumps from outside a loop inside it; graph reducibility allows to establish the same order between loops in the
% bytecode and source code level and to compile the invariants to the correct places in the bytecode.
% \todo{put it elsewhere}

%\todo{limitations : registers that are used with two different types in the method bytecode}
