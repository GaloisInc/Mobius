\section{A short overview of JML}\label{SecJMLOverview}

This section gives a short introduction to JML, by means of an
example. Throughout the rest of this paper, we assume that the
reader is familiar with JML, its syntax and its semantics. For a
detailed overview of JML we refer to its reference
manual~\cite{JMLReferenceManual05}. Where necessary, we refer to the
appropriate sections of this manual. A detailed overview of the tools
which support JML can be found in~\cite{BurdyCCEKLLP05}.


%\lstset{numbers=left,numberstyle=\small,stepnumber=1,numbersep=5pt}
\begin{figure}[th!]
%\begin{lstlisting}[frame=trbl] 
{\small
\begin{verbatim}
/* @author Hermann Lehner, Aleksy Schubert
 * The Bill class provides an abstract implementation of the bill 
 * functionality. It calculates the aggregate cost for series of
 * investments based on the implementation of the method which gives 
 * the cost of a single round (to be implemented in subclasses). */
abstract class Bill {

  private int sum;  //@ invariant sum>=0;
 
  /* This method gives a cost of a single round.
   * @param x is the number of the particular round
   * @return the cost of the investment in this round, it's not
   *         greater than <code>x</code> */
  //@ ensures 0 <= \result && \result <= x;
  abstract int round_cost(int x) throws Exception;
  
  /* This method calculates the cost of the whole series of investments.
   * @return <code>true</code> when the calculation is successful and
   *         <code>false</code> when the calculation cannot be performed */
  //@ requires n > 0;
  //@ ensures sum <= \old(sum)+n*(n+1)/2;
  public boolean produce_bill(int n){
    try{
      //@ loop_modifies sum, i;
      //@ loop_invariant 0 <= i && 0 <= sum && i <= n + 1 && sum <= \old(sum)+(i-1)*i/2;
      for (int i=1;i<=n;i++) 
      { this.sum = this.sum + round_cost(i); }
      return true;
    } catch (Exception e){ return false; }
  }
}
\end{verbatim}
}
%\end{lstlisting}
\caption{Class \texttt{Bill} with JML annotations} 
\label{FigJMLSpec}
\end{figure}

To illustrate the different features of JML, Figure~\ref{FigJMLSpec}
shows an example class specification, defining the class
\texttt{Bill}. It contains an abstract method
\texttt{round\unsc cost}, that computes the cost of a particular
round. The method \texttt{produce\unsc bill} is supposed to sum up the
costs of the different rounds. 

In order not to interfere with the standard Java compiler, JML
specifications are written as special comments (tagged with
\texttt{@}). Method specifications contain preconditions (keyword
\jmlKey{requires}), postconditions (\jmlKey{ensures}) and frame
conditions (\jmlKey{assignable}). The latter specify which variables
\emph{may} be modified by a method. In a method body, one can
annotate all statements with an \jmlKey{assert} predicate and loops
also with invariants (\jmlKey{loop\unsc invariant}), variants
(\jmlKey{decreases}) and loop frame conditions (\jmlKey{loop\unsc
modifies}). The latter is a non-standard extension of JML, introduced
in~\cite{BurdyRL03}, which we found useful to make program
verification more practical. One can also specify class invariants,
\emph{i.e.}\ properties that should hold in all visible states of the
execution, and constraints, describing a relation that 
holds between any two pairs of consecutive visible states (where
visible states are the states in which a method is called or returned
from).

The predicates in the different conditions are side-effect free Java
boolean expressions, extended with specification-specific keywords,
such as \jmlKey{\bsl result}, denoting the return value of a non-void
method, and \jmlKey{\bsl old}, indicating that an expression should be
evaluated in the pre-state of the method. 

JML allows to declare special specification-only variables: logical
variables (with keyword \jmlKey{model}) and so-called \jmlKey{ghost}
variables, that can be assigned to in special \jmlKey{set}
annotations.

In Figure~\ref{FigJMLSpec}, the specification for \texttt{round\unsc
cost} states that the result of the method should be positive, but
less than the number of the round. The specification for
\texttt{produce\unsc bill} requires that we compute at least one
round, and then ensures an upper-bound on the outcome of the
method. We use a loop invariant and loop frame condition to prove the
method body correct. Finally, the class invariant specifies that the
\texttt{sum} field is always positive.
     


%JML also allows the declaration of special JML variables, that are
%used only for specification purposes. These variables are declared in
%comments with the \jmlKey{ghost} modificator and may be used only in
%specification clauses. Those variables can also be assigned. Ghost
%variables are usually used for expressing properties which can not be
%expressed with the program variables.

