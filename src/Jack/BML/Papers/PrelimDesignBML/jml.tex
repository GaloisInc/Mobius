\section{A short overview of JML}\label{SecJMLOverview}

This section gives a short introduction to JML, by means of an
example. Throughout the rest of this paper, we will assume that the
reader is familiar with JML, its syntax and its semantics. For a
detailed overview of JML we refer to its reference
manual~\cite{JMLReferenceManual05}. Where necessary, we refer to the
appropriate sections of this manual. A detailed overview of the tools
which support JML can be found in~\cite{BurdyCCEKLLP05}.

\lstset{numbers=left,numberstyle=\small,stepnumber=1,numbersep=5pt}
\begin{figure}[t]
\begin{lstlisting}[frame=trbl] 
abstract class Bill {
  
  private int sum;
  //@ invariant sum>=0;
 
/**
 * This method gives a cost of a single round.
 * @param x is the number of the particular round
 * @return the cost of the investment in this round, it's not
 *  greater than <code>x</code> 
 */ 
//@ensures 0 <= \result && \result <= x;
abstract int round_cost(int x) throws Exception;

/**
 * This method calculates the cost of the whole series of investments.
 *
 * @return <code>true</code> when the calculation is successful and
 * <code>false</code> when the calculation cannot be performed
 */
//@requires n > 0;
//@ensures sum <=\old(sum)+n*(n+1)/2;
public boolean produce_bill(int n){
int i;
try{
//@ loop_modifies sum, i;
//@ loop_invariant 0 <= i && 0 <= sum && i <= n + 1 && sum <=  \old(sum)+(i-1)*i/2;
      for (i=1;i<=n;i++) 
      {
        this.sum = this.sum + round_cost(i);
      }
      return true;
    } catch (Exception e){
      return false;
    }
  }
}
\end{lstlisting}
\caption{\sc class \mbox{\rm \lstinline!ListArray!} with JML annotations} 
\label{FigExampleJML}
\end{figure}

Figure~\ref{FigExampleJML} shows a typical example of a simple JML
specification.  In order not to interfere with the standard Java
compiler, JML specifications are written as special Java comments
(tagged with \texttt{@}). Method specifications contain preconditions
(keyword \jmlKey{requires}), postconditions \jmlKey{ensures} and
frameconditions (\jmlKey{assignable}). The latter specify which
variables \emph{may} be modified by a method. In the method body, one
can annotate loops with invariants (\jmlKey{loop\unsc invariant} and loop
frame conditions (\jmlKey{loop\unsc modifies}). The latter is a
non-standard extension of JML, introduced in~\cite{BRL-JACK}, which we
found useful to make program verification more practical. Finally, one
can also specify class invariants, \emph{i.e.}\ a property that should
hold in all visible states of the execution, and history constraints,
describing a relation that should hold between any two pairs of
consecutive visible states (where visible states are the states in
which a method is called or returned from). 

The predicates in the different conditions are side-effect free Java
boolean expressions, extended with specification-specific keywords,
such as \jmlKey{\bsl result}, denoting the return value of a non-void
method, and \jmlKey{\bsl old}, indicating that an expression should be
evaluated in the pre-state of the method.

The \jmlKey{spec\unsc public} denotes that a variable is publicly
visible in specifications. JML also allows to declare special
specification-only variables: logical variables (with keyword
\jmlKey{model}) and so-called ghost variables, that can be assigned to
in special \jmlKey{set} annotations.

Figure~\ref{FigExampleJML} specifies the method ... 
     


JML also allows the declaration of special JML variables, that are
used only for specification purposes. These variables are declared in
comments with the \jmlKey{ghost} modificator and may be used only in
specification clauses. Those variables can also be assigned. Ghost
variables are usually used for expressing properties which can not be
expressed with the program variables.

