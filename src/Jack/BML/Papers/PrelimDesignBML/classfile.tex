\section{Encoding BML specifications in the class file format}
\label{SecClassfile}

To store BML specifications together with the bytecode that it
specifies, we need a way to encode them in the class file format. We
do this using so-called user-specific attributes for Java class files.

\begin{itemize}
\item \textbf{Java compiler independance } 
Class files containing BML specification must not depend on any non
optimizing compiler.
    
      To do this, the process of the Java source compilation is
      separate from the JML compilation. More particularly, the
      \JMLtoBML (short for the compiler from JML to BML) \ compiler
      takes as input a Java source file annotated with JML
      specification and its Java class produced by a non optimizing
      compiler containing a debug information.% As we shall see later
      in the coming sections, the debug data plays a role in the
      compilation of the JML specification into BML.
      
      %In other words, we would like that the compilation of BML specification  is not attached to a particular Java compiler. 
      %This makes BML independant from Java source compilation.
      % Note, however that we impose as a restriction that the compiler should be not optimizing. 
       % generating debug information~\footnote{the debug information is necessary for the compilation
      % from JML to BML as we shall see in the coming sections}.

\item \textbf{JVM compatibility } 
            The class files augmented with the BML specification must
            be executable by any implementation of the JVM
            specification.  % why do we do so?  Because the JVM
            specification does not allow inlining of any user specific
            data in the bytecode instructions BML annotations must be
            stored separately from the method body (the list of
            bytecode instructions which represents its body).
	  
	    
	    % how 

In particular, the BML specification is written in the so called user
defined attributes in the class file.  The JVM specification defines
the format of those attributes and mandates that any user specific
information should be stored in such attributes. Note, that attribute
which encodes the specification referring to a particular bytecode
instruction contains information about the index of this
instruction. For instance, BML loop invariants are stored in a user
defined attribute in the class file format which contains the
invariant as well as the index of the entry point instruction of the
loop.
	    
	    %comparison

	    Thus, BML encoding is different from the encoding of JML
	    specification where annotations are written directly in
	    the source text as comments at a particular point in the
	    program text or accompany a particular program
	    structure. For instance, in Fig. \ref{replaceSrc} the
	    reader may notice that the loop specification refers to
	    the control structure which follows after the
	    specification and which corresponds to the loop.  This is
	    possible first because the Java source language is
	    structured, and second because writing comments in the
	    source text does not violate the Java or the JVM
	    specifications.
	  

            %  However, on bytecode level we
	    %  could not write directly in the bytecode of a method body, as this will corrupt the performance of any standard Java Virtual Machine.
	    %  That's why specification is written outside the bytecode text and contains also information about the instruction to which the specification
	    %  refers. Then, as bytecode does not have control structures specification will always refer to a particular instruction in the bytecode. 
	    %  For instance, loops on bytecode are identified by a unique loop entry instruction and thus, a loop invariant must hold basically every time
	    %  the corresponding loop entry instruction is reached.

\item \textbf{Compactness} and \textbf{Efficiency}

      Although opposite, we consider those two features together
      because they are mutually dependent. By the first, we mean that
      the class files augmented with BML should be as compact as
      possible.  The second feature refers to that tools supporting
      BML should not be slowed down by the processing of the BML
      specification and more precisely we refer verification condition
      generator tools.  This is an important condition if verification
      is done on devices with limited resources.

      For fulfilling these conditions, BML is designed to correspond
      to a subset of the desugared version of JML.  In particular, it
      brings a relative compactness of the class file as well as makes
      the verification procedure more efficient.

      % compactness   

      We first see in what sense this allows the class file
      compactness. Because every kind of BML specification clause is
      stored in a different user defined attribute, supporting all
      constructs of JML would mean that class files may contain a
      large number of attributes which would increase considerably the
      class file size. Of course, the size of a BML specification
      depends also on how much detailed is the specification, the more
      detailed it is, the larger size it would have.
      
      % efficiency

      Because BML corresponds to a desugared version of JML, this
      means that on verification time the BML specification does not
      need much processing and thus, it can be easily translated to
      the data structures used in the verification scheme. This makes
      BML suitable for verification on devices with limitted
      resources.
      
\end{itemize}


Recall that a class file contains all the information related to a
single class or interface, i.e.~its class name, interfaces
implemented by the class, its super class and the methods and fields
it declares. The Java Virtual Machine Specification~\cite{JVMspec} 
prescribes the mandatory elements of the class
file: the constant pool, the field information and the method
information. The constant pool is the table which is used to construct
the runtime constant pool upon class or interface creation. This will
serve for loading, linking and resolution of references used in the
class. The JVM specification allows to add user-specific information to the 
class
file (\cite[\S4.7.1]{JVMspec}), by defining user-specific attributes,
following the structure prescribed by the JVM specification. 
We use these to encode
the BML specifications. For each class, we add the following global
user-specific attributes:
\begin{itemize}
\item lists of the model and ghost fields used in the specification;
if a model or a ghost field is dereferenced in the specification, then
a constantFieldRef is added to the constant pool as the Java compiler
would do for any dereferenced Java field\footnote{The JVM
specification does not
allow to create a separate constant pool for specification-only
variables, since every constant that occurs in the class file
\emph{must} occur in the standard constant pool.};
\item a list of the class invariants (both static and object); and
\item a list of the history constraints (both static and object).
\end{itemize}

The lists of model and ghost fields have the following format:\\ 

\textbf{
\begin{tabular}{l}
Ghost\unsc Field\unsc attribute \{\\
\hspace*{1em}
\begin{tabular}{l}
u2  attribute\unsc name\unsc index; \\
u4  attribute\unsc length;\\
u2  fields\unsc count;\\
\{\begin{tabular}[t]{l} 
    u2 access\unsc flags; \\  
    u2 name\unsc index;\\
    u2 descriptor\unsc index;\\
  \end{tabular}\\
\} fields[fields\unsc count];\\
\end{tabular}\\
\}
\end{tabular}
}\\

This should be understood as follows: the name of the attribute is
given as an index into the constant pool. This constant pool entry
will be representing a string (either \texttt{"Model\unsc Field"} or
\texttt{"Ghost\unsc Field"}). Next we have the length of the
attribute, which should be 2 + 6*\textbf{fields\unsc count} (the number of
fields stored in the list). The \textbf{fields} table then stores all
ghost and model fields. For each field we store its access flags
(e.g.~\texttt{public} or \texttt{private}), and the name index and 
descriptor index, both referring to the constant pool. The first must be a
string, representing the (unqualified) name of the variable, the
latter is a field descriptor, containing e.g.~type
information.  The information as
\textbf{u2} and \textbf{u4} specifies the size of the attribute, 2 and
4 bytes, respectively.

In a similar way, we specify the format for the attributes containing
the list of class invariants and history constraints. The type of
invariants and history constraints is specified by the 
\textbf{type} entry: when it is \textbf{1} the invariant (or history
constraint) is defined over objects, when it is \textbf{0} the
invariant (or constraint) is static.

\noindent
\begin{center}
\begin{tabular}{p{7cm}p{10cm}}
\textbf{
\begin{tabular}{l}
JMLClassInvariant\unsc attribute \{ \\
\hspace*{0.1em}\begin{tabular}{l}
u2 attribute\unsc name\unsc index;\\ 
u4 attribute\unsc length;\\ 
u2  invariant\unsc count;\\
\{\begin{tabular}[t]{l} 
        u1 type;\\
	formula invariant;\\ 
\end{tabular}\\
\} invariants[invariant\unsc count];  
\end{tabular}\\
\}  
\end{tabular}
}
&
\textbf{
\begin{tabular}{l}
JMLHistoryConstraints\unsc attribute \{ \\ 
\hspace*{1em}\begin{tabular}{l}
u2 attribute\unsc name\unsc index;\\ 
u4 attribute\unsc length;\\ 
%formula attribute\unsc formula;\\ 
u2  history\unsc constr\unsc count;\\
\{\begin{tabular}[t]{l} 
        u1 type;\\
	formula constraint;\\ 
\end{tabular}\\
\} history\unsc constr[history\unsc constr\unsc count];
\end{tabular}\\
\}
\end{tabular}
}
\end{tabular}
\end{center}

The JVM specification prescribes that the table with method 
information at least
contains the code of each method. We add attributes for the method
specification, a table with set statements, a table with assert
statements, a table with assume statements and a table with loop
specifications.  The attribute with the lightweight behaviour
specifications is formatted as follows (heavyweight behaviour
specifications are handled similarly): 

\noindent\begin{flushleft}\textbf{    
\begin{tabular}{l}
JMLMethod\unsc attribute \{ \\ 
\hspace*{1em}
\begin{tabular}[t]{l}
u2 attribute\unsc name\unsc index;\\ 
u4 attribute\unsc length;\\ 
formula requires\unsc formula;\\
u2 spec\unsc count;\\
\{\begin{tabular}[t]{l}
  formula spec\unsc requires\unsc formula; \\
  u2 assignable\unsc count;\\
  formula assignable[assignable\unsc count];\\
  formula ensures\unsc formula;\\
  u2 exsures\unsc count;\\
  \{\begin{tabular}[t]{l}
    u2 exception\unsc index; \\
    formula exsures\unsc formula;\\
    \end{tabular}\\
  \} exsures[exsures\unsc count];\\
  \end{tabular}\\
\} spec[spec\unsc count];   \\
\end{tabular}\\
\}
\end{tabular}
}
\end{flushleft}

The global requires formula is the disjunction of all preconditions in
the different specification cases of the method. For each
specification case, we then have a precondition
(\textbf{spec\unsc requires\unsc formula}), a list of assignable expressions,
a postcondition (\textbf{ensures\unsc formula}) and a list of exceptional
postconditions (stored in the \textbf{exsures} attribute). If a clause
is not explicitly specified, its default value will be stored
here. Notice that for each list of elements we get two attributes: one
to store the number of elements, and one attribute actually containing
the elements.

The tables with set, assert and assume statements are very
similar. For each statement we use \textbf{index} to denote the point
in the bytecode to which the statement is associated. For the set
statement, expression \textbf{e1} is a ghost variable, \textbf{e2}
denotes the expression that will be assigned to \textbf{e1}. For the
assert and assume statements, the formula \textbf{predicate} is the
predicate that is supposed to hold at this point in the program
execution. We only give the format for the assert statement table
here, the assume statement table is similar.\\

\begin{tabular}{p{8cm}p{8cm}}
\textbf{  
\begin{tabular}[t]{l}
Set\unsc attribute \{\\
\hspace*{1em}\begin{tabular}{l}
u2 attribute\unsc name\unsc index;\\
u4 attribute\unsc length;\\
u2 set\unsc count;\\
\{\begin{tabular}[t]{l}
  u2 index; \\
  expression e1; \\
  expression e2; \\
  \end{tabular}\\
\} set[set\unsc count];\\
\end{tabular}\\
\}
\end{tabular}
}

&
\textbf{
\begin{tabular}[t]{l}
Assert\unsc attribute \{\\
\begin{tabular}{l}
u2 attribute\unsc name\unsc index;\\
u4 attribute\unsc length;\\
u2 assert\unsc count;\\
\{\begin{tabular}[t]{l}
  u2 index; \\
  formula predicate; \\
\end{tabular}\\
\} assert[assert\unsc count];\\
\end{tabular}\\
\}
\end{tabular}
}
\end{tabular}\\


Finally, loop specifications consist of the following elements: an
\textbf{index} to the bytecode instruction that corresponds to the
entry of the loop, a list of variables that may be modified by the
loop, a loop invariant, and a \textbf{decreases} clause, which is the
loop variant, i.e.~the expression that allows to prove
termination of the loop. If the specification does not contain a loop
variant, we indicate this, using a special tag for the
\textbf{decreases} clause. This gives the following attribute format.\\

\textbf{     
\begin{tabular}{l}
JMLLoop\unsc specification\unsc attribute \{\\
\begin{tabular}{l}
u2 attribute\unsc name\unsc index;\\
u4 attribute\unsc length;\\
u2 loop\unsc count;\\
\{\begin{tabular}{l}
  u2 index;\\
  u2 modifies\unsc count;\\
  formula modifies[modifies\unsc count];\\
  formula invariant;\\
  expression decreases;\\
  \end{tabular}\\
\} loop[loop\unsc count];\\
\end{tabular}\\
\}
\end{tabular}
}\\
