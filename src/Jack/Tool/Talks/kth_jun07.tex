%% -*- coding: iso-8859-1; ispell-local-dictionary: "english"; compile-command: "make" -*-
\documentclass[table,10pt]{beamer} 
\usefonttheme{serif}

\usecolortheme{deepblue}
\usetheme{keepMetalAlive}
\usepackage[normalem]{ulem}
\usepackage{alltt}
\usepackage{xspace}
\newcommand{\eg}{e.g.\xspace}
\newcommand{\ie}{i.e.\xspace}
\newcommand{\etc}{etc.\xspace}
\newcommand{\etal}{et al.\xspace}
\newcommand{\cf}{cf.\xspace}
\newcommand{\wrt}{w.r.t.\xspace}
\def \bsl       {\symbol{92}}
\def \unsc      {\symbol{95}}
\newcommand{\one}{\emph{(i)}\xspace}
\newcommand{\two}{\emph{(ii)}\xspace}
\newcommand{\three}{\emph{(iii)}\xspace}
\newcommand{\four}{\emph{(iv)}\xspace}

\newcommand{\hoare}[1]{\(\{\)#1\(\}\)}
\newcommand{\sequ}[1]{\(\{\)#1\(\}_s\)}
\newcommand{\conc}[1]{\(\{\)#1\(\}_c\)}
\newcommand{\seqconc}[2]{\sequ{#1} \(\leadsto\) \conc{#2}}
\newcommand{\perm}[1]{\([\)#1\(]\)}

\newcommand{\both}[2]{\perm{#1}\quad\hoare{#2}}


%\usepackage{colortbl}
\usepackage{colordvi}

\institute{\ \\ \ \\\includegraphics[scale=0.3,viewport=125 525 400 400]{mobius_logo_only.pdf}\includegraphics[scale=0.2,viewport=900 1000 400 400]{Logo-INRIA-couleur.pdf}}
\newcommand{\insertpicture}{
\ifnum\thesection = 0
\else
  \ifnum\thesection < 6
  \includegraphics[scale=0.2]{metapost-figures/pdf/fig-50\thesection.pdf}
  \else
  \fi
\fi
}

\title{Towards modular verification of concurrent object-oriented programs}

\author{Marieke Huisman\\
  \hfill\\
  INRIA Sophia Antipolis\\
  \hfill\\
  Joint work with Cl\'ement Hurlin}

\date{}

\begin{document}

\frame{
  \maketitle
  \thispagestyle{empty}
}

\section{Introduction}

\frame{
The next challenge for program verification: \alert{multi-threading}

\begin{itemize}
\pause
\item Easy to make programming mistakes
\pause
\item Requires reasoning about interferences between threads
\pause
\item Global understanding of system necessary for local reasoning
\pause
\item Interleaving semantics (or not?)
\pause
\item Imposing programming style (or not?)
\pause
  \item Use of design patterns (Doug Lea)
  \end{itemize}
}

% \frame{

%   \tableofcontents

% }

\section{Java Memory Model}

\frame{
\begin{itemize}
\item Describes legal executions of multithreaded program
\item New since Java 1.5
\item Fixes security bug related with final fields
\item Allows compiler optimisations
\end{itemize}
}

\subsection{Unexpected behaviours}
\frame{
Consequences for \alert{incorrectly} synchronised programs\\
(initially: \texttt{x = y = 0}):\medskip\\
\noindent\begin{tabular}{p{4cm}p{4cm}}
Thread 1: & Thread 2: \\
\fbox{
\parbox{1.2cm}{
\texttt{x = 1;}\\
\texttt{j = y;}}
}
&
\fbox{
\parbox{1.2cm}{
\texttt{y = 1;}\\
\texttt{i = x;}}
}
\end{tabular}\\
\bigskip
\pause
Possible results of running Thread 1 and 2 in parallel:
\begin{itemize}
\item \texttt{i = 1, j = 0}
\item \texttt{i = 0, j = 1}
\item \texttt{i = 1, j = 1}
\pause
\item \alert{\texttt{i = 0, j = 0}}
\end{itemize}
}

\frame{
Consequences for \alert{incorrectly} synchronised programs\\
(initially: \texttt{x = y = 0}):\medskip\\
\noindent\begin{tabular}{p{4cm}p{4cm}}
Thread 1: & Thread 2: \\
\fbox{
\parbox{1.2cm}{
\texttt{x = 1;}\\
\texttt{j = y;}}
}
&
\fbox{
\parbox{1.2cm}{
\texttt{y = 1;}\\
\texttt{i = x;}}
}
\end{tabular}\\
\bigskip
\begin{itemize}
\item Two statements in the thread are independent, can be swapped
\item No synchronisation between threads to enforce that one action must
\Blue{happen before} another action
\item Thus: \texttt{j = y; y = 1; i = x; x = 1} well-formed execution
\end{itemize}
}

\subsection{Data Race Freeness Guarantee}
\frame{
\begin{itemize}
\item Correctly synchronised programs: sequentially consistent
executions do not  contain data races
\item For such programs, all executions sequentially consistent
\item Thus: all behaviours can be described with (fine-grained)
interleaving semantics
\item Formalisation of Java Memory Model and DRF Guarantee (with
Gustavo Petri)
\item Revival of RCC (Race Condition Checker, Flanagan, Abadi, Freund)
by Radu Grigore (UCD)
\end{itemize}
\
}

\section{Verification of program behaviour}
\subsection{Modular verification of sequential programs}
\frame[containsverbatim]{
\begin{alltt}
//@ requires P;
//@ ensures Q;
void call(){ ... }

void method(MyObject o){
  o.call();
  // (1) }
\end{alltt}
Can we rely on \alert{\texttt{Q[o/this]}} at (1)?
}

\subsection{Atomicity}
\frame{
\begin{itemize}
\item Atomicity: at most one instruction in a method is sensitive to
interference of other threads
\item However: atomicity does not directly consider method specification
\item Atomicity often conditional property, e.g.:
\begin{itemize}
\item Caller holds lock
\item Parameter protected by lock
\item Parameter local to thread
\end{itemize}
\end{itemize}
}

\subsection{Invariants}
\frame{
\begin{itemize}
\item Visible state semantics for sequential programs: \alert{all}
invariants of \alert{all} 
allocated objects hold in \alert{all} visible states
\item Concurrent setting: avoid that one thread can (temporarily)
break an invariant, while another thread relies on it
\item Adaptation of semantics needed to modularise invariants:
\begin{itemize}
\item Identification of relevant invariants 
\item Rely-guarantee style reasoning
\item Strong invariants
\item Locking invariants
\end{itemize}
\item Similar issues for constraints
\end{itemize}
}

\subsection{Verification of method specifications}
\frame{

  Difficulties arise from sharing of data:
  \begin{itemize}
  \item Read/write or write/write conflicts: data race
  \item Read of stale value: synchronization issue
  \item Meaning of shared data in specifications
  \end{itemize}

  \bigskip
  \pause
  
  Patterns to avoid this problem:
  \begin{itemize}
  \item Share only \alert{immutable} data
  \item \alert{Protect} shared data with locks
  \item Do \alert{not} share
  \end{itemize}

}

\subsection{Immutable objects}

\frame[containsverbatim]{
  \begin{itemize}
  \item Object cannot change state
  \item Methods safe because they do not rely on assignable fields
  \end{itemize}
}

\frame[containsverbatim]{
%  \begin{itemize}
%  \item Object cannot change state.
%  \item Methods safe because they do not rely on assignable fields.
%  \end{itemize}

%  \begin{small}
\begin{alltt}
class Fraction\{
  \alert{final} int n;
  \alert{final} int d;

  public plus(Fraction f)\{
    return new Fraction(n * f.d + f.n * d, d * f.d); \}

  public boolean equals(Fraction f)\{
    return n * f.d == d * f.n; \}
\}
\end{alltt}
%  \end{small}
}

\subsection{Writable objects}
\frame{
  Locking is useful to avoid undesired concurrent access to objects:
  \begin{itemize}
  \item Via \texttt{synchronized} methods/blocks
  \item Via objects inheriting from \texttt{java.util.concurrent.Locks.Lock} (since Java 1.5)
  \end{itemize}
}

\frame[containsverbatim]{
%  Locking is useful to avoid undesired concurrent access to objects:
%  \begin{itemize}
%  \item Via \texttt{synchronized} methods/blocks.
%  \item Via objects inheriting from \texttt{java.util.concurrent.Locks.Lock} (since 1.5).
%  \end{itemize}
\begin{alltt}
class Even\{
  private /*@ spec_public */ int n = 0; 

  //@ \alert{requires \bsl haslock(this);}
  //@ requires n % 2 == 0;
  //@ ensures  n % 2 == 0;
  public \alert{synchronized} int next()\{
    n++;
    n++;
  \}
\}
\end{alltt}
% \hfill \hyperlink{access example}{\beamergotobutton{example}} \hypertarget{even}{}
}

\frame{
  Various designs associated to locking:
  \begin{itemize}
  \item Fully \texttt{synchronized} objects: all methods \texttt{synchronized}
  \item For efficiency: synchronization only when necessary (one lock per functionality)
  \item Read-write lock: one writer or multiple readers
  \end{itemize}

  \bigskip

}

\frame[containsverbatim]{
  \begin{small}
    \begin{alltt}
class Car\{
  Wheel w1; \dots; Wheel w4;
  final Lock lw = new ReentrantLock();

  final Body b;
  final Lock lb = new ReentrantLock();

  //@ requires lockset.has(lw);
  public void changeWheel(Wheel new, int n)\{ \dots \}

  public void paint()\{ 
    lb.lock();
    \dots
    lb.unlock();  
  \}
\}      
    \end{alltt}
  \end{small}
}

\subsection{Lock free algorithms}
\frame{
  \begin{itemize}
  \item Objects shared, not protected, access different data
  \item Gains in performance
  \item But needs proof that threads read and write independent data
  \item Typical example: fork-join algorithms
  \end{itemize}
}
% \subsection{True-concurrency}

% \frame[containsverbatim]{

%   \begin{itemize}
%   \item Objects where multiple threads execute simultaneously.    
%   \end{itemize}

%   \begin{small}
% \begin{verbatim}







% \end{verbatim}
%   \end{small}
% }

% \frame[containsverbatim]{

%   \begin{itemize}
%   \item Objects where multiple threads execute simultaneously.    
%   \end{itemize}

%   \begin{small}
% \begin{verbatim}
% class DoubleInt{
%   private int a; 
%   private int b; 

%   public void initA(int x){ ... }
%   public void initB(int x){ ... }
%   public void initAll(int x, int y){ ... }
% }
% \end{verbatim}
%   \end{small}
% }

\subsection{Thread ownership and transfer}

\frame{
  \begin{itemize}
 \item Thread ownership allows to reason about lock-free algorithms
 \item Single capacity to write data guarantees thread locality
 \item Spreading of read-only capacities preserves data race freeness
 \item Objects transferred between threads:
  \begin{itemize}
  \item upon thread creation
  \item upon tread finish (via join)
  \item via shared objects (e.g. containers)
  \end{itemize}
  \end{itemize}

}

%\frame[containsverbatim]{

%%  Objects are transferred between threads:
%%  \begin{itemize}
%%  \item When a thread is created.
%%  \item When a thread dies.
%%  \item Via already shared objects.
%%  \end{itemize}

%  \begin{columns}
%    \begin{column}{0.5\textwidth}
%%      \begin{small}
%\begin{verbatim}
%public static void main(){
%  DoubleInt a = new DoubleInt();

%  Thread t = new MyThread(a);
%  t.start(); 
  
%  a.initA(1);

%  t.join();
%}
%\end{verbatim}
% %     \end{small}
%      \end{column}
%      \begin{column}{0.5\textwidth}
%      %  \begin{small}
%\begin{verbatim}









%\end{verbatim}
%   % \end{small}
%      \end{column}
%  \end{columns}
%}

\frame[containsverbatim]{

%  Objects are transferred between threads:
%  \begin{itemize}
%  \item When a thread is created.
%  \item When a thread dies.
%  \item Via already shared objects.
%  \end{itemize}

  \begin{columns}
    \begin{column}{0.5\textwidth}
%      \begin{small}
\begin{verbatim}
public static void main(){
  DoubleInt a = new DoubleInt();

  Thread t = new MyThread(a);
  t.start(); // initialises B
  
  a.initA(1);

  t.join();
}
\end{verbatim}
 %     \end{small}
      \end{column}
      \begin{column}{0.5\textwidth}
  %      \begin{small}
\begin{verbatim}
class DoubleInt{
  private int a; 
  private int b; 

  public void initA(int x){ ... }
  public void initB(int x){ ... }
  public void initAll
    (int x, int y){ ... }
}
\end{verbatim}
   % \end{small}
      \end{column}
  \end{columns}
}

\section{Annotation System}

\frame{
 
  The different behaviors shown before are expressed by permission system:
  \begin{itemize}
  \item Permissions are owned by threads or locks
  \item Permissions cannot be duplicated arbitrarily
  \item Permissions are created when objects are created
  \item Requirements/assumptions about permissions are expressed with pre/postconditions
  \end{itemize}
}

\subsection{Immutability}
\frame[containsverbatim]{

  \begin{small}
\begin{alltt}
class Fraction\{
  final int n;
  final int d;

  \alert{//@ permission_set (*)p = n : (0,1), d : (0,1);}

  \alert{//@ requires true;}
  \alert{//@ ensures (*)p;}
  public Fraction(int n, int d)\{
    this.n = n;
    this.d = d; \}

  \alert{//@ requires p, f.p;}
  \alert{//@ ensures p, f.p;}
  public boolean equals(Fraction f)\{
    return n * f.d == d * f.n; \}
\}
\end{alltt}
  \end{small}
}

%% \subsection{Locking}
%% \frame[containsverbatim]{
%%   \begin{small}
%%     \begin{alltt}
%% class Car\{
%%   Wheel w1; \dots; Wheel w4;
%%   final Lock lw = new ReentrantLock();

%%   final Body b;
%%   final Lock lb = new ReentrantLock();

%%   //@ permission_set \{lw\} = w1 : (1,0) p, \dots, w4 : (1,0) p;
%%   //@                \{lb\} = b : (0,1), b : q;

%%   //@ requires \{lw\};
%%   public void changeWheel(Wheel new, int n)\{ \dots \}

%%   //@ requires true;
%%   public void paint()\{ 
%%     lb.lock();
%%     \dots
%%     lb.unlock(); \}\}      
%%     \end{alltt}
%%   \end{small}
%% }

\subsection{Lock free algorithms}
\frame[containsverbatim]{

  \begin{small}
\begin{alltt}
class DoubleInt\{
  private int a; 
  private int b; 

  \alert{//@ permission_set p = a : (1,0);}
  \alert{//@                q = b : (1,0);}

  \alert{//@ requires p;}
  public void initA(int x)\{ \dots \}

  \alert{//@ requires q;}
  public void initB(int x)\{ \dots \}

  \alert{//@ requires p,q;}
  public void initAll(int x, int y)\{ \dots \}
\}
\end{alltt}
  \end{small}
}

\subsection{Permission splitting/combining in action}

\frame[containsverbatim]{

  \begin{columns}
    \begin{column}{0.5\textwidth}
      \begin{small}
\begin{alltt}
public static void main()\{
  DoubleInt a = new DoubleInt();
    \alert{\(tid \longrightarrow a \longrightarrow p,q\)}






  



\}
\end{alltt}
      \end{small}
    \end{column}
    \begin{column}{0.5\textwidth}
      \begin{small}
\begin{verbatim}
class DoubleInt{
  ...

  //@ ensures p,q;
  public DoubleInt();
}

class MyThread{
  final DoubleInt a;

  //@ ensures this.a == a;
  public MyThread(DoubleInt a){
    this.a = a;
  }

  //@ requires a.q;
  //@ ensures a.q;
  public run(){
    a.initB(2);  
  }
}
\end{verbatim}
      \end{small}
    \end{column}
  \end{columns}
}

\frame[containsverbatim]{

  \begin{columns}
    \begin{column}{0.5\textwidth}
      \begin{small}
\begin{alltt}
public static void main()\{
  DoubleInt a = new DoubleInt();
    \alert{\(tid \longrightarrow a \longrightarrow p,q\)}

  Thread t = new MyThread(a);




  



\}
\end{alltt}
      \end{small}
    \end{column}
    \begin{column}{0.5\textwidth}
      \begin{small}
\begin{verbatim}
class DoubleInt{
  ...

  //@ ensures p,q;
  public DoubleInt();
}

class MyThread{
  final DoubleInt a;

  //@ ensures this.a == a;
  public MyThread(DoubleInt a){
    this.a = a;
  }

  //@ requires a.q;
  //@ ensures a.q;
  public run(){
    a.initB(2);  
  }
}
\end{verbatim}
      \end{small}
    \end{column}
  \end{columns}
}

\frame[containsverbatim]{

  \begin{columns}
    \begin{column}{0.5\textwidth}
      \begin{small}
\begin{alltt}
public static void main()\{
  DoubleInt a = new DoubleInt();
    \alert{\(tid \longrightarrow a \longrightarrow p,q\)}

  Thread t = new MyThread(a);

  t.start();
    \alert{\(tid \longrightarrow a \longrightarrow p\)}
    {\(t \longrightarrow a \longrightarrow q\)}
  




\}
\end{alltt}
      \end{small}
    \end{column}
    \begin{column}{0.5\textwidth}
      \begin{small}
\begin{verbatim}
class DoubleInt{
  ...

  //@ ensures p,q;
  public DoubleInt();
}

class MyThread{
  final DoubleInt a;

  //@ ensures this.a == a;
  public MyThread(DoubleInt a){
    this.a = a;
  }

  //@ requires a.q;
  //@ ensures a.q;
  public run(){
    a.initB(2);  
  }
}
\end{verbatim}
      \end{small}
    \end{column}
  \end{columns}
}

\frame[containsverbatim]{

  \begin{columns}
    \begin{column}{0.5\textwidth}
      \begin{small}
\begin{alltt}
public static void main()\{
  DoubleInt a = new DoubleInt();
    \alert{\(tid \longrightarrow a \longrightarrow p,q\)}

  Thread t = new MyThread(a);

  t.start();
    \alert{\(tid \longrightarrow a \longrightarrow p\)}
    \(t \longrightarrow a \longrightarrow q\)
  
  a.initA(1);

  t.join();
    \alert{\(tid \longrightarrow a \longrightarrow p,q\)}
\}
\end{alltt}
      \end{small}
    \end{column}
    \begin{column}{0.5\textwidth}
      \begin{small}
\begin{verbatim}
class DoubleInt{
  ...

  //@ ensures p,q;
  public DoubleInt();
}

class MyThread{
  final DoubleInt a;

  //@ ensures this.a == a;
  public MyThread(DoubleInt a){
    this.a = a;
  }

  //@ requires a.q;
  //@ ensures a.q;
  public run(){
    a.initB(2);  
  }
}
\end{verbatim}
      \end{small}
    \end{column}
  \end{columns}
}

\subsection{Expressing different states}
\frame{
  Various states seen in these patterns:
  \begin{itemize}
    \item Object not shared and writable by a single thread
    \item Object shared and immutable
    \item Object shared and writable
  \end{itemize}

  \bigskip
  \pause

  In the annotation system:
  \begin{itemize}
  \item Each state expressed by a permission set
  \item State changes expressed by casting permission sets
  \item Permission cast allowed when all permissions owned by single thread
  \item Object can change from unshared to shared state, and vice versa
  \end{itemize}
}

\frame[containsverbatim]{
  \begin{small}
    \begin{alltt}
class Matrix \{
  final int[][] elems;

  \alert{//@ permission_set (2)r = elems : (0,1) (0,1)(0,1);}
  \alert{//@ permission_set w = elems : (0,1) (0,1)(1,0);}

  \alert{//@ requires w;}
  public void setValue(int i, int j, int k)\{ 
    elems[i][j] = k;
  \}

  \alert{//@ requires r;}
  public Matrix[] split()\{ \dots \}
\}
    \end{alltt}
  \end{small}
}

\subsection{Changing state}
\frame[containsverbatim]{
  \begin{small}
    \begin{alltt}
\alert{//@ requires a.w, b.w;}
static public void main(Matrix a, Matrix b)\{
  while(true)\{
    \alert{//@ loop_invariant a.w, b.w;}
    
    askUser(a,b);     // write permission needed

    \alert{//@ ((2)r) a;}
    \alert{//@ ((2)r) b;}

    FutureTask f = new FutureTask(new Multiplication(a,b));
    f.start();        // matrices shared in readonly mode
    
    f.get();          // permission gathered
    \alert{//@ (w)a;}
    \alert{//@ (w)b;}
  \}
\}
    \end{alltt}
  \end{small}
}

%% \subsection{Locking}
%% \frame{
%%   \begin{itemize}
%%   \item Previous locking behaviors shown can be expressed in the system.
%%   \item More interestingly, locking can be combined with permission counting:
%%     \begin{itemize}
%%     \item Shared synchronized access.
%%     \item Unsynchronized local access.
%%     \end{itemize}
%%   \end{itemize}
%% }

%% \frame[containsverbatim]{
%%   \begin{small}
%%     \begin{alltt}
%% class Color \{
%%   int quantity = 0;

%%   //@ permission_set (2)s = \alert{\{this\}};
%%   //@ permission_set tl = quantity : (1,0);

%%   //@ \alert{locks \{this\}} = quantity : (1,0);

%%   //@ requires s;
%%   void use(int q)\{ 
%%     synchronized(this)\{ \dots \}
%%   \}
	
%%   //@ requires tl;
%%   void fill(int q)\{ \dots \}
%% \}      
%%     \end{alltt}
%%   \end{small}
%% }

%% \frame[containsverbatim]{
%%   \begin{small}
%%     \begin{alltt}
%% public static void main()\{
%%   Color c  = new Color();
%%   c.fill(10);

%%   //@ c = ((2)s) c;

%%   Painter p1 = new Painter(c); p1.start();
%%   Painter p2 = new Painter(c); p2.start();
%%   p1.join();
%%   p2.join();

%%   //@ c = (tl) c;

%%   c.fill(10);
%%   \}
%% \}    \end{alltt}
%%   \end{small}
%% }

\subsection{Exploiting capacities}

\frame[containsverbatim]{
  \begin{alltt}
/*@ <1,0> @*/ class Quantity\verb|{| 
  public volatile int i; //@ invariant i >= 0;

  //@ requires initial >= 0;
  //@ ensures i == initial;
  public Quantity(int initial)\verb|{| i = initial; \verb|}|

  //@ requires plus >= 0;
  //@ modifies i;
  //@ ensures i >= plus;
  public void add(int plus)\verb|{| i += plus; \verb|}|\verb|}|
\end{alltt}
}

\frame[containsverbatim]{
\begin{alltt}
/*@ <*,*> @*/ class CommonWarehouse\verb|{|
  //@ \alert{requires q : (1,0);}
  //@ \alert{ensures  q : (0,0);}
  static void lend(Quantity q)\verb|{| ... \verb|}|
    
  //@ \alert{requires q : (0,0);}
  //@ \alert{ensures  q : (1,0);}
  static void takeBack(Quantity q)\verb|{| ... \verb|}|\verb|}|
\end{alltt}
}

\frame[containsverbatim]{
\begin{alltt}

  //@ requires j >= 0 && k >= 0;
  //@ ensures \verb|\|result.i >= k;
  public static Quantity main(int j, int k)\verb|{|
    Quantity q = new Quantity(j);
      \alert{\perm{q : (1,0)}    \seqconc{q.i = j}{q.i = j}}
    CommonWarehouse.lend(q);
      \alert{\perm{q : (0,0)}    \seqconc{q.i = j}{\(\top\)}}
    CommonWarehouse.takeBack(q);
      \alert{\perm{q : (1,0)}    \seqconc{q.i = j}{\(\top\)}}
    q.add(k);
      \alert{\perm{q : (1,0)}    \seqconc{q.i \(\geq\) k}{q.i \(\geq\) k}  }
   return q; 
  \verb|}|
\end{alltt}
}

\section{Conclusions}
\subsection{Related Work}

\frame{

  \begin{itemize}
  \item JML's \texttt{monitors\_for}
  \item Jacobs \etal's work on Spec\# (Leuven)
  \item Boyland's fractional permission system
  \item Separation logic (Reynolds \etal)
  \end{itemize}

}

\subsection{Towards modular verification of concurrent object-oriented programs}
\frame{

\begin{itemize}

\item Verification of behaviour of programs without data races
\item Class invariants and history constraints: need for alternatives
\item Thread ownership allows to express which treads can read or write
data
\item Thread ownership can encode a variety of programming patterns
    \begin{itemize}
    \item Objects shared as immutable
    \item Objects shared and protected with locks
    \item Objects shared, not protected, access to different fields
    \item Synchronized concurrent access and local unsynchronized access
    \item Permission splitting/gathering at thread creation/join
    \end{itemize}
\item Implementation of runtime checker
\end{itemize}
}

\subsection{Future work}
\frame{
  \begin{itemize}
  \item Develop alternatives to invariants and constraints
  \item Encoding of annotation system in Boogie:
    \begin{itemize}
    \item Checks of methods pre/postconditions
    \item Permission splitting/gathering
    \end{itemize}
  \item Verification condition generation based on permission information
  \end{itemize}

}

%% \section*{Usual patterns}
%% \subsection*{Locking}
%\section*{}
%\frame[containsverbatim]{

%  \begin{columns}
%    \begin{column}{0.5\textwidth}
%      \begin{small}
%\begin{alltt}
%class Even\{
%  private int n = 0;

%  //@ requires n % 2 == 0;
%  //@ ensures  n % 2 == 0;
%  public \alert{synchronized} int next()\{
%    n++;
%    n++;
%  \}
%\}
%\end{alltt}
%      \end{small}
%    \end{column}
%    \begin{column}{0.5\textwidth}
%      \begin{center}
%        % cette commande vient de xcolor.sty et il faut passer ``table'' en option � beamer pour l'avoir
%        \rowcolors[]{1}{chaleurred!40}{chaleurred!10}
%        \begin{tabular}{c | c} 
%          thread 1 & thread 2 \\ \hline
%          read 0 & \\
%          write 1 &\\
%          & read 1 \\
%          & write 2\\
%          read 2 & read 2 \\
%          & write 3 \\
%          write 3 & \\
%        \end{tabular}
%      \end{center}
%    \end{column}
%  \end{columns}
  
%  \bigskip

%  \hfill \hyperlink{even}{\beamergotobutton{back}} \hypertarget{access example}{}

%}



\end{document}
