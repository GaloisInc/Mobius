\nextslide{Un exemple}\small
\begin{tt}
\begin{tabbing}

{\purple int} i; {\purple int}[] tab = {\purple new int}[3];\\
/*\=@ {\purple loop\_modifies} \= i, tab[*]; \+\\
  @ {\purple loop\_invariant}  (0 <= i) \&\& (i <= tab.length) \&\& \\
  @        \>({\purple $\backslash$forall} int j; (0 <= j) \&\& (j < i); tab[j] == 0);\\
  @ {\purple decreases} (tab.length - i);\\
  @*/ \- \\
{\purple for} (i \== 0; i < tab.length; i++) \{  \\
  \>tab[i] = 0;  \\
\}
\end{tabbing}
\end{tt}
\rarrow Jack g\'en\`ere 13 obligations de preuve, 
dont {\purple 5  concernent} la boucle.

Nous allons r\'esoudre l'obligation de preuve de la 
{\purple conservation de l'invariant}.
\nextslide{Preuve de la conservation de l'invariant}\small
intelements : REFERENCES \rarrow t\_int \rarrow t\_int\\
l\_i : t\_int\\
newObject\_9 : REFERENCES\\
hyp1 : 0 $\leq$ l\_i\\
%hyp2 : newObject\_9 $<>$ null\\
hyp4 : $\forall$ l\_j : t\_int,
       0 $\leq$ l\_j $\wedge$ l\_j $<$ l\_i \rarrow intelements newObject\_9 l\_j = 0\\
%hyp5 : $\sim$ instances newObject\_9\\
hyp8 : interval 0
         (j\_sub
            (overridingCoupleRef t\_int arraylength newObject\_9 3 newObject\_9)
            1) l\_i\\
hyp9 : l\_i $<$
       overridingCoupleRef t\_int arraylength newObject\_9 3 newObject\_9\\
hyp10 : l\_i $\leq$
        overridingCoupleRef t\_int arraylength newObject\_9 3 newObject\_9\\
----------------------------------------------------------------------------- (1/1)\\
$\forall$ l\_j : t\_int,
0 $\leq$ l\_j $\wedge$ l\_j $<$ l\_i + 1 \rarrow\\
overridingCoupleZ t\_int (intelements newObject\_9) l\_i 0 l\_j = 0

\nextslide{Preuve (1)}
\small
{\tt startJack.}\\
\begin{it}
1 subgoal\\
...\\
hyp9 : l\_i $<$\\
       overridingCoupleRef t\_int arraylength newObject\_9 3 newObject\_9\\
H : 0 $\leq$ l\_j\\
H0 : l\_j $<$ l\_i + 1\\
H4 : l\_i $\leq$\\
     overridingCoupleRef t\_int arraylength newObject\_9 3 newObject\_9 - 1\\
----------------------------------------------------------------------------- (1/1)\\
overridingCoupleZ t\_int (intelements newObject\_9) l\_i 0 l\_j = 0
\end{it}

\nextslide{Preuve (2)}\small
{\tt solveOverriding; trivial}\\
\begin{it}
1 subgoal\\
...\\
hyp4 : $\forall$ l\_j : t\_int,\\
0 $\leq$ l\_j $\wedge$ l\_j $<$ l\_i \rarrow intelements newObject\_9 l\_j = 0\\
H : 0 $\leq$ l\_j\\
H0 : l\_j $<$ l\_i + 1\\
H2 : l\_i $<>$ l\_j\\
----------------------------------------------------------------------------- (1/1)\\
intelements newObject\_9 l\_j = 0\\
\end{it}

{\tt destrJle (l\_j $\leq$ l\_i); autoJack.}\\
{\it Proof Completed.}

\nextslide{Preuve (3)}
En r\'esum\'e, on obtient le script de preuve:
\begin{center}
\begin{minipage}{6cm}
\begin{verbatim}
startJack.
solveOverriding; trivial.
destrJle (l_j <= l_i); autoJack.
\end{verbatim}
\end{minipage}
\end{center}