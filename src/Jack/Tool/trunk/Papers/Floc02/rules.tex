
\documentclass[a4paper]{llncs}

\newcommand{\modif}{\textit{modifiable}}
\newcommand{\inria}{\textsc{Inria}}
\newcommand{\lemme}{\textsc{Lemme}}
\newcommand{\sem}[1]{\ensuremath{\mbox{[\![} {#1} \mbox{]\!]}\/}}

\usepackage{lscape} % it's needed to turn a page
\usepackage{amsmath} % it is needed by the \sem command
\usepackage{alltt} % environment alltt
\message{<Paul Taylor's Proof Trees, 2 August 1996>}
%% Build proof tree for Natural Deduction, Sequent Calculus, etc.
%% WITH SHORTENING OF PROOF RULES!
%% Paul Taylor, begun 10 Oct 1989
%% *** THIS IS ONLY A PRELIMINARY VERSION AND THINGS MAY CHANGE! ***
%%
%% 2 Aug 1996: fixed \mscount and \proofdotnumber
%%
%%      \prooftree
%%              hyp1            produces:
%%              hyp2
%%              hyp3            hyp1    hyp2    hyp3
%%      \justifies              -------------------- rulename
%%              concl                   concl
%%      \thickness=0.08em
%%      \shiftright 2em
%%      \using
%%              rulename
%%      \endprooftree
%%
%% where the hypotheses may be similar structures or just formulae.
%%
%% To get a vertical string of dots instead of the proof rule, do
%%
%%      \prooftree                      which produces:
%%              [hyp]
%%      \using                                  [hyp]
%%              name                              .
%%      \proofdotseparation=1.2ex                 .name
%%      \proofdotnumber=4                         .
%%      \leadsto                                  .
%%              concl                           concl
%%      \endprooftree
%%
%% Within a prooftree, \[ and \] may be used instead of \prooftree and
%% \endprooftree; this is not permitted at the outer level because it
%% conflicts with LaTeX. Also,
%%      \Justifies
%% produces a double line. In LaTeX you can use \begin{prooftree} and
%% \end{prootree} at the outer level (however this will not work for the inner
%% levels, but in any case why would you want to be so verbose?).
%%
%% All of of the keywords except \prooftree and \endprooftree are optional
%% and may appear in any order. They may also be combined in \newcommand's
%% eg "\def\Cut{\using\sf cut\thickness.08em\justifies}" with the abbreviation
%% "\prooftree hyp1 hyp2 \Cut \concl \endprooftree". This is recommended and
%% some standard abbreviations will be found at the end of this file.
%%
%% \thickness specifies the breadth of the rule in any units, although
%% font-relative units such as "ex" or "em" are preferable.
%% It may optionally be followed by "=".
%% \proofrulebreadth=.08em or \setlength\proofrulebreadth{.08em} may also be
%% used either in place of \thickness or globally; the default is 0.04em.
%% \proofdotseparation and \proofdotnumber control the size of the
%% string of dots
%%
%% If proof trees and formulae are mixed, some explicit spacing is needed,
%% but don't put anything to the left of the left-most (or the right of
%% the right-most) hypothesis, or put it in braces, because this will cause
%% the indentation to be lost.
%%
%% By default the conclusion is centered wrt the left-most and right-most
%% immediate hypotheses (not their proofs); \shiftright or \shiftleft moves
%% it relative to this position. (Not sure about this specification or how
%% it should affect spreading of proof tree.)
%
% global assignments to dimensions seem to have the effect of stretching
% diagrams horizontally.
%
%%==========================================================================

\def\introrule{{\cal I}}\def\elimrule{{\cal E}}%%
\def\andintro{\using{\land}\introrule\justifies}%%
\def\impelim{\using{\Rightarrow}\elimrule\justifies}%%
\def\allintro{\using{\forall}\introrule\justifies}%%
\def\allelim{\using{\forall}\elimrule\justifies}%%
\def\falseelim{\using{\bot}\elimrule\justifies}%%
\def\existsintro{\using{\exists}\introrule\justifies}%%

%% #1 is meant to be 1 or 2 for the first or second formula
\def\andelim#1{\using{\land}#1\elimrule\justifies}%%
\def\orintro#1{\using{\lor}#1\introrule\justifies}%%

%% #1 is meant to be a label corresponding to the discharged hypothesis/es
\def\impintro#1{\using{\Rightarrow}\introrule_{#1}\justifies}%%
\def\orelim#1{\using{\lor}\elimrule_{#1}\justifies}%%
\def\existselim#1{\using{\exists}\elimrule_{#1}\justifies}

%%==========================================================================

\newdimen\proofrulebreadth \proofrulebreadth=.05em
\newdimen\proofdotseparation \proofdotseparation=1.25ex
\newdimen\proofrulebaseline \proofrulebaseline=2ex
\newcount\proofdotnumber \proofdotnumber=3
\let\then\relax
\def\hfi{\hskip0pt plus.0001fil}
\mathchardef\squigto="3A3B
%
% flag where we are
\newif\ifinsideprooftree\insideprooftreefalse
\newif\ifonleftofproofrule\onleftofproofrulefalse
\newif\ifproofdots\proofdotsfalse
\newif\ifdoubleproof\doubleprooffalse
\let\wereinproofbit\relax
%
% dimensions and boxes of bits
\newdimen\shortenproofleft
\newdimen\shortenproofright
\newdimen\proofbelowshift
\newbox\proofabove
\newbox\proofbelow
\newbox\proofrulename
%
% miscellaneous commands for setting values
\def\shiftproofbelow{\let\next\relax\afterassignment\setshiftproofbelow\dimen0 }
\def\shiftproofbelowneg{\def\next{\multiply\dimen0 by-1 }%
\afterassignment\setshiftproofbelow\dimen0 }
\def\setshiftproofbelow{\next\proofbelowshift=\dimen0 }
\def\setproofrulebreadth{\proofrulebreadth}

%=============================================================================
\def\prooftree{% NESTED ZERO (\ifonleftofproofrule)
%
% first find out whether we're at the left-hand end of a proof rule
\ifnum  \lastpenalty=1
\then   \unpenalty
\else   \onleftofproofrulefalse
\fi
%
% some space on left (except if we're on left, and no infinity for outermost)
\ifonleftofproofrule
\else   \ifinsideprooftree
        \then   \hskip.5em plus1fil
        \fi
\fi
%
% begin our proof tree environment
\bgroup% NESTED ONE (\proofbelow, \proofrulename, \proofabove,
%               \shortenproofleft, \shortenproofright, \proofrulebreadth)
\setbox\proofbelow=\hbox{}\setbox\proofrulename=\hbox{}%
\let\justifies\proofover\let\leadsto\proofoverdots\let\Justifies\proofoverdbl
\let\using\proofusing\let\[\prooftree
\ifinsideprooftree\let\]\endprooftree\fi
\proofdotsfalse\doubleprooffalse
\let\thickness\setproofrulebreadth
\let\shiftright\shiftproofbelow \let\shift\shiftproofbelow
\let\shiftleft\shiftproofbelowneg
\let\ifwasinsideprooftree\ifinsideprooftree
\insideprooftreetrue
%
% now begin to set the top of the rule (definitions local to it)
\setbox\proofabove=\hbox\bgroup$\displaystyle % NESTED TWO
\let\wereinproofbit\prooftree
%
% these local variables will be copied out:
\shortenproofleft=0pt \shortenproofright=0pt \proofbelowshift=0pt
%
% flags to enable inner proof tree to detect if on left:
\onleftofproofruletrue\penalty1
}

%=============================================================================
% end whatever box and copy crucial values out of it
\def\eproofbit{% NESTED TWO
%
% various hacks applicable to hypothesis list 
\ifx    \wereinproofbit\prooftree
\then   \ifcase \lastpenalty
        \then   \shortenproofright=0pt  % 0: some other object, no indentation
        \or     \unpenalty\hfil         % 1: empty hypotheses, just glue
        \or     \unpenalty\unskip       % 2: just had a tree, remove glue
        \else   \shortenproofright=0pt  % eh?
        \fi
\fi
%
% pass out crucial values from scope
\global\dimen0=\shortenproofleft
\global\dimen1=\shortenproofright
\global\dimen2=\proofrulebreadth
\global\dimen3=\proofbelowshift
\global\dimen4=\proofdotseparation
\global\count255=\proofdotnumber
%
% end the box
$\egroup  % NESTED ONE
%
% restore the values
\shortenproofleft=\dimen0
\shortenproofright=\dimen1
\proofrulebreadth=\dimen2
\proofbelowshift=\dimen3
\proofdotseparation=\dimen4
\proofdotnumber=\count255
}

%=============================================================================
\def\proofover{% NESTED TWO
\eproofbit % NESTED ONE
\setbox\proofbelow=\hbox\bgroup % NESTED TWO
\let\wereinproofbit\proofover
$\displaystyle
}%
%
%=============================================================================
\def\proofoverdbl{% NESTED TWO
\eproofbit % NESTED ONE
\doubleprooftrue
\setbox\proofbelow=\hbox\bgroup % NESTED TWO
\let\wereinproofbit\proofoverdbl
$\displaystyle
}%
%
%=============================================================================
\def\proofoverdots{% NESTED TWO
\eproofbit % NESTED ONE
\proofdotstrue
\setbox\proofbelow=\hbox\bgroup % NESTED TWO
\let\wereinproofbit\proofoverdots
$\displaystyle
}%
%
%=============================================================================
\def\proofusing{% NESTED TWO
\eproofbit % NESTED ONE
\setbox\proofrulename=\hbox\bgroup % NESTED TWO
\let\wereinproofbit\proofusing
\kern0.3em$
}

%=============================================================================
\def\endprooftree{% NESTED TWO
\eproofbit % NESTED ONE
% \dimen0 =     length of proof rule
% \dimen1 =     indentation of conclusion wrt rule
% \dimen2 =     new \shortenproofleft, ie indentation of conclusion
% \dimen3 =     new \shortenproofright, ie
%                space on right of conclusion to end of tree
% \dimen4 =     space on right of conclusion below rule
  \dimen5 =0pt% spread of hypotheses
% \dimen6, \dimen7 = height & depth of rule
%
% length of rule needed by proof above
\dimen0=\wd\proofabove \advance\dimen0-\shortenproofleft
\advance\dimen0-\shortenproofright
%
% amount of spare space below
\dimen1=.5\dimen0 \advance\dimen1-.5\wd\proofbelow
\dimen4=\dimen1
\advance\dimen1\proofbelowshift \advance\dimen4-\proofbelowshift
%
% conclusion sticks out to left of immediate hypotheses
\ifdim  \dimen1<0pt
\then   \advance\shortenproofleft\dimen1
        \advance\dimen0-\dimen1
        \dimen1=0pt
%       now it sticks out to left of tree!
        \ifdim  \shortenproofleft<0pt
        \then   \setbox\proofabove=\hbox{%
                        \kern-\shortenproofleft\unhbox\proofabove}%
                \shortenproofleft=0pt
        \fi
\fi
%
% and to the right
\ifdim  \dimen4<0pt
\then   \advance\shortenproofright\dimen4
        \advance\dimen0-\dimen4
        \dimen4=0pt
\fi
%
% make sure enough space for label
\ifdim  \shortenproofright<\wd\proofrulename
\then   \shortenproofright=\wd\proofrulename
\fi
%
% calculate new indentations
\dimen2=\shortenproofleft \advance\dimen2 by\dimen1
\dimen3=\shortenproofright\advance\dimen3 by\dimen4
%
% make the rule or dots, with name attached
\ifproofdots
\then
        \dimen6=\shortenproofleft \advance\dimen6 .5\dimen0
        \setbox1=\vbox to\proofdotseparation{\vss\hbox{$\cdot$}\vss}%
        \setbox0=\hbox{%
                \advance\dimen6-.5\wd1
                \kern\dimen6
                $\vcenter to\proofdotnumber\proofdotseparation
                        {\leaders\box1\vfill}$%
                \unhbox\proofrulename}%
\else   \dimen6=\fontdimen22\the\textfont2 % height of maths axis
        \dimen7=\dimen6
        \advance\dimen6by.5\proofrulebreadth
        \advance\dimen7by-.5\proofrulebreadth
        \setbox0=\hbox{%
                \kern\shortenproofleft
                \ifdoubleproof
                \then   \hbox to\dimen0{%
                        $\mathsurround0pt\mathord=\mkern-6mu%
                        \cleaders\hbox{$\mkern-2mu=\mkern-2mu$}\hfill
                        \mkern-6mu\mathord=$}%
                \else   \vrule height\dimen6 depth-\dimen7 width\dimen0
                \fi
                \unhbox\proofrulename}%
        \ht0=\dimen6 \dp0=-\dimen7
\fi
%
% set up to centre outermost tree only
\let\doll\relax
\ifwasinsideprooftree
\then   \let\VBOX\vbox
\else   \ifmmode\else$\let\doll=$\fi
        \let\VBOX\vcenter
\fi
% this \vbox or \vcenter is the actual output:
\VBOX   {\baselineskip\proofrulebaseline \lineskip.2ex
        \expandafter\lineskiplimit\ifproofdots0ex\else-0.6ex\fi
        \hbox   spread\dimen5   {\hfi\unhbox\proofabove\hfi}%
        \hbox{\box0}%
        \hbox   {\kern\dimen2 \box\proofbelow}}\doll%
%
% pass new indentations out of scope
\global\dimen2=\dimen2
\global\dimen3=\dimen3
\egroup % NESTED ZERO
\ifonleftofproofrule
\then   \shortenproofleft=\dimen2
\fi
\shortenproofright=\dimen3
%
% some space on right and flag we've just made a tree
\onleftofproofrulefalse
\ifinsideprooftree
\then   \hskip.5em plus 1fil \penalty2
\fi
}

%==========================================================================
% IDEAS
% 1.    Specification of \shiftright and how to spread trees.
% 2.    Spacing command \m which causes 1em+1fil spacing, over-riding
%       exisiting space on sides of trees and not affecting the
%       detection of being on the left or right.
% 3.    Hack using \@currenvir to detect LaTeX environment; have to
%       use \aftergroup to pass \shortenproofleft/right out.
% 4.    (Pie in the sky) detect how much trees can be "tucked in"
% 5.    Discharged hypotheses (diagonal lines).
 %showing demonstration


\title{The \modif~clause: semantics, verification and application}

\author{
  N\'estor Cata\~no and Marieke Huisman  \\
  \institute{
       \inria~Sophia-Antipolis, France \\
       \lemme~Project
  } 
  \email{\{Nestor.Catano, Marieke.Huisman\}@sophia.inria.fr}
}


\begin{document}
\fussy
\maketitle
\pagestyle{plain}


%%%%%
\begin{table}
\rule{\linewidth}{0.25mm}
\\[2.5ex]
\begin{tabular}{ll}
\textsf{(Meth-Dec)}\,\,\,&
\begin{prooftree} 
\texttt{m(}\overrightarrow{\texttt{o}}\texttt{).body}\
\overrightarrow{\textsf{mod}}\ 
Y\cup \{\overrightarrow{\texttt{o}}\}\cup \{
\texttt{m(}\overrightarrow{\texttt{o}}\texttt{).locvars}\} 
\justifies
\texttt{m(}\overrightarrow{\texttt{o}}\texttt{)}\ \textsf{mod}\ Y
\end{prooftree}
\end{tabular}
\\[2.5ex]
\rule{\linewidth}{0.25mm}
\end{table}
%%%%%%%%%



%%%%%%%%%%
\begin{table}
\rule{\linewidth}{0.25mm}
\\[2.5ex]
\begin{tabular}{ll}
\textsf{(Mod-To-Exp)} & 
\begin{prooftree}
e\ \textsf{modEXP}\ Y
\justifies
e\ \textsf{mod}\ Y
\using
e\in Expression
\end{prooftree}
\end{tabular}
\\[2.5ex]
\rule{\linewidth}{0.25mm}
\end{table}
%%%%%%%%%





%%%%%%%%%
\begin{table}
\rule{\linewidth}{0.25mm}
\\[2.5ex]
\begin{tabular}{ll}
\textsf{(Assg)}\,\,\, & 
\begin{prooftree}
e_1\underline{\in}Y,\ \ e_1\
\textsf{modPE}\ Y,\ \ e_2\ \textsf{modEXP}\ Y
\justifies
e_1\oplus \textup{e}_2\ \textsf{modEXP}\ Y
\using
\oplus \in \{\texttt{=,+=,-=,*=,/=}\}
\end{prooftree}
\\[3.0ex] 
\textsf{(Var-Decl-Assg)}\,\, & 
\begin{prooftree}
\rule[1ex]{0em}{1.5ex}
\texttt{x}\underline{\in}Y,\ \ e\ \textsf{modEXP}\ Y
\justifies
\texttt{T x =}\ e\ \textsf{modEXP}\ Y
\end{prooftree}
\\[3.0ex] 
\textsf{(Var-Decl)}\,\,\, & 
\begin{prooftree}
\justifies
\texttt{T x}\ \textsf{modEXP}\ Y
\end{prooftree}
\\[3.0ex]
\textsf{(Pre-Plus)} &
\begin{prooftree}
e\underline{\in}\ Y,\ \ e\ \textsf{modPE}\ Y
\justifies
\texttt{++}e\ \textsf{modEXP}\ Y
\end{prooftree}
\\[3.0ex]
\textsf{(Pre-Minus)} &
\begin{prooftree}
e\underline{\in}\ Y,\ \ e\ \textsf{modPE}\ Y
\justifies
\texttt{-}\strut\texttt{-}\ e\ \textsf{modEXP}\ Y
\end{prooftree}
\\[3.0ex]
\textsf{(Post-Plus)}\,\, &
\begin{prooftree}
e\underline{\in}Y,\ \ e\ \textsf{modPE}\ Y
\justifies
e\texttt{++}\ \textsf{modEXP}\ Y
\end{prooftree}
\\[3.0ex] 
\textsf{(Post-Minus)} &
\begin{prooftree}
e\underline{\in}\ Y,\ \ e\ \textsf{modPE}\ Y
\justifies
e\ \texttt{-}\strut\texttt{-}\ \textsf{modEXP}\ Y
\end{prooftree}
\\[3.0ex]
\textsf{(Binary)} & 
\begin{prooftree} 
e_1\ \textsf{modEXP}\ Y,\ \ e_2\ \textsf{modEXP}\ Y
\justifies
e_1\oplus e_2\ \textsf{modEXP}\ Y
\using
\oplus \in \{
	\begin{array}{l}
		\texttt{<,<=,>,>=,==,!=,||}	\\
		\texttt{\&\&,+,-,*,/,\,\&,\^\ ,|}
	\end{array}
	\}
\end{prooftree}
\\[3.0ex]
\textsf{(Unary)} & 
\begin{prooftree} 
e\ \textsf{modEXP}\ Y
\justifies
\oplus \ e\ \textsf{modEXP}\ Y
\using
\oplus \in \{\texttt{+,-,$\sim$,!}\}
\end{prooftree}
\\[3.0ex] 
\textsf{(Instance)} & 
\begin{prooftree} 
e\ \textsf{modEXP}\ Y
\justifies
e\ \texttt{instanceof C}\ \textsf{modEXP}\ Y
\end{prooftree}
\\[3.0ex] 
\textsf{(Cast)} & 
\begin{prooftree}
e\ \textsf{modEXP}\ Y
\justifies
\texttt{(T)}e\ \textsf{modEXP}\ Y
\end{prooftree}
\\[3.0ex] 
\textsf{(Conditional)}\,\, & 
\begin{prooftree} 
e_1\ \textsf{modEXP}\ Y,\ \ e_2\ \textsf{modEXP}\ Y,\ \ e_3\
\textsf{modEXP}\ Y
\justifies
e_1\texttt{?}e_2\texttt{:}e_3\ \textsf{modEXP}\ Y
\end{prooftree}
\\[3.0ex]
\end{tabular}
\rule{\linewidth}{0.25mm}
\end{table}
%%%%%%%%%%



%%%%%%%%%
\begin{table}
\rule{\linewidth}{0.25mm}
\\[2.5ex]
\begin{tabular}{ll}
\textsf{(Exp-To-Pe)} & 
\begin{prooftree}
e\ \textsf{modPE}\ Y
\justifies
e\ \textsf{modEXP}\ Y
\using
e\in Post\!\!-\! expression
\end{prooftree}
\end{tabular}
\\[2.5ex]
\rule{\linewidth}{0.25mm}
\end{table}
%%%%%%%%%



%%%%%%%%%%%%
\begin{table}
\rule{\linewidth}{0.25mm}
\\[3.0ex]
\begin{tabular}{ll}
\textsf{(Pe-to-Prm)} &
\begin{prooftree}
e\ \textsf{modPRM}\ Y
\justifies
e\ \textsf{modPE}\ Y
\using
e\in \ \textsf{Primary}
\end{prooftree}
\\[3.0ex]
\textsf{(Meth-Inv)}\,\, &
\begin{prooftree}
\sem{e\texttt{.m(}\overrightarrow{\texttt{o}})\texttt{.modifies}}
[\overrightarrow{\texttt{o}}\backslash \overrightarrow{\texttt{q}},
\texttt{this}\backslash e]\sqsubseteq Y,\
e\ \textsf{modPE}\ Y,\
\overrightarrow{\texttt{q}}\ \overrightarrow{\textsf{modEXP}}\ Y
\justifies
e\texttt{.m(}\overrightarrow{\texttt{q}}\texttt{)}\ \textsf{modPE}\ Y
\end{prooftree}
\\[3.0ex]
\textsf{(Pe-To-Ps)} &
\begin{prooftree}
e_1\ \textsf{modPE}\ Y,\ e_2\ \textsf{modPS}\ Y
\justifies
e_1\ \texttt{.}\ e_2\ \textsf{modPE}\ Y
\end{prooftree}
\\[3.0ex]
\end{tabular}
\rule{\linewidth}{0.25mm}
\end{table}
%%%%%%%%%%%%




%%%%%%%%%%%%
\begin{table}
\rule{\linewidth}{0.25mm}
\\[3.0ex]
\begin{tabular}{ll}
\textsf{(Id-Fld)}\,\, &
\begin{prooftree}
\justifies
\texttt{x}\ \textsf{modPRM}\ Y
\end{prooftree}
\\[3.0ex]
\textsf{(Super)}\,\, & 
\begin{prooftree}
\justifies
\texttt{super}\ \textsf{modPRM}\ Y
\end{prooftree}
\\[3.0ex]
\textsf{(This)}\,\, & 
\begin{prooftree}
\justifies
\texttt{this}\ \textsf{modPRM}\ Y
\end{prooftree}
\\[3.0ex]
\textsf{(Static)}\,\, &
\begin{prooftree}
\justifies
\texttt{A}\ \textsf{modPRM}\ Y
\end{prooftree}
\\[3.0ex]
\textsf{(Const)}\,\, &
\begin{prooftree}
\justifies
\texttt{b}\ \textsf{modPRM}\ Y
\using
\texttt{b}\in Constant
\end{prooftree}
\\[3.0ex]
\textsf{(Arr-Fld)}\,\, &
\begin{prooftree}
e\ \textsf{modEXP}\ Y
\justifies
\texttt{a[}e\texttt{]}\ \textsf{modPRM}\ Y
\end{prooftree}
\\[3.0ex]
\textsf{(Meth)}\,\, &
\begin{prooftree}
\sem{\texttt{this.m(}\overrightarrow{\texttt{o}}\texttt{).modifies}}[\overrightarrow{\texttt{o}} 
\backslash \overrightarrow{\texttt{q}}]\sqsubseteq Y,\ \
\overrightarrow{\texttt{q}}\ \overrightarrow{\textsf{modEXP}}\
\textsc{Y}
\justifies
\texttt{m(}\overrightarrow{\texttt{q}}\texttt{)}\ \textsf{modPRM}\
\textsc{Y}
\end{prooftree}
\\[3.0ex]
\textsf{(New-Exp)}\,\, & 
\begin{prooftree}
\overrightarrow{e}\ \overrightarrow{\textsf{modEXP}}\ Y
\justifies
\texttt{new T(}\overrightarrow{e}\texttt{)}\ \textsf{modPRM}\ Y
\end{prooftree}
\\[3.0ex]
\textsf{(New-Arr)}\,\, & 
\begin{prooftree}
e\ \textsf{modEXP}\ Y
\justifies
\texttt{new T[}{e}\texttt{]}\ \textsf{modPRM}\ Y
\end{prooftree}
\end{tabular}
\\[3.0ex]
\rule{\linewidth}{0.25mm}
\end{table}
%%%%%%%%%%





%%%%%%%%%%%%
\begin{table}
\rule{\linewidth}{0.25mm}
\\[3.0ex]
\begin{tabular}{ll}
\textsf{(Id-Fld)}\,\, &
\begin{prooftree}
\justifies
\texttt{x}\ \textsf{modPS}\ Y
\end{prooftree}
\\[3.0ex]
\textsf{(Super)}\,\, & 
\begin{prooftree}
\justifies
\texttt{super}\ \textsf{modPS}\ Y
\end{prooftree}
\\[3.0ex]
\textsf{(This)}\,\, & 
\begin{prooftree}
\justifies
\texttt{this}\ \textsf{modPS}\ Y
\end{prooftree}
\\[3.0ex]
\textsf{(Static)}\,\, &
\begin{prooftree}
\justifies
\texttt{A}\ \textsf{modPS}\ Y
\end{prooftree}
\\[3.0ex]
\textsf{(Arr-Fld)}\,\, &
\begin{prooftree}
e\ \textsf{modEXP}\ Y
\justifies
\texttt{a[}e\texttt{]}\ \textsf{modPS}\ Y
\end{prooftree}
\end{tabular}
\\[3.0ex]
\rule{\linewidth}{0.25mm}
\end{table}
%%%%%%%%%%


\begin{table}
\rule{\linewidth}{0.25mm}
\\[3.0ex]
\begin{tabular}{ll}
\textsf{(Mod-To-Statement)} & 
\begin{prooftree}
e\ \textsf{modSTM}\ Y
\justifies
e\ \textsf{mod}\ Y
\using
e\in Statement
\end{prooftree}
\end{tabular}
\\[3.0ex]
\rule{\linewidth}{0.25mm}
\end{table}
%%%%%%%%%%
%\textsf{(Seq)}\,\, & 
%\begin{prooftree}
%s\ \textsf{modSTM}\ Y,\ \ t\
%\textsf{modSTM}\ Y
%\justifies
%s\texttt{;}t\ \textsf{modSTM}\ Y
%\end{prooftree}
%\\[3.0ex] 

%%%%%%%%%%%%
\begin{table}
\rule{\linewidth}{0.25mm}
\\[3.0ex]
\begin{tabular}{ll}
\textsf{(If-Then)}\,\, & 
\begin{prooftree}
c\ \textsf{modEXP}\ Y,\ \ \overrightarrow{s}\ \overrightarrow{\textsf{mod}}\ Y
\justifies
\texttt{if(}c\texttt{)\{}\overrightarrow{s}\ \texttt{\}}\ 
\textsf{modSTM}\ Y
\end{prooftree}
\\[3.0ex]
\textsf{(If-Then-Else)}\,\, & 
\begin{prooftree}
c\ \textsf{modEXP}\ Y,\ \ \overrightarrow{s}\ \overrightarrow{\textsf{mod}}\ Y,\ \ \overrightarrow{t}\ \overrightarrow{\textsf{mod}}\ Y
\justifies
\texttt{if(}c\texttt{)\{}\overrightarrow{s}\texttt{\}else\{}\overrightarrow{t}\texttt{\}}\
\textsf{modSTM}\ Y
\end{prooftree}
\\[3.0ex]
\textsf{(While)}\,\,\, & 
\begin{prooftree}
c\ \textsf{modEXP}\ Y,\ \ \overrightarrow{s}\ \overrightarrow{\textsf{mod}}\ Y
\justifies
\texttt{while(}c\texttt{)\{}\overrightarrow{s}\texttt{\}}
\end{prooftree}
\\[3.0ex]
\textsf{(Skip)} &  
\begin{prooftree}
\justifies
\texttt{skip}\ \textsf{modSTM}\ Y
\end{prooftree}
\\[3.0ex]
\textsf{(Break)} & 
\begin{prooftree}
\justifies
\texttt{break}\ \textsf{modSTM}\ Y
\end{prooftree}
\\[3.0ex]
\textsf{(Break-Lbl)} &
\begin{prooftree}
\justifies
\texttt{break\! lbl}\ \textsf{modSTM}\ Y
\end{prooftree}
\\[3.0ex]
\textsf{(Continue)} & 
\begin{prooftree} 
\justifies
\texttt{continue}\ \textsf{modSTM}\ Y
\end{prooftree}
\\[3.0ex]
\textsf{(Continue-Lbl)}\,\, & 
\begin{prooftree} 
\justifies
\texttt{continue\! lbl}\ \textsf{modSTM}\ Y
\end{prooftree}
\\[3.0ex]
\textsf{(Return)} & 
\begin{prooftree} \justifies
\texttt{return}\ \textsf{modSTM}\ Y
\end{prooftree}
\\[3.0ex]
\textsf{Return-Exp}\,\, & 
\begin{prooftree} 
e\ \textsf{modEXP}\ Y
\justifies
\texttt{return}\ e\ \textsf{modSTM}\ Y
\end{prooftree}
\end{tabular}
\\[3.0ex]
\rule{\linewidth}{0.25mm}
\end{table}
%%%%%%%%%%




%%%%%%%%%%%%
\begin{table}
\rule{\linewidth}{0.25mm}
\\[3.0ex]
\begin{tabular}{ll}
\textsf{(In-Var)} &
\begin{prooftree}
\texttt{x}\in \textsc{Y}
\justifies
\texttt{x}\underline{\in}Y
\end{prooftree}
\\[3.0ex]
\textsf{(In-Arr)} &
\begin{prooftree}
\texttt{a[}e\texttt{]}\in Y
\justifies
\texttt{a[}e\texttt{]}\underline{\in}Y
\end{prooftree}
\\[3.0ex]
\textsf{(In-Exp)} &
\begin{prooftree}
e\texttt{.x}\in Y
\justifies
e\texttt{.x}\underline{\in} Y
\end{prooftree}
\\[3.0ex]
\textsf{(In-Exp-Arr)}\,\, &
\begin{prooftree}
e\texttt{.a[}e_1\texttt{]}\in Y
\justifies
e\texttt{.a[}e_1\texttt{]}\underline{\in} Y
\end{prooftree}
\\[3.0ex]
\textsf{(In-Fld-Var)} &
\begin{prooftree}
\backslash \texttt{fields\_of(this)}\in Y
\justifies
\texttt{x}\underline{\in} Y
\end{prooftree}
\\[3.0ex]
\textsf{(In-Fld-Arr)}\,\, &
\begin{prooftree}
\backslash \texttt{fields\_of(a)}\in Y
\justifies
\texttt{a[}e_1\texttt{]}\underline{\in} Y
\end{prooftree}
\\[3.0ex]
\textsf{(In-Fld-Exp-Var)}\,\, &
\begin{prooftree}
\backslash \texttt{fields\_of(}e\texttt{)}\in Y
\justifies
e\texttt{.x}\underline{\in} Y
\end{prooftree}
\\[3.0ex]
\textsf{(In-Fld-Exp-Arr)}\,\, &
\begin{prooftree}
\backslash \texttt{fields\_of(}e\texttt{.a)}\in Y
\justifies
e\texttt{.a[}e_1\texttt{]}\underline{\in} Y
\end{prooftree}
\\[3.0ex]
\textsf{(In-Reach-Var)} &
\begin{prooftree}
\backslash
\texttt{fields\_of(}\backslash\texttt{reach(}e\texttt{))}\in Y,\ \
\texttt{this}\in \backslash \texttt{reach(}e\texttt{)} 
\justifies
\texttt{x}\underline{\in} Y
\end{prooftree}
\\[3.0ex]
\textsf{(In-Reach-Arr)} &
\begin{prooftree}
\backslash \texttt{fields}\_\texttt{of(}\backslash
\texttt{reach(}e_1\texttt{))}\in Y,\ \ \texttt{a}\in \backslash
\texttt{reach(}e\texttt{)} 
\justifies
\texttt{a[}e\texttt{]}\underline{\in} Y
\end{prooftree}
\\[3.0ex]
\textsf{(In-Reach-Exp)} &
\begin{prooftree}
\backslash \texttt{fields\_of(}\backslash
\texttt{reach(}e_1\texttt{))}\in Y,\ \ e\in \backslash
\texttt{reach(}e_1\texttt{)}
\justifies
e\texttt{.x}\underline{\in} Y
\end{prooftree}
\\[3.0ex]
\textsf{(In-Reach-Exp-Arr)}\,\, &
\begin{prooftree}
\backslash \texttt{fields\_of(}\backslash
\texttt{reach(}e\texttt{))}\in Y,\ \ e_2\texttt{.a}\in \backslash
\texttt{reach(}e\texttt{)}
\justifies
e_2\texttt{.a[}e_1\texttt{]}\underline{\in} Y
\end{prooftree}
\\[3.0ex]
\textsf{(Times-Arr)}\,\, &
\begin{prooftree}
\texttt{a[}*\texttt{]}\in Y
\justifies
\texttt{a[}e_1\texttt{]}\underline{\in} Y
\end{prooftree}
\\[3.0ex]
\textsf{(Interv-Arr)}\,\, & 
\begin{prooftree}
\texttt{a[}i..j\texttt{]}\in Y,\ i\leq e\leq j
\justifies
\texttt{a[}e\texttt{]}\underline{\in} Y
\end{prooftree}
\end{tabular}
\rule{\linewidth}{0.25mm}
\end{table}
%%%%%%%%%

\[
\begin{tabular}{ll}
\end{tabular}
\]



\end{document}












Hi Marieke,

I have an initial version of the "assignable checker" implementing the rules we defined. I didn't test it properly, but I ran some examples whit it.

I executed this tool with the class 'Jour.java', 'Mois.java', 'Anne.java' and 'Date.java' coming the "purse" distribution. So, you can try it from ~ncatano/verificard/JML. The main class is called CheckModClause and it uses the classes assinable.Assignable assignable.ASTContext and assignable.util (in addition to some other JML classes).

How can one use it ?

1. java CheckModClause file1.java ... filen.java 
  will check file1.java by using file2,...,filen files as 'Context' 

As an example:

"java CheckModClause Date.java Annee.java Mois.java Jour.java DateException.java"
checks Date.java

Some particularities:

1. (Times-Array) and (Interv-Arr) were not considered. The second because you can't know the value of 'e' in executable and the first because it's expressable using '\fields(of)

2. modifies \nothing and modifies \everything were not considered but they will.


Some particularities concerning the example 'Date.java' I executed:

1. I left out all comments from Annee.java Date.java Mois.java Jour.java and DateException.java and all 'modifies' go after 'requires' ones

2. The class constructors  were not considered

3. Class Date.java

 3.1 method getDate : //@modifies bArray[*] --> //@modifies \fields_of(bArray)
 3.2 method getDate : //@assume ...   --> //assume ...
 3.3 method setDate : //@modifies jour  --> //@modifies this.jour (ETC...)
 3.3 method setDate: if(!p) then q else t --> if(p) then t else q
 3.4 method setDate:  Jour.check(j) --> com.gemplus.pacap.utils.Jour.check(j) (ETC)
 3.5 method setDate: jour = j --> this.jour = j (ETC)
 3.6 method setDate: the parameters of ....throwIt(-) was replace by a 'var'.
 3.7 method before: annee --> this.annee (ETC)
 3.8 method after: see 3.7


I'm sending you the output I got from this example.

Finally, I hope keep on improving the tool by means of your inputs (suggestions) and tests.


thanks,
Nestor,
