
While JML is easily accessible to Java developers, actually writing
the specifications of a smart card application is labour-intensive and
error-prone, as it is easy to forget some annotations. There
exist tools which assist in writing these annotations,
\emph{e.g.}~Daikon~\cite{ErnstCGN2001:TSE} and Houdini~\cite{FlanaganL01}
use heuristic methods to produce annotations for simple safety and
functional invariants.  However, these tools cannot be guided by the
user---they do not require any user input---and in particular cannot
be used to synthesise annotations from realistic security policies.

We describe here, a method that, given a security policy,
automatically annotates a Java (Card) application, in such a way that
if the application respects the annotations then it also respects the
security policy. The generation of annotations proceeds in two phases:
synthesising and weaving.
\begin{enumerate}
\item Based on the security policy we \emph{synthesise} core annotations, 
specifying the behaviour of the methods directly involved.
\item Next we propagate these annotations to all methods directly or
indirectly invoking the methods that form the core of the security
policy, thus \emph{weaving} the security policy throughout the
application. 
\end{enumerate} 

To show the usefulness of our approach, we applied the algorithm to
several realistic examples of smart card applications. When doing
this, we actually found violations against the security policies
documented for some of these applications.

This section is organised as follows. Subsection~\ref{SecHighLevelSecProp}
introduces several typical high-level security properties. Next,
Subsection~\ref{SecVerif} presents the process to weave these properties
throughout applications. Subsequently, Subsection~\ref{SecResults}
discusses the application of our method to realistic
examples.






