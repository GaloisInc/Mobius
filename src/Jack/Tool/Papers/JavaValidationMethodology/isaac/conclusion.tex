\subsection{Conclusions}\label{SecConcl} 

We have developed a mechanism to synthesise JML annotations from
high-level security properties. The mechanism has been implemented as
a front-end for tools accepting JML-annotated Java programs; we use
it in combination with JACK. The resulting tool set has been
successfully applied to the area of smart cards, both to verify secure
applications, and to discover programming errors in insecure
ones. Our broad conclusion is that the tool set contributes to
effectively carrying out formal security analyses, while also being
reasonably accessible to security experts without intensive training
in formal techniques.

Currently, we are developing solutions to hide the complexity of
generating core annotations from the user. To this end, we plan to
develop appropriate formalisms for expressing high-level security
properties, and a compiler that translates properties expressed in
these formalisms into appropriate JML core-annotations. Possible
formalisms include security automata, for which appealing visual
representations can be given, or more traditional logics, such as
temporal logic.  In the latter case, we believe that it will be necessary
to rely on a form of security patterns reminiscent of the
specification patterns developed by Dwyer \emph{et
al.}~\cite{DwyerAC98}, and also to consider extensions of JML
with temporal logic~\cite{TrentelmanH02}.


Further, we intend to apply our methods and tools in other contexts,
and in particular for mobile phone applications. In particular, this
will require extending our tools to other Java technologies that,
unlike Java Card, feature recursion and multi-threading.


% that allow to specify the behaviour of an
%application.




 
