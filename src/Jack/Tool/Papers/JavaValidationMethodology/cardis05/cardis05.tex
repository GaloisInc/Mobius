% Pertinence of compiling bytecode into native code for embedded devices
Enabling Java on embedded and restrained systems is an important challenge for today's industry and research groups~\cite{Mulchandani1998}. Java brings features like execution safety and low-footprint program code that make this technology appealing for embedded devices which have obvious memory restrictions, as the success of Java Card witnesses. However, the memory footprint and safety features of Java come at the price of a slower program execution, which can be a problem when the host device already has a limited processing power. As of today, the interest of Java for smart cards is still growing, with next generation operating systems for smart cards that are closer to standard Java systems~\cite{Lagosanto2002,Grimaud2003}, but runtime performance in still an issue. To improve the runtime performances of Java systems, a common practice is to translate some parts of the program bytecode into native code.

% Cost of native code
Doing so removes the interpretation layer and improves the execution speed, but also greatly increases the memory footprint of the program: it is expected that native code is about three to four times the size of its Java counterpart, depending on the target architecture. This is explained by the less-compact form of native instructions, but also by the fact that many safety-checks that are implemented by the virtual machine must be reproduced in the native code. For instance, before dereferencing a pointer, the virtual machine checks whether it is \texttt{null} and, if it is, throws a \texttt{NullPointerException}. Every time a bytecode that implements such safety-behaviors is compiled into native code, these behaviors must be reproduced as well, leading to an explosion of the code size. Indeed, a large part of the Java bytecode implement these safety mechanisms.

% Usefulness of runtime checks
Although the runtime checks are necessary to the safety of the Java virtual machine, they are most of the time used as a protection mechanism against programming errors or malicious code: A runtime exception should be the result of an exceptional, unexpected program behavior and is rarely thrown when executing sane code - doing so is considered poor programming practice. The safety checks are therefore without effect most of the time, and, in the case of native code, uselessly enlarge the code size.

% Our contribution
Several studies proposed to factorize these checks or in some case to eliminate them, but none proposed a complete elimination without hazarding the system security. In this paper, we use formal proofs to ensure that run-time checks can never be true into a program, which allows us to completely and safely eliminate them from the generated native code. The programs to optimize are JML-annotated against runtime exceptions and verified by the Java Applet Correctness Kit (JACK~\cite{BRL-JACK}). We have been able to remove almost all of the runtime checks on tested programs, and obtained native ARM thumb code which size was comparable to the original bytecode.

\subsection{Java and Ahead-of-Time Compilation}
\label{sec:sota}

Compiling Java into native code common on embedded devices. This section gives an overview of the different compilation techniques of Java programs, and points out the issue of runtime exceptions.

\subsubsection{Ahead-of-Time \& Just-in-Time Compilation}

% JIT & AOT compilations
Ahead-of-Time (AOT) compilation is a common way to improve the efficiency of Java programs. It is related to Just-in-Time (JIT) compilation by the fact that both processes take Java bytecode as input and produce native code that the architecture running the virtual machine can directly execute. AOT and JIT compilation differ by the time at which the compilation occurs. JIT compilation is done, as its name states, just-in-time by the virtual machine, and must therefore be performed within a short period of time which leaves little room for optimizations. The output of JIT compilation is machine-language. On the contrary, AOT compilation compiles the Java bytecode way before the program is run, and links the native code with the virtual machine. In other words, it translates non-native methods into native methods (usually C code) prior to the whole system execution. AOT compilers either compile the Java program entirely, resulting in a 100\% native program without a Java interpreter, or can just compile a few important methods. In the latter case, the native code is usually linked with the virtual machine. AOT compilation have no or few time constraints, and can generate optimized code. Moreover, the generated code can take advantage of the C compiler's own optimizations.

% Why JIT is not applicable to embedded devices
JIT compilation in interesting by several points. For instance, there is no prior choice about which methods must be compiled: the virtual machine compiles a method when it appears that doing so is beneficial, e.g. because the method is called often. However, JIT compilation requires embedding a compiler within the virtual machine, which needs resources to work and writable memory to store the compiled methods. Moreover, the compiled methods are present twice in memory: once in bytecode form, and another time in compiled form. While this scheme is efficient for decently-powerful embedded devices such as PDAs, it is inapplicable to very restrained devices like smartcards or sensors. For them, ahead-of-time compilation is usually preferred because it does not require a particular support from the embedded virtual machine outside of the ability to run native methods, and avoids method duplication. AOT compilation has some constraints, too: the compiled methods must be known in advance, and dynamically-loading new native methods is forbidden, or at least very unsafe.

% Runtime exceptions
Both JIT and AOT compilers must produce code that exactly mimics the behavior of the Java virtual machine. In particular, the safety checks performed on some bytecode must also be performed in the generated code.

\subsubsection{Java Runtime Exceptions}
\label{sec:runtimeexceptions}
% Why runtime exceptions
The JVM (Java Virtual Machine)~\cite{VMSpec} specifies a safe execution environment for Java programs. Contrary to native execution, which does not automatically control the safety of the program's operations, the Java virtual machine ensures that every instruction operates safely. The Java environment may throw predefined runtime exceptions at runtime, like the following ones:

If the JVM detects that executing the next instruction will result in an inconsistency or an illegal memory access, it throws a runtime exception, that may be caught by the current method or by other methods on the current stack. If the exception is not caught, the virtual machine exits. This safe execution mode implies that many checks are made during runtime to detect potential inconsistencies. 

Of the 202 bytecodes defined by the Java virtual machine specification, we noticed that 43 require at least one runtime exception check before being executed. While these checks are implicitly performed by the bytecode interpreter in the case of interpreted code, they must explicitly be issued every time such a bytecode is compiled into native code, which leads to a code size explosion. Ishizaki et al. measured that bytecodes requiring runtime checks are frequent in Java programs: for instance, the natively-compiled version of the SPECjvm98 \texttt{compress} benchmark has 2964 exception check sites for a size of 23598 bytes. As for the \texttt{mpegaudio} benchmark, it weights 38204 bytes and includes 6838 exception sites~\cite{Ishizaki1999}. The exception check sites therefore make a non-neglectable part of the compiled code.

\subsection{Optimizing Ahead-of-Time Compiled Java Code}
\label{sec:method}

%verification procedure
Verifying that a bytecode program does not throw Runtime exceptions using JACK involves several stages:
\begin{enumerate}
\item writing the JML specification at the source level of the application, which expresses that no runtime exceptions are thrown.
\item compiling the Java sources and their JML specification\footnote{the BCSL specification is inserted in user defined attributes in the class file and so does not violate the class file format}.
\item generating the verification conditions over the bytecode and its BCSL specification, and proving the verification conditions~\ref{proofs}. During the calculation process of the verification conditions, they are indexed with the index of the instruction in the bytecode array they refer to and the type of specification they prove (e.g. that the proof obligation refers to the exceptional postcondition in case an exception of type \texttt{Exc} is thrown when executing the instruction at index \texttt{i} in the array of bytecode instructions of a given method). Once the verifications are proved, information about which instructions can be compiled without runtime checks is inserted in user defined attributes of the class file.
\item using these class file attributes in order to optimize the generated native code. When a bytecode that has one or more runtime checks in its semantics is being compiled, the bytecode attribute is checked in order to make sure that the checks are necessary. It indicates that the exceptional condition has been proved to never happen, then the runtime check is not generated.
\end{enumerate}

Our approach benefits from the accurateness of the JML specification and from the bytecode verification condition generator. Performing the verification over the bytecode allows to easily establish a relationship between the proof obligations generated over the bytecode and the bytecode instructions to optimized.

In the rest of this section, we explain in detail all the stages of the optimization procedure.

\subsubsection{Methodology for Writing Specification Against Runtime Exception}

We now illustrate with an example which annotations must be generated in order to check if a method may throw an exception. Figure~\ref{fig:jmlexample}\footnote{although the analysis that we describe is on bytecode level, for the sake of readability, the examples are also given on source level} shows a Java method annotated with a JML specification. The method \verb!clear! declared in class \verb!Code_Table! receives an integer parameter \verb!size! and assigns \verb!0! to all the elements in the array field \verb!tab! whose indexes are smaller than the value of the parameter \verb!size!. The specification of the method guarantees that if every caller respects the method precondition and if every execution of the method guarantees its postcondition then the method \verb!clear! never throws an exception of type or subtype \verb!java.lang.Exception!\footnote{Note that every Java runtime exception is a subclass of \texttt{java.lang.Exception}}. This is expressed by the class and method specification contracts.
First, a class invariant is declared which states that once an instance of type \verb!Code_Table! is created, its array field \verb!tab! is not null. The class invariant guarantees that no method will throw a \verb!NullPointerException! when dereferencing (directly or indirectly) \verb!tab!.

\begin{figure}
\begin{verbatim}
final class Code_Table {
  private/*@spec_public */short tab[];

  //@invariant tab != null;

  ...

  //@requires size <= tab.length;
  //@ensures true;
  //@exsures (Exception) false;
  public void clear(int size) {
  1  int code;
  2  //@loop_modifies code, tab[*];
  3  //@loop_invariant code <= size && code >= 0;
  4  for (code = 0; code < size; code++) {
  5    tab[code] = 0;
     }
  }
}
\end{verbatim}

\caption{A JML-annotated method}
\label{fig:jmlexample}
\end{figure}

The method precondition requires the \verb!size! parameter to be smaller than the length of \verb!tab!. The normal postcondition, introduced by the keyword \verb!ensures!, basically says that the method will always terminate normally, by declaring that the set of final states in case of normal termination includes all the possible final states, i.e. that the predicate \verb!true! holds after the method's normal execution\footnote{Actually, after terminating execution the method guarantees that the first \texttt{size} elements of the array tab will be equal to 0, but as this information is not relevant to proving that the method will not throw runtime exceptions we omit it}. On the other hand, the exceptional postcondition for the exception \texttt{java.lang.Exception} says that the method will not throw any exception of type \texttt{java.lang.Exception} (which includes all runtime exceptions). This is done by declaring that the set of final states in the exceptional termination case is empty, i.e. the predicate \texttt{false} holds if an exception caused the termination of the method. The loop invariant says that the array accesses are between index \verb!0! and index \verb!size - 1! of the array \verb!tab!, which guarantees that no loop iteration will cause a \verb!ArrayIndexOutOfBoundsException! since the precondition requires that \verb!size <= tab.length!.

Once the source code is completed by the JML specification, the Java source is compiled using a normal non-optimizing Java compiler that generates debug information like \textrm{LineNumberTable} and \textrm{LocalVariableTable}, needed for compiling the JML annotations. From the resulting class file and the specified source file, the JML annotations are compiled into BCSL and inserted into user-defined attributes of the class file. 

For generating the verification conditions, we use a bytecode verification condition generator (vcGen) based on a bytecode weakest precondition calculus~\cite{JBL05MP}. 

\subsubsection{From Program Proofs to Program Optimizations }
\label{proofs}
In this phase, the bytecode instructions that can safely be executed without runtime checks are identified. Depending on the complexity of the verification conditions, Jack can discharge them to the fully automatic prover Simplify, or to the Coq and AtelierB interactive theorem prover assistants.

There are several conditions to be met for a bytecode instruction to be optimized safely -- the precondition of the method the instruction belongs to must hold every time the method is invoked, and the verification condition related to the exceptional termination must also hold. In order to give a flavor of the verification conditions we deal with, figure~\ref{proofs:vc} shows part of the  verification condition related to the possible \texttt{ArrayIndexOutOfBounds} exceptional termination of instruction \verb!11 sastore! in figure~\ref{bcsl}, which is actually provable.

\begin{figure}
\[
  \begin{array}{ll}
    \begin{array}{l}
        \ldots \\
     \texttt{length} ( \texttt{tab}(  \register{0}) \leq \register{2}_{15} \vee \register{2}_{15}< 0   \\
       \wedge \\
        \register{2}_{15 } \geq 0 \\
	\wedge \\
       \register{2}_{15 }< \register{1} \\
       \wedge \\
        \register{1} \leq  \texttt{length} ( \texttt{tab}( \register{0} ) ) \\
        \end{array}
     &  \Rightarrow  \Myfalse
  \end{array}\]
\caption{The verification condition for the \texttt{ArrayIndexOutOfBoundException} check related to the \texttt{sastore} instruction of figure~\ref{bcsl}}
\label{proofs:vc}
\end{figure}

Once identified, proved instructions can be marked in user-defined attributes of the class file so that the compiler can find them.

\subsubsection{More Precise Optimizations}

\label{section:optimprecise}

As we discussed earlier, in order to optimize an instruction in a method body, the method precondition must be established at every call site and the method implementation must be proved not to throw an exception under the assumption that the method precondition holds. This means that if there is one call site where the method precondition is broken then no instruction in the method body will be optimized.

Actually, the analysis may be less conservative and therefore more precise. We illustrate with an example how
one can achieve more precise results.

Consider the example of figure \ref{fig:jmlpreciseex}. On the left side of the figure, we show source code for method \verb!setTo0! which sets the \verb!buff! array element at index \verb!k! to 0. On the right side, we show the bytecode of the same method. The \texttt{iastore} instruction at index \texttt{3} may throw two different runtime exceptions: \texttt{NullPointerException}, or \texttt{ArrayIndexOutOfBoundException}. For the method execution to be safe (i.e. no Runtime exception is thrown), the method requires some certain conditions to be fulfilled by its callers. Thus, the method's precondition states that the \verb!buff! array parameter must not be null and that the \verb!k! parameter must be inside the bounds of \verb!buff!. If at all call sites we can establish that the \verb!buff! parameter is always different from null, but there are sites at which an unsafe parameter \verb!k! is passed the optimization for \texttt{NullPointerException} is still safe although the optimization for \texttt{ArrayIndexOutOfBoundException} is not possible. In order to obtain this kind of preciseness, a solution is to classify the preconditions of a method with respect to what kind of runtime exception they protect the code from. For our example, this classification consists of two groups of preconditions. The first is related to \texttt{NullPointerException}, i.e. \texttt{buff != null} and the second consists of preconditions related to \texttt{ArrayIndexOutOfBoundException}, i.e. \verb! k >= 0 && k <= buff.length!. Thus, if the preconditions of one group are established at all call sites, the optimizations concerning the respective exception can be performed even if the preconditions concerning other exceptions are not satisfied.

\begin{figure}
\begin{minipage}[b]{0.5\linewidth}
\begin{verbatim}
...

//@requires buff != null;
//@requires k >= 0 ;
//@requires k <= buff.length;
//@ensures true;
//@exsures (Exception) false;
public void setTo0(int k,int[] buff)
{
  buff[k] = 0;
}
\end{verbatim}
\end{minipage}
\hspace{.5cm}
\begin{minipage}[b]{0.4\linewidth}
 \begin{verbatim}
 0 aload_2
 1 iload_1
 2 iconst_0
 3 iastore
 4 return
\end{verbatim}
\end{minipage}
\caption{The source code and bytecode of a method that may throw several exceptions}
\label{fig:jmlpreciseex}
\end{figure}

\subsection{Experimental Results}
\label{sec:experiments}

This section presents an application and evaluation of our method on various Java programs.

\subsubsection{Methodology}

We have measured the efficiency of our method on two kinds of programs, that implement features commonly met in restrained and embedded devices. \benchname{crypt} and \benchname{banking} are two smartcard-range applications. \benchname{crypt} is a cryptography benchmark from the Java Grande benchmarks suite, and \benchname{banking} is a little banking application with full JML annotations used in~\cite{BRL-JACK}. \benchname{scheduler} and \benchname{tcpip} are two embeddable system components written in Java, which are actually used in the JITS~\cite{JITSWebsite} platform. \benchname{scheduler} implements a threads scheduling mechanism, where scheduling policies are Java classes. \benchname{tcpip} is a TCP/IP stack entirely written in Java, that implements the TCP, UDP, IP, SLIP and ICMP protocols. These two components are written with low-footprint in mind ; however, the overall system performance would greatly benefit from having them available in native form, provided the memory footprint cost is not too important.

For every program, we have followed the methodology described in section \ref{sec:method} in order to prove that runtime exceptions are not thrown in these programs. We look at both the number of runtime exception check sites that we are able to remove from the native code, and the impact on the memory footprint of the natively-compiled methods with respect to the unoptimized native version and the original bytecode. The memory footprint measurements were obtained by compiling the C source file generated by the JITS AOT compiler using GCC 4.0.0 with optimization option \texttt{-Os}, for the ARM platform in thumb mode. The native methods sizes are obtained by inspecting the .o file with \texttt{nm}, and getting the size for the symbol corresponding to the native method.

Regarding the number of eliminated exception check sites, we also compare our results with the ones obtained using the JC virtual machine mentioned in~\ref{sec:relatedwork}, version 1.4.6. The results were obtained by running the \texttt{jcgen} program on the benchmark classes, and counting the number of explicit exception check sites in the generated C code. We are not comparing the memory footprints obtained with the JITS and JC AOT compilers, for this result would not be pertinent. Indeed, JC and JITS have very different ways to generate native code. JITS targets low memory footprint, and JC runtime performance. As a consequence, a runtime exception check site in JC is heavier than one in JITS, which would falsify the experiments. Suffices to say that our approach could be applied on any AOT compiler, and that the most relevant measurement is the number of runtime exception check sites that remains in the final binary - our measurements on the native code memory footprint are just here to evaluate the size impact of exception check sites.

\subsubsection{Results}
\label{results}
Table \ref{tab:nbexcsites} shows the results obtained on the four tested programs. The three first columns indicate the number of check sites present in the bytecode, the number of explicit check sites emitted by JC, and the number of check sites that we were unable to prove useless and that must be present in our optimized AOT code. The last columns give the memory footprints of the bytecode, unoptimized native code, and native code from which all proved exception check sites are removed.

\begin{table}
\caption{Number of exception check sites and memory footprints when compiled for ARM thumb}
\begin{center}
  \begin{tabular}{|l|r@{\extracolsep{0.2cm}}rrrrr|}
    \hline
    \multirow{2}*{Program} & \multicolumn{3}{c}{\# of exception check sites} & \multicolumn{3}{c|}{Memory footprint (bytes)}\\
    \cline{2-4} \cline{5-7} & Bytecode & ~~~~~~JC & Proven AOT & Bytecode & Naive AOT & Proven AOT\\
    \hline
    \benchname{crypt} & 190 & 79 & 1 & 1256 & 5330 & 1592\\
    \benchname{banking} & 170 & 12 & 0 & 2320 & 5634 & 3582\\
    \benchname{scheduler} & 215 & 25 & 0 & 2208 & 5416 & 2504\\
    \benchname{tcpip} & 1893 & 288 & 0 & 15497 & 41540 & 18064\\
    \hline
  \end{tabular}
\end{center}
\label{tab:nbexcsites}
\end{table}

On all the tested programs, we were able to prove that all but one exception check site could be removed. The only site that we were unable to prove from \benchname{crypt} is linked to a division, which divisor is a computed value that we were unable to prove not equal to zero. JC has to retain 16\% of all the exception check sites, with a particular mention for \benchname{crypt}, which is mainly made of array accessed and had more remaining check sites.

The memory footprints obtained clearly show the heavy overhead induced by exception check sites. Despite of the fact that the exception throwing convention has deliberately been simplified for our experiments, optimized native code is less than half the size of the non-optimized native code. The native code of \benchname{crypt}, which heavily uses arrays, is actually made of exception checking code at 70\%.

Comparing the size of the optimized native versions with the bytecode reveals that proved native code is just slightly bigger than bytecode. The native code of \benchname{crypt} is 27\% bigger than its bytecode version. Native \benchname{scheduler} only weights 13.5\% more that its bytecode, \benchname{tcpip} 16.5\%, while \benchname{banking} is 54\% heavier. This last result is explained by the fact that, being an application and not a system componant, \benchname{banking} includes many native-to-java method invokations for calling system services. The native-to-java calling convention is costly in JITS, which artificially increases the result.

Finally, table \ref{tab:implication} details the human work required to obtain the proofs on the benchmark programs, by comparing the amount of JML code with respect to the comments-free source code of the programs. It also details how many lemmas had to be manually proved.

\begin{table}
\caption{Human work on the tested programs}
\begin{center}
  \begin{tabular}{|l|r@{\extracolsep{0.5cm}}rrr|}
    \hline
    \multirow{2}*{Program} & \multicolumn{2}{c}{Source code size (bytes)} & \multicolumn{2}{c|}{Proved lemmas}\\
    \cline{2-3} \cline{4-5} & ~~~~~~~~~Code & JML & Automatically & Manually\\
    \hline
    \benchname{crypt} & 4113 & 1882 & 227 & 77 \\
    \benchname{banking} & 11845 & 15775 & 379 & 159\\
    \benchname{scheduler} & 12539 & 3399 & 226 & 49\\
    \benchname{tcpip} & 83017 & 15379 & 2233 & 2191\\
    \hline
  \end{tabular}
\end{center}
\label{tab:implication}
\end{table}

On the three programs that are annotated for the unique purpose of our study, the JML overhead is about 30\% of the code size. The \benchname{banking} program was annotated in order to prove other properties, and because of this is made of more JML annotations than actual code. Most of the lemmas could be proved by Simplify, but a non-neglectable part needed human-assistance with Coq. The most demanding application was the TCP/IP stack. Because of its complexity, nearly half of the lemmas could not be proved automatically.

The gain in terms of memory footprint obtained using our approach is therefore real. One may also wonder whether the runtime performance of such optimized methods would be increased. We did the measurements, and only noticed a very slight, almost undetectable, improvement of the execution speed of the programs. This is explained by the fact that the exception check sites conditions are always false when evaluated, and therefore the amount of supplementary code executed is very low. The bodies of the proved runtime exception check sites are, actually, dead code that is never executed.


