We propose a bytecode verification framework with the following components: a bytecode specification language, a compiler from source
 program annotations into bytecode annotations and a verification condition generator over Java bytecode.

In a client-producer scenario, these features bring to the producer means to supply the sufficient specification information 
which will allow the client to establish trust in the code, especially when the client policy is potentially complex and a fully automatic specification inference
will fail. On the other hand, the client is supplied with a procedure to check the untrusted annotated code. 

  

Our approach is tailored to Java bytecode.
The Java technology is widely applied to mobile and embedded components because of its portability across platforms. 
For instance, its dialect JavaCard is largely used in smart card applications and the J2ME Mobile Information Device Profile 
(MIDP) finds application in GSM mobile components. 
%In this article, we propose a static verification technique using formal methods for sequential Java bytecode programs.

The proposed scheme is composed of several components.
 We define a bytecode specification language, called BCSL, and supply a compiler from 
 the high level Java specification language JML~\cite{JMLRefMan} to BCSL. 
 BCSL supports a JML subset which is expressive enough to specify rich functional properties. 
The specification is inserted in the class file format in newly defined attributes and thus makes not
 only the code mobile but also its specification. These class
 file extensions do not affect the JVM performance.
We define a bytecode logic in terms of weakest precondition calculus for the sequential Java bytecode language. 
The logic gives rules for almost all Java bytecode instructions and supports the Java specific features such as
exceptions, references, method calls and subroutines.  
 We have implementations of a verification condition generator based on the weakest precondition calculus and of
 the JML specification compiler. Both are integrated in the Java Applet Correctness Kit tool (JACK)~\cite{BRL-JACK}.

  The full specifications of the JML compiler, the weakest precondition predicate transformer definition and its proof of correctness can be found in~\cite{JBL05MP}.
  
The remainder of this section is organized as follows: 
Subsection~\ref{architecture_s} reviews scenarios in which the architecture may be appropriate to use; 
 Subsection~\ref{bcSpecLg} presents the bytecode specification language BCSL and the JML compiler; Subsection~\ref{wpbc} discusses the main
features of the \wpi (short for weakest precondition calculus); Subsection~\ref{pogEquiv} discusses the relationship between the verification conditions for JML annotated source and BCSL annotated bytecode.















