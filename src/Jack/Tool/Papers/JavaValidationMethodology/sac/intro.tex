Establishing trust in software components that originate from untrusted or unknown producers is an important issue in areas such as
 smart card applications, mobile phones, bank cards, ID cards and whatever scenario where untrusted code should be installed and executed.
%The present paper addresses the  problem of establishing trust in software components that originate from untrusted or unknown producers. 
%The question concerns important areas like 
%smart card applications, mobile phones, bank cards, ID cards and whatever scenario where untrusted code should be installed and executed.

In particular, the state of the art proposes different solutions. For example,
 the verification may be performed over the source code. In this case, the code receiver should make the compromise to trust the compiler, which is problematic. Bytecode verification techniques \cite{Ljbc} are another solution, which does not require to trust the compiler. 
The bytecode verifier performs static analysis directly over the bytecode yet, it can only guarantee that 
the code is well typed and well structured. %that the bytecode does not violate the proper function of the virtual machine. 
The Proof Carrying Code paradigm (PCC) and the certifying compiler \cite{DesNecLee98} are another alternative.
 In this architecture, untrusted code
is accompanied by a proof for its safety w.r.t. to some safety property and the code receiver has just to generate the 
verification conditions and type check the proof against them. 
The proof is generated automatically by the certifying compiler for properties like well typedness or safe memory access. 
As the certifying compiler is designed to be completely automatic, it will not be able to deal with rich functional or security properties. 
 
We propose a bytecode verification framework with the following components: a bytecode specification language, a compiler from source
 program annotations into bytecode annotations and a verification condition generator over Java bytecode.
%\begin{itemize}
%  \item a bytecode specification language and a compiler from source program annotations into bytecode annotations.
%   Thus, bytecode can benefit from the source specification but does not need to be accompanied by its source code. 
%     \item verification condition generator over Java bytecode, which supports
%  the bytecode annotation. 
%  \end{itemize}
%  

In a client-producer scenario, these features bring to the producer means to supply the sufficient specification information 
which will allow the client to establish trust in the code, especially when the client policy is potentially complex and a fully automatic specification inference
will fail. On the other hand, the client is supplied with a procedure to check the untrusted annotated code. 

  

Our approach is tailored to Java bytecode.
The Java technology is widely applied to mobile and embedded components because of its portability across platforms. 
For instance, its dialect JavaCard is largely used in smart card applications and the J2ME Mobile Information Device Profile 
(MIDP) finds application in GSM mobile components. 
%In this article, we propose a static verification technique using formal methods for sequential Java bytecode programs.

The proposed scheme is composed of several components.
 We define a bytecode specification language, called BCSL, and supply a compiler from 
 the high level Java specification language JML~\cite{JMLRefMan} to BCSL. 
 BCSL supports a JML subset which is expressive enough to specify rich functional properties. 
The specification is inserted in the class file format in newly defined attributes and thus makes not
 only the code mobile but also its specification. These class
 file extensions do not affect the JVM performance.
We define a bytecode logic in terms of weakest precondition calculus for the sequential Java bytecode language. 
The logic gives rules for almost all Java bytecode instructions and supports the Java specific features such as
exceptions, references, method calls and subroutines.  
 We have implementations of a verification condition generator based on the weakest precondition calculus and of
 the JML specification compiler. Both are integrated in the Java Applet Correctness Kit tool (JACK)~\cite{BRL-JACK}.

  The full specifications of the JML compiler, the weakest precondition predicate transformer definition and its proof of correctness can be found in~\cite{JBL05MP}.
  
The remainder of the paper is organized as follows: 
Section~\ref{architecture_s} reviews scenarios in which the architecture may be appropriate to use; 
 Section~\ref{bcSpecLg} presents the bytecode specification language BCSL and the JML compiler; Section~\ref{wpbc} discusses the main
features of the \wpi (short for weakest precondition calculus); Section~\ref{pogEquiv} discusses the relationship between the verification conditions for JML annotated source and BCSL annotated bytecode; Section~\ref{conclusion} concludes with future work.  















