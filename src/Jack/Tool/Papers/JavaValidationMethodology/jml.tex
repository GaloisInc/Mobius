\chapter{Java Validation}
\section{Java}
\section{JML}
JML~\cite{Leavens-Baker-Ruby99b,Leavens-Baker-Ruby03}, the
``Java Modeling Language'', is useful for specifying detailed
designs of Java classes and interfaces.  JML is a behavioral interface
specification language for Java; that is, it specifies both the behavior
and the syntactic interface of Java code.  The syntactic interface of
a Java class or interface consists of its method signatures,
the names and types of its fields, etc.
This is what is commonly meant by an application programming
interface (API).
The behavior of such an API can be precisely documented in JML annotations;
these describe the intended way that programmers should
use the API.  In terms of behavior, JML can detail, for example, the
preconditions and postconditions for methods as well as class
invariants.

An important goal for the design of JML is that it should be easily
understandable by Java programmers. This is achieved by staying as
close as possible to Java syntax and semantics.  Another important
design goal is that JML {\em not} impose any particular design method
on users; instead, JML should be able to document Java programs
designed in any manner \cite{Leavens-Baker-Ruby03}.

The work on JML was started by Gary Leavens and his colleagues and
students at Iowa State University, but has grown into a cooperative,
open effort.  Several groups worldwide are now building tools that
support the JML notation and are involved with the ongoing design of
JML\@.  The open, cooperative nature of the JML effort is important
both for tool developers and users, and we welcome participation by
others.  For potential users, the fact that there are several tools
supporting the same notation is clearly an advantage.  For tool
developers, using a common syntax and semantics can make it much
easier to get users interested. After all, one of the biggest hurdles
to using a new specification-centric tool is often the lack of
familiarity with the associated specification language.

\medskip

The next section introduces the JML notation and the section~\ref{related} discusses some related
languages and tools, such as OCL and other runtime assertion checkers.

%=====================================================================
\subsection{The JML Notation}
\label{notation}

JML blends Eiffel's Design by Contract approach~\cite{Meyer97} with
the Larch~\cite{Guttag-Horning93} tradition (and others that space
precludes mentioning).  Because JML supports quantifiers such as
\verb_\forall_ and \verb_\exists_, and because JML allows ``model''
(i.e., specification-only) fields and methods, specifications can
easily be made more precise and complete than is typical for Eiffel software.
However, like Eiffel, JML uses Java's expression syntax in assertions,
thus JML's notation is easier for programmers to learn than notations
based on a language-independent specification language, like the Larch Shared
Language~\cite{Leavens-Baker-Ruby03,Leavens-etal03a} or
OCL~\cite{WarmerKleppe99}.

%% Note that I have used the newly specified constructor to illustrate
%% a point about ESC/Java below, so beware if you try to save space by
%% deleting it. -- Gary

JML assertions are written as special
annotation comments in Java code,
either after \verb_//@_ or between \verb_/*@ ... @*/_,
so that they are ignored by Java compilers but can be used
by tools that support JML\@.  Within annotation comments JML extends the
Java syntax with several keywords.  It also extends Java's expression syntax with several
operators; these begin with a backslash so they do not
clash with existing Java identifiers.

The central ingredients of a JML specification are preconditions
(given in {\tt requires} clauses), postconditions (given in {\tt
  ensures} clauses), and (class and interface) invariants.  These are
all expressed as boolean expressions in JML's extension to Java's
expression syntax.

In addition to ``normal'' postconditions, the language also supports
``exceptional'' postconditions, specified in {\tt signals} clauses.
These can be used to specify what must be true when a method throws an
exception. 

\medskip


%% [[[I wonder if, in this journal version, the model variable
%% feature of JML deserves a more systematic treatment. -- Gary]]]
\begin{itemize}
\item Model variables, which play the role of abstract values for
  abstract data types~\cite{Cheon-etal03}, allow specifications that
  hide implementation details.
    
\item An extensive library that provides Java types that can be used
  for describing behavior mathematically.  This library includes such
  concepts as sets, sequences, and relations.  It is similar to
  similar libraries of mathematical concepts found in VDM, Z, LSL, or
  OCL, but allows such concepts to be used directly in assertions,
  since they are embodied as Java objects.
  
\item A semantics that prevents side-effects in assertions.  This
  both allows assertion checks to be used safely during debugging, and
  supports mathematical reasoning about assertions.
  This semantics works conservatively, by allowing a method to only be
  used in assertions only if it is declared as 
  {\tt pure}, meaning the method does not have any side-effects and
  does not perform any input or output \cite{Leavens-Baker-Ruby03}.
%\footnote{Unfortunately, some methods that one would like to be pure,
%  such as String's equals method, are not.  This is a topic of current
%  research.}
\begin{verbatim}
  /*@ pure @*/ int getBalance() { ... }
\end{verbatim}
  then this method can be used in the specification instead of the field
  {\tt balance}.
\item Finally, JML supports all the Java modifiers ({\tt public},
  {\tt protected}, and {\tt private}) for expressing visibility.  For
  example, invariants can be declared {\tt protected} if
  they are not observable by clients but intended for use by
  programmers of subclasses.  
\end{itemize}

\subsection{JML vs.\ OCL}
\label{related}

Despite the similarity in the acronyms, JML is {\em very} different in
its aims from UML~\cite{RumbaughJacobsonBooch98}.
The most basic difference is that the UML aims to cover all phases of
analysis and design with many notations, and tries to be independent
of programming language, while JML only deals with detailed designs
(for APIs) and is tied to Java.  The ``model'' in
JML refers to abstract, specification-only fields that can be used to
describe the behavior of various types.  By contrast, the ``model'' of
UML refers to the general modeling process (analysis and design) and
is not limited to abstractions of individual types.
%% [[[Text cut, as it's incorporated above in a better form, I
%% hope. -- Gary]]]
% Unlike UML, which attempts
% to have notations for all phases of the analysis and design process,
% JML has the much more modest aim of describing the behavior of Java
% classes and interfaces and recording detailed design and
% implementation decisions.

JML does have some things in common with the Object Constraint
Language (OCL)~\cite{WarmerKleppe99}, which is part of the UML
standard.  Like JML, OCL can be used to specify invariants and pre-
and postconditions.  An important difference is that JML explicitly
targets Java, whereas OCL is not specific to any one programming
language.  One could say that JML is related to Java in the same way
that OCL is related to UML\@.

JML clearly has the disadvantage that it can not be used for, say, C++
programs, whereas OCL can.  But it also has obvious advantages when it
comes to syntax, semantics, and expressivity.  Because JML sticks to
the Java syntax and typing rules, a typical Java programmer will
prefer JML notation over OCL notation, and, for instance, prefer to
write (in JML):
\begin{verbatim}
     invariant pin != null && pin.length == 5;
\end{verbatim}
rather than the OCL:
\begin{verbatim}
     inv: pin <> null and pin->size() = 5
\end{verbatim}

JML supports all the Java modifiers such as {\tt static}, {\tt
  private}, {\tt public}, etc., and these can be used to record
detailed design decisions for different readers.
Furthermore, there are legal Java
expressions that can be used in JML specifications but that cannot be
expressed in OCL\@.
%% [[[Like what?  Can we give an example?]]]
% Thus JML can specify more details about Java programs than would be
% possible in OCL\@.

More significant than these limitations, or differences in syntax, are
differences in semantics.  JML builds on the (well-defined) semantics
of Java. So, for instance, {\tt equals} has the same meaning in JML
and Java, as does {\tt ==}, and the same rules for overriding,
overloading, and hiding apply.  One cannot expect this for OCL\@.  In
fact, a semantics for OCL was only recently
proposed~\cite{brucker.ea:proposal:2002}.

In all, we believe that a language like JML, which is tailored to
Java, is better suited for recording the detailed design of a Java
programs than a generic language like OCL\@.  Even if one uses UML in
the development of a Java application, it may be better to use JML
rather than OCL for the specification of object constraints,
especially in the later stages of the development.

%=====================================================================
%=====================================================================
\subsection{The JML tool suite}
\label{tools}

Since JML specifications are meant to be read and written by ordinary
Java programmers, it is important to support the conventional ways
that these programmers create and use documentation.  Consequently,
the {\tt jmldoc} tool
produces browsable HTML pages containing both the
API and the specifications for Java code, in the style of pages
generated by Javadoc~\cite{Friendly95}.

%=====================================================================
The JML compiler (\texttt{jmlc}), developed at Iowa State University,
is an extension to a Java compiler and compiles Java programs
annotated with JML specifications into Java
bytecode~\cite{Cheon03,Cheon-Leavens02b}.  The compiled bytecode includes
runtime assertion checking instructions that check JML specifications
such as preconditions, normal and exceptional postconditions,
invariants, and history constraints.  The execution of such assertion
checks is transparent in that, unless an assertion is violated, and
except for performance measures (time and space), the behavior of the
original program is unchanged.  The transparency of runtime assertion
checking is guaranteed, as JML assertions are not allowed to have any
side-effects~\cite{Leavens-etal03a}.





\section{Extended Static Checking with ESC/Java}
\label{escjava}

ESC/Java tool~\cite{Flanagan-Et-Al02}, developed at Compaq Research,
performs what is called ``extended static
checking''~\cite{ESC:Overview,10yearsESC},
% KRML: I would prefer \emph{extended static checking}, but such a
% change would probably require similar changes to other places of the
% document where double quotes are used.  For example, one would
% probably change ``model'' in:
%     JML allows ``model'' (i.e., specification-only) fields
% to \emph{model}.  Ditto for ``candidate annotations'' below.
% [[[I'd also like that... -- Gary]]]
compile-time checking that goes well beyond type checking.  It can
check relatively simple assertions and can check for certain kinds of
common errors in Java code, such as dereferencing \texttt{null},
indexing an array outside its bounds, or casting a reference to an
impermissible type.  ESC/Java supports a subset of JML and also checks
the consistency between the code and the given JML annotations.  The
user's interaction with ESC/Java is quite similar to the interaction
with the compiler's type checker: the user includes JML annotations in
the code and runs the tool, and the tool responds with a list of
possible errors in the program.

JML annotations affect ESC/Java in two ways.  First, the given JML
annotations help ESC/Java suppress spurious warning messages.   Second,
annotations make ESC/\-Java do additional checks.  
In these two ways, the use of JML annotations enables ESC/Java to
produce warnings not at the source locations where errors manifest
themselves at runtime, but at the source locations where the errors
are committed.

An interesting property of ESC/Java is that it is neither sound nor
complete; that is, it neither warns about all errors, nor does it
warn only about actual errors.  This is a deliberate design choice:
the aim is to increase the cost-effectiveness of the tool.  In some
situations, convincing a mechanical checker of the absence of some
particular error may require a large number of JML annotations
(consider, for example, a hypothetical program that dereferences
\texttt{null} if four of the program's large-valued integer variables
satisfy the equation in Fermat's Last Theorem).  To make the tool more
cost-effective, it may therefore be prudent to ignore the possibility
of certain errors, which is what ESC/Java has been designed to do.
The ESC/Java User's Manual~\cite{escjava:userman} contains a list of
all cases of unsoundness and incompleteness in ESC/Java.

Under the hood, ESC/Java is powered by detailed program semantics and
an automatic (non-interactive) theorem prover.  ESC/Java translates a
given JML-an\-no\-tat\-ed Java program into verification
conditions~\cite{LeinoSaxeStata:JavaViaGC,FlanaganSaxe:POPL01},
logical formulas that are valid if and only if the program is free of
the kinds of errors being analyzed.  Any verification-condition
counterexamples found by the theorem prover are turned into
programmer-sensible warning messages, including the kind and source
location of each potential error.  The ESC/Java User's
Manual~\cite{escjava:userman} also provides a detailed description of
the semantics of JML annotations, as they pertain to ESC/Java.

\subsection{ESC/Java2}
\label{escjava2}

Development of version 1 of ESC/Java ceased with the dissolving of the
SRC group at Compaq.  Consequently Cok and Kiniry have in progress a
version 2 of ESC/Java, built on the source code release provided by
Compaq and HP.  This version has the following goals:
\begin{itemize}
\item to migrate the code base of ESC/Java and the code accepted by
  ESC/Java to Java 1.4;
\item to update ESC/Java to accept annotations consistent with current
  version of JML;
\item to increase the general use of the tool by packaging it in an
  easy-to-use form;
\item to increase the amount of JML that ESC/Java statically checks,
  consistent with the original engineering goals of ESC/Java;
\item and, over time, to update the associated tools of the ESC suite
  (Calvin, Houdini, RCC) in a similar manner.
\end{itemize}

There is currently an alpha version of ESC/Java2 available
(from \url{http://www.cs.kun.nl/ita/research/} 
\url{projects/sos/projects/escjava.html}).
This release includes the following improvements with respect to the original
ESC/\-Java.

%% [[[I worried that the following had too much detail for most
%% readers. I'm especially thinking about the second point below,
%% which detailed all the features of JML that are not thoroughly type
%% checked. I edited to try to correct this. --Gary]]]

\begin{itemize}
\item Parses Java 1.4 (the old version only parsed Java 1.3). 
  In particular ESC/Java2 handles the Java
  \verb$assert$ statement.  The user has the option of treating the
  statement as in Java, namely, as a conditional statement that
  might throw an exception.  Alternatively, ESC/Java2 will translate
  the Java \verb$assert$ statement into a JML \verb$assert$ statement,
  producing a static warning if the condition in the
  statement appears not to be always true.
\item Parses all of current JML.  This is a somewhat moving target,
  since JML is the subject of current discussion and research.
  Nevertheless the core part of JML is stable and that is the portion
  that ESC/Java2 attempts to statically check.  Some
  of the more esoteric features of JML (e.g. model programs)
  % the clauses in the \verb$code_contract$
  % section, the resource clauses \verb$\working_space$, \verb$space$,
  % \verb$\duration$, and the frame and datagroup features)
  are only parsed, are not thoroughly type checked, and are ignored for
  purposes of static checking.
\item Allows specifications to be placed in (multiple) files separate from the
  implementation, using JML's refinement features.  ESC/Java2 makes checks
  using the appropriate combination of all available specifications and
  implementations.  It also checks these specifications for consistency.
\item Follows the JML semantics for specification inheritance.
  The constructs specific to ESC/Java version 1
  (\verb$also_requires$, etc.)
  % \verb$also_ensures$, \verb$also_modifies$, and \verb$also_exsures$).
  were dropped.
\end{itemize}

Many other details in the static checking have also been improved in
ESC/Java2, making it more suitable for general use.

%% [[[Cut text below:]]]
%   The integration and inheritance of
%   \verb$non_null$ annotations was also improved.
% \item The static checking provided in ESC/Java has been expanded in
%   ESC/Java2 in these ways: checking of the \texttt{assert} statement;
%   checking of \texttt{initially} and \texttt{constraint} clauses;
%   improvements in handling \texttt{modifies} (\texttt{assignable})
%   clauses; checking of \verb$\not_modified$ expressions; allowing and
%   checking the application of \verb$\typeof$ to primitive types;
%   handling of \texttt{forall} and \texttt{old} method specification
%   clauses.

There are two major areas of development of ESC/\-Java2 that will
improve overall usability of the tool, besides performance
improvements.  The first is the use of model variables and method
calls in annotation expressions.  Model variables are an important
abstraction mechanism in writing specifications and model methods
allow much more readable and compact specifications.  This is a
current topic of research and experimentation and some partial
facilities in this direction are a part of the current alpha release
of ESC/Java2.

The second important needed advance is to provide some level of
checking of the frame conditions (i.e., JML's \texttt{modifies} or
\texttt{assignable} clauses).  It is an acknowledged unsoundness of
ESC/Java that these are not checked and faulty \texttt{modifies}
clauses can be a subtle source of errors.  In particular, the default
\texttt{modifies} clause in some JML specifications is that everything
is potentially modified; this interpretation is not currently
implemented.

\section{Program Verification with LOOP}
\label{loop}

The University of Nijmegen's LOOP tool~\cite{Jacobs-etal98,Huisman01}
translates JML-annotated Java code into proof obligations for the
theorem prover PVS~\cite{OwreRRSS96}.  One can then try to prove these
obligations, interactively, in PVS\@.  The translation from JML to
formal proof obligations builds on a formal semantics for sequential
Java that has has been formalized in PVS, and which has been extended
to provide a formal semantics for a large part of JML\@.  The
verification of the proof obligations is accomplished using a Hoare
Logic~\cite{JacobsPoll01a} and a weakest-precondition
calculus~\cite{Jacobs03} for Java and JML\@.

Interactive theorem proving is very labor-intensive, but allows
verification of more complicated properties than can be handled by
extended static checking with ESC/Java.  A recent paper describing a
case study with the LOOP tool, giving the best impression of the state
of the art, is now available~\cite{BreunesseBJ02}.

%% It seems there isn't enough to say about this to justify its own section.
 \section{Krakatoa}
 \label{krakatoa}
 
 Krakatoa~\cite{krakatoa03a} is a
 program verification tool for JML-annotated code that is currently
 under development.  It is similar to LOOP, but produces proof
 obligations for the theorem prover Coq, and
 currently covers only a subset of Java.

%=====================================================================
\section{Applications of JML to Java Card}
\label{applications}

Although JML is able to specify arbitrary sequential Java programs,
most of the serious applications of JML and JML tools up to now
have targeted Java Card.  Java Card$^{TM}$ is a dialect of Java specifically
designed for the programming of the latest generation of smartcards.
Java Card is adapted to the hardware limitations of smartcards; for
instance, it does not support floating point numbers, strings, object
cloning, or threads.

Java Card is a well-suited target for the application of formal
methods.  It is a relatively simple language with a restricted API\@.
Moreover, Java Card programs, called ``applets'', are small, typically
on the order of several KBytes of bytecode.  Additionally, correctness
of Java Card programs is of crucial importance, since they are used in
sensitive applications such as bank cards and mobile phone SIMs.  (An
interesting overview of security properties that are relevant for Java
Card applications is available~\cite{MarletLM01}.)

JML, and several tools for JML, have been used for Java Card,
especially in the context of the EU-supported project VerifiCard
(www.verificard.org).  JML has been used to write a formal
specification of almost the entire Java Card
API~\cite{PollBergJacobs01}.  This experience has shown that JML is
expressive enough to specify a non-trivial existing API\@.  The
runtime assertion checker has been used to specify and verify a
component of a smartcard operating system~\cite{PollHarteldeJong02}.

ESC/Java has been used with great success to verify a realistic
example of an electronic purse implementation in Java
Card~\cite{CatanoHuisman02}. This case study was instrumental in
convincing industrial users of the usefulness of JML and feasibility
of automated program checking by ESC/Java for Java Card applets.  This
provided the motivation for the development of the JACK tool discussed
earlier, which is specifically designed for Java Card programs.  One
of the classes of the electronic purse mentioned above has provided
the most serious case study to date with the LOOP
tool~\cite{BreunesseBJ02}.

As witnessed by the development of the JACK tool by Gemplus, Java Card
smartcard programs may be one of the niche markets where formal
methods have a promising future. Here, the cost that companies are
willing to pay to ensure the absence of certain kinds of bugs is quite
high.  It seems that, given the current state of the art, using static
checking techniques to ensure relatively simple properties (e.g., that
no runtime exception ever reaches the top-level without being caught)
seems to provide an acceptable return-on-investment.  It should be
noted that the very simplicity of Java Card is not without its
drawbacks.  In particular, the details of its very primitive
communication with smartcards (via a byte array buffer) is not easily
abstracted away from.  It will be interesting to investigate if J2ME (Java 2
Micro Edition), which targets a wider range of electronic consumer
products, such as mobile phones and PDAs, is also an interesting
application domain for JML\@.


