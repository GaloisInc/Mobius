\chapter{Java Validation}
\section{Java}
\section{JML}
JML~\cite{Leavens-Baker-Ruby99b,Leavens-Baker-Ruby03}, the
``Java Modeling Language'', is a behavioral interface
specification language for Java; that is, it specifies both the behavior
and the syntactic interface of Java code.  The syntactic interface of
a Java class or interface consists of its method signatures,
the names and types of its fields, etc.
This is what is commonly meant by an application programming
interface (API).
The behavior of such an API can be precisely documented in JML annotations;
these describe the intended way that programmers should
use the API.  In terms of behavior, JML can detail, for example, the
preconditions and postconditions for methods as well as class
invariants.

An important goal for the design of JML is that it should be easily
understandable by Java programmers. This is achieved by staying as
close as possible to Java syntax and semantics.  Another important
design goal is that JML {\em not} impose any particular design method
on users; instead, JML should be able to document Java programs
designed in any manner \cite{Leavens-Baker-Ruby03}.

JML uses Java's expression syntax in assertions,
thus JML's notation is easy for programmers to learn.  
Because JML supports quantifiers such as
\verb_\forall_ and \verb_\exists_, and because JML allows ``model''
(i.e., specification-only) fields and methods, specifications can
easily be made precise and complete.
JML assertions are written as special
annotation comments in Java code,
so that they are ignored by Java compilers but can be used
by tools that support JML\@.  Within annotation comments JML extends the
Java syntax with several keywords.  It also extends Java's expression syntax with several
operators.
The central ingredients of a JML specification are preconditions
(given in {\tt requires} clauses), postconditions (given in {\tt
  ensures} clauses), and (class and interface) invariants.  These are
all expressed as boolean expressions in JML's extension to Java's
expression syntax.
In addition to ``normal'' postconditions, the language also supports
``exceptional'' postconditions, specified in {\tt signals} clauses.
These can be used to specify what must be true when a method throws an
exception. 

%=====================================================================
%=====================================================================
\subsection{The JML tool suite}
\label{tools}

Since JML specifications are meant to be read and written by ordinary
Java programmers, it is important to support the conventional ways
that these programmers create and use documentation.  Consequently,
the {\tt jmldoc} tool
produces browsable HTML pages containing both the
API and the specifications for Java code, in the style of pages
generated by Javadoc~\cite{Friendly95}.

%=====================================================================
The JML compiler (\texttt{jmlc}), developed at Iowa State University,
is an extension to a Java compiler and compiles Java programs
annotated with JML specifications into Java
bytecode~\cite{Cheon03,Cheon-Leavens02b}.  The compiled bytecode includes
runtime assertion checking instructions that check JML specifications
such as preconditions, normal and exceptional postconditions,
invariants, and history constraints.  The execution of such assertion
checks is transparent in that, unless an assertion is violated, and
except for performance measures (time and space), the behavior of the
original program is unchanged.  The transparency of runtime assertion
checking is guaranteed, as JML assertions are not allowed to have any
side-effects~\cite{Leavens-etal03a}.




\subsection{Extended Static Checking with ESC/Java2}
\label{escjava}

ESC/Java tool~\cite{Flanagan-Et-Al02}, developed at Compaq Research,
performs what is called ``extended static
checking''~\cite{ESC:Overview,10yearsESC},
compile-time checking that goes well beyond type checking.  It can
check relatively simple assertions and can check for certain kinds of
common errors in Java code, such as dereferencing \texttt{null},
indexing an array outside its bounds, or casting a reference to an
impermissible type.  ESC/Java supports a subset of JML and also checks
the consistency between the code and the given JML annotations.  The
user's interaction with ESC/Java is quite similar to the interaction
with the compiler's type checker: the user includes JML annotations in
the code and runs the tool, and the tool responds with a list of
possible errors in the program.

JML annotations affect ESC/Java in two ways.  First, the given JML
annotations help ESC/Java suppress spurious warning messages.   Second,
annotations make ESC/\-Java do additional checks.  
In these two ways, the use of JML annotations enables ESC/Java to
produce warnings not at the source locations where errors manifest
themselves at runtime, but at the source locations where the errors
are committed.

An interesting property of ESC/Java is that it is neither sound nor
complete; that is, it neither warns about all errors, nor does it
warn only about actual errors.  This is a deliberate design choice:
the aim is to increase the cost-effectiveness of the tool.  

ESC/Java translates a
given JML-an\-no\-tat\-ed Java program into verification
conditions~\cite{LeinoSaxeStata:JavaViaGC,FlanaganSaxe:POPL01},
logical formulas that are valid if and only if the program is free of
the kinds of errors being analyzed.  Any verification-condition
counterexamples found by the theorem prover are turned into
programmer-sensible warning messages, including the kind and source
location of each potential error.  The ESC/Java User's
Manual~\cite{escjava:userman} also provides a detailed description of
the semantics of JML annotations, as they pertain to ESC/Java.

\subsubsection{ESC/Java2}
\label{escjava2}

Development of version 1 of ESC/Java ceased with the dissolving of the
SRC group at Compaq.  Consequently Cok and Kiniry have in progress a
version 2 of ESC/Java, built on the source code release provided by
Compaq and HP.  This version has the following goals:
\begin{itemize}
\item to migrate the code base of ESC/Java and the code accepted by
  ESC/Java to Java 1.4;
\item to update ESC/Java to accept annotations consistent with current
  version of JML;
\item to increase the general use of the tool by packaging it in an
  easy-to-use form;
\item to increase the amount of JML that ESC/Java statically checks,
  consistent with the original engineering goals of ESC/Java;
\item and, over time, to update the associated tools of the ESC suite
  (Calvin, Houdini, RCC) in a similar manner.
\end{itemize}

There are two major areas of development of ESC/\-Java2 that will
improve overall usability of the tool, besides performance
improvements.  The first is the use of model variables and method
calls in annotation expressions.  Model variables are an important
abstraction mechanism in writing specifications and model methods
allow much more readable and compact specifications. 

The second important needed advance is to provide some level of
checking of the frame conditions (i.e., JML's \texttt{modifies} or
\texttt{assignable} clauses).  It is an acknowledged unsoundness of
ESC/Java that these are not checked and faulty \texttt{modifies}
clauses can be a subtle source of errors.  In particular, the default
\texttt{modifies} clause in some JML specifications is that everything
is potentially modified; this interpretation is not currently
implemented.

%=====================================================================
\section{Applications of JML to Java Card}
\label{applications}

Although JML is able to specify arbitrary sequential Java programs,
most of the serious applications of JML and JML tools up to now
have targeted Java Card.  Java Card$^{TM}$ is a dialect of Java specifically
designed for the programming of the latest generation of smartcards.
Java Card is adapted to the hardware limitations of smartcards; for
instance, it does not support floating point numbers, strings, object
cloning, or threads.

Java Card is a well-suited target for the application of formal
methods.  It is a relatively simple language with a restricted API\@.
Moreover, Java Card programs, called ``applets'', are small, typically
on the order of several KBytes of bytecode.  Additionally, correctness
of Java Card programs is of crucial importance, since they are used in
sensitive applications such as bank cards and mobile phone SIMs.  (An
interesting overview of security properties that are relevant for Java
Card applications is available~\cite{MarletLM01}.)

JML, and several tools for JML, have been used for Java Card,
especially in the context of the EU-supported project VerifiCard
(www.verificard.org).  JML has been used to write a formal
specification of almost the entire Java Card
API~\cite{PollBergJacobs01}.  This experience has shown that JML is
expressive enough to specify a non-trivial existing API\@.  The
runtime assertion checker has been used to specify and verify a
component of a smartcard operating system~\cite{PollHarteldeJong02}.



