The objective of this section is to demonstrate how the user can
annotate and verify programs in order to obtain an upper bound on
memory consumption. We begin by describing the principles of our
approach, then turn to establish its soundness, and finally show
how it can be applied to non-trivial examples involving recursive
methods and exceptions.


\subsubsection{Principles}
Let us begin with a very simple memory consumption policy which aims
at enforcing that  programs do not consume more than
some fixed amount of memory \Max . To enforce this policy, we first
introduce a ghost variable \Mem\ that represents at any given point of
the program the memory used so far. Then, we annotate the program both
with the policy and with additional statements that will be used to
check that the application respects the policy.



\paragraph{The precondition} of the method \method\ should ensure
that there must be enough free memory for the method
execution. Suppose that we know an upper bound of the allocations done
by method \method\ in any execution. We will denote this upper
bound by \allocMethod{\method}. Thus there must be at least
\allocMethod{\method}\ free memory units from the allowed \Max\ when
method \method\ starts execution. Thus the precondition for \method\ is:
$$
//@ \ \requires \ \Mem + \allocMethod{\method}  \leq \Max.
$$
The precondition of the program entry point (i.e., the \textit{main} 
method from which an application may start its execution) should 
also give the initial memory used by the virtual machine, i.e.
require that variable \Mem\ is equal to some fixed constant.

\paragraph{The normal postcondition} of the method $\method$ must
guarantee that the memory allocated during a normal execution of
$\method$ is not more than some fixed number \allocMethod{\method}\
of memory units. Thus for the method $\method$ the postcondition is:
$$
%\begin{array}{l}
//@ \ \ensures \ \Mem \leq \oldp(\Mem) + \allocMethod{\method}.
%\end{array}
$$

\paragraph{The exceptional postcondition} of the method $\method$ must
specify that the memory allocated during an execution of $\method$ 
terminating by throwing an exception \texttt{Exception} is not more
than \allocMethod{\method} units. Thus for the method $\method$ the
exceptional postcondition is:
$$
\begin{array}{l}
//@ \ \exsures{Exception} \\
\ \ \ \ \ \ \ \ \ \ \ \ \ \  \Mem \leq \oldp(\Mem) + \allocMethod{\method}.
\end{array}
$$




\paragraph{For every instruction that allocates memory} the ghost
variable \Mem\ must be updated accordingly. For the purpose of
this paper, we only consider dynamic object creation with the 
bytecode \new; arrays are left for future work and briefly discussed 
in the conclusion. 

In order to perform the update for \new\ bytecodes, we assume given a
function $\texttt{allocInst}: Class \rightarrow int$ gives an
estimation of the memory used by an instance of a class. Then at every
program point where a bytecode \srcCode{\new \ A} is found, the ghost
variable \Mem\ must be incremented by $\allocInstance{A}$. This is
achieved by inserting a ghost assignment associated with any \new\
instruction, as shown below:
$$
%\begin{array}{l}
\srcCode{\new \ A} \ \ \ \ // \ghostSet \ \Mem = \Mem + \allocInstance{A}.
%\end{array}
$$

\subsubsection{Correctness} An important question is whether our approach
guarantees that the memory allocated by a given program conforms to
the memory consumption policy imposed by BCSL annotations. We can
prove that our approach is correct by instrumenting the operational
semantics of the bytecode language to reflect memory
consumption. Concretely, this is achieved by extending states with the
special variable \Mem, and describing for each bytecode and for ghost
assignments the effect of the weakest precondition calculus on \Mem\
(in the fragment of the language considered, the only instruction to
modify memory is \new, thus the only instruction whose weakest
precondition calculus has an effect on \Mem\ is \new).

We can then prove the correctness of the annotations w.r.t. the
instrumented operational semantics, under the proviso that ghost
assignments triggered by object creation are compatible with the
instrumented operational semantics. 

%\subsubsection{Examples}
%We illustrate hereafter our approach by several examples, coping with
%recursive and overridden methods and with exceptions.

%\paragraph{Inheritance and overridden methods} Overriding methods
%are treated as follows: whenever a call to a method \method\ is
%performed, we require that there is enough free memory space for the
%maximal consumption by all the methods that override or are overridden
%by \method. In Fig. \ref{classExt} we show a class \srcCode{A}
%and its extending class \srcCode{B}, where \srcCode{B} overrides the
%method \method\ from class \srcCode{A}. Method \method\ is
%invoked by method $n$. Given that the dynamic type of the
%parameter passed to $n$ is not known, we cannot know which of the
%two methods will be invoked. This is the reason for requiring enough
%memory space for the execution of any of these methods.



%%After the method execution we consider the extreme case where there is executed the method \method\ that consumes the most.

%\begin{figure}[!t]%[!htp]
%Specification of method $m$ in class A:
%$$
%\begin{array}{ll}
%//@ \ \requires & \Mem + k  \leq \Max \\
%//@ \ \modifies & \Mem \\
%//@ \ \ensures & \Mem  \leq \oldp(\Mem) + k
%\end{array}
%$$

%Specification for method $m$ in class B:
%$$
%\begin{array}{ll}
%//@ \ \requires & \Mem + l  \leq \Max \\
%//@ \ \modifies & \Mem \\
%//@ \ \ensures & \Mem  \leq \oldp(\Mem) + l
%\end{array}
%$$

%%\begin{verbatim}
%%$$
%%\begin{array}{l}
%\begin{alltt}
%\srcCode{\textbf{public void n (\srcCode{A} a)}}
%... 
%//\small{\textit{Prove } \srcCode{\Mem +} \textit{max(l,k)} \srcCode{<= \Max}}
%\srcCode{invokevirtual <A.m>}
%//\small{\textit{Assume } \srcCode{\Mem <= \oldp(\Mem) +} \textit{max(l,k)}}
%...
%\end{alltt}
%%\end{verbatim}
%%\end{array}
%%$$

%\caption{\sc Example of overridden methods}
%\label{classExt}
%\end{figure}


%\paragraph{Recursive Methods} In Fig. \ref{recMeth} the bytecode
%of the recursive method $m$ and its specification are shown. For
%simplicity we show only a simplified version of the bytecode; we
%assume that the constructors for the class \srcCode{A} and \srcCode{C}
%do not allocate memory. Besides the pre- and the postcondition, the
%specification also includes information about the termination of the
%method: \variant\ $\local{1}$, meaning that the local variable
%$\local{1}$ decreases on every recursive call down to $0$, guaranteeing 
%that the execution of the method will terminate.

 
%%Now we explain why such a precondition is required for method \textbf{m} in order to specify the property for constraint memory consumption. 

%\begin{figure}[!t]
%\begin{alltt}
%public class D \{
% public void m (int i) \{
%   if (i > 0) \{ new A(); m(i-1); new A(); \}
%   else \{ new C(); new A(); \} \} \}
%\end{alltt}
%$$
%\begin{array}{l}
% //@ \ \requires \ \Mem + \allocInstance{A} + \allocInstance{C}) \\
%\ \ \ \ \ \ \ \ \ \ \ \ \  + \local{1}*2*\allocInstance{A} \le \Max \\
%//@ \ \variant \ \local{1} \\
%//@ \  \ensures  \ \local{1} \ge 0 \ \wedge \ \Mem \leq \oldp(\Mem)\ +  \\
%\ \ \ \ \ \ \ \ \ \oldp(\local{1})*2*\allocInstance{A}\ + \\
%\ \ \ \ \ \ \ \ \ \allocInstance{A} +  \allocInstance{C})
%\end{array}$$

%\begin{alltt}
%\srcCode{\textbf{public void m()}}
%\srcCode{0 \load\_1} //\small{\textit{Local var. loaded on the stack of \textbf{m}}}
%\srcCode{1 ifle 12} //\small{\textit{If \local{1} <= 0 jump}}
%\srcCode{2 new <A>} //\small{\textit{Here \local{1} > 0  } }
%//set \Mem = \Mem + \allocInstance{A}
%\srcCode{3 invokespecial <A.<init>>}
%\srcCode{4 aload\_0}
%\srcCode{5 iload\_1}
%\srcCode{6 iconst\_1}
%\srcCode{7 isub} //\small{\textit{\local{1} decremented with 1}}
%\srcCode{8 invokevirtual <D.m>} //\small{\textit{Recursive call}}
%\srcCode{9 new <A>}
%//set \Mem = \Mem + \allocInstance{A}
%\srcCode{10 invokespecial <A.<init>>}
%\srcCode{11 goto 16}
%\srcCode{12 new <A>} //\small{\textit{Target of the jump at \srcCode{1}}}
%//set \Mem = \Mem +  \allocInstance{A}
%\srcCode{13 invokespecial <A.<init>>}
%\srcCode{14 new  <C>}
%//set \Mem = \Mem + \allocInstance{C}
%\srcCode{15 invokespecial <C.<init>>}
%\srcCode{16 return}
%\end{alltt}

%\caption{\sc Example of a recursive method}
% \label{recMeth}
%\end{figure}

%We explain first the precondition. If the condition of line
%\srcCode{1} is not true, the execution continues at line
%\srcCode{2}. In the sequential execution up to line \srcCode{7}, the
%program allocates at most $\allocInstance{A}$ memory units and
%decrements by $1$ the value of $\local{1}$. The instruction at line
%\srcCode{8} is a recursive call to $m$, which either will take the
%same branch if $\local{1} > 0 $ or will jump to line \srcCode{12}
%otherwise, where it allocates at most $\allocInstance{A} +
%\allocInstance{C}$ memory units. On returning from the recursive call
%one more allocation will be performed at line \srcCode{9}.  Thus $m$
%will execute, $\local{1}$ times, the instructions from lines
%\srcCode{2} to \srcCode{7}, and it finally will execute all the
%instructions from lines \srcCode{12} to \srcCode{16}.

%The postcondition states that the method will perform no more than
%$\oldp(\local{1})$ recursive calls (i.e., the value of the register
%variable in the pre-state of the method) and that on every recursive
%call it allocates no more than two instances of class \texttt{A} (one
%corresponding to line \srcCode{2} and the other to line \srcCode{9})
%and that it will finally allocate one instance of class \texttt{A}
%(line \srcCode{12}) and another of class \texttt{C} (line
%\srcCode{14}).

%For proving the correctness of this method, 18 proof obligations were
%generated with Jack, most of which were automatically proved in Coq
%using its standard tactics.


%\paragraph{Loops} must also be annotated with appropriate invariants,
%and with variants that guarantee their termination. If we know that
%some expression $e$ bound by \verb!0! decreases at every
%iteration of the loop, and that each loop iteration will not allocate
%more than $k$ units, then we can strengthen the loop invariant
%to:
%$$\begin{array}{l}
%//@ \ \modifies \ i, \Mem \\
%//@ \ \invariant \ 0 \le e \wedge \ \Mem \le \atState{\Mem}{Before_{l}} + e * k \\

%//@ \ \variant \ e
%\end{array}$$
%$\atState{\Mem}{Before_{l}}$ is a special variable denoting the value
%of the consumed memory just before entering for the first time the
%loop \progLoop{l}. At every iteration the consumed memory must not go
%beyond the upper bound given for the body of loop.

%\paragraph{More precise specification} We can be more precise in
%specifying the precondition of a method by considering the field
%values of an instance, for example. Let \method\ be the method shown
%in Fig. \ref{excMeth} and assume no allocations are done in the
%constructor of the class \texttt{A}. The first line of \method\
%initialises one of the fields of field \texttt{b}. Since nothing
%guarantees that \texttt{b} is not \NULL, the execution may terminate
%with \texttt{NullPointerException}. Depending on the values of the
%parameters passed to \method, the memory allocated will be
%different. The precondition specifies the required memory space
%depending on whether the field \texttt{b} is \NULL\ or not.  In the
%normal postcondition we state that the method has allocated an object
%of class \texttt{A}. The exceptional postcondition states that no
%allocation is performed if \texttt{NullpointerException} causes the
%termination.


%\begin{figure}[!t]
%%\begin{figure}[t]
%$$
%\begin{array}{l}
%//@\  \requires\ \ !(\local{1} == \NULL) \Rightarrow  \\
%\ \ \ \ \ \ \ \ \ \ \ \ \ \ \phantom{\local{1}} \Mem +  \allocInstance{A} \le \Max \\
%//@\  \modifies\ \ \Mem \\
%//@\  \ensures \ \Mem \le \oldp(\Mem) +  \allocInstance{A} \\
%//@\   \exsures{NullPointerException} \\
%\hspace*{4.5cm} \Mem == \oldp(\Mem)
%\end{array}$$\\
%\begin{tabular}{lr}
%\begin{minipage}[t]{7cm}
%\begin{alltt}
%\srcCode{0 aload\_0}
%\srcCode{1 getfield<C.b>}
%\srcCode{2 iload\_2}
%\srcCode{3 putfield <B.i>}
%\srcCode{4 new <A>}
%//set \Mem = \Mem +
%      \allocInstance{A}
%\srcCode{5 dup}
%\srcCode{6 invokespecial <A.<init>>}
%\srcCode{7 astore\_1}
%\srcCode{8 return}
%\end{alltt}
%\end{minipage}
% &
%\begin{minipage}[t]{4cm}
%\begin{alltt}
%public class C
%\{
%  B b;
%  public void m (A a, int i)
%  \{
%      b.i = i ;
%      a = new A();
%  \}
%\}
%\end{alltt}
%\end{minipage}
%\end{tabular}
%\caption{\sc Example of a method with possible exceptional termination}
%\label{excMeth}
%\end{figure}

