\chapter{Evaluations}
\subsection{Industrial evaluations}
Two industrial evaluations have been carried internally by industrial
partners Axalto and Oberthur. The purpose of the present section is to
summarize the outcomes of these evaluations, that respectively focused
on the PayFlex case study and on a file system. More details can be
found in internal documents by the partners.

\subsubsection{Verification of banking case studies}
As a first test for Jack, we have studied a little banking
application.  This section presents different metrics concerning the
evaluation of the tool on this package.
\begin{table}
 \begin{tabular}{|l|c|c|c|c|c|c|c|} \hline
 Classes & Java & JavaDoc & JML & Proof & Automatic & Time to PO & Time to \\
  & lines & lines & lines & obligations & proof & generate (s) & prove (s)\\  \hline
 Transfert\_src  & 116 & 34 & 150 & 359 & 91\% & 22,5 & 238 \\
 AccountMan\_src & 105 & 51 & 236 & 269 & 82\% & 12,7 & 195 \\
 Currency\_src   &  93 & 20 &  28 &  50 & 96\% &  7,6 &  17 \\
 Balance\_src    &  64 & 38 &  58 & 335 & 95\% & 16,5 & 191 \\
 Spending\_rule  &  40 & 33 &  62 &  42 & 67\% & 13,6 & 217 \\ \hline
 \end{tabular}
\caption{banking applet metrics}
\label{MetricsTable}
\end{table}
Different remarks can be made from Table \ref{MetricsTable},
concerning the cost of adding JML annotations, the performance 
of the tool, as well as the cost associated to the proof.
The case study was also used to evaluate the following points:
\begin{itemize}
\item \emph{Cost of the annotations}
A first remark concerns the cost of the annotations.  The metrics
given here only concern the number of lines but one can see that the
documentation size (JavaDoc and JML) is one and half greater than the
code size.  So, writing the JML specification seems to be a costly
activity.  This remark can be moderated by two points: this
development was the first that we made, and annotations were added to
already existing code. So it suffers from its lack of abstraction, and
the annotations are really verbose. Moreover the time to specify is to
be compared to the time to test.

\item \emph{Interactive phase}
The automatic phases are quite responsive with some seconds to
generate proof obligations and attempt to discharge them
automatically.  Furthermore, the automatic proof rating is reasonable.
It is quite greater than the usual value for a B development (around
80\%).

Nevertheless, after automatic proof step, there remain 111 lemmas to
prove using the Atelier B interface.  An expert needs between 4 and 5
days to prove them.
\end{itemize}
Then, the new version of Jack has been evaluated on the Payflex case
study with the automatic mode using Simplify combined with the
interactive mode using Coq. Relevant changes on the Coq mode include:
\begin{itemize}
\item the transitivity relation for subtypes is now explicit:
      as a result, proofs about the payflex file system are better handled
      when one must decide whether a file of type T is also of type T' 
or not.
  
\item  the display of proof obligation hypothesis is clearer and more 
readable than on previous version

  
\item an prototype editor for coq inside Eclipse has been
  developed. It can be used as an alternative to ProofGeneral or
  CoqIDE.

  
\item the use of pure (i.e. side-effect free) method calls in JML
  annotations is better handled: the method call is no longer
  automatically replaced by its postcondition, instead it is displayed
  as a Java method call and the user can unfold it when desired (as is
  done in the Krakatoa tool for instance)
  
\item JML model variables (i.e. used for specification purpose only)
  are now well supported, and one can also use some methods declared
  with the keyword 'native' (it is not standard JML) in
  annotations.These methods are native with respect to a particular
  prover, that is they must be defined directly in the prover. That is
  a way to develop some 'specification libraries' to reuse when
  needed.
\end{itemize}
In order to reduce the overhead of writing annotations, Axalto has
also worked on the JML specification of the JC applet in general (as
the tool uses as input the JML specification). Some theoretical
results on the verification of specific properties have been obtained,
and a paper has been issued on the subject\cite{Rousset}. We start the
development of a tool with graphical interface to assist the writing
of annotations: from an UML class diagram or the code itself of an
application, it proposes some specification patterns to the user
(non-null by default for references, ranges for variable of integer
type and arrays length), then translation to adequate class invariants
and method preconditions is performed. In practice, annotations
generated in that manner reveal sufficient to prove that each method
'does not go wrong' (i.e. is runtime error free) and all the
verification conditions are filled in automatic mode with Simplify
using tools like Krakatoa or Jack.

\subsubsection{Verification of a file system}
This section describes the evaluation of the JACK product by Oberthur
Card Systems to specify Java Card programs in a smart card technology
context.

\subsubsection{File system specification}
For our evaluation, we choose a case study in line with the objectives
of the INSPIRED project. We specify a large part of a file system on
smart card. The file system stores user data and manages access
control on these data. It is for example an important security feature
of e-passports.

In our evaluation, we consider a realistic file system containing
three kinds of files:

\begin{itemize}
\item Elementary files (EF): contain user data. There are several types of EF:
\begin{itemize}
\item Binary files: are files without internal structure
\item Record files: where data are stored in records. These files can be:
\begin{itemize}
\item With records of variable length
\item With record of fixed length
\item With a cyclic structure
\end{itemize}
\end{itemize}
\item Dedicated files (DF): are similar to \lq\lq folders\rq\rq\; they contain
other EF or DF

\item Master file (MF): it is a particular DF that is the root of the
file system

\end{itemize}
We specified a complete set of commands that are:\\
\begin{tabular}{l}
CREATE FILE \\
SELECT FILE \\
READ BINARY \\
UPDATE BINARY \\
READ RECORD \\
UPDATE RECORD \\
APPEND RECORD \\
DELETE FILE
\end{tabular}\\
These commands are specified in ISO-IEC 7816 as APDU (Application
Protocol Data Unit). We recall the APDU structure. It comprises:
\begin{itemize}
\item CLA (1 byte): define the command class 
\item  INS (1byte): command instruction
\item  P1 (1 byte) and P2 (1 byte): parameters of the command
\item  LC (1 byte): command length
\item  Data field (N bytes): data of the command
\item LE (1 byte): expected length of the response
\end{itemize}
The response APDU contains two fields:
\begin{itemize} 
\item Body: containing the data returned by the card
\item SW1||SW2: status of the command execution (9000 means \lq\lq correct
execution\rq\rq  and others values are reserved for error and warning
status).

\end{itemize}
Our specification uses heavily an existing Java Card API specification
developed by Erik Poll in the VERIFICARD project. Other important
ingredients are: 
\begin{itemize}
\item Specification of basic operations on APDU (offset
computation, slicing\ldots) 
\item Arithmetic operation: this point is quite difficult because we
  use heavily the cast between the types short (signed) and byte
  (unsigned) Operations on Bits: ({\tt \&\&}, {\tt <<}, \ldots) : to
  avoid a complex specification, we put them as axioms.
\end{itemize}
Another important ingredient is the access condition control
that is performed by the following method:

 checkAccessCondition(oper,fichier)

that take two parameters:
\begin{itemize}
\item
oper: describes the requested operation. Possible values are:
\\
\begin{tabular}{l}
AMB\_CREATE\_DF \\
AMB\_CREATE\_EF \\
AMB\_EF\_READ  \\
AMB\_EF\_APPEND\_RECORD \\
AMB\_DF\_DELETE  \\
AMB\_EF\_DELETE \\
AMB\_EF\_WRITE 
\end{tabular}\\
\item fichier: is the reference on the file on which the operation will act. 
\end{itemize}.
The result of the method has two possible values:  ACCESS GRANTED 
or ACCESS DENIED.

\paragraph{Complete example of specification}

We present in this section the JML specification of he READ BINARY
command. The corresponding method has the following signature:

readBinary(byte p1,byte p2,byte Lc,byte[] buffer, short offset)
We distinguish three different steps for the specification development: 
\begin{itemize}
\item Lookup the file
\item Offset computation
\item Copy of the found data in the output buffer
\end{itemize}

\subparagraph{Lookup the file}
The parameters p1, p2 are used to identify the file. A private method
is especially defined for this operation.

\begin{lstlisting}
ElementaryFile getBinary(byte P1,byte p2)}.

/*@ requires true;
@ ensures ((byte)(p1 & (byte)(0x80) == 0)
@               ==> \result == m_oCurrentFile;
@ ensures ((byte)(p1 & (byte)(0x80) != 0)
@               ==> (\exists ElementaryFile ef;
@                       ef.m_bSFI == (p2 & 0x7F);
@                       \result == ef)
@ ensures \typeof(\result) <: \type(Elementaryfile)
@*/
\end{lstlisting}
This file research is based on a of a linked list that we completely specified, for operations: create, insert, remove. 

\subparagraph{Offset computation}
This part is performed by the following method:
\begin{lstlisting}
getOffset(byte p1,byte p2)

/*@ requires true;
@ modifies \nothing;
@ ensures ((byte)(p1 & 0x80)== 0x80)==>
@               \result == Util.makeShort((byte)0,P2);//(1)
@ ensures ((byte)(p1 & 0x80)!= 0x80)==>
@               \result == Util.makeShort(P1,P2);//(2)
@ ensures \result >= 0;//(3)
@*/
\end{lstlisting}
\subparagraph{Copy data}
This step is performed without difficulties in JML. 

We put together all the previous pieces of specification to build the specification of the READ BINARY command.
\begin{lstlisting} 
/*@ requires apdu != null;
@ requires apduR != null;
@ modifies m_oCurrentFile;
@ modifies ISOException.systemInstance.theSw[0],
ISOException.systemInstance.systemInstance._reason;
@ ensures (\exists TransparentFile ef;
@               ef == getBinaryFile(p1, p2);
@               ef.checkfFileStateAccessRead() ==>
@               checkAccessCondition(File.AMB_DEF_READ, ef)
@                       == ACCESS_GRANTED
@               ==> (
@                       (((byte)0x00 & length)+getUpdateOffset(p1, p2)<= ef.length) ==>
@                       (
@                               (\forall short s;
@                                       0 <=s
@                                       && s<((byte)0x00 & length);
@                                            ef.data[s+getUpdateOffset(p1,p2)] ==apduR[s])
@                               &&
@                               \result == ((byte)0x00 & length)
@                       ))
@               &&
@               (
@                       (((byte)0x00 & length)+getUpdateOffset(p1, p2)> ef.length) ==>
@                       (
@                               (\forall short s;
@                                       0 <=s
@                                       && s < (ef.length -getUpdateOffset(p1, p2));
@                                               ef.data[s+getUpdateOffset(p1, p2)]== apduR[s])
@                               &&
@                               \result == (ef.length -getUpdateOffset(p1, p2))
@                       )
@               )
@               &&
@               ((byte)(p1 & (byte)0x80) == (byte)0x80
@               ==> ef == currentFile
@       )));
@       signals (ISOException ex) true;
@*/
\end{lstlisting}
A complete example of proof As example, we extract from our
development a proof from the APPEND RECORD command. It concerns 
the creation of a record and constraints on its size.

The proof is composed of the following sections: first, it contains
a header to import all the necessary libraries:
\begin{lstlisting}
Add LoadPath "c:\coq\lib\theories\bool" as Coq.Bool.
Require Import Bool.
Add LoadPath "c:\coq\lib\theories\ZArith" as Coq.ZArith.
Require Import ZArith.
Add LoadPath "c:\coq\lib\theories\Logic" as Coq.Logic.
Require Import Classical.
Require Import "C:\eclipse\workspace\IDONE\JPOs
\com_oberthurcs_javacard_common_filesystem_VariableRecordElementaryFile".

Load "C:\eclipse\workspace\IDONE\JPOs\localTactics.v".

Open Scope Z_scope.
Open Scope J_Scope.
\end{lstlisting}
Then, it contains a declaration of Java Card variables:
\begin{lstlisting}
Section JackProof.
Variable Result_setData_3: bool.
Variable byteelements_0: REFERENCES -> t_int -> t_byte.
Variable filesystem_Record_m_sLength_0: 
REFERENCES -> t_short.
Variable filesystem_Record_m_baData_0: 
REFERENCES -> REFERENCES.
Variable filesystem_Record_index_0: 
REFERENCES -> t_short.
Variable filesystem_Record_state_0:
 REFERENCES -> t_short.
Variable util_LinkedObject_next_0: 
REFERENCES -> REFERENCES.
Variable util_LinkedObject_indice_0: 
REFERENCES -> t_short.
Variable util_LinkedObject_previous_0: 
REFERENCES -> REFERENCES.
Variable util_LinkedObject_jmlPrevious_0: 
REFERENCES -> REFERENCES.
Variable util_LinkedObject_jmlNext_0: 
REFERENCES -> REFERENCES.
Variable f_java_lang_Object_owner_0: REFERENCES -> REFERENCES.
Variable newObject_8: REFERENCES.
Variable this: REFERENCES.
Variable l_p_baSource: REFERENCES.
Variable l_p_sOffset: t_short.
Variable l_p_sLength: t_short.
Variable l_p_sRecordSize: t_short.
\end{lstlisting}
Then one finds hypotheses coming from the specification of called methods:
\begin{lstlisting}
Variable hyp1 : (Result_setData_3 = true).
Variable hyp2 : (not ((newObject_8 = null))).
Variable hyp3 : (not ((newObject_8 = null))).
Variable hyp4 : (not ((newObject_8 = null))).
Variable hyp5 : (not ((instances newObject_8))).
Variable hyp6 : (newObject_8 <> null).
Variable hyp7 : (((j_lt 255 l_p_sRecordSize)) ->
((filesystem_Record_m_sLength_0 newObject_8) = 255)).
Variable hyp8 : (((j_le l_p_sRecordSize 0)) ->
((filesystem_Record_m_sLength_0 newObject_8) = 255)).
Variable hyp9 : (((((j_lt 0 l_p_sRecordSize)) /\ ((j_le l_p_sRecordSize 255))))
 -> ((filesystem_Record_m_sLength_0 newObject_8) = l_p_sRecordSize)).
Variable hyp10 : forall x1258, (x1258 <> newObject_8) ->((f_m_sLength x1258) 
= (filesystem_Record_m_sLength_0 x1258)).
Variable hyp11 : (forall (x1259:REFERENCES), 
(~((singleton REFERENCES (filesystem_Record_m_baData_0 newObject_8)) x1259)) 
-> ((byteelements_0 x1259) = (byteelements x1259))).
Variable hyp12 : forall (Result_checkData_4: bool),
        ((((((((((((j_le 0 l_p_sLength)) /\ 
((j_le l_p_sLength (arraylength (filesystem_Record_m_baData_0 newObject_8)))))) /\ 
((j_le (j_add l_p_sLength l_p_sOffset) (arraylength l_p_baSource))))) ->
(Result_checkData_4 = true))) /\ 
((((((((j_lt l_p_sLength 0)) \/ 
((j_lt (arraylength (filesystem_Record_m_baData_0 newObject_8)) l_p_sLength)))) \/ 
((j_lt (arraylength l_p_baSource) (j_add l_p_sLength l_p_sOffset))))) ->
(Result_checkData_4 = false))))) ->
(Result_setData_3 = Result_checkData_4))).
Variable hyp13 : forall (Result_checkData_5: bool),
        ((((((((((((j_le 0 l_p_sLength)) /\ 
((j_le l_p_sLength (arraylength (filesystem_Record_m_baData_0 newObject_8)))))) /\ 
((j_le (j_add l_p_sLength l_p_sOffset) (arraylength l_p_baSource))))) ->
(Result_checkData_5 = true))) /\ 
((((((((j_lt l_p_sLength 0)) \/ 
((j_lt (arraylength (filesystem_Record_m_baData_0 newObject_8)) l_p_sLength))))
 \/ ((j_lt (arraylength l_p_baSource) (j_add l_p_sLength l_p_sOffset))))) ->
(Result_checkData_5 = false))))) ->
(((Result_checkData_5 = true)) ->
(forall (l_s171: t_short), 
(((((j_le 0 l_s171)) /\ ((j_lt l_s171 l_p_sLength)))) ->
(((byteelements_0 (filesystem_Record_m_baData_0 newObject_8)) l_s171) 
= ((byteelements_0 l_p_baSource) (j_add l_s171 l_p_sOffset)))))))).
Variable hyp14 : forall x1260, (x1260 <> newObject_8) ->
((f_m_baData x1260) = (filesystem_Record_m_baData_0 x1260)).
\end{lstlisting}
Statements concerning variable that are not modified in this method
\begin{lstlisting}
Variable hyp15 : forall x1261, (x1261 <> newObject_8) ->
((f_index x1261) = (filesystem_Record_index_0 x1261)).
Variable hyp16 : forall x1262, (x1262 <> newObject_8) ->
((f_state x1262) = (filesystem_Record_state_0 x1262)).
Variable hyp17 : forall x1263, (x1263 <> newObject_8) ->
((f_next x1263) = (util_LinkedObject_next_0 x1263)).
Variable hyp18 : forall x1264, (x1264 <> newObject_8) ->
((f_indice x1264) = (util_LinkedObject_indice_0 x1264)).
Variable hyp19 : forall x1265, (x1265 <> newObject_8) ->
((f_previous x1265) = (util_LinkedObject_previous_0 x1265)).
Variable hyp20 : forall x1266, (x1266 <> newObject_8) ->
((f_jmlPrevious x1266) = (util_LinkedObject_jmlPrevious_0 x1266)).
Variable hyp21 : forall x1267, (x1267 <> newObject_8) ->
((f_jmlNext x1267) = (util_LinkedObject_jmlNext_0 x1267)).
Variable hyp22 : forall x1268, (x1268 <> newObject_8) ->
((f_owner x1268) = (f_java_lang_Object_owner_0 x1268)).
Variable hyp23 : (instances this).
Variable hyp24 : (subtypes (typeof this) (class c_VariableRecordElementaryFile)).
\end{lstlisting}
Hypotheses extracted from the class where is performed the proof:
\begin{lstlisting}
Variable hyp25 : 
((union REFERENCES instances (singleton REFERENCES null)) l_p_baSource).
Variable hyp26 : (((l_p_baSource <> null)) ->
(subtypes (typeof l_p_baSource) (array (class c_byte) 1))).
Variable hyp29 : (l_p_baSource <> null).
Variable hyp30 : (j_le 0 l_p_sOffset).
\end{lstlisting}
The statement of the lemma to prove:
\begin{lstlisting}
Lemma l:
forall (Result_checkData_0: bool),
        ((((((((((((j_le 0 l_p_sLength)) /\ 
((j_le l_p_sLength (arraylength (filesystem_Record_m_baData_0 newObject_8)))))) /\ 
((j_le (j_add l_p_sLength l_p_sOffset) (arraylength l_p_baSource))))) ->
(Result_checkData_0 = true))) /\ 
((((((((j_lt l_p_sLength 0)) \/ 
((j_lt (arraylength (filesystem_Record_m_baData_0 newObject_8)) l_p_sLength)))) \/ 
((j_lt (arraylength l_p_baSource) (j_add l_p_sLength l_p_sOffset))))) ->
(Result_checkData_0 = false))))) ->
(((Result_checkData_0 = true)) ->
((((((filesystem_Record_m_sLength_0 newObject_8) = l_p_sRecordSize)) \/ 
(((filesystem_Record_m_sLength_0 newObject_8) = 255)))) /\ 
((forall (l_s128: t_short), (((((j_le 0 l_s128)) /\ ((j_lt l_s128 l_p_sLength)))) ->
(((byteelements_0 (filesystem_Record_m_baData_0 newObject_8)) l_s128)
 = ((byteelements_0 l_p_baSource) (j_add l_s128 l_p_sOffset)))))))))).
\end{lstlisting}
The proof:
\begin{lstlisting}
Proof with autoJack; arrtac.
intros.
split.
generalize (dec_Zle  l_p_sRecordSize 0).
intro.
inversion_clear H1.
right.
apply (hyp8 H2).
generalize (Znot_le_gt l_p_sRecordSize 0 H2).
intro.
generalize (dec_Zle  l_p_sRecordSize 255).
unfold Decidable.decidable.
intro.
inversion_clear H3.
generalize (Zgt_lt l_p_sRecordSize 0 H1).
intro.
left.
auto.
generalize (Znot_le_gt l_p_sRecordSize 255  H4).
intro.
generalize (Zgt_lt l_p_sRecordSize 255 H3).
intro.
right.
apply (hyp7 H5).
intros.
rewrite H0 in H.
generalize (hyp13 true H).
intros.
cut (true = true).
intro.
apply (H2 H3 l_s128 H1).
reflexivity.
Qed.
End JackProof.
\end{lstlisting}
\paragraph{Conclusion}
We experiment the JML specification on a non-trivial example, showing that this language is suitable for smart card applications. Especially because we use a specification style that is very close from the implementation. However we remark that in some cases, we are obliged to use complex techniques like: auxiliary functions, sequence memorisation\ldots


Furthermore, Jack is based on an old JML syntax and unfortunately it
does not support some very interesting functionalities:
\begin{itemize}
\item Model methods: that are methods exclusively used for specification and that can be used as a toolbox to solve common problems

\item Refinement: with the file jml-refined one can refine JML
specification at several level of precision
\end{itemize}
In our experiment, we tried to used JML in a Common Criteria
certification framework. Our conclusion is that it fulfil partially
the requirements of the development activity (ADV): FSP (functional
specifications), HLD (high level design) and LLD (low level design),
because it permits the complete description of the interfaces at each
of these levels.

\begin{quote}
ADV FSP.3.3C The functional specification shall describe the
purpose and method of use of all external TSF
interfaces, providing complete details of all effects,
exceptions and error messages.

ADV HLD.3.8C The high-level design shall describe the purpose
and method of use of all interfaces to the subsystems
of the TSF, providing complete details
of all effects, exceptions and error messages.

ADV LLD.2.5C The low-level design shall describe the purpose
and method of use of all interfaces to the modules
of the TSF, providing complete details of
all effects, exceptions and error messages.
\end{quote}
However it is not possible to formally prove the correspondence between these levels (FSP, HLD and LLD). A semi-formal correspondence based on matrix is possible. To have a formal proof of correspondence, the jml-refined functionality is necessary.    




\section{Comparison between source and bytecodes \\ proofs}  \label{results}

The purpose of this section is to give a comparison between bytecode and source proof obligations.
In particular, we illustrate this by the proof obligations of the example program in Fig.\ref{replaceSrc}.
%We have an implementation of the JML compiler ( subsection \ref{comJML}) and the bytecode verification condition generator based on the weakest precondition calculus (Section \ref{wpbc}) which are integrated in JACK. Both of the verification condition generators perform the same simplifications over the verification conditions 
%, e.g. eliminate verification conditions that contain contradictory hypothesis or trivial goals (equal to true). 

%The performed tests show that JML compilation augments around twice the file size. 
%For the example in Fig.~\ref{replaceSrc}, the class file without the specification extensions is 548 bytes, 
%and the class with the BCSL extension BCSL is 954 bytes. 
%The size of the bytecode specification is proportional to the source specification: 
%the bigger is the source specification, the greater will be the size of the class file. 
We studied the relationship between the source code proof obligations generated 
by the standard feature of JACK and the bytecode proof obligations generated by our implementation over the corresponding bytecode
 produced by a non optimizing compiler over the examples given in \cite{JPVC03JKM}. The proof obligations were the same modulo 
program variables names and basic types.

 We return now to our example from the previous sections and give in Fig. \ref{vcEnsures} one of the proof obligations on source 
and bytecode level respectively concerning the postcondition correctness. The verification conditions on bytecode and source level
 have the same shape modulo names (see Section \ref{comJML} for how names are compiled). Also in Section
 \ref{comJML}, we discussed the compilation of the JML postcondition from Fig. \ref{replaceSrc}. Particularly,
 we saw that the compiler has to transform the source postcondition in an equivalent formula and we
 gave the compilation in Fig. \ref{postCompile}. 

Despite those transformations, the source and bytecode goal respectively (which are actually the postcondition) on bytecode and source level are not only
semantically equivalent but syntactically the same (except for the variable names ). Still, in the bytecode proof obligation we have one more hypothesis than on source level. The extra hypothesis in the bytecode proof obligation is related to the fact that the result type is boolean but the JVM encodes boolean expressions as integers (which is trivially true). This means that the proof obligations have also the same shape.

 Another important issue is the impact of simple optimizations like dead code elimination on the relationship between source and bytecode proof obligations. 
In this case, the compiler does not generate the dead code and the bytecode verification condition generator will neither ``see'' it. 
Even though the source contains the never taken branch as the condition is equivalent to false, this will result in a trivially true
verification condition which the JACK source verification condition generator will discard.

The equivalence between source and bytecode proof obligations can be exploited in PCC scenarios, as we discussed in Section \ref{architecture} where 
the producer generates the program certificate over the source code in scenarios where a complete automatic certification (e.g. the certifying compiler) will not work.
 
We aim to formally give evidence that the proof obligations on non optimized bytecode and source programs are syntactically the same (modulo names and types). 

%Fig. \ref{vcLoopPreserv} shows the proof obligations for the loop preservation. As you can see the hypothesis and the goal have the same ``shape'' on bytecode and source code and the differences are due to the variable names.



% \begin{figure}{!h}

% $$\begin{array}{ll}
%Hypothesis \ on \ bytecode:  & Hypothesis \ on \ source \ level:  \\
%% & \\

%
%\begin{array}{l}
% \register{1} \neq \\
%\#19 (\register{0})[\register{2}\_at\_ins\_22]
%\end{array}  
%
%&  
%\begin{array}{l}
% \srcVar{obj} \neq \\
%  ListArray.list(\this)[\srcVar{i}\_at\_ins\_26] 
%\end{array}   \\
%
%
%
% & \\
%
%\#19(\register{0}) \neq \Mynull &  ListArray.list( \this) \neq \Mynull\\
%
%
%& \\
%
%\begin{array}{l}
%  len(\#19 (\register{0})) > \\
% \register{2}\_at\_ins\_22 
%\end{array}
%%& 
%\begin{array}{l}
%  len(ListArray.list(\this)) > \\
%\srcVar{i}\_at\_ins\_26
%\end{array}         \\ 

%

% & \\
%
% \register{2}\_at\_ins\_22 \geq 0 &   \srcVar{i}\_at\_ins\_26    \geq 0    \\
%
%
% & \\
%\begin{array}{l}
%  \register{2}\_at\_ins\_22 < \\
%  len(\#19(\register{0}))
%\end{array} &
%
%\begin{array}{l}
%  \srcVar{i}\_at\_ins\_26 <\\
%  len(ListArray.list(\this))
%\end{array}   \\
%
%
% & \\
%\begin{array}{l}
%  \register{2}\_at\_ins\_22 \leq \\
%  len( \#19(\register{0}))
%\end{array} 
%&  
%\begin{array}{l} 
%  \srcVar{i}\_at\_ins\_26 \leq \\
%  len(ListArray.list(\this))
%\end{array}   \\
%
%
% &\\
%% \register{2}\_at\_ins\_22 \geq 0 &   \srcVar{i}\_at\_ins\_26 \geq 0 \\
%
%
%
%% &\\
% \begin{array}{l} 
%         \forall  var(0). \ 0 \leq var(0) \wedge var(0) < (\register{2}\_at\_ins\_22) \Rightarrow \\
%                \Myspace    \#19(\register{0})[var(0)] \neq \register{1}
%      \end{array} &        
%      \begin{array}{l} 
%             \forall  var(0). \ 0 \leq var(0) \wedge var(0) < (\srcVar{i}\_at\_ins\_26) \Rightarrow \\
%                 \Myspace       ListArray.list(\this)[var(0)] \neq \srcVar{obj}
%      \end{array}  \\
%%
% typeof(\register{0}) <: ListArray &    typeof(this) <: ListArray     \\
%
%& \\
%& \\
%Goal \ on \ bytecode: & Goal \ on \ source \ level: \\
%
%& \\
%
%  \begin{array}{l}
%               1 + \register{2}\_at\_ins\_22 \leq  len(ListArray.list(\register{0}))  \\
%
%               1 + \register{2}\_at\_ins\_22 \geq 0 \\
%
%               \forall  var(0). 0 \leq var(0) \wedge var(0) < 1 + \register{2}\_at\_ins\_22 \Rightarrow \\
%                   \Myspace  ListArray.list(\register{0})[var(0)] \neq \register{1} 
%
%       \end{array}
%& 
%
%       \begin{array}{l}
%             1 + \srcVar{i}\_at\_ins\_26 \leq  len(ListArray.list(this))  \\
%	     \\
%             1 + \srcVar{i}\_at\_ins\_26 \geq 0 \\
%	     \\
%             \forall  var(0). 0 \leq var(0) \wedge\\
%	     \Myspace  var(0) < 1 + \srcVar{i}\_at\_ins\_26 \Rightarrow \\
%                  \Myspace  ListArray.list(this)[var(0)] \neq \srcVar{obj} 
%       \end{array}   
%
 
%\end{array}$$



%\caption{Source and Bytecode verification condition for loop preservation for method \texttt{ListArray.isElem} }
%\label{vcLoopPreserv}
%\end{figure}








\begin{figure*}[!h]

\begin{center}
\begin{tabular}{|l|l|}
\hline
\bf{Hypothesis \ on \ bytecode:}  & \bf{Hypothesis \ on \ source \ level:}  \\
\hline 
$\register{2}\_at\_ins\_20 \geq $ 
& $ \srcVar{i}\_at\_ins\_26 \geq$ \\

$len(\#19(\register{0})) $ & $  len(ListArray.list(\this)) $ \\
\hline 

$\#19(\register{0}) \neq \Mynull$ 
& $ ListArray.list(\this) \neq \Mynull$ \\

\hline 
$ \register{2}\_at\_ins\_20) \leq$ 
&  $  \srcVar{i}\_at\_ins\_26  \leq   $ \\
$ len(\#19(\register{0}))  $ & $ len(ListArray.list(\this))  $ \\
\hline

$\register{2}\_at\_ins\_20 \geq 0  $ 
& $ \srcVar{i}\_at\_ins\_26  \geq 0 $ \\

\hline

$\forall  var(0). \  0 \leq var(0) \wedge  $ & $\forall  var(0). \  0 \leq var(0) \wedge $ \\
$ var(0) < \register{2}\_at\_ins\_20 \Rightarrow $ & $  var(0) < \srcVar{i}\_at\_ins\_26 \Rightarrow $\\
$ \#19(\register{0})[var(0)] = \register{1}   $ & $  ListArray.list(\this)[var(0)] = \srcVar{obj}  $ \\

\hline

 $typeof(\register{0}) <: ListArray$ & $typeof( \this) <:  ListArray$  \\
\hline

$0=0 \vee 0=1$ & \\

& \\

\hline
\bf{Goal on bytecode:} & \bf{Goal on source level:} \\
\hline
$\Myfalse  \iff $ & $\Myfalse \iff  $ \\
 $ \exists  var(0) . \ 0 \leq var(0) \wedge$ 
& $ \exists  var(0) . \ 0 \leq var(0) \wedge$ \\

$\Myspace \Myspace var(0) < len(\#19(\register{0})) \wedge$ 
& $\Myspace \Myspace  var(0) < len(ListArray.List(this)) \wedge $\\
       
$\Myspace \Myspace \#19(\register{0})[var(0)] = \register{1} $ 
&$\Myspace \Myspace  ListArray.List(this)[var(0)] = \srcVar{obj}  $ \\

\hline
\end{tabular}
\end{center}

Note: $\expression\_at\_ins\_n$ denotes the value of  
expression $\expression$ at the bytecode instruction at index (source line)  $n$ 

\caption{\sc Comparison of source and bytecode verification conditions}
\label{vcEnsures}
\end{figure*}

\clearpage
 















%\subsubsection{Example}
% We give a simple example of how the \wpi \ works. Block $\blockm{6}$ (starts at instr. \texttt{6}) in Fig.~\ref{blockBC} ends with a branching instruction and in the case when the condition is true (the current element of the array is not equal to the first parameter of the method \texttt{replace}) the execution will continue at $\blockm{19}$. Below we give the part of the weakest precondition for block $\blockm{6}$ in case the control flows to block $\blockm{19}$( the condition of its last instruction holds and in this case 
%the predicate $pre(b^{6}, b^{19})$ is $\wpi(\blockm{19})$).  The implications with conclusion \Myfalse \ stand for the possible exceptions \texttt{NullPointer} and \texttt{ArrayIndexOutOfBound} exceptions that may be thrown (as no postcondition is specified explicitly for these cases of abnormal termination, the one by default is taken). 

%\input wpExample.tex


\section{Bytecode Verifier}
The tools have also been tested on a bytecode verifier java implementation. A termination proof has been provided.
A specific implementation has been coded with on one hand the main loop which remain unchanged whatever the specifications of the virtual machine Java chosen, and other instructions and memory states which depends on selected model.
\subsection{Implementation and Modelisation}
The main loop is in a package which contains abstract classes: 
the instructions and the states are implemented in a more generic way.
The package containing the implementation is composed from the instructions for the standard Java types and of the states of memory typing. 
\subsubsection {Memory states}
The memory states are represented by the State class, which is an abstract class.  It does not contain any precise definition of the memory: 
one has no information on the stack or on the local variables table. 
The implementation is relatively simple: 
it is a class which contains a type stack and a table of the types of the local variables. 
Functions allowing to read simply these structures and to generate verification error in the cases of misuse are defined.   
\subsubsection{Instructions}
The instructions are also represented by an abstract class: 
the class Instruction.  
Since in the Kildall algorithm each instruction is associated to a memory state,  the Instruction class has a field of the State type. 
An instruction can also have one or more successors. 
This relation is represented by a field which is the list of the successors of the instruction.  
One of the other aspects is the fact that on associate to each instruction a boolean field to determine if it has been modified or not.

Several properties of the bytecode verifier are formalised in this class.
First of all one verifies that the successors of the instruction are well included in the others instructions of the program. 
If these successors pointed towards external instructions, an verification error would be returned. 

The others important properties concern the pure function {\tt
buildNewState}.  This function builds the typing state of the
execution of an instruction on the current state.  This construction
can fail if the instruction tries for example to pop an element when
the stack is empty.  If it succeeds, a new non null state is built.


Around ten instructions have been implemented: {\tt load} and {\tt
blind} for the access to local variables, {\tt push} and {\tt pop} to
obtain or put element on the stack, {\tt op1} and {\tt op2} which is
two operators who consume both the two top element of the stack and
which replaces them by a result of a certain type, {\tt ifle} and {\tt
jump} instructions of jump towards another instruction successor, {\tt
nop} the instruction which does not do anything and finally {\tt stop}
which is an instruction which does not have a successor.  These
instructions have an associated type in the OperandType class, who can
be None, Type1 or Type2.  Those are the minimal instructions to have a
Java-like program.

\subsubsection {The main loop}
The main loop is implemented in the Verifier class.  It is not an
abstract class because it uses the properties of the State and
Instruction abstract classes to verify an instruction set on
particular states.  This class provides two functions, the function
{\tt verify} in which the loop is written and the function {\tt check}
which verify an instruction.

%The m \ 'ethode check ensures that all \ 'states of the successors of an instruction donn \ 'ee,    are larger or \ 'equal that the \ 'states before the ex \ 'ecution of the m \ 'ethode. This   propri \ 'and \ 'E seems simple \ `has to express but it implies several Pr \ 'erequis.  First of all it should be guaranteed that the successors of the instructions point all worms of   valid instructions. Then that all the instructions are diff \ 'erentes of no one and that theirs  \ 'states are too diff \ 'erents of no one them.    The Pr \ calculation weaker 'econdition of Jack forces us \ `has to add these propri \ 'and \ 'be    Li \ 'ees \ `with the S \ 'emantic of the language Java.  .   %Pour to facilitate the evidence I have \ 'and \ 'E oblig \ 'E  %de to add a certain number of assertions.    

The {\ tt verify} method is the main loop of the bytecode verifier.
It contains two nested loops. The internal one is a {\tt for} loop
which iterates on the instructions and verify all the quoted
instructions (as described in the Kildall algorithm). The termination
of the internal loop is easy to prove.  The {\tt for} loop executes as
many time as there are numbers in the table.  The external {\tt while}
loop stops the algorithm when no more instruction typing state is
modified.  This termination is not obvious to prove, especially with
JML, since it only allow to prove loop termination by giving an
integer variant.

Since the states have to be used to show the algorithm termination,
one has to make correspond each state with an integer. Thus at each
loop iteration, the integer associated with the state either increase
or preserve the same value; and it exists a maximum value.


\begin{figure}[ht]  
\begin{center}    
\begin{tabular}{p {0.4 \textwidth} c c c c}  
{\bf Classes:} & State & Instruction & Verifier \\  
{\bf Lines of code:} & 14 & 47 & 66 \\  
{\bf Lines of annotations:} & 20 & 54 & 81 \\  \raggedright 
{\bf Proof obligations:} & 26 & 129 & 627 \\  \raggedright 
{\bf Automatically proved proof obligations:} & 17 & 93 & 112 \\  
{\bf Average length of a non-automatic proof:} & 3 & 6 & 12 \\    
\end{tabular}  
\end{center}  
\caption{Some statistics on proof}  
\label{stats}  
\end{figure}    
\subsection{Proofs}
The first proofs are relatively easy.  The State class is proved
almost automatically; these proofs are not due to the code of the
methods but some standard verification Jack adds to each methods,
mostly to verify that all the public invariants are not broken after
the execution of each method.  Since the methods in the State class
are quite simple (and do not break the invariants), the automation is
good for these kind of proof obligations.  The only special case is
for the constructor where it is necessary to break a disjunction ({\tt
instance S $\vee$ S = null}) to prove the invariants.


The Instruction class has also been relatively easy to prove.  A
significant number of proof was done automatically (approximately 90
\%) and as for the State class this was mainly the verifications of
invariants; then majority of proof could be trivially resolved as most
of the methods are observer or accessors to private fields.  Two
methods verify some properties over the instructions, namely {\tt
checkDomain} which checks if the successors of the instruction are
contained in the program and {\tt isSuccessor} which test if the
instruction passed as a parameter is in the list of the successors of
the current instruction.  Thes two methods contains loops, so the
proof obligations generated are quite differents. We have to use some
arithmetic to prove their termination, which is not automated since it
does not appear often.  Finally the Verifier class was harder to
prove. One of the main reason is that the main method contains 2
loops; the inner one, easy to verify (a bit more difficult than the
ones contained in the Instruction class) because it simply consult
each intruction of the instruction array representing the program, but
with much more properties expressed on it. The proof obligations
generated for the main loop containing the inner loop are harder to
prove because the loop terminates only if it reaches a fixpoint, so it
cannot be expressed by simple arithmetic like the previous ones.  The
lemmas of this class were containing too many hypotheses to be
automatically proved.  In fact for each class some properties are
brought from the previous ones, this adds lots of hypothesis but it
adds too to the complexity of the proofs.  That's why there was some
kind of exponentional growth in the size of the proof obligations for
each class when they were defined.  Around 500 proof obligations had
to be solved manually.  Some of them were obvious and were resolved
with quite the same script, but the script cannot be automated.  Some
of them were complex: the proof script became little large (an average
of 30 steps).  As one should expect the lemmas concerning the loop
invariant of the verify method and its initialization were the most
difficult to prove.


%The first proofs are relatively easy. 
%The State class is proved almost automatically; 
%except for the constructor where it is necessary to break a disjunction ({\tt instance S $\vee$ S = null}) to prove the invariants.

%The Instruction class has also been relatively easy to prove. 
%A significant number of proof was done automatically (approximately 90 \%); 
%then majority of proof could be trivially resolved, except some lemma concerning  a loop termination.    

%Finally the Verifier class was harder to prove.
%The lemmas were containing too many hypotheses to be automatically proved.
%Around 500 proof obligations have to be resolved manually.
%Some of them was obvious and were resolved with quite the same script, but the script cannot be automated.
%Some of them was complex: the proof script became little large (an average of 30 steps).
%The lemmas concerning the loop invariant of the verify method and its initialization were the most difficult.

\section{JavaCard applet security properties}
To show the usefulness of the security property propagation, we applied it to
several realistic examples of smart card applications. When doing
this, we actually found violations against the security policies
documented for some of these applications.
\subsection{High-level Security Properties for Applets}
\label{SecHighLevelSecProp}

The properties that we consider can be divided in several groups,
related to different aspects of smart cards. First of all there are
properties dealing with the so-called \emph{applet life cycle},
describing the different phases that an applet can be in. Many actions
can only be performed when an applet is in a certain phase. Second,
there are properties dealing with the transaction mechanism, the Java
Card solution for having atomic updates. Further there are properties
restricting the kind of exceptions that can occur, and finally, we
consider properties dealing with access control, limiting the possible
interactions between different applications. For each group we present
some example properties. For all these properties encodings into JML
annotations exist.

\paragraph {Applet life cycle}

A typical applet life cycle defines phases as {\it loading},
{\it installation}, {\it personalization}, {\it selectable},
{\it blocked} and {\it dead}
(see \emph{e.g.}~\/\cite{MarletLM01}).  Each phase corresponds to a
different moment in the applet's life. First an applet is loaded on
the card, then it is properly installed and registered with the Java
Card Runtime Environment. Next the card is personalized,
\emph{i.e.}~all information about the card owner, permissions, keys
\emph{etc.} is stored. After this, the applet is selectable, which means
that it can be repeatedly selected, executed, and deselected. However,
if a serious error occurs, for example there
have been too many attempts to verify a pin code, the card can get
blocked or even become dead. From the latter state, no recovery is
possible.

In many of these phases, restrictions apply on who can perform
actions, or on which actions can be performed. These restrictions give
rise to different security properties, to be obeyed by the applet.

\begin{quote}
\textbf{Authenticated initialization} Loading, installing and 
personalizing the applet can only be done by an authenticated
authority.\smallskip\\
\textbf{Authenticated unblocking} When the card is blocked,
only an authenticated authority can execute commands and possibly
unblock it.\smallskip\\
\textbf{Single personalization} An applet can be
personalized only once.
\end{quote}


\paragraph {Atomicity}

A smart card does not include a power supply, thus a brutal retrieval
from the terminal could interrupt a computation and bring the system in
an incoherent state. To avoid this, the Java Card
specification prescribes the use of a transaction mechanism to
control synchronized updates of sensitive data. A 
statement block surrounded by the methods \texttt{beginTransaction()} and
\texttt{commitTransaction()} can be considered atomic.
If something happens while executing the transaction (or if
\texttt{abortTransaction()} is executed), the card will
roll back its internal state to the state before the transaction was
begun.

To ensure the proper functioning and prevent abuse of this mechanism,
several security properties can be specified.

\begin{quote}
\textbf{No nested transactions} Only one level of transactions
is allowed.\smallskip\\
\textbf{No exception in transaction} All exceptions that may be thrown
inside a transaction, should also be caught inside the
transaction.\smallskip\\
\textbf{Bounded retries}
No pin verification may happen within a transaction.
\end{quote} 
The second property ensures that the \texttt{commitTransaction} will
always be executed. If the exception is not caught, the
\texttt{commitTransaction} would be ignored and the transaction would
not be finished. The last property excludes pin verification within a
transaction. If this would be allowed, one could abort the transaction
every time a wrong pin code has been entered. As this rolls
back the internal state to the state before the transaction was
started, this would also reset the retry counter, thus allowing an
unbounded number of retries. Even though the specification of the Java
Card API prescribes that the retry counter for pin verification cannot
be rolled back, in general one has to check this kind of properties.

\paragraph{Exceptions}

Raising an exception at the top level can reveal
information about the behavior of the application and in principle it
should be forbidden. However, sometimes it is necessary to pass on
information about a problem that occurred. Therefore, the Java
Card standard defines so-called ISO exceptions, where a pre-defined
status word explains the problem encountered. These exceptions are the
only exceptions that may be visible at top-level; all other exceptions
should be caught within the application.

\begin{quote}
\textbf{Only ISO exceptions at top-level}
No exception should be visible at top-level, except ISO
exceptions.
\end{quote}

\paragraph {Access control} 

Another feature of Java Card is an isolation mechanism between
applications: the firewall. The firewall ensures that several
applications can securely co-exist on the same card, while managing
limited collaboration between them: classes and interfaces defined in
the same package can freely access each other, while external classes
can only be accessed via explicitly shared
interfaces. Inter-application communication via shareable interfaces
should only take place when the applet is selectable, in all other
phases of the applet life cycle only authenticated authorities are
allowed to access the applet. 
\begin{quote}
\textbf{Only selectable applications shareable} An application is accessible
via a shareable interface only if it is selectable.
\end{quote}


\subsection{Automatic Verification of Security Properties}\label{SecVerif}
As explained above, we are interested in the verification of
high-level security properties that are not directly related to a
single method or class, but that guarantee the overall
well-functioning of an application. Writing appropriate JML
annotations for such properties is tedious and error-prone, as they
have to be spread all over the application. Therefore, we propose a
way to construct such annotations automatically. First we synthesize
core-annotations for methods directly involved in the property.  For
example, when specifying that no nested transactions are allowed, we
annotate the methods \texttt{beginTransaction},
\texttt{commitTransaction} and
\texttt{abortTransaction}. Subsequently, we propagate the necessary 
annotations to all methods (directly or indirectly) invoking these
core-methods.  The generated annotations are sufficient to respect the
security properties, \emph{i.e.}~if the applet does not violate the
annotations, it respects the corresponding high-level security
property.

Whether the applet respects its annotations can be established with
JACK~\cite{BRL-JACK}.
Since
for most security properties the annotations are relatively
simple---but there are many---it is important that these verifications
are done automatically, without any user interaction. The results in
Section~\ref{SecResults} show that for the generated annotations all
correct proof obligations can indeed be automatically discharged.

\section{Comparison between source and bytecodes \\ proofs}  \label{results}

The purpose of this section is to give a comparison between bytecode and source proof obligations.
In particular, we illustrate this by the proof obligations of the example program in Fig.\ref{replaceSrc}.
%We have an implementation of the JML compiler ( subsection \ref{comJML}) and the bytecode verification condition generator based on the weakest precondition calculus (Section \ref{wpbc}) which are integrated in JACK. Both of the verification condition generators perform the same simplifications over the verification conditions 
%, e.g. eliminate verification conditions that contain contradictory hypothesis or trivial goals (equal to true). 

%The performed tests show that JML compilation augments around twice the file size. 
%For the example in Fig.~\ref{replaceSrc}, the class file without the specification extensions is 548 bytes, 
%and the class with the BCSL extension BCSL is 954 bytes. 
%The size of the bytecode specification is proportional to the source specification: 
%the bigger is the source specification, the greater will be the size of the class file. 
We studied the relationship between the source code proof obligations generated 
by the standard feature of JACK and the bytecode proof obligations generated by our implementation over the corresponding bytecode
 produced by a non optimizing compiler over the examples given in \cite{JPVC03JKM}. The proof obligations were the same modulo 
program variables names and basic types.

 We return now to our example from the previous sections and give in Fig. \ref{vcEnsures} one of the proof obligations on source 
and bytecode level respectively concerning the postcondition correctness. The verification conditions on bytecode and source level
 have the same shape modulo names (see Section \ref{comJML} for how names are compiled). Also in Section
 \ref{comJML}, we discussed the compilation of the JML postcondition from Fig. \ref{replaceSrc}. Particularly,
 we saw that the compiler has to transform the source postcondition in an equivalent formula and we
 gave the compilation in Fig. \ref{postCompile}. 

Despite those transformations, the source and bytecode goal respectively (which are actually the postcondition) on bytecode and source level are not only
semantically equivalent but syntactically the same (except for the variable names ). Still, in the bytecode proof obligation we have one more hypothesis than on source level. The extra hypothesis in the bytecode proof obligation is related to the fact that the result type is boolean but the JVM encodes boolean expressions as integers (which is trivially true). This means that the proof obligations have also the same shape.

 Another important issue is the impact of simple optimizations like dead code elimination on the relationship between source and bytecode proof obligations. 
In this case, the compiler does not generate the dead code and the bytecode verification condition generator will neither ``see'' it. 
Even though the source contains the never taken branch as the condition is equivalent to false, this will result in a trivially true
verification condition which the JACK source verification condition generator will discard.

The equivalence between source and bytecode proof obligations can be exploited in PCC scenarios, as we discussed in Section \ref{architecture} where 
the producer generates the program certificate over the source code in scenarios where a complete automatic certification (e.g. the certifying compiler) will not work.
 
We aim to formally give evidence that the proof obligations on non optimized bytecode and source programs are syntactically the same (modulo names and types). 

%Fig. \ref{vcLoopPreserv} shows the proof obligations for the loop preservation. As you can see the hypothesis and the goal have the same ``shape'' on bytecode and source code and the differences are due to the variable names.



% \begin{figure}{!h}

% $$\begin{array}{ll}
%Hypothesis \ on \ bytecode:  & Hypothesis \ on \ source \ level:  \\
%% & \\

%
%\begin{array}{l}
% \register{1} \neq \\
%\#19 (\register{0})[\register{2}\_at\_ins\_22]
%\end{array}  
%
%&  
%\begin{array}{l}
% \srcVar{obj} \neq \\
%  ListArray.list(\this)[\srcVar{i}\_at\_ins\_26] 
%\end{array}   \\
%
%
%
% & \\
%
%\#19(\register{0}) \neq \Mynull &  ListArray.list( \this) \neq \Mynull\\
%
%
%& \\
%
%\begin{array}{l}
%  len(\#19 (\register{0})) > \\
% \register{2}\_at\_ins\_22 
%\end{array}
%%& 
%\begin{array}{l}
%  len(ListArray.list(\this)) > \\
%\srcVar{i}\_at\_ins\_26
%\end{array}         \\ 

%

% & \\
%
% \register{2}\_at\_ins\_22 \geq 0 &   \srcVar{i}\_at\_ins\_26    \geq 0    \\
%
%
% & \\
%\begin{array}{l}
%  \register{2}\_at\_ins\_22 < \\
%  len(\#19(\register{0}))
%\end{array} &
%
%\begin{array}{l}
%  \srcVar{i}\_at\_ins\_26 <\\
%  len(ListArray.list(\this))
%\end{array}   \\
%
%
% & \\
%\begin{array}{l}
%  \register{2}\_at\_ins\_22 \leq \\
%  len( \#19(\register{0}))
%\end{array} 
%&  
%\begin{array}{l} 
%  \srcVar{i}\_at\_ins\_26 \leq \\
%  len(ListArray.list(\this))
%\end{array}   \\
%
%
% &\\
%% \register{2}\_at\_ins\_22 \geq 0 &   \srcVar{i}\_at\_ins\_26 \geq 0 \\
%
%
%
%% &\\
% \begin{array}{l} 
%         \forall  var(0). \ 0 \leq var(0) \wedge var(0) < (\register{2}\_at\_ins\_22) \Rightarrow \\
%                \Myspace    \#19(\register{0})[var(0)] \neq \register{1}
%      \end{array} &        
%      \begin{array}{l} 
%             \forall  var(0). \ 0 \leq var(0) \wedge var(0) < (\srcVar{i}\_at\_ins\_26) \Rightarrow \\
%                 \Myspace       ListArray.list(\this)[var(0)] \neq \srcVar{obj}
%      \end{array}  \\
%%
% typeof(\register{0}) <: ListArray &    typeof(this) <: ListArray     \\
%
%& \\
%& \\
%Goal \ on \ bytecode: & Goal \ on \ source \ level: \\
%
%& \\
%
%  \begin{array}{l}
%               1 + \register{2}\_at\_ins\_22 \leq  len(ListArray.list(\register{0}))  \\
%
%               1 + \register{2}\_at\_ins\_22 \geq 0 \\
%
%               \forall  var(0). 0 \leq var(0) \wedge var(0) < 1 + \register{2}\_at\_ins\_22 \Rightarrow \\
%                   \Myspace  ListArray.list(\register{0})[var(0)] \neq \register{1} 
%
%       \end{array}
%& 
%
%       \begin{array}{l}
%             1 + \srcVar{i}\_at\_ins\_26 \leq  len(ListArray.list(this))  \\
%	     \\
%             1 + \srcVar{i}\_at\_ins\_26 \geq 0 \\
%	     \\
%             \forall  var(0). 0 \leq var(0) \wedge\\
%	     \Myspace  var(0) < 1 + \srcVar{i}\_at\_ins\_26 \Rightarrow \\
%                  \Myspace  ListArray.list(this)[var(0)] \neq \srcVar{obj} 
%       \end{array}   
%
 
%\end{array}$$



%\caption{Source and Bytecode verification condition for loop preservation for method \texttt{ListArray.isElem} }
%\label{vcLoopPreserv}
%\end{figure}








\begin{figure*}[!h]

\begin{center}
\begin{tabular}{|l|l|}
\hline
\bf{Hypothesis \ on \ bytecode:}  & \bf{Hypothesis \ on \ source \ level:}  \\
\hline 
$\register{2}\_at\_ins\_20 \geq $ 
& $ \srcVar{i}\_at\_ins\_26 \geq$ \\

$len(\#19(\register{0})) $ & $  len(ListArray.list(\this)) $ \\
\hline 

$\#19(\register{0}) \neq \Mynull$ 
& $ ListArray.list(\this) \neq \Mynull$ \\

\hline 
$ \register{2}\_at\_ins\_20) \leq$ 
&  $  \srcVar{i}\_at\_ins\_26  \leq   $ \\
$ len(\#19(\register{0}))  $ & $ len(ListArray.list(\this))  $ \\
\hline

$\register{2}\_at\_ins\_20 \geq 0  $ 
& $ \srcVar{i}\_at\_ins\_26  \geq 0 $ \\

\hline

$\forall  var(0). \  0 \leq var(0) \wedge  $ & $\forall  var(0). \  0 \leq var(0) \wedge $ \\
$ var(0) < \register{2}\_at\_ins\_20 \Rightarrow $ & $  var(0) < \srcVar{i}\_at\_ins\_26 \Rightarrow $\\
$ \#19(\register{0})[var(0)] = \register{1}   $ & $  ListArray.list(\this)[var(0)] = \srcVar{obj}  $ \\

\hline

 $typeof(\register{0}) <: ListArray$ & $typeof( \this) <:  ListArray$  \\
\hline

$0=0 \vee 0=1$ & \\

& \\

\hline
\bf{Goal on bytecode:} & \bf{Goal on source level:} \\
\hline
$\Myfalse  \iff $ & $\Myfalse \iff  $ \\
 $ \exists  var(0) . \ 0 \leq var(0) \wedge$ 
& $ \exists  var(0) . \ 0 \leq var(0) \wedge$ \\

$\Myspace \Myspace var(0) < len(\#19(\register{0})) \wedge$ 
& $\Myspace \Myspace  var(0) < len(ListArray.List(this)) \wedge $\\
       
$\Myspace \Myspace \#19(\register{0})[var(0)] = \register{1} $ 
&$\Myspace \Myspace  ListArray.List(this)[var(0)] = \srcVar{obj}  $ \\

\hline
\end{tabular}
\end{center}

Note: $\expression\_at\_ins\_n$ denotes the value of  
expression $\expression$ at the bytecode instruction at index (source line)  $n$ 

\caption{\sc Comparison of source and bytecode verification conditions}
\label{vcEnsures}
\end{figure*}

\clearpage
 















%\subsubsection{Example}
% We give a simple example of how the \wpi \ works. Block $\blockm{6}$ (starts at instr. \texttt{6}) in Fig.~\ref{blockBC} ends with a branching instruction and in the case when the condition is true (the current element of the array is not equal to the first parameter of the method \texttt{replace}) the execution will continue at $\blockm{19}$. Below we give the part of the weakest precondition for block $\blockm{6}$ in case the control flows to block $\blockm{19}$( the condition of its last instruction holds and in this case 
%the predicate $pre(b^{6}, b^{19})$ is $\wpi(\blockm{19})$).  The implications with conclusion \Myfalse \ stand for the possible exceptions \texttt{NullPointer} and \texttt{ArrayIndexOutOfBound} exceptions that may be thrown (as no postcondition is specified explicitly for these cases of abnormal termination, the one by default is taken). 

%\input wpExample.tex

\section{Low-Footprint Java-to-Native Compilation}
% Pertinence of compiling bytecode into native code for embedded devices
Enabling Java on embedded and restrained systems is an important challenge for today's industry and research groups~\cite{Mulchandani1998}. Java brings features like execution safety and low-footprint program code that make this technology appealing for embedded devices which have obvious memory restrictions, as the success of Java Card witnesses. However, the memory footprint and safety features of Java come at the price of a slower program execution, which can be a problem when the host device already has a limited processing power. As of today, the interest of Java for smart cards is still growing, with next generation operating systems for smart cards that are closer to standard Java systems~\cite{Lagosanto2002,Grimaud2003}, but runtime performance in still an issue. To improve the runtime performances of Java systems, a common practice is to translate some parts of the program bytecode into native code.

% Cost of native code
Doing so removes the interpretation layer and improves the execution speed, but also greatly increases the memory footprint of the program: it is expected that native code is about three to four times the size of its Java counterpart, depending on the target architecture. This is explained by the less-compact form of native instructions, but also by the fact that many safety-checks that are implemented by the virtual machine must be reproduced in the native code. For instance, before dereferencing a pointer, the virtual machine checks whether it is \texttt{null} and, if it is, throws a \texttt{NullPointerException}. Every time a bytecode that implements such safety-behaviors is compiled into native code, these behaviors must be reproduced as well, leading to an explosion of the code size. Indeed, a large part of the Java bytecode implement these safety mechanisms.

% Usefulness of runtime checks
Although the runtime checks are necessary to the safety of the Java virtual machine, they are most of the time used as a protection mechanism against programming errors or malicious code: A runtime exception should be the result of an exceptional, unexpected program behavior and is rarely thrown when executing sane code - doing so is considered poor programming practice. The safety checks are therefore without effect most of the time, and, in the case of native code, uselessly enlarge the code size.

% Our contribution
Several studies proposed to factorize these checks or in some case to eliminate them, but none proposed a complete elimination without hazarding the system security. In this paper, we use formal proofs to ensure that run-time checks can never be true into a program, which allows us to completely and safely eliminate them from the generated native code. The programs to optimize are JML-annotated against runtime exceptions and verified by the Java Applet Correctness Kit (JACK~\cite{BRL-JACK}). We have been able to remove almost all of the runtime checks on tested programs, and obtained native ARM thumb code which size was comparable to the original bytecode.

\subsection{Java and Ahead-of-Time Compilation}
\label{sec:sota}

Compiling Java into native code common on embedded devices. This section gives an overview of the different compilation techniques of Java programs, and points out the issue of runtime exceptions.

\subsubsection{Ahead-of-Time \& Just-in-Time Compilation}

% JIT & AOT compilations
Ahead-of-Time (AOT) compilation is a common way to improve the efficiency of Java programs. It is related to Just-in-Time (JIT) compilation by the fact that both processes take Java bytecode as input and produce native code that the architecture running the virtual machine can directly execute. AOT and JIT compilation differ by the time at which the compilation occurs. JIT compilation is done, as its name states, just-in-time by the virtual machine, and must therefore be performed within a short period of time which leaves little room for optimizations. The output of JIT compilation is machine-language. On the contrary, AOT compilation compiles the Java bytecode way before the program is run, and links the native code with the virtual machine. In other words, it translates non-native methods into native methods (usually C code) prior to the whole system execution. AOT compilers either compile the Java program entirely, resulting in a 100\% native program without a Java interpreter, or can just compile a few important methods. In the latter case, the native code is usually linked with the virtual machine. AOT compilation have no or few time constraints, and can generate optimized code. Moreover, the generated code can take advantage of the C compiler's own optimizations.

% Why JIT is not applicable to embedded devices
JIT compilation in interesting by several points. For instance, there is no prior choice about which methods must be compiled: the virtual machine compiles a method when it appears that doing so is beneficial, e.g. because the method is called often. However, JIT compilation requires embedding a compiler within the virtual machine, which needs resources to work and writable memory to store the compiled methods. Moreover, the compiled methods are present twice in memory: once in bytecode form, and another time in compiled form. While this scheme is efficient for decently-powerful embedded devices such as PDAs, it is inapplicable to very restrained devices like smartcards or sensors. For them, ahead-of-time compilation is usually preferred because it does not require a particular support from the embedded virtual machine outside of the ability to run native methods, and avoids method duplication. AOT compilation has some constraints, too: the compiled methods must be known in advance, and dynamically-loading new native methods is forbidden, or at least very unsafe.

% Runtime exceptions
Both JIT and AOT compilers must produce code that exactly mimics the behavior of the Java virtual machine. In particular, the safety checks performed on some bytecode must also be performed in the generated code.

\subsubsection{Java Runtime Exceptions}
\label{sec:runtimeexceptions}
% Why runtime exceptions
The JVM (Java Virtual Machine)~\cite{VMSpec} specifies a safe execution environment for Java programs. Contrary to native execution, which does not automatically control the safety of the program's operations, the Java virtual machine ensures that every instruction operates safely. The Java environment may throw predefined runtime exceptions at runtime, like the following ones:

If the JVM detects that executing the next instruction will result in an inconsistency or an illegal memory access, it throws a runtime exception, that may be caught by the current method or by other methods on the current stack. If the exception is not caught, the virtual machine exits. This safe execution mode implies that many checks are made during runtime to detect potential inconsistencies. 

Of the 202 bytecodes defined by the Java virtual machine specification, we noticed that 43 require at least one runtime exception check before being executed. While these checks are implicitly performed by the bytecode interpreter in the case of interpreted code, they must explicitly be issued every time such a bytecode is compiled into native code, which leads to a code size explosion. Ishizaki et al. measured that bytecodes requiring runtime checks are frequent in Java programs: for instance, the natively-compiled version of the SPECjvm98 \texttt{compress} benchmark has 2964 exception check sites for a size of 23598 bytes. As for the \texttt{mpegaudio} benchmark, it weights 38204 bytes and includes 6838 exception sites~\cite{Ishizaki1999}. The exception check sites therefore make a non-neglectable part of the compiled code.

\subsection{Optimizing Ahead-of-Time Compiled Java Code}
\label{sec:method}

%verification procedure
Verifying that a bytecode program does not throw Runtime exceptions using JACK involves several stages:
\begin{enumerate}
\item writing the JML specification at the source level of the application, which expresses that no runtime exceptions are thrown.
\item compiling the Java sources and their JML specification\footnote{the BML specification is inserted in user defined attributes in the class file and so does not violate the class file format}.
\item generating the verification conditions over the bytecode and its BML specification, and proving the verification conditions~\ref{proofs}. During the calculation process of the verification conditions, they are indexed with the index of the instruction in the bytecode array they refer to and the type of specification they prove (e.g. that the proof obligation refers to the exceptional postcondition in case an exception of type \texttt{Exc} is thrown when executing the instruction at index \texttt{i} in the array of bytecode instructions of a given method). Once the verifications are proved, information about which instructions can be compiled without runtime checks is inserted in user defined attributes of the class file.
\item using these class file attributes in order to optimize the generated native code. When a bytecode that has one or more runtime checks in its semantics is being compiled, the bytecode attribute is checked in order to make sure that the checks are necessary. It indicates that the exceptional condition has been proved to never happen, then the runtime check is not generated.
\end{enumerate}

Our approach benefits from the accurateness of the JML specification and from the bytecode verification condition generator. Performing the verification over the bytecode allows to easily establish a relationship between the proof obligations generated over the bytecode and the bytecode instructions to optimized.

In the rest of this section, we explain in detail all the stages of the optimization procedure.

\subsubsection{Methodology for Writing Specification Against Runtime Exception}

We now illustrate with an example which annotations must be generated in order to check if a method may throw an exception. Figure~\ref{fig:jmlexample}\footnote{although the analysis that we describe is on bytecode level, for the sake of readability, the examples are also given on source level} shows a Java method annotated with a JML specification. The method \verb!clear! declared in class \verb!Code_Table! receives an integer parameter \verb!size! and assigns \verb!0! to all the elements in the array field \verb!tab! whose indexes are smaller than the value of the parameter \verb!size!. The specification of the method guarantees that if every caller respects the method precondition and if every execution of the method guarantees its postcondition then the method \verb!clear! never throws an exception of type or subtype \verb!java.lang.Exception!\footnote{Note that every Java runtime exception is a subclass of \texttt{java.lang.Exception}}. This is expressed by the class and method specification contracts.
First, a class invariant is declared which states that once an instance of type \verb!Code_Table! is created, its array field \verb!tab! is not null. The class invariant guarantees that no method will throw a \verb!NullPointerException! when dereferencing (directly or indirectly) \verb!tab!.

\begin{figure}
\begin{verbatim}
final class Code_Table {
  private/*@spec_public */short tab[];

  //@invariant tab != null;

  ...

  //@requires size <= tab.length;
  //@ensures true;
  //@exsures (Exception) false;
  public void clear(int size) {
  1  int code;
  2  //@loop_modifies code, tab[*];
  3  //@loop_invariant code <= size && code >= 0;
  4  for (code = 0; code < size; code++) {
  5    tab[code] = 0;
     }
  }
}
\end{verbatim}

\caption{\sc A JML-annotated method}
\label{fig:jmlexample}
\end{figure}

The method precondition requires the \verb!size! parameter to be smaller than the length of \verb!tab!. The normal postcondition, introduced by the keyword \verb!ensures!, basically says that the method will always terminate normally, by declaring that the set of final states in case of normal termination includes all the possible final states, i.e. that the predicate \verb!true! holds after the method's normal execution\footnote{Actually, after terminating execution the method guarantees that the first \texttt{size} elements of the array tab will be equal to 0, but as this information is not relevant to proving that the method will not throw runtime exceptions we omit it}. On the other hand, the exceptional postcondition for the exception \texttt{java.lang.Exception} says that the method will not throw any exception of type \texttt{java.lang.Exception} (which includes all runtime exceptions). This is done by declaring that the set of final states in the exceptional termination case is empty, i.e. the predicate \texttt{false} holds if an exception caused the termination of the method. The loop invariant says that the array accesses are between index \verb!0! and index \verb!size - 1! of the array \verb!tab!, which guarantees that no loop iteration will cause a \verb!ArrayIndexOutOfBoundsException! since the precondition requires that \verb!size <= tab.length!.

Once the source code is completed by the JML specification, the Java source is compiled using a normal non-optimizing Java compiler that generates debug information like \textrm{LineNumberTable} and \textrm{LocalVariableTable}, needed for compiling the JML annotations. From the resulting class file and the specified source file, the JML annotations are compiled into BML and inserted into user-defined attributes of the class file. 

For generating the verification conditions, we use a bytecode verification condition generator (vcGen) based on a bytecode weakest precondition calculus~\cite{JBL05MP}. 

\subsubsection{From Program Proofs to Program Optimizations }
\label{proofs}
In this phase, the bytecode instructions that can safely be executed without runtime checks are identified. Depending on the complexity of the verification conditions, Jack can discharge them to the fully automatic prover Simplify, or to the Coq and AtelierB interactive theorem prover assistants.
There are several conditions to be met for a bytecode instruction to be optimized safely -- the precondition of the method the instruction belongs to must hold every time the method is invoked, and the verification condition related to the exceptional termination must also hold.
Once identified, proved instructions can be marked in user-defined attributes of the class file so that the compiler can find them.

\subsubsection{More Precise Optimizations}

\label{section:optimprecise}

As we discussed earlier, in order to optimize an instruction in a method body, the method precondition must be established at every call site and the method implementation must be proved not to throw an exception under the assumption that the method precondition holds. This means that if there is one call site where the method precondition is broken then no instruction in the method body will be optimized.

Actually, the analysis may be less conservative and therefore more precise. We illustrate with an example how
one can achieve more precise results.

Consider the example of figure \ref{fig:jmlpreciseex}. On the left side of the figure, we show source code for method \verb!setTo0! which sets the \verb!buff! array element at index \verb!k! to 0. On the right side, we show the bytecode of the same method. The \texttt{iastore} instruction at index \texttt{3} may throw two different runtime exceptions: \texttt{NullPointerException}, or \texttt{ArrayIndexOutOfBoundException}. For the method execution to be safe (i.e. no Runtime exception is thrown), the method requires some certain conditions to be fulfilled by its callers. Thus, the method's precondition states that the \verb!buff! array parameter must not be null and that the \verb!k! parameter must be inside the bounds of \verb!buff!. If at all call sites we can establish that the \verb!buff! parameter is always different from null, but there are sites at which an unsafe parameter \verb!k! is passed the optimization for \texttt{NullPointerException} is still safe although the optimization for \texttt{ArrayIndexOutOfBoundException} is not possible. In order to obtain this kind of preciseness, a solution is to classify the preconditions of a method with respect to what kind of runtime exception they protect the code from. For our example, this classification consists of two groups of preconditions. The first is related to \texttt{NullPointerException}, i.e. \texttt{buff != null} and the second consists of preconditions related to \texttt{ArrayIndexOutOfBoundException}, i.e. \verb! k >= 0 && k <= buff.length!. Thus, if the preconditions of one group are established at all call sites, the optimizations concerning the respective exception can be performed even if the preconditions concerning other exceptions are not satisfied.

\begin{figure}
\begin{minipage}[b]{0.5\linewidth}
\begin{verbatim}
...

//@requires buff != null;
//@requires k >= 0 ;
//@requires k <= buff.length;
//@ensures true;
//@exsures (Exception) false;
public void setTo0(int k,int[] buff)
{
  buff[k] = 0;
}
\end{verbatim}
\end{minipage}
\hspace{.5cm}
\begin{minipage}[b]{0.4\linewidth}
 \begin{verbatim}
 0 aload_2
 1 iload_1
 2 iconst_0
 3 iastore
 4 return
\end{verbatim}
\end{minipage}
\caption{\sc The source code and bytecode of a method that may throw several exceptions}
\label{fig:jmlpreciseex}
\end{figure}

%\subsection{Experimental Results}
%\label{sec:experiments}

%This section presents an application and evaluation of our method on various Java programs.

%\subsubsection{Methodology}

%We have measured the efficiency of our method on two kinds of programs, that implement features commonly met in restrained and embedded devices. \benchname{crypt} and \benchname{banking} are two smartcard-range applications. \benchname{crypt} is a cryptography benchmark from the Java Grande benchmarks suite, and \benchname{banking} is a little banking application with full JML annotations used in~\cite{BRL-JACK}. \benchname{scheduler} and \benchname{tcpip} are two embeddable system components written in Java, which are actually used in the JITS~\cite{JITSWebsite} platform. \benchname{scheduler} implements a threads scheduling mechanism, where scheduling policies are Java classes. \benchname{tcpip} is a TCP/IP stack entirely written in Java, that implements the TCP, UDP, IP, SLIP and ICMP protocols. These two components are written with low-footprint in mind ; however, the overall system performance would greatly benefit from having them available in native form, provided the memory footprint cost is not too important.

%For every program, we have followed the methodology described in section \ref{sec:method} in order to prove that runtime exceptions are not thrown in these programs. We look at both the number of runtime exception check sites that we are able to remove from the native code, and the impact on the memory footprint of the natively-compiled methods with respect to the unoptimized native version and the original bytecode. The memory footprint measurements were obtained by compiling the C source file generated by the JITS AOT compiler using GCC 4.0.0 with optimization option \texttt{-Os}, for the ARM platform in thumb mode. The native methods sizes are obtained by inspecting the .o file with \texttt{nm}, and getting the size for the symbol corresponding to the native method.

%Regarding the number of eliminated exception check sites, we also compare our results with the ones obtained using the JC virtual machine mentioned in~\ref{sec:relatedwork}, version 1.4.6. The results were obtained by running the \texttt{jcgen} program on the benchmark classes, and counting the number of explicit exception check sites in the generated C code. We are not comparing the memory footprints obtained with the JITS and JC AOT compilers, for this result would not be pertinent. Indeed, JC and JITS have very different ways to generate native code. JITS targets low memory footprint, and JC runtime performance. As a consequence, a runtime exception check site in JC is heavier than one in JITS, which would falsify the experiments. Suffices to say that our approach could be applied on any AOT compiler, and that the most relevant measurement is the number of runtime exception check sites that remains in the final binary - our measurements on the native code memory footprint are just here to evaluate the size impact of exception check sites.

%\subsubsection{Results}
%\label{results}
%Table \ref{tab:nbexcsites} shows the results obtained on the four tested programs. The three first columns indicate the number of check sites present in the bytecode, the number of explicit check sites emitted by JC, and the number of check sites that we were unable to prove useless and that must be present in our optimized AOT code. The last columns give the memory footprints of the bytecode, unoptimized native code, and native code from which all proved exception check sites are removed.

%\begin{table}
%\caption{Number of exception check sites and memory footprints when compiled for ARM thumb}
%\begin{center}
%  \begin{tabular}{|l|r@{\extracolsep{0.2cm}}rrrrr|}
%    \hline
%    \multirow{2}*{Program} & \multicolumn{3}{c}{\# of exception check sites} & \multicolumn{3}{c|}{Memory footprint (bytes)}\\
%    \cline{2-4} \cline{5-7} & Bytecode & ~~~~~~JC & Proven AOT & Bytecode & Naive AOT & Proven AOT\\
%    \hline
%    \benchname{crypt} & 190 & 79 & 1 & 1256 & 5330 & 1592\\
%    \benchname{banking} & 170 & 12 & 0 & 2320 & 5634 & 3582\\
%    \benchname{scheduler} & 215 & 25 & 0 & 2208 & 5416 & 2504\\
%    \benchname{tcpip} & 1893 & 288 & 0 & 15497 & 41540 & 18064\\
%    \hline
%  \end{tabular}
%\end{center}
%\label{tab:nbexcsites}
%\end{table}

%On all the tested programs, we were able to prove that all but one exception check site could be removed. The only site that we were unable to prove from \benchname{crypt} is linked to a division, which divisor is a computed value that we were unable to prove not equal to zero. JC has to retain 16\% of all the exception check sites, with a particular mention for \benchname{crypt}, which is mainly made of array accessed and had more remaining check sites.

%The memory footprints obtained clearly show the heavy overhead induced by exception check sites. Despite of the fact that the exception throwing convention has deliberately been simplified for our experiments, optimized native code is less than half the size of the non-optimized native code. The native code of \benchname{crypt}, which heavily uses arrays, is actually made of exception checking code at 70\%.

%Comparing the size of the optimized native versions with the bytecode reveals that proved native code is just slightly bigger than bytecode. The native code of \benchname{crypt} is 27\% bigger than its bytecode version. Native \benchname{scheduler} only weights 13.5\% more that its bytecode, \benchname{tcpip} 16.5\%, while \benchname{banking} is 54\% heavier. This last result is explained by the fact that, being an application and not a system componant, \benchname{banking} includes many native-to-java method invokations for calling system services. The native-to-java calling convention is costly in JITS, which artificially increases the result.

%Finally, table \ref{tab:implication} details the human work required to obtain the proofs on the benchmark programs, by comparing the amount of JML code with respect to the comments-free source code of the programs. It also details how many lemmas had to be manually proved.

%\begin{table}
%\caption{Human work on the tested programs}
%\begin{center}
%  \begin{tabular}{|l|r@{\extracolsep{0.5cm}}rrr|}
%    \hline
%    \multirow{2}*{Program} & \multicolumn{2}{c}{Source code size (bytes)} & \multicolumn{2}{c|}{Proved lemmas}\\
%    \cline{2-3} \cline{4-5} & ~~~~~~~~~Code & JML & Automatically & Manually\\
%    \hline
%    \benchname{crypt} & 4113 & 1882 & 227 & 77 \\
%    \benchname{banking} & 11845 & 15775 & 379 & 159\\
%    \benchname{scheduler} & 12539 & 3399 & 226 & 49\\
%    \benchname{tcpip} & 83017 & 15379 & 2233 & 2191\\
%    \hline
%  \end{tabular}
%\end{center}
%\label{tab:implication}
%\end{table}

%On the three programs that are annotated for the unique purpose of our study, the JML overhead is about 30\% of the code size. The \benchname{banking} program was annotated in order to prove other properties, and because of this is made of more JML annotations than actual code. Most of the lemmas could be proved by Simplify, but a non-neglectable part needed human-assistance with Coq. The most demanding application was the TCP/IP stack. Because of its complexity, nearly half of the lemmas could not be proved automatically.

%The gain in terms of memory footprint obtained using our approach is therefore real. One may also wonder whether the runtime performance of such optimized methods would be increased. We did the measurements, and only noticed a very slight, almost undetectable, improvement of the execution speed of the programs. This is explained by the fact that the exception check sites conditions are always false when evaluated, and therefore the amount of supplementary code executed is very low. The bodies of the proved runtime exception check sites are, actually, dead code that is never executed.



\section{Memory consumption}

%\input BML/cmdBML.tex

\newcommand{\code}{\textit{code}}
\newcommand{\indexComp}{\textit{index}}





\section{Introduction} \label{bcsl}
This section presents a bytecode level specification language, called for short BML and a compiler from a
 subset of the high level Java specification language JML to BML. 

% motivation

 Before going further, we discuss what advocates the need of a low level specification language.
Traditionally, specification languages were tailored for high level languages.  
Source  specification allows to express complex functional or security properties about programs.
Thus, they are / can successfully be used 
for software audit and validation. Still, source specification in the context of mobile code does not help a lot for several reasons.


First, the executable / interpreted code  may not be accompanied by its specified  source. Second, it is more reasonable for the 
code receiver to check the executable code than its source code, especially if he is not willing to trust the compiler. 
Third, if the client has complex requirements and even if the code respects them, in order to establish them, 
the code should be specified. Of course, for properties like well typedness this specification can be inferred automatically,
but in the general case this problem is not decidable. 
Thus, for more sophisticated policies, an automatic inference will not work.

 It is in this perspective, that we propose to make the Java
bytecode benefit from the source specification by defining the BML language and a compiler from JML towards BML.    

% what does the language support?
 BML supports the most important features of JML. Thus, we can express functional properties of Java
 bytecode programs in the form of method pre and postconditions, class and object invariants, assertions
 for particular program points like loop invariants. To our knowledge BML does not have predecessors that are tailored 
 to Java bytecode.  

 In section \ref{BCSLprelim}, we give an overview of the main features of JML. A detailed overview of BML is given in section \ref{BCSLgrammar}.  
  As we stated before, we support also a compiler from the high level specification language JML into BML. The 
 compilation process from JML to BML is discussed in section  \ref{BCSLcompile}.
 The full specification of the new user defined Java attributes in which the JML specification is compiled is given in the appendix.





\subsection{Modeling Memory Consumption}\label{sec:verif}
The objective of this section is to demonstrate how the user can
annotate and verify programs in order to obtain an upper bound on
memory consumption. We begin by describing the principles of our
approach, then turn to a discussion for its soundness, and finally show
how it can be applied to non-trivial examples involving recursive
methods and exceptions.


\subsection{Principles}
Let us begin with a very simple memory consumption policy which aims
at enforcing that  programs do not consume more than
some fixed amount of memory \Max . To enforce this policy, we first
introduce a ghost variable \Mem{} that represents at any given point of
the program the memory used so far. Then, we annotate the program both
with the policy and with additional statements that will be used to
check that the application respects the policy.



\paragraph{The precondition} of the method $\method$ should ensure that
there must be enough free memory for the method execution. Suppose
that we know an upper bound of the allocations done by method $\method$
in any execution. We will denote this upper bound by
\allocMethod{\method}. Thus there must be at least
\allocMethod{\method}\ free memory units from the allowed \Max\ when
method $\method$ starts execution. Thus the precondition for the method
$\method$ is:
$$
\requires \ \Mem + \allocMethod{\method}  \leq \Max.
$$

%\todo{ to leave this paragraph or not. It is about the initialization of the variable \Mem} 
The precondition of the
program entry point (i.e., the method from which an application
may start its execution) should state that the program has not
allocated any memory, i.e. require that variable \Mem \ is  0:
$$
\requires \ \Mem == 0.
$$
\paragraph{The normal postcondition} of the method $\method$ must
guarantee that the memory allocated during a normal execution of
$\method$ is not more than some fixed number \allocMethod{\method}\
of memory units. Thus for the method $\method$ the postcondition is:
$$
\ensures \  \Mem \leq \old{\Mem} + \allocMethod{\method}.
$$

\paragraph{The exceptional postcondition} of the method $\method$ must
say that the memory allocated during an execution of $\method$ that
terminates by throwing an exception \texttt{Exception} is not more
than \allocMethod{\method}\ units. Thus for the method $\method$ the
exceptional postcondition is:

$$
\exsures{Exception} \  \Mem \leq \old{\Mem} + \allocMethod{\method}.
$$


\paragraph{Loops} must also be annotated with appropriate invariants. 
%Assuming that we know that loop $\progLoop{l}$ iterates no more than $\maxIter{l}$ as well as an upper bound  $\allocLoop{l}$ of the allocations done per iteration in $l$. 
Let us assume that loop $\progLoop{l}$ iterates no more than $\maxIter{l}$ and let $\allocLoop{l}$ be an upper bound of the memory allocated per iteration in $l$.
Below we give a general form of loop specification w.r.t. the property for constraint memory consumption. The loop invariant of a loop $\progLoop{l}$ states that at every iteration the loop body is not going to allocate more than $\allocLoop{l}$ memory units and that the iterations are no more than $\maxIter{l}$. We also declare an expression which guarantees loop termination, i.e. a variant (here an integer expression whose values decrease at every iteration  and is always bigger or equal to 0).
$$\begin{array}{ll}
\modifies &  \ i, \Mem \\
\invariant: & \ \Mem \le \atState{\Mem}{Before_{l}} + i * \allocLoop{l} \\
                & \wedge \\
                & i \le \maxIter{l}\\
\variant: & \maxIter{l} - i \\
\end{array}$$
 A special variable appears in the invariant, $\atState{\Mem}{Before_{l}}$. It denotes the value of the consumed memory just before entering for the first time the loop \progLoop{l}. At every iteration the consumed memory must not go beyond the upper bound given for the body of loop.

\paragraph{For every instruction that allocates memory} the ghost
variable \Mem\ must also be updated accordingly. For the purpose of
this paper, we only consider dynamic object creation with the bytecode
\new; arrays are left for future work and briefly discussed in the
conclusion. 

The function $\allocInstanceOnly: Class \rightarrow int$ gives an estimation of the memory used by an instance of a class. 
Note that the memory allocated for a class instance is specific to the implementation of the virtual machine.
%In order to perform the update for \new\ bytecodes, we must assume given a function $allocInstance: Class \rightarrow int$ which maps classes to an estimation of the memory that any instance of the class may occupy. 
At every program point where a bytecode \srcCode{\new \ A} is found, the ghost variable \Mem\ must be incremented by $\allocInstance{A}$. This
is achieved by inserting a ghost assignment immediately after any \new\ instruction, as shown below:
$$
\begin{array}{l}
\srcCode{\new \ A} \\
 // \set \ \Mem = \Mem + $\allocInstance{A}$.
\end{array}
$$

\subsection{Correctness}
%An important question is if the annotations that we prescribe here guarantees that the memory used in a program is not more than a fixed upper bound \Max. 
We want to guarantee that the memory allocated by a given program is bounded by a constant \Max.
We can prove that our annotation is correct w.r.t. to the policy for constraint memory use, by instrumenting the operational semantics of the bytecode language given in
Chapter \ref{prelim}, Section \ref{opSem}. The instrumented operational semantics
manipulates states as before, but it is extended with the special variable \Mem. Thus, states in the new semantics have the form:

$$ \configMem{\config{\heap}{\counterOnly}{\stackOnly}{\locVarOnly}{\pc}}{\Mem} $$ 

%The variable \Mem \ changes its value only for instructions that allocate space in the heap, i.e. \new\ instructions:

%$$\small{\frac{
%\begin{array}[c]{c}
%\ \InstAt(m,\pc)=\new \ A ,
%\end{array}}
%{\begin{array}[t]{c} \config{h,\fram{m,\pc,l,v::s},\stf, \Mem} \to_{\new\ A} \\ \config{h + \allocInstance{A},\fram{m,\pc+1,l,s},\stf ,\Mem + \allocInstance{A}}
%\end{array}}}$$



The other instructions do not affect \Mem, so the corresponding rules of the operational semantics are as before. As we saw in the previous section to every
instruction of the form $\new\ A$ we attach the annotation $\set\ \Mem = \Mem + \allocInstance{A}$. The proof obligation generator converts this annotation into new value for the variable \Mem:

$$
\begin{array}{l}
wp(\set \ \Mem = \Mem + \allocInstance{A}, \psi) = \\
\ \ \ \ \ \ \ \ \ \ \ \ \psi[ \Mem \leftarrow \Mem + \allocInstance{A} ]
\end{array}
$$

We can prove that whenever the allocated space in the heap increments, 
the ghost variable \Mem\ also increments, which is a sufficient condition to guarantee the correctness of the annotations. 
So far we do not deal with garbage collection (see discussion in Section \ref{sec:conc}).

\subsection{Examples}
We illustrate hereafter our approach by several examples. 
%\alarm{talk about number of proof obligations, which are discharged automatically in Coq, etc}

\subsubsection{Inheritance and overridden methods} Overriding methods are treated as follows: whenever a call is performed to a method \method,
we require that there is enough free memory space for the maximal
consumption by all the  methods that override or are overridden by
\method. In Fig. \ref{classExt} we show a class \verb!A! and its
extending class \verb!B!, where \verb!B! overrides the method \method\ from class \verb!A!. Method \method\ is invoked by $n$. Given that the dynamic type of the parameter passed to $n$ is not known, we cannot know which of the two
methods will be invoked. This is the reason for requiring enough memory space for the execution of any of these methods.
%After the method execution we consider the extreme case where there is executed the method \method\ that consumes the most.

\begin{figure}[!htp]
Specification of method \methodd{} in class A:
$$
\begin{array}{ll}
\requires & \Mem + k  \leq \Max \\
\modifies & \Mem \\
\ensures & \Mem  \leq \old{\Mem} + k
\end{array}
$$

Specification for method \methodd{} in class B:
$$
\begin{array}{ll}
\requires & \Mem + l  \leq \Max \\
\modifies & \Mem \\
\ensures & \Mem  \leq \old{\Mem} + l
\end{array}
$$

\begin{verbatim}
method n(A a)
...
//{ prove Mem <= Mem +max(l,k) }
invokevirtual m <A>
//{ assume Mem <= \old{Mem} + max(l,k)}
...
\end{verbatim}
\caption{\sc Example of overridden methods}
\label{classExt}
\end{figure}


\subsubsection{Recursive Methods} In Fig. \ref{recMeth} the bytecode of the recursive method \methodd{} and its specification is shown. 
 We show a simplified version of the bytecode; we assume that the constructors for the class \srcCode{A} and \srcCode{C}
 do not allocate memory. Besides the precondition and the postcondition, the specification also includes information 
 about the termination of the method: \variant\ $\local{1}$, meaning that the local variable $\local{1}$ decreases on every recursive call down to and no more than $0$, guaranteeing
 that the execution of the method will terminate.
 
%Now we explain why such a precondition is required for method \textbf{m} in order to specify the property for constraint memory consumption. 

We explain first the precondition. If the condition of line \srcCode{1} is not true, the execution continues at line \srcCode{2}.

\begin{figure}[!hbp]
\begin{alltt}
public class D \{
  public void m( int i) \{
    if (i > 0) \{
      new A();
      m(i - 1);
      new A();
    \} else \{
      new C();
      new A();
   \}
  \}
\}
\end{alltt}

$$
\begin{array}{ll}
 \requires & ( \Mem + \local{1}*2*\allocInstance{A} + \\
           &  \allocInstance{A} + \allocInstance{C}) \le \Max \\
 \variant  & \local{1} \\
 \ensures  & \local{1} \ge 0 \\
           & \wedge \\
           & \Mem <= \old{\Mem} +  \\
	   & \old{\local{1}}*2*\allocInstance{A} + \allocInstance{A}\\
           &  +  \allocInstance{C})
\end{array}$$

\begin{alltt}
\srcCode{\textbf{public void m()}}
//\small{\textit{local variable loaded on} }
//\small{\textit{the operand stack of method \textbf{m}}}
\srcCode{0 \load\_1}
//\small{ \textit{ if \local{1} <= 0 jump}}
\srcCode{1 ifle 12}
\srcCode{2 new <A>} //\small{ \textit{ here \local{1} > 0  } }
//set \Mem = \Mem +  \allocInstance{A}
\srcCode{3 invokespecial <A.<init>>}
\srcCode{4 aload\_0}
\srcCode{5 iload\_1}
\srcCode{6 iconst\_1}
//\small{\textit{\local{1} decremented with 1}}
\srcCode{7 isub}
//\small{ \textit{ recursive call with the new value of \local{1}}}
\srcCode{8 invokevirtual <D.m>}//
\srcCode{9 new <A>}
//set \Mem = \Mem +  \allocInstance{A}
\srcCode{10 invokespecial <A.<init>>}
\srcCode{11 goto 16}
//\small{\textit{target of the jump at \srcCode{1}}}
\srcCode{12 new <A>}
//set \Mem = \Mem +  \allocInstance{A}
\srcCode{13 invokespecial <A.<init>>}
\srcCode{14 new  <C>}
//set \Mem = \Mem +  \allocInstance{C}
\srcCode{15 invokespecial <C.<init>>}
\srcCode{16 return}
\end{alltt}

\caption{\sc Example of a recursive method}
 \label{recMeth}
\end{figure}

In the sequential execution up to line \srcCode{7}, the program allocates at most $\allocInstance{A}$ memory units and decrements by $1$ the value of $\local{1}$. The instruction at line \srcCode{8} is a recursive call to \methodd{}, which either will take the same branch if $\local{1} > 0 $ or will jump to line \srcCode{12} otherwise, where it allocates at most $\allocInstance{A} +  \allocInstance{C}$ memory units. On returning from the recursive call one more allocation will be performed at line \srcCode{9}.
 Thus \methodd{} will execute, $\local{1}$ times, the instructions from lines \srcCode{4} to \srcCode{35}, 
and it finally will execute all the instructions from lines  \srcCode{12} to \srcCode{16}.
The postcondition states that the method will perform no more
than $\old{\local{1}}$ recursive calls (i.e., the value of the register variable in the pre-state of the method) and that on every recursive call it allocates no more than two instances of class \texttt{A} and that it will finally allocate one instance of class \texttt{A} and another of class \texttt{C}.


\subsubsection{More precise specification} We can be more precise in specifying the precondition of a method by considering what are the field values of an instance, for example. Suppose that we have the method \method\ as shown in Fig. \ref{excMeth}. We assume that in the constructor of the class \texttt{A} no allocations are done. The first line of the method \method\ initializes one of the fields of field \texttt{b}. Since nothing guarantees that field \texttt{b} is not \Mynull, the execution may terminate with
\texttt{NullPointerException}. Depending on the values of the parameters passed to \method, the memory allocated will be different. The precondition establishes what is the expected space of free resources depending on if the field
\texttt{b} is \Mynull  or not. In particular we do not require anything for
the free memory space in the case when \texttt{b} is \Mynull. In the
normal postcondition we state that the method has allocated an
object of class \texttt{A}. The exceptional postcondition states
that no allocation is performed if \texttt{NullpointerException} causes the execution termination.

\begin{figure}[!hbp]
$$
\begin{array}{ll}
 \requires &  \local{1} != \Mynull \Rightarrow  \\
           & \phantom{\local{1}} \Mem +  \allocInstance{A} \le \Max \\
       %& \wedge \\
       %&  \local{1} == \Mynull \Rightarrow  \\
           %& \phantom{\local{1}} \Mem +  \allocInstance{B} + \allocInstance{A}   \le \Max \\
  \modifies & \Mem \\
  \ensures  & \Mem \le \old{\Mem} +  \allocInstance{A} \\
  \exsures{NullPointerException}  & \Mem == \old{\Mem}   \\
\end{array}$$

\begin{tabular}{lr}
\begin{minipage}[t]{170pt}
\begin{alltt}
\srcCode{0 aload\_0}
\srcCode{1 getfield<C.b>}
\srcCode{2 iload\_2}
\srcCode{3 putfield <B.i>}
\srcCode{4 new <A>}
//set \Mem = \Mem +
      \allocInstance{A}
\srcCode{5 dup}
\srcCode{6 invokespecial <A.<init>>}
\srcCode{7 astore\_1}
\srcCode{8 return}
\end{alltt}
\end{minipage}
 &
\begin{minipage}[t]{170pt}
\begin{alltt}
public class C \{
  B b;
  public void m(A a, int i) \{
    b.i = i ;
    a = new A();
  \}
\}
\end{alltt}
\end{minipage}
\end{tabular}
\caption{\sc Example of a method with possible exceptional termination}
\label{excMeth}
\end{figure}

\subsection{Inferring Memory Allocation}\label{sec:infer}
In the previous section, we have described how the memory consumption
of a program can be modelled in BCSL and verified using an appropriate
verification environment. While our examples illustrate the benefits
of our approach, especially regarding the precision of the analysis,
the applicability of our method is hampered by the cost of providing
the annotations manually. In order to reduce the burden of manually
annotating the program, one can rely on annotation assistants that
infer automatically some of the program annotations (indeed such
assistants already exist for loop invariants~\cite{NimmerE02:ISSTA} and class
invariants~\cite{log04:vmcai}). In this section, we describe an
implementation of an annotation assistant dedicated to the analysis of
memory consumption, and illustrate its functioning on an example.


\subsubsection{Annotation assistant}
The annotation assistant performs two tasks. First, it inserts the
ghost assignments on appropriate places; for this task, the user must
provide annotations about the memory required to create objects of the
given classes. 

Second, it inserts pre- and postconditions for each method. In this case, variants for loops and recursive methods may be given by the user or be
synthesised through appropriate mechanisms.  Based on this
information, the annotation assistant recursively computes the memory
allocated on each loop and method. Essentially, it finds the maximal
memory that can be allocated in a method by exploring all its possible
execution paths.

The function $\allocMethod{.}$ is defined as follows:
\begin{itemize}
\item \textbf{Input:} Annotated bytecode of a method \method, and memory
policies for methods that are called by \method;

\item \textbf{Output:} Upper bound of the memory allocated by \method;

\item \textbf{Body:} The first step is to compute the loop structure
of the method, then to compute an upper bound to the memory allocated
by each loop using its variant, and then to compute an upper bound to
the memory allocated along each execution path.
\end{itemize}



%A pseudo-code of the algorithm for inferring an upper bound for method
%allocations is given in Fig.~\ref{methodAlloc}.  Essentially, it finds
%the maximal memory that can be allocated in a method by exploring all
%its possible execution paths. in Fig.~\ref{methodAlloc} the auxiliary
%function $allocPath(\cdot)$ infers the allocations done by the set of
%execution paths ending with the same \return\ instruction.

%\begin{figure}[t]
%function $\allocMethod{.}$\\
%\textbf{Input:} Bytecode of a method $m$. \\
%\textbf{Output:} Upper bound of the memory allocated by $m$. \\
%\textbf{Body:}
%\begin{enumerate}
%   \item Detect all the loops in $m$;
%  \item For every loop $l$ determine $\loopSet{l}$, $\loopEntry{l}$ and $\loopEndsSet$;
%   \item Apply the function $\allocated{\cdot}$ to each instruction $i_k$, such that $i_k = \return$;
%  \item Take the maximum of the results given in the previous step:  $max_{i_k = \return } \allocated{i_k}$.
%\end{enumerate}
%\caption{\sc Inference algorithm}
%\label{methodAlloc}
%\end{figure}

%Inferring the memory allocated inside loops is done by the function $\allocLoopWithEnd{\cdot}{\cdot}$, which is invoked by $allocPath$ whenever the current instruction belongs to a loop. The specification of the function is shown in Fig. \ref{fig:loopPath} (where $P = max_{\instrAt{k} \in preds(\loopEntry{l'} ) - \loopEndsSet{\progLoop{l'}}}$).

%\begin{figure}[!ht]
%$\allocated{\instrAt{s}}$ = 
%$$ \left\{ \begin{array}{l}
%\allocIns{\instrAt{s} } \hspace*{1.8cm}  \mbox{if  $\instrAt{s}$  has  no  predecessors} \\
%            \allocLoop{\loopEntry{l}} \ + \\
%\ \ \ \ \            max_{\instrAt{k} \in preds(\instrAt{s} )-\loopEndsSet{\progLoop{l}}}( \allocated{\instrAt{k}} ) \\
%\hspace*{4cm}  \mbox{if  $\instrAt{s}\in \loopSet{\progLoop{l}}$} \\
%\allocIns{\instrAt{s}} \ + \ max_{\instrAt{k} \in preds(\instrAt{s} )}( \allocated{\instrAt{k}} ) \\
%\hspace*{4cm} \mbox{otherwise}
%\end{array}
%\right.
%$$
%\caption{\sc Definition of the function $\allocated{\instrAt{s}}$} 
%\label{fig:allocMethod}
%\end{figure}


%\begin{figure}[!ht]
%$\allocLoopWithEnd{\loopEntry{l}}{\instrAt{s}} = $
%$$ 
%\left\{\begin{array}{l}

% \allocIns{\loopEntry{l}}   \hspace*{1.8cm} \mbox{if $\instrAt{s} = \loopEntry{l}$} \\
%  \allocLoop{\loopEntry{l'}} \ + \\
%\ \ \ \ \      P(\allocLoopWithEnd{\loopEntry{l}}{\instrAt{k}}) \\
%\hspace*{2cm}  \mbox{if $\instrAt{s} \in  \loopSet{\progLoop{l'}} \ \land \ \progLoop{l'}$ is  nested in $\progLoop{l}$} \\

%     \allocIns{\instrAt{s}} \ + \\
%\ \ \ \ \     max_{\instrAt{k} \in preds(\instrAt{s} )}(\allocLoopWithEnd{\loopEntry{l}}{\instrAt{k}}) \\
% \hspace*{5cm} \mbox{otherwise}
%\end{array} \right.
%$$
% \caption{\sc Definition of the function $\allocLoopWithEnd{\loopEntry{l}}{\instrAt{s}}$}
%\label{fig:loopPath}
%\end{figure}

The annotation assistant currently synthesises only simple memory
policies (i.e., whenever the memory consumption policy does not depend
on the input).  Furthermore, it does not deal with arrays,
subroutines, nor exceptions, and is restricted to loops with a unique
entry point. The latter restriction is not critical because it
accommodates code produced by non-optimising compilers. However, a
pre-analysis could give us all the entry points of more general loops,
for instance by the algorithms given in \cite{CJPS05cmu}; our approach
may be thus applied straightforwardly. How to treat arrays is
briefly discussed in the conclusion.


\subsubsection{Example}

Let us consider the bytecode given in Fig. \ref{inf:src}, which is a
simplified version of the bytecode corresponding to the source code
given in the right of the figure. For simplicity of presentation, we
do not show all the instructions (the result of the inference
procedure is not affected). Method \method\ has two branching
instructions, where two objects are created: one instance of class \texttt{A}
and another of class \texttt{B}. Our inference algorithm gives that
$\allocMethod{\method} =$ $\allocInstance{A} +$ $\allocMethod{A.init}
+ \allocInstance{B} + \allocMethod{B.init}$.

%Due to limitation on space, we do not explain the details of such inference, which is given in Fig. \ref{inf:ex} ($\instrAt{k}$ refers to the bytecode instruction at position $k$).

\begin{figure}[!hbp]
\begin{tabular}{lr}
\begin{minipage}[t]{4.3cm}
\begin{alltt}
\begin{small}
\srcCode{0 aload\_1} 
\srcCode{1 ifnonnull 6 } 
\srcCode{2 new <A>}
... 
\srcCode{4 invokespecial <A.<init>>} 
\srcCode{6 aload\_2}
\srcCode{7 ifnonnull 12}
\srcCode{8 new <B>} 
... 
\srcCode{10 invokespecial <B.<init>>}
...
\srcCode{12 return}
\end{small}
\end{alltt}
\end{minipage} &

\begin{minipage}[t]{4cm}
\begin{alltt}
\small{
public void 
 m (A a , B b )   \{
  if (a == null) \{
    a = new A(); \}
  if (b == null) \{
    b = new B(); \}\}
}
\end{alltt}
\end{minipage}
\end{tabular}
\caption{\sc Example}
\label{inf:src}
\end{figure}

%The procedure presented above terminates as an acyclic
%representation of the control flow graph is used.


