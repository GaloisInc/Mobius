\parindent 1cm
\parskip 0.2cm
\topmargin 0.2cm
\oddsidemargin 1cm
\evensidemargin 0.5cm
\textwidth 15cm
\textheight 21cm

\newcommand{\comment}[1]{{\sf #1}}
\newcommand{\alarm}[1]{\marginpar{#1}}

\newtheorem{definition}{Definition}
\newtheorem{lemma}{Lemma}
\newtheorem{theorem}{Theorem}


\def \bsl       {\symbol{92}}
\def \unsc      {\symbol{95}}

\newcommand{\todo}[1]{ \textbf{#1}}
\newcommand{\fig}[1]{ Fig.}
\newcommand{\jmlKey}[1]{\texttt{#1}}% wrapping jml keywords
\newcommand{\java}[1]{\texttt{#1}}
\newcommand{\stack}[1]{\texttt{st(#1)}}% element on top stack 
\newcommand{\counter}{\texttt{ct}}

\newcommand{\true}{\texttt{true}}
\newcommand{\false}{\texttt{false}}

\newcommand{\wpi}{\textit{wp}}

\newcommand{\instr}[1]{\texttt{#1}}

\newcommand{\substitution}[2]{[\tt{#1} \leftarrow \tt{#2}]}
% thegrammar for the bytecode specification language
\newcommand{\ClassSpec}{\rm{ClassSpec}}
\newcommand{\MethodSpec}{\rm{MethodSpec}}
\newcommand{\SpecCase}{\textrm{SpecCase}}
\newcommand{\jmlStmt}[1]{\textrm{#1}}
\newcommand{\interMethodSpec}{\rm{InterMethodSpec}}
\newcommand{\loopSpec}{\rm{loopSpec}}
\newcommand{\assert}{\rm{assertSpec}}

\newcommand{\ArithExpr}{\texttt{Arithmetic\_Expr}}
\newcommand{\expression}{\mathcal{E} }

\newcommand{\integer}{\texttt{int} }
\newcommand{\register}[1]{\texttt{lv[#1]} }
\newcommand{\reference}{\texttt{ref} }
\newcommand{\intLiteral}{\texttt{int\_literal} }
\newcommand{\Mynull}{\texttt{null}}
\newcommand{\this}{\texttt{this}}
\newcommand{\fieldAccess}[1]{\texttt{field\_cp\_index(}#1 \texttt{)}}
\newcommand{\arrayAccess}[2]{#1[#2] }

\newcommand{\result}{\jmlKey{$\backslash$result}}
\newcommand{\old}[1]{\jmlKey{$\backslash$old(}#1\jmlKey{)}}
\newcommand{\typeof}[1]{\jmlKey{$\backslash$typeof(}#1 \jmlKey{)}}
\newcommand{\EXC}{\texttt{EXC}}

\newcommand{\excPost}{\psi^{exc}}

\newcommand{\Myspace}{\phantom{aa}}
\newcommand{\predicate}{ \mathcal{P}} 
\newcommand{\Myfalse}{\textit{false}}
\newcommand{\Mytrue}{ \textit{true} }
\newtheorem{defn}{Definition} 



% abstractCtrlFlow.tex
\newcommand{\execRel}{\rightarrow} % the execution relation
\newcommand{\blockm}[1]{ \tt{b^{#1}} }
\newcommand{\blockSeq}[1]{ \tt{b_{seq}^{#1}} }
\newcommand{\pathm}[2]{\blockm{#1} \execRel^{*} \blockm{#2} }

\newcommand{\blockPost}[1]{ \it{post(}\tt{b_{seq}^{#1}}\it{)}}

\newcommand{\invariant}{\textit{I}}

\newcommand{\srcVar}[1]{\texttt{#1} }

\newcommand{\method}{\texttt{m} }

% recuperé dans le prelude du code généré par l'Atelier B.
% Doit permettre d'écrire à peu-près correctement <+
\def\famletter#1{\ifcase #1 0\or 1\or 2\or 3\or 4\or 5\or 6\or 7\or
    8\or 9\or A\or B\or C\or D\or E\or F\fi}
\font\msx=msam10
\newfam\msxfam \textfont\msxfam=\msx
\edef\fx{\famletter\msxfam}
\mathchardef    \dres       "2\fx43
\def    \lover      {\mathbin{{\dres} \llap{$-\!\!\!\!-\!$}}}

\def\keywords{\noindent\bf Keywords: \vspace{0pt}
\it\normalsize\normalsize}
\def\endkeywords{\par}

\def\acknowledgement{\vspace{4pt} \noindent{\bf Acknowledgements} \\ \vspace{1pt}
\normalsize\normalsize}
\def\endkeywords{\par}

\newcommand{\JACK}{\texttt{JACK}}
\newcommand{\ESC}{\texttt{ESC/Java}}
\newcommand{\LOOP}{\texttt{LOOP}}


\title{Methodology for Java Application Validation}

\author{L. Burdy, ...}

\date{ }
