\chapter{Introduction}


Smart cards are trusted personal devices whose characteristics are
regulated by the ISO 7816 standard. As other trusted personal devices,
smartcards are designed to store and process confidential data, and
can act as tokens to provide users with a secure electronic
representation in a large network. They are widely deployed and used
in application areas such as mobile telecommunications, banking,
transportation, electronic identity, and digital rights management
(DRM). Further, they hold the promise to play a key role in the
e-society, especially as a means to guarantee users a personalized,
global, and secure access to applications and services.


The prominent role played by trusted personal devices in security
sensitive applications make them an ideal target for
attacks. Traditionally, the main concern with smartcards has been with
hardware attacks in which the attacker gains access to confidential
information or disturbs the functioning of the card through
observation (e.g. of power or electro-magnetic radiations) or invasion
(e.g. overriding sensors or attaching probes). This issue is studied in
Deliverable D8.1.

The trusted personal device remains a specific domain where post
issuance corrections are very expensive due to the deployment process
and the mass production. Furthermore, the emergence of new generation
trusted personal devices increasingly connected to networks and
providing execution support for complex programs and the prospect of
logical attacks has urged the trusted personal devices industry to
improve the quality of their software, as logical attacks are
potentially easier to launch than physical attacks (for example they
do not require physical access to the device, and are easier to
replicate from one device to the other), and may have a huge impact.
In particular, a malicious attacker spreading over the network and
disconnecting or disrupting devices massively could have deep consequences.

This deliverable reports on the development of methodologies and tools
that increase confidence in applications.  For concreteness, we focus 
on Java applications that can be executed on devices that embed Java
Virtual Machines (JVM) or their variants, in particular Java Card
Virtual Machines (JCVM). Java enabled devices are a natural choice for
formal methods because:
\begin{inparaenum}[i)]
\item they are widely deployed in the field;
\item they feature mechanisms that contribute to the security of the
platform and the applications that execute over it;
\item detailed informal specifications of the Java platform are publicly
available, and can be scrutinized.
\end{inparaenum}
However, it should be clear that the methods presented in this document
are relevant to other execution platforms for trusted personal devices.


\section{Security issues}
While the focus of the deliverable is on application validation,
security is a holistic property of a system, and formal techniques
must therefore be employed at different levels to provide strong
guarantees about the security of a TPD and its applications.
Essentially, the levels are: the hardware, platform, the libraries,
and the applications.

The need to consider security at those levels is illustrated for
example by the case study described on Page 55 in Deliverable D8.2,
which is concerned with secure platforms. The development of secure
API is discussed in Deliverables D7.1 and D7.2. As previously mentioned,
hardware security is discussed in Deliverable D8.1.


\paragraph*{Platform} The TPD security architecture guarantees that
downloaded applications are innocuous and comply with some basic
policies related to typing, initialization or access control. Such
basic policies are the cornerstones upon which the overall security of
the smartcard will rely. Therefore it is important to verify that the
security architecture does enforce these basic policies as
intended. Thus, an important application of formal methods to TPD
security is platform verification, which aims at providing an abstract
model of the Java platform and security architecture, and at proving
that the security functions play their expected role.

\paragraph*{Libraries} However, it is not sufficient to show that
security functions are correctly designed. In particular, one also has
to ensure that other components of the infrastructure, in particular
API, are correctly designed and implemented. For the purpose of this
deliverable, where the focus is on Java based TPD, the Java API and
the Global Platform API constitute two prominent components of the
infrastructure whose correct design is central to security. 


\paragraph*{Applications} 
Platform and libraries verification is a fundamental step towards
guaranteeing the security of smartcards, and a prerequisite for Common
Criteria evaluations at the highest levels. Nevertheless, the
guarantees offered by the Java security architecture are limited, and
further verifications must be performed to verify that applications
make a legitimate use of the infrastructure, and do not attempt any
hostile action.

Thus, application validation is another important application of
formal methods to TPD security. To date, testing campaigns remain the
primary means to ensure the quality of applications. However, testing
campaigns are expensive and only provide partial guarantees with
regard to the reliability of software. Therefore, it is important to
develop other advanced techniques for applet validation.

There are many facets to applet validation, each with its own
objectives and techniques:
\begin{itemize}
\item one can enhance existing security architectures to enforce
security properties not addressed by current architectures, in
particular confidentiality and availability.  Verification can be
performed by enhanced bytecode verification mechanisms;


\item one can abandon the realm of type systems and its associated
benefits and choose develop logical methods for specifying and
verifying either automatically or efficiently a specific class of
security properties. Verification can be performed by (possibly
efficient and hence incomplete) logic-based proof inference
mechanisms;




\item one can exploit the expressive power of logical methods to
require that applications, or at least sensitive fragments of
applications, are subjected to functional verification, i.e. to
verifications that establish their correctness in terms of
functionality as well as security.
\end{itemize}



\section{Logical verification of security properties using JML}
In order to provide precise analyzes with a limited overhead, we
advocate an integrated approach where validation techniques of
increasing strength are used, starting from automated techniques such
as testing and moving towards formal validation using a combination of
automated and interactive tools. In addition, we aim at overcoming the
difficulty of introducing formal techniques in industrial processes by
providing notations and tools hiding the mathematical formalisms and
by integrating formal techniques into classical developers environment
so as to allow users to benefit from formal techniques without having
to learn new formalisms and to become experts.

All the tools and results presented in this document were developed
with this goal in mind, notably the choice of JML as assertion
language and the development of JACK and its associated feature.
Using those techniques, Java developers should be able to validate
their code, or at least to get a good assurance on its correctness.


\subsection{JML}
JML~\cite{Leavens-Baker-Ruby99b,Leavens-Baker-Ruby03}, the ``Java
Modeling Language'', is a behavioral interface specification language
for Java; that is, it specifies both the behavior and the syntactic
interface of Java code.  The syntactic interface of a Java class or
interface consists of its method signatures, the names and types of
its fields, etc.  This is what is commonly meant by an application
programming interface (API).  The behavior of such an API can be
precisely documented in JML annotations; these describe the intended
way that programmers should use the API.  In terms of behavior, JML
can detail, for example, the preconditions and postconditions for
methods as well as class invariants. These specifications are given as
annotations of the Java source file. More precisely, they are included
as special Java comments, either after the symbols \lstinline!//@! or
enclosed between \lstinline!/*@! and
\lstinline[basicstyle=\normalfont\ttfamily\small\sl]!@*/!. For example,
the general schema for the annotation of a method is the following:
\begin{lstlisting}
/*@ behavior
  @   requires <precondition>;
  @   ensures <postcondition if no exception raised>;
  @   signals(E) <postcondition when exception E raised>;
  @   assignable <modified fields and variables>;
  @*/
\end{lstlisting}
where \lstinline!requires! specifies the conditions on variables, fields
and method parameters at the beginning of the method call so that the
conditions after \lstinline!ensures! hold at the end of the method
call and the conditions after \lstinline!signals(E)! hold if an
exception is raised and not caught inside the analyzed method.  The
underlying model is a an extension of Hoare-Floyd logic: if the
precondition holds at the beginning of the method call, then
postconditions (with and without exceptions) will hold after the
call. The \lstinline!assignable! clause specifies side-effect affected
variables and is used during the weakest precondition calculus for
method invocations.

An important goal for the design of JML is that it should be easily
understandable by Java programmers. This is achieved by staying as
close as possible to Java syntax and semantics.  Another important
design goal is that JML {\em not} impose any particular design method
on users; instead, JML should be able to document Java programs
designed in any manner \cite{Leavens-Baker-Ruby03}.

JML uses Java's expression syntax in assertions,
thus JML's notation is easy for programmers to learn.  
Because JML supports quantifiers such as
\verb_\forall_ and \verb_\exists_, and because JML allows ``model''
(i.e., specification-only) fields and methods, specifications can
easily be made precise and complete.
JML assertions are written as special
annotation comments in Java code,
so that they are ignored by Java compilers but can be used
by tools that support JML\@.  Within annotation comments JML extends the
Java syntax with several keywords.  It also extends Java's expression syntax with several
operators.
The central ingredients of a JML specification are preconditions
(given in {\tt requires} clauses), postconditions (given in {\tt
  ensures} clauses), and (class and interface) invariants.  These are
all expressed as boolean expressions in JML's extension to Java's
expression syntax.
In addition to ``normal'' postconditions, the language also supports
``exceptional'' postconditions, specified in {\tt signals} clauses.
These can be used to specify what must be true when a method throws an
exception. 

\paragraph*{Styles of specification}
Due to its expressiveness and versatility, the JML specification
language supports several styles of specifications; the choice of one
style of specification over the others depends on the purpose of the
verification effort. In a nutshell, one can either opt for lightweight
specifications in which one introduces enough annotations to reason
about some specific safety property, such as the absence of
exceptions, or heavyweight specifications where functional behavior is
considered. There is of course a great liberty in how \lq\lq
lightweight\rq\rq\ or \lq\lq heavyweight\rq\rq\ a specification should
be, and different styles can be used in different parts of an
application.

In addition, one may opt for defensive specifications, in which methods
are annotated with preconditions that prevent exceptions to occur, or
offensive specifications, which use appropriate clauses to specify 
exceptional postconditions.

\subsection{Verification techniques and tools}

JML specifications correctness can be verified either during runtime
or statically~\cite{BurdyCCEKLLP03}. To be verified during runtime, the
source code must have been compiled using \texttt{jmlc}, which is a
enhanced Java compiler for JML annotated code. This compiler adds to
the generated program assertions checking instructions corresponding
to the JML specifications of the program: preconditions, postconditions 
and loop or class invariants. An exception is raised during the execution 
if a JML condition fails. The JML runtime assertion checker can be used
for unit testing~\cite{CL02:ecoop}.


For the static verification of Java programs, several tools are
available using (variations of) JML as specification language. These
tools adopt different compromises between soundness and automation,
and thus it is useful to use them in combination, starting from
automatic but unsound tools, and pursuing with sound but interactive
tools.  Among these tools, ESC/Java2~\cite{CK04:cassis} offers the higher
level of automation as it does not require any user interaction and
relies on the Simplify automatic prover. It is particularly useful for
checking null pointers or array bounds limits; however it is unsound
and incomplete.

In order to further increase the level of reliability of applications,
we propose a methodology based on static verification using
JACK~\cite{BRL03:fme}, a tool that generates proof obligations that
can be discharged using proof assistants or automatic provers.


\section{Main contributions}
The work reported in this deliverable builds upon the JACK tool,
that was initially developed within Gemplus. The tool development
was transferred to INRIA at the beginning of the project, with
the objective to improve and increase its functionalities so as
to address the needs of INSPIRED. The main contributions of the
work carried within INSPIRED are:
\begin{itemize}
\item support for verifying high-level security properties
\item support for verifying bytecode programs
\item validation of the methodology for estimating resource
usage and optimizing code for low-footprint.
\end{itemize}
In order to carry the above tasks and evaluations, it has also
been necessary to make many improvements to the tool itself.
It is not in the scope of the present document to describe these 
improvements.


\section{Contents of the deliverable}
This document is organized as follows, the next chapter introduce the
assertion language JML, chapter 2 describes the JACK tool with its
extension feature, chapter 3 presents some evaluations done with the
tools and the last chapter concludes.
