%=====================================================================
\section{Tools for JML}
\label{tools}

For a specification language, just as for a programming language, a
range of tools is necessary to address the various needs of the
specification language's users such as reading, writing, and checking
JML annotations.\footnote{Most of these tools rely upon an underlying
  tool that type checks a Java program and its JML specification.}


%% Isn't some transition text necessary here, as we jump right into the
%% detailed discussion of each of these tools without a prelude?
%% -- Joe
%% [[[How's this?  -- Gary]]]

The rest of this paper describes the various tools that are currently
available for JML.  The following categorization serves also as an
organization for the immediately following sections of this paper.

\smallskip

\begin{itemize}
\item {\bf Runtime assertion checking and testing:}
\begin{itemize}
\item One way of checking the correctness of JML specifications is by
  runtime assertion checking, i.e., simply running the Java code and
  testing for violations of JML assertions.  Such runtime assertion
  checks are accomplished by using the JML compiler {\tt jmlc}
  (Section~\ref{jmlc}).
\item Given that one often wants to do runtime assertion checking in
  the testing phase, there is also a {\tt jmlunit} tool
  (Section~\ref{jmlunit}) which combines runtime assertion checking
  with unit testing.
\end{itemize}
\item {\bf Static checking and verification:} More ambitious than
  testing if the code satisfies the specifications at runtime, is
  verifying that the code satisfies its specification statically.
  Verifying a specification gives more assurance in the correctness of
  code as it establishes the correctness for all possible execution paths,
  whereas runtime assertion checking is limited by the execution paths
  exercised by the test suite being used.  Of course, correctness of a
  program with respect to a given specification is not decidable in
  general.  A verification tool must trade off the level of automation
  it offers (i.e., the amount of user interaction it requires) and the
  complexity of the properties and code that it can handle.  There are
  several tools for statically checking or verifying JML assertions
  providing different levels of automation and supporting different
  levels of expressivity in specifications.
\begin{itemize}
\item The program checker ESC/Java  can
  automatically detect certain common errors in Java code and check
  relatively simple assertions.
\item ESC/Java2 extends ESC/Java
  to support more of the JML syntax and to add other functionality.
\item The CHASE tool checks some aspects of
  frame conditions.
\item The LOOP tool translates JML-annotated code
  to proof obligations that one can then try to prove using the
  theorem prover PVS\@.  The LOOP tool can handle more complex
  specifications and code than automatic checkers like ESC/Java can,
  but at the price of more user interaction.
\item Krakatoa~\cite{krakatoa03a} is similar to LOOP, but integrates
  with the Coq theorem prover.  It is currently under development and
  handles a subset of Java.
\item The program checker JACK offers similar
  functionality to ESC/Java, but is more ambitious in attempting real
  program verification.
\end{itemize}

\item {\bf Generating specifications: } In addition to
  these tools for checking specifications, there are also tools that
  help a developer write JML specifications, with the aim of reducing
  the cost of producing JML specifications.
\begin{itemize}
\item The Daikon tool (Section~\ref{daikon})
  infers likely invariants by observing the runtime behavior
  of a program.
\item The Houdini tool (Section~\ref{houdini}) postulates annotations
  for code, then uses ESC/Java to check them.
\item The {\tt jmlspec} tool can produce a skeleton of a
  specification file from Java source and can compare the interfaces of
  two different files for consistency.
\end{itemize}

\item {\bf Documentation}: Finally, in spite of all the tools
  mentioned above, ultimately human beings must 
  read and understand JML specifications.  The {\tt jmldoc} tool
  (Section~\ref{jmldoc}) helps the reading process by producing
  browsable HTML from JML 
  specifications, in the style of Javadoc.  The resulting HTML is a
  convenient format for reading large bodies of JML specifications.

\end{itemize}

\subsection{Documentation Generation}
\label{jmldoc}

%% Why is this a separate subsection here, if it was just discussed
%% above?  I suggest moving (some of) this discussion up into the item
%% above. -- Joe

Since JML specifications are meant to be read and written by ordinary
Java programmers, it is important to support the conventional ways
that these programmers create and use documentation.  Consequently,
the {\tt jmldoc} tool (authored by David Cok)
produces browsable HTML pages containing both the
API and the specifications for Java code, in the style of pages
generated by Javadoc~\cite{Friendly95}.

This tool reuses the parsing and checking performed by the JML checker
and connects it to the doclet API underlying Javadoc.  In this way,
\texttt{jmldoc} remains consistent with the definition of JML and
creates HTML pages that are very familiar to a user of Javadoc.  The
{\tt jmldoc} tool combines and displays in one place all of the
specifications that pertain to a given class, interface, method, or
field; it combines annotations across a series of source files that
constitute successive refinements of a given class or interface, as
well as the relevant specifications of overridden methods.

%=====================================================================
\subsection{Runtime Assertion Checking and Testing}

\subsubsection{Runtime Assertion Checking}
\label{jmlc}

The JML compiler (\texttt{jmlc}), developed at Iowa State University,
is an extension to a Java compiler and compiles Java programs
annotated with JML specifications into Java
bytecode~\cite{Cheon03,Cheon-Leavens02b}.  The compiled bytecode includes
runtime assertion checking instructions that check JML specifications
such as preconditions, normal and exceptional postconditions,
invariants, and history constraints.  The execution of such assertion
checks is transparent in that, unless an assertion is violated, and
except for performance measures (time and space), the behavior of the
original program is unchanged.  The transparency of runtime assertion
checking is guaranteed, as JML assertions are not allowed to have any
side-effects~\cite{Leavens-etal03a}.

The JML language provides a rich set of specification facilities to
write abstract, complete behavioral specifications of Java program
modules~\cite{Leavens-etal03a}.  It opens a new possibility in runtime
assertion checking by supporting abstract specifications written in
terms of specification-only declarations such as model fields, ghost
fields, and model methods.  Thus the JML compiler represents a
significant advance over the state of the art in runtime assertion
checking as represented by Design by Contract tools such as
Eiffel~\cite{Meyer97}, or by Java tools such as
iContract~\cite{Kramer98} or Jass~\cite{Bartetzko-etal01}.  The
\texttt{jmlc} tool also supports advances such as (stateful) interface
specifications, multiple inheritance of specifications, various forms
of quantifiers and set comprehension notation, support for strong and
weak behavioral subtyping~\cite{Liskov-Wing94,Dhara-Leavens96}, and a
contextual interpretation of undefinedness~\cite{Leavens-etal03a}.

In sum, the JML compiler brings ``programming benefits'' to formal
interface specifications by allowing Java programmers to use JML
specifications as practical and effective tools for debugging,
testing, and design by contract.

\subsubsection{Unit Testing}
\label{jmlunit}

A formal specification can be viewed as a test
oracle~\cite{Peters-Parnas98,Antoy-Hamlet00}, and a runtime assertion
checker can be used as the decision procedure for the test
oracle~\cite{Cheon-Leavens02}.  This idea has been implemented as a
unit testing tool for Java (\texttt{jmlunit}), by combining JML with
the popular unit testing tool JUnit~\cite{Beck-Gamma98}.  The
\texttt{jmlunit} tool, developed at Iowa State University, frees the
programmer from writing most unit test code and significantly
automates unit testing of Java classes and interfaces.

The tool generates JUnit test classes that rely on the JML runtime
assertion checker.  The test classes send messages to objects of the
Java classes under test.  The testing code catches assertion violation
errors from such method calls to decide if the test data violate the
precondition of the method under test; such assertion violation errors
do not constitute test failures.  When the method under test satisfies
its precondition, but otherwise has an assertion violation, then the
implementation failed to meet its specification, and hence the test
data detects a failure~\cite{Cheon-Leavens02}.  In other words, the
generated test code serves as a test oracle whose behavior is derived
from the specified behavior of the class being tested.  The user is
still responsible for generating test data; however the test classes
make it easy for the user to supply such test data. In addition, the
user can supply hand-written JUnit test methods if desired.

Our experience shows that the tool allows us to perform unit testing
with minimal coding effort and detects many kinds of errors.
Ironically, about half of our test failures were caused by
specification errors, which shows that the approach is also useful for
debugging specifications.  In addition, the tool can report assertion coverage
information, identifying assertions that are always true or always
false, and thus indicating deficiencies in the set of test cases.
However, the approach requires specifications to be fairly complete
descriptions of the desired behavior, as the quality of the generated
test oracles depends on the quality of the specifications. Thus, the
approach trades the effort one might spend in writing test cases for
effort spent in writing formal specifications.

%\section{Static Techniques for JML}
%\label{static}
%
%More ambitious than {\em testing} if the code satisfies the specifications
%at runtime, by doing runtime assertion checking, is
%{\em verifying} that the code
%satisfies its specifications statically.
%This can give more assurance in the correctness of code:
%verification establishes the correctness for all possible inputs,
%unlike runtime assertion checking, which is always limited by the test
%suite being used.
%
%Correctness of a program with respect to
%a given specification is not decidable in general.
%A verification tool must trade off the level of automation the tool
%offers (i.e., the amount of user interaction it requires)
%and the expressiveness of the properties and code that it can handle.
%There are several tools for verification of JML assertions,
%that provide different levels of automation
%and support different levels of expressivity in specifications.

%=====================================================================
\subsection{Generating Specifications} 

\subsubsection{Invariant Detection with Daikon}
\label{daikon}

Most tools that work with JML assume the existence of a JML
specification, then verify code against the specification.  Writing
the JML specification is left to a programmer.  Because this task can
be time-consuming, tedious, and error-prone, the Daikon invariant
detector~\cite{Ernst2000:PhD,ErnstCGN2001:TSE} provides assistance in
creating a specification.  Daikon outputs observed program properties
in JML syntax (as well as other output formats) and automatically inserts
them into a target program.

The Daikon tool dynamically detects likely program invariants.  In
other words, given program executions, it reports properties that were
true over those executions.  The set of reported properties is also
known as an ``operational abstraction.''  Invariant detection operates
by observing values that a program computes at runtime, generalizing
over those values, and reporting the resulting properties.  The
properties reported by Daikon encompass numbers ($x \le y$, $y=ax+b$),
collections ($x \in \mathit{mytree}$, $\mathit{mylist}$ {\it is
  sorted\/}), pointers ($\mathit{n} = \mathit{n.child.parent}$),
and conditionals ($\mathit{if\,} p \neq \mathit{null\; then\; p.value}
> x$); a complete list appears in the Daikon user manual.  Daikon is
available at {\small\url{http://pag.lcs.mit.edu/daikon/}}.  Several
other implementations of dynamic invariant detection
exist~\cite{HangalL02,PytlikRKR2003,HenkelD2003}, but they do not
presently produce output in JML format.

Like any dynamic analysis, the accuracy of the inferred invariants
depends in part on the quality and completeness of the test cases, and
other executions may falsify some of the reported properties.
(Furthermore, the actual behavior of the program is not necessarily
the same as its intended behavior.)  However, Daikon uses static
analysis, statistical tests, and other mechanisms to reduce the number
of false positives~\cite{ErnstCGN2000:Relevance}.  Even if a property
is not true in general, Daikon's output provides valuable information
about the test suite over which the program was run.  Combining
invariant detection with a static verifier such as ESC/Java helps to
overcome the problems of both techniques: the unsoundness of the
dynamic analysis and the static analysis's need for annotations.

Even with modest test suites, Daikon's output is remarkably accurate.
In one set of experiments~\cite{NimmerE02:ISSTA}, over 90\% of the
properties that it reported were verifiable by ESC/Java (the other
properties were true, but were beyond the capabilities of ESC/Java),
and it reported over 90\% of the properties that ESC/Java needed in
order to complete its verification.  For example, if Daikon generated
100 properties, users had only to delete less than 10 properties and
to add another 10 properties in order to have a verifiable set of
properties.  In another experiment~\cite{NimmerE02:FSE}, users who
were provided with Daikon output (even from unrealistically bad test
suites) performed statistically significantly better on a program
verification task than did users who did not have such assistance.

In addition to aiding the task of static checking as described above,
operational abstractions generated by the Daikon invariant detector
have been used to generate and improve test
suites~\cite{HarderME03,XieN2003:ASE,GuptaH2003}, to automate
theorem-proving~\cite{NeWinE02:TR841,NeWinEGKL04:STTT}, to identify
refactoring opportunities~\cite{KataokaEGN01}, to aid program
analysis~\cite{DodooDLE02,DodooLE2003:TR}, and to to detect anomalies
and bugs~\cite{RazKS2002,GroceV2003,Brun2003}, among other uses.

\subsubsection{Inferring annotations with Houdini}
\label{houdini}

An obstacle to using program verification tools such as ESC/Java on
legacy code is the lack of annotations in such a program.  The
warnings
% that ESC/Java then produces
more likely point out missing annotations than errors in the code.
The Houdini tool~\cite{FlanaganL01,Houdini-theory:IPL} attempts
to alleviate this problem by supplying many of the missing
annotations.

Houdini works by making up ``candidate annotations'' for the given
program.  These candidate annotations compare fields and array lengths
to -1, 0, 1, constants used in array constructors, \texttt{null},
\texttt{true}, and \texttt{false} (depending on the type of the
field), and indicate that arrays and subarrays contain no null
elements.  To find which of the candidate annotations hold for the
program, Houdini repeatedly invokes ESC/Java, removing those candidate
annotations that ESC/Java finds to be inconsistent with the code.
When all remaining candidate annotations are consistent with the code,
Houdini invokes ESC/Java a final time to produce warnings that are
then presented to the user.  Houdini thus retains the precision of
ESC/Java, trading quick turnaround for a reduced annotation effort.

Note that any user-supplied JML annotations in the program still get
used by Houdini, since they become part of each invocation of
ESC/Java.  Thus, the benefits of using JML annotations are the same
for Houdini as for ESC/Java, but Houdini can find program errors from
a smaller set of user-supplied JML annotations.

