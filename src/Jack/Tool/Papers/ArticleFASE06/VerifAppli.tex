\section{Conformity of the Environment w.r.t. $PH$}
\label{sec-verif}


In this section, we propose the annotations allowing to  
verify that the set $\Sigma_C$ of executions of an environment satisfy $PH$.

\begin{normalsize}
%\marginpar{bien introduire la figure}
%In the case of liveness properties, the generation
%of annotations on the class $Transaction$ is not sufficient to ensures 
%the verification. We also need ($MPH$) on the
%environment of the class. Therefore, as explained in Section~\ref{SecIntro}, 
%our method to verify
%liveness properties consists in doing two tasks.  The first one requires
%showing that the class $Transaction$ establishes the 
%liveness property by
%satisfying the appropriate JML annotations, as discussed in
%Sect.~\ref{sec-until}, assuming ($MPH$) on the environment.

%deals with the second task, consisting in showing that the
%class runs in an appropriate environment, \emph{i.e.,} an environment
%that will actually enable it to establish the liveness property, by
%calling the appropriate methods often enough. 
%that respects Hypothesis~($MPH$). 
  For example, consider the class $Transaction$ whose surrounding
  environment has to respect a hypothesis $PH$ (see
  Section~\ref{sec-until}). Instances of $Transaction$ are used by a class
  $Session$ through a method entitled \texttt{performTransaction()} (see
  Fig.\ref{fig-perform}).  The executions of this method constitute
  $\Sigma_{Transaction}$. We need to ensure that it respects
  the hypothesis $PH$.




%Suppose %for example 
%that \texttt{performTransaction()} has the following body:
%%the class in Fig.~\ref{FigExample} that this method is the following.
\end{normalsize}

\begin{figure}
{\scriptsize
\begin{multicols}{2}
\begin{verbatim}
public boolean performTransaction(TransactionSystem a)
  {while (true)   
    {switch(a.state)
       { case 1:a.initialiseTransaction();                  
                break;            
         case 2:a.sendToBank();break;
         case 3:try{a.pinValidation();}
                catch(Exception e){}
                break;
         case 4:a.complete();break;                   
         case 5:a.printReceipt();break;                                      
         case 6: return true;
         case 7: return false;
       }
   }
  }
\end{verbatim}
\end{multicols}
}
\caption{The \texttt{performTransaction()} method}\label{fig-perform}
\end{figure}

%{\small
%\begin{verbatim}
%public void performTransaction{
%    initialiseTransaction();}
%\end{verbatim}
%}
Section~\ref{sec-preservation} 
presents conditions implying that $\Sigma_C \models PH$. %the condition of preservation and the correctness of the 
%method; 
Then Section~\ref{sec-checking} explains how the verification
of these conditions are implemented in our tool and presents
the experimental results. 
%
\subsection{Clauses $\Sigma_C \models PH$}
\label{sec-preservation}

%We need a simulation relation $R$ between the executions 
%of the methods of the class $Transaction$ agreeing with ($MPH$)
%and the executions of the methods of the class $Transaction$ in
%the application ($Session$).
%This relation $R$ needs to preserve safety as well as liveness properties.
%It is well-known that simulation preserves safety properties~\cite{ClarkeGP99}.
%For liveness properties, $R$ has to be divergence sensitive, i.e., 
%no new deadlock and no new livelock in the concrete system~\cite{glabbeek93}
%are allowed. 
%We introduce in this paper a simulation notion very closed to this
%one, but which permits deadlock or livelock introduction \textit{under some
%conditions}.
%For this verification, we actually stay in the case where
%the environment is allowed to modify the internal state of the 
%class only through invocation of the methods of the class - thus
%we disallow direct assignment of attributes or internal
%modifications through aliasing. 

In order to verify that $\Sigma_C \models PH$, we define the following clauses. 
\begin{itemize}
%\item The environment $S$ calls the methods of $T$ within
%their preconditions. ($C_{\tau}$)
\item[$C_{dead}$] The environment executions $\Sigma_{C}$ cannot terminate, either
normally nor by throwing an exception when $Q \wedge \neg R$
holds on the current state. 
\item[$C_{live}$] The environment executions $\Sigma_{C}$ cannot diverge, i.e.,
not performing any progress - by
infinite loop, infinite recursive call or call to a method
within diverging conditions when $Q \wedge \neg R$
holds on the current state.
\end{itemize}

\begin{lemma}[Condition of $PH$ Satisfaction]
\label{prop-preservation}
Let $PH$ be the hypothesis of a class $C$ environment for  $Q \leadsto R (J,V,M)$. 
Assuming that the environment of $C$ calls the methods in
$\mathcal{PM}_C$ within their preconditions, %Second $C$ under $MPH$ satisfy 
%the property $Q \leadsto_{JML} R (J,V,M)$. Then the environment $S$ 
%using $C$ satisfy also $Q \leadsto_{JML} R (J,V,M)$ iff the
%following conditions are satisfied:
$\Sigma_C \models PH$ iff $\Sigma_{C} \models C_{live}$ and $\Sigma_{C} \models C_{dead}$.
\end{lemma}
The sketch of the proof 
can be found in Appendix~\ref{sec-proof-proposition-preservation}. 

%\begin{proposition}
%\label{prop-preservation-2}
%If $\sigma$ is a $\tau$-simulation of $\sigma'$, and both
%$\sigma$ and $\sigma'$ verify  $Q \leadsto_{JML} R (J,V,M)$
%then $\sigma'$ verify $C_{dead}$ and $C_{live}$
%\end{proposition}




%\begin{enumerate}
%\item The environment execution cannot terminates - normaly
%or abruptly when
%$Q \wedge \neg R$ holds.
%\item The environment execution cannot diverge - by a infinite
%loop not containing a progress method or a call to a method within 
%divergence condition $Q \wedge \neg R$ holds.
%\item The environment cannot call a method out of its precondition.
%\end{enumerate}
%\end{proposition}
%\marginpar{proof to be rewritted: it is only some notes}
%\begin{proof}
%We show that then these conditions are satisfied, the relation
%between the executions of the environment and the executions
%of the class under $MPH$ is a divergence sensitive $\tau$ simulation.
%Let $\sigma$ be an execution of the method of the class under $MPH$,
%and let $\sigma'$ be an execution of the environment using the class.
%The proof is done by induction on the length of the execution.
%\begin{itemize}
%\item The initials states are in relation.
%\item Suppose that the environment is in a state $s'_j$ in relation
%with a state $s_i$. Then:
%   \begin{itemize}
%   \item a progress method is called, then it was possible to call
%the same method in the execution with $MPH$ because the precondition
%of the method holds + divergence condition
%   \item the execution environment terminates. Then  $\sigma_i$ is
%$ s_i xrightarrow{skip} s_{i+1} \dot$. Where all states are the same.
%Because  $Q \wedge \neg R$ does not hold, this execution satisfy $MPH$.
%   \item a non diverging block of code without call to a progress method ->
%finite number of $\tau$. diverging block with call to progress method ->
%regularly call -> verify $MPH$
%diverging bloc without call to progress method. -> only when 
%$\neg$ $Q \wedge \neg R$ -> this divergence already exists in the abstract 
%level. 
%   \end{itemize}
%
%\end{itemize}
%
%\end{proof}

% We are able to verify by proof that
% the environment verify  simulation conditions 
% given in Proposition~\ref{prop-preservation} using, like
% for the verification of the Task 1, 
% the standard JML framework. The 
% method is explained in the next section.




\subsection{Checking the Clauses $C_{live}$ and $C_{dead}$} 
\label{sec-checking}



%The first condition, $C_{\tau}$, implying that each
%method must be called within its precondition, can be checked using 
%proof tools for JML~\cite{BurdyRL03,marche03jlap}.
%This verification is performed in two step. %First, we 
%perform a decomposition of the environment into atomic 
%methods; Second we verify $C_{dead}$ on this decomposition;
%Third we verify  $C_{live}$ on the decomposition.
%First we verify $C_{dead}$ using standard JML annotations; 
%Second, we perform a decomposition of the environment into
%atomic methods and we  verify  $C_{live}$ on the decomposition.

We present JML annotations to verify $C_{dead}$ and $C_{live}$.

%\begin{definition}[JML Annotation for $C_{dead}$]
We call $\mathcal{A}_{dead}$ the annotations for a method in the environment which invokes methods of the class $C$.\\ 
\begin{tabular}{l}
\hspace{8em}\verb+/*@ behavior+\\
\hspace{8em}...\\
\hspace{8em}\verb+  @ diverges true;+\\
\hspace{8em}\verb+  @ ensures +($\neg Q \vee R$);\\
\hspace{8em}\verb+  @ signals (Exception e) +($\neg Q \vee R$)\verb+;+\\
\hspace{8em}\verb+  @*/+
\end{tabular} \hspace{4em} ($\mathcal{A}_{dead}$)


\begin{proposition}
\label{prop-deadlock}
%Let $C$ be a class where the liveness property 
%$Q \leadsto R (J,V,M)$ is satisfied under $PH$. 
%Let $S$ be a environment using this clas throw a
%method $m$. $C_{dead}$ is satisfied on the environment
%iff $m$ satisfies the folloving JML method specification ($\mathcal{A}_{dead}$
%We verify on the decomposition that there is no deadlock 
%using two invariants.
%\begin{enumerate}
%The first invariant implies that if the execution ends then $ Q \wedge \neg R$ %does
%not hold. The second one implies that when $ Q \wedge \neg R$, the execution
%cannot be blocked.
Let $\Sigma_{C}$ be the set of the executions of a method in the
environment which invokes methods of the class $C$. 

$\Sigma_{C} \models C_{dead}$ iff
$\Sigma_C : \mathcal{A}_{dead}$.
\end{proposition}
The proof of Proposition~\ref{prop-deadlock} is immediate by the 
JML semantics. %and can be can be found in 
%Appendix~\ref{sec-proof-proposition-CDead}.

We call $\mathcal{A}_{Slive}$ the following JML annotations for a method in
the environment which invokes methods of the class $C$.\\  
%
\begin{tabular}{l}
\hspace{8em}\verb+/*@ behavior+\\
\hspace{8em}...\\
\hspace{8em}\verb+  @ diverges false;+\\
\hspace{8em}\verb+  @*/+
\end{tabular} \hspace{12em} ($\mathcal{A}_{Slive}$)



\begin{proposition}
\label{prop-livelock}
%Let $C$ be a class where the liveness property 
%$Q \leadsto R (J,V,M)$ is satisfied under $MPH$. 
%Let $S$ be a environment using this clas throw a
%method $m$. $C_{live}$ is satisfied on the environment
%if $m$ satisfy the folloving $\mathcal{A}_{live_S}$ JML method specification:\\
Let $\Sigma_{C}$ be the set of the executions of a method in the
environment which invokes methods of the class $C$. 
If $\Sigma_C : \mathcal{A}_{Slive}$
then $\Sigma_{C} \models C_{live}$.
\end{proposition}
\begin{proof}
  Following the JML semantics, when the predicate of the JML
  diverges clause is set to false the method under consideration
  cannot diverge.  If the method in the environment which invokes methods
  of the class $C$ never diverges, it does not diverge when $\neg Q \wedge
  R$ holds.\qed
\end{proof}

This sufficient condition is, in general, too restrictive since it forbids
a divergence of the environment. The next section gives means to weaken
this too strong requirement. 

\subsection{Weakening of $C_{Slive}$}

In the general case, i.e., when the environment $S$
may diverge, %standard JML annotations 
%do not allow to directly specify
%$C_{live}$ on the environment. 
$C_{live}$ cannot be directly specified by standard JML
annotations. Take the example of Fig.\ref{fig-perform}, we
replace the statement \verb+return false+ by a statement
\verb+a = new Transaction()+. %Then, the method environment
%may not terminate, i.e. may infinitely perform a 
%new \textit{transaction}s. 
We denote this environment \texttt{performTransaction2}. There
exists an execution of \texttt{performTransaction2} that
never terminates - when we always perform a new transaction.


Then, to ensures $C_{live}$, we must verify that
until a state where $\neg Q \vee R$ holds, each Java 
statement must not diverge. That cannot be check by
standard JML annotations.
Therefore, we need to perform a decomposition
of the environment $S$ into a environment $S'$
such that the verification of $C_{live}$ on $S'$ can
be performed using JML Annotations and such that
$$\Sigma_{TS_{S'}} \models C_{live} 
      \Rightarrow  
          (\Sigma_{TS_S} \models C_{live})$$


 %into atomic statement, i.e., statement without
%while loop, on which
%we can perform the verification and $C_{live}$ using standard
%JML annotations.
The decomposition is performed by rewriting 
rules~(see Fig~\ref{fig-rewritting}) whose 
correctness can be established using the Java 
semantics~\cite{TheseMarieke}. These rules must be understood
as follows: the \textit{bottom} method specification $m$ is rewritten 
into the \textit{up} methods specification. The environment $S'$
obtained using these rules is only composed of atomic methods.

The proof of the rewriting rule $R_{while}$
rule is given in Appendix~\ref{sec-while}, the other proofs are given 
in~\cite{reportTACAS}. During the decomposition, we observe call to
progress variables by introducing an 
\textit{ghost} integer \texttt{callStack} representing 
the execution stack. %- in this example, 
%we have no \textit{nested calls} so an integer is sufficient. 
%\marginpar{parageaphe a reecrire}



%%%%%%%%%%%%%%%%%%%%%%%%%%%%%%%%%%%%%%%%%%%%%%%%%%%%%
%%%%%%%%% rewritting rules
%%%%%%%%%%%%%%%%%%%%%%%%%%%%%%%%%%%%%%%%%%%%%%%%%%%%
\begin{figure}
{\scriptsize
%%%%%%%%%%%%%%%%%%%
%%%%% switch %%%%%%
%%%%%%%%%%%%%%%%%%%
\begin{tabular}{ll}
\begin{tabular}{c}
\begin{tabular}{lcl}

\begin{tabular}{l}

/*@ requires $P$ \&\&\\
~~~@ $\mathsf{E_{switch}}$ == $\mathsf{E_{case1}}$;\\
~~~@ ensures $Q$;\\
~~~@*/\\
m()\_switch\_case1()\{\\ 
$\mathsf{body_{case1}}$\\
\}\\
\end{tabular}
&
\quad ...
&
\quad
\begin{tabular}{l}
/*@ requires $P$ \&\&\\
~~~@ $\mathsf{E_{switch}}$ == $\mathsf{E_{casen}}$;\\
~~~@ ensures $Q$;\\
~~~@*/\\
m\_switch\_casen()\{\\ 
$\mathsf{body_{casen}}$\\
\}\\

\end{tabular}
\end{tabular}\\
\hline



\begin{tabular}{l}
/*@ requires $P$;\\
~~@ ensures $Q$;\\
~~@*/\\
 m()\{\\
switch($\mathsf{E_{switch}}$)\{\\
\qquad case $\mathsf{E_{case1}}$: $\mathsf{body_{case1}}$;break;\\
\qquad ...\\
\qquad case $\mathsf{E_{casen}}$: $\mathsf{body_{casen}}$;break;\\
%\qquad default: $\mathsf{body_{default}}$;break;\\
\}\\

\end{tabular} 
\end{tabular}$(R_{switch})$

& \\


%~
%& 
%\begin{tabular}{c}
%\begin{tabular}{l}
%m()\{\\
%//@ callstack = $END$;\\
%return($X$);\\
%\}\\
%\end{tabular}\\
%\hline
%\begin{tabular}{l}
%m()\{\\
%return($X$);\\
%\}\\
%\end{tabular} 
%\end{tabular}($R_{return}$)\\




%%%%%%%%%%%%%%%%%
%%%% While %%%%%%
%%%%%%%%%%%%%%%%%

$\frac{
\begin{tabular}{l}
/*@ requires $\mathsf{expr_{while}}$;\\
~~~@ requires $P$;\\
~~~@ ensures $P$;\\
~~~@*/\\
m\_while()\{\\
\quad $\mathsf{body_{while}}$;\\
\}\\
\end{tabular}\qquad
\begin{tabular}{l}
/*@ requires $\neg \mathsf{expr_{while}}$;\\
~~~@ requires $P$;\\
~~~@ ensures $Q$;\\
~~~@*/\\
m\_end\_while()\{\\
\quad $\mathsf{body_{endwhile}}$;\\
\}\\
\end{tabular}
}{\begin{tabular}{l}
/*@ requires $P$;\\
~~~@ ensures $Q$;\\
~~~@*/\\
m()\{\\
\quad while ($\mathsf{expr_{while}}$)\{\\
\qquad$\mathsf{body_{while}}$;\\
\quad\}\\
\quad$\mathsf{body_{endwhile}}$\\
\quad\}\\
\end{tabular}
}(R_{while})$ 


&

%\begin{tabular}{c}
%\begin{tabular}{l}
%m()\{\\
%//@ callstack = $method$;\\
%method();\\
%\}\\
%\end{tabular}\\
%\hline
%\begin{tabular}{l}
%m()\{\\
%method();\\
%\}\\
%\end{tabular} 
%\end{tabular}($R_{call}$)










\end{tabular}\\
}


\label{fig-rewritting}
\caption{Rewriting rules used to decompose \texttt{performTransaction2}}
\end{figure}


%%%%%%%%%%%%%%%%%%%%%%%%%%%%%%%%%%%%%%%%%%%%%%%%%%%%%%%
%%%%%%%%%%%%%%%%%%%%%%%%%%%%%%%%%%%%%%%%%%%%%%%%%%%%%%

%We can notice on this figure that we introduce
%an integer \texttt{callStack} representing the execution stack - 
%in this example, we have no \textit{nested calls}, so an integer
%ùis sufficient. Notice also that  \texttt{initialize\-Transaction()}
%has now a precondition \texttt{callStack \-== CALL\_INIT\-IA\-LIZE\_\-TRAN\-SAC\-TION},



%The decomposition is performed using rewriting 
%rules~(see Appendix~\ref{appendix-rewriting}), whose 
%correctness can be established using the Java 
%semantics~\cite{TheseMarieke}. Using JML
%ghost variables, we modelize the execution stack and the exception
%mechanism. Figure~\ref{fig-decomposition} give an extract of the decomposition
%of the environment. We can notice on this figure that we introduce
%an integer \texttt{callStack} representing the execution stack - 
%in this example, we have no \textit{nested calls}, so an integer
%is sufficient. Notice also that  \texttt{initialize\-Transaction()}
%has now a precondition \texttt{callStack \-== CALL\_INIT\-IA\-LIZE\_\-TRAN\-SAC\-TION},
%meaning that the method can only be executed when called by the environment.
%When the execution ends normally, the ghost variable \texttt{callStack} is
%set to  \texttt{END\_EXECUTION}.

An extract of the decomposition of \texttt{performTransaction2()}
is given in~\ref{fig-decomposition}.

\begin{figure}
{\scriptsize
\begin{multicols}{2}
\begin{verbatim}
...
 //@ ghost int callStack = 0;
private  int WHILE_STATEMENT = 0;
private  int CALL_INITIALIZE_TRANSACTION = 1;
...


/*@ requires callStack == 0; 
  @ requires state == 1;
  @ assignable callStack;
  @ ensures callStack 
  @    == CALL_INITIALIZE_TRANSACTION;
  @*/
public void performTransaction_case1()
  {         
  //@ set callStack = CALL_INITIALIZE_TRANSACTION;  
  initialize_Transation();  
  }
...
\end{verbatim}
%/*@ private normal_behavior
%  @ requires state == GETTING_SPECIFICS_STATE;
%  @ requires callStack
%  @   == CALL_INITIALIZE_TRANSACTION; 
%  @ modifies state,callStack;
%  @ ensures  state == SENDING_TO_BANK_STATE 
%  @     ||  state ==ASKING_DO_ANOTHER_STATE;
%  @ ensures callStack == 0; 
%  @*/
%public  void initializeTransaction(){
%  state = SENDING_TO_BANK_STATE;
%  //@ callStack = 0;
%}
%\end{verbatim}
\end{multicols}
}
\label{fig-decomposition}
\caption{Extract from a Decomposition}
\end{figure}

\begin{definition}[Necessary and Sufficient Annotation for $C_{live}$]
Let $S'$ the environment obtained by decomposition of $S$. Let  $M_{S'}$
the set of methods of this environment and  $M_{C}$ the set of all
methods of $C$. We define the JML annotations 
$\mathcal{A}_{live}$ as follows:\\
\begin{tabular}{l}
\verb+//@ invariant+ 0 $< V_{live}$ ;\\
\verb+//@ constraint+ $Q \wedge \neg R$ \verb+=> \old(+$V_{live}$\verb+)+ $<~V_{live}$ \verb+for +$M_{env}$;\\
\verb+//@ invariant + $Q \wedge \neg R$ \verb+=>+ $\bigwedge_{m \in (M_{S'} \cup M_{C})} $\verb+(callstack == +$m$ \verb+ => !+$diverges(m)$\verb+);+
\end{tabular}\\
Where $V_{live}$ is an expression over the variables 
of $S'$ returning an integer
\end{definition}
             
\begin{proposition}
\label{prop-livelock2}
%This verification is performed by checking that the atomic method
%of the environment decrease a well-founded variant $V_{live}$ given by the user. 
%That 
%implies that the environment cannot perform continuously 
%external actions.
%Let $M_{env}$ the set of the methods of the environment
%after decomposition,
%then $C_live$ is
%checked using the following annotations:\\
The executions of the environment $S'$ obtained by decomposition of $S$ 
verify $C_{live}$ iff we can found the \textit{variant} expression $V_{live}$
such that $S'$ is conform to JML annotations $\mathcal{A}_{live}$,i.e.,
\begin{center}
\begin{tabular}{c}
$\exists V_{live} : (\mathcal{P}red_{S'} \rightarrow \mathsf{int}) . (S': \mathcal{A}_{live})$\\
\hline
$\Sigma_{TS_{S'}} \models C_{live} $
\end{tabular}
\end{center}

\end{proposition}



%The proof of this proposition 
%can be found in Appendix~\ref{sec-proof-proposition-CLive}.
%The variables involved in the variant are most of the time
%environment variables or variables, like \texttt{callStack},
%introduced during the decomposition.
%In the case of the example, where there is no while-loop
%nor call to external methods, the variant can be generated
%automatically.


\subsubsection{Experimental Results}

For technical reasons, due to the back-end proof obligation 
generator used (Jack), we produces two files. The first one, 
checked there exists no deadlocks introduced during the integration,
the second one that there is no livelock introduction. 


The table~\ref{table-refinement} gives the result of the proof
of the two output files. The verification is done by the 
Jack tool, using the prover Simplify
\footnote{Results obtained with a Intel Pentium IV 3 ghtz. 500Mo RAM} . The reader can be surprised by
the number of proof obligations generated, but most of them are obvious goal. 
\begin{table}
\label{table-refinement}
\begin{tabular}{|l|l|l|l|}
\hline
Verification & Number of PO & Proved Automatically & Time to perform proof \\
Deadlock freeness & 104 & 100 \% & 3 seconds \\
Livelock freeness & 42 & 100 \% & 1 seconds \\
\hline
\end{tabular}
\caption{Result of the Analyze of the Environment}
\end{table}




%%% Local Variables: 
%%% mode: latex
%%% TeX-master: "main2"
%%% End: 
