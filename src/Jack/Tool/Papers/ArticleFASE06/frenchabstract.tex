


Cet article propose une méthode pour vérifier des propriétés de vivacités dans le cadre d'une extension de JML. Cette vérification est composée de deux sous-tâches: (1) générer les annotations JML appropriées permettant de vérifier que la classe respecte la propriété de vivacité. (2) Montrer que l'environnement préserve la propriété de vivacité par une preuve de raffinement. Pour la génération des annotations JML, la propriété de vivacité doit être complétée avec un variant et un invariant (à l'instar du variant et de l'invariant utilisé dans une preuve de terminaison de boucle). Nous pouvons alors prouver que sous certaines hypothèses sur l'environnement, la propriété de vivacité est satisfaite. La seconde tâche se réduit à prouver que l'environnement respecte les hypothèses énoncées. Nous illustrons notre méthode à l'aide d'un exemple. 