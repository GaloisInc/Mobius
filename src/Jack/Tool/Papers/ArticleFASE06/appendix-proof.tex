\subsection{Proof of Theorem  \ref{until}}
\label{sec-proof-theorem-loop}

\begin{lemma}
\label{lemma-loop}
The JML annotations that ensure the verification of the liveness
primitive~\textsf{Loop} are equivalent to the following 
proof obligations:\:\footnote{where
\(\mathsf{term}(M)\) is the set of events 
\(\{ m~\mathbf{terminates}, m~\mathbf{exceptional}, m~\mathbf{normal} \mid m \in M\}\), 
\(\mathsf{requires}(m)\)  denotes the precondition, and
\(\mathsf{diverges}(m)\) the diverges
clause --the conditions under which the method might
diverge --, of method \(m\), respectively.}.


\begin{footnotesize}
\begin{gather}
\forall i.\:(\sigma_i \models  Q ) \Rightarrow ( \sigma_i \models J ) \\
\forall i.\:\sigma_i \models (V \in \mathbb{N})\\
\forall i\:, n.\:(\sigma_i \models  (Q \wedge J \wedge V = n) \Rightarrow
((\sigma_{i+1} \models \neg  R \wedge \mathsf{term}(M))
%\wedge \sigma_i \models V = n \wedge \sigma_i \models event(M)
\Rightarrow \sigma_{i+1} \models Q))\\
\forall i\:, n.\:(\sigma_i \models  (Q \wedge J \wedge V = n) \Rightarrow
((\sigma_{i+1} \models \neg  R \wedge \mathsf{term}(M))
%\wedge \sigma_i \models V = n \wedge \sigma_i \models event(M)
\Rightarrow \sigma_{i+1} \models V < n))\\
\forall i\:, n.\:(\sigma_i \models  (Q \wedge J \wedge V = n)
\Rightarrow
((\sigma_{i+1} \models \neg  R \wedge \mathsf{term}(\overline{M})) 
%\wedge \sigma_i \models V = n \wedge \sigma_i \models
%event(\overline{M}) 
\Rightarrow \sigma_{i+1} \models V \leq n))\\
\forall i.\:(\sigma_i \models (Q \wedge J) 
%~ \wedge ~ \sigma_i \models J  ~\wedge ~ 
\Rightarrow (\sigma_{i+1} \models \neg  R ~ \Rightarrow  
%(\exists m.\:\sigma_{i+1} \models (\mathsf{requires}(m) \wedge \neg \mathsf{divrges}(m)))))
\end{gather}
\end{footnotesize}
\end{lemma}

This lemma comes from the Java/JML semantics~\cite{TheseMarieke}.
%Following Theorem~\ref{until} and the JML
%semantics~\cite{TheseMarieke}, the following JML annotations are
%sufficient to imply the validity of the
%\emph{Loop} primitive.


\begin{proof}[Theorem  \ref{until}]
The proof is performed using proof obligation given in Lemma~\ref{lemma-loop}.
Because the proof obligation (3) obviously implies that $Q$ always
holds until $R$ holds, we only give the proof of termination, i.e.,
we show that the proof obligations (1--6) implies that:
\[ \forall i. ((i \geq 0
\wedge \sigma_i \models Q ) ~
\Rightarrow ~ (\exists j. j > i \wedge \sigma_j \models R)) \]
The proof is by  contradiction. Suppose that :
\[ \exists i. ((i \geq 0
\wedge \sigma_i \models Q )) ~
\wedge ~ (\forall j. j > i \Rightarrow \sigma_j \not \models R) \]
Proof obligation (3) and (1)  implies that there exists 
an $i \in \mathbb{N}$ such that:
$$  (\forall j > i ~ . ~ \sigma_j \models (Q \wedge J)  \wedge \sigma_{j+1} 
\not \models R) \qquad \qquad (a)$$
Since ($H_S$) holds on $\sigma$, also holds  
on the suffix of $\sigma$ begining at the 
indice $i$: 
  $$\sigma_i \models F^{\infty} M ~ \mathtt{called} $$
Using ($a$) and the proof obligation (6), we deduce
that the method does not diverge, so:
  $$(\sigma_i \models F^{\infty} M ~ \mathtt{terminates} $$
It now implies, using (4),(5) and ($a$) that the variant
$V$ infinitely decreases. Then, the proof obligation (2) 
cannot be established, this is a contradiction. \qed 
\end{proof}


\subsection{Proof of Proposition~\ref{prop-hs}}
\label{sec-proof-proposition-MPH}

\begin{proof}

Let $\sigma$ be an execution of the environment. Suppose
that $H_S'$ holds, then
$$ \exists i . (\forall j \geq i ~ . ~ \sigma_j \models (\neg Q \vee R))$$
Because in each state of the execution after the index $i$, 
the property $Q \Rightarrow R$ holds, the $Loop$ primitive is
obviously verified on the suffix of the execution beginning
at the i$^{th}$ state.
We now show that the property is verified on the 
totality of the execution. 
We first suppose that $\sigma_i \models R$, then, the primitive 
is verified. 
Second, suppose that $\sigma_i \models \neg R$, then, 
$\sigma_i \models \neg Q$. There are two cases.
\begin{enumerate}
\item If $i = 0$, the primitive is
verified. 
\item If $i > 0$ we show that $Q$ cannot
hold at the index $i-1$.
Suppose that $\sigma_{i-1} \models Q$, then,
using the proof obligation($3$), that implies
that $\sigma_{i} \models Q$ - a contradiction
with the hypothesis. Otherwise $\sigma_{i-1} \models \neg Q$,
then the property is still verified on the suffix of the
execution at the index $i-1$.\qed 

\end{enumerate}
 
\end{proof}


\subsection{Proof of $MPH$}
\label{sec-proof-proposition-MPHminimal}

\begin{proof}
We first show that, assuming that the class $C$ 
satisfy the generated JML annotations, all execution satisfying $MPH$ 
satisfy $Q \leadsto R (J,V,M)$. Second we whow that an execution 
satisfying $Q \leadsto R (J,V,M)$ satisfy also $MPH$.
\begin{enumerate}
\item  Assuming that the class $C$ 
satisfy the generated JML annotations, we have shown that $H_S$ 
implies  $Q \leadsto R (J,V,M)$ and that  $H'_S$ implies 
also  $Q \leadsto R (J,V,M)$. So, because $MPH$ is the 
disjunction of $H_S$ and $H'_S$, we conclude that $MPH$
implies $Q \leadsto R (J,V,M)$.
\item Suppose an execution $\sigma$ which 
does not satisfy the ($MPH$). So, $\sigma$ satisfies
its negation:
$$ \mathsf{F^{\infty}}  (Q \wedge \neg R) \wedge (\mathsf{G^{\infty}} \neg (M ~ \mathbf{called}))$$
First, $(\mathsf{G^{\infty}} \neg (M ~ \mathbf{called}))$ means that
there exists an index $k$ such that for all indexes greater than $k$, 
progressive
methods are no more called, i.e., there exists an index $j \geq k$
such that the internal state of the class always stays the same. 

Second, $ \mathsf{F^{\infty}}  (Q \wedge \neg R)$ denotes that there always
exists a state in the future where $(Q \wedge \neg R)$ holds. So, for
all indexes greater than $j$, $(Q \wedge \neg R)$ holds. 
The execution cannot satisfy the modality $Q \leadsto_{JML} R$.
\end{enumerate}
\qed
\end{proof}

\subsection{Proof of Lemma~\ref{prop-preservation}}
\label{sec-proof-proposition-preservation}

\begin{proof}
%We show that, under the hypothesis:
%\item ($H_2$) The action of $S$ do not modify 
%the value of the variables of $C$.
%\end{itemize}


First, we show that $C_{Dead} \wedge C_{Live} \Rightarrow MPH$; 
Second we show that $MPH \Rightarrow  C_{Dead} \wedge C_{Live} $.
\begin{enumerate}
\item Let $\sigma$ be an execution of the environment. 
   First suppose that the execution terminates, either
        normally or throwing an exception, at an index $k$. Then $\sigma_k$
        is:
        $$ s_k \rightarrow s_k \rightarrow s_k ...$$
        Because $C_{Dead}$ holds on the execution, we deduce that 
        $s_k \models (\neg Q \vee R)$. So $\sigma$ satisfies the 
        \textsf{PLTL} property $G^{\infty}(\neg Q \vee R)$. We can 
        conclude that $\sigma$ satisfies $MPH$.

      Second suppose now that the execution does not terminate, i.e., the
        execution is infinite. Then the environment can have two behaviors:
       \begin{itemize}
         \item First, the environment infinitely often call a method in the set
         $M$ of the progress methods of $C$, then 
         $\sigma \models F^{\infty} M~\mathbf{called}$. 
         \item Second, the environment $S$ performs a finite number of call
         to the methods of $C$. That means, that there exists an index $k$, 
         such that $\sigma_k \models G^{\infty} \neg M~\mathbf{called}$, where
         $M$ is the set of progress method of $C$. Following $C_{Live}$, 
         we obtain 
         $$s_k \models (\neg Q \vee R).$$  
         Remembering that environment cannot modify the value of the variables
         of $C$, we can write:
         $$ \forall i \geq k . (\sigma_i  \models (\neg Q \vee R))$$
         The preceding predicate is the definition of 
         $$ \sigma \models G^{\infty}(\neg Q \vee R).$$
         We can conclude that $\sigma$ satisfies $MPH$. 
       \end{itemize} 
 \item We now show that $MPH \Rightarrow  C_{Dead} \wedge C_{Live} $. This
proof is performed by contradiction. We 
show that $\neg C_{Dead} \vee \neg C_{Live} \Rightarrow \neg ($MPH$)$.
\begin{itemize}
\item Suppose that $C_{Dead}$ does not hold on $\sigma$ and suppose that 
$\sigma$ terminates on a state where $(Q \wedge \neg R)$ holds, then $\sigma$
satisfies $$(\mathsf{F}^{\infty}(Q \wedge \neg R)) 
\wedge (\mathsf{G^{\infty}} \neg (M \mathbf{called})$$. 
This is the negation of $MPH$.
\item Suppose that $C_{Live}$ does not hold on $\sigma$. So  
$\sigma$ can diverge on a state $s_k$ where $(Q \wedge \neg R)$ holds.
The divergence of the environment implies that progress methods will no more
be called on $\sigma_k$, it means that:
$$\sigma \models (\mathsf{G^{\infty}} \neg (M \mathbf{called})$$
Moreover, the environment cannot modify the value of the variable of $C$ so
 $ \forall i \geq k . (\sigma_i  \models (Q \wedge \neg R)).$
That implies that  $\sigma \models (\mathsf{G}^{\infty}(Q \wedge \neg R)).$
So, $\sigma \models (\mathsf{F}^{\infty}(Q \wedge \neg R)).$
We conclude that 
 $$\sigma \models (\mathsf{F}^{\infty}(Q \wedge \neg R)) 
\wedge (\mathsf{G^{\infty}} \neg (M \mathbf{called})).$$ 
This is the negation of $MPH$.
\end{itemize} 
\end{enumerate}
%We have prove that $C_{Dead} \wedge C_{Live} \Rightarrow MPH$ and
% $MPH \Rightarrow  C_{Dead} \wedge C_{Live} $. \qed
\end{proof}

%\begin{proof}
%We prove this proposition by first showing that ($C_{\tau}$)
%implies that each execution of $S+T$ is a $\tau$-
%simulation of an execution of a environment using
%$T$. Second we show that $\tau$-
%simulations verifying ($C_{dead}$) and ($C_{live}$) 
%preserve $Q \leadsto_{JML} R (J,V,M)$.
%Let $\sigma$ be an execution of $T$  and
%let $\sigma'$ be an execution of $T$ in $S$.
%\begin{enumerate}
%\item Trivially, we have l($s_0$) $\equiv$ l($s'_0$). Now 
%suppose that until an index $i$ of $\sigma'$ and l($s_i$) $\equiv$ l($s'_j$), we have $\tau$-
%simulation relation between $\sigma$ and $\sigma'$, then.
%\begin{itemize}
%\item The environment stop the execution or diverges then 
%we have a $\tau$-simulation relation between $\sigma$ and $\sigma'$.
%\item The environment performs an external action. This action cannot
%modify the internal state of the class, so l($s_i$) $\equiv$ l($s'_{i+1}$)
%then l($s'_{i+1}$) $\equiv$ l($s'_j$). We have the $\tau$-simulation relation
%until the index $i+1$.
%\item The environment call a method of the class. This invokation leads to
%a state  $s'_{i+1}$. Because this method is call in its precondition, it it
%also possible to call the same method in $s_j$, leading to a state $s_{j+1}$.
%then, because ($s_j$) $\equiv$ $s'_{i}$ and the same method is invoked, then
% $s_{j+1}$ $\equiv$ $s'_{i+1}$. We have the $\tau$-simulation relation
%until the index $i+1$.
%\end{itemize}
%We can conclude that under the condition ($C_{\tau}$, the executions
%of  $S+T$ are $\tau$-simulation of the executions of $T$.
%\item We show that a $\tau$-simulation that satisfies $C_{dead}$
%and $C_{live}$ preserves $Q \leadsto_{JML} R (J,V,M)$. 
%\begin{itemize}
%\item Suppose that there exists a state of $\sigma'$ with a deadlock then
%this state must verify  $\neg Q \vee R$ ($C_{dead}$). Then the execution
%verify $Q \leadsto_{JML} R (J,V,M)$.
%\item Suppose that there exists a state of $\sigma'$ with a livelock then
%this state must verify  $\neg Q \vee R$ ($C_{dead}$). Then the execution
%verify $Q \leadsto_{JML} R (J,V,M)$.
%\item Suppose that there is no livelock nor deadlock in the execution $\sigma'$,
%then the relation between $\sigma$ and $\sigma'$ is a  divergence s
%ensitive simulation~\cite{glabbeek93}. This kind of simulation preserve
%liveness properties~\cite{bellegarde00}, so $Q \leadsto_{JML} R (J,V,M)$.
%\end{itemize}
%\end{enumerate}\qed
%\end{proof}

%\subsection{Proof of Proposition~\ref{prop-preservation-2}}
%\label{sec-proof-proposition-preservation-2}

%\begin{proof}
%This proof is done by contradiction. 
%\begin{itemize}
%\item Suppose that $C_{dead}$ does not hold on the
%execution  $\sigma'$ then the execution stops in
%a state where  $Q \wedge \neg R$. The liveness 
%property is not satisfied.
%\item Suppose that $C_{live}$ does not hold on the
%execution  $\sigma'$ then the execution diverges in
%a state where  $Q \wedge \neg R$. The liveness 
%property is not satisfied.
%\end{itemize}\qed
%\end{proof}


%\subsection{Proof of Proposition~\ref{prop-deadlock}}
%\label{sec-proof-proposition-CDead}
%\begin{proof}
%\end{proof}


%\subsection{Proof of Proposition~\ref{prop-livelock}}
%\label{sec-proof-proposition-CLive}
%\begin{proof}
%The environment execution can diverge by:
%\begin{enumerate}
%\item A call to a method within the diverging conditions.
%\item An infinite loop.
%\end{enumerate}

%\end{proof}



\subsection{Proof of $R_{while}$}
\label{sec-while}

The proof of rewritting rules for non-nested 
environment is done using the Hoare semantics 
for Java program~\cite{TheseMarieke}. The rules
used in this proof are given iné\ref{fig-Mariekerules}. 
We assume that all Java predicates are side-effects free.

\begin{proof}
Let $S$ and $Q$ be two JML predicates, let $body_{while}$ and
$body_{endwhile}$ be  Java
Statements, let $expr_{while}$ be a side-effect-free Java expression.
Consider the \textit{bottom} side of our rewriting rule $R_{while}$:
\begin{center}
\texttt{
 \begin{tabular}{l}
//@ requires $P$;\\
//@ ensures $Q$;\\
m()\{
while ($\mathsf{expr_{while}}$)\{
$\mathsf{body_{while}}$
\}
$\mathsf{body_{endwhile}}$
\}
\end{tabular}}
\end{center}
In Hoare formalism~\cite{TheseMarieke}, $m$ satisfy its JML
annotations if the following partial correctness triple is satisfied:
\begin{center}
\textsf{\{P\} WHILE ($expr_{while}$)($body_{while}$);$body_{endwhile}$\{Q\}}
\end{center}
Using \textit{Composition Rule}~\cite{TheseMarieke}, we obtain the
followings two Hoare triple.
\begin{center}
\textsf{
\hspace{7em} 
\{P\} WHILE ($Expr_{while}$)($body_{while}$)\{R\} \hspace{8em}($a$)\\
\hspace{10em}
\{R\} $body_{endwhile}$\{Q\} \hspace{13.5em}($b$)\\
}
\end{center}
Using \textit{Partial WHILE Rule}~\cite{TheseMarieke} on ($a$),
we obtain:
\begin{center}
\textsf{
\hspace{8em} \{P $\wedge Expr_{while}$\} ($body_{while}$) \{P\}
\hspace{9em} ($a_1$)\\
\hspace{9em} \{P $\wedge \neg (Expr_{while})$\} skip \{R\}
\hspace{10em} ($a_2$)\\
}
\end{center}
Now using \textit{partial skip rule} and \textit{partial composition rule}
on ($a_2$) and ($b$), we obtain the following Hoare triple:
\begin{center}
\hspace{8em} \{P$\wedge \neg (Expr_{while})$\} $body_{endwhile}$\{Q\}
\hspace{8em} ($c$)\\
\end{center}
Correctness of ($a_1$) and ($c$) can be verified in the JML framework 
by proving the correctness of the JML annotations of the following
methods:
\begin{center}
\texttt{
\begin{tabular}{l}
/*@ requires $\mathsf{expr_{while}}$;\\
~~~@ requires $P$;\\
~~~@ ensures $P$;\\
~~~@*/\\
m\_while()\{
\quad $\mathsf{body_{while}}$;
\}\\
\end{tabular}
\begin{tabular}{l}
/*@ requires $\neg \mathsf{expr_{while}}$;\\
~~~@ requires $P$;\\
~~~@ ensures $Q$;\\
~~~@*/\\
m\_end\_while()\{
\quad $\mathsf{body_{endwhile}}$;
\}\\
\end{tabular}
}
\end{center}
This is the \textit{top} side of our rewriting rule.
\qed


\end{proof}

\begin{figure}
Partial Composition rule
$$\frac{\{P\}S\{R\} \quad \{R\}T\{Q\}}{\{P\}S;T\{Q\}}$$
Partial WHILE rule
$$\frac{\{P \wedge \neg C\} skip \{Q\} \quad \{P \wedge C\} (S)\{P\} }{\{P\}WHILE(C)(S)\{Q\}} $$
Partial SKIP rule
$$\frac{\{P\}skip;S\{R\}}{\{P\}S\{Q\}}$$

\caption{Hoare Triple Rules}
\label{fig-Mariekerules}
\end{figure}

