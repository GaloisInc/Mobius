\parindent 1cm
\parskip 0.2cm
\topmargin 0.2cm
\oddsidemargin 1cm
\evensidemargin 0.5cm
\textwidth 15cm
\textheight 21cm

\def\lstlanguagefiles{lstlangjml.sty}
\lstloadlanguages{Jml}

\newcommand{\benchname}[1]{\texttt{#1}}

\newcommand{\comment}[1]{{\sf #1}}
\newcommand{\alarm}[1]{\marginpar{#1}}

%\newtheorem{definition}{Definition}
%\newtheorem{lemma}{Lemma}
%\newtheorem{theorem}{Theorem}


\def \bsl       {\symbol{92}}
\def \unsc      {\symbol{95}}

\newcommand{\todo}[1]{ \textbf{#1}}
\newcommand{\fig}[1]{ Fig.}
\newcommand{\jmlKey}[1]{\texttt{#1}}% wrapping jml keywords
\newcommand{\java}[1]{\texttt{#1}}
\newcommand{\stack}[1]{\texttt{st(#1)}}% element on top stack 
\newcommand{\counter}{\texttt{ct}}

\newcommand{\true}{\texttt{true}}
\newcommand{\false}{\texttt{false}}

\newcommand{\wpi}{\textit{wp}}

\newcommand{\instr}[1]{\texttt{#1}}

\newcommand{\substitution}[2]{[\tt{#1} \leftarrow \tt{#2}]}
% thegrammar for the bytecode specification language
\newcommand{\ClassSpec}{\rm{ClassSpec}}
\newcommand{\MethodSpec}{\rm{MethodSpec}}
\newcommand{\SpecCase}{\textrm{SpecCase}}
\newcommand{\jmlStmt}[1]{\textrm{#1}}
\newcommand{\interMethodSpec}{\rm{InterMethodSpec}}
\newcommand{\loopSpec}{\rm{loopSpec}}
%\newcommand{\assert}{\rm{assertSpec}}

\newcommand{\ArithExpr}{\texttt{Arithmetic\_Expr}}
\newcommand{\expression}{\mathcal{E} }

\newcommand{\integer}{\texttt{int} }
\newcommand{\register}[1]{\texttt{lv[#1]} }
\newcommand{\reference}{\texttt{ref} }
\newcommand{\intLiteral}{\texttt{int\_literal} }
\newcommand{\Mynull}{\texttt{null}}
\newcommand{\this}{\texttt{this}}
\newcommand{\fieldAccess}[1]{\texttt{field\_cp\_index(}#1 \texttt{)}}
\newcommand{\arrayAccess}[2]{#1[#2] }

\newcommand{\result}{\jmlKey{$\backslash$result}}
\newcommand{\oldp}[1]{\jmlKey{$\backslash$old(}#1\jmlKey{)}}
\newcommand{\typeof}[1]{\jmlKey{$\backslash$typeof(}#1 \jmlKey{)}}
\newcommand{\EXC}{\texttt{EXC}}

\newcommand{\excPost}{\psi^{exc}}

\newcommand{\Myspace}{\phantom{aa}}
\newcommand{\predicate}{ \mathcal{P}} 
\newcommand{\Myfalse}{\textit{false}}
\newcommand{\Mytrue}{ \textit{true} }
\newtheorem{defn}{Definition} 



% abstractCtrlFlow.tex
\newcommand{\execRel}{\rightarrow} % the execution relation
\newcommand{\blockm}[1]{ \tt{b^{#1}} }
\newcommand{\blockSeq}[1]{ \tt{b_{seq}^{#1}} }
%\newcommand{\pathm}[2]{\blockm{#1} \execRel^{*} \blockm{#2} }

\newcommand{\blockPost}[1]{ \it{post(}\tt{b_{seq}^{#1}}\it{)}}

\newcommand{\invariant}{\textit{I}}

\newcommand{\srcVar}[1]{\texttt{#1} }

\newcommand{\method}{\texttt{m}}

% recuperé dans le prelude du code généré par l'Atelier B.
% Doit permettre d'écrire à peu-près correctement <+
\def\famletter#1{\ifcase #1 0\or 1\or 2\or 3\or 4\or 5\or 6\or 7\or
    8\or 9\or A\or B\or C\or D\or E\or F\fi}
\font\msx=msam10
\newfam\msxfam \textfont\msxfam=\msx
\edef\fx{\famletter\msxfam}
\mathchardef    \dres       "2\fx43
\def    \lover      {\mathbin{{\dres} \llap{$-\!\!\!\!-\!$}}}

\def\keywords{\noindent\bf Keywords: \vspace{0pt}
\it\normalsize\normalsize}
\def\endkeywords{\par}

\def\acknowledgement{\vspace{4pt} \noindent{\bf Acknowledgements} \\ \vspace{1pt}
\normalsize\normalsize}
\def\endkeywords{\par}

\newcommand{\JACK}{\texttt{JACK}}
\newcommand{\ESC}{\texttt{ESC/Java}}
\newcommand{\LOOP}{\texttt{LOOP}}


\date{}
\newcommand{\bbb}{\mathbb{B}}
% moved here from carmel.tex
\newcommand{\spp}{\hspace{1.5cm}}


\newcommand{\semb}{\llbracket}
\newcommand{\seme}{\rrbracket}
\newcommand{\sem}[1]{ \semb #1 \seme}
\newcommand{\ia}[1]{ \semb #1 \seme}
\newcommand{\fia}[1]{ \mathcal{F}\semb #1 \seme}
\newcommand{\ria}[1]{ \mathcal{R}\semb #1 \seme}

\newcommand{\Coq}{{\sf Coq}}
\newcommand{\ocaml}{\textsc{ocaml}}

\newcommand{\memvar}{\text{Mem}}

\newcommand{\AbVal}{\widehat{\text{Val}}}
\newcommand{\Val}{{\text{Val}}}
%\newcommand{\Stack}{{\text{Stack}}}
\newcommand{\AbStack}{{\widehat{\text{Stack}}}}
\newcommand{\SafeCallStack}{{\text{SafeCallStack}}}
\newcommand{\OneCall}{{\text{OneCall}}}
\newcommand{\Var}{{\text{Var}}}
\newcommand{\LocalVar}{{\text{LocalVar}}}
\newcommand{\AbLocalVar}{{\widehat{\text{LocalVar}}}}
\newcommand{\State}{{\text{State}}}
\newcommand{\Trace}{{\text{Trace}}}
\newcommand{\AbState}{{\widehat{\text{State}}}}
\newcommand{\progCount}{{\text{progCount}}}
\newcommand{\fieldName}{{\text{fieldName}}}
\newcommand{\methodName}{{\mathrm{methodName}}}
\newcommand{\MM}{{\methodName}}
\newcommand{\PP}{{\progCount}}
\newcommand{\className}{{\text{className}}}
\newcommand{\varName}{{\text{varName}}}
\newcommand{\Adress}{{\text{Adress}}}
\newcommand{\Instruction}{{\text{Instruction}}}
\newcommand{\InstAt}{{\text{InstAt}}}
\newcommand{\Constraint}{{\text{Constraint}}}
\newcommand{\St}{\widehat{\text{St}}}

\newcommand{\config}[1]{{\langle\!\langle #1 \rangle\!\rangle}}
\newcommand{\fram}[1]{{\left\langle #1 \right\rangle}}
\newcommand{\num}{{\text{num}}}
\newcommand{\reff}{{\text{ref}}}
\newcommand{\nul}{{\text{null}}}
\newcommand{\some}{{\text{some}}}
\newcommand{\nameClass}{{\text{nameClass}}}
\newcommand{\class}{{\text{class}}}
\newcommand{\newObject}{{\text{newObject}}}
\newcommand{\newArray}{{\text{newArray}}}
\newcommand{\lengthArray}{{\text{lengthArray}}}
\newcommand{\fieldValue}{{\text{fieldValue}}}
\newcommand{\Value}{{\text{Value}}}
\newcommand{\classes}{{\text{classes}}}
\newcommand{\RefValue}{{\text{RefValue}}}
\newcommand{\Location}{{\text{Location}}}
\newcommand{\methodLookup}{{\text{methodLookup}}}
\newcommand{\nbArgument}{{\text{nbArgument}}}
\newcommand{\nameMethod}{{\text{nameMethod}}}

\newcommand{\ClassName}{{{\text{ClassName}}}}
\newcommand{\FieldName}{{{\text{FieldName}}}}
\newcommand{\default}{{{\text{default}}}}
\newcommand{\END}{{{\text{END}}}}
\newcommand{\AbHeap}{{\widehat{\text{Heap}}}}

\newcommand{\Sinit}{{\mathcal{S}_{\mathit{init}}}}

\newcommand{\instructionAt}{{\text{instructionAt}}}


%\newcommand{\abs}{\mbox{${\cal S}\!\mathit{t}$}}
\newcommand{\abs}{\mbox{$\Sigma$}}
\newcommand{\addr}{\mbox{\em addr}}
\newcommand{\pc}{\mathit{pc}}
\newcommand{\stf}{{\mathit{sf}}}
\newcommand{\cl}{{\mathit{cl}}}

\newcommand{\analyze}{\text{\tt analyse}}
\newcommand{\Analyze}{\ensuremath{\mathit{Unbounded}(P)}}
\newcommand{\Program}{\text{Program}}

\newenvironment{constraint}{%
\noindent
\hspace{.3mm}$
\begin{array}[t]{l}}{%
\end{array}$\\[0.2cm]
}

\newenvironment{contraint}{%
\noindent
\hspace{1cm}$
\begin{array}[t]{l}}{%
\end{array}$\\[0.5cm]
}




\newcommand{\AbPop}{\widehat{\text{pop}}}
\newcommand{\AbPush}{\widehat{\text{push}}}
\newcommand{\AbTop}{\widehat{\text{top}}}
\newcommand{\AbBinop}{\widehat{\text{binop}}}
\newcommand{\AbApply}{\widehat{\text{apply}}}
\newcommand{\AbSubst}{\widehat{\text{subst}}}
\newcommand{\Num}{{{\text{Num}}}}
\newcommand{\AbNum}{{\widehat{\text{Num}}}}
\newcommand{\AbRef}{{\widehat{\text{Ref}}}}
\newcommand{\AbObject}{{\widehat{\text{Object}}}}
%\newcommand{\AbHeap}{{\widehat{\text{Heap}}}}
\newcommand{\Flow}{\text{\tt Flow}}

\newenvironment{myalltt}{\vspace*{-3pt}\begin{alltt}}{\end{alltt}\vspace*{-3pt}}

%%
%% Added by Gerardo (28/09/2004)
%%

\newcommand{\Rule}[2]
{
\frac{#1}
{
\begin{array}{l}
#2
\end{array}
}
}

\newcommand\Loop{\ensuremath{\mathit{Loop}}}
\newcommand\Pred{\ensuremath{\mathit{Pred}}}
\newcommand\BC{\ensuremath{\mathit{BC}}}
\newcommand\MutRec{\ensuremath{\mathit{MutRecR}}}
\newcommand\Rec{\ensuremath{\mathit{Rec}}}
\newcommand\Anc{\ensuremath{\mathit{Anc}}}
\newcommand\LoopCall{\ensuremath{\mathit{LoopCall}}}
\newcommand\Call{\ensuremath{\mathit{Call}}}
\newcommand\Q{\ensuremath{\mathit{Q}}}
\newcommand\tr{\ensuremath{\mathit{tr}}}

\newcommand\End{\ensuremath{\mathtt{end}}}
\newcommand\nop{\ensuremath{\mathtt{nop}}}
\newcommand\push{\ensuremath{\mathtt{push}}}
\newcommand\pop{\ensuremath{\mathtt{pop}}}
\newcommand\dup{\ensuremath{\mathtt{dup}}}
\newcommand\swap{\ensuremath{\mathtt{swap}}}
\newcommand\numop{\ensuremath{\mathtt{numop}}}
\newcommand\load{\ensuremath{\mathtt{load}}}
\newcommand\store{\ensuremath{\mathtt{store}}}
\newcommand\inc{\ensuremath{\mathtt{inc}}}
\newcommand\add{\ensuremath{\mathtt{add}}}
\newcommand\sub{\ensuremath{\mathtt{sub}}}
\newcommand\goto{\ensuremath{\mathtt{goto}}}
\newcommand\If{\ensuremath{\mathtt{if}}}
\newcommand\looksw{\ensuremath{\mathtt{lookupswitch}}}
\newcommand\tabsw{\ensuremath{\mathtt{tableswitch}}}
\newcommand\newarray{\ensuremath{\mathtt{newarray}}}
%\newcommand\default{\ensuremath{\mathtt{default}}}
\newcommand\new{\ensuremath{\mathtt{new}}}
\newcommand\checkc{\ensuremath{\mathtt{checkcast}}}
\newcommand\getst{\ensuremath{\mathtt{getstatic}}}
\newcommand\putst{\ensuremath{\mathtt{putstatic}}}
\newcommand\instof{\ensuremath{\mathtt{instanceof}}}
\newcommand\getfd{\ensuremath{\mathtt{getfield}}}
\newcommand\getfdt{\ensuremath{\mathtt{getfield this}}}
\newcommand\putfd{\ensuremath{\mathtt{putfield}}}
\newcommand\invdef{\ensuremath{\mathtt{invokedefinite}}}
\newcommand\invvir{\ensuremath{\mathtt{invokevirtual}}}
\newcommand\invint{\ensuremath{\mathtt{invokeinterface}}}
\newcommand\return{\ensuremath{\mathtt{return}}}
\newcommand\arrlh{\ensuremath{\mathtt{arraylength}}}
\newcommand\arrld{\ensuremath{\mathtt{arrayload}}}
\newcommand\arrst{\ensuremath{\mathtt{arraystore}}}
\newcommand\throw{\ensuremath{\mathtt{throw}}}
\newcommand\jsr{\ensuremath{\mathtt{jsr}}}
\newcommand\ret{\ensuremath{\mathtt{ret}}}
\newcommand\thiss{\ensuremath{\mathtt{this}}}
%\newcommand\instr{\ensuremath{\mathtt{instr}}}
\newcommand\const{\ensuremath{\mathtt{const}}}
\newcommand\Array{\ensuremath{\mathsf{array}}}
\newcommand\Int{\ensuremath{\mathsf{int}}}
\newcommand\byte{\ensuremath{\mathsf{byte}}}
\newcommand\short{\ensuremath{\mathsf{short}}}
\newcommand\bool{\ensuremath{\mathsf{bool}}}
\newcommand\Ctxt{\ensuremath{\mathsf{Ctxt}}}

\newcommand\warn{\ensuremath{<!>}}
\newcommand\trace{\overline{s}}
\newcommand\defi{\stackrel{\mathrm{def}}{=}}
\newcommand\lub{\sqcup}

\newcommand\Size{\ensuremath{\mathit{Size}}}
\newcommand\Pre{\ensuremath{\mathrm{Pre}}}
\newcommand\Post{\ensuremath{\mathrm{Post}}}
\newcommand\old{\ensuremath{\backslash\mathrm{old}}}

%%%%%%%%%%%%%%%%%%%%%%%%%%%%%%%%%%%%%%%%%%%%%%%%%%%%%%%
%%
%% Mariela's definitions
%%
%%%%%%%%%%%%%%%%%%%%%%%%%%%%%%%%%%%%%%%%%%%%%%%%%%%%%%%


%\newcommand{\pathm}[2]{\blockm{#1} $\ll^{*}$ \blockm{#2} }



%%%%%%%%%Specification commands
\newcommand{\annotation}{BML}
\newcommand{\Apredicate}{\textit{P}}


\newcommand{\requires}{\texttt{requires}}
\newcommand{\ensures}{\texttt{ensures}}
\newcommand{\exsures}[1]{\texttt{exsures(#1)}}
\newcommand{\assert}{\texttt{assert}}

\newcommand{\variant}{\texttt{variant}}

\newcommand{\declare}{\texttt{declare}}
\newcommand{\ghost}{\texttt{Model}}
\newcommand{\ghostSet}{\texttt{set}}
\newcommand{\modifies}{\texttt{modifies}}
\newcommand{\ensemble}[2]{#1 .. #2}
\newcommand{\maxIter}[1]{ iter^#1 }
\newcommand{\progLoop}[1]{\textit{#1}}
\newcommand{\memConsAt}[1]{\Mem^l}

\newcommand{\atState}[2]{#1^{#2} }

%\newcommand{\Mem}{\texttt{MemUsed}}
\newcommand{\Mem}{\texttt{Mem}}
%\newcommand{\old}{\texttt{old}}
\newcommand\Max{\ensuremath{\texttt{Max}}}
\newcommand{\allocated}[1]{allocPath(#1)}
\newcommand{\srcCode}[1]{\texttt{#1}}
\newcommand{\local}[1]{\texttt{localVar}(#1)}



%%%%%%%%%%%% allocation function
\newcommand{\visited}{\texttt{visited}}



%\newcommand{\instrAt}[1]{i_{#1}}
%\newcommand{\instanceOfAlloc}[1]{instanceOfAllocates( #1 )}

\newcommand{\allocInstance}[1]{\texttt{allocInst(#1)}}
\newcommand{\allocLoop}[1]{\texttt{loopCon(#1)}} % function that returns directly the allocations done by a loop : multiplied by the max iterations it can do
\newcommand{\allocMethod}[1]{\texttt{mthdCon(#1)}} % returns the allocations done in a method
\newcommand{\allocLoopWithEnd}[2]{alloc\_loop\_path(#1 , #2)} % returns the allocations done in a loop for a particular path that starts at the start instruction of a loop and that % ends with an insstruction that leads back to the start instructions

\newcommand{\allocIns}[1]{alloc\_instr(#1)} % function that returns that the allocation by the argument

\newcommand{\numLoop}[1]{\textit{numberLoop}(#1)}
\newcommand{\loopEndsSet}[1]{loopEndSet(#1)}
\newcommand{\loopSet}[1]{loopSet(#1)}
\newcommand{\loopEntry}[1]{entry(#1)} % predicate that says that the instruction is an entry to a loop

\newcommand{\backedge}[2]{backedge(#1,#2)} % the start of the backedge  and the end of the backedge


%\newcommand{\wpi}[3]{ \rm{wp}( \srcCode{#1}, #2, #3) } % wp for instructions
\newcommand{\wpExe}[1]{ \rm{wp}(#1) } % wp for blocks
\newcommand{\normalPost}{\psi^{n}}


%\newcommand{\stack}[1]{St(#1)}
\newcommand{\topStack}{c}
\newcommand{\javaNull}{null}
\newcommand{\Ref}[1]{ref_{#1} }
\newcommand{\NULL}{\texttt{null}}



\newcommand{\prevIns}[1]{prev(#1 )}
\newcommand{\nextIns}[1]{next(#1 )}
\newcommand{\targetIns}[1]{target(#1)}



%%%%%%%%%%%%%%%%%%%%%%%%%%%%%%%%%%%%%%%%%%%%%%%%%%%%%


