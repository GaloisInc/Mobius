\section{Related Work}\label{SecRelated}
% Discuss briefly the work by Schneider "Enforceable security policies" and its
% follow up by Ligatti. Talk about it language to specify security properties.

In~\cite{Aktug07} Aktug and Naliuka present ConSpec, a language for security
policy specification.
The ConSpec automaton is similar to our MVA, in particular it
has guarded transitions with actions that can be executed at the entry or exit
(exceptional) of methods. Although they give a formal semantics for a ConSpec
automaton that may allow the user to reason about the property being enforced,
the inlining of the monitor and it correctness are stated informally.
\marginnote{AT: I later found the paper Provably Correct Runtime Monitoring where
they are much more formal, but since you work with Gurov, it may be easier for
you to write about it.}

Cheon and Perumandla propose and extension to JML to specify the allowed
sequences of methods calls in a regular expression-like notation~\cite{Cheon07}.
These specifications are succinct, however, many interesting
security properties cannot be expressed with this formalism. Even our simple
example is out of their scope, because it contains a counter used
only by the specification.

There are tools that translate properties specified in some format to JML
annotations.
One of them is F2J, a tool~\cite{Hubbers03} developed by Hubbers \emph{et al.},
which translates finite state machine specifications into JML annotations and
can also generate a code skeleton for a smart card applet.
Another one is JAG~\cite{Giorgetti06} by Giorgetti and Groslambert, whose input
are properties specified in (a subset of) temporal logic.
The main distinction of these formalisms and ours is that they do not distinguish
between method entry and exit, thus the translation approach is different.
Moreover, none of these works have proved correctness of their translation
algorithms.

%Use of Aspects to do monitoring?
