\section{Related Work}\label{SecRelated}
% Discuss briefly the work by Schneider "Enforceable security policies" and its
% follow up by Ligatti. Talk about it language to specify security properties.

Security automata~\cite{Schneider99} are widely used for monitoring
security properties. The originality of our work lies in considering
them as specifications in a general specification language, with the
ultimate goal of static verification.

Closely related to our approach is work by Aktug \emph{et
al.}~\cite{Aktug07,Aktug08,AktugDG08}, who define a formal language
for security policy specifications, ConSpec, that is similar to our
PA, except that it does not distinguish between normal and exceptional
termination of a method. They prove that a monitor can be inlined into
the program's bytecode, by adding first-order logic annotations, and
then they use a weakest precondition computation that essentially
works the same as the annotation propagation algorithm that we plan to
use~\cite{PavlovaBBHL03} to produce a fully annotated, verifiable
program. In contrast, our algorithm is defined for source code, and
connects with the general-purpose specification language JML. This
allows the use of JML verification tools, to verify the actual policy
adherence. And of course, correctness of our inlining algorithm has
been proven with a theorem prover.

Cheon and Perumandla propose an extension to JML to specify allowed
sequences of methods calls in a regular expression-like
notation~\cite{Cheon07}.  This results in succinct specifications, but
of limited expressiveness. Even our simple example is out of their
scope, because it contains a counter used only by the
specification. Further, they only target runtime verification.

Several tools exist that translate temporal properties into JML
annotations: AutoJML~\cite{Hubbers03} translates finite state machine
specifications into JML annotations and can also generate a code
skeleton for a smart card applet; JAG~\cite{Giorgetti06} translates
properties in (a subset of) temporal logic, including liveness
properties.  However, they typically do not distinguish between method
entry and exit, and moreover, correctness of the translation algorithm
has not been proven.

For more information about policy languages, monitor inlining and
specifying policy adherence, we refer to Section 4.10 of Aktug's
thesis~\cite{Aktug08}. 
%Use of Aspects to do monitoring?
