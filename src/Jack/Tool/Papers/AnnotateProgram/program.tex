\section{Programs and Semantics}\label{SecProgram}

This section first defines an abstract syntax of programs, followed by
their semantics. Both are fairly standard, except that the semantics
is parametrised on the treatment of specifications. In particular, we
define a run-time checking and a monitoring semantics, that evaluate
differently upon method call and exit.

% commented out: version with different constants
%\[
%\begin{array}{rcl}
%\NumExpr & = & \I(i : \mathbb{Z}) \mid \Plus(n_1, n_2 : \NumExpr) \mid \ldots \mid
%               \EvalN(n : \Name) \\
%\BoolExpr & = & \ttt \mid \fff \mid \Not(b : \BoolExpr) \mid \Conj(b_1,
%b_2 : \BoolExpr) \mid \ldots \mid \\
%          &   & \Eq(e_1, e_2 : \Expr) \mid \EvalB(n
%: \Name) \\
%\RefExpr & = & \Null \mid \EvalR(n : \Name)\\
%\Expr & = & \BExpr(b : \BoolExpr) \mid
%            \NExpr(n : \NumExpr) \mid
%            \RExpr(r : \RefExpr) \mid\\
%      &   & \CondExpr(c : \BoolExpr, e_1, e_2 : \Expr) \mid
%            \Assign(n : \Name, e : \Expr) \mid \\
%      &   & \Call(o : \Expr, n : \Name, p : \Expr) \mid
%            \Const(v : \Val) \\
%\Stmt & = & \Skip \mid
%            \Sequence(s_1, s_2 : \Stmt) \mid
%            \IfThenElse(c : \BoolExpr, s_1, s_2 : \Stmt) \mid\\
%      &   & \While(c : \BoolExpr, s : \Stmt) \mid
%            \StmtExpr(e : \Expr) \mid
%            \Throw \mid\\
%      &   & \TryCatch(t: \Stmt, e : \Excpt c, f : \Stmt) \mid
%            \Set(n : \Name, e : \Expr) \mid\\
%      &   & \CaseJML(b : \listof{\BoolExpr \times \Stmt}) \mid
%            \Assert(e : \BoolExpr)
%\end{array}
%\]

\begin{figure}[t]
\[{\small
\begin{array}{rcl}
\Expr & = & \Plus(n_1, n_2 : \Expr) \mid
            \EvalN(n : \Name) \mid
            \Not(b : \Expr) \mid
            \Conj(b_1, b_2 : \Expr) \mid \\
      &   & \Eq(e_1, e_2 : \Expr) \mid
            \EvalB(n : \Name) \mid
            \EvalR(n : \Name) \mid
            \CondExpr(c, e_1, e_2 : \Expr) \mid\\
      &   & \Assign(n : \Name, e : \Expr) \mid
            \Call(o : \Expr, mn : \Name, p : \Expr) \mid
            \Const(v : \Val) \\
\Stmt & = & \Skip \mid
            \Sequence(s_1, s_2 : \Stmt) \mid
            \IfThenElse(c : \Expr, s_1, s_2 : \Stmt) \mid\\
      &   & \While(c : \Expr, s : \Stmt) \mid
            \StmtExpr(e : \Expr) \mid
            \Throw(e : \Excpt) \mid\\
      &   & \TryCatch(t: \Stmt, e : \Excpt, c, f : \Stmt) \mid
            \Set(n : \Name, e : \Expr) \mid\\
      &   & \CaseJML(b : \listof{\Expr \times \Stmt}) \mid
            \Assert(e : \Expr)
\end{array}}
\]
\caption{Abstract syntax of expressions and statements}
\label{FigExprStmt}
\end{figure}

\begin{figure}[t]
\[{\small
\begin{array}{rcl}
\Method & = \opr & \name : \Name,
                   \param : \LocalVarDecl,
                   \lvars : \setof{\LocalVarDecl},
                   \body : \Stmt,\\
        &        & \res : \Expr,
                   \restype : \Type,
                   \pre, \post : \Expr \rightarrow \Expr, \\
        &        & \preset, \postset : \Expr \rightarrow \Stmt,
                   \excset : \Excpt \rightarrow \Stmt \clr \\
\Class & = \opr & \name : \Name,
                  \super : \Name_{\bot},
                  \fields : \setof{\FieldDecl},
                  \methods : \setof{\Method},\\
        &       & \inv : \Expr,
                  \ghostvars : \setof{\FieldDecl} \clr\\
\Program & = \opr & \classes : \setof{\Class} \clr
\end{array}}
\]
\caption{Abstract Syntax for Programs}\label{FigProgram}
\end{figure}

\subsection{Program Syntax}\label{SecSyntax}

Our language is a restricted subset of (sequential) Java, abstracting
away from typical object-oriented features. This means in particular
that we abstract away from method resolution; instead we assume that
the annotated class contains method bodies for the relevant methods,
thus method lookup is trivial. Similarly, we assume that lookup of the
class invariant returns the complete class invariant, including those
invariants that are inherited from superclasses. Moreover, we consider
only a few exceptions, and assume that methods have only one
parameter. We believe, however, that our formalisation contains all
constructs that are relevant for proving correctness of our approach,
and that implementing the algorithm for the full language is mainly an
engineering issue (see Section~\ref{SecConcl}).

Figure~\ref{FigExprStmt} defines expressions and statements as a
mutually recursive data type (we use the term \emph{body} to denote
either an expression or a statement). Notice that we have some special
language constructs that are used to represent JML annotations: \Set,
to update ghost variables (\emph{i.e.}, specification-only variables),
\CaseJML, to abbreviate a list of conditional ghost variable updates,
and \Assert, to evaluate a condition on the program state. A standard
program semantics ignores these statements, but the annotated program
semantics evaluates them.
% Therefore, the state of an annotated program
% is extended with a ghost variable store.
% , which is updated according to the \Set or \CaseJML statement.
% If the condition of an \Assert evaluates to false, a special \JMLExc is thrown to
% signal the annotation violation.
% If the condition of an \Assert evaluates to false, a \JMLExc is thrown.

Figure~\ref{FigProgram} describes the syntax for methods,
classes and programs. To ensure that every method has an appropriate
return expression, it is part of the method signature.
Furthermore, methods can be annotated with pre- and postconditions, and
classes with invariants. To support our annotation generation
algorithm, we define special annotations called
\(\preset\), \(\postset\) and \(\excset\). These annotations describe
the updates to the ghost variables at method entry, exit and
exceptional exit, respectively.
Pre- and postcondition and the different method specification-level set
annotations are functions to allow the use of the method parameter,
the method result and the possible return exception\footnote{Notice
that the type of \pre, \post, \preset and \postset could be made more specific:
the first expression argument is a side-effect free expression.}.
% The semantics will apply them appropriately.

A program is said to be \emph{wellformed} if
\begin{inparaenum}[(\itshape i\upshape)]
\item names of fields, local variables and ghost variables are
disjoint and are not reserved words;
\item class names are unique;
\item method names are unique;
\item every variable name that is used is declared; and
\item only ghost variables are the target of \Set statements.
\end{inparaenum}
We have stated only the wellformedness conditions necessary to our
correctness proofs.

\subsection{Natural Semantics}\label{SecSemantics}
The behaviour of a program is described via a big step semantics. We
closely follow Von Oheimb's formalisation of Java~\cite{Oheimb01},
with simplifications wherever possible, due to our simplified
program syntax. A judgement $\etp{P}{e,\sigma}{v,\sigma'}$, means that the body
$e$ evaluates to $v$, while transforming the state $\sigma$ into $\sigma'$, in
the context of the program \(P\). Note that \(v\) is \One for
normally terminating \emph{statements}, while \(v\) is \(\bot\) whenever
the evaluation finishes in an exceptional state.

A basic program state \(\Pstate\) is composed of an optional exception
and a store.  The store maps every field and local variable to a
value.

\[
{\small
\begin{array}{rcl}
\Pstate & = \opr & \ex : \Excp_{\bot}, \st : \PStore \clr\\
\PStore & = \opr & \fvs : \Name \mapsto \Val, \lvs : \Name \mapsto \Val \clr
\end{array}}
\]

% Commented out to save space
% When bodies are evaluated, uncaught exceptions are propagated.
% Exceptions can only be caught explicitly \emph{within} a \(\TryCatch\)
% statement.  All evaluation rules require explicitly that the
% initial state is not exceptional.

% Commented out to save space
% As mentioned above, we define both the semantics of an (annotated)
% program and of a monitored program by parametrising the semantics with
% functions that specify how the specification constructs are to be
% evaluated.
Since annotated or monitored programs
contain more information than unannotated programs, the
evaluation rules are parametrised with types
\(\FullProgram\) and \(\FullState\). For each instantiation we give
mappings \program and \progstate to the basic program type \Program
and the basic program state \Pstate.

The evaluation rules are fairly standard, therefore we refer to Von
Oheimb and the PVS formalisation for more details.
Figure~\ref{FigEvalRules} shows the evaluation of normally
terminating method calls (where for clarity of presentation, we have
left out several checks that intermediate states are not exceptional).
First the receiver is evaluated, resulting in reference \(r\).
Next, the parameter is evaluated, resulting in value \act. If \(r\) is not null,
the method definition \md is looked up.
The local variable store is updated assigning
%the receiver reference
\(r\) to \texttt{this}, initialising the
method's local variables and assigning the actual parameter to the
formal parameter. The old local variable store is kept as \oldlvs.
%to be restored after method termination.
Next, an appropriate action upon method entry is taken, as specified by the
relation \(\gammain\), one of the parameters of the semantics.
% ~--~below we will discuss different instantiations.
Then the method body, and method result
expression are evaluated. Since this rule applies to normal method
termination only, the semantics' parameter for normal method termination
\(\gammanorm\) is evaluated. Last, the local store is set back to
\(\oldlvs\). %, and this terminates the method call.
Similar rules exist that describe what happens upon exceptional termination,
% of a method,
when a method is called upon a null reference \emph{etc.}



\begin{figure}[t]
\[{\small
\begin{array}{c}
\sigma_0.\progstate.\ex = \bot \qquad        % no exception in sigma_0
\etp{P}{o, \sigma_0}{r,\sigma_1} \qquad      % evaluate receiver
\etp{P}{p, \sigma1}{\act, \sigma_2}\\        % evaluate argument
%\sigma_2.\progstate.\ex = \bot \qquad        % no exceptions in sigma_2
r \not= \Null \qquad                         % receiver not null
\md = \lookupmthd(P, r, \mn) \\
\oldlvs = \sigma_2.\progstate.\st.\lvs \qquad
\sigma_3 = \updatelvs(\sigma_2, r, \md.\lvars, md.\param, \act) \\
\gammain(P, \md, r, \Const(\act), \sigma_3, \sigma_4) \qquad
\etp{P}{\md.body, \sigma_4}{\One,\sigma_5} \\
\etp{P}{\md.\res, \sigma_5}{v,\sigma_6}\qquad
%\ex(\progstate(\sigma_6)) = \bot
\gammanorm(P, \md, r, \Const(v), \sigma_6, \sigma_7)\\
%\ex(\progstate(\sigma_7)) = \bot\\
\hline
\etp{P}{\Call(o,\mn,p), \sigma_0}{v, \sigma_7
(\progstate.\st.\lvs := \oldlvs)}
\end{array}}
\]
\caption{Evaluation rule for normal termination of method
calls}\label{FigEvalRules}
\end{figure}

Besides the relations \gammain and \gammanorm, the semantics
is also parametrised with
\gammaexc, describing what happens upon exceptional termination of a
method, and \deltaset, \deltacase, and \deltaassert, describing the
semantics of the \Set, \CaseJML and \Assert statements,
respectively. In a standard program semantics, where specifications
are ignored, all these parameters are instantiated % basically
with the identity relation.

%\paragraph{Standard program semantics}
%To evaluate a program without considering the specifications,
%\(\FullProgram\) and \(\Program\) and
%\(\FullState\) and \(\Pstate\) are identical, thus \program and
%\progstate are identify functions. The specification evaluation
%functions are instantiated as identity relations, \emph{i.e.,}
%\(\gammain(P, \md, r, \tau_1, \tau_2) = \gammanorm(\ldots) =
%\gammaexc(\ldots) = (\tau_1 = \tau_2)\), and \(\deltaeval(P,
%\EvalG(n), \tau_1, v, \tau_2, \ldots) = \deltaset(\ldots) = \deltacase(\ldots) = (\tau_1 =
%\tau_2)\).


\paragraph{Annotated Program Semantics}

The program state of an annotated program is extended with a store for
ghost variables:
\[{\small
\Astate = \opr \pstate : \Pstate, \gvs : \Name \mapsto \Val \clr}
\]
The types \FullProgram and \Program coincide, while \FullState is
instantiated as \Astate, and the mapping \progstate is defined as
\pstate. Figure~\ref{FigAnnotatedSem} shows some of the
instantiations of the semantics parameters; % (presented in rule format);
the other instantiations are similar. Relation \gammain uses an
auxiliary relation \(\beta\) which checks a boolean expression \(e\) and
raises a special \JMLExc if it evaluates to false. Upon method
entry, the class invariant and precondition are evaluated. If they
fail, a \JMLExc is thrown, otherwise the method's \preset statement is
executed. Finally, we ensure that the program store is not changed.
The function
\deltaset updates a ghost variable: it first evaluates the expression
and if this did not result in an exceptional state, it updates the
value of the ghost variable\footnote{We use \(\tau(\gvs.n := v)\) to
abbreviate that the \(n^{th}\) entry of the \gvs component of \(\tau\) is
updated to \(v\).} appropriately.

% Relation with the code:
% r = a, act = arg, \sigma_1 = s1, \sigma_2 = s2, \beta = check_assertion
\begin{figure}[t]
\[{\small
\begin{array}[t]{c}
\invar = \lookupinv(P, r) \qquad
\beta(P, \invar, \sigma_1, \tau_1) \qquad
\beta(P, \md.\pre(\act), \tau_1, \tau_2) \\
\etp{P}{md.\preset(\act), \tau_1}{v,\tau_2} \qquad v\in\{\bot,\One\} \qquad
\sigma_1.\pstate.\st = \sigma_2.\pstate.\st\\
\hline
\gammain(P, \md, r, \act, \sigma_1, \sigma_2)
\smallskip\\


\etp{P}{e, \sigma_1}{v, \tau} \qquad
\pif{v = \B(\ttt)}{\sigma_2 = \tau}{\sigma_2 = \tau (\ex := \JMLExc)}\\
\hline
\beta(P, e, \sigma_1, \sigma_2)

\smallskip\\

\etp{P}{e, \sigma_1}{v, \tau} \qquad
\pif{\tau.\pstate.\ex = \bot}{\sigma_2 = \tau (\gvs.n := v)}{\sigma_2 = \tau}\\
\hline
\deltaset(P, \Set(e, n), \sigma_1, \sigma_2)
\end{array}}
\]
\caption{Instantiation of semantics for runtime annotation evaluation}
\label{FigAnnotatedSem}
\end{figure}


\paragraph{Monitored Program Semantics}
The parametric program semantics is also instantiated for monitored
programs. This semantics is only defined when the MVA is compatible
with the program. An MVA \(a\) is said to be compatible with a program
\(P\), denoted \(a \sqsubseteq P\), if
\begin{inparaenum}[(\itshape i\upshape)]
\item the program contains the class \(c\) that is being monitored,
\item all variables declared as program variables in
\(a\) are fields of the class \(c\) with the correct type, and
\item every event name corresponds to a method in the class.
\end{inparaenum}
A monitored program is a product of an MVA and a program. The state
of a monitored program consists of the states of the MVA and the
program, and a flag \stuck. If the MVA is partial, the flag
\stuck is set when \(\Delta_a\) is not defined for a
certain input. If the flag is set, this means that the security policy
is violated, and the program should be stopped. If the MVA is total,
the \stuck flag will never be set. Instead, violation of the security
policy is modelled by the MVA reaching the error state \halted.

\[
{\small
\begin{array}{rcl}
\Mprogram & =  & \opr \mva : \MVA, \program : \Program \clr\\
\Mstate & = & \opr \mvastate : \MVAstate, \progstate : \Pstate, \stuck
: \mathbb{B} \clr
\end{array}}
\]
Thus, \FullProgram gets instantiated as \Mprogram and \FullState as
\Mstate, with mappings \program and \progstate. Now we can give
appropriate instantiations for the \(\gamma\)- and
\(\delta\)-relations. In fact, the \(\delta\)-relations are the same as
for the standard or annotated program semantics (depending on whether
we wish the monitored program to evaluate annotations).
Here we show how \gammain changes the state
of the monitor, using \(\hat\gammain\) for the necessary checks at
method entrance in the program itself (\emph{i.e.}, the evaluation of
the method specification).
\[
{\small
\begin{array}{c}
\hat\gammain(P, \md, r, \act, \sigma_1, \tau) \\
\pif{\tau.\pstate.\ex = \bot}
    {\sigma_2 = \delta(\entry)(P, \md, \act, \tau)}
    {\sigma_2 = \tau}\\
\hline
\gammain(P, \md, r, \act, \sigma_1, \sigma_2)
\end{array}}
\]
\marginnote{\small isn't $\hat\gammain$ the same defined in
Fig~\ref{FigAnnotatedSem}? If so why don't we refer to it?}

\noindent where
% This is a simplified version of on_method_MVA in MonitoredProgram
\[
{\small
\begin{array}{rcl}
\delta(\ev)(P, \md, \act, \sigma) & = &
\begin{array}[t]{l}
\mathsf{let\ }
\begin{array}[t]{rcl}
  e & = & \opri \etype := \ev, \mname := \md.\name \clri\\
 \tau & = & \Delta_{P.\mva}(\sigma.\mvastate, \sigma.\progstate, e, \act)
\end{array}
\mathsf{in\ }\\
\pif{\sigma.\stuck \vee \tau = \bot}
    {\sigma (\stuck := \ttt)}
    {\sigma (\mvastate := \tau)}
\end{array}
\end{array}}
\]


%% This example does not add much is a bit disconnected with the previous text
% \paragraph{Example}
%
% To be able to generate the annotations capturing the security policy
% specified by the LimitSMS automaton in Figure~\ref{FigExample}, below
% is a class declaration of class \texttt{Messaging}, declaring the
% methods used by the automaton, plus a method \texttt{receiveSMS}, and
% a field \texttt{counter}. To illustrate our annotation algorithm, we
% do not need to give an implementation here; it is only important to
% know that the methods actually have an implementation.
%
% {\small\begin{verbatim}
% class Messaging {
%   int counter;
%   void sendSMS(){/* body of sendSMS */}
%   void receiveSMS(){/* body of receiveSMS */}
%   void reset() {/* body of reset */} }
% \end{verbatim}}

%\caption{Implementation of class \texttt{Messaging}}\label{FigExampleImplem}
%\end{figure}
%\subsubsection{PVS formalisation}
%As mentioned above, the syntax and semantics for MVA and programming
%language have been formalised in PVS.

% To formalise the program semantics in PVS is not
%completely straightforward. PVS does not allow one to define mutually
%recursive definitions, and moreover for every recursive method call,
%termination has to be shown. As a solution, we define a function
%\texttt{deriv\_expr} parametrised with
%natural number \(n\) and function \texttt{deriv\_stmt}, where the
%argument \texttt{deriv\_stmt} is only applicable to natural numbers
%strictly less than \(n\). The number \(n\) provides an upperbound on
%the number of recursive method calls that are going to be made.  The
%function \texttt{deriv\_stmt}, applied to natural number \(m\) can
%then directly call the \texttt{deriv\_expr} function, with
%\texttt{deriv\_stmt} as argument, but restricted to calls with arguments
%strictly less than \(m\). The parameters \gammain, \gammanorm
%\emph{etc.} are all parametrised with \texttt{deriv\_expr} and
%\texttt{deriv\_stmt}, but again restricted to be applicable only for
%strictly smaller natural numbers. We can then prove that the natural
%number argument is always strictly decreasing, and this is sufficient
%to guarantee termination.







