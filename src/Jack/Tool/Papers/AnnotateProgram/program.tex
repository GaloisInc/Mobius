\section{Programs and Semantics}\label{SecProgram}

This section first defines an abstract syntax of programs, and then it
defines the semantics. The program syntax and semantics are fairly
standard. However, an important feature is that the semantics is
parametrised on what happens with specifications. In particular, we
define a run-time checking and a monitoring semantics, that evaluate
differently upon method call and exit.

% commented out: version with different constants
%\[
%\begin{array}{rcl}
%\NumExpr & = & \I(i : \mathbb{Z}) \mid \Plus(n_1, n_2 : \NumExpr) \mid \ldots \mid
%               \EvalN(n : \Name) \\
%\BoolExpr & = & \ttt \mid \fff \mid \Not(b : \BoolExpr) \mid \Conj(b_1,
%b_2 : \BoolExpr) \mid \ldots \mid \\
%          &   & \Eq(e_1, e_2 : \Expr) \mid \EvalB(n
%: \Name) \\
%\RefExpr & = & \Null \mid \EvalR(n : \Name)\\
%\Expr & = & \BExpr(b : \BoolExpr) \mid
%            \NExpr(n : \NumExpr) \mid
%            \RExpr(r : \RefExpr) \mid\\
%      &   & \CondExpr(c : \BoolExpr, e_1, e_2 : \Expr) \mid
%            \Assign(n : \Name, e : \Expr) \mid \\
%      &   & \Call(o : \Expr, n : \Name, p : \Expr) \mid
%            \Const(v : \Val) \\
%\Stmt & = & \Skip \mid
%            \Sequence(s_1, s_2 : \Stmt) \mid
%            \IfThenElse(c : \BoolExpr, s_1, s_2 : \Stmt) \mid\\
%      &   & \While(c : \BoolExpr, s : \Stmt) \mid
%            \StmtExpr(e : \Expr) \mid
%            \Throw \mid\\
%      &   & \TryCatch(t: \Stmt, e : \Excpt c, f : \Stmt) \mid
%            \Set(n : \Name, e : \Expr) \mid\\
%      &   & \CaseJML(b : \listof{\BoolExpr \times \Stmt}) \mid
%            \Assert(e : \BoolExpr)
%\end{array}
%\]

\begin{figure}[t]
\[
\begin{array}{rcl}
\NumExpr & = & \Plus(n_1, n_2 : \NumExpr) \mid \EvalN(n : \Name) \\
\BoolExpr & = & \Not(b : \BoolExpr) \mid \Conj(b_1,
b_2 : \BoolExpr) \mid \Eq(e_1, e_2 : \Expr) \mid \EvalB(n : \Name) \\
\RefExpr & = & \EvalR(n : \Name)\\
\Expr & = & \BExpr(b : \BoolExpr) \mid
            \NExpr(n : \NumExpr) \mid
            \RExpr(r : \RefExpr) \mid\\
      &   & \CondExpr(c : \BoolExpr, e_1, e_2 : \Expr) \mid
            \Assign(n : \Name, e : \Expr) \mid \\
      &   & \Call(o : \Expr, n : \Name, p : \Expr) \mid
            \Const(v : \Val) \\
\Stmt & = & \Skip \mid
            \Sequence(s_1, s_2 : \Stmt) \mid
            \IfThenElse(c : \BoolExpr, s_1, s_2 : \Stmt) \mid\\
      &   & \While(c : \BoolExpr, s : \Stmt) \mid
            \StmtExpr(e : \Expr) \mid
            \Throw \mid\\
      &   & \TryCatch(t: \Stmt, e : \Excpt, c, f : \Stmt) \mid
            \Set(n : \Name, e : \Expr) \mid\\
      &   & \CaseJML(b : \listof{\BoolExpr \times \Stmt}) \mid
            \Assert(e : \BoolExpr)
\end{array}
\]
\caption{Abstract syntax of expressions and
statements}\label{FigExprStmt}
\end{figure}

\begin{figure}[t]
\[
\begin{array}{rcl}
\Method & = \opr & \name : \Name,
                   \param : \LocalVarDecl,
                   \lvars : \setof{\LocalVarDecl},
                   \body : \Stmt,\\
        &        & \res : \Expr,
                   \restype : \Type,
                   \pre, \post : \Expr \rightarrow \BoolExpr, \\
        &        & \preset, \postset : \Expr \rightarrow \Stmt,
                   \excset : \Excpt \rightarrow \Stmt \clr \\
\Class & = \opr & \name : \Name,
                  \super : \Name_{\bot},
                  \fields : \setof{\FieldDecl},
                  \methods : \setof{\Method},\\
        &       & \inv : \BoolExpr,
                  \ghostvars : \setof{\FieldDecl} \clr\\
\Program & = \opr & \classes : \setof{\Class} \clr
\end{array}
\]
\caption{Abstract Syntax for Programs}\label{FigProgram}
\end{figure}

\subsection{Program Syntax}\label{SecSyntax}

The language that we consider is a restricted subset of (sequential)
Java, abstracting away from the typical object-oriented features. This
means in particular that we abstract away from method resolution;
instead we assume that the annotated class contains method bodies for
the relevant methods, thus method lookup is trivial. Similarly, we
assume that lookup of the class invariant returns the complete class
invariant, including those invariants that are inherited from
superclasses. Moreover, we consider only a few exceptions, and assume
that methods have only one parameter. However, we believe that our
formalisation contains all constructs that are relevant for proving
correctness of our approach, and that implementing the technique for
the full language is mainly an engineering issue (see
Section~\ref{SecImplem}).


Figure~\ref{FigExprStmt} defines expressions and statements as a
mutually recursive data type (we use the term \emph{body} to denote
either expressions or statements). Notice that we have a few language
constructs that are used to represent JML annotations: \Set, to update
ghost variables (\emph{i.e.}, specification-only variables), \CaseJML,
to abbreviate a list of conditional ghost variable updates, and
\Assert, to evaluate a condition on the program state. A standard
program semantics ignores these statements, but the annotated program
semantics does evaluates them. Therefore, the state of an annotated
program is extended with a ghost variable store, that is updated
according to the \Set or
\CaseJML statement. If an \Assert statement is evaluated, and its
condition evaluates to false, a \JMLExc is throw and the program
stops~--~otherwise execution continues normally.


Finally, Figure~\ref{FigProgram} describes the syntax for methods,
classes and programs. To ensure every method has an appropriate return
expression, this expression is part of the method signature.  Further,
methods can be annotated with pre- and postconditions, and classes
with invariants. To support our annotation generation algorithm, for
each method we define special annotations, called
\(\preset\), \(\postset\) and \(\excset\). These annotations describe
the updates to the ghost variables at method entry, exit and
exceptional exit, respectively. To allow the use of the method
parameter, the method result and the possible return exception, pre-
and postcondition and the different specification-level set
annotations are functions. The semantics will apply these
appropriately.

A program is said to be \emph{wellformed} if
\begin{inparaenum}[(\itshape i\upshape)]
\item names of fields, local variables and ghost variables are
disjoint;
\item the names used in the program are not reserved words;
\item class names are unique;
\item method names are unique;
\item every variable name that is used is declared; and
\item only ghost variables are the target of \Set statements.
\end{inparaenum}
Notice that we decided to state only the wellformedness conditions
that were necessary to complete our correctness proofs.

\subsection{Natural Semantics}\label{SecSemantics}
The behaviour of a program is described using a big step semantics. We
closely follow Von Oheimb's formalisation of Java~\cite{Oheimb01},
with simplifications wherever possible because of our simplified
program syntax. For every body, we derive a judgement
$\etp{P}{e,\sigma}{v,\sigma'}$, meaning that  $e$
evaluates to $v$, while transforming state $\sigma$ into $\sigma'$ in
the context of program \(P\). Notice that \(v\) is \One for
(normally) terminating \emph{statements}, while \(v\) is \(\bot\) whenever
evaluation finishes in an exceptional state.

A basic program state \(\Pstate\) is composed of an optional exception
and a store.  The store maps every field and local variable to a
value.

\[
\begin{array}{rcl}
\Pstate & = \opr & \ex : \Excp_{\bot}, \st : \PStore \clr\\
\PStore & = \opr & \fvs : \Name \mapsto \Val, \lvs : \Name \mapsto \Val \clr
\end{array}
\]

When bodies are evaluated, uncaught exceptions are
propagated. Exceptions can only by caught explicitly within a
\(\TryCatch\) statement. All other evaluation rules require explicitly
that the initial state is not exceptional.

As mentioned above, we define both the semantics of an (annotated)
program and of a monitored program. We do this by parametrising the
semantics with functions that specify how the specification constructs
are to be evaluated. Since annotated or monitored programs and/or
their states contain more information than basic programs, the
evaluation rules are defined over parametric types
\(\FullProgram\) and \(\FullState\). For each instantiation we give
mappings \program and \progstate to the basic program type \Program
and the basic program state \Pstate.

The evaluation rules are fairly standard, therefore we refer to Von
Oheimb and the PVS formalisation for most of the
rules. Figure~\ref{FigEvalRules} shows the rule that evaluates
normally terminating method calls. For simplicity, we have left out
several checks that intermediate states are not exceptional (but these
are present in the PVS formalisation). First the receiver is
evaluated, resulting in a reference \(r\). Next, the parameter is
evaluated, resulting in a value \act. If \(r\) is not null, the method
definition \md can be looked up. The local variable store is updated
with the receiver reference \(r\), assigned to \texttt{this},
initialising the methods local variables and assigning the actual
parameter to the formal parameter. The old local variable store is
kept, to be restored after method termination. Next, appropriate
actions upon method entry are taken, as specified by the relation
\(\gammain\). This function is one of the parameters of the
semantics~--~below we will discuss different instantiations. Next the
method body, and method result expression are evaluated. Since we are
considering here the case where the method terminates normally, the
parameter for normal method termination \(\gammanorm\) is
evaluated. Last, the local store is set back to its original contents,
and this terminates the method call. Similar rules exist that describe
what happens upon exceptional termination of a method, when a method
is called upon a null reference \emph{etc.}



\begin{figure}[t]
\[
\begin{array}{c}
\ex(\progstate(\sigma_0)) = \bot \qquad % no exception in sigma_0
\etp{P}{o, \sigma_0}{r,\sigma_1} \qquad      % evaluate receiver
\etp{P}{p, \sigma1}{\act, \sigma_2}\\          % evaluate argument
%\ex(\progstate(\sigma_2)) = \bot \qquad   % no exceptions
r \not= \Null \qquad                    % receiver not null
\md = \lookupmthd(P, r, \mn) \\
\oldlvs = \sigma_2.\progstate.\st.\lvs \qquad
\sigma_3 = \updatelvs(\sigma_2, r, \md.\lvars, md.\param, \act) \\
\gammain(P, \md, r, \Const(\act), \sigma_3, \sigma_4) \qquad
\etp{P}{\md.body, \sigma_4}{\One,\sigma_5} \\
\etp{P}{\md.\res, \sigma_5}{v,\sigma_6}\qquad
%\ex(\progstate(\sigma_6)) = \bot
\gammanorm(P, \md, r, \Const(v), \sigma_6, \sigma_7)\\
%\ex(\progstate(\sigma_7)) = \bot\\
\hline
\etp{P}{\Call(o,\mn,p), \sigma_0}{v, \sigma_7
(\progstate.\st.\lvs := \oldlvs)}
\end{array}
\]
\caption{Evaluation rule for normal termination of method
calls}\label{FigEvalRules}
\end{figure}

Besides the relations \gammain and \gammanorm, the semantics
is also parametrised with
\gammaexc, describing what happens upon exceptional termination of a
method, and \deltaset, \deltacase, and \deltaassert, describing the
semantics of the \Set, \CaseJML and \Assert statement,
respectively. In a standard program semantics, where specifications
are ignored, all these parameters are instantiated basically with the
identity relation.

%\paragraph{Standard program semantics}
%To evaluate a program without considering the specifications,
%\(\FullProgram\) and \(\Program\) and
%\(\FullState\) and \(\Pstate\) are identical, thus \program and
%\progstate are identify functions. The specification evaluation
%functions are instantiated as identity relations, \emph{i.e.,}
%\(\gammain(P, \md, r, \tau_1, \tau_2) = \gammanorm(\ldots) =
%\gammaexc(\ldots) = (\tau_1 = \tau_2)\), and \(\deltaeval(P,
%\EvalG(n), \tau_1, v, \tau_2, \ldots) = \deltaset(\ldots) = \deltacase(\ldots) = (\tau_1 =
%\tau_2)\).


\paragraph{Annotated Program Semantics}


The program state of an annotated program is extended with a store for
ghost variables:
\[
\Astate = \opr \pstate : \Pstate, \gvs : \Name \mapsto \Val \clr
\]
The types \FullProgram and \Program coincide, while \FullState is
instantiated as \Astate, and the mapping \progstate is defined as
\pstate. Figure~\ref{FigAnnotatedSem} shows some of the
instantiations of the semantics parameters; the other instantiations
are similar. Relation \gammain uses an auxiliary relation \(\beta\)
which checks boolean expression \(e\) and raises a special \JMLExc if
the expression is not true. Upon method entry, the class invariant and
precondition are evaluated. If they fail, a \JMLExc is thrown,
otherwise the method's \preset statement is executed. Finally, we
ensure that the program store is not changed. The function
\deltaset updates a ghost variable: it first evaluates the expression
and if this did not result in an exceptional state, it updates the
value of the ghost variable\footnote{We use \(\tau(\gvs.n := v)\) to
abbreviate that the \(n\) entry of the \gvs entry of \(\tau\) is
updated.} appropriately.

\begin{figure}[t]
\[
\begin{array}[t]{c}
\invar = \lookupinv(P, r) \qquad
\beta(P, \invar, \sigma_1, \tau_1) \qquad
\beta(P, \md.\pre(\act), \tau_1, \tau_2) \\
\etp{P}{\preset(md)(\act), \tau_1}{v,\tau_2} \qquad v\in\{\bot,\One\} \qquad
\st(\pstate(\sigma_1)) = \st(\pstate(\sigma_2))\\
\hline
\gammain(P, \md, r, \act, \sigma_1, \sigma_2)
\smallskip\\


\etp{P}{e, \sigma_1}{v, \tau} \qquad
\pif{v = \B(\ttt)}{\sigma_2 = \tau}{\sigma_2 = \tau (\ex := \JMLExc)}\\
\hline
\beta(P, e, \sigma_1, \sigma_2)

\smallskip\\

\etp{P}{e, \sigma_1}{v, \tau} \qquad
\pif{\ex(\pstate(\tau)) = \bot}{\sigma_2 = \tau (\gvs.n :=
v)}{\sigma_2 = \tau}\\
\hline
\deltaset(P, \Set(e, n), \tau_1, \tau_2)
\end{array}
\]
\caption{Instantiation of semantics for runtime annotation evaluation}
\label{FigAnnotatedSem}
\end{figure}

\paragraph{Monitored Program Semantics}
We also instantiate the general program semantics for monitored
programs. This semantics is only defined when the MVA compatible with
the program. An MVA \(a\) is said to be compatible with program \(P\),
denoted \(a \sqsubseteq P\) if
\begin{inparaenum}[(\itshape i\upshape)]
\item the program contains the class \(c\) that is being monitored,
\item all variables declared as program variables in
\(a\) are fields of the class \(c\) with the correct type, and
\item every event name corresponds to a method in the class.
\end{inparaenum}
A monitored program is a product of the MVA and the program. The state
of a monitored program consists of the states of the MVA and the
program, and a flag \stuck. If the MVA is partial, the flag
\stuck is set when \(\Delta_a\) is not defined for a
certain input. If the flag is set, this means that the security policy
is violated, and the program should be stopped. If the MVA is total,
the \stuck flag will never be set. Instead, violation of the security
policy is modelled by the MVA reaching the error state \halted.


\[
\begin{array}{rcl}
\Mprogram & =  & \opr \mva : \MVA, \program : \Program \clr\\
\Mstate & = & \opr \mvastate : \MVAstate, \progstate : \Pstate, \stuck
: \mathbb{B} \clr
\end{array}
\]
Thus, \FullProgram gets instantiated as \Mprogram and \FullState as
\Mstate, with mappings \program and \progstate. Now we can give
appropriate instantiations for the \(\gamma\)- and
\(\delta\)-relations. In fact, the \(\delta\)-relations are the same as
for the standard or annotated program semantics (depending on whether
we wish the monitored program to evaluate annotations that are
possibly already there). Here we show how \gammain changes the state
of the monitor, using \(\hat\gammain\) for the necessary checks at
method entrance in the program itself (\emph{i.e}, the evaluation of
the method specification).

\[
\begin{array}{c}
\hat\gammain(P, \md, r, \act, \sigma_1, \tau) \\
\pif{\ex(\pstate(\tau)) = \bot}
    {\sigma_2 = \delta(\entry)(P, \md, \act, \tau)}
    {\sigma_2 = \tau}\\
\hline
\gammain(P, \md, r, \act, \sigma_1, \sigma_2)
\end{array}
\]

\noindent where
\[
\begin{array}{l}
\delta(\ev)(P, \md, \act, \sigma) =\\
\qquad
\begin{array}[t]{l}
\mathsf{let\ }
\begin{array}[t]{rcl}
  e & = & \opri \etype := \ev, \mname := \md.\name \clri\\
 \tau & = & \Delta_a(\mvastate(\sigma), \progstate(\sigma), e, \act)
\end{array}
\mathsf{in\ }\\
\pif{\sigma.\stuck \vee \tau = \bot}
    {\sigma (\stuck := \ttt)}
    {\sigma (\mvastate := \tau)}
\end{array}
\end{array}
\]


\paragraph{Example}

To be able to generate the annotations capturing the security policy
specified by the LimitSMS automaton in Figure~\ref{FigExample}, below
is a class declaration of class \texttt{Messaging} declaring the
methods used by the automaton, plus a method \texttt{receiveSMS}, and
a field \texttt{counter}. To illustrate our annotation algorithm, we
do not need to give an implementation here, it is only important to
know that the methods actually have an implementation.
\begin{figure}[t]
\begin{verbatim}
class Messaging {
  int counter;
  void sendSMS(){ /* body sendSMS  */}
  void receiveSMS(){... }
  void reset() { /* body reset */}
}
\end{verbatim}
\caption{Implementation of class \texttt{Messaging}}\label{FigExampleImplem}
\end{figure}
%\subsubsection{PVS formalisation}
%As mentioned above, the syntax and semantics for MVA and programming
%language have been formalised in PVS.

% To formalise the program semantics in PVS is not
%completely straightforward. PVS does not allow one to define mutually
%recursive definitions, and moreover for every recursive method call,
%termination has to be shown. As a solution, we define a function
%\texttt{deriv\_expr} parametrised with
%natural number \(n\) and function \texttt{deriv\_stmt}, where the
%argument \texttt{deriv\_stmt} is only applicable to natural numbers
%strictly less than \(n\). The number \(n\) provides an upperbound on
%the number of recursive method calls that are going to be made.  The
%function \texttt{deriv\_stmt}, applied to natural number \(m\) can
%then directly call the \texttt{deriv\_expr} function, with
%\texttt{deriv\_stmt} as argument, but restricted to calls with arguments
%strictly less than \(m\). The parameters \gammain, \gammanorm
%\emph{etc.} are all parametrised with \texttt{deriv\_expr} and
%\texttt{deriv\_stmt}, but again restricted to be applicable only for
%strictly smaller natural numbers. We can then prove that the natural
%number argument is always strictly decreasing, and this is sufficient
%to guarantee termination.







