
\section{Programs and Semantics}\label{SecProgram}


\begin{figure}[t]
\[
\begin{array}{rcl}
\Type & = & \Bool \mid \Int \mid \Ref \mid \Void\\
\NumExpr & = & \I(i : \mathbb{I}) \mid \Plus(n_1, n_2 : \NumExpr) \mid \ldots \mid 
               \EvalN(n : \Name) \\
\BoolExpr & = & \ttt \mid \fff \mid \Not(b : \BoolExpr) \mid \Conj(b_1,
b_2 : \BoolExpr) \mid \ldots \mid \\
          &   & \GT(n_1, n_2 : \NumExpr) \mid \EvalB(n
: \Name) \\
\RefExpr & = & \Null \mid \EvalR(n : \Name)\\
\Val & = & \B(b : \mathbb{B}) \mid \I(i : \mathbb{I}) \mid \Null \mid
\R(i : \mathbb{I}) \mid \One \mid \bot\\
\end{array}
\]
\caption{Variable types, numerical, boolean and reference expressions
and values}
\label{FigProgramBase}
\end{figure}

\begin{figure}[t]
\[
\begin{array}{rcl}
\Expr & = & \BExpr(b : \BoolExpr) \mid
            \NExpr(n : \NumExpr) \mid
            \RExpr(r : \RefExpr) \mid\\
      &   & \CondExpr(c : \BoolExpr, e_1, e_2 : \Expr) \mid
            \Assign(n : \Name, e : \Expr) \mid \\
      &   & \Call(o : \Expr, n : \Name, p : \Expr) \mid
            \EvalG(n : \Name) \smallskip\\
\Stmt & = & \Skip \mid
            \Sequence(s_1, s_2 : \Stmt) \mid
            \IfThenElse(c : \BoolExpr, s_1, s_2 : \Stmt) \mid\\
      &   & \While(c : \BoolExpr, s : \Stmt) \mid
            \StmtExpr(e : \Expr) \mid
            \Throw \mid\\
      &   & \TryCatch(t, c, f : \Stmt) \mid
            \Set(n : \Name, e : \Expr) \mid\\
      &   & \CaseJML(b : \listof{\BoolExpr \times \Stmt})
\end{array}
\]
\caption{Abstract syntax of expressions and
statements}\label{FigExprStmt}
\end{figure}

\begin{figure}[t]
\[
\begin{array}{rcl}
\FieldDecl & = \opr & \type : \Type, \name : \Name, \init : \Val \clr\\
\ArgDecl & = \opr & \type : \Type, \name : \Name \clr\\
\LocalVarDecl & = \opr & \type : \Type, \name : \Name \clr\\
\GhostVarDecl & = \opr & \type : \Type, \name : \Name, \init : \Val \clr\\
\end{array}
\]
\caption{Different types for variable declarations}\label{FigDeclarations}
\end{figure}

\begin{figure}[t]
\[
\begin{array}{rcl}
\Method & = \opr & \name : \Name, 
                   \param : \ArgDecl, 
                   \lvars : \setof{\LocalVarDecl},
                   \body : \Stmt,\\
        &        & \res : \Expr,
                   \restype : \Type,
                   \pre : \Expr,
                   \post : \Expr, \\
        &        & \preset : \Stmt,
                   \postset : \Stmt,
                   \excset : \Stmt \clr \\
\Class & = \opr & \name : \Name,
                  \super : \Name_{\bot},
                  \fields : \setof{\FieldDecl},
                  \methods : \setof{\Method},\\
        &       & \inv : \Expr,
                  \ghostvars : \setof{\GhostVarDecl} \clr\\
\Program & = \opr & \classes : \setof{\Class} \clr
\end{array}
\]
\caption{Abstract Syntax for Programs}\label{FigProgram}
\end{figure}

\subsection{Program Syntax}\label{SecSyntax}

Before defining the format of automata that we use to express security
properties and the annotation generation algorithm, we first formally
define the abstract syntax of programs. We will use a PVS-like
notation to declare abstract data types and records. Let \(\Name\) be
an infinite, set of names, containing the special names \textsf{this}
and \textsf{super}. Figure~\ref{FigProgramBase} defines the variable
types, numerical, boolean and reference expressions. It also defines
the type of values that a variable can contain. Notice that we have
special type \(\Void\), inhabited by \(\One\) to model methods without
results. We use \(\mathbb{B}\) and \(\mathbb{I}\) to denote the
standard sets of booleans and integers, respectively. Reference values
contain a number, which can be considered as the location where the
object is stored; \(\bot\) is used to denote the outcome of an
undefined expression. 

Next, Figure~\ref{FigExprStmt} defines the abstract data types for
program expressions and statements. Notice that we have a few language
constructs that are used to represent JML annotations (\EvalG, \(\Set\) and
\(\CaseJML\), to lookup and update ghost variables, respectively). 
The standard program semantics ignores these statements, but the
annotated program semantics evaluates these, and updates the store of
the specification accordingly. The language that we consider is a
restricted subset of (sequential) Java. In particular, we consider
that there are only a few exceptions, and that methods have
only one parameter. However, we believe that it contains all
constructs that are relevant for proving correctness of our
approach. Implementing the technique for the full language is only an
engineering issue (see Section~\ref{SecImplem}).

Figure~\ref{FigDeclarations} defines the different kinds of variable
declarations that can occur in a program. Notice that field and ghost
variable declarations are explicitly initialised, while parameters and
local variables are not. Ghost variables are one of the JML concepts
that occur in our program syntax. They will be used to represent the
control behaviour of the monitoring automaton. The \(\Set\) statement
is used to update ghost variables. The \(\CaseJML\) statement
abbreviates a list of conditional ghost variable updates.  Further,
methods can be annotated with pre- and postconditions, and classes
with invariants. To support our annotation generation algorithm, we
also specify for each method special annotations, called
\(\preset\),
\(\postset\) and \(\excset\). These annotations describe the updates
to the ghost variables at method entry, exit and exceptional exit,
respectively. The annotation generation algorithm proceeds in two
steps: first we generate the special pre- and post-set annotations,
using the \(\CaseJML\) construct, then we translate this further into
``standard'' JML. Finally, Figure~\ref{FigProgram} describes the
syntax for methods, classes and programs. To ensure every method has
an appropriate return expression, this expression is part of the
method signature.

\subsection{Natural Semantics}\label{SecSemantics}
The behaviour of a program is described using a big step semantics. We
basically follow Von Oheimb's formalisation of Java~\cite{vOheimb?},
with simplifications wherever possible because of our simplified
program syntax. However, in our setting we use both the semantics of
an (annotated) program and of a monitored program (as defined in the
next section). To be able to do this in a uniform way, we modularise
the evaluation rules with functions that describe how the annotations
or monitoring automaton is handled at a particular point. This allows
us to define standard program semantics, runtime annotation evaluation
semantics and monitoring program semantics by varying only a small set
of functions.

We use the following notations: 
$\stp{c,\sigma}{\sigma'}$  denotes that execution of statement $c$
transforms state $s$ into $s'$ in the context of program P, while
$\etp{c,\sigma}{v,\sigma'}$  denotes that expression $e$ evaluates to
$v$, while transforming state $s$ into $s'$ in the context of program
\(P\). A basic program state \(\Pstate\) is composed of an optional
exception and a store. We only distinguish a few basic exceptions: 
\(\Excp = \Throwable \mid \NullPointer\). The store maps every field
and local variable to a value.

\[
\begin{array}{rcl}
\Pstate & = \opr & \ex : \Excp_{\bot}, \st : \PStore \clr\\
\PStore & = \opr & \fvs : \Name \mapsto \Val, \lvs : \Name \mapsto \Val \clr
\end{array}
\]

The program state of an annotated program is extended with a store for
ghost variables:
\[
\Astate = \opr \pstate : \Pstate, \gvs : \Name \mapsto \Val \clr
\]

The evaluation rules assume that we have a program \(p\) of type
\(\FullProgram\), and a state \(\sigma\) of type
\(\FullState\). Further, they assume that there are functions
\(\program\) and \(\progstate\) that return the basic program, and the
basic program state, respectively. These types and functions are
instantiated differently for each semantics. 

\begin{figure}[t]
\[
\begin{array}{c}
\begin{array}{lc}
\br{\Call}     &
\begin{array}{c}
\Norm(\progstate(\sigma_0))\\
\etp{o, \sigma_0}{r,\tau_1}\\
\etp{p, \tau1}{\act, \tau_2}\\
\tau_3 =  \pif{\exc(\progstate(\tau_2)) \neg= \bot \wedge r = \Null}
              {\tau_2 \with \progstate.\ex := \NullPointer}{\tau_2}\\
\md = lookup\_mthd(P, r, \mn, \act)\\
\gammain(P, \md, r, \tau_3, \tau_4) \\
\oldval = \tau_4.\progstate.\st.\lvs (\md.\param.\name)\\
\tau_5 = \tau_4 \with \progstate.\st.\lvs := (\md.\param \mapsto \act) \\
%\_3 = \sigma_2(lvs:=init\_vars(\md.lvars[this \mapsto a][\md.pn \mapsto pv]))\\
\stp{\md.body, \tau_5}{\tau_6}\\
\etp{\md.\res, \tau_6}{v,\tau_7}\\
\pif{\exc(\progstate(\tau_6)) = \bot}{\gammanorm(P, \md, r, \tau_7,
\tau_8)}{\gammaexc(P, \md, r, \tau_7, \tau_8)}\\
\hline
\etp{\Call(o,\mn,p), \Norm(\sigma_0)}{v, \tau_8 \with
\progstate.\st.\lvs := \oldval}
\end{array}
\end{array}
\medskip\\
\begin{array}{ccc}
\begin{array}{lc}
\br{\Set}     &
\begin{array}{c}
\Norm(\progstate(\sigma_0))\\
\deltaset(P, \sigma_0, n, e, \sigma_1)\\
\hline
\stp{\Set(n, e), \sigma_0}{\sigma_1}
\end{array}
\end{array}
&\ \ &
\begin{array}{lc}
\br{\CaseJML}     &
\begin{array}{c}
\Norm(\progstate(\sigma_0))\\
\deltacase(P, \sigma_0, b, \sigma_1)\\
\hline
\stp{\CaseJML(b), \sigma_0}{\sigma_1}
\end{array}
\end{array}
\end{array}

\end{array}
\]
\caption{Evaluation rules for method calls, \(\Set\) and \CaseJML statements}\label{FigEvalRules}
\end{figure}

When statements or expressions are evaluated, uncaught exceptions are
propagated. Exceptions can only by caught explicitly within a
\(\TryCatch\) statement. The evaluation rules for all statements
explicitly require that the initial state is not exceptional. We use
\(\Norm(\sigma)\) as abbreviation for basic program states of the form
\((\# \ex :=  \bot, \st := \sigma \#)\). Most of the evaluation rules
are standard, and we do not give them explicitly here, but instead
refer to Von Oheimb's thesis. The only exception to this are the rules
that involve annotations, \emph{i.e.,} the rules for method call,
looking up and updating ghost variables. Figure~\ref{FigEvalRules}
shows the adapted rule for method calls. The functions \gammain, \gammanorm,
\gammaexc specify what happens upn method entry, normal termination
of a method and exceptional termination of a method. The rules for
looking up and updating ghost variables (\EvalG, \Set and \CaseJML
completely rely on functions \deltaeval, \deltaset and \deltacase,
that are appropriately instantiated for each kind of semantics.

To evaluate a program without
considering the specifications, 
\(\FullProgram\) and \(\Program\) and
\(\FullState\) and \(\Pstate\) are identical, thus \program and
\progstate are identify functions. Also the specification evaluation
functions are instantiated as identity relations, \emph{i.e.,}
\(\gammain(P, \md, r, \tau_1, \tau_2) = \gammanorm(\ldots) = \gammaexc(\ldots) = (\tau_1 = \tau_2)\), and \(\deltaeval(P, \tau_1,
\tau_2, \ldots) = \deltaset(\ldots) = \deltacase(\ldots) = (\tau_1 =
\tau_2)\). 

Figure~\ref{FigAnnotatedSem} shows the instantiation of the semantics
for a runtime annotation evaluation semantics.

\begin{figure}[t]
\[
\begin{array}{rcl}
\FullProgram & = & \Program\\
\FullState & = & \Astate\\
\program & = & \id\\
\progstate & = & \pstate\smallskip\\

\gammain(P, \md, r, \tau_1, \tau_2) & = & \\
\gammanorm(P, \md, r, \tau_1, \tau_2) & = & \\
\gammaexc(P, \md, r, \tau_1, \tau_2) & = & \smallskip\\

\deltaeval(P, \EvalG(n), \tau_1, v, \tau_2) & = & \\
\deltaset(P, \Set(e, n), \tau_1, \tau_2) & = & \\
\deltacase(P, \CaseJML(b), \tau_1, \tau_2) & = & \\
\end{array}
\]
\caption{Instantation of semantics for runtime annotation evaluation}
\label{FigAnnotatedSem}
\end{figure}

\subsubsection{PVS formalisation}
To formalise the program semantics in PVS is not completely
straightforward. PVS does not allow one to define mutually recursive
definitions, and moreover for every recursive method call, termination
has to be shown. As a solution, we defined a function
\texttt{deriv\_expr} parametrised with a
natural number \(n\), and a function \texttt{deriv\_stmt}, where the
argument \texttt{deriv\_stmt} is only applicable to natural numbers
strictly less than \(n\). The number \(n\) provides an upperbound on
the number of recursive method calls that are going to be made.  The
function \texttt{deriv\_stmt}, applied to natural number \(m\) can
then directly call the \texttt{deriv\_expr} function, with
\texttt{deriv\_stmt} as argument, but restricted to calls to numbers
strictly less than \(m\). 
The parameters \gammain, \gammanorm \emph{etc.} are all parametrised
with \texttt{deriv\_expr} and \texttt{deriv\_stmt}, but again
restricted to be applicable only for strictly smaller natural
numbers. We can then prove for both functions the natural number
argument is going to decrease strictly, and this is sufficient to
guarantee termination.






