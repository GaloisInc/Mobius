\section{Conclusions \& Future Work }\label{SecConcl}

This paper presents an algorithm to inline security automata, in the
form of JML annotations. The translation is defined in several
steps. All steps are formalised and proven correct, using the PVS
theorem prover. The algorithm might seem trivial, but several
subtleties complicated the correctness proof.

% First of all,
% for the main step, where the monitor is encoded in method-level set
% annotations, it is not straightforward to state the correct relation
% that is maintained between the two program executions. In particular,
% once the \halted state is reached, the correspondence between the two
% program states is not preserved anymore, because the monitored program
% might continue executing, while the annotated program throws an
% exception. Also, to prove correctness of this translation step, it was
% important to specify the evaluation of pre- and postcondition, class
% invariant and pre- and post-set annotations in the right order, to ensure
% that the relation was reestablished as quickly as possible~--~so that
% the induction hypothesis could be applied for the other
% steps. Adding the additional assert to the pre-set annotation was also
% crucial for correctness.
% Finally, a last important issue here was the behaviour of the
% \TryCatch statement, where exceptions in the \emph{finally} block can
% overwrite other exceptions.

The formalisation has been developed for a subset of Java. We believe
that extending it to full (sequential) Java would be relatively
straightforward. However, generalising to properties that are not
restricted to a single class or to multithreading might be more
challenging.

The ultimate goal of our work is to statically verify adherence to the
security policy. To achieve this, a weakest precondition calculus can
be used to generate pre- and postconditions, based on the generated
\Set annotations. In earlier work, we presented such a
propagation algorithm~\cite{PavlovaBBHL04}, and proved correctness for
a limited case (instance variables and branches are not
considered). It is future work to overcome these limitations.
%  We will then
% also exploit the existing PVS formalisation, to prove the formal
% correctness of the propagation with a theorem prover.
