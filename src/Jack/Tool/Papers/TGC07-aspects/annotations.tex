The weakest precondition calculus we are targetting relies on program and aspects specifications that are
precise enough to make the verification conditions generated provable.
The annotation language we use is like Pipa \cite{ZhaoR03} (the extension
of JML for Aspects), but in a limited form. The annotation language is a JML level 0 type language 
\cite{Leavens-etal07}, to which
we remove the universe types, but to which we add model methods as well as purity, and some constructs
to handle aspects found in Pipa. In our annotations we also allow behavioural specifications
because JML's can be desugared as shown in \cite{RaghavanL00} and Pipa's specific constructs
 as how we show below.

\subsubsection{JML} JML is a behavioural specification language for Java. It is made of numerous keyword: you 
can express methods pre and post conditions with it (the keywords {\tt requires} and {\tt ensures}), exceptional
postcondition ({\tt exsures}). There
are notions of frame conditions (the {\tt assignable} clauses). One can also express assertions ({\tt assert} 
construct), and loop invariants.

\paragraph{Invasive model methods} 
Model methods are an interesting construct of JML: these are specification
methods that can be with or without side-effects, their effects being determined by their specifications.
They can have a body, but in our framework we are interested by bodyless model methods. They are used to 
assume an effect over the program, and so state that if their requires are satisfied, the given effect (given by
the ensure clause) will be satisfied.
In this paper we won't talk about pure model methods which could be use in specifications. Here we rather talk
about invasive ones: just like invasive aspects they are meant to state that the program has been modified 
non explicitly.
\begin{figure}
\begin{center}\begin{minipage}{3cm}
\bcode
ghost int i;\\
...\\
requires i > 0;\\
assignable i;\\
ensures i == 1;\\
model void m();
\ecode
\end{minipage}\end{center}
\caption{A model method definition}
\label{model_meth_def}
\end{figure}


\paragraph{Model methods call}
To be able to use model methods with the full expressivity desired, we add a new annotation to JML, the simple 
annotation method call. Its syntax would be simply the model method call in the middle of annotations as shown 
on figure \ref{model_meth}. This construct is natural from a JML point of view because for most of the non 
exceptional cases  it could be desugared easily into 
more primitive JML. Hence a normal model call could be desugared into 3 construct (it is similar to what is done
in guarded commands \cite{BarnettL05}):
\begin{itemize} 
\item an assert of the require clause, followed by 
\item an  instruction to state that all the variable assignable have new values: it can be represented by JML 
keyword {\tt fresh} which is usually used for method specifications, but here could be used to annotate the method, 
and finally 
\item the assume of the method's postcondition.
\end{itemize}
Nevertheless, the exceptional cases being difficult to desugar, it is interesting to have the model method calls in
JML specification.

\begin{figure}
\begin{center}\begin{minipage}{4cm}
The program code:
\bcode
...\\
//@ m();\\
//@ assert(i == 1);\\
...
\ecode
Could be desugared to:
\bcode
...\\
//@ assert (i > 0);\\
//@ fresh(i);\\
//@ assume (i == 1);\\
//@ assert(i == 1);\\
...
\ecode
\end{minipage}\end{center}
\caption{A non exceptional model method call desugared}
\label{model_meth}
\end{figure}
\subsubsection{Pipa}
Pipa is a behavioural specification language based on JML. 
You can specify advices as well as pointcuts~\cite{pointcuts07} with Pipa.
In this paper we will only treat the advices specifications. 
The specification of an advice is similar to the one of a method call. It has a pre and post condition,
an exceptional postcondition and a frame condition. The specification are not different for JML's method's 
specification, being applied on advices.
Pipa mainly adds two construct plus a specific semantic we are interested in:
the specific constructs for around advices, and their semantic of the ensure clause. These construct will have
to be desugared in case of behavioural specifications.



\paragraph{Around advice specifications} The difficulty of specification of around advices is due to the presence
of the proceed construct. It has to be stated in the method specification and in JML there are no keywords to
represent it. To represent the proceed in specification, Pipa uses the predicate {\tt proceed}
as well as the {\tt then} construct as what was stated in Clifton and Leavens paper~\cite{clifton02spectators}.
The specification of an around advice is divided in two the part before the proceeds (the part before the 
execution of the target instruction) and the part after the proceeds. These two parts have both pre and post 
conditions as well as a frame condition. There is also for the first part a new keyword proceeds which take a 
boolean condition which is an invariant for an around advice which determined if the instruction must be executed.

\begin{figure}
\begin{center}\begin{minipage}{3cm}\bcode
int f;\\
...\\
/*\=@ \ \ \=requires f > 0;\+ \\
@ \>ensures f == 1;\\
@ \>proceeds true;\\
@ then\\
@ \> requires f == 2;\\
@ \> ensures f == 3;\\
@*/\-\\
void \= around() : call (void m()) \{\\
\>...\\
\} \ecode\end{minipage}\end{center}
\caption{Around annotations}
\label{arround_annot}
\end{figure}

\paragraph{Desugaring around advice specifications}
The desugaring of the proceed construct is very easy: it is transformed into the predicate {\tt proceed(bool)} 
and conjuncted to the ensures specifications of the part before the then of the given case. 
If the proceed was omitted it is added in the ensures as {\tt proceed(false)}.

Each around advice behaviour is desugared arount the {\tt then} construct: each part of each 
behaviour taking place before the {\tt then} construct are desugared with each other as 
stated for standard JML behaviour in~\cite{clifton02spectators}.
And the part after the {\tt then} construct is given the same treatment.

\paragraph{Semantic of the ensure clause}
The other interesting element in Pipa is the semantic of the ensure clause which slightly differs from the one in
JML. It enables to express properties about the caller variables which were passed to the advice while 
weaving. For instance if you have a variable {\tt x} inside a method, 
and the advice is weaved on a method call that uses {\tt x}, you can express properties over {\tt x} in the 
postcondition. In normal JML it makes no sense to express a property like that, and if {\tt x} is mentionned
in the postcondition, it would have been the old value of {\tt x}. It is especially useful to express real 
invasive aspects. In our framework we will use this semantic for {\it any} postcondition, {\it e.g.} 
for normal method calls as well.
The desugaring is not changed by this peculiar semantic.