In this paper we have described a verification framework 
for taking into account meta-programming concern in the verification.
These concerns were modelled using AspectJ aspects specified with Pipa.
The verification was made using a standard guarded command language, and
an abstraction of aspects as model classes.
The concerns were by choice invasive, so no real modular verification was
possible.

A great part of the framework is meant to exist as part of the 
Mobius PVE~\cite{MobiusPVE07}, like the translation of annotated bytecode
to BoogiePL~\cite{javatrans07}, or the weakest precondition calculus
over a BoogiePL-like language. But there is lots of implementation
to be done still: JML should be extended with the Pipa constructs 
(to our knowledge no implementation of Pipa exists), the abstraction
from advices to models has to be implemented, and the weaving too.

An interesting difficulty arises when using this framework: when a user first
specify the base program, his specifications are not complete enough to enable 
verification. There should be a way to enrich automatically these 
specifications, maybe by using specification weaving like what is presented
in Moxa~\cite{moxa05}.