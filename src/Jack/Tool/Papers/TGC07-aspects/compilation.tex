The next step once the program is fully annotated and the specification desugared is the compilation of the
advices to their effects on the program solely, something  which can be expressed by simple bodyless model 
methods.
This transformation is direct, but the specifications of the advices have to be
 enriched with the pointcuts conditions that cannot be determined syntactically.
This enrichment is to make the weaving of the models simpler.

\subsubsection{Translation of the specifications}
The annotations are translated directly to the model methods. The method has the same signature
 of the corresponding advice, and the specifications are copied entirely to the model method. It works smoothly
for most of the cases except for the around advice which is split into 2 model methods: the one corresponding
to the first part, and the one corresponding to the part after the {\tt then} construction. 
The {\tt proceed} instruction has to be translated to JML as well: it becomes a global 
ghost variable of type bool.
For an around advice the model method corresponding to the first part has to be modified consequently.
The ensures has the equality {\tt (proceed == proc\_cond)} where {\tt proc\_cond} was the argument 
to the annotation {\tt proceed}. The assignable clause is also modified: the {\tt proceed} global variable is 
added to it.
\subsubsection{Translation of pointcuts}
The pointcuts are not directly translated to the model methods, though the conditions over the pointcut undecidable
at compile time are added to the require clause.
