In this paper we focus on a verification framework adapted for the
verification of AspectJ programs annotated with Pipa.  The pointcut
semantic has been properly
defined~\cite{DBLP:conf/popl/AvgustinovHOMSTV07} though the advice
weaving semantic is still partly missing~\cite{weaving06}.  Pipa is an
annotation language which was inspired by Clifton and Leavens'
work~\cite{clifton02observers}. There are some extensions to it like
pointcuts annotations~\cite{pointcuts07}, or Moxa \cite{moxa05}.


\paragraph{Modular verification}
Verification in Aspect Oriented Programming language is often linked
to a modular approach.  It is orthogonal to the aim of this paperh: we
tailor verification of aspects which were not specially wanted by the
user, and are most-likely imposed by the environment.  Clifton and
Leavens~\cite{clifton02observers,clifton02spectators,cliftonPhd}
propose a programming discipline to define and specify aspects, using
an extension of JML as specification language.  They define two kind
of aspects, the \emph{spectators} and \emph{assistants} and propose to
explicitly allows the weaving of these aspects. This approach allows
efficient modular reasonning and easy implementations because they
show a way of weaving specifications using a control-flow graph
analysis. In their 2004 work~\cite{shriram04} Krishnamurthi {\it et
al} present a model-checking framework for verification of programs
containing aspects. It is quite different from our work as it presents
a generic framework for aspects. In an unpublished paper~\cite{cesar}
Kunz presents a Hoare logic for modular verification of aspects. His
paper is quite formal, but his approach remains less modular for the
specification around advices, especially for the {\tt proceeds}.

\paragraph{Verification of Java programs annotated with JML}
 

% Shmuel Katz et al. \cite{Katz06,GoldmanK06} propose
% a classification of aspects as \emph{spectative}, \emph{regulative} or
% \emph{invasive}, to simplify program verification by focusing on the







