\paragraph*{Non-interference:}
Non-interference of advices code with respect to the base code behavior is a
common property of practical AOP examples. 
Dantas and Walker \cite{DantasW06} define the notion of
\emph{harmless advice}, as advices that may only interfere by performing extra
I/O or by preventing termination.
They propose an information-flow type system over a core AOP language
\cite{WalkerZL03} to check harmlessness with respect to the main program.
Clifton and Leavens \cite{clifton02spectators} classify aspects as
\emph{spectators} or \emph{assistants}; the former include aspects that only
modify the state space they own and do not alter the control flow.
On the other hand, \emph{assistants} can interfere with the original behavior of
the program but only if \emph{explicitly accepted} by the original
program. 
Shmuel Katz et al. \cite{Katz06,GoldmanK06} propose a classification of
aspects as \emph{spectative}, \emph{regulative} or \emph{invasive}, to simplify
program verification by focusing on the properties that may be affected by the
introduction of an aspect. 
More concretely it model-checks a state machine representation of
the aspect merged with the representation of the underlying program,
analyzing how already valid properties are influenced.
Following the result of this analysis, a concrete static procedure to classify
advices is proposed. 

\paragraph*{JML-based verification:}
\alarm{Talk about JML}
Pipa \cite{ZhaoR03} is proposed as an extension to JML
\cite{Leavens-etal07} for AspectJ \cite{AspectJ06}, 
to support specification for aspects invariants, pre- and post-conditions for
advices and variable introductions.
The main motivation is to transfer the application of current tools for Java
programs to the AspectJ language, by extending an AspectJ to Java compiler
with a simultaneous translation of a Pipa specification into a standard JML
specification.

\alarm{Talk about Cesar Work}



