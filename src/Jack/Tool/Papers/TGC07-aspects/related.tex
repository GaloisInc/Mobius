The verification framework is adapted for the verification of AspectJ
programs annotated with Pipa.  The pointcut semantic of AspectJ has
been properly defined~\cite{DBLP:conf/popl/AvgustinovHOMSTV07} as well
as the advice weaving semantic~\cite{weaving04} though only parts have
been formalized~\cite{weaving06}.  Pipa~\cite{ZhaoR03} is an
annotation language which was inspired by Clifton and Leavens'
work~\cite{clifton02observers}. There are some extensions to it like
pointcuts annotations~\cite{pointcuts07}, or Moxa \cite{moxa05}.  Our
framework relies on verification based on an intermediate language,
like what is done in ESC/Java2~\cite{FlanaganLLNSS02},
Boogie~\cite{BarnettCDJL05}, or Krakatoa~\cite{MarcheP-MU04}. The work
is made to be adapted on a BoogiePL-like guarded command which is
coupled with a VCGen~\cite{BarnettL05,FlanaganS01}.

\paragraph{Modular verification of Aspects}
Verification in Aspect Oriented Programming language is often linked
to a modular approach.  It is orthogonal to the aim of this paperh: we
tailor verification of aspects which were not specially wanted by the
user, and are most-likely imposed by the environment.  Clifton and
Leavens~\cite{clifton02observers,clifton02spectators,cliftonPhd}
propose a programming discipline to define and specify aspects, using
an extension of JML as specification language.  They define two kind
of aspects, the \emph{spectators} and \emph{assistants} and propose to
explicitly allows the weaving of these aspects. This approach allows
efficient modular reasonning and easy implementations because they
show a way of weaving specifications using a control-flow graph
analysis. In their 2004 work~\cite{shriram04} Krishnamurthi {\it et
al} present a model-checking framework for verification of programs
containing aspects. It is quite different from our work as it presents
a generic framework for aspects. In an unpublished paper~\cite{cesar},
Kunz presents a Hoare logic for modular verification of aspects. The
paper is quite formal, but he does not add special annotations for method
specifications as Clifton and Leavens. For this reason his approach remains 
less modular than Clifton's, but more expressive.




% Shmuel Katz et al. \cite{Katz06,GoldmanK06} propose
% a classification of aspects as \emph{spectative}, \emph{regulative} or
% \emph{invasive}, to simplify program verification by focusing on the







