The Security Manager is an important system component in the Java
Virtual Machine (JVM).  It forbids access to some specific methods
depending on a given security policy.  It modifies both the behavior
of the Java system libraries and of the JVM, for it throws exceptions
in case of security policy violation.  The difficulty for the
verification of a environment containing it lies in the fact that its
use mixes library calls (easy to specify) as well as direct VM calls
(hard to take in account because they modify directly the Java
semantic of the execution). 

In this article we use a modelisation of a SecurityManager
seen as an aspect component, and we define a verification framework for
Pipa and AspectJ to take in account the presence of this aspect
component in the verification of programs.