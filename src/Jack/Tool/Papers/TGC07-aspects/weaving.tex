
First all the pointcut are abstracted as a syntactical pointcut. Then they are ordered using the compilation
order.
They are ordered in a list made of couples (instr, (model\_list)) where model\_list is the list of model methods 
that correspond to a specific syntactical match.

%
\begin{figure}
\bcode
weave \=(instr; Linstr, (instr, (model :: model\_list)) :: Lmatch) -> \\
\>model:: weave (instr; Linstr, (instr, (model\_list)) :: Lmatch);\\
weave \=(instr; Linstr, (instr, (instr :: model\_list)) :: Lmatch) -> \\
\>instr:: weave (instr; Linstr, (instr, (model\_list)) :: Lmatch);\\
weave (instr:: Linstr, (instr', (model :: model\_list)) :: Lmatch) -> \\
\>instr <> instr',  weave (instr; Linstr, Lmatch);\\
weave (instr:: Linstr, (instr, (nil)) :: Lmatch) -> \\
\>weave (Linstr, Lmatch);\\
weave (instr:: Linstr, nil) -> \\
\>instr :: weave (Linstr, Lmatch).
\ecode
\caption{The algorithm to weave method}
\label{weaving_algo}
\end{figure}
Then for each program point that matches