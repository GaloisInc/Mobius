At this stage of the transformation, all the program source is turned
into guarded commands. On one side we have the main program part 
which still does not contain any aspects. On the other side
we have the related to aspects elements:
\begin{enumerate}
\item 
we have model methods which represents the behaviour of before and
after advices, model methods that represent the first part of around
advices, model that represent the second part and,
\item
methods representing the advices, and for around advices, proceeds 
model methods.
\end{enumerate}

Now the weaving can be done over the implementation part of the
guarded commands program. The semantic of the weaving we base our work
upon has been presented in two papers \cite{weaving04,weaving06}.  The
more formal of the two~\cite{weaving06} lacking a proper semantic for
around advices.

We weave in the code the model methods representing the advices.  For
before and after advices we use exactly the weaving semantic presented
in Belblidia and Debbabi's paper.
In our case the four rules become:

\fixme{write nicely the four rules}.

Since the around advice is not treated in~\cite{weaving06} and is 
mentionned vaguely in~\cite{weaving04}, we derive from the later
a generic weaving method, adapted to the way they were specified.

\fixme{write the stupid weaving}

\fixme{prove the correctness by reduction}






%
