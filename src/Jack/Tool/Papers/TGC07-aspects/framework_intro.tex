The verification framework we are building in this paper 
is an adaptation of static 
verification based on weakest precondition calculus for aspects. 
Here we will consider 
the SecurityManager as an invasive aspect, 
so we must take into the weaving process. 

The verification is made of several steps:
\begin{enumerate}
\item the program and it's aspects are annotated using Pipa,
\item the behavioural specifications are desugared
\item the specifications of the aspects are compiled into model methods
\item the model methods are weaved to the program
\item a weakest precondition calculus is made on the transformed program, and verification
conditions are generated, and finally,
\item the verification conditions can be discharged automatically or interactively.
\end{enumerate}
This verification procedure is indeed made of 3 main parts: the annotation
part (step 1 and 2), the compilation and program transformation part (step 3 and 4)
and the verification part (steps 5 and 6).
The crucial part here being the second, since the third can be done with any verification tool
supporting static verification with Java and JML, and part 1 is only simple desugaring, and has been
extensively described in \cite{RaghavanL00}.