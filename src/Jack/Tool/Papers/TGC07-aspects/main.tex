% This is LLNCS.DEM the demonstration file of
% the LaTeX macro package from Springer-Verlag
% for Lecture Notes in Computer Science,
% version 2.3 for LaTeX2e
%
\documentclass[draft]{llncs}
%
\usepackage{makeidx}  % allows for indexgeneration
\usepackage[inline, nomargin]{fixme}
\usepackage{listings}
\usepackage{float}
\usepackage{macros}
%
\begin{document}

\newcommand{\rarrow}{$\rightarrow$}
\newcommand{\conj}{$\wedge$}
\newcommand{\disjonc}{$\vee$}
\newcommand{\s}{\,}
\newcommand{\btab}{\begin{tt}\begin{tabbing}}
\newcommand{\etab}{\end{tabbing}\end{tt}}
\newcommand{\bcode}{\begin{tt}\begin{small}\begin{tabbing}}
\newcommand{\ecode}{\end{tabbing}\end{small}\end{tt}}
%
\frontmatter          % for the preliminaries
%

\mainmatter              % start of the contributions
%
\title{Taking into account Java's Security Manager for static verification}
%
\titlerunning{Taking in account Java's Security Manager}  % abbreviated title (for running head)
%                                     also used for the TOC unless
%                                     \toctitle is used
%
\author{Julien Charles \and C\'esar Kunz}
%
\authorrunning{Julien Charles and C\'esar Kunz}   % abbreviated author list (for running head)
%
%%%% list of authors for the TOC (use if author list has to be modified)
\tocauthor{Julien Charles, C\'esar Kunz}
%
\institute{INRIA Sophia-Antipolis\\
\email{\{julien.charles, cesar.kunz\}@inria.fr}}

\maketitle              % typeset the title of the contribution

\begin{abstract}
The Security Manager is an important system component in the Java Virtual Machine (JVM). It forbids access
to some specific methods depending on a given security policy. It modifies both the behavior of 
the Java system libraries and of the JVM, for it throws exceptions in case of security policy 
violation.  The difficulty for the verification of a environment containing it 
lies in the fact that its use mixes library calls (easy to specify) as well as
direct VM calls (hard to take in account because they modifies directly the Java semantic of the execution). 
Moreover its use  could be modelled solely as a JVM level component, which would make the
use of this component more homogenous as well as the application of its security policies it applies 
more trustable.

In this article we use a JVM-level modelisation of a SecurityManager seen as an aspect component, 
and we define a verification framework to verify statically programs that takes in account the presence 
of this component, and more generally aspects.
\end{abstract}
%
\section{Introduction}
Java is a language commonly used on small device, and more specifically on Trusted Personnal Devices, devices
of heterogenous nature like cellphones or JavaCards. Thus the verification of the different components Java
platform is made of is crucial; and even more so taking in account their specific caracteristic which would
change the general behaviour of a JVM. Verification of specific components is scattered through litterature: 
there has been verifications of Bytecode Verifiers, as well as Access Contoller. As far as our knowledge goes
though~\cite{HartelMoreau01}, the SecurityManager of the Virtual Machine is not taken into account when
doing static verification of Java programs.


The main goal of this paper is to verify an implementation of a Security Manager, and especially to use 
this verification as a mean to specify more precisely the behaviour of Java Virtual Machine in order to
verify more accurately Java programs.

\subsection{The Security Manager}
The Security Manager applies the security policies on two levels. First on the library level, each time a
writing or reading operation is called for instance a call to the Security Manager is made and if the
caller has not respected a given security policy, a SecurityException is thrown, several other operations
of the same type are handled. Second on the Virtual Machine level, each time a class is loaded, it checks on
a meta level with the use of the ClassLoader (which for this paper will be considered as a part of the
JVM) if the currently inspected class has the right to access a specified type from an outside package.
The second set of property is hard to take in account for a program verification, because it changes the 
behavior of type resolution which is done to the program.
A practical way to model these changes  would be to consider the Security Manager as an Aspect which
would weave at the cutting points representing type resolution.
\begin{figure}
\bcode
pa\=ckage a;\\
public class Main \{\+\\
  st\=atic \{\+\\
    Sy\=stem.setSecurityManager(new SecurityManager() \{\+\\
      pu\=blic void checkPackageAccess(String target) \{\+\\
        if\=(target.equals("b"))\\
          \>throw new SecurityException("That is true");\\
      \}\});\-\-\\
  \}\\
 \\
  public static void main(String[] args) \{\+\\
    System.out.println("Nextline will throw an exception");\\
    b.A a = new b.A();\-\\
  \}\-\\
\}
\ecode
The output of the program:
\bcode
Next line will throw an exception\\
Excepti\=on in thread "main" java.lang.SecurityException: That is true\+\\
	at a.b.Main\$1.checkPackageAccess(Main.java:9)\\
	\dots
\ecode
\caption{An invasive Security Manager}
\end{figure}


\subsection{A Security Manager modeled as Aspects}
Aspect Oriented Programming (AOP) is a recent paradigm that offers modularity
though it is ortogonal to  
the usual Object Oriented Programming paradigm. AOP enables to weave code
directly into a program 
thus changing the behaviour of given constructs. Two new notions have been
introduced through AOP: 
\begin{itemize}
\item the concept of cutting points, points in the program where code could be
inserted, and 
\item advice, the code to be inserted at the specified cutting point.
\end{itemize}

Modeling the Security Manager as an invasive Aspect is quite natural: as we have seen in the above example 
when used it can change the behaviour of the JVM at the type access step. The type is checked only on the
first call, and
%
\begin{figure}
\bcode
pu\=blic aspect SecurityManager \{\+\\

Set$<$String$>$ s = new HashSet$<$String$>$();\\
po\=intcut(Object o) : \=target(o) \&\& !within(SecurityManager)\+ \\
           \>\&\& call(public * funName(..)) \{\\
    String pkg = o.getClass().getPackage();\\
    if\=(!s.contains(pkg)) \{\+\\     
       s.add(pkg);\\
       if\=(pkg.equals("b"))\\
           \>throw new SecurityException("That is true");\\\-\\ 
    \}\-\\
\}\-\\
\}
\ecode
\caption{An Aspect implementation of the invasive SecurityManager}
\end{figure}



\subsection{Related Work}
In this paper we focus on a verification framework adapted for the
verification of AspectJ programs annotated with Pipa.  The pointcut
semantic of AspectJ has been properly
defined~\cite{DBLP:conf/popl/AvgustinovHOMSTV07} as well as the advice
weaving semantic~\cite{weaving04} though only parts have been
formalized~\cite{weaving06}.  Pipa is an annotation language which was
inspired by Clifton and Leavens' work~\cite{clifton02observers}. There
are some extensions to it like pointcuts
annotations~\cite{pointcuts07}, or Moxa \cite{moxa05}.  Our framework
relies on verification based on an intermediate language, like what is
done in ESC/Java2~\cite{FlanaganLLNSS02}, Boogie~\cite{BarnettCDJL05},
or Krakatoa~\cite{MarcheP-MU04}. The work is made to be adapted on a
BoogiePL-like guarded command which is coupled with a
VCGen~\cite{BarnettL05,FlanaganS01}.

\paragraph{Modular verification of Aspects}
Verification in Aspect Oriented Programming language is often linked
to a modular approach.  It is orthogonal to the aim of this paperh: we
tailor verification of aspects which were not specially wanted by the
user, and are most-likely imposed by the environment.  Clifton and
Leavens~\cite{clifton02observers,clifton02spectators,cliftonPhd}
propose a programming discipline to define and specify aspects, using
an extension of JML as specification language.  They define two kind
of aspects, the \emph{spectators} and \emph{assistants} and propose to
explicitly allows the weaving of these aspects. This approach allows
efficient modular reasonning and easy implementations because they
show a way of weaving specifications using a control-flow graph
analysis. In their 2004 work~\cite{shriram04} Krishnamurthi {\it et
al} present a model-checking framework for verification of programs
containing aspects. It is quite different from our work as it presents
a generic framework for aspects. In an unpublished paper~\cite{cesar},
Kunz presents a Hoare logic for modular verification of aspects. The
paper is quite formal, but he does not add special annotations for method
specifications as Clifton and Leavens. For this reason his approach remains 
less modular than Clifton's, but more expressive.




% Shmuel Katz et al. \cite{Katz06,GoldmanK06} propose
% a classification of aspects as \emph{spectative}, \emph{regulative} or
% \emph{invasive}, to simplify program verification by focusing on the









\paragraph{Contents of the paper?}
In this paper we will restrict to the Aspects that can easily represent a
Security Manager, the invasive 
ones. This focus removes all the modularity that was aimed at in the other
verification framework mentioned above,
because it's the specific case where the Aspects have effects and likely break
the original behaviour of the program.




\section{The verification framework}
The verification framework we are building in this paper is an adaptation of
static 
verification based on weakest precondition calculus for aspects. Here we will
consider 
the SecurityManager as an invasive aspect, so we must take into the weaving
process. 

To keep the calculus simple, we do the generation of the verification
conditions in 3 steps. 
First we abstract their specification from the aspects, which gives us a model
of the aspects. 
Then we statically weave these model aspects into the program, it is a
syntactic weaving so 
it is supposed to be lightweight, and finally we verify them
using a standard weakest precondition calculus.


\subsection{Specifications}
The verification framework we are targetting relies on program and
aspects specifications that are precise enough to make the
verification conditions generated provable.  The annotation language
we use is Pipa~\cite{ZhaoR03} (the extension of JML for AspectJ).  The
choice was obvious since it is to our knowledge the only proper
annotation language defined for aspects. Here we won't treat point
cuts specification that Pipa allows.  We will use a JML level
0~\cite{Leavens-etal07} like subset of Pipa.  We won't allow universe
types, but we will allow other non level 0 constructs like the model
methods, and some constructs to handle aspects found in Pipa.
We will also permit behavioural specifications because JML's can
be desugared as shown in \cite{RaghavanL00}.

\subsubsection{JML} 
JML is a behavioural specification language for Java. It is made of
numerous keyword: you can express methods pre and post conditions with
it (the keywords {\tt requires} and {\tt ensures}), exceptional
postcondition ({\tt exsures}). There are notions of frame conditions
(the {\tt assignable} clauses). One can also express assertions ({\tt
assert} construct), and loop invariants.

\paragraph{Invasive model methods} 
Model methods are an interesting construct of JML: these are
specification methods that can be with or without side-effects, their
effects being determined by their specifications.  They can have a
body, but in our framework we are interested by bodyless model
methods. They are used to assume an effect over the program, and so
state that if their requires are satisfied, the given effect (given by
the ensure clause) will be satisfied.  In this paper we won't talk
about pure model methods which could be use in specifications. Here we
rather talk about invasive ones: just like invasive aspects they are
meant to state that the program has been modified non explicitly.
\begin{figure}
\begin{center}
\begin{tabular}{lll}
\begin{minipage}{3cm}
\bcode
ghost int i;\\
...\\
requires i > 0;\\
assignable i;\\
ensures i == 1;\\
model void m();
\ecode
\end{minipage} & 
\ \ \ \ \ \ 
&
\begin{minipage}{3cm}
\bcode
...\\
i++;\\
//@ m();\\
//@ assert(i == 1);\\
...
\ecode \end{minipage}

\end{tabular}
\end{center}
\caption{Model method definition and use}
\label{model_meth_def}
\end{figure}


\paragraph{Model methods call}
To be able to use model methods with the full expressivity desired, we
add a new annotation to JML, the simple annotation method call. Its
syntax would be simply the model method call in the middle of
annotations as shown on figure \ref{model_meth}. It can be easily 
translated into guarded commands (like the one there \cite{BarnettL05}),
but we will define it more precisely in Subsection \ref{gc}.

\subsubsection{Pipa}
Pipa is a behavioural specification language based on JML.  You can
specify advices as well as pointcuts~\cite{pointcuts07} with Pipa.  In
this paper we will only treat the advices specifications.  The
specification of an advice is similar to the one of a method call. It
has a pre and post condition, an exceptional postcondition and a frame
condition. The specification are not different for JML's method's
specification, being applied on advices.  Pipa mainly adds two
construct plus a specific semantic we are interested in: the specific
constructs for around advices, and their semantic of the ensure
clause.



\paragraph{Around advice specifications} 
The difficulty of specification of around advices is due to the
presence of the proceed construct. It has to be stated in the method
specification and in JML there are no keywords to represent it. To
represent the proceed in specification, Pipa uses the predicate {\tt
proceed} as well as the {\tt then} construct as what was stated in
Clifton and Leavens paper~\cite{clifton02spectators}.  The
specification of an around advice is divided in two the part before
the proceeds (the part before the execution of the target instruction)
and the part after the proceeds. These two parts have both pre and
post conditions as well as a frame condition. There is also for the
first part a new keyword proceeds which take a boolean condition which
is an invariant for an around advice which determined if the
instruction must be executed.

\begin{figure}
\begin{center}
\begin{tabular}{ll} \begin{minipage}{3cm}\bcode
int f;\\
...\\
/*\=@ \ \ \=requires f > 0;\+ \\
@ \>ensures f == 1;\\
@ \>proceeds true;\\
@ then\\
@ \> requires f == 2;\\
@ \> ensures f == 3;\\
@*/\-\\
void \= around() : call (void m()) \{\\
\>...\\
\} \ecode\end{minipage}



\end{tabular}
\end{center}

\caption{Around annotations}
\label{arround_annot}
\end{figure}


\subsection{Desugaring specifications}
In the seminal paper about Pipa~\cite{ZhaoR03}, the authors stated
that annotations written with Pipa could be translated back to JML
annotation, and they propose to weave Pipa annotation to JML
annotation, being unclear how to do it.  Here what we understand by
translating Pipa annotation to JML annotations is removing most of the
Pipa specific annotations. We turn the advices into methods so 
we can apply the behaviours desugaring described in Raghavan and Leavens
paper \cite{RaghavanL00}.

\paragraph{Proceed}
Proceed construct is transformed into the predicate {\tt
proceed(bool)} and conjuncted to the ensures specifications of the
part before the then of the given case.  If the proceed was omitted it
is added in the ensures as {\tt proceed(false)}.


\paragraph{then}
The {\tt then} is the only Pipa construct that cannot be desugared
into proper JML. It is used in the contract of around advices, and it
modifies the semantic of the {\tt proceed()} instruction in the code.
Behaviours of around advices is desugared around the {\tt then}
construct: each part of each behaviour taking place before the {\tt
then} construct are desugared with each other as stated for standard
JML behaviour in~\cite{clifton02spectators}.  And the part after the
{\tt then} construct is given the same treatment.



\subsection{Compilation to invasive model methods}
The next step once the program is fully annotated and the
specification desugared is the compilation of the advices to their
effects on the program solely, something which can be expressed by
simple bodyless model methods.  This transformation is direct, but the
specifications of the advices have to be enriched with the pointcuts
conditions that cannot be determined syntactically.  This enrichment
is to make the weaving of the models simpler.

\subsubsection{Translation of the specifications}
The annotations are translated directly to the model methods. The
method has the same signature of the corresponding advice, and the
specifications are copied entirely to the model method. It works
smoothly for most of the cases except for the around advice which is
split into 2 model methods: the one corresponding to the first part,
and the one corresponding to the part after the {\tt then}
construction.  The {\tt proceed} instruction has to be translated to
 JML as well: it becomes a global ghost variable of type bool.  For an
 around advice the model method corresponding to the first part has to
 be modified consequently.  The ensures has the equality {\tt (proceed
 == proc\_cond)} where {\tt proc\_cond} was the argument to the
 annotation {\tt proceed}. The assignable clause is also modified: the
 {\tt proceed} global variable is added to it.
\subsubsection{Translation of the point cuts}
The pointcuts are not directly translated to the model methods, though
the conditions over the pointcut which are not purely syntactical (the
ones not taking only patterns as arguments) are added to the
specification of the method.  More precisely the algorithm is the
following:
\begin{enumerate}
\item 
the point cuts are fully resolved and unfolded to simple boolean
expressions (only a combination of {\tt $||$}, {\tt \&\&}, or {\tt !}
with the point cuts specific keywords).
\item 
then the point cuts expression is ordered in two group: the pattern
related (the one which take an argument of type pattern) and the
others. Typically it will be separated by an {\tt and} ({\tt \&\&}),
but it won't be always the case.
\item 
if the sepation between the pattern point cuts and the other point
cuts is:
\begin{enumerate}
\item 
an {\tt and}: the method {\tt requires} clause is conjuncted with the
expression of the non pattern point cuts and a new behaviour is added
to the methods, where only the {\tt requires} is precised and is the
negation of the expression of the non pattern point cuts.
\item 
a {\tt or}: the method is duplicated, with one version with conjuncted
 to its {\tt requires} clause the expression of the non pattern point
 cuts. This case should most likely be an error in the specification
 of the point cut.
\end{enumerate}
\end{enumerate}
Once the model methods are generated with their specifications, they
can be weaved to the program.



\subsection{Guarded commands}
\label{gc}
\begin{itemize} 
\item 
an assert of the require clause, followed by
\item 
an instruction to state that all the variable assignable have new
values: it can be represented by JML keyword {\tt fresh} which is
usually used for method specifications, but here could be used to
annotate the method, and finally
\item the assume of the method's postcondition.
\end{itemize}
Nevertheless, the exceptional cases being difficult to desugar, it is
interesting to have the model method calls in JML specification.

\begin{figure}
\begin{center}\begin{minipage}{4cm}
The program code:
\bcode
...\\
//@ m();\\
//@ assert(i == 1);\\
...
\ecode
Could be desugared to:
\bcode
...\\
//@ assert (i > 0);\\
//@ fresh(i);\\
//@ assume (i == 1);\\
//@ assert(i == 1);\\
...
\ecode
\end{minipage}\end{center}
\caption{A non exceptional model method call desugared}
\label{model_meth}
\end{figure}
\subsection{Weaving of the models}
At this stage of the transformation, all the program source is turned
into guarded commands. The complete code is composed of the guarded
command version of the base program, the guarded command version of
the advices, and the model abstraction of the aspects.  What has to be
done at this stage is the weaving of the models to the normal code.

The model abstraction is made of:
\begin{enumerate}
\item 
methods which represents the behaviour of before and after advices,
model methods that represent the first part of around advices, model
that represent the second part and,
\item
methods representing the advices, and for around advices, proceeds
model methods.
\end{enumerate}
The weaving consists in adding these methods call in the guarded
commands program. The methods are inserted at the same points as they
would on bytecode. The semantic of the weaving we base our work upon
has been presented in two papers \cite{weaving04,weaving06}.  We use
the rules which were defined in~\cite{weaving06}. Since we operate on
the guarded command version of the program, we need a correspondence
between the bytecode program and the guarded command program. This
will be assured by a correspondence table.  This table has been
computed during the translation from bytecode to guarded commands in
the previous stage.

\begin{figure}[ht]\vspace{-0.4cm}
\begin{mathpar}
%\inferrule [noshadow]
%{
%{\neg \ {\tt isShadow(}m, pc{\tt )} \vee ads = []}
%}
%{
%{\langle \varepsilon, m, pc, ads, nextpc, gcs \rangle \rightarrow }
%{\langle \varepsilon, m, nextpc, \varepsilon.advices, nextpc, gcs \rangle}
%}
%\and

%\inferrule [nomatch]
%{
% {{\tt isShadow(}m, pc{\tt )} \wedge ads \neq []}  \ 
% {\wedge \  \neg \ {\tt matchPcut(}\varepsilon, head(ads).pointcut, m,
%  pc{\tt )}}
%}
%{
%\langle \varepsilon, m, pc, ads, nextpc, gcs \rangle \rightarrow 
%\langle \varepsilon, m, nextpc, tail(ads), nextpc, gcs \rangle
%}
%\\[3ex]
%\and
\inferrule[before]
{
 {{\tt isShadow(}m, pc{\tt )} \wedge ads \neq []}  \ \\
 {\wedge \  
  {\tt matchPcut(}\varepsilon, head(ads).pointcut, m, pc{\tt )}} \ 
 {\wedge\ 
   {\tt head(}ads{\tt ).kind = Before}}\ \\ \wedge\ 
 (\varepsilon', gcs') = {\tt insertBeforeAdvice(}\varepsilon, m, pc,
 head(ads), gcs {\tt )}
}
{
\langle \varepsilon, m, pc, ads, nextpc, gcs \rangle \rightarrow 
\langle \varepsilon', m, pc, tail(ads), nextpc, gcs' \rangle
}
%\\[3ex]
\and

\inferrule[after]
{
 {{\tt isShadow(}m, pc{\tt )} \wedge ads \neq []}  \ \\
 {\wedge \  
  {\tt matchPcut(}\varepsilon, head(ads).pointcut, m, pc{\tt )}} \ 
 {\wedge\ 
   {\tt head(}ads{\tt ).kind = After}}\  \\\wedge\  
 (\varepsilon', gcs') =  {\tt insertAfterAdvice(}
  \varepsilon, m, pc,head(ads), nextpc, gcs {\tt )}
}
{
\langle \varepsilon, m, pc, ads, nextpc, gcs \rangle \rightarrow 
\langle \varepsilon', m, pc, tail(ads), nextpc, gcs' \rangle
}
%\\[3ex]
\end{mathpar}
\vspace{-0.4cm}
\caption{The two weaving rules which are modified}
\label{weaving_rules}
\vspace{-0.4cm}
\end{figure}

The weaving we use is the one presented in~\cite{weaving06}. We use the four
 rules defined in the paper, which takes as an entry an environment 
$\varepsilon$, the current method $m$, the code pointer in the method $pc$,
the list of advices $ads$ which is composed of the advices name with their
syntactic pointcuts and $nextpc$, a pointer to the next instruction in the
program.
We add a new parameter, {\it gcs},
a tuple containing the  guarded command program and 
the bytecode/guarded commands correspondence table 
(Figure \ref{weaving_rules}). The main difference
in the algorithm is for the {\small  BEFORE} and {\small AFTER} rules, where
{\it gcs} is modified and not the code. The the {\tt insertBeforeAdvice} and 
{\tt insertAfterAdvice} are modified in order to add the guarded commands
translation of a simple static method call to the method {\tt aspectOf()}, 
followed by a call to the advice model method. In the case of the 
SecurityManager, we only have to weave a before advice, as shown in Figure
\ref{weaved_prog}.


The around advice weaving is not properly treated in~\cite{weaving06}
 or in~\cite{weaving04}. In our framework the around advice was
 compiled in two methods. So we derive from the later a generic
 weaving method, adapted to the way they were specified. The key idea is
to weave the around-before method like a normal around advice, and add
two guarded command at the end of the weaved method, a non deterministic goto
that goes to the begining of the pointcut or after the pointcut, and at the 
beginning of the pointcut add the assertion {\tt proceed = true}.


\vspace{-0.4cm}
\begin{figure}[h]
\begin{center}
\begin{tabular}{ll} \begin{minipage}{3cm}
\bcode
sta\=rt: \+ \\
    ...\\
    stack[0] := new(heap, b.A);\\
%(Invokevirtual java\_lang\_Class.getPackage)
     arg0 := stack[0]; // advice method call \\
    pre\_heap := heap;\\
    assert arg0 != null;\\
    assert requires(SM.beforeAnyMethod,
     pre\_heap, arg0);\\
    havoc heap;\\
    goto SM.beforeAnyPublicMethod\_normal, 
SM.beforeAnyMethod\_excp;\\
\< SM.beforeAnyMethod\_excp:\\
    havoc stack[0]\\
    assume alloc (stack[0], heap) $\wedge$ typeof(stack[0]) $<$: Throwable;\\
    assume exsures(\=SM.beforeAnyMethod,\\
\> (pre\_heap, arg0), (heap, stack[0]));\\
    goto handler;\\
\< SM.beforeAnyMethod\_normal:\\
    havoc stack[0];\\
    assume ensures(\=SM.beforeAnyMethod, \\
\>(pre\_heap, arg0), (heap, stack[0]));\\
    arg0 := stack[0]; // constructor call \\
    ...
\ecode
\end{minipage}
\end{tabular}
\end{center}
\vspace{-0.4cm}
\caption{The weaved guarded command base program}
\label{weaved_prog}
\vspace{-0.4cm}
\end{figure}
\vspace{-0.4cm}
%


\subsection{The weakest precondition calculus}

In our framework aspects are considered as methods calls that are transferred the whole
control flow, and are permitted to modify local variables in the caller methods. Therefore when aspects are weaved
there wp is just like the one of a method call.
For instance if an advice is weaved before an instruction i, the effective execution of the program
will be of

The aspect are compiled into a list containing the.
The predicate transformers associated with an advice is similar to the weakest precondition
clause toward an invasive method call i.e. a method call that would be able to modify local variables
as well as the other program variables.

One of the amusing fact about this approach is that the weakest precondition calculus is parametized by a new 
semantic which is given by the aspects.

General form of the final wp rule:
\bcode

\\
\ecode
where befores, arounds and afters are the compiled predicate
\subsection{Definition of the weakest precondition calculus on the programs}
\subsection{Proof of Correctness}
%
\section{Verifying a security manager}
\subsection{Intro/Definition/related work: sys component verif}
\subsection{An aspect modelisation}
\subsection{An instanciation example}
\section{Conclusion}
\subsection{Future work: reflexivity anyone?}
\subsection{Configurable PCC}
%
% ---- Bibliography ----
%


\bibliographystyle{plain}
\bibliography{bibli,aspects}
%




\end{document}
