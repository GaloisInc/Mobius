% This is LLNCS.DEM the demonstration file of
% the LaTeX macro package from Springer-Verlag
% for Lecture Notes in Computer Science,
% version 2.3 for LaTeX2e
%
\documentclass[draft]{llncs}
%
\usepackage{makeidx}  % allows for indexgeneration
\usepackage[inline, nomargin]{fixme}
\usepackage{listings}
\usepackage{float}
%
\begin{document}

\newcommand{\rarrow}{$\rightarrow$}
\newcommand{\conj}{$\wedge$}
\newcommand{\disjonc}{$\vee$}
\newcommand{\s}{\,}
\newcommand{\btab}{\begin{tt}\begin{tabbing}}
\newcommand{\etab}{\end{tabbing}\end{tt}}
\newcommand{\bcode}{\begin{tt}\begin{small}\begin{tabbing}}
\newcommand{\ecode}{\end{tabbing}\end{small}\end{tt}}
%
\frontmatter          % for the preliminaries
%

\mainmatter              % start of the contributions
%
\title{Taking into account Java's Security Manager for static verification}
%
\titlerunning{Taking in account Java's Security Manager}  % abbreviated title (for running head)
%                                     also used for the TOC unless
%                                     \toctitle is used
%
\author{Julien Charles \and C\'esar Kunz}
%
\authorrunning{Julien Charles and C\'esar Kunz}   % abbreviated author list (for running head)
%
%%%% list of authors for the TOC (use if author list has to be modified)
\tocauthor{Julien Charles, C\'esar Kunz}
%
\institute{INRIA Sophia-Antipolis\\
\email{\{julien.charles, cesar.kunz\}@inria.fr}}

\maketitle              % typeset the title of the contribution

\begin{abstract}
The Security Manager is an important system component in the Java Virtual Machine (JVM). It forbids access
to some specific methods depending on a given security policy. It modifies both the behavior of 
the Java system libraries and of the JVM, for it throws exceptions in case of security policy 
violation.  The difficulty for the verification of a environment containing it 
lies in the fact that its use mixes library calls (easy to specify) as well as
direct VM calls (hard to take in account because they modifies directly the Java semantic of the execution). 
Moreover its use  could be modelled solely as a JVM level component, which would make the
use of this component more homogenous as well as the application of its security policies it applies 
more trustable.

In this article we use a JVM-level modelisation of a SecurityManager seen as an Aspect component, 
and we define a verification framework to verify statically programs that takes in account the presence 
of this component. 
\end{abstract}
%
\section{Introduction}
Java is a language commonly used on small device, and more specifically on Trusted Personnal Devices, devices
of heterogenous nature like cellphones or JavaCards. Thus the verification of the different components Java
platform is made of is crucial; and even more so taking in account their specific caracteristic which would
change the general behaviour of a JVM. Verification of specific components is scattered through litterature: 
there has been verifications of Bytecode Verifiers, as well as Access Contoller. As far as our knowledge goes
though~\cite{HartelMoreau01}, the SecurityManager of the Virtual Machine is not taken into account when
doing static verification of Java programs.


The main goal of this paper is to verify an implementation of a Security Manager, and especially to use 
this verification as a mean to specify more precisely the behaviour of Java Virtual Machine in order to
verify more accurately Java programs.

\subsection{The Security Manager}
The Security Manager applies the security policies on two levels. First on the library level, each time a
writing or reading operation is called for instance a call to the Security Manager is made and if the
caller has not respected a given security policy, a SecurityException is thrown, several other operations
of the same type are handled. Second on the Virtual Machine level, each time a class is loaded, it checks on
a meta level with the use of the ClassLoader (which for this paper will be considered as a part of the
JVM) if the currently inspected class has the right to access a specified type from an outside package.
The second set of property is hard to take in account for a program verification, because it changes the 
behavior of type resolution which is done to the program.
A practical way to model these changes  would be to consider the Security Manager as an Aspect which
would weave at the cutting points representing type resolution.
\begin{figure}
\bcode
pa\=ckage a;\\
public class Main \{\+\\
  st\=atic \{\+\\
    Sy\=stem.setSecurityManager(new SecurityManager() \{\+\\
      pu\=blic void checkPackageAccess(String target) \{\+\\
        if\=(target.equals("b"))\\
          \>throw new SecurityException("That is true");\\
      \}\});\-\-\\
  \}\\
 \\
  public static void main(String[] args) \{\+\\
    System.out.println("Nextline will throw an exception");\\
    b.A a = new b.A();\-\\
  \}\-\\
\}
\ecode
The output of the program:
\bcode
Next line will throw an exception\\
Excepti\=on in thread "main" java.lang.SecurityException: That is true\+\\
	at a.b.Main\$1.checkPackageAccess(Main.java:9)\\
	\dots
\ecode
\caption{An invasive Security Manager}
\end{figure}
\subsection{Verification of Aspects}
Aspect Oriented Programming (AOP) is a recent paradigm that offers modularity though it is ortogonal to 
the usual Object Oriented Programming paradigm. AOP enables to weave code directly into a program
thus changing the behaviour of given constructs. Two new notions have been introduced through AOP:
\begin{itemize}
\item the concept of cutting points, points in the program where code could be inserted, and
\item advice, the code to be inserted at the specified cutting point.
\end{itemize}
Adding Aspect code to a program most of the time doesn't change its general behaviour, and in these case
the modular properties of the program are kept.
There is a classification of aspects into three groups: speculative, regulative or invasive \fixme{quote the 
article}.

Speculative and regulative Aspects enables modularity. 
\fixme{talk about model checking, JML verification + Cesar work}


In this paper we will restrict to the Aspects that can easily represent a Security Manager, the invasive
ones. This focus removes all the modularity that was aimed at in the other verification framework
mentionned above,
because it's the specific case where the Aspects have effects and likely break the original behaviour of
the program.

\subsection{A Security Manager modeled as Aspects}
Modeling the Security Manager as an invasive Aspect is quite natural: as we have seen in the above example 
when used it can change the behaviour of the JVM at the type access step. The type is checked only on the
first call, and
%
\begin{figure}
\bcode
pu\=blic aspect SecurityManager \{\+\\

Set$<$String$>$ s = new HashSet$<$String$>$();\\
po\=intcut(Object o) : \=target(o) \&\& !within(SecurityManager)\+ \\
           \>\&\& call(public * funName(..)) \{\\
    String pkg = o.getClass().getPackage();\\
    if\=(!s.contains(pkg)) \{\+\\     
       s.add(pkg);\\
       if\=(pkg.equals("b"))\\
           \>throw new SecurityException("That is true");\\\-\\ 
    \}\-\\
\}\-\\
\}
  
\ecode
\caption{An Aspect implementation of the invasive SecurityManager}
\end{figure}
\section{The verification framework}
The verification framework we are building in this paper is an adaptation of static
verification based on weakest precondition calculus for aspects. Here we will consider
the SecurityManager as an invasive aspect, so we must take into the weaving process.

To keep the calculus simple, we do the generation of the verification conditions in 3 steps.
First we abstract their specification from the aspects, which gives us a model of the aspects.
Then we statically weave these model aspects into the program, it is a syntactic weaving so
it is supposed to be lightweight, and finally we verify them
using a standard weakest precondition calculus.

\subsection{Annotations}
The weakest precondition calculus we are targetting relies on program and aspects specifications that are
precise enough to make the verification conditions generated provable.
The annotation language we use is like Pipa \cite{ZhaoR03} (the extension
of JML for Aspects), but in a limited form. The annotation language is a JML level 0 type language 
\cite{Leavens-etal07}, to which
we remove the universe types, but to which we add model methods as well as purity, and some constructs
to handle aspects found in Pipa. The other restriction we add is that we don't treat behavioural specifications
because they can be desugared \cite{RaghavanL00}.

\subsubsection{JML} JML is a behavioural specification language for Java. It is made of numerous keyword: you 
can express methods pre and post conditions with it (the keywords {\tt requires} and {\tt ensures}
\paragraph{Purity} 

\paragraph{Invasive model methods} 
Model methods are an interesting construct of JML: these are specification
methods that can be with or without side-effects, their effects being determined by their specifications.
They can have a body, but in our framework we are interested by bodyless model methods. They are used to 
assume an effect over the program, and so state that if their requires are satisfied, the given effect (given by
the ensure clause) will be satisfied.
In this paper we won't talk about pure model methods which could be use in specifications. Here we rather talk
about invasive ones: just like invasive aspects they are meant to state that the program has been modified 
non explicitly.
\subsubsection{Pipa}
 For our case study we will limit the scope of annotations
to simple JML annotations: we allow to specify loop invariants, methods pre and post conditions
and assertions. For Pipa we take mainly its constructs {\tt proceed} and {\tt then} which are made to
specify around advices.
The methods will not use JML's semantic for ensures annotations but rather the one of Pipa.
\fixme{talk about behavioural limitation}
\paragraph{Semantic of the proceed}
\paragraph{Semantic of the ensure clause}
\subsection{Compilation to invasive model methods}
The main idea is to compile aspects to their effects, which would be modeled by methods that are defined
solely by their specifications. 
They are specifically
\subsubsection{Translation of the specifications}
The annotations are translated directly to the model methods. The method has the same signature
that of the aspect, and the specifications are copied entirely to the model method. It works smoothly
for most of the case except for the around advice which is split into 2 model methods: the one corresponding
to the before part, with added the proceed instruction as conjuncted in the ensures.
\subsubsection{Translation of pointcuts}
The pointcuts are not directly translated to the model methods, though the conditions over the pointcut undecidable
at compile time are added to the require clause.
\subsection{Weaving of the models}
First all the pointcut are abstracted as a syntactical pointcut. Then they are ordered using the compilation
order.
They are ordered in a list made of couples (instr, (model\_list)) where model\_list is the list of model methods 
that correspond to a specific syntactical match.

%
\begin{figure}
\bcode
weave \=(instr; Linstr, (instr, (model :: model\_list)) :: Lmatch) -> \\
\>model:: weave (instr; Linstr, (instr, (model\_list)) :: Lmatch);\\
weave \=(instr; Linstr, (instr, (instr :: model\_list)) :: Lmatch) -> \\
\>instr:: weave (instr; Linstr, (instr, (model\_list)) :: Lmatch);\\
weave (instr:: Linstr, (instr', (model :: model\_list)) :: Lmatch) -> \\
\>instr <> instr',  weave (instr; Linstr, Lmatch);\\
weave (instr:: Linstr, (instr, (nil)) :: Lmatch) -> \\
\>weave (Linstr, Lmatch);\\
weave (instr:: Linstr, nil) -> \\
\>instr :: weave (Linstr, Lmatch).
\ecode
\caption{The algorithm to weave method}
\label{weaving_algo}
\end{figure}
Then for each program point that matches

\subsection{The weakest precondition calculus}

In our framework aspects are considered as methods calls that are transferred the whole
control flow, and are permitted to modify local variables in the caller methods. Therefore when aspects are weaved
there wp is just like the one of a method call.
For instance if an advice is weaved before an instruction i, the effective execution of the program
will be of

The aspect are compiled into a list containing the.
The predicate transformers associated with an advice is similar to the weakest precondition
clause toward an invasive method call i.e. a method call that would be able to modify local variables
as well as the other program variables.

One of the amusing fact about this approach is that the weakest precondition calculus is parametized by a new 
semantic which is given by the aspects.

General form of the final wp rule:
\bcode

\\
\ecode
where befores, arounds and afters are the compiled predicate
\subsection{Definition of the weakest precondition calculus on the programs}
\subsection{Proof of Correctness}
%
\section{Verifying a security manager}
\subsection{Intro/Definition/related work: sys component verif}
\subsection{An aspect modelisation}
\subsection{An instanciation example}
\section{Conclusion}
\subsection{Future work: reflexivity anyone?}
\subsection{Configurable PCC}
%
% ---- Bibliography ----
%


\bibliographystyle{plain}
\bibliography{bibli,aspects}
%




\end{document}
