
\documentclass[10pt,notitlepage,twoside]{article}
%\usepackage{opsem,fancyhdr,amsfonts}
%\pagestyle{fancy}

%%%Macros
\newcommand{\eframe}[4]{\langle #1, #2, #3, #4 \rangle}
\newcommand{\cf}{\mathit{sf}}
\newcommand{\aut}{a}
\newcommand{\automata}[5]{\langle #1,#2,#3,#4,#5 \rangle^A} 
\newcommand{\qu}{\alpha}
\newcommand{\theautomaton}{\automata{Q}{\delta}{\qu_0}{V}{I}}
\newcommand{\floor}[1]{\lfloor #1 \rfloor}
%\newcommand{\em}{\mathit{em}}
\newcommand{\emdef}[2]{(#1,#2)}
\newcommand{\cons}{\texttt{::}}

\newcommand{\excval}[1]{\texttt{E}(#1)}
\newcommand{\retval}[1]{\texttt{R}}
\newcommand{\normval}{\texttt{N}}
\newcommand{\assertval}[1]{\texttt{A}}
\newcommand{\exc}{\mathit{ex}}
\newcommand{\brk}{\mathit{brk}}

\newcommand{\means}[1]{\lceil #1 \rceil}
\newcommand{\eval}[2]{P \vdash #1 \Downarrow #2}
\newcommand{\emeval}[2]{em \vdash #1 \Downarrow #2}
\newcommand{\evale}[2]{#1 \Downarrow #2}
\newcommand{\checktrue}[2]{#1\models #2}
\newcommand{\evalg}[2]{G \vdash #1 \Downarrow #2}
\newcommand{\state}[1]{\langle #1 \rangle}
% annotation
\newcommand{\genv}{\gamma}
\newcommand{\gcomp}{\Gamma}

\newcommand{\annvar}[1]{\mathtt{#1}}

\newcommand{\pre}{\mathit{pre}}
\newcommand{\post}{\mathit{post}}
\newcommand{\ppre}{\mathit{ppre}}
\newcommand{\ppreann}{\overline{\mathit{ppre}}}
\newcommand{\ppost}{\mathit{ppost}}
\newcommand{\mod}{\mathit{mod}}
\newcommand{\fv}{\mathit{fv}}
\newcommand{\true}{\mathit{true}}
\newcommand{\false}{\mathit{false}}



\newcommand{\ssub}[1]{_{{}^#1}}
\newcommand{\ssup}[1]{^{{}_#1}}
\newtheorem{theorem}{Theorem}
\newtheorem{lemma}{Lemma}
\newtheorem{corollary}{Corollary}
\newtheorem{definition}{Definition}
\newtheorem{example}{Example}
%% End macros

\title{Security Automata, Program Annotations and High-level Security Properties}

\author{Igor Siveroni}
\date{}

\begin{document}

\maketitle

\section{Introduction}

To do:
\begin{itemize}
\item Security Automata. Examples.
\item Core Annotations.
\item Two propagation functions: wp-based and cfg-based.
\item Check that recursion works.
\item Soundness of cfg-based. Try completenes/equality with a more abstract semantics.
\item Completeness of wp-based.
\item Examples. What can and can't be done e.g properties about local variables after event (method call).
\end{itemize}

\section{The Language}

\subsection{Syntax}

We define in Figure~\ref{fig:syntax} the syntax of a simple imperative procedural language (core subset) with guard statements and program annotations.
The core subset corresponds to a simple procedural language with global variables.
The language deals with two kind of variables: program variables (\textrm{PVar}) and ghost variables (\textrm{Ghost}).
Statements of the core subset of the language deal solely with program variables, which are introduced by the global variable declaration (global variables) at the begining of the program or as parameters in method declarations (local variables).

Ghost variables are a special type of global variables introduced by the user to model program properties as annotations.
Annotations are added in two ways: as method  pre- and post- conditions, and in the form of guard statements.
Annotations use program and ghost variables. Ghost variables are updated with a special set statement.



\begin{figure}[tbp]
\begin{displaymath}
\begin{array}{c}
\hline
\ \\
x,y,z \in \textrm{PVar}\;\;\;\; X,Y,Z \in \textrm{Ghost} \\
\ \\
\begin{array}{rcl}
\multicolumn{3}{l}{P \in  \textrm{Program},\;d  \in  \textrm{MDecl},\;m  \in  \textrm{MethodName}}   \\
P & ::= & \texttt{global}\;(x = c;)^*  \;\; \texttt{ghost}\;(X = c;)^* \;\;  d^{+}  \\
d & ::= & \texttt{requires}\;\; p \;\; \texttt{pre-set}\;\; \bar{s}   \\
  &     & \texttt{ensures}\;\; p \;\; \texttt{post-set}\;\; \bar{s} \\
& & \texttt{method } m\textrm{(}x_1,\ldots,x_n)\; [= s] \texttt{ end} 
\end{array} \\
\ \\
\begin{array}{rcl}
\multicolumn{3}{l}{e  \in \textrm{Exp},\; b,p,q  \in  \textrm{BExp},\;s  \in  \textrm{Stmt}  } \\
e & ::= & c \mid x \mid e_1 \;aOp\; e_2     \\  
b & ::= & \texttt{true} \mid \texttt{false} \mid e_1 \;\mathit{bOp}\; e_2 \mid
           \texttt{not}\;b \mid b_1 \;\texttt{and}\; b_2 \mid b_1 \;\texttt{or}\; b_2   \\
s & ::= & \texttt{skip} \mid x \texttt{:=}\; e \mid
         \texttt{if }b\;s_1\;s_2 \mid
        \texttt{while}\;[I]\; b\texttt{ do }s \mid s_1; s_2 \mid \\
 & & m(e_1,\ldots,e_{|m|})  \mid \texttt{return} \mid 
 \texttt{assert}\;b \mid \texttt{set}\;\bar{s} \\
\bar{s} & ::= & (X := \bar{e})^* \\
& &  \mbox{where } fv(\bar{e}) \subseteq \textrm{Ghost} \\
\end{array} \\
\ \\
\hline
\end{array}
\end{displaymath}
\caption{Syntax of Programs}\label{fig:syntax}
\end{figure}


A program is made of a global variable declaration, a ghost variable declaration and  a non-empty sequence of (annotated) method declarations.
All global and ghost variables must be declared at the beginning of the program and must initialisation values.
A method delaration is made of the method's annotations, the method's name, the list of parameters (variable names) and the method's body (a statement). Declarations without a body correspond to native methods. 

% statements
% syntax of anotations


% notation
The global and ghost variable declarations, found in $P.\mathit{globals}$ and  $P.\mathit{ghost}$,  are parsed as mappings from (the declared) variables to (their initial) constants.
Method declarations, $P.\mathit{methods}$, are denoted by tuples. We write $P\|m\|$ to obtain the tuple that denotes the declaration of method $m$:
\begin{displaymath}
P\|m\| = \langle p,\bar{s}_0,q,\bar{s}_1,s_\bot,x_1 :: \ldots ::x_{|m|} \rangle
\end{displaymath}
where the first four elements ($p,\bar{s}_0,q,\bar{s}_1$) correspond to the method's annotations (\texttt{requires}, \texttt{pre-set}, \texttt{ensures} and \texttt{post-set}, respectively) $s_\bot$ its body and $|m|$  the method's arity. 
We shall write $\|m\|$ whenever $P$ is clear from the context. 

\begin{definition}[Valid Program]
A program is valid if it satisfies the syntactic rules of Figure~\ref{fig:syntax}, all method names used by method call statements correspond to the methods declared in the program i.e. belong to $P.\mathit{methods}$, all execution paths end with \texttt{return} and there exists a non-native method called \texttt{main}.
\end{definition}

% language subsets.

%%% End of annotations

\subsection{Operational Semantics} \label{sec:opsem}


The operational semantics of the language is divided into two parts: execution of statements and evaluation of expressions. Similarly, the semantics deals with two kind of values, expression values ($\textrm{ExpVal}$) and statement values ($\textrm{StmtVal}$).

The first part of the semantics reasons about the changes performed to the program's state by the execution of a statement. 
A program state $\state{\genv,\rho} $ is made of two components, the local environment $\rho$ that maps local variables to constants, and the environment of  global and ghost variables $\genv$.
\begin{displaymath}
\begin{array}{rcl}
\sigma & \in & \textrm{State} = \textrm{Genv} \times \textrm{Lenv} \\
\sigma & ::= & \state{\genv,\rho} \\
\genv & \in & \textrm{Genv} = (\textrm{PVar} \cup \textrm{Ghost}) \rightarrow \textrm{ValConst} \\
\rho & \in & \textrm{Lenv} = \textrm{PVar} \rightarrow \textrm{ValConst}
\end{array}
\end{displaymath}
Execution is associated to one parameter: the program $P$ that is being executed.
We write $\eval{\state{\genv,\rho},s}{\state{\genv',\rho'},w}$ if  $\state{\genv',\rho'}$ is  the new state after the execution statement $s$ in program state $\state{\genv,\rho}$, and statement value $w$ is its result.

%%%%%%
%%%%%%  Natural Semantics
\begin{figure}[tbp]
\begin{displaymath}
\begin{array}{c}
\hline
\ \\
% skip
\eval{\state{\genv,\rho},\texttt{skip}}{\state{\genv',\rho}}
\;\;\;\;\;
% return - normal
\eval{\state{\genv,\rho},\texttt{return}}{\state{\genv,\rho},\retval{}} \\
\ \\
% assignment - local
\begin{array}{c}
x \not\in dom(\genv) \\
\eval{\state{\genv,\rho},e}{c} \\
\hline
\eval{\state{\genv,\rho},x \texttt{:=}e}{\state{\genv,\rho[x \mapsto c]}}
\end{array}
\;\;\;\;\;\;\;
% assignment - global
\begin{array}{c}
x \in dom(\genv) \\
\eval{\state{\genv,\rho},e}{c} \\
\hline
\eval{\state{\genv,\rho},x \texttt{:=}e}{\state{\genv[x \mapsto c],\rho}}
\end{array} \\
\ \\
% conditional - normal
\begin{array}{c}
\eval{\state{\genv,\rho},b}{v_b} \\
\eval{\state{\genv,\rho},(v_b \;?\; s_1 : s_2)}{\state{\genv',\rho'}} \\
\hline
\eval{\state{\genv,\rho},{(\texttt{if }b\;s_1\;s_2)}}{\state{\genv',\rho'}}
\end{array} 
\;\;\;\;\;
% conditional - break
\begin{array}{c}
\eval{\state{\genv,\rho},b}{v_b} \\
\eval{\state{\genv,\rho},(v_0 \;?\; s_1 : s_2)}{\state{\genv',\rho'},\brk} \\
\hline
\eval{\state{\genv,\rho},{(\texttt{if }b\;s_1\;s_2)}}{\state{\genv',\rho'},\brk}
\end{array} \\
\ \\
% while loop - false
\begin{array}{c}
\eval{\state{\genv,\rho},b}{\texttt{ff}} \\
\hline
\eval{\state{\genv,\rho},{(\texttt{while}\;b\;\texttt{do}\;s)}}{\state{\genv,\rho}}
\end{array}
\;\;\;
% while loop - true
\begin{array}{c}
\eval{\state{\genv,\rho},b}{\texttt{tt}} \wedge \eval{\state{\genv,\rho},s}{\state{\genv'',\rho''}}  \\
 \eval{\state{\genv'',\rho''},{(\texttt{while}\;b\;\texttt{do}\;s)}}{\state{\genv',\rho'}} \\
\hline
\eval{\state{\genv,\rho},{(\texttt{while}\;b\;\texttt{do}\;s)}}{\state{\genv',\rho'}}
\end{array} \\ 
\ \\
% while loop - exception or return
\begin{array}{c}
\eval{\state{\genv,\rho},b}{\texttt{tt}} \\
 \eval{\state{\genv,\rho},s}{\state{\genv',\rho'},\brk}  \\
\hline
\eval{\state{\genv,\rho},{(\texttt{while}\;b\;\texttt{do}\;s)}}{\state{\genv',\rho'},\brk}
\end{array}
\;\;\;
% 
\begin{array}{c}
\eval{\state{\genv,\rho},b}{\texttt{tt}} \wedge \eval{\state{\genv,\rho},s}{\state{\genv'',\rho''}}  \\
\eval{\state{\genv'',\rho''},{(\texttt{while}\;b\;\texttt{do}\;s)}}{\state{\genv',\rho'},\brk} \\
\hline
\eval{\state{\genv,\rho},{(\texttt{while}\;b\;\texttt{do}\;s)}}{\state{\genv',\rho'},\brk}
\end{array} \\ 
\ \\
% sequencing - normal 
\begin{array}{c}
\eval{\state{\genv,\rho},s_1}{\state{\genv'',\rho''}}    \\
\eval{\state{\genv'',\rho''},s_2}{\state{\genv',\rho'}} \\
\hline
\eval{\state{\genv,\rho},{(s_1;s_2)}}{\state{\genv',\rho'}}
\end{array} 
\;\;\;\;\;\;\;
% sequencing - break
\begin{array}{c}
\eval{\state{\genv,\rho},s_1}{\state{\genv',\rho'},\brk}   \\
\hline
\eval{\state{\genv,\rho},{(s_1;s_2)}}{\state{\genv',\rho'},\brk}
\end{array} \\
\ \\
\begin{array}{c}
\eval{\state{\genv,\rho},s_1}{\state{\genv'',\rho''}} \\
 \eval{\state{\genv'',\rho''},s_2}{\state{\genv',\rho'},\brk)} \\
\hline
\eval{\state{\genv,\rho},{(s_1;s_2)}}{\state{\genv',\rho'},\brk}
\end{array} \\
\ \\
\hline
\end{array}
\end{displaymath}
\caption{Statement Evaluation Rules}\label{fig:bigstep-statements}
\end{figure}



The execution of an statement can end either without interruption  or abruptly. The latter is the result of the execution of a return statement or a failed assert - exceptions are treated similarly (Section~\ref{sec:exceptions}).
Statement values are defined by:
\begin{displaymath}
\begin{array}{rcl}
w & \in & \textrm{StmtVal} \\
w & ::= & \normval \mid \brk \\
\brk & \in & \textrm{BreakVal} \\
\brk & ::= & \retval{} \mid \assertval{} 
\end{array}
\end{displaymath}
We write $\eval{\sigma,s}{\sigma',\retval{}}$ if the execution of $s$ was terminated by a $(\texttt{return})$ statement.
Termination due to the failure of checking an assertion is expressed by $\eval{\sigma,s}{\sigma',\assertval{}}$.
Otherwise we say a statement returns $\normval$ though we may  write $\eval{\sigma,s}{\sigma'}$ instead.

%%%%%%
%%%%%%  Natural Semantics - Expressions
\begin{figure}[tbp]
\begin{displaymath}
\begin{array}{c}
\hline
\ \\
\eval{\state{\genv,\rho},c}{c}  \;\;\;\;\;\;\;\;
\begin{array}{c}
x \not\in dom(\genv) \\
\hline
\evale{\state{\genv,\rho},x}{\rho(x)}
\end{array} 
\;\;\;\;
\begin{array}{c}
x \in dom(\genv) \\
\hline
\evale{\state{\genv,\rho},x}{\genv(x)}
\end{array} \\
\ \\
% binary operation - normal
\begin{array}{c}
\evale{\state{\genv_0,\rho},e_1}{\genv_1,c_1} \\
 \evale{\state{\genv_1,\rho},e_2}{\genv_2,,c_2} \\
c = \means{\mathit{op}}(c_1,c_2) \\
\hline
\evale{\state{\genv,\rho},{e_1\;\mathit{op}\;e_2}}{c}
\end{array} \\
\ \\
\hline
\end{array}
\end{displaymath}
\caption{Expression Evaluation Rules}\label{fig:bigstep-expressions}
\end{figure}

Expression evaluation is denoted by the judgement $\evale{\state{\genv,\rho},e}{w}$ and defined according to the rules shown in Figure~\ref{fig:bigstep-expressions}.
The evaluation of an expression does not change the state and, therefore, is not included in the right-hand side of the judgement above.
For the time being, expression values only consist of constants:
\begin{displaymath}
\begin{array}{rcl}
v & \in &  \textrm{ExpVal} \\
v & ::= & c  \\
\end{array}
\end{displaymath}

%The judgement $P \vdash \sigma \rightarrow \sigma'$ describes the changes
% in state $s$ after the execution of one operational step.  


%%%%%%
%%%%%%  Natural Semantics
\begin{figure}[tbp]
\begin{displaymath}
\begin{array}{c}
\hline
\ \\
%% call function
\begin{array}{l}
\mathit{call}(P,\genv,s_0,\vec{x},\vec{c})  =  \state{\genv',\rho'} \\
\;\;\;\;\mbox{where } \rho'' = \bot[x_i \mapsto c_i] \wedge 
      \eval{\state{\genv,\rho''},s_0}{\state{\genv',\rho'},\retval{}} \\
\mathit{call}(P,\genv,\bot,\vec{x},\vec{c}) =  \state{\means{m}(\genv,\vec{c}),\bot}
\end{array} \\
\ \\
\multicolumn{1}{l}{%
\mbox{Let: }s = m(e_1,\ldots,e_{|m|}), \|m\| = \langle p,\bar{s}_0,q,\bar{s}_1, s_\bot,\vec{x} \rangle \mbox{ and } \vec{c} = c_1 \cons \ldots \cons c_{|m|}:} \\
\ \\
%%%% method call
%  Method call
\begin{array}{c}
1 \leq i \leq |m|:\;\evale{\state{\genv,\rho},e_i}{c_i} \\
\checktrue{\state{\genv,[\vec{x} \mapsto \vec{c}]}}{p} \wedge
\eval{\genv,\bar{s}_0}{\genv_0} \\
\state{\genv',\rho'} = \mathit{call}(P,\genv_0,s_\bot,\vec{x},\vec{c}) \\
\checktrue{\state{\genv',\rho'}}{q} \wedge 
\eval{\genv',\bar{s}_1}{\genv_1}   \\
\hline
\eval{\state{\genv,\rho},s}{\state{\genv_1,\rho}}
\end{array} 
\;\;
%%%% method call - failure
\begin{array}{c}
1 \leq i \leq |m|:\;\eval{\state{\genv,\rho},e_i}{c_i} \\
\checktrue{\state{\genv,[\vec{x} \mapsto \vec{c}]}}{\neg p} \\
\hline
\eval{\state{\genv,\rho},m(e_1,\ldots,e_{|m|})}{\state{\genv',\rho},\assertval{}}
\end{array} \\
\ \\
\begin{array}{c}
1 \leq i \leq |m|:\;\evale{\state{\genv,\rho},e_i}{c_i} \\
\checktrue{\state{\genv,[\vec{x} \mapsto \vec{c}]}}{p} \wedge
\eval{\genv,\bar{s}_0}{\genv_0} \\
\state{\genv',\rho'} = \mathit{call}(P,\genv_0,s_\bot,\vec{x},\vec{c}) \\
\checktrue{\state{\genv',\rho'}}{\neg q} \\
\hline
\eval{\state{\genv,\rho},s}{\state{\genv',\rho'},\assertval{}} 
\end{array} \\
\ \\
\multicolumn{1}{l}{\mbox{Guard Statements:}} \\
\ \\
% assert - true
\begin{array}{c}
\checktrue{\state{\genv,\rho}}{b} \\
\hline
\eval{\state{\genv,\rho},\texttt{assert}\;b}{\state{\genv,\rho}}
\end{array}
\;\;\;\;\;\;
% assert -false
\begin{array}{c}
\checktrue{\state{\genv,\rho}}{\neg b} \\
\hline
\eval{\state{\genv,\rho},\texttt{assert}\;b}{\state{\genv,\rho},\assertval{}}
\end{array} \\
\ \\
% set
\begin{array}{c}
\eval{\state{\genv,\rho},e_i}{c_i} \\
\hline
\eval{\state{\genv,\rho},\texttt{set}\;(X_i:=e_i)^*}{\state{\genv[X_i \mapsto c_i],\rho}}
\end{array} \\
\ \\
\hline
\end{array}
\end{displaymath}
\caption{Execution Rules - Guard Statements and Annotated Methods}\label{fig:bigstep-annotations}
\end{figure}


Statement execution is defined according to the evaluation rules shown in Figures~\ref{fig:bigstep-statements} and ~\ref{fig:bigstep-annotations}.
Figure ~\ref{fig:bigstep-statements} describes the execution of statements that do not involve (directly) manipulation of ghost variables and program assertions.
The skip command does not modify the program's state. An assignment statement first evaluates the right-hand side expression and then updates the respective environnment (local or global) depending on the type of variable being assigned: all global variables are always part of the global environment. 
%% return break consitions




The rules in Figure~\ref{fig:bigstep-annotations} correspond to the execution of statements that directly deal with ghost variables and program annotations. 
Succesful execution of a method, besides the usual conditions,  depends on the success of the assertion of the method's \texttt{requires} and \texttt{ensures} annotations , before and after the execution of the method's body, respectively.
Ghost variables are maintained by the global environment.
Updates to global variables can only be performed via \texttt{set} statements.
Assert statements do not modify the program state but succeed if and only if the enclosed assertion is true.
Failure on any of the explicit run-time checks generate the special break value $\assertval{}$ value, which is propagated through the program's execution following the rules on Figures~\ref{fig:bigstep-statements}, until it reaches the top entry point.


Program execution, defined below, starts with a call to the entry method \texttt{main}. The program's arguments are  bound to \texttt{main}'s parameters and the initial global environment is  set to values indicated by the global and ghost variable declarations.

\begin{definition}[Program Execution]
Let $\vec{v}$ be the vector of arguments passed to program $P$. Program execution, denoted by $P(\vec{v})$, is defined as follows:
\begin{displaymath}
P(\vec{v}) \Downarrow \genv',w \;\Leftrightarrow\; \eval{\state{\genv,\bot},\texttt{main}(\vec{v})}{\state{\genv',\bot},w} 
\end{displaymath}
where $\genv = P.\mathit{globals} \cup P.\mathit{ghost}$ and $n = |\texttt{main}|$ .
\end{definition}



\section{Execution Monitoring and Security Automata}\label{sec:em}
% Variables
% Automaton \aut
% actions c


An execution monitor (EM) is a mechanism that enforces security policies by monitoring the execution steps of some system (target) and terminating the target's execution if it is about to violate the security policy being inforced~\cite{Schneider2000}.
In this section, we are concerned with the modelling of EMs by security automata.

We start by defining security automata as a special type of automata that receives input from the execution of a program. We then formalize the semantics of the EM by pairing a security automaton with the operational semantics defined in section~\ref{sec:opsem}

% States - \qu
\subsection{Security Automata}\label{sec:automata}


We define a security automaton as a 5-tuple with fields of the form:
\begin{displaymath}
\begin{array}{rcl}
\aut & \in & \textrm{SecAut} = \mathcal{P}(\textrm{AState}) \times \Delta \times \textrm{AState} \times \textrm{Var} \times \textrm{Input}  \\
\aut & ::= & \theautomaton
\end{array}
\end{displaymath}
where $Q$ denotes the set of the automaton states, $q_0$ the initial state, $V$ the set of state variables that define $Q$, $I$ the set of input symbols and $\delta$ the transition function between program states.

% $a = \theautomaton$ has the following properties:
\begin{itemize}
\item \textbf{Input symbols.} The security automata considered in this paper deal with  input elements that  correspond to the execution of a program's statements as defined by the operational semantics of section~\ref{sec:opsem}.
More precisely, the input symbols indicate the moments before and after the execution of a statement $s$, as defined by the syntax below:
\begin{displaymath}
\begin{array}{rcl}
\imath  & \in & \textrm{Input}\;\;\; \mbox{  (Input Symbols)} \\
\imath & ::= & \texttt{begin}(s) \mid \texttt{end}(s,w) \;\;\;\;\;w \in \textrm{StmtVal} \\
I & \subseteq & \textrm{Input} 
\end{array}
\end{displaymath}
In addition to marking the \texttt{begin} and \texttt{end} of the execution of a statement, \texttt{end} end input symbol indicate if the respective execution ended normally or by a \texttt{return} statement e.g. \texttt{end}$(s,\retval{})$.

A security automaton is concerned with the execution of a particular set of operations. This set of operations is denoted by  $I$, the set of input symbols generated by the execution of the respective statements.
For example, is we only want to reason about the occurrence of calls to methods  $m_1$ and $m_2$ then we should have:
\begin{displaymath}
I = \{ \texttt{begin}(m(e_1,\ldots,e_{|m|})) \mid m \in \{m_1,m_2\} \}
\end{displaymath}

\item \textbf{States and variables.} A security automaton state is defined as an assignment of integer values to state variables. The set of variables that make out the automaton's state is determined by $V$. Thus, given automaton $\theautomaton$:
\begin{displaymath}
\begin{array}{rcl}
\qu \in \textrm{AState} & = & (\textrm{Ghost} \rightarrow \textrm{Integer}) \cup \{ \texttt{bad} \}  \\
\qu_0 \in Q  \subseteq  \mathcal{P}(\textrm{AState}) & s.t. & \forall \qu \in Q.\; dom(\qu) = V  \\
\texttt{bad} \in Q & & 
\end{array}
\end{displaymath}
that is, every valid state must have a value associated to every variable in $V$. We also include a special state \texttt{bad} that should always be part of valid states $Q$. 

\item \textbf{Transition function and input symbols.} Transitions between automata states are defined by the transition function $\delta$. The transition function takes as input a state and an input symbol, and returns a next state, as defined by:
%State changes triggered by method calls.
\begin{displaymath}
\begin{array}{rcll}
\delta & \in & \Delta = \textrm{AState} \times I \rightarrow \textrm{AState} & \mbox{(Transition Function)} \\
\end{array}
\end{displaymath}

The transition function $\delta$ must be completely  defined by the execution of a set of actions applied to states. We restrict these actions to a special type of assignment statements $X := e$ involving solely variables in  $V$. 
The set of actions that define the transition function $\delta$ is encoded by $\floor{\delta}$:
\begin{displaymath}
\begin{array}{rcll}
\floor{\delta} & \in & \textrm{AState} \times \textrm{Input} \rightarrow \textrm{Actions}^* & \mbox{(get Actions Function)} \\
sa  & \in &  \textrm{Actions} \\
sa & ::= & (X := e) & fv(e) \subseteq V
\end{array}
\end{displaymath}
The domain and range of a transition function is restricted to the set of valid states and input symbols defined by the automaton. 
Given automaton $\aut = \theautomaton$ and  actions $\floor{\delta} = (X_i := e_i)_{i=1\ldots n}$, the following must be true:
\begin{displaymath}
\begin{array}{ll}
\bullet & \delta \in Q \times I \rightarrow Q \;\wedge \floor{\delta} \in Q \times I \rightarrow \textrm{Actions}^*  \\
\bullet & dom(\delta) = dom(\floor{\delta}) \;\wedge \\
\bullet & \forall \imath \in I.\; \delta(\texttt{bad},\imath) = \texttt{bad}\\
\bullet & \forall (q \times \imath) \in dom(\delta): 
        \delta(\qu,\imath) = \qu[X_i \mapsto c_i]_{i \in 1\ldots n},\\
& \;\;\mbox{ where} \evale{\state{\qu,\bot},e_i}{c_i}
\end{array}
\end{displaymath}
where the latter spells out the direct relation between $\delta$ and $\floor{\delta}$. 
The use of actions becomes an advantage when translating transition functions  into program annotations. For example, the action $X := X + 1$ replaces several assertions of the form $X := c$, one per valid initial value of $X$. 
A similar generalisation could be applied to the domain of the transtion function by grouping the states asociated to a particular action, and for a given input, under a predicate (or pre-condition). For example, given input $X$, action $X := X + 1$ should be applied only if $X > 0$.

\end{itemize}


%\begin{definition}[Semantics of Security Automata]
%\begin{displaymath}
%a \vdash \qu \stackrel{\imath}{\rightarrow} \qu'  \;\;\Leftrightarrow\;\; a = \theautomaton\wedge (\qu' = \delta(\qu,\imath))
%\end{displaymath}
%\end{definition}



\subsection{Execution Monitor Semantics}

Program execution is monitored by a security automaton that captures the desired security property. An execution monitor (EM) is defined by a program $P$ and automaton $\aut = \theautomaton$ such that $I$, the set of input symbols of the automaton, depend on the domains used by the operational semantics of the language.

%is a subset of the output symbols of the language operational semantics
% ($I \subseteq \textrm{Output}$).

An EM state $em$ is made of the program's state $\sigma$ and the automaton's state $\qu$. 
\begin{displaymath}
\begin{array}{rcl}
EM & \in & \mathrm{EMonitor} \\
EM & ::= & \emdef{P}{\aut}
\end{array} 
\;\;\;\;\;\;
\begin{array}{rcl}
em & \in & \mathrm{EMState} \\
em & ::= & (\sigma,\qu) 
\end{array} 
\end{displaymath}



The operational semantics of an EM is defined as a combination of the language's operational semantics and the automaton's transition system. The execution of a program's statement generates an output symbol (that corresponds to the begining or end of the execution of the statement) which, if recognized as a valid input symbol by the automaton $a$, yields a change of state in $\qu$.  



\begin{definition}[EM Operational Semantics]
The operational semantics of $em = \emdef{P}{\aut}$ is defined as a relation between EM states $(\sigma,\qu)$, and denoted by the judgement $\emdef{P}{\aut} \vdash (\sigma,\qu) \Downarrow (\sigma',\qu')$.
The rules that define the semantics of the EM are obtained by rewriting the operational semantics rules using the new judgement $\emdef{P}{\aut} \vdash (\sigma,\qu) \Downarrow (\sigma',\qu')$ and adding feeding the aotutomaton with the respective input symbols corrresponding to the begeniing and end to the execution of the statement. Thus, for every rule of the form
\begin{displaymath}
\begin{array}{c}
\ldots \mathit{conditions} \ldots \\
\eval{\sigma,s_1}{\sigma_1,w_1} \ldots 
\eval{\sigma_{n-1},s_n}{\sigma_n,w_n} \\
\hline
\eval{\sigma,s}{\sigma',w}
\end{array} \\
\end{displaymath}
we should have
\begin{displaymath}
\begin{array}{c}
\ldots \mathit{conditions} \ldots \\
\begin{array}{ccc}
\imath = \texttt{begin}(s) & & \imath' = \texttt{end}(s,w) \\
\qu_0 = \left \{ \begin{array}{ll}
\delta(\qu,\imath) & \imath \in I \\
\qu & \imath \not\in I
\end{array} \right . 
& & 
\qu' = \left \{ \begin{array}{ll} 
\delta(\qu_n,\imath') & \imath' \in I \\
\qu_n & \imath' \not\in I
\end{array} \right . 
\\
em \vdash (\sigma,\qu_0),s_1 \Downarrow (\sigma_1,\qu_1),w_1 & \ldots & em \vdash (\sigma_{n-1},\qu_{n-1}),s_n \Downarrow (\sigma,\qu_n),w 
\end{array} \\
\hline
em \vdash (\sigma,\qu),s \Downarrow (\sigma',\qu'),w
\end{array}
\end{displaymath}
where $\aut = \theautomaton$. Thus, an axiom rule 
\begin{displaymath}
\begin{array}{c}
\Phi(\sigma,s,\sigma',w) \\
\hline
\eval{\sigma,s}{\sigma',w}
\end{array} \\
\end{displaymath}
is replaced by
\begin{displaymath}
\begin{array}{c}
\Phi(\sigma,s,\sigma',w)  \\
\begin{array}{ccc}
\imath = \texttt{begin}(s) & & \imath' = \texttt{end}(s,w) \\
\qu_0 = \left \{ \begin{array}{ll}
\delta(\qu,\imath) & \imath \in I \\
\qu & \imath \not\in I
\end{array} \right . 
& & 
\qu' = \left \{ \begin{array}{ll} 
\delta(\qu_0,\imath') & \imath' \in I \\
\qu_n & \imath' \not\in I
\end{array} \right . 

\end{array} \\
\hline
em \vdash (\sigma,\qu),s \Downarrow (\sigma',\qu'),w
\end{array}
\end{displaymath}

\end{definition}

The semantics defined above interweaves the small-step style of the automaton semantics into the big-step style of the source language semantics. This is analoguos to the automaton performing a left-to-right, root-to-branches crawl of the derivation tree. 

A more succint specification can be obtained by using a small-step semantics of the source language. On one hand, the resulting EM semantics is straightforward. On the other hand, the extra machinery needed makes proofs cumbersome, in particular those dealing with pre-- and post-conditions.

EM execution, defined below, starts with a call to the entry method \texttt{main} of the program. The program' arguments are bound to \texttt{main}'s parameters, the automaton is set to its initial state and the initial global environment is set to values indicated by the global and ghost variable declarations.

\begin{definition}[EM execution]\label{def:em-execution}
Let $em = (P,\aut)$ and  $\vec{v} = v_1\cons\ldots\cons v_n$ the vector of arguments passed to the program. EM execution, denoted by $em(\vec{v})$ is defined as follows:
\begin{displaymath}
em(\vec{v}) \Downarrow (\genv',\qu'),w \;\Leftrightarrow\;  \emeval{(\state{\genv,\bot},\qu_0),\texttt{main}(\vec{v})}{(\state{\genv',\bot)},\qu'),w}
\end{displaymath}
where $\genv = P.\mathit{globals} \cup P.\mathit{ghost}$ and $n = |\texttt{main}|$ .
\end{definition}




The following lemma states that the EM semantics respects the semantics of the source language:

\begin{lemma}
Let $em = \emdef{P}{\aut}$ and $\aut = \theautomaton$.
If $\qu \in Q$ and $em \vdash (\sigma,\qu),s \Downarrow (\sigma',\qu'),w$ then $\eval{\sigma,s}{\sigma',w}$. Similarly, if $\eval{\sigma,s}{\sigma',w}$ and $\qu \in Q$ then $\exists \qu'$ such that $em \vdash (\sigma,\qu),s \Downarrow (\sigma',\qu'),w$.
\end{lemma}

\subsection{Examples}

\begin{example}[Transactions]\label{ex:transactions-aut}
Let's say we have a system that implements atomic memory updates with a transaction mechanism. Memory updates that need to be atomic must be placed inside a transaction. In this system, transactions are started with a call to the \texttt{beginT} native method, and terminated with a call to the \texttt{commitT}  or \texttt{abortT} native methods.

We would like to make sure that transactions are correctly formed and that nested transactions do not exceed a pre-determined depth $n > 0$. Correctly formed transactions are those that have matching calls to the methods that begin and end transactions.

Transaction depth is modelled by $T$, a variable initially set of zero. The elements of the automaton $\aut_n = \theautomaton$ that checks the desired property are:
\begin{displaymath}
\begin{array}{rcl}
Q & = &  \{ (T,i) \mid 0 \leq i \leq n  \}  \\
\delta & = &  \{ ((T,i),\texttt{begin(beginT())},(T,i+1)) \mid 0 \leq i < n \} \cup \\
& & \{ ((T,i),\texttt{begin(s)},(T,i-1)) \mid (1 \leq i \leq n) \wedge s \in \{\texttt{commitT()},\texttt{abortT()}   \} \}    \\
\qu_0 & = & (T,0) \\
V & = & \{ T \} \\
I & = & \{ \texttt{begin}(m()) \mid m \in \{ \texttt{beginT}, \texttt{commitT}, \texttt{abortT} \}  \}
\end{array}
\end{displaymath}
The actions associated to the transition function are:
\begin{displaymath}
\begin{array}{l}
\floor{\delta}(\qu,i) = \left \{ \begin{array}{ll}
T \texttt{:= } T + 1 & \mbox{if } i = \texttt{begin(beginT())} \wedge 0 \leq \qu(d) < n \\
T \texttt{:= } T - 1 & \mbox{if } i = \texttt{begin}(m())\; \wedge\\
 &  m \in  \{\texttt{commitT()},\texttt{abortT()}   \} \wedge 1 \leq \qu(d) \leq n
\end{array} \right . \\
\end{array}
\end{displaymath}

\end{example}



\begin{example}[Inside method m, do not call n] \end{example}

\begin{example}[Inside method m, always call n] \end{example}

\begin{example}[Inside method m, if call n then call p]  \end{example}


% **********************************************************************

\section{Implemeting Execution Monitors: Generation of Program Annotations from Security Automata}\label{sec:annotation}

Execution monitors specified by security automata, as defined in section \ref{sec:em}, can be implemented by encoding the automaton into the program being checked.
This encoding is obtained by declaring a ghost variable for each state variable, and implementing the transition function and transition actions as a series of guard (assert and set) statements and as method annotations.


\subsection{Encoding states and the transition function}

The transition function is the core of the semantics of a security automaton.
A transition function, as  defined in section~\ref{sec:automata}, takes as input an initial state and a token that signals the beginning or end of the execution of a  statement of the program. Tokens can also be extended to denote the occurrence of a group of events such as any call of a particular method of the assignment of a variable - the former is considered by the current definition.


Given that the sequence (or possible sequences)  of input symbols is already fixed by the control flow of the program, we encode the transition function by making sure that the ocurrence of each valid input symbol i.e. statement, is wrapped around:
\begin{itemize}
\item Checks that ensure that only valid automaton states reach the program point (generated by function $\mathit{cond}$), and
\item Ghost variable updates that implement state changes (generated by function $\mathit{action}$). 
\end{itemize}

The conditions checked by the encoding can be either pre-conditions or post-conditions, depending on the type of input symbol considered. Pre-conditions are generated for \texttt{begin}(-) symbols and post-conditions for symbols of the \texttt{end}(-) form.
The condition $\mathit{cond}(\imath)$ generated for input symbol $\imath$ creates a disjunction with all the valid states that receive $\imath$ as input. It  is defined as follows:

\begin{displaymath}
\mathit{cond}(\imath) = \bigvee \{ \qu \mid \exists \qu' \neq \texttt{bad} \wedge (\qu,\texttt{begin}(s),\qu') \in \delta  \} \\
\end{displaymath}

Similarly, the assignments $\mathit{action}(\imath)$ generated by the occurrence of input symbol $\imath$ are:

\begin{displaymath}
\mathit{action}(\imath) = \floor{\delta}(\qu,\imath) \;\mbox{iff}\; \exists \qu s.t. (\qu,\imath) \in dom(\delta) 
\end{displaymath}

However, the function above is only defined if the following condition is true:
\begin{displaymath}
\forall \imath. (\forall \qu,\qu' \in \{ \qu_0 \mid (\qu_0,\imath) \in \delta \wedge \delta(\qu_0,\imath) \neq \texttt{bad} \}.
\floor{\delta}(\qu,\imath) = \floor{\delta}(\qu',\imath) )
\end{displaymath}

This condition imposes a serious restriction to the type of automata that can be encoded in this language. 

\subsection{The annotation function}


The function $\mathit{annotate}$ takes as arguments a program and a security automaton, and returns the original program annotated with the encoding of the automaton.
The first part of the transformation (above) encodes the initial state of the automaton by adding  $\qu_0$  to the initial mapping of ghost variables of the program. Global variables are left untouched.
Given program $P$ and automaton $\aut = \theautomaton$, $\mathit{trans}$ is defined as follows:
\begin{displaymath}
\begin{array}{l}
P' = \mathit{trans}(P,\aut) \;\Leftrightarrow \\
\;\;\begin{array}{rcl}
P'.\mathit{globals} & = & P.\mathit{globals} \\
P'.\mathit{ghost} & = & P.\mathit{ghost} :: \qu_0 \\
P'.\mathit{methods} & = & \{ \mathit{trans}(P,\aut,d) \mid d \in P.\mathit{methods} \}
\end{array}
\end{array}
\end{displaymath}



Method declarations (including its body) are affected by transitions that depend on input that report the method's invocation or return, and by transitions that depend on the execution of any of the instructions on its body.
Thus, the $\mathit{requires}$ component of every method declaration is updated if there exists a good transition that takes as input the symbol corresponding to the begining of the method's call i.e. \texttt{begin}($m$(-)). If that is the case, the corresponding initial states, transformed into predicates, are added to the method's $\mathit{requires}$ set.

\begin{displaymath}
\mathit{cond}(\texttt{begin}(m())
\end{displaymath}

At the same time, all actions associated to method invocation  must be translated to their respective \texttt{set} annotations. The following actions are appended to the \texttt{preset} annotations of the method:
\begin{displaymath}
\mathit{action}(\texttt{begin}(m())
\end{displaymath}
Similarly, transitions that depend on method returns - reported by input symbols of the form \texttt{end}($m$(-)) - update the $\mathit{ensures}$ and $\mathit{post-set}$ component of the respective method declarations. For every method $m$:

The complete definition of $\mathit{trans}(P,\aut,d)$ is:
\begin{displaymath}
\begin{array}{l}
d' = \mathit{trans}(P,\aut,d) \;\Leftrightarrow\;  \\
\;\;\ \begin{array}{rcl}
d'.\mathit{requires} & = & d.\mathit{requires} \cup \mathit{cond}(\texttt{begin}(m()))  \\
d'.\mathit{preset} & = & d.\mathit{preset} \cup \mathit{action}(\texttt{begin}(m()))  \\
d'.\mathit{ensures} & = & d.\mathit{ensures} \cup \mathit{cond}(\texttt{end}(m())) \\
d'.\mathit{postset} & = & d.\mathit{postset} \cup \mathit{action}(\texttt{end}(m()))  \\
d'.\mathit{body} & = & \texttt{set}\;\floor{\delta}(-,m(-))  ; \mathit{trans}(P,\aut,P.\mathit{body}) 
\end{array}
\end{array}
\end{displaymath}

A similar encoding is applied to transitions that take as input the execution of instructions other than method invocation.

\begin{displaymath}
\begin{array}{l}
\mathit{trans}(P,\aut,s) = s_0; s; s_1 \\
\mbox{where}\;\;
\begin{array}[t]{l}
s_0 = \texttt{assure}\; \mathit{cond}(\texttt{begin}(s));\; \texttt{set}\;\mathit{action}(\texttt{begin}(s)) \\
s_1 = \texttt{assure}\; \mathit{cond}(\texttt{end}(s));\; \texttt{set}\;\mathit{action}(\texttt{end}(s))
\end{array}
\end{array}
\end{displaymath}


\begin{example}[Transactions - Execution monitor]\label{ex:transactions-em}
See Figure~\ref{fig:transactions-em}. Explain.
\end{example}

\begin{figure}[tbp]
\ttfamily
\begin{tabbing}
global T = 0; \\
\ \\
main() \{ m(); \} \\
\ \\
m() \\
\{ \=openT(); \\
\> x = 5; \\
\> closeT(); \} \\
\ \\
\= requires: T = 0  \\
pre-set: T = 1 \\
ensures: T = 1 \\
openT()  \\
\ \\
requires: T = 1  \\
pre-set: T = 0 \\
ensures: T = 0 \\
\=closeT() 
\end{tabbing}
\caption{Example: Transactions}\label{fig:transactions-em}
\end{figure}

\subsection{Correctness of Annotations}

The correctness of the annotation function is establised by proving that  the program annotations generated by  $\mathit{annotate}$ correctly encode the original EM. First we show that the execution of an EM with a valid auntomaton state and the execution of the transformed program with a transformed state (where the automaton state is transformed to a mapping and appended to the global environment) both yield to equivalent states. 

\begin{lemma}
Let  $em = \emdef{P}{\aut}$, $\aut = \theautomaton$ and  $\qu \in Q$ . If there exists $P'$ s.t. $P' = \mathit{annotate}(P,\aut)$ then
\begin{displaymath}
em \vdash (\state{\genv,\rho},\qu),s \Downarrow (\state{\genv',rho'},\qu'),w
\;\Leftrightarrow\;
P \vdash \state{\genv::\qu,\rho},s \Downarrow \state{\genv'::\qu',\rho'},w
\end{displaymath}
\end{lemma}

%% Program level.
Then we show, using the result above, that the execution of the EM and the transformed program, for any set of initial arguments, yield to the same result.

\begin{theorem}
Let  $em = \emdef{P}{\aut}$, $\aut = \theautomaton$ and  $\qu \in Q$ . If there exists $P'$ s.t. $P' = \mathit{annotate}(P,\aut)$ then
\begin{displaymath}
em(v_1,\ldots,v_n) \Downarrow (\genv,\qu),w
\;\Leftrightarrow\;
P'(v_1,\ldots,v_n) \Downarrow \genv::\qu,w
\end{displaymath}

\end{theorem}





\section{Static Checking and Propagation of Annotations}\label{sec:propagation}

The annotation function defined in section~\ref{sec:annotation} provides a systematic method for encoding high-level security properties - defined by security automata - as execution monitors.
Execution monitors are implemented by extending the original language with guard statements, method annotations (a special case of guard statements), and special assigment statements for ghost variables, a set of external variables used for property-related annotations.

Execution monitors are run-time/dynamic checkers. This means that a program is said to satisfy  the desired encoded property if it successfully terminates  i.e. all guard statements in the execution path were successful.

We are interested in checking the program statically, that is, checking the desired property without actually executing the program. We denote static checking with the $\models$ symbol and write $P \models P$ when program $P$ is accepted by the static checker.
Static checking is correct if it is safe:

\begin{definition}[Static Checker Correctness]
A static checker is correct if all programs accepted by the checker are correct: If $P \models P$ then $P \vdash P$.
\end{definition}

\subsection{Static Checking}

We are concerned with a particular static checking technique, a technnique based on Hoare-logic style program annotations.
This technique is based on the following:
\begin{itemize}
\item A program is correct if all its methods satisfy their pre/post-conditions, that is, if we can prove that a method's post-condition (\texttt{ensures} annotation) is satisfied given that its pre-condition (\texttt{requires} annotation)  is true.
\item Method calls are abstracted away. The overall effect of a method call is replaced by its post-condition, provided its pre-condition is true. Otherwise, verification fails.
\end{itemize}

\begin{definition}[Static Checking]
We could make this more precise, perhaps using wp.
\end{definition}

We want to use static checking to verify high-level security  properties e.g. properties encoded by the security automata and implemented by execution monitors.
For example, we would like verify that a particular program uses transactions correctly. This property is encoded by the automaton defined in example~\ref{ex:transactions-aut} and implemented as an execution monitor in example~\ref{ex:transactions-em}.
Clearly, the program (em) shown in figure~\ref{fig:transactions-em}  is correct i.e. it terminates without interruptions caused by errors reported by guard statements.
However, the program does not pass the static checker. The absence of annotations in method \texttt{m}'s declaration makes it impossible to satisfy the proof obligations of the call \texttt{openT()}  i.e. \texttt{T = 0}. 

\begin{figure}[tbp]
\ttfamily
\begin{tabbing}
global T = 0; \\
\ \\
requires: T = 0 \\
main() \{ m(); \} \\
\ \\
requires: T = 0 \\
m() \\
\{ \=openT(); \\
\> x = 5; \\
\> closeT(); \} \\
\ \\
\= requires: T = 0  \\
pre-set: T = 1 \\
ensures: T = 1 \\
openT()  \\
\ \\
requires: T = 1  \\
pre-set: T = 0 \\
ensures: T = 0 \\
\=closeT()
\end{tabbing}
\caption{Transactions Example with  propagated pre-conditions}\label{fig:transactions-pre}
\end{figure}

The previous example shows a correct program that is rejected by the static checker.
In order to increase the number of correct programs that pass the static checker we must strengthen the method annotations. This is done by propagating prove obligations from inside the method's body to the method's declaration and appending them to the method's annotations. The new method annotations affect the proof obligations of the caller methods, triggering the re-calculation of their new method annotations. The process continues all the way up the control flow graph.

For example, the proof obligations generated by \texttt{openT()} in figure~\ref{fig:transactions-em} should be propagated all the way up the program generating the new annoated program shown in figure~\ref{fig:transactions-pre}.


%% Note: assert annotations can be removed as well.

The pre- and post-conditions required by the prove obligations in the body are appended to the ensure and requires elements of the method declation.
The propagation of annotations is performed  by the $\mathit{prop}$ function, defined below.



\begin{definition}[Propagation of Annotations]
The function $\mathit{prop}$ applied to program $P$ generates a new program  according to the following rules:
\begin{displaymath}
\begin{array}{l}
P' = \mathit{prop}(P) \Leftrightarrow \\
\;\;\;\;
\begin{array}{l}
(P'.\mathit{globals} = P.\mathit{globals}) \wedge (P'.\mathit{ghost} = P.\mathit{ghost}) \;\wedge \\
P'.\mathit{methods} = \{ \mathit{prop}(P,m) \mid m \in P.\mathit{methods}  \}
\end{array} \\
m' = \mathit{prop}(P,d)  \Leftrightarrow \\
\;\;\;\;
\begin{array}{l}
m'.\mathit{requires} = \ppre(P,m)  \\
m'.\mathit{preset} = m.\mathit{preset} \\
m'.\mathit{ensures} = \ppost(P,m)  \\
m'.\mathit{postset} = m.\mathit{postset} \\
m'.\mathit{body} = m.\mathit{body}
\end{array}
\end{array}
\end{displaymath}
where $\ppre$ and $\ppost$ can either use the definitions of sections~\ref{sec:wp} or~\ref{sec:cf}. 
\end{definition}

\subsection{WP-based Propagation}\label{sec:wp}

The propagation of annotations based on weakest pre-conditions (strongest post-conditions) is defined  by the rules in Figure~\ref{fig:propagation}. The algorithm collects proof obligations imposed by assert annotations and method calls (ensures annotation) inside the method's body.

Annotations are collected in sets. The $\Rightarrow$ and update operators are extended to sets in the obvious way:
\begin{displaymath}
p \Rightarrow \Psi = \{ p \Rightarrow q \mid q \in \Psi \}
\;\;\mbox{ and }\;\;
\Psi[e/x] = \{ p[e/x] \mid p \in Psi \}
\end{displaymath}

Comment: Note that we no longer carry the set of modified variables since updates are processed in-situ. A similar change could be applied to the control flow based algorithm.

Completeness result for subset - no annoations inside while loops.



\begin{figure}[tbp]
\begin{displaymath}
\begin{array}{rcl}
\hline 
\ppre(P,m) & = & m.\mathit{requires} \cup \ppre(P,\mathit{body}(m),\emptyset)[e_i/X_i]  \\
& & \mbox{where}\; m.\mathit{preset} = [e_i/X_i] \\
\ppost(P,m) & = & m.\mathit{ensures} \cup \ppost(P,\mathit{body}(m),\emptyset) \\
\ \\
\ppre(P,\texttt{skip},\Psi) & = & \Psi  \\
\ppre(P,x \texttt{:=}\; e,\Psi) & = & \Psi[e/x] \\
\ppre(P,\texttt{if }e_0\;s_1\;s_2,\Psi) & = & (e_0 \Rightarrow \ppre(P,s_1,\Psi)) \cup (\neg e_0 \Rightarrow \ppre(P,s_2,\Psi)) \\
\ppre(P,\texttt{while }b\texttt{ do }s,\Psi) & = & (\neg b \Rightarrow \Psi) \cup (b \Rightarrow \ppre(s,\ppre(P,\texttt{while }b\texttt{ do }s,\Psi))   \\
\ppre(P,s_1; s_2,\Psi) & = & \ppre(P,s_1,\ppre(P,s_2,\Psi))  \\
\ppre(P,m(e_1,\ldots,e_{|m|}),\Psi) & = & \ppre(P,m)  \\
\ppre(P,\texttt{return},\Psi) & = & \emptyset  \\
\ppre(P,\texttt{assert}\;p,\Psi) & = & \Psi \cup \{ p \} \\
\ppre(P,\texttt{set}\;(X_i = e_i),\Psi) & = & \Psi[e_i/X_i] \\
\ \\
\mbox{Post conditions...to do:}\\
\ppost(P,\texttt{skip},\Psi) & = & \true  \\
\ppost(P,x \texttt{:=}\; e,\Psi) & = & \true \\
\ppost(P,\texttt{if }e_0\;s_1\;s_2,\Psi) & = & \ppost(P,s_1,\Psi) \vee \ppost(P,s_2,\Psi) \\
\ppost(P,\texttt{while }b\texttt{ do }s,\Psi) & = & \ldots    \\
\ppost(P,s_1; s_2,\Psi) & = & \ppost(P,s_2,\Psi) \wedge \ppost(P,s1,\Psi\cup\mod(s_2))  \\
\ppost(P,m(e_1,\ldots,e_{|m|}),\Psi) & = & \{ q \mid q \in \ppost(P,m) \wedge \fv(q) \cap \Psi \neq \phi   \}\\
\ppost(P,\texttt{return},\Psi) & = & \true \\
\ppost(P,\texttt{assert}\;p,\Psi) & = & \Psi \\
\ppost(P,\texttt{set}\;(X_i = e_i)*,\Psi) & = & \{ \}  \\
\hline
\end{array}
\end{displaymath}
\caption{Weakest Pre-Condition and Strong Post-Condition based Propagation of Annotations}\label{fig:wp-propagation}
\end{figure}


\newpage

%% \section{Approximate Propagation of Annotations}
\subsection{Control Flow based Propagation}\label{sec:cf}

This is the approximation algorithm based on~\cite{PaBaBu03}.
It uses old notation - change.
\begin{figure}[tbp]
\begin{displaymath}
\begin{array}{rcl}
\hline 
\ppre(P^\alpha,m) & = & \alpha.\pre(m) \cup \ppre(P^\alpha,\mathit{body}(m),\emptyset) \\
\ppost(P^\alpha,m) & = & \alpha.\post(m) \cup \ppost(P^\alpha,\mathit{body}(m),\emptyset) \\
\ \\
\ppre(P^\alpha,\texttt{skip},V) & = & \phi  \\
\ppre(P^\alpha,x \texttt{:=}\; e,V) & = & \phi \\
\ppre(P^\alpha,\texttt{if }e_0\;s_1\;s_2,V) & = & \ppre(P^\alpha,s_1,V) \cup \ppre(P^\alpha,s_2,V) \\
\ppre(P^\alpha,\texttt{while }b\texttt{ do }s,V) & = & \ppre(P^\alpha,s,V) \\
\ppre(P^\alpha,s_1; s_2,V) & = & \ppre(P^\alpha,s_2,V\cup\mod(s_1)) \cup \ppre(P^\alpha,s_1,V)  \\
\ppre(P^\alpha,m(e_1,\ldots,e_{|m|}),V) & = & \{ q \mid q \in \ppre(P^\alpha,m) \wedge \fv(q) \cap V \neq \phi  \} \\
\ppre(P^\alpha,\texttt{return},V) & = & \emptyset  \\
\ \\
\ppost(P^\alpha,\texttt{skip},V) & = & \true  \\
\ppost(P^\alpha,x \texttt{:=}\; e,V) & = & \true \\
\ppost(P^\alpha,\texttt{if }e_0\;s_1\;s_2,V) & = & \ppost(P^\alpha,s_1,V) \vee \ppost(P^\alpha,s_2,V) \\
\ppost(P^\alpha,\texttt{while }b\texttt{ do }s,V) & = & \ppost(P^\alpha,s,V)  \\
\ppost(P^\alpha,s_1; s_2,V) & = & \ppost(P^\alpha,s_2,V) \wedge \ppost(P^\alpha,s1,V\cup\mod(s_2))  \\
\ppost(P^\alpha,m(e_1,\ldots,e_{|m|}),V) & = & \{ q \mid q \in \ppost(P^\alpha,m) \wedge \fv(q) \cap V \neq \phi   \}\\
\ppost(P^\alpha,\texttt{return},V) & = & \true \\
\hline
\end{array}
\end{displaymath}
\caption{Control Flow based Propagation of Annotations}\label{fig:propagation}
\end{figure}

\begin{theorem}
Let $P^\alpha \in \overline{\textrm{Program}}$. Then 
$P \models \alpha \Leftrightarrow P \models \mathit{prop}(P,\alpha)$.
\end{theorem}


\section{Adding Exceptions}\label{sec:exceptions}

Moved from first sections - requieres special treatment.

\begin{displaymath}
\begin{array}{rcl}
s & ::= & \ldots
    \texttt{throw}\; \exc \mid \texttt{try}\;s_1\;\texttt{catch}(\exc)\; s_2   \\
\exc & \in & \textrm{Exceptions}
\end{array}
\end{displaymath}

The set \textrm{Exceptions} contains the runtime exceptions defined for the language. The set is equipped with a transitive, reflexive and non-conmutative ordering relation $\preceq$ used for exception handling conditions.

New configutation - value.
Exception hangling - semantics. \\
New annotations. \\
New Rules\\

%%%%%%
%%%%%%  Natural Semantics
\begin{figure}
\begin{displaymath}
\begin{array}{c}
\hline
\ \\
% binary operation - break
\begin{array}{c}
\eval{\state{\genv_0,\rho},e_1}{\genv_1,\excval{\exc}} \; \vee \\
(\eval{\state{\genv_0,\rho},e_1}{\genv_1,c} \wedge \eval{\state{\genv_1,\rho},e_2}{\genv_2,\excval{\exc}}) \\
\hline
\eval{\state{\genv,\rho},{(e_1\;\mathit{op}\;e_2)}}{\genv',\excval{\exc}}
\end{array} \\
\ \\
% Method call -exception arguments
\begin{array}{c}
((n \leq |m|) \wedge 1 \leq i \leq n:\;\eval{\state{\genv_{i-1},\rho},e_i}{\genv_i,c_i} \wedge \eval{\state{\genv_{n-1},\rho},e_n}{\genv_n,\excval{\exc}})  \\
\hline
\eval{\state{\genv,\rho},m(e_1,\ldots,e_{|m|})}{\genv',\excval{\exc}}
\end{array} \\
\ \\
% Method call - exception at call
%  Method call - exception
\begin{array}{c}
1 \leq i \leq |m|:\;\eval{\state{\genv_{i-1},\rho},e_i}{\genv_i,c_i} \\
\|m\| = \langle s_0,x_1,\ldots,x_{|m|} \rangle \wedge
\rho' = \bot[x_i \mapsto c_i] \\
\eval{\state{\genv_{|m|},\rho'},s_0}{\state{\genv_{|m|+1 },\rho'},\excval{\exc}} \\
\hline
\eval{\state{\genv,\rho},m(e_1,\ldots,e_{|m|})}{\genv',\excval{\exc}}
\end{array} 
\;\;\;
%  Method call - native - exception 
\begin{array}{c}
m.\mathit{isNative}  \\
1 \leq i \leq |m|:\;\eval{\state{\genv_{i-1},\rho},e_i}{\genv_i,c_i} \\
\excval{\exc} = \means{m}(c_1,\ldots,c_{|m|}) \\
\hline
\eval{\state{\genv,\rho},m(e_1,\ldots,e_{|m|})}{\genv',\excval{\exc}}
\end{array} \\
\ \\
% throw 
\eval{\state{\genv,\rho},{(\texttt{throw}\; \exc)}}{\state{\genv,\rho},\excval{\exc}} \\
\ \\
% try - catch - Normal
\begin{array}{c}
\eval{\state{\genv_0,\rho},s_1}{\state{\genv_1,\rho'}} \;\vee  \\
\eval{\state{\genv_0,\rho},s_1}{\state{\genv_1,\rho},\excval{\exc'}} \wedge (\exc' \preceq \exc) \wedge \eval{\state{\genv_1,\rho''},{s_2}}{\state{\genv_2,\rho'}}   \\
\hline
\eval{\state{\genv,\rho},{(\texttt{try}\;s_1\;\texttt{catch}(\exc)\;s_2)}}{\state{\genv',\rho'}}
\end{array} \\
\ \\
% try - catch - return
\begin{array}{c}
\eval{\state{\genv_0,\rho},s_1}{\state{\genv_1,\rho'},\retval{-}}\; \vee \\
\eval{\state{\genv_0,\rho},s_1}{\state{\genv_1,\rho''},\excval{\exc'}} \wedge (\exc' \preceq \exc) \wedge \eval{\state{\genv_1,\rho''},{s_2}}{\state{\genv_2,\rho'},\retval{-}}   \\
\hline
\eval{\state{\genv,\rho},{(\texttt{try}\;s_1\;\texttt{catch}(\exc)\; s_2)}}{\state{\genv',\rho'},\retval{-}}
\end{array} \\
\ \\
% try - catch - exc
\begin{array}{c}
(\eval{\state{\genv_0,\rho},s_1}{\state{\genv_1,\rho'},\excval{\exc'}} \wedge \exc'\not\preceq\exc)\; \vee \\
P \vdash \state{\genv_0,\rho},s_1 \Downarrow \state{\genv_1,\rho''},\excval{\exc''} \wedge (\exc'' \preceq \exc) \wedge \eval{\state{\genv_1,\rho''},{s_2}}{\state{\genv_2,\rho'},\excval{\exc'}}   \\
\hline
\eval{\state{\genv,\rho},{(\texttt{try}\;s_1\;\texttt{catch}(\exc)\; s_2)}}{\state{\genv',\rho'},\excval{\exc'}}
\end{array} \\
\ \\
\hline
\end{array}
\end{displaymath}
\caption{Exception Handling Rules}\label{fig:bigstep-exceptions}
\end{figure}
































%%%% Bibliography
\bibliographystyle{plain}
\bibliography{propagation}

\end{document}