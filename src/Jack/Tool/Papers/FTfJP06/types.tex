\section{Native types}
One of the interest of these native methods would be to use them to define  
specification-only libraries. But to do so, native methods are not expressive 
enough: primitive Java types and object types as defined in JML are not 
really good to handle some Coq native functions and expressions.
A good thing would be to be able to use directly some Coq types within our 
specifications.
A solution can be to embed some Coq types within a Java type 
(like \code{Reference}).
First we need some new axioms to get and set the Coq special values, together with a reduction rule. 
For instance if we want to manage a list type
we would need the relation \code{setList}:
\btab
Variable setList: Reference \rarrow \ list \rarrow \ Reference.
\etab
and the relation \code{getList}:
\btab
Variable getList:  Reference \rarrow \ list.
\etab
and the rewriting rule:
\btab
Axiom getsetList: forall r l, getList (setList r l) = l.
\etab 
This method is bad because we have to add many axioms, and this is not really 
done in a natural way.
To properly do this kind of manipulation we would need some types directly defined 
in the target environment,  some 'native' types.

\subsection{Definition}
We have extended the native construct to the type definitions, in order to be able to map
existing libraries well defined in the target prover language to some specification written in
JML. With this construct we can declare a type and some operations on it (with the native methods)
that will be entirely defined in the target language.

The syntax is the same as for the 'native' pure methods:
\btab
/*\=@ pub\=lic native class MyNativeType \{\+\\
   @ \> public native boolean myNativeMethod();\\
   @ \> public native static MyNativeType myStaticMethod1();\\
   ...
\etab
The \code{native} keyword is mandatory for native methods inside the native classes 
since static verification tools (notably Krakatoa) usually allow defining specification only 
methods, which are pure methods defined only by their specifications.

These native types are not standard Java/JML classes, they are more akin of a functional type:
\begin{itemize}
\item they do not inherit from the Object class, they are outside the Java type hierarchy, and they do not
subtype one another;
\item they are not instances, in fact they are not even some references but 
they could be binded to a reference type in the target language;
\item they have no default initializer: if a specification variable is declared with this type it must
be initialized to a value returned by a method or taken from another variable of the same type;
\item it has no constructors: since constructors are used in Java to initialize the object (they return 
nothing) there is no semantic in initializing the newly created 'object' from a native type
\end{itemize}
On the opposite they allow method calls \`a la Java on them. There are two kind of method
calls on these types:
\begin {itemize}
\item the static call, which is just a normal method call from a method defined 
inside a specification library.\\
For instance we can have this kind of calls:
\btab
//@ assert MyNativeType.my\=StaticMethod1() !=\\ \>MyNativeType.myStaticMethod2();
\etab
where \code{myStaticMethod1()} and  \code{myStaticMethod2()} return  values of type \code{MyNativeType}, 
and are two native static methods.
\item the instance call, where the variable on which the native method is called is passed as a 
parameter. For instance:
\btab
//@ assert MyNativeType.myStaticMethod1().myNativeMethod();
\etab
which correspond to a call to \code{myNativeMethod()}, with only one argument passed to the method: the
result of \code{myStaticMethod1()} call. This is a valid for JML assertion since the method 
\code{myNativeMethod()} is
pure (native in fact) and returns a boolean.
\end{itemize}
\subsection{Soundness of the method}
Defined this way, the native types are sound with respect to the way they are implemented in the 
target environment. 

In Jack, for the Simplify output, the native types and methods are just uninterpreted function
symbols. There is no axioms on them,
 so there is less chances to be able to  prove automatically the proof obligations 
using these types. So the proof obligations are as sound as it would have been without these
new constructs. Jack logic for Simplify could be extendable, in order to allow to add axioms on the
native methods and types, but it has not been implemented yet.

In Coq the native types, just like the native methods are defined in a library.
Jack's Coq plugin was modified in order to call the user defined type every\-time  a variable tagged as 
native is translated from Jack to Coq. Letting the user define by himself the type and the properties 
over them can lead to unsoundness. One way to avoid this is to encourage the user to only use only 
functional definitions, like definitions or fixpoints to define the native types 
and the operations over them. 
 The aim is to allow to use mostly executable 
constructs of Coq which will not modify the logic of Jack for Coq, and not add
unnecessary axioms.

If we do runtime verification the native types can be mapped to a Java implementation. The only 
requirement is for every native methods defined to be pure and that it does not throw any exception. 
The definition will be sound with respect to JML and Java interpretation.
\subsection{Native libraries}
The native construct in JML enables to easily declare libraries 
that will be used within the specifications like JML's model classes. 
For instance, if we want to have a library on sets we could define it this way:
\btab
/*\=@ pub\=lic native class ObjectSet \{\+\\
  @\> public native static ObjectSet create();\\
   @\> public native static ObjectSet add(ObjectSet os, Object o);\\
   @\> public native boolean member(Object o);\\
   @ ....\\
   @*/
\etab
On the Coq level an easy way to bind this to a library, is to map it to the Coq library ListSet:
\btab
Definition ObjectSet := set Reference. \\
Definition ObjectSet\_create := empty\_set.\\
Definition ObjectSet\_add (os: ObjectSet) (o: Reference) :=  set\_add o os.\\
 Definition ObjectSet\_member (this: ObjectSet) (o: Reference) := set\_mem o this.\\
 ....
\etab

First we define the Coq type on which the \code{ObjectSet} native type will be bound,
it is a set of objects. For Jack, objects are seen as \code{Reference}, hence the first definition
where \code{ObjectSet} is defined by \code{set Reference}.
Then we define the method \code{create} by the definition \code{ObjectSet\_create}, 
which is static and takes no argument. When it is created the set is empty.
The \code{add} method is defined by \code{ObjectSet\_add} in Coq it is directly bound to the 
\code{set\_add method} of the library \code{ListSet} of Coq. It returns a new \code{ObjectSet} where 
\code{o} has been added to
the preexisting \code{ObjectSet} \code{os}.
The \code{member} method is an instance method, it tells if the \code{ObjectSet} 
on which it is called rightly
contain the argument \code{o} which is an object. It is mapped directly to the 
\code{set\_mem method} of the \code{ListSet} library.


In fact when we define a library this way, there are two distinct sort of methods:
\begin{itemize}
\item the modifiers which are implemented solely as static methods. 
Since native types are of a functional nature (all their methods must be without any side-effect), 
each time we want to modify a data of this type we must create a new object.
\item the observers which can be instance methods or static methods.
\end{itemize}
This way of defining the libraries forces the programmer to make a clear distinction 
between modifiers and observers.

   
 It gives a way to have a Set library in JML a bit like what is done for \code{JMLObjectSet}
\cite{LPCCR-03-JML}, but with differences: modifiers here are all static
(which is not the case for  \code{JMLObjectSet}),
 the \code{ObjectSet} type is outside the Java class hirarchy since it is a native type,
so it does not have 
to define all the inherited methods from the class Object 
(namely \code{equals(Object obj)}, \code{hashCode()} or 
\code{wait()}...) and are not interesting when defining a library to use sets within specifications.