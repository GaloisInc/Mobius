\section{Conclusion}
We have added the \code{native} keyword inside of Jack to help solve the different 
proofs. It can be used to build  native libraries that could replace or 
at least complete elegantly the model classes, adding more expressiveness
to define such types and  simplifying their definitions.

Defining abstractions over program variables with model fields and native types
 enable one to refine Coq data structures to Java programs. 
The proof obligations using these abstractions are not so easy to prove using Jack. 
One needs lots of intermediate lemmas to do so. 
With this methodology, one must first prove some properties on the library, 
like that the translation function \code{toSet} has the same number 
of elements as the originating array, if each element in the array
was different otherwise it will have less elements.

We have done the same kind of work over a list library to help implement a QuickSort. The size of the 
library that only binds the Coq constructs with the JML native methods and types is 251 lines long,
the lemmas proved to help do the proof obligations are 569 lines long. 
With Jack, around 230 proof obligation are generated, half of them are solved automatically, 
and the proof scripts of each solved interactively range from 2 or 3 lines (most of the proofs) to 20. 
Here with the model
variables about just as much are solved automatically, and afterward the longest proof takes around 
10 lines.
First we have to prove properties on the library, and after the proof is easier to do. 
For multiple use,
it is really good to have such JML libraries, fully proven, enabling faster proof and program developments.

Jack doesn't check yet if the defined native method or type well-fit with the declared ones.
It will be an interesting development to add this verification.


Another work would be to do  refinement from the 
target language not only on the data level but also on the program
level. A nice way to handle this would be to use JML's model program construct
together with the native methods and types. We could imagine 
to implement a program with our favorite prover, then bind it
to JML's model program construct using the native methods and types,
 and finally have to prove that the program behave the same
as the model program which is in fact the prover-implemented program.

The only problem is that the model program construct is not implemented at
all inside the main static program verification tools, neither in ESC/Java2 nor
 in Krakatoa nor in Jack. 




