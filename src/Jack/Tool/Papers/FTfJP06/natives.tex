
\documentclass{llncs}
%
\usepackage{makeidx}  % allows for indexgeneration
%

\newcommand{\rarrow}{$\rightarrow$}
\newcommand{\conj}{$\wedge$}
\newcommand{\disjonc}{$\vee$}
\newcommand{\s}{\,}
\newcommand{\btab}{\begin{tt}\begin{small}\begin{tabbing}}
\newcommand{\etab}{\end{tabbing}\end{small}\end{tt}}
\newcommand{\code}[1]{\begin{tt}\begin{small}#1\end{small}\end{tt}}
\begin{document}
%
%\mainmatter          
%
\pagestyle{headings}  % switches on printing of running heads


%
\title{Adding native specifications to JML}

\author{Julien Charles }

\institute{INRIA Sophia-Antipolis, France \\
  {\tt\email{Julien.Charles@sophia.inria.fr}}}

\maketitle

\begin{abstract}
In the specification language JML we can see pure methods as a way to 
express user-defined predicates that will simplify the annotations. 
We take this idea a step further in allowing 
to only declare these predicates in JML without giving an explicit definition.
The explicit definition is done directly in the language
to which the Java program and the specifications are 
translated. To this end we introduce a new keyword to JML, the keyword 
\code{native}. 
To facilitate these definitions we have enabled the user to define 
also \code{native} types in the same way.
In this paper we will describe these new constructs as well as their 
implementation in Jack, 
 and their application to JML's libraries and model fields.
\end{abstract}
%


\section{Jack - a Java verification environment} \label{javaVerif:jack} 
As Java has been gaining popularity in industry since the nineties of the twentieth century,
it also attracted the research interest.   
Thus the nineties upto nowadays have given rise to several verification tools tailored to Java
 based on the principles of Hoare logic. Among the ones that gained most popularity are
esc/java developed at Compaq \cite{escjava}, the Loop tool \cite{jacobs03java}, Krakatoa, Jack \cite{BRL-JACK} etc.

\section{Pure methods}
JML's pure methods are methods that can be used in specifications.
They cannot mutate already existing objects but they can allocate new objects.
Nonetheless, the pure keyword is not an alias for JML {\tt modifies $\backslash$nothing}, 
it implies also that the methods terminates, giving a result or throwing an 
exception.
For instance if a constructor only modifies the object that is being 
created and terminates properly
it can be considered as pure.
A mean to verify whether a method is pure or not according to JML
can be found in \cite{salcianu05}. 



In dynamic program verification pure methods are usually built from their source code. 
The method is first thought on the Java level, without side effects, and afterward
the user writes its specifications on the JML level; in order to be able
to use it in JML annotations.

In static program verification,  pure methods can be built directly
from their specifications, since most of the tools replace  pure method's calls 
 by the instanciation of the pure method's specifications.


\subsection{Jack's implementation}
In Jack the notion of purity used is as in JML a kind of observational purity \cite{naumann05}, 
but a constructor that
only modifies the fields of the newly created objects is not considered 
as pure.
In Jack's weakest precondition calculus the specifications are considered 
as lightweight: the method calls are replaced by their specifications 
inside the calculus.
The replacement with its specifications is done with the normal 
specifications in case the method terminated normally or with the 
exceptional specification if the pure method terminated on an exception.
 
For a method defined in a Java file: 
\btab
 /*\=@ requires tab != null;\+\\
  @  modifies $\backslash$nothing;\\
  @ ensures   $\backslash$result == (0 $\le$ i) \&\& (i $\le$ tab.length);\\
  @ exsures false;\\
  @*/\-\\
publi\=c static /*@ pure @*/  withinBounds(int[] tab, int i) \{\+\\
       return (0 $\le$ i) \&\& (i $\le$ tab.length);\-\\
\}
\etab

with this method call within the annotations: 
\btab
withinBounds(tab,i)
\etab

the method call will be directly replaced in the weakest precondition calculus by:
\btab
(tab $!=$ null) \rarrow \ (0 $\le$ i) \conj \ (i $\le$ tab.length)
\etab
Since the specification
of the method has an {\tt exsures false} clause, there is no exceptional case.
In a way, this method will be replaced by its specifications like for a macro.



\subsection{Specification macros}


When specifying a program with JML one of the main problem is the 
growth of the size of the annotations. 
The way static verification tools usually define the handling 
of pure methods, we can use them to do some specification macros.
The method calls will be replaced by their specifications when the 
annotations will be interpreted. It is useful to avoid the growth.

If we have for instance a property to tell an array is sorted we would 
prefer read the annotation:
\btab
is\_sorted(tab)
\etab
instead of:
\btab
 $\backslash$forall \=int i; 0 $\le$ i \&\& i $<$ tab.length;\+\\ 
 $\backslash$forall int j; \=0 $\le$ j \&\& j $<$ tab.length; \+\\(i $<$ j) ==$>$ tab[i] $\le$ tab[j];
\etab
This method makes the annotations clearer, but as annotations 
grows big, proof obligations grows big too. In order to ease the readability 
of the proof obligations, we would like to keep track of the pure method name
that was used as the macro in order to see what part of the specification we 
are proving. That's why we changed the pure method's substitution in Jack.

In Jack, we decided to have a couple Definition / hypothesis.
Now a functional definition is generated of the form:
\btab
mypurefun\_norm  Args Result := (requires Args) \rarrow \ (ensures Args Result)\\
mypurefun\_exc  Args Result := (requires Args) \rarrow \ (exsures Args Result)
\etab
where {\tt requires} is a predicate that is on the arguments of the pure function and 
which correspond to JML {\tt requires} clause the same for {\tt ensures} and {\tt exsures} 
which are predicates that correspond to JML's {\tt ensures} and {\tt exsures} clause respectively.
These functions are then called within the hypothesis at the places where they were used in the code. 

It is nearly what is done in Krakatoa\cite{MPMU-04-JLAP}, as
Krakatoa use a functional definition of the pure method if it can generate it
but otherwise use an axiomatisation of it like in ESC/Java\cite{COKK-04-ESCJ}.
The axiomatisation is done in 3 parts: the pure method is first declared as a variable, 
there is some hypothesis  using it and giving it its properties (which correspond to its
specifications), and then the variable
is used within the lemma which has to be proved (for more detailed comparison between 
the different technique see: \cite{COK-04-METH, DarvasMueller-05}).

This way of defining Definition/Hypothesis doesn't change anything for automatic 
proof of the proof obligations with prover like Simplify. 
%However, with Coq 
%it add an extra step to the proof which is to unfold the definition 
%within the hypothesis. Nevertheless,  
With Coq,
it facilitate the readability of the proof obligation for the user which 
is a critical point, notably when doing an interactive proof.

\subsection{Pure as Predicates}
Some of the properties we have to express are not so easy to deal with on the JML level. 
We want to be able to prove lemmas concerning pure methods, and also have 
relations over  variables without specifying any property on the relation. 

So we decided to be able to define pure methods directly within the language in which the proof 
obligations are generated or the JML annotations are interpreted. 
We added a new keyword to JML in order to allow it: the {\tt native} keyword. 
If a method is declared within a specification as native, the method will not be defined nor specified
 in JML at all, it will only be declared. Its specifications will be to the target prover or environment 
discretion.

Since the native methods are declared within the specifications they must be pure:
they must not have any side-effect, they can only
create new objects, they have to be terminating. But native is
more restrictive than pure: a native method must not throw any exception.


For instance we can have the property {\tt withinBounds} declared as native, inside the specification:
\btab
//@ public native static boolean withinBounds(int [] tab, int i); 
\etab
If interpreted with a dynamic program verification tool, it
can be defined with the Java method:
\btab
public \= static boolean withinBounds(int [] tab, int i) \{\+\\
    return (tab $!=$ null) \&\& (0 $\le$ i) \&\& (i $\le$ (tab.length));\-\\
\}
\etab
If interpreted with a static program verification tool,
it can be defined this way in Coq:
\btab
Def\=inition withinBounds := \+\\
fun \= (tab i) =>\+\\
       (and (not (tab = null)) (and (0 $\le$ i) (i $\le$ (arraylength tab))).
\etab
In Simplify it will be seen as an uninterpreted function symbol, just a relation on the arguments.

In Jack, the binding from the JML declaration to the {\tt native} language is done automatically.
The arguments passed to the method are the same as the one whose the method was declared with
except:
\begin{itemize}
\item if the method is an instance method, an extra argument {\tt this} is added by Jack 
  at the beginning of the method when it is translated
\item if one of the argument is an array, the array dereferencing relation
(to do array access) and its length relation are also given
\end{itemize}


For static program verification, this construct can be really useful, especially if 
it is used in the pure macro fashion. 
Even though we lose the ability to express JML's behaviour
with these specification macros, 
we can now easily prove properties over specifications in the target prover language. 
Once these properties are proved, they can be added as a help
to ease the automatic solving to some of the proofs of the proof obligations.

In  ESC/Java or Krakatoa it is permitted to define a pure method in
specifications only, but its definition/specifications must be written in Java.
Krakatoa is indeed a front-end for Java and JML to the Why tool \cite{Why-Tool}. 
In this tool there exists a mechanism 
which allow similar definitions as the native keyword: the {\tt parameter} construct.
The main difference with the native keyword is that {\tt parameter} is more tool dependent, 
 it is used implicitly within annotations and it is not used on the JML level.



\section{Native types}
One of the interest of these native methods would be to use them to define  
specification-only libraries. But to do so, native methods are not expressive 
enough: primitive Java types and object types as defined in JML are not 
really good to handle some Coq native functions and expressions.
A good thing would be to be able to use directly some Coq types within our 
specifications.
A solution can be to embed some Coq types within a Java type 
(like \code{Reference}).
First we need some new axioms to get and set the Coq special values, together with a reduction rule. 
For instance if we want to manage a list type
we would need the relation \code{setList}:
\btab
Variable setList: Reference \rarrow \ list \rarrow \ Reference.
\etab
and the relation \code{getList}:
\btab
Variable getList:  Reference \rarrow \ list.
\etab
and the rewriting rule:
\btab
Axiom getsetList: forall r l, getList (setList r l) = l.
\etab 
This method is bad because we have to add many axioms, and this is not really 
done in a natural way.
To properly do this kind of manipulation we would need some types directly defined 
in the target environment,  some 'native' types.

\subsection{Definition}
We have extended the native construct to the type definitions, in order to be able to map
existing libraries well defined in the target prover language to some specification written in
JML. With this construct we can declare a type and some operations on it (with the native methods)
that will be entirely defined in the target language.

The syntax is the same as for the 'native' pure methods:
\btab
/*\=@ pub\=lic native class MyNativeType \{\+\\
   @ \> public native boolean myNativeMethod();\\
   @ \> public native static MyNativeType myStaticMethod1();\\
   ...
\etab
The \code{native} keyword is mandatory for native methods inside the native classes 
since static verification tools (notably Krakatoa) usually allow defining specification only 
methods, which are pure methods defined only by their specifications.

These native types are not standard Java/JML classes, they are more akin of a functional type:
\begin{itemize}
\item they do not inherit from the Object class, they are outside the Java type hierarchy, and they do not
subtype one another;
\item they are not instances, in fact they are not even some references but 
they could be binded to a reference type in the target language;
\item they have no default initializer: if a specification variable is declared with this type it must
be initialized to a value returned by a method or taken from another variable of the same type;
\item it has no constructors: since constructors are used in Java to initialize the object (they return 
nothing) there is no semantic in initializing the newly created 'object' from a native type
\end{itemize}
On the opposite they allow method calls \`a la Java on them. There are two kind of method
calls on these types:
\begin {itemize}
\item the static call, which is just a normal method call from a method defined 
inside a specification library.\\
For instance we can have this kind of calls:
\btab
//@ assert MyNativeType.my\=StaticMethod1() !=\\ \>MyNativeType.myStaticMethod2();
\etab
where \code{myStaticMethod1()} and  \code{myStaticMethod2()} return  values of type \code{MyNativeType}, 
and are two native static methods.
\item the instance call, where the variable on which the native method is called is passed as a 
parameter. For instance:
\btab
//@ assert MyNativeType.myStaticMethod1().myNativeMethod();
\etab
which correspond to a call to \code{myNativeMethod()}, with only one argument passed to the method: the
result of \code{myStaticMethod1()} call. This is a valid for JML assertion since the method 
\code{myNativeMethod()} is
pure (native in fact) and returns a boolean.
\end{itemize}
\subsection{Soundness of the method}
Defined this way, the native types are sound with respect to the way they are implemented in the 
target environment. 

In Jack, for the Simplify output, the native types and methods are just uninterpreted function
symbols. There is no axioms on them,
 so there is less chances to be able to  prove automatically the proof obligations 
using these types. So the proof obligations are as sound as it would have been without these
new constructs. Jack logic for Simplify could be extendable, in order to allow to add axioms on the
native methods and types, but it has not been implemented yet.

In Coq the native types, just like the native methods are defined in a library.
Jack's Coq plugin was modified in order to call the user defined type every\-time  a variable tagged as 
native is translated from Jack to Coq. Letting the user define by himself the type and the properties 
over them can lead to unsoundness. One way to avoid this is to encourage the user to only use only 
functional definitions, like definitions or fixpoints to define the native types 
and the operations over them. 
 The aim is to allow to use mostly executable 
constructs of Coq which will not modify the logic of Jack for Coq, and not add
unnecessary axioms.

If we do runtime verification the native types can be mapped to a Java implementation. The only 
requirement is for every native methods defined to be pure and that it does not throw any exception. 
The definition will be sound with respect to JML and Java interpretation.
\subsection{Native libraries}
The native construct in JML enables to easily declare libraries 
that will be used within the specifications like JML's model classes. 
For instance, if we want to have a library on sets we could define it this way:
\btab
/*\=@ pub\=lic native class ObjectSet \{\+\\
  @\> public native static ObjectSet create();\\
   @\> public native static ObjectSet add(ObjectSet os, Object o);\\
   @\> public native boolean member(Object o);\\
   @ ....\\
   @*/
\etab
On the Coq level an easy way to bind this to a library, is to map it to the Coq library ListSet:
\btab
Definition ObjectSet := set Reference. \\
Definition ObjectSet\_create := empty\_set.\\
Definition ObjectSet\_add (os: ObjectSet) (o: Reference) :=  set\_add o os.\\
 Definition ObjectSet\_member (this: ObjectSet) (o: Reference) := set\_mem o this.\\
 ....
\etab

First we define the Coq type on which the \code{ObjectSet} native type will be bound,
it is a set of objects. For Jack, objects are seen as \code{Reference}, hence the first definition
where \code{ObjectSet} is defined by \code{set Reference}.
Then we define the method \code{create} by the definition \code{ObjectSet\_create}, 
which is static and takes no argument. When it is created the set is empty.
The \code{add} method is defined by \code{ObjectSet\_add} in Coq it is directly bound to the 
\code{set\_add method} of the library \code{ListSet} of Coq. It returns a new \code{ObjectSet} where 
\code{o} has been added to
the preexisting \code{ObjectSet} \code{os}.
The \code{member} method is an instance method, it tells if the \code{ObjectSet} 
on which it is called rightly
contain the argument \code{o} which is an object. It is mapped directly to the 
\code{set\_mem method} of the \code{ListSet} library.


In fact when we define a library this way, there are two distinct sort of methods:
\begin{itemize}
\item the modifiers which are implemented solely as static methods. 
Since native types are of a functional nature (all their methods must be without any side-effect), 
each time we want to modify a data of this type we must create a new object.
\item the observers which can be instance methods or static methods.
\end{itemize}
This way of defining the libraries forces the programmer to make a clear distinction 
between modifiers and observers.

   
 It gives a way to have a Set library in JML a bit like what is done for \code{JMLObjectSet}
\cite{LPCCR-03-JML}, but with differences: modifiers here are all static
(which is not the case for  \code{JMLObjectSet}),
 the \code{ObjectSet} type is outside the Java class hirarchy since it is a native type,
so it does not have 
to define all the inherited methods from the class Object 
(namely \code{equals(Object obj)}, \code{hashCode()} or 
\code{wait()}...) and are not interesting when defining a library to use sets within specifications.
\section{Application of the native libraries}
\subsection{Ghost variables}
The direct application of native libraries is their use with  ghost variables. 
In JML ghost variables are specification-only variables.
In the Jack implementation as well as for all the other JML tools, 
these variables are treated as normal variables
except they can only be used in specifications and they can only be modified 
using the JML set construct.
 
Since they are only defined within the specifications, ghost variables can have a native type. 
To use our ObjectSet library we could have for instance a variable mySet. It would be 
declared as followed in a Java program:
\btab
 //@ ghost \=ObjectSet mySet = ObjectSet.create(); // native types variables\\
\>  must be initialized since they have no default initializer
 \etab
 We could add elements to our ObjectSet using JML's set instruction (with the static modifier):
 \btab
 //@ set mySet = ObjectSet.add(mySet, new Object());
 \etab
and  finally we could then use it within an assertion (with the instance accessor):
\btab
 //@ assert mySet.member(new Object());
 \etab
The only difference in the use of native types for ghost variables instead of Java types 
is the reduced number of properties and proof obligations we have on them. For instance 
there's no point for the assertion to generate a proof obligation  to verify that the program 
is not dereferencing a null pointer. We had to initialize the ghost variable 
with a first value, for which we have {\it a priori} no hypothesis  (except if we now the
 implementation  of the natives, which is prover or target system dependent).

The native types add some expressiveness to JML annotations and are a way to implement 
easily the JML model classes libraries.
Ghost variables are useful when specifying a program, but sometimes we would like to be
sure specifcation variables model real program behaviours. That's why we used our native
construct with JML's model variables.



\subsection{Model fields}
\subsubsection{Definition}
Model fields are specification variables defined by a representation function or with a 
representation relation that maps a program variable to a model variable. Here we will 
only interest ourselves in defining the model field by a representation function (detailed
hints on the implementation of model fields for static verification can be found in
 \cite{LeinoMueller06,breunesse03verifying}). One must 
use 3 constructs in JML to declare a model field:
\begin{itemize}
\item first it must be declared with the model keyword: 
\btab 
//@ model MyType myModel
\etab
\item then it must be linked to a program variable with an abstraction function:
\btab
//@ represents myModel $\leftarrow$ myFun(progVar);
\etab
\item finally we must specify that when the program field is modified the model field is modified too.
\btab
//@ depends myModel $\leftarrow$ progVar;
\etab
\end{itemize}
The represents and depends clauses are strong invariants: they must not be broken 
in any way in any state of the program.

Since one of the aims of model fields is to gain abstraction from the program, 
we could do this abstraction with a native construct. Program variables can be 
abstracted to a native type using a native abstraction function that will link
 the native type with the Java type. The native method would have this signature inside of JML:
\btab
//@ public native static MyNativeType translate(MyJavaType var);
\etab
Once defined we can use it smoothly within the {\tt represents} clause.
\subsubsection{Modeling an array with sets}
One of the immediate applications would be to implement such a translation function to 
model a Java array of Objects with sets, using the sets defined as a native library 
(as defined in subsection 3.3).
To be able to model an array with our native sets, the only thing missing is
an abstraction function from array to sets. This function would have this declaration:
\btab
//@ public static native ObjectSet toSet(Object [] tab);
\etab
Translated through Jack it would be linked with an {\tt ObjectSet\_toSet} definition in Coq:
\btab
Fix\=point\= \ toSet\_intern (tab: Reference)\\ 
\>(refelements: Reference \rarrow \ Z  \rarrow \ Reference)\+\\
\>(len: nat) \{struct len\} : set Reference :=\\
match len with\\
$|$ S n =$>$ set\_add \=(intelements tab (Z\_of\_nat n)) \\\>(toSet\_intern tab refelements n)\\
$|$ S0 =$>$ empty\_set\\
end.\-\\
Definition ObjectSet\_toSet :=\+\\ 
fun\= \ (tab\=: Reference) 
(refelements: Reference \rarrow \ Z  \rarrow \ Reference) \+\\ \>
(arraylength: Reference \rarrow \ Z) =$>$ \\
if \=(tab = null)\\
 \> empty\_set\\
  else \+\\
match (arraylength tab) with\\
$|$ Zpos p =$>$ toSet\_intern tab refelements (nat\_of\_P p)\\
$|$ \_ =$>$ empty\_set\\
end.
\etab
where {\tt refelements} is a dereferencing relation and {\tt arraylength} the relation to get the length of an array. This translation function is built around 2 functions:
\begin{itemize}
\item a first one ({\tt ObjectSet\_toSet}) determining the value of the result if the {\tt tab} parameter is {\tt null} 
\item a recursive function ({\tt toSet\_intern}) that add to the set each element from the array
\end{itemize}
The only thing missing from these definition are the lemmas to ease the proofs. 
Typical examples are: 
\begin{itemize}
\item
\btab
forall \=tab arraylength,
tab $<>$ null \rarrow \+\\
(forall \=i, 0 $\le$ i \conj i $<$ (arraylength tab) \rarrow \+\\
(forall \=refelements, ObjectSet\_member  \+\\
(ObjectSet\_toSet tab refelements arraylength) \\
(refelements tab i)). 
\etab
each element in the array {\tt tab} is a member of the corresponding set defined by the translation
function.
\item
\btab
forall \=tab arraylength,\+\\
(forall \=ref,  ObjectSet\_member \+\\(ObjectSet\_toSet tab intelements arraylength)  ref
\rarrow \\
exists i, (refelements tab i) = ref). 
\etab
for each element of the result set of the translation, there exist an element in the 
array from which the set was translated.
\end{itemize}
\section{Conclusion and Future Work}\label{conclusion}
This article describes a bytecode weakest precondition calculus applied to a bytecode specification language (BCSL).
BCSL is defined as suitable extensions of the Java class file format.
Implementations for a proof obligation generator and a JML compiler to BCSL have been developed and are part of the Jack 1.8 release\footnote{http://www-sop.inria.fr/everest/soft/Jack/jack.html}.
At this step, we have built a framework for Java program verification.
 This validation can be done at source or at bytecode level in a common environment: for instance, to prove lemmas ensuring bytecode correctness all the current and future provers plugged in Jack can be used.

We are now aiming to complete our architecture for establishing trust in untrusted code - in particular extending the present work to a PCC architecture for establishing non trivial requirements.  
%Properties that can be verified are properties expressible in the JML specification language. Design by contract properties (used in interface design) can be easily expressed and sent through a network with this framework. What should be pointed out is that we do not deal with such low level properties like for example memory allocation or time constraints.What the approach proposes is suitable for verifying static properties (invariant) concerning objects: it can be relations between values, or conditions over expressions that the program treats.
In this way, several important directions for future work are:
\begin{itemize}
\item perform case studies and strengthen the tool with more experiments.
\item find an efficient representation and validation of proofs in order to construct a PCC framework for Java bytecode. We would like to build a PCC framework where the proofs are done interactively over the source code
and then compiled down to bytecode. Actually, as we stated in Section \ref{results} the proof obligations generated over a source program and over its compilation with non optimizing compiler are syntactically equivalent modulo name and types. 
\item an extension of the framework applying previous research results in automated annotation generation for Java bytecode (see~\cite{PBBHL}). The client thus will have the possibility to verify a security policy by propagating properties in the loaded code and then by verifying that the code verify the propagated properties.
%\item correctness of the semantics of the weakest precondition calculus proposed, which we will do over the bytecode operational semantics. 

\end{itemize}
%Finally, we are currently proving the correctness of the semantics of the weakest precondition calculus proposed, the proof is built over the bytecode operational semantics and will ensure the soundness of our weakest precondition calculus.


\ \\
\begin{small}
\begin{it}
{\bf Acknowledgements:} I thank Lilian Burdy for having coined the 'native'
 keyword and for his valuable help on Jack, 
I also thank Gilles Barthe, Benjamin Gr\'egoire, Marieke Huisman, 
Mariela Pavlova, Erik Poll,
Sylvain Boulm\'e,  Christine Paulin-Mohring and the anonymous
reviewers for their helpful comments.
\end{it}
\end{small} 

%
% ---- Bibliography ----
%
\bibliography{natives}
\bibliographystyle{plain}
\end{document}
