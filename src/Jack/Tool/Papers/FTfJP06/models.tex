\section{Application of the native libraries}
\subsection{Ghost variables}
The direct application of native libraries is their use with  ghost variables. 
In JML ghost variables are specification-only variables.
In the Jack implementation as well as for all the other JML tools, 
these variables are treated as normal variables
except they can only be used in specifications and they can only be modified 
using the JML set construct.
 
Since they are only defined within the specifications, ghost variables 
can have a native type. 
To use our ObjectSet library we could have for instance a variable mySet. 
It would be declared as followed in a Java program:
\btab
 //@ ghost \=ObjectSet mySet = ObjectSet.create(); // native types variables\\
\>//  must be initialized since they have no default initializer
 \etab
 We could add elements to our ObjectSet using JML's set instruction (with the static modifier):
 \btab
 //@ set mySet = ObjectSet.add(mySet, new Object());
 \etab
and  finally we could then use it within an assertion (with the instance accessor):
\btab
 //@ assert mySet.member(new Object());
 \etab
The only difference in the use of native types for ghost variables instead 
of Java types is the reduced number of properties and proof obligations 
we have on them. For instance there's no point for the assertion to generate 
a proof obligation  to verify that the program is not dereferencing a 
null pointer. We had to initialize the ghost variable with a first value, 
for which we have {\it a priori} no hypothesis  (except if we know the
 implementation  of the natives, which is prover or target system dependent).

The native types add some expressiveness to JML annotations and are a way 
to implement easily the JML model classes libraries.
Ghost variables are useful when specifying a program, but sometimes 
we would like to be sure specifcation variables model real program behaviours.
That's why we used our native construct with JML's model variables.



\subsection{Model fields}
\subsubsection{Definition}
Model fields are specification variables defined by a representation 
function or with a representation relation that maps a program variable 
to a model variable. 
Here we will only interest ourselves in defining the model field by 
a representation function (detailed hints on the implementation of 
model fields for static verification can be found in 
 \cite{LeinoMueller06,breunesse03verifying}). 
One must use 3 constructs in JML to declare a model field:
\begin{itemize}
\item first it must be declared with the model keyword: 
\btab 
//@ model MyType myModel
\etab
\item then it must be linked to a program variable with an abstraction 
function:
\btab
//@ represents myModel $\leftarrow$ myFun(progVar);
\etab
\item finally we must specify that when the program field is modified 
the model field is modified too.
\btab
//@ depends myModel $\leftarrow$ progVar;
\etab
\end{itemize}
The \code{represents} and \code{depends} clauses are strong invariants: 
they must not be broken in any way in any state of the program.

Since one of the aims of model fields is to gain abstraction from the program, 
we could do this abstraction with a native construct. Program variables can be 
abstracted to a native type using a native abstraction function that will link
 the native type with the Java type. 
The native method would have this signature inside of JML:
\btab
//@ public native static MyNativeType translate(MyJavaType var);
\etab
Once defined we can use it smoothly within the \code{represents} clause.
\subsubsection{Modeling an array with sets}
One of the immediate applications would be to implement such 
a translation function to model a Java array of Objects with sets, 
using the sets defined as a native library 
(as defined in subsection 3.3).
To be able to model an array with our native sets, the only thing missing is
an abstraction function from array to sets. 
This function would have this declaration:
\btab
//@ public static native ObjectSet toSet(Object [] tab);
\etab
Translated through Jack it would be linked with an \code{ObjectSet\_toSet} 
definition in Coq:
\btab
Fix\=point\= \ toSet\_intern (tab: Reference)\\ 
\>(refelements: Reference \rarrow \ Z  \rarrow \ Reference)\+\\
\>(len: nat) \{struct len\} : set Reference :=\\
match len with\\
$|$ S n =$>$ set\_add \=(intelements tab (Z\_of\_nat n)) \\
\>(toSet\_intern tab refelements n)\\
$|$ S 0 =$>$ empty\_set\\
end.\-\\
Definition ObjectSet\_toSet :=\+\\ 
fun\= \ (tab\=: Reference) 
(refelements: Reference \rarrow \ Z  \rarrow \ Reference) \+\\ \>
(arraylength: Reference \rarrow \ Z) =$>$ \\
if \=(tab = null)\\
 \> empty\_set\\
  else \+\\
match (arraylength tab) with\\
$|$ Zpos p =$>$ toSet\_intern tab refelements (nat\_of\_P p)\\
$|$ \_ =$>$ empty\_set\\
end.
\etab
where \code{refelements} is a dereferencing relation and \code{arraylength} 
the relation to get the length of an array. 
This translation function is built around 2 functions:
\begin{itemize}
\item a first one (\code{ObjectSet\_toSet}) determining the value of 
the result if the \code{tab} parameter is \code{null} 
\item a recursive function (\code{toSet\_intern}) that add to 
the set each element from the array
\end{itemize}
The only thing missing from these definition are the lemmas to ease 
the proofs. Typical examples are: 
\btab
$\forall$ \=tab arraylength,
tab $<>$ null \rarrow \+\\
($\forall$ \=i, 0 $\le$ i \conj i $<$ (arraylength tab) \rarrow \+\\
($\forall$ \=refelements, ObjectSet\_member  \+\\
(ObjectSet\_toSet tab refelements arraylength) \\
(refelements tab i)). 
\etab
each element in the array \code{tab} is a member of the corresponding set 
defined by the translation function.
\btab
$\forall$ \=tab arraylength,\+\\
($\forall$ \=ref,  ObjectSet\_member \+\\(ObjectSet\_toSet tab intelements arraylength)  ref
\rarrow \\
$\exists$ i, (refelements tab i) = ref). 
\etab
for each element of the result set of the translation, there exist an element in the 
array from which the set was translated.
