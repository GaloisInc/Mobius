\section{Conclusions and Future Work}\label{SecConcl}

This paper described the main characteristics of JACK, the Java Applet
Correctness Kit, a tool set for the validation of security and
functional behaviour properties for Java applications. We have focused
in particular on the features that distinguish JACK from other similar
tools:
\begin{itemize}
\item the integration into a standard IDE;
\item a user interface that helps to understand the proof obligations;
\item the implementation of several algorithms to generate
annotations;
\item support for the verification of both source code \emph{and}
bytecode; and
\item support for interactive verification, both practical
(development of user interface and tactics) and theoretical (native
construct to link JML (or BML) specifications with the logic of the
underlying theorem prover).
\end{itemize}

The JACK tool has been used for several small to medium-scale case
studies. First of all, we have shown how BML annotations can be used
to guarantee resource policies related to memory consumptions on
bytecode applications~\cite{gmg05:sefm}. In addition, we have also
shown how the verification of exception-freeness at bytecode level can
be used to reduce the footprint of Java-to-native compilation schemes:
any condition for a run-time exception for which we can show that it
will never happen, does not have to be checked at
run-time~\cite{DBLP:conf/cardis/CourbotPGV06}.


Development and maintenance of a verification tool for a realistic
programming language is a major effort. During the last decade several
such tools have been developed (see the related work section). This
has resulted in a drastic improvement of the technologies available to
verify applications. However, we believe that now the moment has come
to combine the different technologies, and to bundle this into one
powerful verification tool. Development of such a tool is one of the
goals of the IST \textsf{Mobius} project. It is foreseen that all technology
developed around JACK that distinguish it from other verification
tools will be integrated in this single verification tool. 

Work around this single \textsf{Mobius} tool should also address
limitations of the JML technology, and increase support for
automation, both at the level of specification and verification. In
addition, future work should also provide support for refinement in
JML, as well as for using automated testing techniques for proof
obligations that have not been discharged automatically. This would
make the use of formal techniques in Java validation cost effective,
and provide rigorous and automated support to facilitate the security
evaluations of applications for trusted personal devices.

\paragraph{Related work}
\begin{itemize}
\item Other Java/JML tools: ESC/Java, Jive, Key, LOOP, jmlc
\item Spec\#
\item JMLEclipse
\item Bytecode verification platforms EVP, JVer
\item Fradet for annotation generation?
\item Claude's JML-ADT
\end{itemize}


\subsection*{Acknowledgements}
