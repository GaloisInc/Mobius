\documentclass[]{llncs}

\title{FMCO Tutorial Proposal:\\
JACK~---~a tool for validation of security and behaviour of Java applications}

\author{Gilles Barthe\inst{1} \and
        Lilian Burdy\inst{2} \and
        Julien Charles\inst{1} \and
        Benjamin Gr\'egoire\inst{1} \and
        Marieke Huisman\inst{1} \and
        Jean-Louis Lanet\inst{3} \and
        Mariela Pavlova\inst{1} \and
        Antoine Requet\inst{3}}
\institute{INRIA Sophia Antipolis, France \and Clearsy, France \and
Gemplus, France}

\begin{document}
\maketitle

%\begin{abstract}
%JACK is a tool set for the validation of Java applications. The
%security requirements and behaviour of the application can be
%specified with JML. JACK computes the verification conditions that
%ensure correctness of the application. The verification conditions can
%be verified with different automatic and interactive provers.

%To support the whole development process of secure applications, JACK
%allows to do the verification both at source code and at bytecode
%level. Specifications of source code programs can be automatically
%compiled into bytecode specifications. 

%JACK also provides support for writing specifications, as it
%implements several algorithms to automatically generate annotations.
%\end{abstract}

\paragraph{Motivation}
Over the last years, the use of trusted personal devices, such as
mobile phones, PDAs and smart cards, has become more and more
widespread. As they are often used with security-sensitive
applications, they are an ideal target for attacks. Traditionally,
research has been focusing on avoiding hardware attacks, but the
emergence of new generation trusted personal devices that are
increasingly connected to networks and provide execution support for
complex programs has put forward the need to improve the quality of
software to avoid logical attacks.  Logical attacks are potentially
easier to launch than physical attacks (for example they do not
require physical access to the device, and are easier to replicate
from one device to the other), and may have a huge impact.  In
particular, a malicious attacker spreading over the network and
disconnecting or disrupting devices massively could have deep
consequences.

\paragraph{A tool for application validation}
We report here on the development of the tool JACK. JACK can be used
throughout the development process to increase confidence in the
quality of an application. JACK allows to verify Java applications
that are annotated with JML\footnote{The Java Modeling Language, see
\texttt{http://www.jmlspecs.org}.}. An advantage of using JML is that
there is wide range of tools and techniques available that use JML as
specification language, \emph{i.e.}\ JML specifications can be used to
improve testing, they can be simulated, and in particular they can be
verified. We distinguish two kinds of verification: at runtime, using
\texttt{jmlc}, or statically. Several tools provide
static verification of JML-annotated programs, adopting different
compromises between soundness and automation.  Among these tools, JACK
aims at achieving a high level of soundness, while keeping
automation. Therefore, JACK uses a fully automatic weakest
precondition calculus to generate proof obligations that can be
discharged both by automatic and by interactive theorem provers. The
automatic prover that is used is Simplify, a well-known prover for
program verification, the interactive theorem prover that is used is
Coq.

During the tutorial we illustrate how JACK can be used for the
development of a secure application for a trusted personal device, by
means of several examples. In particular, we show how a high-level
security policy can be weaven as annotations in the application. This
can be done at source code, but also at bytecode level. JACK's user
interface, integrated in Eclipse IDE\footnote{See
\texttt{http://www.eclipse.org}.} allows to inspect the different
proof obligations that are generated for these annotations. Further,
we will also show how JACK is used to verify complex functional
behaviour specifications, by providing advanced support for
interactive verification.


\paragraph{Integration with Eclipse}
Because of the tight integration with Eclipse, the developer does not
have to change tools to validate the application. We will show the
special JACK perspective, that allows to inspect the generated proof
obligations in different views (in a Java-like syntax, or in the
language of the prover). Moreover, syntax colouring of the original
source code allows to see to which parts of the application and
specification the proof obligation relates.

\paragraph{Annotation generation}
One of the drawbacks of using JML-like annotations for specification
is that writing them is labour-intensive and error-prone and it is
easy to forget some. We provide two different annotation generators:
the first generator computes frame conditions and simple preconditions
that are sufficient to avoid runtime exceptions, the second generator
takes as input a security policy and generates and propagates
annotations in such a way that if the application respects the
annotations, then it also respects the security policy.

\paragraph{Verification of bytecode programs}
A possible approach to the secure loading of applications on trusted
personal devices is the use of proof-carrying code. In such a
framework, the applications come equipped with a specification and a
proof that allow the client to establish trust in the application. To
achieve this, we need the ability to reason about bytecode. JACK
provides a verification condition generator for BML, the Bytecode
Modeling Language, a bytecode variation of BML. In addition, it
provides a compiler from JML to BML, for which we have shown that it
basically preserves proof obligations (and thus in principle source
code level proofs can be reused at bytecode level).

\paragraph{Support for interactive verification}
As interactive program verification is known to be labour-intensive,
several advanced Coq tactics have been developed that help with this
verification. To be able to write expressive specifications, a
\texttt{native} construct has been proposed for JML, that allows to
link JML constructs directly with the logic of the underlying
prover. This allows to develop the theory about these constructs
directly in the logic of the theorem prover, which makes specification
and verification much simpler.

\paragraph{Literature}
Several papers have been published around JACK; here we list a few.

\nocite{BRL03:fme,Charles06,m+04:cardis,BurdyP06}

\bibliographystyle{plain}
\bibliography{../specification.bib,/net/home/gbarthe/bib/string,/net/home/gbarthe/bib/gilles,/net/home/gbarthe/bib/article,/net/home/gbarthe/bib/book,/net/home/gbarthe/bib/lncs,/net/home/gbarthe/bib/misc,/net/home/gbarthe/bib/proceedings,/net/home/gbarthe/bib/techrep,/net/home/gbarthe/bib/thesis,/net/home/gbarthe/bib/software,/net/home/gbarthe/bib/web}

\end{document}
