


\section{Support for Interactive Verification}\label{SecInteractive}
One of the interesting feature in Jack is its use for interactive 
verification. When automatic provers fail to solve a proof obligation,
the user can try to solve it interactivly (instead of the usual 
informal verification). It is comparable to what is done within the tool
 Krakatoa but Jack has the advantage to be integrated in Eclipse.
More specifically, Jack's case viewer allows to edit every proof that 
failed automatic verification separately and 
it also shows the proof source from the Java code. 
\subsection{The Coq plugin}
The generated  proof obligations are translated into Coq language, mostly
straight-forwardly. The main concern in this plugin is the proof readibility,
and proof reusability.
That's why we developped a set of facilities for variable name handling,
pretty printing, the reuse of proofs, the ease to build the proofs and the 
user extensions.

\paragraph{Pretty printing}
In Jack, facilities for better readibility already exists.
A difference is made between short variable names, 
when no other variable is named this way,
 and long variable names, when there can be an ambiguity. 
When doing interactive verification the scope of these disambiguations 
can be mistaken as Jack do it execution case wise and not lemma wise.
This disambiguation is redone (lemma wise) 
in the Coq plugin to add better proof readability.
The main pretty printing is done directly through Coq's own features for
pretty printing. Each proof obligation is passed to Coq top-level so it
can be printed in an optimized way.

\paragraph{Lemma generation} 
A special attention has also be given when generating the proof obligations 
into a file. First it has a name composed of the full class name, the method
name, the execution case and the goal. It is done so the user can retrieve 
the proof obligation with Jack whenever he wants it. Each file is stocked
locally in a directory named {\tt proofs} 
for when the lemma is regenerated or reopened; the proof it was containing
can be replayed (and eventually fail if the conditions have changed too much).
The normal hypothesis are separated from the invariants; given different names.
It is specifically useful when a goal has been modified (and considered 
unproved) by the modification of an invariant. If the proof was not involving
the use of any invariant; it can be replayed without any changing.

\paragraph{Automations}
The interaction with the theorem has been made tighter than the usual 
interactive verification's one.
Jack has 3 {\it interactive-automatic} modes. 
By {\it automatic} we mean we 
try to apply a generic proof script to several goals, 
and if it fails we try another one. 
The three automatic modes are: the light mode, where we use a proof script
called lightAutoJack to prove the goals; 
the tough mode, where we use the proof script toughAutoJack, 
and the semi-auto mode where the user can write a proof script that will
be tried on several lemmas.



\subsection{Jack with Coq in Eclipse}
Jack's Coq plugin relies on a translation of Jack's Java logic in
Coq as well as several user extension. Since everything is generated
inside the Eclipse's workspace, and the user extensions have to be 
developped as part of the interactive verification process,
 a smoother interaction of Coq within
the Eclipse environment was necessary.

\paragraph{Coq editor}
We added a way to interact directly with Coq through Eclipse's Java 
environment. Mostly it is a way to directly use Coq and edit Coq files within
Eclipse. It is alike what has been done for Isabelle in Proof General Eclipse 
(!!!reference!!!), but for Coq and in a more light-weight fashion.
It has keyboard shortcuts similar to CoqIde (the current Coq graphical
interface which is written in OCaml). 
The main feature we have is the Outline view; 
which sums up the view of a Coq file in a tree-like fashion (especially useful
with to see modules hierarchy). There is a search by tags feature which
permits to search a whole project for the definition of a specific keyword; and
it is done incrementaly with Eclipse's builder feature. Of course, there is 
syntax highlighting for Coq files and you can interactively play a Coq file
as in the usual interactive provers IDE.


\paragraph{User extensions: tactics}

One of the key point in interactive verification is the level of difficulty
to manipulate the proof assistant in order to solve a proof obligation.
In order to help the user build proof scripts that are both intuitive
to read and to make, we have used the tactic mechanism of Coq.
These tactics are written in a pseudo language named Ltac \cite{DEL-00-LTAC}
which permits to combine primitive tactics.
The Coq plugin contains a library of those which, based on Jack's Coq 
implemented Java logic, can help for obvious or really specific proof steps
in Jack's proof obligations. All these tactics are situated in a library 
used when a new proof obligation is loaded. Tactics can also be used in 
really specific user oriented context. There is a special file called 
'user extension' which is meant to contain user-defined
special lemmas and special tactics.



\subsection{Radical user extensions: Native Specifications}



