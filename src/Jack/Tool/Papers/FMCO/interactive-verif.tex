

\section{Support for Interactive Verification}\label{SecInteractive}
When automatic provers fail to solve a proof obligation,
the user can try to change the specifications of the program or
 solve the proof obligation interactively.
Using the interactive verification support of JACK,
the proof obligation can be proven or in case it is unprovatble it
can be analysed thoroughly to find the source of the error.
This verification is done mainly with the Coq proof assistant. 

First we will discuss the features of JACK's Coq plugin, then its 
integration within the IDE Eclipse. Finally we will present
 JACK's specific annotation keyword for interactive verification: the 
\native keyword. 

\subsection{The Coq Plugin}
Proof readability and proof reusability is crucial in interactive 
verification; in contrast to automatic verification 
where proof obligations are simply sent to the automatic prover 
and it is of no importance whether the proof obligation is human-readable or not.
Therefore we developed a set of facilities for pretty printing proof obligations, 
to reuse the proofs, in particular to allow replaying the proofs 
when the specifications have changed, and for proof construction.

JACK uses short variable names in proof obligations as much as possible, 
but in case of ambiguity long variable names are used.
Basically JACK generates all variable names for all proof obligations of one 
file in a row whereas in interactive verification the variables are only 
considered within the scope a single proof obligation.
The Coq Plugin corrects this in redoing the disambiguation per proof 
obligation.
This results in better proof readability as the variable names are shorter.
The main pretty printing is done directly through Coq's own pretty printing 
features. 


Special attention has also been given when storing the proof obligations 
to a file. First, the file has a human-readable name so the user can retrieve 
the proof obligation as well as its proof script easily. 
If the lemma is regenerated or reopened he can step through the proof 
(and eventually fail to achieve it if the conditions have changed too much), 
and it does not have to be rewritten from scratch.
The different kind of hypotheses are separated from each other and
are given different names.
This is in particular important when a proof obligation has been modified 
(and considered unproved) by the modification of the specifications. 
If the proof did not involve of one of the modified hypotheses
it remains valid because most of the names remained
unchanged. This facilitates greatly the reuse of proofs.

One of the key points in interactive verification is the level of difficulty
to manipulate the proof assistant in order to construct a proof.
To help the user  build proof scripts that are both intuitive
to read and to make, we have used the tactic mechanism of Coq~\cite{DEL-00-LTAC}.
As JACK generates proof obligations for automatic verification, numerous hypotheses
 are 
added to help the automatic theorem prover. For interactive verification 
these hypotheses are usually useless (and annoying). 
Therefore we have developed tactics made to clean up the proof obligations. 
There are also tactics\footnote{
See \texttt{http://www-sop.inria.fr/everest/soft/Jack/doc/plugin/coq/Prelude/}
for a full description of the different tactics.}
 to solve common proof patterns
generated by JACK: \begin{inparaenum}
\item to solve arithmetic goals,
\item when the solving of a proof leads to an absurd case, 
\item the ones to help solving
array specific proof obligations 
\item the ones to solve assignment proof obligations.
\end{inparaenum}
 The Coq Plugin also allows automatic resolution of proof 
obligations using generic proof scripts.
This library of tactics and the automatitions mechanism are 
useful but usually the user likes to define his own application
specific tactics.
A user can define his own tactic in the Coq Plugin,
 so he can use them in the interactive solving or 
with the automatic resolution.








\subsection{JACK with Coq in Eclipse}
An important feature of JACK is that everything can be done inside Eclipse.
Therefore, the Coq Plugin contains an editor for Coq, CoqEditor. 
CoqEditor provides support for proving developing user extensions.

CoqEditor is a way to interact directly with Coq through Eclipse's Java 
environment, so the user can process and edit Coq files. 
It is similar to what has been done for Isabelle in 
Proof General Eclipse \cite{WintersteinAL05}, but more light-weight.
It has keyboard shortcuts similar to CoqIde (the current Coq graphical
interface, written in OCaml\footnote{Available via the
Coq distribution (\texttt{http://coq.inria.fr}).}). 
CoqEditor has an outline view,
which sums up the structures of the currently edited Coq file 
in a tree-like fashion (which is especially useful to see the modules hierarchy). 
CoqEditor has also an incremental indexing feature which allows the user
to jump directly from a keyword to its definition.
Of course, there is syntax highlighting for Coq files 
and one can interactively evaluate a Coq file.


\subsection{native specifications}
\begin{figure}[t!]
{\small In JML we define:
\begin{verbatim}
/*@ public native class IntList {
     public native IntList append (IntList l);
     public native static IntList create();
     ...
     public native static IntList toList (int [] tab);
   } @*/ \end{verbatim}}

{\small And in Coq:
\begin{verbatim}
Definition IntList := list t_int.
Definition IntList_create : IntList := nil .  
Definition IntList_append:   list t_int ->  list t_int -> list t_int := app.
... \end{verbatim}}
\caption{The definition of native type \texttt{IntList}}\label{CoqAnnot} 
\end{figure}
Sometimes users would like more expressivity: instead of having
tactics and extensions based solely on JACK's Java logic, a user may want to
build his own logic and use it with JML annotations. That is 
exactly the purpose of the \native construct \cite{Charles06}.

The \native construct is made to express some notions contained within
the annotation directly in the logic of the prover. It can be used 
as \native types and as  \native methods.
A \native method is a specification method that is like a {\tt pure} method 
but  it does not throw any exception and is used uniquely through the specifications. 
A \native type is a type to use with specification methods (\native methods as well as 
JML's {\tt model} methods). Both are references to a constructs 
defined in the proof obligation's target language: usually \native types are bounded to 
types and \native methods to functions definitions.


If we define a \native type IntList and we bind it to a list of integer type 
 (Figure \ref{CoqAnnot}), we can use the list library of the target prover (here Coq) 
in the proofs. The specifications of the
{\tt sort} method can be defined as specifications using our new type 
(Figure \ref{sortnat}). We get more readable annotations;
the annotations are more natural for the user because instead
 of relying upon Java logic over the array, we get it defined directly into
our target proof obligations language syntax.
The user can also define more easily lemmas to help prove some of the proof
obligations and to add automations for proof scripts.


\begin{figure}[t!]
{\small In JML we define:
\begin{verbatim}
//@ ghost IntList list;
/*@ requires (tab!=null) && list.equals(IntList.toList(tab))
    assignable tab[0 .. (tab.length -1)], list;
    ensures list.equals(IntList.toList(tab)) &&
             list.isSorted() && list.isPermutation();
 @*/
public void sort() {if(tab.length > 0) sort(0, tab.length -1);}\end{verbatim}}
\caption{\texttt{Sort} with natives}\label{sortnat} 
\end{figure}




