%
% $Id: idebug.tex,v 1.1.1.1 2002/12/29 12:36:07 kiniry Exp $
%

%\documentclass[twoside]{article}
\documentclass{article}
%\usepackage{warn}
\usepackage{html}
\usepackage{url}
%\usepackage{alltt}
%\usepackage{epsfig}
%\usepackage{times}

\title{IDebug:\\An Advanced Debugging Framework for Java\thanks{This
document describes IDebug version 1.0}}

\author{Joseph R. Kiniry\\
  Department of Computer Science,\\
  California Institute of Technology,\\
  Mailstop 256-80,\\
  Pasadena, CA 91125}

\date{September, 1998}

\begin{document}

\maketitle

%======================================================================
\begin{abstract}
  
  The IDebug debugging framework is an advanced debugging framework for
  Java.  This framework provides the standard core debugging and
  specification constructs such as assertions, debug levels and categories,
  stack traces, and specialized exceptions.  Debugging functionality can be
  fine-tuned to a per-thread and/or a per-class basis, debugging contexts
  can be stored to and recovered from persistent storage, and several
  aspects of the debugging run-time are configurable at the meta-level.
  Additionally, the framework is designed for extensibility.  Planned
  improvements include support for debugging distributed object systems via
  currying call stacks across virtual machine contexts and debug
  information logging with a variety of networking media including unicast,
  multicast, RMI, distributed events, and JavaSpaces. Finally, we are
  adding support for debugging mobile agent systems by providing mobile
  debug logs.

\end{abstract}

%=====================================================================
\section{Introduction}

Programming technologies have evolved greatly over the years. New
programming models have emerged, new languages have gained popularity,
new tools have been adopted, and yet several core debugging constructs
have not changed.  We believe that the two primary constructs for
general debugging are the \emph{execution trace} and the
\emph{assertion}.

%----------------------------------------------------------------------
\subsection{Object-Oriented Debugging}

Debugging object-oriented programs is not the same as debugging
procedural ones.  Because most object models enforce modularity and
encapsulation, one must test both the implementation and the interface
of a class.

A specification of an interface is called a
\emph{Contract}~\cite{HamiltonPowellMitchell93,HelmHollandGangopadhyay90,Holland92,Meyer88}.
A class's contract specifies the externally visible behavior that a
class guarantees.

Contracts are typically specified via three constructs:
\emph{preconditions}, \emph{postconditions}, and
\emph{invariants}.  Using these three constructs, the safety
properties of a class can be completely specified.

%----------------------------------------------------------------------
\subsection{Debugging in Java}

Surprisingly, given its popularity, the Java programming language
provides very few built-in constructs for debugging classes.

Typically, a Java programmer relies upon language features and
development tools for debugging.  Java provides array bounds checking,
static type checking, variable initialization testing, and exceptions
to assist in code debugging.  While programming environments provide
sophisticated source-code debuggers, most developers seem fixated on
using primitive \texttt{println}'s to debug their code.

Java is missing several traditional core debugging constructs, the
most critical of which is \emph{assertions}.  Assertions are program
statements of the form ``at this point in execution the following
\emph{must} be true''.  They are used to specify predicates that must
remain inviolate for a program to exhibit correct behavior.
Typically, if an assertion is violated, a program is aborted. In
object-oriented systems we often have options other than halting the
program execution (e.g. throwing an exception).

%----------------------------------------------------------------------
\subsection{Debugging Frameworks}

A \emph{framework} is a collection of classes that provides a unified
model and interface to a specific piece of functionality.  A framework
in Java is typically implemented as a collection of classes organized
into a \emph{package}.

%----------------------------------------------------------------------
\subsection{IDebug: A Debugging Framework}

The IDebug debugging framework is implemented as a set of Java Beans (Java
components) collected into the package \texttt{IDebug}.  These classes can
be used either (a) as ``normal'' classes with standard manual debugging
techniques, or (b) as components within visual Java Bean programming tools.

This package provides many standard core debugging and specification
constructs, such as the previously discussed assertions, debug levels
and categories, call stack, and specialized exceptions.  Debug levels
permit a developer to assign a ``depth'' to debug statements.  This
creates a lattice of information that can be pruned at runtime
according to the demands of the current execution.  Call stack
introspection is provided as part of the Java language specification.
The IDebug framework uses the call stack to support a runtime
user-configurable filter for debug messages based upon the current
execution context of a thread.  Finally, a set of specialized
exceptions are provided for fine-tuning the debug process.

Additionally, the framework supports extensions for debugging
distributed systems.  One problem typical of debugging distributed
systems is a loss of context when communication between two non-local
entities takes place.  E.g. When object $A$ invokes a method $m$ on
object $B$, the thread within $m$ does not have access to the call
stack from the calling thread in $A$.  Thus, the IDebug package
supports what we call \emph{call stack currying}.  Information such as
source object identity, calling thread call stack, and more is
available to the debugging framework on both sides of a communication.
Such information can be curried across arbitrary communication mediums
(sockets, RMI, etc.).

The IDebug package is also being extended to support the debugging of
mobile agent systems.  Mobile agent architectures can support
disconnected computing.  For example, an object $O$ can migrate from
machine $A$ to machine $B$, which might then becomes disconnected from
the network (i.e.  absolutely no communication can take place between
$B$ and $A$).  Since $B$ cannot communicate with $A$, and printing
debugging information on $B$'s display might not be useful or
possible, $B$ must log debugging information for later inspection.  To
support this functionality, the IDebug package will provide
serializable debug logs.  These logs can be carried around by a mobile
object and inspected at a later time by the object's owner or
developer.

The IDebug package is very configurable.  Debugging functionality can
be fine-tuned on a per-thread basis.  Each thread can have its own
debugging context.  The context specifies the classes of interest to
that particular thread.  I.e. If a thread $T$ specifies that it is
interested in a class $C$ but not a second class $D$, then debugging
statements in $C$ will be considered when $T$ is inside of $C$, but
debugging statements in $D$ will be ignored at all times.  These
debugging contexts can be stored to and recovered from persistent
storage.  Thus, ``named'' special-purpose contexts can be created for
reuse across a development team.

%=====================================================================
\section{Requirements}

In this section we will briefly present our project analysis including
our project concept dictionary, a review of our requirements for the
debugging package, and our goals.

%----------------------------------------------------------------------
\subsection{Project Dictionaries}

\begin{table}[htbp]
  \begin{center}
    \begin{tabular}{|l|p{4in}|}
        \hline
        \multicolumn{2}{|c|}{Project Dictionary}
        \\ \hline\hline

        \textbf{Assertion} & 
        An \emph{assertion} is a \emph{predicate} which states a logical
        sentence which evaluates to \emph{true} or \emph{false}.  The
        assertion is typically embedded in program code. An error is
        indicated if, during program execution, the assertion evaluates
        to false. There are three main types of assertions (see
        below): \emph{preconditions}, \emph{postconditions}, and
        \emph{invariants}. \\ \hline

        \textbf{Debug Context} & 
        A \emph{debug context} is a debugging frame of reference.
        More specifically, each thread of control within a
        component can have an independent debug context.  This
        context describes what types of debugging information are
        relevant to that specific thread. \\ \hline

        \textbf{Debug Semantics} & 
        \emph{Debug semantics} are the runtime behavior of the
        debug package, as exhibited by its reactions to
        exceptions, the language and text of its output
        messages, etc.\\ \hline

        \textbf{Invariant} & 
        A condition that must be true at all stable points in
        program execution. There are several types of invariants.
        A \emph{class invariant} is an assertion describing a
        property which holds for all instances of a class and,
        potentially, for all static calls to the class. A
        \emph{loop invariant} for a given loop is an assertion that
        is true at the beginning of the loop and after each
        execution of the loop body. \\ \hline

        \textbf{Output Interface} & 
        The debug package's \emph{output interface} is any
        legitimate output medium.  Example output interfaces
        include the system console, a shell window, a GUI, etc. \\ \hline

        \textbf{Postcondition} & 
        A condition that must be true at the end of a section
        of code. \\ \hline

        \textbf{Precondition} & 
        A condition that must be true at the beginning of a
        section of code. \\ \hline 

        \textbf{Predicate} & 
        Formally, a \emph{predicate} is something that is
        affirmed or denied of the subject in a proposition in
        logic. In other words, it is a logical sentence that
        evaluates to a boolean within specific contexts. \\ \hline

        \textbf{Variant} & 
        A \emph{variant} is a predicate that describes how
        state changes. A \emph{loop variant} is an assertion
        that describes how the data in the loop condition is
        changed by the loop.  Loop variants are used to
        check forward progress in the execution of loops (i.e.
        avoid infinite loops and other incorrect loop behavior). \\ \hline
    \end{tabular}
    \caption{Project Dictionary}
    \label{tab:dictionary}
  \end{center}
\end{table}

At the beginning of the project analysis phase, a dictionary of
concepts was developed so that all designers, developers, and users
would have a clear, common language.  The dictionary of terms is
included in Table~\ref{tab:dictionary} as well as in the design
diagrams directory of the framework deliverable.

Next, we will consider the core, application, and innovative
requirements we agreed upon before designing the IDebug framework.

%----------------------------------------------------------------------
\subsection{Core Requirements}
\label{sec:core}

We require that the IDebug framework support the following
requirements.  IDebug must, \emph{at the minimum}:

\begin{enumerate}
\item \emph{Provide an assertion mechanism.}  Assertions are the core
  construct of any debugging system.  Assertions can be intelligently inserted
  in program code and, if an assertion is violated, an error message is logged
  and a \texttt{AssertionFailedException} is thrown and the thread, and
  possibly the program halt.
\item \emph{Support the output of debugging messages.}  Printing
  miscellaneous debugging messages, perhaps outside the context of the
  primary interface of a component, is essential in a good debugging
  suite.
\item \emph{Support multiple debugging levels.} Different types of
  errors, messages, and situations require different levels of
  response.  An adequate debugging framework should not only support a
  set of debugging levels, but the set should be ordered so that user
  or developer-tunable filtering of debug output can take
  place\footnote{A filtering mechanism could be used instead, though
  is usually more tedious for the tester.}.
\item \emph{Complement the standard Java exception mechanism.} Since
  this is a debugging framework built for the Java language, it should
  work with, not against, the built in exception mechanisms.  In
  particular, prudent use of exception types (\texttt{Runtime} verses
  \texttt{Throwable}) is necessary so that the framework is not overly
  intrusive to the developer\footnote{I.e. If all exceptions were
  \texttt{Runtime} exceptions, the developer would have to bracket
  nearly all code with \texttt{try-catch} blocks.}.
\item \emph{Work with all development environments.}  IDebug must work
  with all development environments, from the most flashy IDE to the
  lowly CLI runtime.  This means that IDebug must be implemented as
  ``100\% Pure Java''; no proprietary extensions or native code may
  be used.
\end{enumerate}

%----------------------------------------------------------------------
\subsection{Application Requirements}

Because we build a wide array of Java applications and components, we
believe that IDebug should support debugging all types of Java
programs.  This means that the framework must provide debug
functionality that complements the following application types.  Each
type of application listed below is followed by a non-unique
implication of that particular application assumption.
\begin{enumerate}
\item \emph{Console-based applications.} Sometimes we will want to
  send messages to an output stream different from C's \textsc{stdout}
  or \textsc{stderr}.
\item \emph{Graphical user interface applications.} Occasionally, one
  wants to send debug messages to independent debugging windows or
  message sub-frames within a large application.
\item \emph{Console-less applications.} If there is no output channel,
  logging debug messages for later retrieval is an excellent course of
  action. 
\item \emph{Independent components (e.g. beans, servlets, doclets,
    etc.).} Independent components should be able to maintain
  independent debugging semantics and contexts.  Conversely, sometimes
  it is useful to have a compositional application share a debug
  context among its components.
\item \emph{Mobile agent/object applications.} If an application has
  mobile sub-components, their debug contexts need to be mobile as
  well, and debugging message output and/or storage should be
  location-independent and/or location-aware.
\item \emph{Distributed applications.} Distributed applications mean
  (at least) distributed control, distributed debugging context, and
  distributed debug messaging.
\end{enumerate}

If a debug package were to support all of the above application types,
we would consider it extremely powerful due to its flexibility.

%----------------------------------------------------------------------
\subsection{Innovative Requirements}
\label{sec:innovation}

Finally, we wish to support a set of innovative debugging
capabilities.  While most of these goals are independent of the target
language, they are facilitated by many of Java's more advanced
features.  The list of innovative requirements includes:
\begin{enumerate}
\item \emph{Support categorized debugging.} Debugging messages,
  errors, warnings, etc. should not only have a value (the debug
  level), but they should have a debug category (a classification).
\item \emph{Support per-class debugging.} A developer should be able
  to selectively turn debugging on or off at a per-class level.
\item \emph{Have a configurable runtime.}  We should not force
  developers to adopt our debugging semantics.  New semantics (debug
  ranges, base categories, etc.) should be configurable at runtime.
\item \emph{Support multiple output interfaces.} All debugging
  messages need not be sent to the same output channel.  E.g. Consider
  messages generated by UNIX's \texttt{syslog} facility.  Some
  messages are sent to the console, some are logged in a file, and
  some are sent directly to the system administrator via email.
\item \emph{Support per-thread debugging.} Each thread within a
  runtime should be able to construct its own debugging context.  More
  precisely, most of the above configurable options (debugging
  categories, classes, semantics, output interface, and level) should
  be configurable on a per-thread basis.  Additionally, these options
  should be changeable at runtime.
\item \emph{Support persistent debug contexts.} Once a debugging
  context is created, it should be possible to send it to persistent
  storage for later access.  This way, debugging contexts can not only
  be shared across sets of components, but they can be shared across
  groups of developers.
\end{enumerate}

If a debug framework were to support all of the above requirements, we
would simply be amazed\footnote{Hint: IDebug supports everything you
  have read so far.}.

Now that we have a common vocabulary and understand the problem domain
and the design goals, we'll consider a design for the debugging
framework.

%=====================================================================
\section{Design}

We will not discuss the full design of IDebug here due to space
considerations.  Interested parties should download and consult the
IDebug package or browse the information online via the IDebug release
\htmladdnormallinkfoot{web page}
{http://www.infospheres.caltech.edu/releases/}.  Only a few
interesting and/or important design points are discussed below.

%----------------------------------------------------------------------
\subsection{Context Configurability}

As mentioned previously, debugging options should be configurable on a
per-thread basis.  On further consideration, we decided that two
configurable settings should not be switchable at runtime: debug
semantics and output interface.

The reason for this decision might not be immediately obvious, but
consider the following two points:
\begin{itemize}
\item Debugging output might be queued due to the temporary
  unavailability of an output channel or user.
\item Source code that uses a debugging package makes explicit
  assumptions about the semantics of the package.  Meaning, while
  debugging semantics might be switchable at runtime by the
  \emph{framework}, it is not (usually) switchable at runtime for the
  application \emph{using} the framework.
\end{itemize}

Due to these factors, the configuration of debugging semantics and
output interface is immutable.  Meaning, once these options are set
for a debugging context, they cannot be changed.

Note that a new context can be \emph{created}.  All the other
flexibility mentioned in Section~\ref{sec:innovation} is fully
configurable at runtime on a per-thread basis.

Now, we'll briefly discuss the implementation and use of the IDebug
framework, version 2.0.

%=====================================================================
\section{Implementation}

IDebug is freely available from the \htmladdnormallinkfoot{KindSoftware's
  Open Source web pages}
{http://www.kindsoftware.com/products/opensource/}.

%----------------------------------------------------------------------
\subsection{Implementation Size and Performance}

\begin{table}[htbp]
  \begin{center}
    \begin{tabular}{|l|l|}
      \hline
      \multicolumn{2}{|c|}{Implementation Summary (with test and
        example code)}
      \\ \hline\hline
      \textbf{Total Number of Packages} & 2 \\ \hline
      \textbf{Total Number of Classes} & 12 \\ \hline
      \textbf{Total Number of KB of Java} & 97.9KB \\
      (includes code, documentation, and whitespace) & \\ \hline
      \textbf{Total Number of KB of classfiles} & \\
      Independent class files & 28.4KB \\
      Jar (compressed) format & 11.7KB \\ \hline
      \textbf{Total Number of Lines of Code}\footnote{Discuss 
        CommentCounter.} & 2069 \\ \hline
      \textbf{Total Number of Lines of Comments} & 1380 \\ \hline
      \textbf{Comments/Code} & 67\% \\ \hline
    \end{tabular}
    \caption{Implementation Summary}
    \label{tab:impl_summary}
  \end{center}
\end{table}

The implementations of both versions of IDebug are summarized in
Table~\ref{tab:impl_summary}.

\paragraph{IDebug Performance.} 
We have not yet performed performance tests on the IDebug package.  In
general, its performance is entirely based upon the speed of the Java
runtime's \texttt{Throwable.printStackTrace()} method and
\texttt{Hashtable} and \texttt{StringBuffer} implementations, since
these classes are at the core of the exception and assertion-handling
mechanisms in IDebug.

A performance profile test of IDebug could reveal performance
weaknesses. In general, any performance tuning would mean replacing
data structures, rathan than changing core algorithms.

In general, performance is not an issue in debugging complex systems,
especially distributed or object-oriented systems.  We make this claim
for two reasons:

First, the debugging phase of an implementation should be part of an
ordered and reasoned test suite, and thus the use of the debugging
framework should also be ordered and have reason.  In other words,
rarely will it be the case that all threads within a complex
application will have all their debugging options turned on
simultaneously.  

Second, we believe that debugging statements should not be written by
hand or statically inserted into program code.  Debug code
should be ``tunable'' at compile time, not just runtime, and thus
debug framework performance should only matter for critical debug
paths, of which there should be few.  

%----------------------------------------------------------------------
\subsection{Framework Extensibility}

The IDebug framework is extensible in two dimensions: debug semantics
and output interfaces.  

\paragraph{IDebug Framework Semantics.}
The semantics of the package can be changed by implementing new
versions of DebugConstantsInterface.  An example of such an extension
is provided in the form of the
\texttt{FrenchConstantsInterface} class in the
\texttt{IDebug.examples} package.  This class
provides an implementation of \texttt{DebugConstantsInterface} that
differs from the default implementation (\texttt{DefaultDebugConstants})
in two ways:
\begin{enumerate}
\item Debug levels range from 1 to 100 instead of 1 to 10, 
\item Default debugging levels have been adjusted for this new
  granularity of debug levels, and
\item Default debug messages, categories, and documentation are
  provided in French.
\end{enumerate}

\paragraph{IDebug Output Interfaces.}
New implementations of the \texttt{DebugOutput} interface can be
designed to support sending debug messages to alternative output
media/channels.  As of version 2.0, the framework comes with two
implementations: \texttt{ConsoleOutput}, which sends messages to the
console of a Java runtime, and \texttt{WriterOutput}, which sends
debug messages to a \texttt{Writer} which can be used as part of a
normal \texttt{java.io} compositional data stream.

%----------------------------------------------------------------------
\subsection{Complementary Tools}

Static debugging statements clutter source code, increase object code
size, and reduce execution speed.  We have developed a application
called JPP, the Java PreProcessor, that solves exactly this
problem.

In short, JPP performs transformations of embedded program
specification, in the form of design by contract\cite{Meyer92a} (DBC)
predicates in documentation comments, into IDebug test code at compile
time.  Future versions of JPP will also perform code beautification,
code standard conformance checking, code metric analysis, and
documentation generation.

%=====================================================================
\section{Conclusion}

IDebug is the most advanced debugging framework available today.  It
is extremely configurable, supports a wide range of Java application
types, and, because it is an open framework, is extensible by the
developer.

\paragraph{Future Work.}

In fact, we encourage developers to extend IDebug.  In particular, we
are interested in hearing about (and including) alternative
implementations of the \texttt{DebugOutput} interface and
\texttt{DefaultDebugConstants}.  We have come up with the following
alternative ideas for output interfaces; perhaps your application
could use one of these or one that we have not thought of:

\begin{itemize}
\item \texttt{DBOutput} --- used to log debugging messages to a
  database via standard JDBC.
\item \texttt{EventSourceOutput} --- send messages to arbitrary
  listeners within a Java virtual machine (perhaps as part of a
  compositional Java Beans-based application).
\item \texttt{FrameOutput} --- to send debugging messages to an
  arbitrary frame within a larger GUI.
\item \texttt{LogOutput} --- to persistently log messages for
  off-line debugging.
\item \texttt{MessageOutput} --- send messages via a
  JMS-conformant messaging infrastructure to a/some remote objects.
\item \texttt{RemoteEventSourceOutput} --- to provide debugging
  messages as distributed events (perhaps as part of a
  Jini\cite{JiniArchOverview98} application).
\item \texttt{ScrollableWindowOutput} --- display messages in an
  independent, scrollable window.
\item \texttt{ServletLogOutput} --- to persistently log messages via
  the servlet developers kit's debugging interface.
\item \texttt{SpaceOutput} --- store debugging events in a
  JavaSpace\cite{JavaSpaces98}.
\end{itemize}

Finally, we are investigating integrating IDebug with Dan Zimmerman's
\"UberNet distributed messaging infrastructure\cite{Zimmerman98}. Our
primary goal is to support the currying of call stacks across
execution contexts.  This would mean that assertions and exceptions on
remote (receiver) machines would have access to the call stack of the
sending thread.

\paragraph{Thanks.}
The author would like to thank the Infospheres Group for help with the
initial problem analysis and early IDebug design.  In particular, the
comments of Mani Chandy, Dan Zimmerman, Wesley Tanaka, and Adam Rifkin
were invaluable.  Also, Nelson Minar used the first version of IDebug
as part of his thesis work; his comments were very helpful.  Matt
Hanna helped review this technical report.  Finally, I'd like to thank
Ron Resnick, Mark Baker, Mary Baxter, and Cici Koenig for their
general support and encouragement in all I do.

%======================================================================

\bibliographystyle{plain}
\bibliography{abbrev,complexity,hypertext,knowledge,languages,meta,metrics,misc,modeling,reuse,softeng,specification,technology,theory,web,conferences}

%======================================================================
% Fin

\end{document}
