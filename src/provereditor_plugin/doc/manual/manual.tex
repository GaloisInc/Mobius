\documentclass[a4paper]{article}
%HEVEA \usepackage{hevea}

\newcommand{\eclipse}{
    \ahref{http://www.eclipse.org/}{Eclipse}}
\newcommand{\proverspec}{ prover specific documentation}
\title{ProverEditor v.0.1.0}

\begin{document}

\part* {ProverEditor v.0.1.0}




This editor is used to edit interactive theorem prover files within
\eclipse.
It is made to use with \ahref{http://coq.inria.fr}{Coq}, 
and in a near future PVS and other 
provers.
It uses a lightweight approach: easy to extend, 
with just a minimal set of commands.

For more technical informations  about it, see the
\ahref{http://gforge.inria.fr/projects/provereditor/}{gforge repository}.

%HEVEA\cutdef{subsection}
\section{Installation and requirements}
\subsection{Requirements}
\begin{itemize}
\item You need a recent Eclipse (preferred 3.1+)
\item A working Coq installation (with a proper ide, coqc and coqtop).
\end{itemize}
\subsection{Installation}
In eclipse go to {\tt Help $>$ Find And Install}; select the option 
{\tt Search for new feature to install}; click on Next; then click on 
{\tt New Remote Site...}, the address of the update site is
{\tt http://www-sop.inria.fr/everest/personnel/Julien.Charles/downloads/ProverEditorUpdateSite/}.
After that you are asked if you want to restart \eclipse. It is better
to answer yes to this question. 

\section{Basic Configurations}
These configurations are {\bf mandatory} if you want Prover Editor to work.
\subsection{Configuring the prover}
Go to {\tt Window $>$ Preferences...}, in the category {\tt ProverEditor}
select the prover you want to configure (right now only the choice of Coq 
is possible), and specify the full path to the binaries of the prover.
The grace time is the time Prover Editor will wait if the prover's top-level
is not responding.

%HEVEA \imgsrc{prover-configuration.jpg}

\subsection{Adding the toplevel view}
It is still not enough to make Prover Editor work. 
You have now to add the Top-Level view to your \eclipse's views. Go in 
 {\tt Window $>$ Show View $>$ Other...} and choose in the category other,
Prover Editor top-level view.

%HEVEA \imgsrc{view-configuration.jpg}

{\bf You're done ! now you can properly use Prover Editor !!!}.


\section{Some uses}

\subsection{Interactions}
\subsubsection{Views}
There are 3 view parts in the interface: 
\begin{itemize}
\item the editor, which is used to view a Coq file. It has syntax highlighting,
and when a part is evaluated, the part cannot be edited anymore (as in most 
interactive theorem prover editors).\\

%HEVEA \imgsrc{coqeditor.jpg}

\item the top-level, use to show the result of the interactions.\\

%HEVEA \imgsrc{toplevel.jpg}

\item the outline not shown for every prover, which able to view a summary
of the currently edited files.\\

%HEVEA \imgsrc{coqoutline.jpg}

\end{itemize}

\subsubsection{Evaluating a file}
All the interactions are done through the usse of a toolbar.\\
%HEVEA \imgsrc{toolbar.jpg}

From left to right, the meanings of the buttons: 
\begin{itemize}
\item the first one launch or reset Prover Editor on the currently read file. 

\item the second is used to progress through a proof
(Ctrl+Alt+Down Arrow)
\item the third is made to undo a step in a proof
(Ctrl+Alt+Up Arrow)
\item the fourth is made to try to progress until the end of the 
currently edited file
\item the fifth is bound to go back to the beginning of a file
\item the sixth one, is to cancel the current interaction with the 
prover. This last button is to use with the first button, as it is useful
to restart everything if we are in an inconsistent state of the editor
(it should not happen really often, but it is useful to be able to 
reset the editor state).
\end{itemize}
Some of these commands are binded to some keyboard shortcuts as detailed
in the next section.
\subsection{Keyboard Shortcuts}
\subsubsection{Set the ProverEditor keyboard shortcuts on}
To activate keyboard shortcuts in ProverEditor, you must go to
{\tt Window $>$ Preferences...} in the category {\tt Keys}, choose 
{\tt Modify} and change the scheme to {\tt Prover Editor}, as on the image. 
After, click on {\tt Apply} or {\tt Ok} to accept the changes.

%HEVEA \imgsrc{shortcuts.jpg}


\subsubsection{Shortcuts}
\begin{itemize}
\item Ctrl+Alt+Down Arrow: Progress in the proof
\item Ctrl+Alt+Up Arrow: Undo a step in the proof
\item Ctrl+Alt+Left: Jump to the previous sentence
\item Ctrl+Alt+Right: Jump to the next sentence
\item Ctrl+K: Delete until the end of line (like in Emacs)
\item F3: Find and highlight the definition of the word under the cursor
(only properly work on a CoqProject).
\end{itemize}


\subsection{Library paths}
If you right-click on a directory, ProverEditor add the item 
{\tt Add to loadpath}. It is used to tell to the prover to look into
the specified directories for its libraries.

%HEVEA \imgsrc{addloadpath.jpg}

This action adds an entry to the file {\tt .prover_paths}, which is then
passed to the ide, the compiler and the top-level. The exact behaviour and 
specifically how it is passed to  the different provers is specific to 
each plugins (see \proverspec).
The format of the file is one directory per line, with path relative
to the project path. To remove a directory just remove its line from 
the file.

\subsection{Pop-up menu}
If you right click on a file editable by Prover Editor (ie. a file from
a prover) Prover Editor add as a menu item a compile button as well as launch
IDE button. The compilation is done through the compiler which was specified 
in the prover configuration dialog, and the IDE is also 
the one specified for the prover configuration. The load path for the 
project is given to the compiler or the IDE.

%HEVEA \imgsrc{popup.jpg}




%HEVEA\cutend
\\ 
\begin{it}
ProverEditor will be part of the \ahref{http://mobius.inria.fr}{Mobius} 
tool suite.\\
\end{it}

\\
%HEVEA \imgsrc{screenshot1.jpg}


\begin{it}
This documentation is maintained since the first birthday of 
\ahref{http://gforge.inria.fr/projects/provereditor/}{ProverEditor} which
was once \ahref{http://www-sop.inria.fr/everest/personnel/Julien.Charles/papers/05-12-12-spops.ps}{CoqEditor for Eclipse}, september the 25th 2006.
\end{it}


\end{document}
