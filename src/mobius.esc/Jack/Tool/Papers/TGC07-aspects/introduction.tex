Java is a language commonly used on small device, with specific flavour
of it being used on heterogenous devices like cellphones or JavaCards. 
Thus taking into account their specific caracteristic each implementations
is important for verification.
Especially if specific system components change the overall behaviour of JVMs. 
Verification of system components is scattered through
litterature: there has been verifications of bytecode verifiers, as
well as access contoller. As far as our knowledge goes
though~\cite{HartelMoreau01}, the Security Manager of the JVM 
has not been verified and its semantic is not taken into
account when usually doing static verification of Java programs.


The main goal of this paper is to verify an implementation of a
Security Manager, and also to give a framework to more generally verify
programs taking into account the semantic differences that different
component of the JVM can bring to executions of program inside 
a JVM.

\vspace{-0.4cm}
\subsection{The Security Manager}
The Security Manager applies the security policies on two
levels. First on the library level: each time a writing or reading
operation is called for instance a call to the Security Manager is
made and if the caller has not respected a given security policy, a
{\tt SecurityException} is thrown.
Second on the JVM level: each time a class is
loaded, it checks on a meta level with the use of the ClassLoader
(which for this paper will be considered as a part of the JVM) if the
currently inspected class has the right to access a specified type
from an outside package.  
The latter is hard to take into account for a program verification, 
because it changes the behavior of type resolution which is part
of the language semantic.  A practical way to
model these changes would be to consider the Security Manager as an
Aspect which would weave at the cutting points representing type
resolution.
\vspace{-0.4cm}
\begin{figure}
\bcode
pa\=ckage a;\\
public class Main \{\+\\
  st\=atic \{\+\\
    Sy\=stem.setSecurityManager(new SecurityManager() \{\+\\
      pu\=blic void checkPackageAccess(String target) \{\+\\
        if\=(target.equals("b"))\\
          \>throw new SecurityException("That is true");\\
      \}\});\-\-\\
  \}\\
 \\
  public static void main(String[] args) \{\+\\
    System.out.println("Nextline will throw an exception");\\
    b.A a = new b.A();\-\\
  \}\-\\
\}
\ecode
The output of the program:
\bcode
Next line will throw an exception\\
Excepti\=on in thread "main" java.lang.SecurityException: That is true\+\\
	at a.b.Main\$1.checkPackageAccess(Main.java:9)\\
	\dots
\ecode
\caption{An invasive Security Manager}
\end{figure}

\vspace{-1cm}
\subsection{Modeling with AspectJ}
Aspect Oriented Programming (AOP) is a paradigm that offers
modularity though it is ortogonal to the usual Object Oriented
Programming paradigm. AOP enables to weave code directly into a
program,  changing the behaviour of given language constructs. Two new
notions have been introduced through AOP:
\begin{itemize}
\item the concept of cutting points, points in the program where code
can be inserted, and
\item advices, code to be inserted at a specified cutting point.
\end{itemize}
AspectJ is one of the most popular of the AOP languages. It is Java-based, 
so it is a natural choice for modelling the Security Manager through aspects.

Implementing the Security Manager as an invasive Aspect is easy.  As
we have seen in the above example, a minimal Security Manager could
change the behaviour of the JVM at the type access step. The
difficulty here would be that the type is checked only on the first
call, for this purpose a field has to be introduced, as shown in
Figure \ref{base_implem}.
%
\vspace{-0.4cm}
\begin{figure}
\bcode
pu\=blic aspect SecurityManager \{\+\\

Set$<$String$>$ s = new HashSet$<$String$>$();\\
po\=intcut anyPublicMethod(Object o) : \=target(o) \&\& !within(SecurityManager)\+ \\
           \>\&\& call( public *(*))\-\\
before(Object o) : anyPublicMethod(o) \{\+\\
    String pkg = o.getClass().getPackage().toString();\\
    if\=(!s.contains(pkg)) \{\+\\     
       s.add(pkg);\\
       if\=(pkg.equals("b"))\\
           \>throw new SecurityException("That is true");\\\-\\ 
    \}\-\\
\}\-\\
\}
\ecode
\caption{An Aspect implementation of the invasive SecurityManager}
\label{base_implem}
\end{figure}
\vspace{-1cm}
\subsection{Verification framework}
\label{framework}
The verification framework we will use to handle the Security Manager concerns 
is an adaptation of static verification techniques based on weakest
precondition calculus. It is inspired by Java extended static verification
tools that use guarded commands language like ESC/Java2~\cite{CokK04} and
the Mobius PVE~\cite{MobiusPVE07}.


The verification process is made of several steps:
First the program and the aspects have to be fully specified:
\begin{enumerate}
\item the program and its aspects are annotated using an aspect specific
behavioural specification language, Pipa,
\item the behavioural specifications are desugared
\item the aspects are abstracted to models
\end{enumerate}
Then the proper compilation of aspects is done:
\begin{enumerate}
\item the program is compiled to bytecode with its specifications
\item the program is transformed into guarded commands
\item the model methods representing advices are weaved into the program
\end{enumerate}
Finally a weakest precondition calculus is made on the transformed
program, and verification conditions are generated in order to be
solved automatically or interactively.

\vspace{-0.4cm}
\subsection{Related Work}
\label{related}
In this paper we focus on a verification framework adapted for the
verification of AspectJ programs annotated with Pipa.  The pointcut
semantic of AspectJ has been properly
defined~\cite{DBLP:conf/popl/AvgustinovHOMSTV07} as well as the advice
weaving semantic~\cite{weaving04} though only parts have been
formalized~\cite{weaving06}.  Pipa is an annotation language which was
inspired by Clifton and Leavens' work~\cite{clifton02observers}. There
are some extensions to it like pointcuts
annotations~\cite{pointcuts07}, or Moxa \cite{moxa05}.  Our framework
relies on verification based on an intermediate language, like what is
done in ESC/Java2~\cite{FlanaganLLNSS02}, Boogie~\cite{BarnettCDJL05},
or Krakatoa~\cite{MarcheP-MU04}. The work is made to be adapted on a
BoogiePL-like guarded command which is coupled with a
VCGen~\cite{BarnettL05,FlanaganS01}.

\paragraph{Modular verification of Aspects}
Verification in Aspect Oriented Programming language is often linked
to a modular approach.  It is orthogonal to the aim of this paperh: we
tailor verification of aspects which were not specially wanted by the
user, and are most-likely imposed by the environment.  Clifton and
Leavens~\cite{clifton02observers,clifton02spectators,cliftonPhd}
propose a programming discipline to define and specify aspects, using
an extension of JML as specification language.  They define two kind
of aspects, the \emph{spectators} and \emph{assistants} and propose to
explicitly allows the weaving of these aspects. This approach allows
efficient modular reasonning and easy implementations because they
show a way of weaving specifications using a control-flow graph
analysis. In their 2004 work~\cite{shriram04} Krishnamurthi {\it et
al} present a model-checking framework for verification of programs
containing aspects. It is quite different from our work as it presents
a generic framework for aspects. In an unpublished paper~\cite{cesar},
Kunz presents a Hoare logic for modular verification of aspects. The
paper is quite formal, but he does not add special annotations for method
specifications as Clifton and Leavens. For this reason his approach remains 
less modular than Clifton's, but more expressive.




% Shmuel Katz et al. \cite{Katz06,GoldmanK06} propose
% a classification of aspects as \emph{spectative}, \emph{regulative} or
% \emph{invasive}, to simplify program verification by focusing on the









\vspace{-0.4cm}
\subsubsection{Outline:}
First, in Section \ref{specs} we will show how the program has to be
 properly annotated. Then in Section \ref{transf} we will
transform the program to ease its verification: the aspects will be turned
into model methods and the main program code into guarded commands.
In section \ref{verif} we will explain how to verify the transformed program
and finally give a conclusion in Section \ref{conclusion}.

