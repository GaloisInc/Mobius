\label{Perspectives}
 At the moment, we have developed a prototype that is becoming a
 usable tool.  We are beginning experimentation outside the lab.  But
 we are considering that there are still many points where the
 approach can be improved.
%\subsection{JML Subset considered}
% JACK currently handles a subset of JML corresponding to lightweight
% specifications.  Future work should improve that subset by allowing
% to use a larger subset of JML including full fledged behavior
% specifications and pure method calls.

\subsubsection{Interactive proof cost}
 Currently, the interactive proof support is rather limited.  Thus,
 proving the remaining proof obligations requires users to directly
 use the \texttt{Atelier B} tool interactive prover.  Such a task can
 be tedious, since the B representation of the generated proof
 obligations can be hard to follow.  Different perspectives exist to
 reduce the interactive proof cost.

% AR: remplac� non-expert par Java: en effet je trouve que �a serais
% plus pratique m�me pour un expert.
\paragraph{Java interface}
 Although full interactive proof will still be reserved to experts,
 providing an interface to the \texttt{Atelier B} interactive prover
 allowing to perform proofs by directly using the Java syntax would
 greatly improve the productivity of those experts.  A first step
 would be to extend the Java view used to display the lemmas.

\paragraph{Test Activity}
 Another way to reduce interactive proof activity is to balance it
 with testing activity.  Methods where lemmas are not automatically
 proved could be tested.  Using \texttt{JmlJunit} can be a good way to
 generate test skeleton on certain cases.  A perspective is to
 integrate \texttt{JmlJunit} in our environment in manner to propose
 different validation level (full proof, proof mixed with test, etc).

\paragraph{Counter-example}
 Another idea to reduce the cost of false lemma detection is to
 provide counter-example when it is possible.  \ESC\ already tempts to
 give such counter-examples.  Studies on that subject could give
 results helpful for developers to understand errors on the code or on
 the specifications.

\subsubsection{Allowing expressions in the target formal language}
Currently, \JACK\ can be seen as a proof-obligations generator from
JML annotated Java programs to B lemmas.  A possible extension would
be to add different target languages for lemma generation, for
instance Coq or PVS. Such extension would allow using different
provers for different lemmas.  On the other side, it is also possible
to enrich the JML specification by adding inline expressions or
variables in a target language of the lemma generator. 
Such an extension would have a role similar to Java ``native methods''
at a specification level. That is, allowing to describe in a
lower-level language things that cannot be described in JML (or that
cannot be described efficiently from a proof point of view).

We are currently investigating such extension mechanisms, that would
allow adding different languages for such use without having to modify
the weakest-precondition calculus core.
