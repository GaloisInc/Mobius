\label{Conclusion}
 The presented work allows formal methods experts to prove Java applet correctness.
 Moreover, it allow Java programmers to obtain an high assurance on their code correctness.
 This leads to the most important point:
it allows non-experts to venture into the formal world.  This is a necessary starting point for such validation
techniques to be widely used.

  The tool has been developed following a main
 objective: let Java developers validate their own code.  We claim
 that JML is well suited to express low level design and conception
 choices and that usage of \JACK\ can replace effectively unitary tests,
 giving developers the means to furnish code with good quality
 and non ambiguous documentation.

 Taking benefits from recent research on Java program validation, we have
 developed an automated tool that generates lemmas from Java classes
 annotated with JML.
 The \JACK\ principles are not really different from the \LOOP\
 team choices.  Nevertheless, one can highlight some important
 differences.
 The \LOOP\ tool describes formally Java semantics and the WP-calculus rules are applied by the prover.
 The main advantage is that the rules have been proved sound with regards to the semantics in the theorem prover.
 Our point of view, is more pragmatic: lemma are generated automatically, using
 Java developed rules that can only be checked by usual validation: test, code-inspection.
 The soundness of the tool cannot be formally proved but on the other hand a big effort is done to present
 lemmas to users in a way that he can understand them and verify
 that they are valid.

 The tool is fully integrated in the eclipse IDE
 and presents lemmas in a visual way that allows developers to form their
 opinion on the lemma validity.  An automatic prover discharges an
 important part of the lemmas.  The remaining lemmas have to be proved
 using the prover interface.  Often, this task cannot be done by developers.
 Different ways are studied to bypass it: expert support, test case
 generation, counter-example detection, etc.

 We are now experimenting on real industrial products.  We are
 trying to collect metrics in order to prove that this kind of
 validation is cost-saving, especially when the cost of testing is
 taken into account.
