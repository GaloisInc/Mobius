
\documentclass[c]{beamer}


\usetheme{default}
\usepackage{listings}
%\usepackage{juanlespin}
\usefonttheme{structurebold}

\usepackage[english]{babel}
\usepackage{pgf}
\usepackage{epsfig}
\usepackage{amsmath,amssymb}
\usepackage[latin1]{inputenc}
\usepackage{multirow}
 
%%%%%%%%%%%%%%CARDIS%%%%%%%%%%%%%%%%%%%%%%
\usepackage{tikz}
\usepackage{graphics}
\usepackage{pgflibraryshapes}

\usepackage{listings}

\lstset{escapeinside={(*@}{@*)}}
\lstset{commentstyle=\color{blue!50!black}\textit,tabsize=2,keywordstyle=\color{green!50!black}}
\lstset{backgroundcolor=\color{lightgray!50}}
%\lstset{numbers=left, numberstyle=\footnotesize}


\setbeamertemplate{navigation symbols}{}
\newcommand{\textttbf}[1]{\texttt{\textbf{#1}}}


\newcommand{\varHook}[1]{\mbox{\slshape #1}}
\newcommand{\codeHook}[1]{\mbox{\ttfamily #1}}

\def \bsl       {\symbol{92}}
\newcommand{\unsc}{\_}

\setbeamercovered{dynamic}%\beamertemplatenavigationsymbolsempty

\title[]{Preliminary design of BML: A behavioral interface specification language for Java bytecode }
\author[mariela.pavlova@sophia.inria.fr]{Lilian Burdy, Marieke Huisman, \textbf{Mariela Pavlova}}

\date[]{Institute:  Ludwig-Maximilian University, Munich, Germany }
\pgfdeclareimage[height=3cm]{mobius_transparent}{mobius_transparent}
\logo{\pgfuseimage{mobius_transparent}}

\AtBeginSection[]{\frame{\frametitle{Outline}\tableofcontents[current]}}

\begin{document}

\begin{frame}
\titlepage
%\logo{mobius_transparent}
\end{frame}
 






%%%%%%%%%%%%%%%%%%%%%%%%%%%%%%%%%%%%%%%%%%%%%%%%%%%%%%%%%%%%%%%%%%%%%%%%%%%%%%%%%%%%%%%%%%%%%%%%%%%%%%
%%%%%%%%%%%%%%%%%%%%%%%%%%%%%%%%%%%%%%%%BYTECODE LANGUAGE%%%%%%%%%%%%%%%%%%%%%%%%%%%%%%%%%%%%%%%%%%%%%%%%%%%
%%%%%%%%%%%%%%%%%%%%%%%%%%%%%%%%%%%%%%%%%%%%%%%%%%%%%%%%%%%%%%%%%%%%%%%%%%%%%%%%%%%%%%%%%%%%%%%%%%%%%%%%%%%%
 \newcommand{\bcIns}{\mbox{ \rm I} }
 \newcommand{\instrs}{\mbox{ \rm IS} }
 \newcommand{\ifCond}{ \instr{if\_cond } }
 \newcommand{\goto}{ \instr{goto } }
 \newcommand{\return}{ \instr{return } }
 \newcommand{\arithOp}{ \instr{arith\_op} }
 \newcommand{\load}{ \instr{load} }
 \newcommand{\store}{ \instr{store} }
 \newcommand{\push}{ \instr{push} }
 \newcommand{\pop}{\instr{pop}}
 \newcommand{\dup}{\instr{dup}}
 \newcommand{\iinc}{\instr{iinc}}
 \newcommand{\new}{\instr{new}}
 \newcommand{\newarray}{\instr{newarray}}
 \newcommand{\putfield}{\instr{putfield}} 
 \newcommand{\getfield}{\instr{getfield}}
 \newcommand{\arrstore}{\instr{type\_astore}} 
 \newcommand{\arrload}{\instr{type\_aload}}
 \newcommand{\arraylength}{\instr{arraylength}}
 \newcommand{\instanceof}{\instr{instanceof}} 
 \newcommand{\checkcast}{\instr{checkcast}} 
 \newcommand{\athrow}{\instr{athrow}}
 \newcommand{\invoke}{\instr{invoke}}
 \newcommand{\jsr}{\instr{jsr}}
 \newcommand{\ret}{\instr{ret}}
%%%%%%%%%%%%%%%%%%%%%%%%%%%%%%%%%%%%%%%%%%%%%%%%%%%%%%%%%%%%%%%%%%%%%%%%%%%%%%%%%%%%%%%%%%%%%%%%%%%%%%
%%%%%%%%%%%%%%%%%%%%%%%%%%%%%%%%%%%%%%%%BYTECODE LANGUAGE%%%%%%%%%%%%%%%%%%%%%%%%%%%%%%%%%%%%%%%%%%%%%%%%%%%
%%%%%%%%%%%%%%%%%%%%%%%%%%%%%%%%%%%%%%%%%%%%%%%%%%%%%%%%%%%%%%%%%%%%%%%%%%%%%%%%%%%%%%%%%%%%%%%%%%%%%%%%%%%%



\newcommand{\todo}[1]{\marginpar{\baselineskip0ex\rule{2,5cm}{0.5pt}\\[0ex]{\textsf{#1}}}}
\newcommand{\fig}[1]{ Fig.#1}
\newcommand{\jmlKey}[1]{\mbox{\rm\textbf{#1}}}% wrapping jml keywords
\newcommand{\java}[1]{\texttt{#1}}
\newcommand{\stack}[1]{\mbox{\rm\textbf{st}}(#1)}% element on top stack 
\newcommand{\stackOnly}{\mbox{\rm St}}
\newcommand{\stackOnlyParam}[1]{\mbox{\rm St}(#1)}
\newcommand{\newStack}{ \lbrack~\rbrack  } % empty stack
\newcommand{\update}[3]{ #1 [ \oplus {#2} \rightarrow {#3} ] }

\newcommand{\counter}{\mbox{\rm\textbf{cntr}} }
\newcommand{\counterOnly}{\mbox{\rm Cntr} }
\newcommand{\topStack}{\mbox{\rm cntr} }
\newcommand{\true}{\mbox{\rm\textbf{true}} } % true from the assertion language
\newcommand{\false}{\mbox{\rm\textbf{false}}} % false from the assertion language



\newcommand{\instr}[1]{\mbox{  \rm #1} }

%\newcommand{\loopStart}[1]{\textbf{loop\_start$\tt{_#1}$}}
%\newcommand{\loopEnd}[1]{\textbf{loop\_end$\tt{_#1}$}}
%\newcommand{\invariant}[1]{\it{I}_{\tt{#1}}}
%\newcommand{\boucle}[1]{\texttt{#1}}
%\newcommand{\loopModifies}[1]{\textbf{Modifies}(\texttt{#1}) }

\newcommand{\ins}[1]{instr_{#1}}
\newcommand{\insOnly}{\texttt{instr}}

% the grammar for the bytecode specification language
%\newcommand{\ClassSpec}{\rm{ClassSpec}}
%\newcommand{\ClassInv}{ClassInv}
%\newcommand{\ClassHistoryConstr}{ClassHistoryConstr}

%\newcommand{\MethodSpec}{\rm{MethodSpec}}


%\newcommand{\specCase}{\textrm{SpecCase}}
%\newcommand{\requires}{requires}
%\newcommand{\ensures}{ensures}
%\newcommand{\exsures}{exsures}
%\newcommand{\modifies}{modifies}

%\newcommand{\jmlStmt}[1]{\textrm{#1}}
%\newcommand{\specExpression}{\mathcal{E}^{spec}}

%\newcommand{\interMethodSpec}{\rm{InterMethodSpec}}
%\newcommand{\loopSpec}{\rm{loopSpec}}
%\newcommand{\assert}{\rm{assert}}

%\newcommand{\ArithExpr}{\texttt{ArithmeticExp} }

%\newcommand{\JMLExpr }{\texttt{JmlExp} }


\newcommand{\integer}{\texttt{int} }
\newcommand{\register}[1]{\mbox{\rm\textbf{reg}}(#1)}

\newcommand{\Mynull}{\jmlKey{null}}
\newcommand{\this}{\texttt{this}}
\newcommand{\fieldAccess}[2]{#1\mbox{\rm\textbf{.}}#2}
\newcommand{\arrayAccess}[2]{\mbox{\rm\textbf{arrayAccess}}( #1, #2) }  %{\mbox{\rm arrAccess}(#1, #2)}
\newcommand{\arrayAccessOnly}{\mbox{\rm arrAccess}}


%\newcommand{\loopInv}{loopInvariant}
%\newcommand{\loopMod}{loopModifies}

\newcommand{\result}{\jmlKey{$\backslash$result}}
%\newcommand{\old}[1]{\jmlKey{$\backslash$old(}#1\jmlKey{)}}
%\newcommand{\typeof}[1]{\jmlKey{$\backslash$typeof(}#1 \jmlKey{)}}
%\newcommand{\TYPE}{\jmlKey{TYPE} }
%\newcommand{\elemtype}[1]{ \jmlKey{$\backslash$elemtype(}#1\jmlKey{)} } 
%\newcommand{\excPost}{\psi^{exc}}

\newcommand{\Myspace}{\phantom{aaa}}
\newcommand{\predicate}{ \mathcal{P}} 
\newcommand{\Myfalse}{ \textit{false} }
\newcommand{\Mytrue}{ \textit{true} }



% abstractCtrlFlow.tex
\newcommand{\execRel}{\rightarrow} % the execution relation
\newcommand{\blockm}[1]{ \tt{b^{#1}} }
\newcommand{\blockSeq}[1]{ \tt{b_{seq}^{#1}} }
\newcommand{\pathm}[2]{\blockm{#1} \execRel^{*} \blockm{#2} }
\newcommand{\instrPost}[1]{ post(\instr{#1} )}
\newcommand{\invariant}{\textit{I}}




%from wp.tex
%\newcommand{\wpExeWithLoops}[1]{ \rm{Wp''}\rm{(#1)} }
%\newcommand{\wpExe}[1]{ \rm{Wp'}\rm{(#1)} }




\newcommand{\excPost}{\psi^{exc}}
\newcommand{\javaNull}{null}
\newcommand{\length}{\mbox{\rm arrLength}} % stands for array length


%%%%%%%%%%%%%%%%%%%%%%%%%%%%%%%%%%%%%%%%%%%%%%%%%%%%%%%%%%%%%%%%%%%%%%%%%%%%%%%%%%%%%%%%%%%%%%%%%%%%%%%%%%%%%%%%%%%%%%%%%%%%%%%%%%%%%%%%%%%%%%%%%%%%%%%%%%%%%%%%%%%%%%%%%%%%%%%%%%%%%%%%%%%%%%%%%%%%%%%%%%%%%%%%%%%%%%%%%%%%%%%%%%%%%%%%%%%%%%WP functions%%%%%%%%%%%%%%%%%%%%%%%%%%%%%%%%%%%%%%%%%%%%%%%%%%%%%%%%%%%%%%%%%%%%%%%%%%%%%%%%%%%%%%%%%%%%%%
%%%%%%%%%%%%%%%%%%%%%%%%%%%%%%%%%%%%%%%%%%%%%%%%%%%%%%%%%%%%%%%%%%%%%%%%%%%%%%%%%%%%%%%%%%%%%%%%%%%%%%%%%%%%%%%%%%%%%%%%%%%%%%%%%%%%%%%% 

\newcommand{\wpi}[3]{\mbox{\rm\textit{wp}}( #1, #2 , #3) }
\newcommand{\fwpi}{\mbox{\rm\textit{wp}}}
\newcommand{\inter}[2]{\mbox{ \rm \textit{inter}}(#1, #2, \methodd)} % predicate that holds between two bytecode blocks
\newcommand{\interOnly}{\mbox{ \rm \textit{inter}}} % the name of the function that calculates predicate that holds between two bytecode blocks
\newcommand{\objects}{\texttt{Objects}}

%%%%%%%%%%%%%%%%%%%%%%%%%%%%%%%%%%%%%%%%%%%%%%%%%%%%%%%%%%%%%%%%%%%%%%%%%%%%%%%%%%%%%%%%%%%%%%%%%%%%%%%%%%%%%%%%%%%%%%%%%%%%%%%%%%%%%%%%%%%%%%%%%%%%%%%%%%%%%%%%%%%%%%%%%%%%%%%%%%%%%%%%%%%%%%%%%%%%%%%%%%%%%%%%%%%%%%%%%%%%%%%%%%%%%%%%%%%%%%Heap%%%%%%%%%%%%%%%%%%%%%%%%%%%%%%%%%%%%%%%%%%%%%%%%%%%%%%%%%%%%%%%%%%%%%%%%%%%%%%%%%%%%%%%%%%%%%%%%%%%%%%%%%%%%%%%%%%%%%%%%%%%%%%%%%%%%%%%% 
%%%%%%%%%%%%%%%%%%%%%%%%%%%%%%%%%%%%%%%%%%%%%%%%%%%%%%%%%%%%%%%%%%%%%%%%%%%%%%%%%%%%%%%%%%%%%%%%%%%%%%%%%%%%%%%%%%%%%%%%%%%%%%%%%%%%%%%% 
\newcommand{\heap}{\mbox{\rm H}}
 \newcommand{\HeapSet}{\mbox{\rm \textsf{HeapType}}}
 \newcommand{\heapFields}{\mbox{\rm \textsf{Fld}}}
 \newcommand{\heapArrays}{\mbox{\rm \textsf{Arr}}}
 \newcommand{\heapLocs}{\mbox{\rm \textsf{Loc}}}
 \newcommand{\heapTypeOf}{\mbox{\rm \textsf{TypeOf }}}
 

\newcommand{\pc}{\mbox{\rm Pc}}
\newcommand{\field}[2]{\texttt{f}_{#2}^{#1}}
\newcommand{\Values}{\mbox{\rm\textit{Values}}}
\newcommand{\locVar}[1]{\mbox{\rm{\textbf{reg}}}(#1) }
\newcommand{\locVarOnly}{\mbox{\rm Reg}}


\newcommand{\AllRefs}{\mbox{\rm\textit{RefType}}} % the set all references  - reff \cup \reffArr  
\newcommand{\reff}{\mbox{\rm\textit{RefClType} }} % the type of simple reference
\newcommand{\reffArr}{\mbox{\rm\textit{RefArrType}}}
\newcommand{\Ref}[1]{\mbox{ \rm \texttt{ref}}_{#1}}
\newcommand{\reference}[1]{\mbox{ \rm \texttt{ref}}_{#1}}
\newcommand{\referenceOnly}{\mbox{ \rm \texttt{ref}} \ } 
\newcommand{\RefArr}[1]{\tt{refArr}_{#1} } % a reference to an array  ref_{type}^{length}
\newcommand{\substitution}[3]{#1 \lbrack #2 \leftarrow #3 \rbrack }
\newcommand{\subst}[2]{ \lbrack #1 \leftarrow #2 \rbrack }
\newcommand{\prevState}[1]{prev(#1)}
\newcommand{\nextState}[1]{next(#1)}


\newcommand{\RefValuesArr}{\mbox{\rm\textit{RefValArr}}}

\newcommand{\comment}[1]{ \{ \textit{ #1} \} }



\newcommand{\numConclusion}[1]{\textit{(#1)}}
\newcommand{\valueAtState}[2]{\it{val}_{#1}(#2)}

\newcommand{\stateTrans}{\hookrightarrow}
\newcommand{\stateTransTransClos}[1]{\longleftarrow_{#1}}

%%%%%%%%%%%%%%%%%%%%%%%%%%%%%%%%%%%%%%%%%%%%%%%%%%%%%%%%%%%%%%%%%%%%%%%%%%%%%%%%%%%%%%%%%%%%%%%%%%%%%%%%%%%%%
%%%%%%%%%%%%%%%%%%%%%%STATE CONFIGURATION%%%%%%%%%%%%%%%%%%%%%%%%%%%%%%%%%%%%%%%%%%%%%%%%%%%%%%%%%%%%%%%%%%%%
%%%%%%%%%%%%%%%%%%%%%%%%%%%%%%%%%%%%%%%%%%%%%%%%%%%%%%%%%%%%%%%%%%%%%%%%%%%%%%%%%%%%%%%%%%%%%%%%%%%%%%%%%%%%%
\newcommand{\configVar}{\mbox{\rm \textit{K}}}
\newcommand{\config}[5]{<{#1},{#2},{#3},{#4},{#5}>} % <\heap, \counter, \stackOnly, \locVarOnly , \pc>
\newcommand{\configFinalNorm}[3]{<{#1},{#2},{#3}>^{norm} }% the final configuration for a method's operational semantics   <\heap, RetValue> 
\newcommand{\configFinalExc}[3]{<{#1},{#2},{#3}>^{exc}}% the final configuration for a method's operational semantics   <\heap, excValue> 
\newcommand{\configFinal}[3]{<{#1},{#2},{#3}>^{final}} % config^final = config^exc + config^norm
\newcommand{\Final}{\mbox{\rm \textit{Final}}} % stands for result or the thrown exception
\newcommand{\SetConfigs}{\mbox{\rm \textit{K}}}
\newcommand{\SetConfigInterm}{\mbox{\rm \textit{K}}^{interm}}
\newcommand{\SetConfigFinal}{\mbox{\rm \textit{K}}^{final}}
\newcommand{\SetConfigFinalNorm}{\mbox{\rm \textit{K}}^{norm}}
\newcommand{\SetConfigFinalExc}{\mbox{\rm \textit{K}}^{exc}}

\newcommand{\termination}{\mbox{\rm End}}
\newcommand{\Res}{\mbox{\rm Res}}
\newcommand{\Exc}{\mbox{\rm Exc}} % the third component of a terminating configuration in  case of an exception

\newcommand{\eval}[2]{ #1 \vDash #2 } % \tau (expr )
\newcommand{\conf}[1]{\tt{<#1>}} % <\tau>

\newcommand{\bottom}{\bot}
%\newcommand{\pstate}[2]{<#1,#2> }
\newcommand{\RefValues}{\mbox{\rm\textit{RefVal}}}
\newcommand{\retValue}[1]{\textrm{returnVal}(#1)} % designates the result of the method
\newcommand{\objCl}[1]{\tt{Obj}_{#1}}% object representing a class instance 
\newcommand{\objArr}[2]{\tt{ObjArr}_{#1}^{#2}} % obj_{type}^{length} object representing an array instance
\newcommand{\modExp}{modExp}% stands for modfied locations by loops and methods


\newcommand{\method}{\mbox{ \rm m}~}
\newcommand{\excIndex}[2]{\mbox{ \rm \texttt{excIndex}}(#1,#2 ) }


%classFileExt.tex
\newcommand{\Myint}{\mbox{\rm\textbf{int}}}
%\newcommand{\intLiteral}{\texttt{int\_const} }
\newcommand{\predicates}{ \mathcal{R} }

\newtheorem{Formula}{Formulas}
\newtheorem{Predicate}{Predicates}

\newcommand{\formulaBc}{\mathcal{P}_{bml}}




%%%%%%%%%%%%%%%%%%Exception Types%%%%%%%%%%%%%%%%%%%%%%%%%5
\newcommand{\excType}{\mbox{\rm \textit{ExcType}}}
\newcommand{\NullPointerExc}{\mbox{ \rm \texttt{NullPntrExc}}}
\newcommand{\NegativeArraySizeExc}{\mbox{ \rm \texttt{NegArrSizeExc}}  }
\newcommand{\ArrIndexOutOfBoundExc}{\mbox{ \rm \texttt{ArrIndBndExc}} }
\newcommand{\ArithExc}{\mbox{ \rm \texttt{ArithExc}} }
\newcommand{\ClassCastExc}{\mbox{\rm \texttt{CastExc}}}
\newcommand{\Throwable}{\mbox{\rm \texttt{Throwable}}}
\newcommand{\ArrStoreExc}{\mbox{\rm \texttt{ArrStoreExc}} }

%%%%%%%%%%%%%%%%%%%%%%%%%%%%%%%%%%%%%%%%%%%%%%%%%%%%%%%%%%%%%%%%%%%%%%%%%%%%%%%%%%%%%%%%%%%%%%%%%%%%%%%%%%%%%
%%%%%%%%%%%%%%%%%%%%%%Class Fields Methods%%%%%%%%%%%%%%%%%%%%%%%%%%%%%%%%%%%%%%%%%%%%%%%%%%%%%%%%%%%%%%%%%%%%%%%%%%%%%%%%
%%%%%%%%%%%%%%%%%%%%%%%%%%%%%%%%%%%%%%%%%%%%%%%%%%%%%%%%%%%%%%%%%%%%%%%%%%%%%%%%%%%%%%%%%%%%%%%%%%%%%%%%%%%%%
 \newcommand{\class}{\mbox{ \rm Cl } }

 \newcommand{\FieldSet}{\mbox{\rm\textbf{Field}} } % the set of fields
 \newcommand{\FieldName}{\mbox{\rm\textbf{FieldName}}}
 \newcommand{\MethodSet}{\mbox{\rm\textbf{Method}} }
 \newcommand{\LoopSpecSet}{\mbox{\rm\textbf{LoopSpec}} }
 \newcommand{\MethodName}{\mbox{\rm\textbf{MethodName}} }
 \newcommand{\isField}[2]{\mbox{\rm\textsf{isfield}}(#1,#2)}
 \newcommand{\ClassSet}{\mbox{\rm\textbf{Class}}} % the  set of fields
 \newcommand{\ClassName}{\mbox{\rm\textbf{ClassName}}} % the set of fields
 
 \newcommand{\clazz}{\mbox{\rm \textit{C}}}
 \newcommand{\fieldd}{\mbox{\rm \textit{f}} }
 \newcommand{\methodd}{\mbox{\rm \texttt{m}}}

 %%%%%%%%%%%%%%%%%Class attributes %%%%%%%%%%%%%%%%%%%%%%%%%%%%%%%%%%%%%%%%%%%%%%%%%%%%%%%%%5
 \newcommand{\fields}{\mbox{\rm \textsf{fields}}}
 \newcommand{\methods}{\mbox{\rm \textsf{methods}}}
 \newcommand{\className}{\mbox{\rm \textsf{className}}}
 \newcommand{\superClass}{\mbox{\rm \textsf{superClass}}}
 \newcommand{\classCP}{\mbox{\rm \textsf{constantPool}}}
 %%%%%%%%%%%%%%%%%Field attributes%%%%%%%%%%%%%%%%%%%%%%%%%%%%%%%%%%%%%%%%%%%%%%%%%%%%%%%%%5
 \newcommand{\fieldName}{\mbox{\rm \textsf{Name}}}
 \newcommand{\fieldType}{\mbox{\rm \textsf{Type}}}
 \newcommand{\declaredIn}{\mbox{\rm \textsf{declaredIn}}}

 %%%%%%%%%%%%%%%%%Method attributes%%%%%%%%%%%%%%%%%%%%%%%%%%%%%%%%%%%%%%%%%%%%%%%%%%%%%%%%%5
 \newcommand{\methodName}{\mbox{\rm\textsf{Name}}}
 \newcommand{\retType}{\mbox{\rm\textsf{retType}}}
 \newcommand{\args}{\mbox{\rm\textsf{args}}}
 \newcommand{\numArgs}{\mbox{\rm\textsf{nArgs}}}
 \newcommand{\body}{\mbox{\rm\textsf{body}}}
 \newcommand{\entryPoint}{\mbox{\rm\textsf{entryPnt}}}
 \newcommand{\excHandlerTable}{\mbox{\rm\textsf{excHndlS}}}
 \newcommand{\exceptions}{\mbox{\rm\textsf{exceptions}}}
\newcommand{\methodLocVar}{\mbox{\rm\textsf{locVars}}}

\newcommand{\excPostSpec}{\mbox{\rm\textsf{excPost}}} % the postcondition that is specified in case the method ends with an exception exc 
\newcommand{\loopSpecTable}{\mbox{\rm\textsf{loopSpecS}}}
\newcommand{\normalPost}{\mbox{\rm\textsf{normalPost}}}
\newcommand{\pre}{\mbox{\rm\textsf{pre}}}
\newcommand{\modif}{\mbox{\rm\textsf{modif}}}

%%%%%%%%%%%%%%%%%Loop attributes%%%%%%%%%%%%%%%%%%%%%%%%%%%%%%%%%%%%%%%%%%%%%%%%%%%%%%%%%
\newcommand{\loopSpec}{\mbox{\rm\textit{loopSpec}}}



%%%%%%%%%%%%%%%%%Intra spec attributes%%%%%%%%%%%%%%%%%%%%%%%%%%%%%%%%%%%%%%%%%%%%%%%%%%%%%%%%%
\newcommand{\intraSpec}{\mbox{\rm\textit{assertion}}}
\newcommand{\atIndex}{\mbox{\rm\textbf{atIndex}}}


%%%%%%%%%%%%%%%%Exception handler%%%%%%%%%%%%%%%%%%%%%%%%%%%%%%%%%%%%%%%%%%%%%%%%%%%%%%%%%%%%%%%%%%%%
 \newcommand{\ExcHandler}{\mbox{\rm \textbf{ExcHandler}}}
 \newcommand{\excHH}{\mbox{\rm\textsf{excH}}} % stands for an instance of an ExceptioHandler type 
 \newcommand{\pcStart}{\mbox{\rm \textsf{startPc}}} 
 \newcommand{\pcEnd}{\mbox{\rm \textsf{endPc}}}
 \newcommand{\pcHandler}{\mbox{\rm \textsf{handlerPc}}}
  \newcommand{\exc}{\mbox{\rm \textsf{exc}}}



%%%%%%%%%%%%%%%%%%%%%%%%%%%%%%%%%%%%%
%%%%%%%%%%%%%%%%HEAP%%%%%%%%%%%%%%%%%%%%%
%%%%%%%%%%%%%%%%%%%%%%%%%%%%%%%%%%%%%
 \newcommand{\referenceDef}{\mbox{\rm ! }} % a function that decides if a reference exists and points to an existing object
 \newcommand{\referenceType}{\mbox{\rm isOfType} } % a function that decides that the reference is of some type 
 \newcommand{\JavaClass}{\mbox{\rm \textit{JClass}}} % the set of Java classes
 \newcommand{\JavaType}{\mbox{\rm\textit{JType}}} % the set of Java classes

 
 \newcommand{\isInList}[2]{\mbox{\rm \textit{inList}}(#1, #2)\ }
 \newcommand{\isInListOnly}{\mbox{\rm \textit{inList}}  }
 \newcommand{\addInList}[2]{\mbox{\rm cons}(#2, #1) } 
 \newcommand{\addInListOnly}{\mbox{\rm cons} } 
 \newcommand{\intersectOnly}{ \cap }
 \newcommand{\intersect}[2]{#1 \cap #2}
 \newcommand{\getFreshRef}[2]{\mbox{\rm getFreshRef}(#1,#2)}
 \newcommand{\getFreshRefOnly}{\mbox{\rm getFreshRef} \ }

 \newcommand{\newRef}[2]{\mbox{\rm newRef}(#1,#2 ) } % creates a new reference in the store
 \newcommand{\newRefOnly}{\mbox{\rm newRef}}
 \newcommand{\newArrRef}[3]{\mbox{\rm newArrRef}(#1,#2,#3) \ } % creates a new reference in the store
 \newcommand{\newArrRefOnly}{\mbox{\rm newArrRef}}

 \newcommand{\getLocations}[1]{\mbox{\rm getLoc}(#1) \ }
 \newcommand{\getLocationsOnly}{\mbox{\rm getLoc} \ } 
 \newcommand{\addNewLocation}[2]{\mbox{\rm allocator}(#1,#2) \ }
 \newcommand{\addNewLocationOnly}{\mbox{\rm allocator} \ }
 \newcommand{\defaultValue}[1]{\mbox{\rm\textsf{defVal}}(#1)} % default value for a type
 \newcommand{\defaultValueOnly}{\mbox{\rm \textit{defVal}}}
 \newcommand{\instanceFlds}[2]{\mbox{\rm \textit{instFlds}}(#1, #2)} % isInstField (field, class )
 \newcommand{\instanceFldsOnly}{\mbox{\rm \textit{instFlds}}}
 \newcommand{\InDom}[2]{\mbox{\rm\textit{inDom} } (#1,#2)} 
 \newcommand{\anyType}{\mbox{\rm \texttt{T}}} % represents any Java Type
 \newcommand{\Arrays}{\mbox{\rm ARR}} % an abstraction for all arrays in the heap

 %%%%%%%%%%%%%%%%%%% the state after  exception is thrown %%%%%%%%%%%%%%%%%%%%%%%%%%%%
 \newcommand{\getStateAfterExc}{\mbox{\rm \textit{getStateOnExc}\ }} 
 \newcommand{\findExcHandler}[3]{\mbox{\rm \textit{findExcHandler}}(#1,#2,#3)} % #1 = exception, #2 = pc , #3 = exception handler table 
 \newcommand{\findExcHandlerOnly}{\mbox{\rm\textit{findExcHandler}}\ } % #1 = exception, #2 = pc , #3 = exception handler table 



%%%%%%%%%%%%%%%%%%%%%%%%%%%%%%%%%%%%%%%%%%%%%%%%%%%%%%%%%%%%%%%%%%%%%%%%%%%%%%%%%%%%%%%%%%%
%%%%%%%%%%%%%%%%%%%%%%%%%%%%%%%%%Subtyping and types%%%%%%%%%%%%%%%%%%%%%%%%%%%%%%%%%%%%%%%%%%%%%%%%%%%%%%%%%%
%%%%%%%%%%%%%%%%%%%%%%%%%%%%%%%%%%%%%%%%%%%%%%%%%%%%%%%%%%%%%%%%%%%%%%%%%%%%%%%%%%%%%%%%%%% 
\newcommand{\subtypeOnly}{\mbox{\rm \textsf{subtype}}} 
\newcommand{\subtype}[2]{\mbox{\rm \textsf{subtype} }(#1,#2)}
 \newcommand{\Object}{\mbox{\rm \texttt{Object}} }


%%%%%%%%%%%%%%%%%%%%%%%%%%%%%%%%%%%%%%%%%%%%%%%%%%%%%%%%%%%%%%%%%%%%%%%%%%%%%%%%%%%%%%%%%%%
%%%%%%%%%%%%%%%%%%%%%%%%%%%%%%%%%constant pool%%%%%%%%%%%%%%%%%%%%%%%%%%%%%%%%%%%%%%%%%%%%%%%%%%%%%%%%%%
%%%%%%%%%%%%%%%%%%%%%%%%%%%%%%%%%%%%%%%%%%%%%%%%%%%%%%%%%%%%%%%%%%%%%%%%%%%%%%%%%%%%%%%%%%% 

\newcommand{\constantPool}{ \mbox{\rm\textbf{CP}}}
\newcommand{\localVariableTable}{\mbox{\rm\textbf{LV}}} 
\newcommand{\locVarEls}{\mbox{\rm\textbf{localVarTableElem}}}% an element of the local variable table
\newcommand{\lineNumberTable}{\mbox{\rm\textbf{LN}}} 
%%%%%%%%%%%%%%%%%%%%%%%%%%%%%%%%%%%%%%%%%%%%%%%%%%%%%%%%%%%%%%%%%%%%%%%%%%%%%%%%%%%%%%%%%%%
%%%%%%%%%%%%%%%%%%%%%%%%%%%%%%%%%BML%%%%%%%%%%%%%%%%%%%%%%%%%%%%%%%%%%%%%%%%%%%%%%%%%%%%%%%%%%
%%%%%%%%%%%%%%%%%%%%%%%%%%%%%%%%%%%%%%%%%%%%%%%%%%%%%%%%%%%%%%%%%%%%%%%%%%%%%%%%%%%%%%%%%%% 
\newcommand{\nonterminal}{ \mbox{\rm\textit{italics}} }
\newcommand{\terminal}{ \mbox{\rm\textbf{boldface}} }
\newcommand{\keyWord}{ \mbox{\rm\textsf{sans serif}} }

\newcommand{\expression}{\mbox{\rm\textit{E}}_{bml} }
\newcommand{\typeExp}{\mbox{\rm\textit{T}}_{bml}}
\newcommand{\expressions}{\mbox{\rm\textit{SpecExp}}_{bml}}

\newcommand{\Constants}{\mbox{\rm\textit{constants}}_{bml}}
\newcommand{\intLiteral}{\mbox{\rm\textit{intLiteral}}}
\newcommand{\signedInt}{\mbox{\rm\textit{signedIntLiteral}}}
\newcommand{\ident}{\mbox{\rm\textit{ident}}}
\newcommand{\idRef}{\mbox{\rm\textit{idRef}}}
\newcommand{\digit}{\mbox{\rm\textit{digit}}}
\newcommand{\digits}{\mbox{\rm\textit{digits}}}
\newcommand{\nonZeroDigit}{\mbox{\rm\textit{nonZerodigit}}}
\newcommand{\boundVar}{\mbox{\rm\textit{boundVar}}}
\newcommand{\bound}{\mbox{\rm\textbf{bv}}}
\newcommand{\unsignedInt}{\mbox{\rm\textit{unsignedInt}}}
\newcommand{\RefValuesSpec}{\mbox{\rm\textit{refVal}}}
\newcommand{\ClassSpec}{\mbox{\rm\textit{classSpec}}}
\newcommand{\MethodSpec}{\mbox{\rm\textit{methodSpec}}}
\newcommand{\specCase}{\mbox{\rm\textit{specCase}}}
\newcommand{\intraMethodSpec}{\mbox{\rm\textit{intraMethodSpec}}}
\newcommand{\modifiesLoc}{\mbox{\rm\textit{locations}}}
\newcommand{\specIndex}{ \mbox{\rm\textit{specIndex}}}
\newcommand{\op}{\mbox{\rm\textit{op}}}
\newcommand{\exsuresList }{\mbox{\rm\textit{exsuresList}}}

\newcommand{\mult}{\mbox{\rm\textbf{mult}}}
\newcommand{\divis}{\mbox{\rm\textbf{div}}}
\newcommand{\modulo}{\mbox{\rm\textbf{rem}}}
\newcommand{\plus}{\mbox{\rm\textbf{+}}}
\newcommand{\minus}{\mbox{\rm\textbf{-}}}

\newcommand{\bmlKeyWords}{ \mbox{\rm\textit{bmlKeyWords}} }

%%%%%%%%%%%%%%%%%%%%%%%%%%%%%%%%%%%%%%%%%%%%%%%%%%%%%%%%%%%%%%%%%% 
%%%%%%%%%%%%%%%%%%%%%%%%% method extension%%%%%%%%%%%%%%%%%%%%%%%%%%%%%%%
%%%%%%%%%%%%%%%%%%%%%%%%%%%%%%%%%%%%%%%%%%%%%%%%%%%%%%%%%%%%%%%%%%%%%%%%%%%% 
\newcommand{\posL}{\mbox{\rm\textsf{pos}}}
\newcommand{\invL}{\mbox{\rm\textsf{invariant}}}
\newcommand{\modifL}{\mbox{\rm\textsf{modif}}}
%%%%%%%%%%%%%%%%%%%%%%%%%%%%%%%%%%%%%%%%%%%%%%%%%%%%%%%%%%%%%%%%%% 
%%%%%%%%%%%%%%%%%%%%%%%%% BML keywords%%%%%%%%%%%%%%%%%%%%%%%%%%%%%%%
%%%%%%%%%%%%%%%%%%%%%%%%%%%%%%%%%%%%%%%%%%%%%%%%%%%%%%%%%%%%%%%%%%%%%%%%%%%% 
\newcommand{\ClassInv}{\mbox{\rm\textbf{ClassInv}}}
\newcommand{\ClassHistoryConstr}{\mbox{\rm\textbf{ClassHistoryConstr}}}
\newcommand{\declare}{\mbox{\rm \textbf{declare}} }
\newcommand{\ghost}{\mbox{\rm\textbf{ghost}}}
\newcommand{\locations}{\mbox{\textbf{locations}}}
\newcommand{\also}{ \mbox{\rm\textbf{also}}}
\newcommand{\requires}{\mbox{\rm\textbf{requires}}}
\newcommand{\ensures}{\mbox{\rm\textbf{ensures}}}
\newcommand{\exsures}{\mbox{\rm\textbf{exsures}}}
\newcommand{\modifies}{\mbox{\rm\textbf{modifies}}}
\newcommand{\assert}{\mbox{\rm \textbf{assert}}}
\newcommand{\set}{\rm\textbf{set}}
\newcommand{\loopInv}{\mbox{\rm\textbf{loopInv}}}
\newcommand{\loopMod}{\mbox{\rm\textbf{loopModif}}}

\newcommand{\loopDecreases}{\mbox{\rm\textbf{loopDecreases}}}
%%%%%%%%%%%%%%%%%%%%%%%%%%%%%%%%%%%%%%%%%%%%%%%%%%%%%%%%%%%%%%%%%% 
%%%%%%%%%%%%%%%%%%%%%%%%%%%%%%%%%%%%%%%%%%%%%%%%%%%%%%%%%%%%%%%%%% 

\newcommand{\jmlStmt}[1]{\textrm{#1}}
\newcommand{\specExpression}{\mathcal{E}^{spec}}

\newcommand{\EXC}{ \mbox{$\backslash$\rm\textbf{EXC}}} % SPEC the spec variable used in exceptional postconditions to denote the thrown exception object




\newcommand{\ArithExpr}{\texttt{ArithmeticExpr} }

\newcommand{\JMLExpr }{\texttt{JmlExp} }

\newcommand{\everything}{\mbox{\rm \textbf{everything}}}
\newcommand{\nothing}{\mbox{\rm  \textbf{nothing}}}
\newcommand{\arrayAccessMod}[2]{\mbox{\rm\textit{arrayModAt}}(#1,#2)}

\newcommand{\all}{ \mbox{\rm all}}




%\newcommand{\result}{\jmlKey{$\backslash$result}}
\newcommand{\old}[1]{\jmlKey{$\backslash$old(}#1\jmlKey{)}}
\newcommand{\typeof}[1]{\jmlKey{$\backslash$typeof(}#1 \jmlKey{)}}
\newcommand{\subtypeSpec}{<:}
\newcommand{\type}[1]{\jmlKey{$\backslash$type(}#1 \jmlKey{)}}
\newcommand{\TYPE}{\jmlKey{$\backslash$TYPE} }
\newcommand{\elemtype}[1]{ \jmlKey{$\backslash$elemtype(}#1\jmlKey{)} } 
%\newcommand{\excPost}{\psi^{exc}}





%%%%%%%%%%%%%%%%%%%%%%%%%%%%%%%%%%%%%%%%%%%%%%%%%%%%%%%%%%%%%%%%%%%%%%%%%%%%%%%%%%%%%%%%%%%%%%%%%%%%%%%%%
%%%%%%%%%%%%%%%%%%%%%%%%%%%%%%%%%%%%5interpetation%%%%%%%%%%%%%%%%%%%%%%%%%%%%%%%%%%%%%%%%%%%%%%%%%%%%%%%%%%%%%%
%%%%%%%%%%%%%%%%%%%%%%%%%%%%%%%%%%%%%%%%%%%%%%%%%%%%%%%%%%%%%%%%%%%%%%%%%%%%%%%%%%%%%%%%%%%%%%%%%%%%%%%%%
\newcommand{\evalExp}[2]{\mbox{ \rm \textit{eval}}( #1,#2 )} % evaluation of an expression #1 in  a state configuration #2
\newcommand{\evalRel}[1]{\mbox{ \rm \textit{rel}}( #1 )} % evaluation of an expression #1 in  a state configuration #2

\newcommand{\interp}[2]{#2 \vDash #1} % interpretation of a predicate  #1 in  a state configuration #2
\newcommand{\interpTwoLines}[2]{ \begin{array}{l} #2  \vDash \\ 
				         #1 
				  \end{array}}

%%%%%%%%%%%%%%%%%%%%%%%%%wpbc%%%%%%%%%%%%%%%%%%%%%%%%%%%%%%%%%%%
\newcommand{\modifLoop}{\mbox{\rm \textit{modifLoop}}}


%%%%%%%%%%%%%%%%%%%%%%%%%%%%%%%%%%%%%%%%%%%%%%%%%%%%%%%%%%%%%%%%%%%%%%%%%%%%%%%%%%%%%%%%%%%%%%%%%%%%%%%%%
%%%%%%%%%%%%%%%%%%%%%%%%%%%%%%%%%%%% local variable table %%%%%%%%%%%%%%%%%%%%%%%%%%%%%%%%%%%%%%%%%%%%%%%%%%%%%%%%%%%%%%
%%%%%%%%%%%%%%%%%%%%%%%%%%%%%%%%%%%%%%%%%%%%%%%%%%%%%%%%%%%%%%%%%%%%%%%%%%%%%%%%%%%%%%%%%%%%%%%%%%%%%%%%%
\newcommand{\nameInd}{\mbox{\rm\textsf{nameIndex}}}
\newcommand{\attLen}{\mbox{\rm\textsf{attrLen}}}
\newcommand{\lvLength}{\mbox{\rm\textsf{lvLength}}}
\newcommand{\lvTab}{\mbox{\rm\textsf{lvTable}}}
%%%%%%%%%%%%%%%%%%%%%%%%%%%%%%%%%%%%%%%%%%%%%%%%%%%%%%%%%%%%%%%%%%%%%%%%%%%%%%%%%%%%%%%%%%%%%%%%%%%%%%%%%
%%%%%%%%%%%%%%%%%%%%%%%%%%%%%%%%%%%% element in the array of local variable table %%%%%%%%%%%%%%%%%%%%%%%%%%%%%%%%%%%%%%%%%%%%%%%%%%%%%%%%%%%%%%
%%%%%%%%%%%%%%%%%%%%%%%%%%%%%%%%%%%%%%%%%%%%%%%%%%%%%%%%%%%%%%%%%%%%%%%%%%%%%%%%%%%%%%%%%%%%%%%%%%%%%%%%%
\newcommand{\lvElStart}{\mbox{\rm\textsf{startPc}}}
\newcommand{\lvElLen}{\mbox{\rm\textsf{length}}}
\newcommand{\descrInd}{\mbox{\rm\textsf{descrInd}}}
\newcommand{\lvElInd}{\mbox{\rm\textsf{index}}}

%%%%%%%%%%%%%%%%%%%%%%%%%%%%%%%%%%%%%%%%%%%%%%%%%%%%%%%%%%%%%%%%%%%%%%%%%%%%%%%%%%%%%%%%%%%%%%%%%%%%%%%%%
%%%%%%%%%%%%%%%%%%%%%%%%%%%%%%%%%%%% the deep expression language %%%%%%%%%%%%%%%%%%%%%%%%%%%%%%%%%%%%%%%%%%%%%%%%%%%%%%%%%%%%%%
%%%%%%%%%%%%%%%%%%%%%%%%%%%%%%%%%%%%%%%%%%%%%%%%%%%%%%%%%%%%%%%%%%%%%%%%%%%%%%%%%%%%%%%%%%%%%%%%%%%%%%%%%
\newcommand{\exprWp}{\mbox{\rm\textit{E}}}
\newcommand{\predWp}{\mbox{\rm\textit{P}}}
\newcommand{\constantsWp}{\mbox{\rm\textit{constants}}}
\newcommand{\expressionsWp}{\mbox{\rm\textit{Expr}}}
\newcommand{\typeExpWp}{\mbox{\rm\textit{T}}}
\newcommand{\refWp}{\mbox{\rm\textbf{ref}}}
\newcommand{\ConstantsWp}{\mbox{\rm\textit{constants}}}

\section{Context and Motivations}


\begin{frame}\frametitle{The need for a specification language  for Java bytecode }

%% Although source verification has shown its utility  for software audit, there are scenarios 
%% in which it will be hardly applicable
\begin{itemize}
 \item Verification of applications  \textit{developed directly in bytecode}
 \item Verification of safety critical applications 
   \textit{ without trusting the compiler}
 \item Establishing trust in mobile code originating from untrusted entity \textit{w.r.t. complex functional and security policies}
\end{itemize}

%\begin{block}{In those cases we need }
%A specification language tailored to bytecode  into which to express the  security properties over bytecode programs
%\end{block}
\end{frame}

\logo{}
\begin{frame}[shrink]\frametitle{Mobile code and  PCC for complex policies }  
\only<1>{
 \begin{center}
     \pgfimage{figs/PPO}
 \end{center}
}

\only<2>{
 \begin{center}
     \pgfimage{figs/PPObml}
 \end{center}
}
\end{frame}

\logo{\pgfuseimage{mobius_transparent}}

\section{Contributions}
\begin{frame}\frametitle{Contributions}
 \begin{itemize}
   \item Bytecode Modeling Language (BML)  
   \item Encoding of BML into the Java class file format
       %%  \begin{itemize}
%%          \item   specification language for Java bytecode programs  
%%          \item inspired by the Java Modeling Language (JML)
%%       \end{itemize}
   
   \item Compiler from JML to BML
\end{itemize}
\end{frame}

\section{JML}

\begin{frame}\frametitle{The formal specification language JML}
 JML stands for  Java Modeling Language and is the de facto specification language of Java
  \begin{itemize}
     \item Contract based approach  (preconditions and postconditions)
     \item Object invariants 
     \item Method annotations  (e.g. loop invariants, loop frame conditions in JACK)  
     \item Special specification constructs (e.g. \old{e}, \result) 
     \item Specification variables  
  \end{itemize}
\end{frame}

\begin{frame}[fragile,shrink]\frametitle{Example of  JML specified Java code}

%% public class ListArray { 
%% private Object[] list;
  
%% //@ requires list != null;
%% //@ modifies list[*];
%% //@ ensures \result ==(\exists int i; 
%% //@ 0 <= i && i < list.length && 
%% //@ \old(list[i]) == obj1 && list[i] == obj2);
%% boolean replace(Object obj1,Object obj2){
%%   int i = 0;
    
%%   //@ loop_modifies i, list[*];
%%   //@ loop_invariant i <= list.length && i >=0 
%%   //@ &&
%%   //@ (\forall int k;0 <= k && k < i ==> 
%%   //@   list[k] != obj1  && 
%%   //@   list[k] == old(list[k]));
%%   for (i = 0; i < list.length; i++ ){
%%     if ( list[i] == obj1){
%%       list[i] = obj2;
%%       return true;	
%%     }
%%   } 
%%   return false;
%% }}
 \begin{lstlisting}[language=java]
//@ requires n > 0;
//@ ensures sum <= \old(sum)+n*(n+1)/2;
public boolean produce_bill(int n){
  try{
    //@ loop_modifies sum, i;
    //@ loop_invariant 0 <=  i && 0 <= sum &&
    //@ i<= n+1 && sum<=\old(sum)+(i-1)*i/2;
    for (int i=1;i<=n;i++) { 
      this.sum = this.sum + round_cost(i);}
    return true;} 
   catch (Exception e){ 
    return false;}} 
 \end{lstlisting}
\end{frame} 

%%%%%%%%%%%%%%%%%%%%%%%%%%%%%%%%%%%%%%%%%%%BML%%%%%%%%%%%%%%%%%%%%%%%%%%%%%%%%%%%%%%%%%



%\begin{frame}[shrink]\frametitle{From source specification to bytecode specification}
%\begin{center}
%\pgfimage[height=\textheight]{figs/PPObml}
%\end{center}
%\end{frame}

\section{BML}


\subsection{Syntax and Semantics}

 \begin{frame}\frametitle{Design features of BML}
    The bytecode counterpart of JML
       \begin{itemize}
          \item Syntax  
	      \begin{itemize}
                \item  corresponds to a subset of JML (JML level 0)
                \item  specific constructs to the JVM %(\stack{\counter}  and \counter)  
              \end{itemize}
	  \item Semantics - the same  as JML
       \end{itemize}
 \end{frame}    

\begin{frame} \frametitle{ Syntax of BML}
\begin{tabular}{lll}
\multicolumn{2}{l}{\varHook{predicate} ::= \(\ldots\)}\smallskip\\
\multicolumn{2}{l}{\varHook{unary-expr-not-plus-minus} ::= \(\ldots\)}\\
\hspace*{1cm} & \(\mid\) \varHook{primary-expr} [\varHook{primary-suffix}]\(\ldots\)\\ 

\multicolumn{3}{l}{\varHook{primary-suffix} ::= \codeHook{.} \varHook{ident}
\(\mid\) \codeHook{(} [\varHook{expression-list}] \codeHook{)}}\\
& \(\mid\) \codeHook{[} \varHook{expression} \codeHook{]}\\

\multicolumn{2}{l}{\varHook{primary-expr} ::= } \\
&\(\mid\) \alert{\codeHook{\#}\varHook{natural}} \\
&\(\mid\) \alert{\codeHook{lv[}\varHook{natural}\codeHook{]}} \\
&\(\mid\) \varHook{bml-primary}\\
&
\multicolumn{2}{l}{\(\mid\) \varHook{constant} \(\mid\)
\codeHook{super}
\(\mid\) \codeHook{true} \(\mid\) \codeHook{false} \(\mid\)} \\
& \(\mid\) \codeHook{null} 
\(\mid\) \codeHook{(}\varHook{expression}\codeHook{)}
\(\mid\) \varHook{jml-primary}\\
 & & \\

%% \end{tabular}
%% \end{frame}


%% \begin{frame}\frametitle{Syntax of BML (2)}

%% \begin{tabular}{ll}

\multicolumn{2}{l}{\varHook{bml-primary} ::= } \\
&\(\mid\) \alert{\codeHook{cntr} } \\
&\(\mid\) \alert{\codeHook{st(}\varHook{additive-expr}\codeHook{)}} \\
&\(\mid\) \alert{\codeHook{length(}\varHook{expression}\codeHook{)}} 
\end{tabular}
\end{frame} 
  
 
\subsection{Encoding in the class file format}

\begin{frame}\frametitle{Encoding BML in the class file. Features}
  \begin{itemize} 
              \item Does not affect  the Java Virtual Machine (JVM)
               
     	      \item Uses user defined class attributes 
		 in compliance with the JVM specification
	     \item Every specification construct has its own attribute (method contracts, internal assertions, object invariants, ghost fields~\ldots)
            \end{itemize}  
\end{frame}

\begin{frame}\frametitle{Encoding of Ghost fields in the class file}

\begin{tabular}[t]{l}
Ghost\unsc Field\unsc attribute \{\\
\hspace*{1em}
\begin{tabular}{l}
u2  attribute\unsc name\unsc index; \\
u4  attribute\unsc length;\\
u2  fields\unsc count;\\
\{\begin{tabular}[t]{l} 
    u2 access\unsc flags; \\  
    u2 name\unsc index;\\
    u2 descriptor\unsc index;\\
  \end{tabular}\\
\} fields[fields\unsc count]; \} \\
\end{tabular}
\end{tabular}

\end{frame}


\subsection{Example of  BML specification}

\begin{frame}[fragile,shrink]\frametitle{Example of a BML specification}

 \begin{lstlisting}[language=jvmis]
{|
requires 
 lv[1] > 0 
ensures 
 lv[0].#24<=\old(lv[0].#24)+lv[1]*(lv[1]+1)/2
|}
0  iconst_1
1  istore_2
2  goto 22 
5  aload_0
6  aload_0
7  getfield #24 <Bill.sum>
10 aload_0
11 iload_2
12 invokevirtual #26 <Bill.round_cost>
15 iadd
16 putfield #24 <Bill.sum>
19 iinc 2 by 1
loop_modifies lv[2], lv[3] 
loop_invariant 
 0<=lv[2]&&0<=lv[0].#24 && lv[2]<=lv[1]+1 &&
 lv[0].#24<=\old(lv[0].#24) + (lv[2]-1)*lv[2]/2
22 iload_2
23 iload_1
24 if_cmple 5 
27 iconst_1
28 ireturn
29 astore_3
30 iconst_0
31 ireturn
\end{lstlisting}
%% \begin{center}
%% \pgfimage[height=5in]{figs/exampleBml}
%% \end{center}
\end{frame}

%%%%%%%%%%%%%%%%%%%%%%%%%Compiler from JML to BML%%%%%%%%%%%%%%%%%%%%%%%%%%%%
\subsection{Compiler from JML to BML}

 \begin{frame}\frametitle{The JML2BML compiler}
   \begin{itemize}
         \item Independent  of any Java compiler
	 \item Proof obligation preservation 
	       in the case of non - optimizing Java compilation 
   \end{itemize}
   \end{frame}


\begin{frame}\frametitle{Compilation process}
     
     \begin{itemize}
         \item Java source file with JML specification 
          \item Compilation of the Java source file (by a compiler which produces \texttt{Line\_Number\_table} and 
 \texttt{Local\_Variable\_table})
	    \item Compilation of ghost variables (declare those in user defined class attribute)
	      %\item Desugaring of JML specification (heavy weight specification )
		\item Linking phase (convert source names into bytecode names)
		  \item Locating the points for intra method specification (e.g. find the correct place for loop invariants)
		    \item Encoding the BML specification into user defined class attributes
       \end{itemize}
\end{frame}




\section{BML verification}

\begin{frame}\frametitle{Verification over bytecode and BML specifications}
 \begin{itemize}
   \item Development in the Jack tool \footnote{http://www-sop.inria.fr/everest/soft/Jack} of a
         verification condition for a prototype version of BML 
   \item  the official version of BML takes part in the European project Mobius  and is the first step in building a PCC framework
           for complex security policies. An ongoing work in Mobius  is the development of a 
	   verification condition generator for Java bytecode and BML 
	   which will be based on the design principles of the esc/java\footnote{http://secure.ucd.ie/products/opensource/ESCJava2} and Jack tools
\end{itemize}
\end{frame}





%\section{More}

%% \begin{frame}\frametitle{Highlights in BML}
%%   \begin{itemize}
%% \item Bytecode cousin of Java Modeling Language (JML)
%% \item Easy reading of specifications
%% \item Specifications stored together with the class file
%% \item Compiler for JML specifications defined
%% \end{itemize}
%% \end{frame}

%%%%%%%%%%%%%%%%%%%%%%%%%%%%%%%%%%%%%
\section{Related work}

\begin{frame}\frametitle{Related work}
\begin{itemize}
      \item The MRG project 
      \item Bytecode logic developed by Fabian Bannwart  and  Peter M{\"u}ller
      \item JVer verifies bytecode annotated with source code level specifications 
      \item The Spec\# Programming system - tailored to C\# and the .NET intermediate language
      \item The Extended Virtual Platform project
\end{itemize}
\end{frame}

\section{Conclusion}




\subsection{Directions for future work}

\begin{frame} \frametitle{Future work. Tool support}
 As part of the Mobius PCC framework, a tool set around BML  will be developed:
   \begin{itemize}
      \item  Extensions of the JML2BML compiler developed in JACK
      \item BML viewer  and  editor
      \item Tool for checking the well - typedness and structure of BML specifications
      \item Verification condition generator using a guarded command language
   \end{itemize}       
\end{frame}




\begin{frame}[shrink]\frametitle{Future work. Integrating BML in PCC for complex policies}
\vspace{3cm}
Differently from the traditional PCC, Mobius proposes an alternative PCC architecture which targets sophisticated functional and security requirements. 
An implementation of such architecture will rely  on BML.


%%  \begin{center}
%%      \pgfimage{figs/PCC}
%%  \end{center}
\end{frame}

\logo{} 

\begin{frame}[shrink]\frametitle{Future work. Integrating BML in PCC for complex policies}

 \begin{center}
     \pgfimage{figs/toolset}
 \end{center}

\end{frame}

\logo{\pgfuseimage{mobius_transparent}}
\begin{frame}\frametitle{To know more about BML } 

\vspace{2cm}
%\begin{itemize}
%\item \alert{Should I say my thesis ?} 
The web page with the BML Reference Manual (under development)
\begin{center}
\texttt{http://www-sop.inria.fr/everest/BML}
\end{center}
%\end{itemize}
\end{frame}

%% \begin{frame}\frametitle{}
%% \vspace{3cm}
%% \hspace{2cm}
%% \large{Questions?} 
%% \end{frame}
% \section{Conclusion}

%% \begin{frame}\frametitle{Conclusion and future work}
%% \begin{itemize}
%%      \item definition of Java bytecode  language BML - support for JML level 0, support a syntax close to the Java bytecode 
           
%%      \item compiler from JML to BML specifications
%%      \item this work is  part of the European project Mobius and is the first step in building a PCC framework
%%            for complex security policies
     
   
%% \end{itemize}
%% \end{frame}




%% \begin{frame}[fragile,shrink]\frametitle{JML.Example}
%%   \begin{lstlisting}[language=java]
%% //@ requires k >= 0 ;
%% //@ ensures \result == k*(k+1)/2;
%% public int sum (int k) {
%%   int sum = 0;		
%%   //@loop_modifies sum,i;
%%   //@loop_invariant i >= 0 && i<=k && 
%%   //@               sum == i*(i+1)/2;
%%   for  (int i = 0; i < k; i++ ) {
%%     sum = sum + i;
%%   } 	
%%   return sum;
%% }
%% \end{lstlisting} 
%% \end{frame}







%%%%%%%%%%%%%%%%%%%%%%%%%%%%%%%%%%%%%%%%%%%%%%CONTRIBUTIONS%%%%%%%%%%%%%%%%%%%%%%%%%%%%%%%%%%%%%%%%%%%%%%%%%%%%%%%%%%%%%%%%%%%%%%%%%%%%%%%%%%%%%%%%%%%%%%%%%%%%%%%%%%%%%%%%%%%%%%%%%




%% \begin{frame}[fragile,shrink]\frametitle{Example}
%% \begin{columns}
%% \begin{column}{5.1cm}
%% {\tiny
%% \begin{lstlisting}[language=java]
%% //@requires k >= 0 ;
%% //@ensures  \result == k*(k+1)/2;
%% public int sum (int k) {
%%   int sum = 0;		
%%   //@loop_modifies  sum,i;
%%   //@loop_invariant i >= 0 && i<=k && 
%%   //@(sum == i*(i+1)/2);
%%   for  (int i = 0; i < k; i++ ) {
%%     sum = sum + i;
%%   } 	
%%   return sum;
%% }
%% \end{lstlisting}}
%% \end{column}

%% \begin{column}{5.1cm}
%% {\tiny
%% \begin{lstlisting}[language=jvmis]
%% //@requires lv[1] >= 0;
%% //@ensures result == lv[1]*(lv[1] + 1)/2;
%% Loop specification
%% //@atIndex 14 
%% //@modifies  lv[2], lv[3]
%% //@invariant lv[3]>=0 && lv[3]<=lv[1] &&
%% //@    lv[2]==lv[3]*(lv[3]+1)/2
%% 0 iconst_0
%% 1 istore_2
%% 2 iconst_0
%% 3 istore_3
%% 4 goto 14 
%% 7 iload_2
%% 8 iload_3
%% 9 iadd
%% 10 istore_2 
%% 11 iinc 3 
%% 14 iload_3
%% 15 iload_1
%% 16 if_icmplt 7 
%% 19 iload_2
%% 20 ireturn
%% \end{lstlisting}}
%% \end{column}
%% \end{columns}
%% \end{frame}

%%%%%%%%%%%%%%%%%%%%%%%%%%%%%%%%%VC gen%%%%%%%%%%%%%%%%%%%%%%%%%%%%%%%%%%%%%%%%%%%%%%






\end{document}

