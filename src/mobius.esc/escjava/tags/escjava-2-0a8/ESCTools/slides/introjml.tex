\documentclass[
pdf,
%a4,
nocolorBG,
%colorBG,
slideColor,
%accumulate,
%draft,
erik,
%frames,
]{prosper}
%%%%%%%%%%%%%%%%%%%%%%%%%%%%%%%%%%%%%%%%%%%%%%%%%%%%%%%%%%%%%%%%%%%%%%%%%%%
\usepackage{alltt}
\usepackage{pstricks,pst-node,pst-text,pst-3d}
\usepackage{textcomp}
%\usepackage{colordvi}
\newcommand{\Red}[1]{{\red #1}}
\usepackage[all]{xy}

%%%%%%%%%%%%%%%%%%%%%%%%%%%%%%%%%%%%%%%%%%%%%%%%%%%%%%%%%%%%%%%%%%%%%%%%%%%

%\newrgbcolor{Yellowish}{0.90 0.85 0.650}

%\newrgbcolor{red}{1 0 0}
%\newrgbcolor{purple}{0.4 0 0.7}
%\newrgbcolor{lightpurple}{0.63 0.13 0.94}

%\newrgbcolor{lime}{0.73 1 0}
\newrgbcolor{green}{0.133 0.56 0} % lichter
%\newrgbcolor{green}{0.10 0.43 0}

%\newrgbcolor{knalblue}{0 0 1}
\newrgbcolor{blue}{.2 .36 .77}
%\newrgbcolor{darkblue}{0.28 0.24 0.55}


%%%%%%%%%%%%%%%%%%%%%%%%%%%%%%%%%%%%%%%%%%%%%%%%%%%%%%%%%%%%%%%%%%%%%%%%%%%

\newcommand{\embf}[1]{\textit{\textbf{#1}}}
\newcommand{\rmbf}[1]{\textrm{\textbf{#1}}}
\newcommand{\ttbf}[1]{\mbox{\tt \textbf{#1}}}
\newcommand{\code}[1]{{\rm \texttt{\textbf{\small #1}}}}

\myitem{1}{\mbox{{$\bullet$}}}
\newcommand{\old}     {\(\backslash\)old}
\newcommand{\vooralle}{\(\backslash\)forall}
\newcommand{\everything}{\(\backslash\)everything}
\newcommand{\nothing}{\(\backslash\)nothing}
\newcommand{\nonnull} {non\verb|_|null}
\newcommand{\result}  {\(\backslash\)result}
\renewcommand{\familydefault}{phv}
\renewcommand{\rmdefault}{phv}
\newif\ifignore

\newrgbcolor{Bluish}{0.9 0.9 1.0}
\newcommand{\doos}[1]{\psshadowbox[fillstyle=solid,
                        fillcolor=Bluish,
                        framearc=0.2,
                        framesep=2mm]
                        {#1}}

%%%%%%%%%%%%%%%%%%%%%%%%%%%%%%%%%%%%%%%%%%%%%%%%%%%%%%%%%%%%%%%%%%%%%%%%%%%%%

\title{\embf{\blue 
       {\huge Introduction to JML
      }}}
\author{\embf{\Large{\red Erik Poll, Joe Kiniry, David Cok}}
       }
\institution{\embf{\large University of Nijmegen; Eastman Kodak Company}}
\slideCaption{{\blue Erik Poll - ESC/Java2 Tutorial - June 2004 -  JML}}
%\Logo{{\blue Erik Poll}}

%%%%%%%%%%%%%%%%%%%%%%%%%%%%%%%%%%%%%%%%%%%%%%%%%%%%%%%%%%%%%%%%%%%%%%%%%%%%%

\begin{document}

\maketitle 

\boldmath

\ifignore
%%%%%%%%%%%%%%%%%%%%%%%%%%%%%%%%%%%%%%%%%%%%%%%%%%%%%%%%%%%%%%%%%%%%%%%%%%%%%
\begin{slide}{Test \hfill}

{\Large Large}
{\large large}
{\normalsize normal}
niks
{\small small}
{\footnotesize footnote}
{\scriptsize script}
{\tiny tiny}

{\bf bf}
{\rm rm}
{\it it}
{\sf sf}
{\sc sc}

\textit{textit}
\textrm{textrm}
\textbf{textbf}
\textsf{textsf}
\textsc{textsc}

\end{slide}
\fi


%%%%%%%%%%%%%%%%%%%%%%%%%%%%%%%%%%%%%%%%%%%%%%%%%%%%%%%%%%%%%%%%%%%%%%%%%%%%%
\begin{slide}{Outline of this talk}
\vspace*{-2ex}

What this set of slides aims to do
\begin{itemize}
\item introduction to JML
\item provide overview of tool support for JML (jmlrac, jmlunit,
  escjava)
\item explain idea of extended static checking and difference with
  runtime assertion checking
\item some more ESC/Java2 tips
\end{itemize}


\end{slide}

%%%%%%%%%%%%%%%%%%%%%%%%%%%%%%%%%%%%%%%%%%%%%%%%%%%%%%%%%%%%%%%%%%%%%%%%%%%%%

%%%%%%%%%%%%%%%%%%%%%%%%%%%%%%%%%%%%%%%%%%%%%%%%%%%%%%%%%%%%%%%%%%%%%%%%%%%%%
\part{{\Large \red The Java Modeling Language \\
    JML \\
    [2ex] {\large\black \texttt{www.jmlspecs.org}}}}
%%%%%%%%%%%%%%%%%%%%%%%%%%%%%%%%%%%%%%%%%%%%%%%%%%%%%%%%%%%%%%%%%%%%%%%%%%%%%

%%%%%%%%%%%%%%%%%%%%%%%%%%%%%%%%%%%%%%%%%%%%%%%%%%%%%%%%%%%%%%%%%%%%%%%%%%%%
\overlays{2}{
\begin{slide}{JML {\footnotesize {\black by Gary Leavens et al.}}}
\vspace*{-3ex}
{\blue Formal specification language} for Java 
\medskip
\begin{itemize}
\item
to specify behaviour of Java classes
\item
to record design \&implementation decisions
\end{itemize}
\medskip
by adding {\green assertions} to Java source code, eg
\medskip
\begin{itemize}
\item ~{\green preconditions}
\item ~{\green postconditions}
\item ~{\green invariants}
\end{itemize}

\medskip

as in Eiffel (Design by Contract), but more expressive.

\medskip

\fromSlide{2}{\doos{Goal: JML should be easy to use for any Java
    programmer.}}


\end{slide}}


%%%%%%%%%%%%%%%%%%%%%%%%%%%%%%%%%%%%%%%%%%%%%%%%%%%%%%%%%%%%%%%%%%%%%%%%%%%%
\begin{slide}{JML}
\vspace*{-3ex}

To make JML easy to use:

\medskip
\begin{itemize}
\item
JML assertions are added as comments in .java file,

between {\green \texttt{\textbf{\small /*@}}} \ldots {\green \texttt{\textbf{\small @*/}}},
or after {\green \texttt{\textbf{\small //@}}},

\medskip
\item
Properties are specified as Java boolean expressions, extended
  with a few operators
({\green \old, \vooralle, \result, \ldots}).

\medskip
\item
using a few keywords
({\green \code{requires},
\code{ensures},
\code{signals},
\code{assignable},
\code{pure},
\code{invariant},
\code{non\_null},
\ldots})

\end{itemize}

\end{slide}


%%%%%%%%%%%%%%%%%%%%%%%%%%%%%%%%%%%%%%%%%%%%%%%%%%%%%%%%%%%%%%%%%%%%%%%%%%%%
\begin{slide}{requires, ensures}
\vspace*{-3ex}

{\blue Pre-} and {\blue post-conditions} for method can be specified.

\begin{alltt}
\texttt{\textbf{\small
{\green /*@}{\blue requires} amount >= 0;
    {\blue ensures}  balance == \old(balance-amount) &&
              \result == balance;
{ \green  @*/}
  public int{\green debit}(int amount) \{ 
    ...
  \}
}}
\end{alltt}

Here \code{\old(balance)} refers to the value of \code{balance}
before execution of the method.

\end{slide}
%%%%%%%%%%%%%%%%%%%%%%%%%%%%%%%%%%%%%%%%%%%%%%%%%%%%%%%%%%%%%%%%%%%%%%%%%%%%
\begin{slide}{requires, ensures}
\vspace*{-3ex}

JML specs can be as strong or as weak as you want.

\begin{alltt}
\texttt{\textbf{\small
{\green /*@}{\blue requires} amount >= 0;
    {\blue ensures} {\blue true};
{ \green  @*/}
  public int{\green debit}(int amount) \{ 
    ...
  \}
}}
\end{alltt}

This default postcondition ``\code{ensures true}'' can be omitted.

\end{slide}

%%%%%%%%%%%%%%%%%%%%%%%%%%%%%%%%%%%%%%%%%%%%%%%%%%%%%%%%%%%%%%%%%%%%%%%%%%%%
\begin{slide}{Design-by-Contract}
\vspace*{-3ex}

Pre- and postconditions define a {\blue contract} between a class and
its clients:
\medskip
\begin{itemize}
\item
Client must {\green ensure precondition}
and may {\green assume postcondition}
\item
Method may {\green assume precondition}
and must {\green ensure postcondition}
\end{itemize}

\bigskip

Eg, in the example specs for \code{debit}, it is the obligation of the
client to ensure that \code{amount} is positive.  The \code{requires}
clause makes this {\green explicit}.

\end{slide}

%%%%%%%%%%%%%%%%%%%%%%%%%%%%%%%%%%%%%%%%%%%%%%%%%%%%%%%%%%%%%%%%%%%%%%%%%%%%
\begin{slide}{signals}
\vspace*{-3ex}

{\blue Exceptional postconditions} can also be specified.
\vspace*{-1ex}
\begin{alltt}
\texttt{\textbf{\small
{\green /*@}{\black requires} amount >= 0;
    {\black ensures}  true;
    {\blue signals (ISOException e) 
               amount > balance         &&
               balance == \old(balance) &&
               e.getReason()==AMOUNT_TOO_BIG;}
{\green   @*/}
  public int{\green debit}(int amount) \{ 
   ...
  \}
}}
\end{alltt}

\end{slide}

%%%%%%%%%%%%%%%%%%%%%%%%%%%%%%%%%%%%%%%%%%%%%%%%%%%%%%%%%%%%%%%%%%%%%%%%%%%%
\begin{slide}{signals}
\vspace*{-3ex}

Exceptions are allowed by default,
i.e.\ the default signals clause is
\begin{alltt}\texttt{\textbf{\small {\blue signals} (Exception){\green true}; }}
\end{alltt}

\medskip
To rule them out, add an explicit
\begin{alltt}\texttt{\textbf{\small {\blue signals} (Exception){\green false}; 
}}

\end{alltt}
or use the keyword {\blue normal\_behavior}
\begin{alltt}
\texttt{\textbf{\small {\green /*@}{\blue normal\_behavior}
       {\black requires} ...
       {\black ensures}  ...              
 {\green   @*/}
}}
\end{alltt}

\end{slide}

%%%%%%%%%%%%%%%%%%%%%%%%%%%%%%%%%%%%%%%%%%%%%%%%%%%%%%%%%%%%%%%%%%%%%%%%%%%%
\begin{slide}{invariant}
\vspace*{-3ex}
{\blue Invariants} (aka {\em class} invariants) are properties that
must be maintained by all methods, e.g.,
\begin{alltt}
\texttt{\textbf{\small{public class {\green Wallet} \{
  public static final short{\green MAX_BAL} = 1000;
  private short{\green balance};
  {\green /*@}{\blue invariant 0 <= balance &&
                      balance <= MAX_BAL;}
  {\green   @*/}
  ...  }}}
\end{alltt} %}

Invariants are implicitly included in all pre- and postconditions.

Invariants must {\em also} be preserved if exception is thrown!
\end{slide}

%%%%%%%%%%%%%%%%%%%%%%%%%%%%%%%%%%%%%%%%%%%%%%%%%%%%%%%%%%%%%%%%%%%%%%%%%%%%
\begin{slide}{invariant}
\vspace*{-4ex}

Invariants document design decisions, e.g.,
\begin{alltt}
\texttt{\textbf{\small public class{\green Directory} \{
 private File[]{\green files};
{\green /*@}{\blue invariant} 
     files != null    
     &&
     (\vooralle int i; 0 <= i && i < files.length;
                   ; files[i] != null &&
                     files[i].getParent() == this);
  {\green @*/} }}
\end{alltt} %}

Making them {\green explicit} helps in understanding the code.
\end{slide}

%%%%%%%%%%%%%%%%%%%%%%%%%%%%%%%%%%%%%%%%%%%%%%%%%%%%%%%%%%%%%%%%%%%%%%%%%%%%
\begin{slide}{non\_null}
\vspace*{-4ex}

Many invariants, pre- and postconditions are about references not
being \texttt{null}.  {\blue non\_null} is a convenient short-hand for
these.

\begin{alltt}
\texttt{\textbf{\small public class Directory \{

  private{\green /*@}{\blue non\_null}{\green @*/} File[] files;

  void createSubdir({\green{/*@}}{\blue non\_null}{\green @*/} String name)\{
   ...
  Directory{\green /*@}{\blue non\_null}{\green @*/} getParent()\{
   ...
}}
\end{alltt} %}

\end{slide}

%%%%%%%%%%%%%%%%%%%%%%%%%%%%%%%%%%%%%%%%%%%%%%%%%%%%%%%%%%%%%%%%%%%%%%%%%%%%
\begin{slide}{assert}
\vspace*{-3ex}

An {\blue \texttt{\textbf{\small assert}}} clause specifies
a property that should hold at some point in the code, e.g.,
\begin{alltt}
\texttt{\textbf{\small if (i <= 0 || j < 0) \{
      ...
  \} else if (j < 5) \{
     {\green //@}{\blue assert i > 0 && 0 < j && j < 5;}
      ...
  \} else \{
     {\green //@}{\blue assert i > 0 && j > 5;}
      ...
  \}  }}
\end{alltt}

\end{slide}

%%%%%%%%%%%%%%%%%%%%%%%%%%%%%%%%%%%%%%%%%%%%%%%%%%%%%%%%%%%%%%%%%%%%%%%%%%%%
\begin{slide}{assert}
\vspace*{-3ex}

JML keyword \code{assert} now also in Java (since Java 1.4).

\medskip

Still, assert in JML is more expressive, for example in

\begin{alltt}
\texttt{\textbf{\small  ...
  for (n = 0; n < a.length; n++) 
       if (a[n]==null) break;
{\green /*@}{\blue assert (\vooralle int i; 0 <= i && i < n; 
                            a[i] != null);                }
  {\green @*/} }}
\end{alltt} 

\end{slide}

%%%%%%%%%%%%%%%%%%%%%%%%%%%%%%%%%%%%%%%%%%%%%%%%%%%%%%%%%%%%%%%%%%%%%%%%%%%%
\begin{slide}{assignable}
\vspace*{-3ex}

{\blue Frame properties} limit possible side-effects of methods.

\begin{alltt}
\texttt{\textbf{\small {\green /*@}   requires amount >= 0;
     {\blue assignable} balance;
         ensures balance == \old(balance)-amount;
{ \green  @*/}
  public int{\green debit}(int amount) \{ 
    ...
}}
\end{alltt}
E.g., \code{debit} can {\em only} assign to the field \code{balance}.

NB this does {\em not} follow from the post-condition.

\medskip

Default assignable clause: \code{assignable \everything}.

\end{slide}

%%%%%%%%%%%%%%%%%%%%%%%%%%%%%%%%%%%%%%%%%%%%%%%%%%%%%%%%%%%%%%%%%%%%%%%%%%%%
\begin{slide}{pure}
\vspace*{-3ex}

A {\blue method without side-effects} is called {\blue pure}.

\begin{alltt}
\texttt{\textbf{\small
  public{\green /*@}{\blue pure}{\green @*/} int getBalance()\{...

  Directory{\green /*@}{\blue pure non\_null}{\green @*/} getParent()\{...

}}
\end{alltt}

Pure method are implicitly \code{assignable \nothing}.

\medskip

Only pure methods can be used {\em in} specifications.

\end{slide}

%%%%%%%%%%%%%%%%%%%%%%%%%%%%%%%%%%%%%%%%%%%%%%%%%%%%%%%%%%%%%%%%%%%%%%%%%%%%
\begin{slide}{visibility}
\vspace*{-4ex}
JML supports the standard Java visibilities: 

\begin{alltt}
\texttt{\textbf{\small {\blue public} int{\blue pub}; {\green private} int{\green priv}; 

 {\black //@} requires i <={\blue pub};
 {\blue public} void{\blue pub1} (int i) \{ ... \}

 {\black //@} requires i <={\blue pub} && i <={\green priv};
 {\green private} void{\green priv1} (int i) { ... } 

 {\black //@} requires i <={\blue pub} && i <={\green priv}; // WRONG !!
 {\blue public} void{\blue pub2}(int i) \{ ... \}
 }}
\end{alltt}

Specs of {\blue public} methods may not refer to {\green private} fields.

\end{slide}

%%%%%%%%%%%%%%%%%%%%%%%%%%%%%%%%%%%%%%%%%%%%%%%%%%%%%%%%%%%%%%%%%%%%%%%%%%%%
\begin{slide}{visibility: spec\_public}
\vspace*{-3ex}

Keyword \code{spec\_public} loosens visibility for specs.\\
Private \code{spec\_public} fields are allowed in public specs, e.g.:

\begin{alltt} \texttt{\textbf{\small{\blue public} int{\blue pub};
 {\green private}{\black /*@}{\blue spec\_public}{\black @*/} int{\green priv};

 {\black //@} requires i <={\blue pub} && i <={\green priv}; // OK
 {\blue public} void{\blue pub2}(int i) \{ ... \} 

}}
\end{alltt}

\bigskip

{\rm Exposing private details is ugly, of course.  A nicer, but more
  advanced alternative in JML is to use public \code{\red model}
  fields to represent (abstract away from) private implementation
  details.}

\end{slide}

%%%%%%%%%%%%%%%%%%%%%%%%%%%%%%%%%%%%%%%%%%%%%%%%%%%%%%%%%%%%%%%%%%%%%%%%%%%%%
\part{{\Large \red Tools for JML}}
%%%%%%%%%%%%%%%%%%%%%%%%%%%%%%%%%%%%%%%%%%%%%%%%%%%%%%%%%%%%%%%%%%%%%%%%%%%%%

%%%%%%%%%%%%%%%%%%%%%%%%%%%%%%%%%%%%%%%%%%%%%%%%%%%%%%%%%%%%%%%%%%%%%%%%%%%%
\overlays{3}{\begin{slide}{tools for JML}
\vspace*{-2ex}

\begin{itemize}
\fromSlide{1}{\item$\!$ {\red parsing} and {\red typechecking}} 
\fromSlide{2}{\item$\!$ {\red runtime assertion checking}:\\
      {\blue test} for violations of assertions {\blue during execution}\\
      {\green jmlrac}}
\fromSlide{3}{\item$\!$ {\red extended static checking}:\\
      {\blue prove} 
      that contracts are never violated
      {\blue at compile-time}\\
      {\green ESC/Java2}\\

      This is program verification, not just testing.}
\end{itemize}
\end{slide}}

%%%%%%%%%%%%%%%%%%%%%%%%%%%%%%%%%%%%%%%%%%%%%%%%%%%%%%%%%%%%%%%%%%%%%%%%%%%%%
\overlays{4}{
\begin{slide}{runtime assertion checking}
\vspace*{-3ex}
{\blue jmlrac compiler} by Gary Leavens et al. at Iowa State Univ.

\begin{itemize}
\item
translates {\blue JML assertions}
into {\blue runtime checks}:
\begin{quote}
  during execution, {\em all} assertions are tested and any violation
  of an assertion produces an Error.
\end{quote}
\fromSlide{2}{\item$\!$
  {\blue cheap \& easy} to do as part of existing testing practice}
\fromSlide{2}{\item$\!$
  {\blue better testing}, because {\green more properties} are tested,
  at {\green more places} in the code}
\end{itemize}

\fromSlide{3}{Of course, an assertion violation can be an {\em error in
    code} {\green or} an {\em error in specification}.}

\fromSlide{4}{
  The {\blue jmlunit} tool combines jmlrac and {\green unit testing}.}
\end{slide}}

%%%%%%%%%%%%%%%%%%%%%%%%%%%%%%%%%%%%%%%%%%%%%%%%%%%%%%%%%%%%%%%%%%%%%%%%%%%%%
\begin{slide}{runtime assertion checking}
\vspace*{-3ex}

jmlrac can generate complicated test-code for free.  E.g., for
\begin{alltt}
\texttt{\textbf{\small{\green /*@} ...
    {\blue signals (Exception) 
                 balance == \old(balance);}
 {\green  @*/}
  public int{\green debit}(int amount) \{ ... \}
}}
\end{alltt} 
it will test that {\green if \code{debit} throws an exception,
the balance hasn't changed, and all invariants still hold}.

\bigskip

{\scriptsize jmlrac even checks \code{\vooralle} if the domain of 
quantification is finite.}

\end{slide}

%%%%%%%%%%%%%%%%%%%%%%%%%%%%%%%%%%%%%%%%%%%%%%%%%%%%%%%%%%%%%%%%%%%%%%%%%%%%%
\overlays{5}{
\begin{slide}{extended static checking}
\vspace*{-3ex}

{\blue ESC/Java(2)}
\begin{itemstep}
\item$\!$
{\blue {\em tries} to {\em prove} correctness of specifications,}\\
{\green at compile-time, fully automatically}
\item$\!$
\Red{\em not sound}: ESC/Java may miss an error that is actually present
\item$\!$
\Red{\em not complete}: ESC/Java may warn of errors that are impossible
\item$\!$
 but {\blue finds lots of potential bugs quickly}
\item
good at proving absence of runtime exceptions {\scriptsize (eg
Null-, ArrayIndexOutOfBounds-, ClassCast-)} and verifying
relatively simple properties.
\end{itemstep}

\end{slide}}

%%%%%%%%%%%%%%%%%%%%%%%%%%%%%%%%%%%%%%%%%%%%%%%%%%%%%%%%%%%%%%%%%%%%%%%%%%%%
\begin{slide}{static checking vs runtime checking}
\vspace*{-3ex}

Important differences:

\begin{itemize}
\item ESC/Java2 checks specs at {\blue compile-time}, \\
      jmlrac checks specs at {\green run-time}

\item ESC/Java2 {\blue proves} correctness of specs,\\
      jml only {\green tests} correctness of specs.
\\
Hence
\begin{itemize}
\item ESC/Java2 independent of any test suite, \\
      results of runtime testing only as good as the test suite,
\item ESC/Java2 provides higher degree of confidence.
\end{itemize}

\end{itemize}

\end{slide}

%%%%%%%%%%%%%%%%%%%%%%%%%%%%%%%%%%%%%%%%%%%%%%%%%%%%%%%%%%%%%%%%%%%%%%%%%%%%
\begin{slide}{static checking vs runtime checking}
\vspace*{-3ex}

One of the assertions below is wrong:
\begin{alltt}
\texttt{\textbf{\small  if (i <= 0 || j < 0) \{
      ...
   \} else if (j < 5) \{
       {\green //@ assert i > 0 && 0 < j && j < 5;}
       ...
   \} else \{
       {\green //@ assert i > 0 && j > 5;}
       ...
   \}  
}}\end{alltt}

Runtime assertion checking {\blue {\em may}} detect this with a
comprehensive test suite.

ESC/Java2 {\blue {\em will}} detect this at compile-time.

\end{slide}

%%%%%%%%%%%%%%%%%%%%%%%%%%%%%%%%%%%%%%%%%%%%%%%%%%%%%%%%%%%%%%%%%%%%%%%%%%%%
\begin{slide}{modular reasoning (1)}
\vspace*{-3ex}

ESC/Java2 reasons about every method individually.
So in

\begin{alltt}\code{ class A\{
  byte[] b;
  public void n() \{ b = new byte[20]; \}
  public void m() \{ n();
                    b[0] = 2;
                    ...       \}
}
\end{alltt}

ESC/Java2 warns that \code{b[0]} may be a null dereference here,\\
even though you can see that it won't be.
\end{slide}

\begin{slide}{modular reasoning (1)}
\vspace*{-3ex}
To stop ESC/Java2 complaining: add a postcondition
\begin{alltt}\code{ class A\{
  byte[] b;
 {\green //@ ensures b != null && b.length = 20;}
  public void n() \{ a = new byte[20]; \}
  public void m() \{ n();
                    b[0] = 2;
                    ...       \} 
}
\end{alltt}
So: property of method that is relied on has to be made explicit.

And: subclasses that override methods have to preserve these.

\end{slide}

%%%%%%%%%%%%%%%%%%%%%%%%%%%%%%%%%%%%%%%%%%%%%%%%%%%%%%%%%%%%%%%%%%%%%%%%%%%%
\begin{slide}{modular reasoning (2)}
\vspace*{-3ex}

Similarly, ESC/Java will complain about \code{b[0] = 2} in

\begin{alltt}\code{ class A\{
  byte[] b;
  public void A() \{ b = new byte[20]; \}
  public void m() \{ b[0] = 2;
                    ...  \}

}
\end{alltt}
Maybe you can see that this is a spurious warning, though this will be
harder than in the previous example: you'll have to inspect {\em all}
constructors and {\em all} methods.

\end{slide}

\begin{slide}{modular reasoning (2)}
\vspace*{-3ex}

To stop ESC/Java2 complaining here: add an invariant

\begin{alltt}\code{ class A\{
  byte[] b;
  {\green //@ invariant b != null && b.length == 20;}
  {\green     // or weaker property for b.length ?}
  public void A() \{ b = new byte[20]; \}
  public void m() \{ b[0] = 2;
                    ...  \}
}
\end{alltt}

So again: properties you rely on have to be made explicit.

\medskip

And again: subclasses have to preserve these properties.

\end{slide}


%%%%%%%%%%%%%%%%%%%%%%%%%%%%%%%%%%%%%%%%%%%%%%%%%%%%%%%%%%%%%%%%%%%%%%%%%%%%
\begin{slide}{assume}
\vspace*{-3ex}

Alternative to stop ESC/Java2 complaining: add an assumption:
\begin{alltt}\code{ 
    ...
    //@{\blue assume} b != null && b.length > 0;
    b[0] = 2;
    ...  
}
\end{alltt}

Especially useful during development, when you're still trying to
discover hidden assumptions, or when ESC/Java2's reasoning power is
too weak.

\medskip

(\code{requires} can be understood as a form of \code{assume}.)

\end{slide}


%%%%%%%%%%%%%%%%%%%%%%%%%%%%%%%%%%%%%%%%%%%%%%%%%%%%%%%%%%%%%%%%%%%%%%%%%%%%
\overlays{1}{\begin{slide}{more JML tools}

\vspace*{-2ex}
\begin{itemize}
\item$\!$ {\blue javadoc-style documentation}:  {\green jmldoc}
\item
Other red {\blue verification} tools:
\begin{itemize}
\item$\!$ {\green LOOP tool + PVS} (Nijmegen)
\item$\!$ {\green JACK} (Gemplus/INRIA)
\item$\!$ {\green Krakatoa tool + Coq} (INRIA)
\end{itemize}
These tools (also) aim at {\blue interactive} verification of complex
properties, whereas ESC/Java2 aims at {\blue automatic} verification
of relatively simple properties.
\item runtime {\blue detection of invariants}: {\green Daikon} (Michael Ernst, MIT)
\item$\!$ {\blue model-checking} multi-threaded programs: {\green Bogor} (Kansas State)
\end{itemize}
See \texttt{www.jmlspecs.org}

\end{slide}}

%%%%%%%%%%%%%%%%%%%%%%%%%%%%%%%%%%%%%%%%%%%%%%%%%%%%%%%%%%%%%%%%%%%%%%%%%%%%
\begin{slide}{Acknowledgements}

\vspace*{-2ex}
Many people and groups have contributed to JML and related tools.

\begin{itemize}
\item {\scriptsize Gary Leavens led the JML effort at Iowa St.
    Contributors include Albert Baker, Clyde Ruby, Curtis
    Clifton, Yoonsik Cheon, Anand Ganapathy, Abhay Bhorkar, Arun
    Raghavan, Kristina Boysen, David Behroozi. Katie Becker, Elisabeth
    Seagren, Brandon Shilling, Katie Becker, Ajani Thomas, and Arthur
    Thomas.}
\item {\scriptsize The ESC project at SRC included K. Rustan M.
    Leino, Cormac Flanagan, Mark Lillibridge, Greg Nelson, Raymie
    Stata, and James Saxe.}
\item {\scriptsize Bart Jacobs led the LOOP (now SoS) group at
    Nijmegen.  Contributors include Erik Poll, Joachim van den Berg,
    Marieke Huisman, Cees-Bart Breunesse, and Joe Kiniry.}
\item {\scriptsize David Cok is a primary contributor to JML and
    ESC/Java2.}
\end{itemize}
  
\end{slide}

%%%%%%%%%%%%%%%%%%%%%%%%%%%%%%%%%%%%%%%%%%%%%%%%%%%%%%%%%%%%%%%%%%%%%%%%%%%%
\begin{slide}{More information}

\vspace*{-2ex}
These websites and mailing lists can provide more information (and have links to even more):

\begin{itemize}

\item JML: www.jmlspecs.org
\item mailing lists:  jmlspecs-interest@lists.sourceforge.net\\
jmlspecs-developers@lists.sourceforge.net 

\item[]
\item ESC/Java2:  www.cs.kun.nl/sos/research/escjava
\item ESC/Java:    www.research.compaq.com/SRC/esc/
\item mailing list:  jmlspecs-escjava@lists.sourceforge.net

\end{itemize}
  
\end{slide}

\end{document}

