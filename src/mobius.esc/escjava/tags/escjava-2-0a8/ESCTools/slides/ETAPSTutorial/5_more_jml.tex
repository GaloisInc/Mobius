\documentclass[
pdf,
%a4,
nocolorBG,
%colorBG,
slideColor,
%accumulate,
%draft,
erik,
%frames,
]{prosper}
%%%%%%%%%%%%%%%%%%%%%%%%%%%%%%%%%%%%%%%%%%%%%%%%%%%%%%%%%%%%%%%%%%%%%%%%%%%
\usepackage{alltt}
\usepackage{pstricks,pst-node,pst-text,pst-3d}
\usepackage{textcomp}
%\usepackage{colordvi}
\newcommand{\Red}[1]{{\red #1}}
\usepackage[all]{xy}

%%%%%%%%%%%%%%%%%%%%%%%%%%%%%%%%%%%%%%%%%%%%%%%%%%%%%%%%%%%%%%%%%%%%%%%%%%%

%\newrgbcolor{Yellowish}{0.90 0.85 0.650}

%\newrgbcolor{red}{1 0 0}
%\newrgbcolor{purple}{0.4 0 0.7}
%\newrgbcolor{lightpurple}{0.63 0.13 0.94}

%\newrgbcolor{lime}{0.73 1 0}
\newrgbcolor{green}{0.133 0.56 0} % lichter
%\newrgbcolor{green}{0.10 0.43 0}

%\newrgbcolor{knalblue}{0 0 1}
\newrgbcolor{blue}{.2 .36 .77}
%\newrgbcolor{darkblue}{0.28 0.24 0.55}


%%%%%%%%%%%%%%%%%%%%%%%%%%%%%%%%%%%%%%%%%%%%%%%%%%%%%%%%%%%%%%%%%%%%%%%%%%%

\newcommand{\embf}[1]{\textit{\textbf{#1}}}
\newcommand{\rmbf}[1]{\textrm{\textbf{#1}}}
\newcommand{\ttbf}[1]{\mbox{\tt \textbf{#1}}}
\newcommand{\code}[1]{{\rm \texttt{\textbf{\small #1}}}}

\myitem{1}{\mbox{{${\black\bullet}$}}}
\newcommand{\old}     {\(\backslash\)old}
\newcommand{\into}     {\(\backslash\)into}
\newcommand{\result}     {\(\backslash\)result}
\newcommand{\vooralle}{\(\backslash\)forall}
\newcommand{\eris}{\(\backslash\)exists}
\newcommand{\everything}{\(\backslash\)everything}
\newcommand{\nothing}{\(\backslash\)nothing}
\newcommand{\nonnull} {non\verb|_|null}



\newcommand{\bsl}{\char'134}
\renewcommand{\result}{\bsl result}
\renewcommand{\old}{\bsl old}

\renewcommand{\familydefault}{phv}
\renewcommand{\rmdefault}{phv}

\newif\ifignore
\newrgbcolor{Bluish}{0.9 0.9 1.0}
\newcommand{\doos}[1]{\psshadowbox[fillstyle=solid,
                        fillcolor=Bluish,
                        framearc=0.2,
                        framesep=2mm]
                        {#1}}

%%%%%%%%%%%%%%%%%%%%%%%%%%%%%%%%%%%%%%%%%%%%%%%%%%%%%%%%%%%%%%%%%%%%%%%%%%%%%

\title{\embf{\blue 
       {\huge Advanced JML  \\ {\large and more tips and pitfalls}
      }}}
\author{\embf{\Large{\red David Cok, Joe Kiniry, and Erik Poll}}
       }
\institution{\embf{Eastman Kodak Company, University College Dublin, \\ and Radboud University Nijmegen}}
\slideCaption{{\blue David Cok, Joe Kiniry \& Erik Poll - ESC/Java2 \& JML Tutorial }}


%%%%%%%%%%%%%%%%%%%%%%%%%%%%%%%%%%%%%%%%%%%%%%%%%%%%%%%%%%%%%%%%%%%%%%%%%%%%%

\begin{document}

\maketitle 

\boldmath

\ifignore
%%%%%%%%%%%%%%%%%%%%%%%%%%%%%%%%%%%%%%%%%%%%%%%%%%%%%%%%%%%%%%%%%%%%%%%%%%%%%
\begin{slide}{Test \hfill}

{\Large Large}
{\large large}
{\normalsize normal}
niks
{\small small}
{\footnotesize footnote}
{\scriptsize script}
{\tiny tiny}

{\bf bf}
{\rm rm}
{\it it}
{\sf sf}
{\sc sc}

\textit{textit}
\textrm{textrm}
\textbf{textbf}
\textsf{textsf}
\textsc{textsc}

\end{slide}
\fi


%%%%%%%%%%%%%%%%%%%%%%%%%%%%%%%%%%%%%%%%%%%%%%%%%%%%%%%%%%%%%%%
\begin{slide}{Core JML}
\vspace*{-3ex}

Remember the core JML keywords were

\begin{itemize}
\item {\green \code{requires}}
\item {\green \code{ensures}}
\item {\green \code{signals}}
\item {\green \code{assignable}}
\item {\green \code{normal\_behavior}}
\item {\green \code{invariant}}
\item {\green \code{non\_null}}
\item {\green \code{pure}}
\item {\green \code{\old}, \code{\vooralle}, \code{\eris}, \code{\result}}
\end{itemize}

\end{slide}

%%%%%%%%%%%%%%%%%%%%%%%%%%%%%%%%%%%%%%%%%%%%%%%%%%%%%%%%%%%%%%%
\begin{slide}{More advanced JML features}
\vspace*{-3ex}

\begin{itemize}
\item Visibility
\item Specifying (im)possibility of exceptions
\item Assignable clauses and datagroups
\item Aliasing
\item Specification inheritance, ensuring behavioural subtyping
\item Specification-only fields: \code{ghost} and \code{mode} fields
\item The semantics of \code{invariant}
\end{itemize}


\end{slide}

%%%%%%%%%%%%%%%%%%%%%%%%%%%%%%%%%%%%%%%%%%%%%%%%%%%%%%%%%%%%%%%%%%%%%%%%%%%%
\part{{\Large \red Visibility }}
%%%%%%%%%%%%%%%%%%%%%%%%%%%%%%%%%%%%%%%%%%%%%%%%%%%%%%%%%%%%%%%%%%%%%%%%%%%%

%%%%%%%%%%%%%%%%%%%%%%%%%%%%%%%%%%%%%%%%%%%%%%%%%%%%%%%%%%%%%%%%%%%%%%%%%%%%
\begin{slide}{Visibility}
\vspace*{-4ex}

JML imposes visibility rules similar to Java, eg.

\begin{alltt}
\code{\scriptsize   public class Bag\{
     ...
     {\green private} int{\green n};

    //@ requires{\green n} > 0;
     {\blue public} int{\blue extractMin}()\{ ... \}
}
\end{alltt}  %}

is not type-correct, because
{\blue public} method {\blue \code{extractMin}}
refers to {\green private} field {\green\code{n}}.

\end{slide}
%%%%%%%%%%%%%%%%%%%%%%%%%%%%%%%%%%%%%%%%%%%%%%%%%%%%%%%%%%%%%%%%%%%%%%%%%%%%
\begin{slide}{Visibility}
\vspace*{-4ex}

\begin{alltt}
\code{\scriptsize {\blue public} int{\blue pub}; {\green private} int{\green priv}; 

 {\black //@} requires i <={\blue pub};
 {\blue public} void{\blue pub1} (int i) \{ ... \}

 {\black //@} requires i <={\blue pub} && i <={\green priv};
 {\green private} void{\green priv1} (int i) { ... } 

 {\black //@} requires i <={\blue pub} && i <={\green priv}; // WRONG !!
 {\blue public} void{\blue pub2}(int i) \{ ... \}
 }
\end{alltt}

\end{slide}

%%%%%%%%%%%%%%%%%%%%%%%%%%%%%%%%%%%%%%%%%%%%%%%%%%%%%%%%%%%%%%%%%%%%%%%%%%%%
\begin{slide}{Visibility: \texttt{spec\_public}}
\vspace*{-3ex}

Keyword \code{spec\_public} loosens visibility for specs.\\
Private \code{spec\_public} fields are allowed in public specs, e.g.:

\begin{alltt}
\code{\scriptsize   public class Bag\{
     ...
     {\green private}{\red /*@ spec\_public @*/} int{\green n};

      //@ requires{\green n} > 0;
     {\blue public} int{\blue extractMin}()\{ ... \}
}
\end{alltt} %}

\bigskip

{\scriptsize  Exposing private details can be ugly, of course.  A nicer, but more
  advanced alternative is to use public \code{model}
  fields to represent (abstract away from) private implementation
  details.}

\end{slide}

%%%%%%%%%%%%%%%%%%%%%%%%%%%%%%%%%%%%%%%%%%%%%%%%%%%%%%%%%%%%%%%%%%%%%%%%%%%%
\part{{\Large \red Exceptions and JML }}
%%%%%%%%%%%%%%%%%%%%%%%%%%%%%%%%%%%%%%%%%%%%%%%%%%%%%%%%%%%%%%%%%%%%%%%%%%%%

%%%%%%%%%%%%%%%%%%%%%%%%%%%%%%%%%%%%%%%%%%%%%%%%%%%%%%%%%%%%%%%%%%%%%%%%%%%%
\begin{slide}{Exceptional specifications}
\vspace*{-3ex}

A method specification can (dis)allow the throwing of exceptions,
and specify a exceptional postcondition that should hold in
the event an exception is thrown.

\medskip

There are some implicit rules for (dis)allowing exceptions.

\medskip

Warning: exceptional specifications are 
easy to get wrong. 

\medskip

Not allowing any exceptions to be thrown is the simplest approach.

\end{slide}

%%%%%%%%%%%%%%%%%%%%%%%%%%%%%%%%%%%%%%%%%%%%%%%%%%%%%%%%%%%%%%%%%%%%%%%%%%%%
\begin{slide}{Exceptions allowed by specs} 
\vspace*{-3ex}

By default, a method is allowed to throw exceptions, but only those listed
in its throws clause. So

\begin{alltt} \code{\scriptsize{\green //@}{\black requires} 0 <= amount && amount <= balance;
  public int{\green debit}(int amount)
                           throws BankException 
   \{ ... \}

}
\end{alltt}
has an implicit clause
\begin{alltt}
\code{\scriptsize{\blue signals} (BankException){\green true};}
\end{alltt}
and an implicit clause
\begin{alltt}
\code{\scriptsize{\blue signals} (Exception e){\green e instanceof BankException};}
\end{alltt}

\end{slide}


%%%%%%%%%%%%%%%%%%%%%%%%%%%%%%%%%%%%%%%%%%%%%%%%%%%%%%%%%%%%%%%%%%%%%%%%%%%%
\begin{slide}{Exceptions allowed by specs}
\vspace*{-3ex}

By default, a method is allowed to throw exceptions, \textit{but only those listed
in its throws clause.} So

\begin{alltt} \code{\scriptsize{\green //@}{\black requires} 0 <= amount && amount <= balance;
  public int{\green debit}(int amount) 
   \{ ... \}

}
\end{alltt}
has an implicit clause
\begin{alltt}
\code{\scriptsize{\blue signals} (Exception){\green false};}
\end{alltt}

NB \code{debit} is now not even allowed to throw an unchecked exception,
even though Java does not require a throws clause for these.

\end{slide}




%%%%%%%%%%%%%%%%%%%%%%%%%%%%%%%%%%%%%%%%%%%%%%%%%%%%%%%%%%%%%%%%%%%%%%%%%%%%
\begin{slide}{Ruling out exceptions}
\vspace*{-3ex}

To forbid a particular exception \code{SomeException}
\begin{enumerate}
\item omit it from throws clause (only possible for unchecked exceptions)
\item add an explicit
\begin{alltt} \code{\scriptsize{\blue signals} (SomeException){\green false};}
\end{alltt}

\item limit the set of allowed exceptions, use a postcondition such as
\begin{alltt}\code{\scriptsize
    signals (Exception e) e instanceof E1 
                       || ... 
                       || e instanceof En;}
\end{alltt}
or, equivalently, us the shorthand for this
\begin{alltt}\code{\scriptsize
    signals_only E1, ... En;}
\end{alltt}

\end{enumerate}

\end{slide}


%%%%%%%%%%%%%%%%%%%%%%%%%%%%%%%%%%%%%%%%%%%%%%%%%%%%%%%%%%%%%%%%%%%%%%%%%%%%
\begin{slide}{Ruling out exceptions}
\vspace*{-3ex}

To forbid \textit{all} exceptions
\begin{enumerate}
\item omit all exceptions from throws clause (only possible for unchecked exceptions)
\item add an explicit
\begin{alltt} \code{\scriptsize{\blue signals} (Exception){\green false};}
\end{alltt}
\item
use keyword {\blue normal\_behavior} to rule out all exceptions
\begin{alltt} \code{\scriptsize {\green /*@}{\blue normal\_behavior}
       {\black requires} ...
       {\black ensures}  ...
    {\green @*/}
}
\end{alltt}
\mbox{\code{normal\_behavior} has implicit
\code{signals(Exception)false}}
\end{enumerate}

\end{slide}



%%%%%%%%%%%%%%%%%%%%%%%%%%%%%%%%%%%%%%%%%%%%%%%%%%%%%%%%%%%%%%%%%%%%%%%%%%%%
\begin{slide}{may vs must throw an exception}
\vspace*{-3ex}

Beware of the difference between 
\begin{quote}
{\green (1) if P holds, then \code{SomeException} is thrown}
\end{quote}
and
\begin{quote}
{\blue (2) if \code{SomeException} is thrown, then P holds}
\end{quote}

\medskip

These are easy to confuse!

\medskip

(2) can be expressed with {\blue \code{signals} }, \\
(1) can be expressed with {\green \code{exceptional\_behavior}}

\end{slide}

%%%%%%%%%%%%%%%%%%%%%%%%%%%%%%%%%%%%%%%%%%%%%%%%%%%%%%%%%%%%%%%%%%%%%%%%%%%%
\begin{slide}{\texttt{exceptional\_behavior}}
\vspace*{-3ex}


\begin{alltt}\code{\scriptsize{\green /*@} exceptional_behavior
      requires amount > balance
      signals (BankException e) 
                e.getReason.equals("Amount too big")
{\green   @*/}
  public int{\green debit}(int amount) \{ ...  \}

}
\end{alltt}

says a \code{BankException} \textit{must} be thrown if \code{amount > balance}

\bigskip

\mbox{\code{{\green normal\_behavior}} has implicit
`\code{{\green signals(Exception)false}}}'

\smallskip
\code{\blue exceptional\_behavior} has implicit
`\code{{\blue ensures false}}'


\end{slide}



%%%%%%%%%%%%%%%%%%%%%%%%%%%%%%%%%%%%%%%%%%%%%%%%%%%%%%%%%%%%%%%%%%%%%%%%%%%%
\begin{slide}{Example}
\vspace*{-3ex}

\begin{alltt}\code{\scriptsize{\green /*@} normal_behavior
        requires amount <= balance;
        ensures  ...
    also
      exceptional_behavior
        requires amount > balance
        signals (BankException e) ...
{\green   @*/}
  public int{\green debit}(int amount) throws BankException 
  \{ ...  \}
}
\end{alltt}

\end{slide}

%%%%%%%%%%%%%%%%%%%%%%%%%%%%%%%%%%%%%%%%%%%%%%%%%%%%%%%%%%%%%%%%%%%%%%%%%%%%
\begin{slide}{Example}
\vspace*{-3ex}

or, equivalently

\begin{alltt}\code{\scriptsize{\green /*@} requires true;
     ensures{\red \old(amount<=balance)} && ...
     signals (BankException e)  
            {\red \old(amount>balance)} && ...
{\green   @*/}
  public int{\green debit}(int amount) throws BankException 
  \{ ...  \}
}
\end{alltt}


\end{slide}



%%%%%%%%%%%%%%%%%%%%%%%%%%%%%%%%%%%%%%%%%%%%%%%%%%%%%%%%%%%%%%%%%%%%%%%%%%%%
\begin{slide}{Exceptional behaviour}
\vspace*{-3ex}

Moral: to keep things simple, disallow exceptions
in specs whenever possible

\bigskip

{\footnotesize
Eg, for

\begin{alltt}\code{\scriptsize public void{\green arraycopy}(int[] src, int destOffset, 
                       int[] dest, int destOffset, int length)
        throws NullPointerException,
               ArrayIndexOutOfBoundsException 

}
\end{alltt}

write a spec that disallows any throwing of exceptions, and only
worry about specifying exceptional behaviour if this is really needed
elsewhere.}

\end{slide}

%%%%%%%%%%%%%%%%%%%%%%%%%%%%%%%%%%%%%%%%%%%%%%%%%%%%%%%%%%%%%%%%%%%%%%%%%%%%
\part{{\Large \red Assignable clauses and datagroups }}
%%%%%%%%%%%%%%%%%%%%%%%%%%%%%%%%%%%%%%%%%%%%%%%%%%%%%%%%%%%%%%%%%%%%%%%%%%%%

%%%%%%%%%%%%%%%%%%%%%%%%%%%%%%%%%%%%%%%%%%%%%%%%%%%%%%%%%%%%%%%%%%%%%%%%%%%%
\begin{slide}{Problems with assignable clauses}
\vspace*{-3ex}

Assignable clauses
\begin{itemize}
\item tend to expose implementation details
\begin{alltt}\code{\scriptsize  private /*@ spec_public @*/ int x;
  ...
  //@ assignable x, ....;
  public void m(...) \{....\} }
\end{alltt}
\item tend to become very long
\begin{alltt}\code{\scriptsize  //@ assignable x, y, z[*],  ....;}
\end{alltt}
\item tend to accumulate
\begin{alltt}\code{\scriptsize  //@ assignable x, f.x, g.y, h.z[*], ....;}
\end{alltt}
\end{itemize}
\end{slide}

%%%%%%%%%%%%%%%%%%%%%%%%%%%%%%%%%%%%%%%%%%%%%%%%%%%%%%%%%%%%%%%%%%%%%%%%%%%%
\begin{slide}{Problems with assignable clauses}
\vspace*{-3ex}


\begin{alltt}\code{\scriptsize public class Timer\{
  /*@ spec_public @*/ int time_hrs, time_mins, time_secs; 
  /*@ spec_public @*/ int alarm_hrs, alarm_mins, alarm_secs; 

  //@ assignable time_hrs, time_mins,  time_secs;
  public void tick() \{ ... \}

  //@ assignable alarm_hrs, alarm_mins, alarm_secs ;
  public void setAlarm(int hrs, int mins, int secs) \{ ... \} 
\}}
\end{alltt}


\end{slide}



%%%%%%%%%%%%%%%%%%%%%%%%%%%%%%%%%%%%%%%%%%%%%%%%%%%%%%%%%%%%%%%%%%%%%%%%%%%%
\begin{slide}{Solution: datagroups}
\vspace*{-3ex}


\begin{alltt}\code{\scriptsize public class Timer\{
 {\red //@ model public JMLDatagroup time, alarm;}
  int time_hrs, time_mins, time_secs;{\red //@ in time;}
  int alarm_hrs, alarm_mins, alarm_secs;{\red //@ in alarm;}

  //@ assignable{\red time};
  public void tick() \{ ... \}

  //@ assignable{\red alarm};
  public void setAlarm(int hrs, int mins, int secs) \{ ... \} 
\}

}
\end{alltt}

\medskip


{\scriptsize \code{time} and \code{alarm} are {\red \code{model} fields}, ie.
specification-only fields}


\end{slide}


%%%%%%%%%%%%%%%%%%%%%%%%%%%%%%%%%%%%%%%%%%%%%%%%%%%%%%%%%%%%%%%%%%%%%%%%%%%%
\begin{slide}{Datagroups}
\vspace*{-3ex}


Datagroups provide an abstraction mechanism for assignable clauses.

\medskip

There's a default datagroup \code{objectState} defined in \code{Object.java}

\medskip

It's good practice to declare that all instance fields are
in \code{objectState}

\end{slide}


%%%%%%%%%%%%%%%%%%%%%%%%%%%%%%%%%%%%%%%%%%%%%%%%%%%%%%%%%%%%%%%%%%%%%%%%%%%%
\begin{slide}{Datagroups can be nested}
\vspace*{-3ex}

\begin{alltt}\code{\scriptsize public class Timer\{
  //@ model public JMLDatagroup time, alarm; {\red //@ objectState}
  int time_hrs, time_mins, time_secs; //@ in time;
  int alarm_hrs, alarm_mins, alarm_secs; //@ in alarm;

  //@ assignable{\red time};
  public void tick() \{ ... \}

  //@ assignable{\red alarm};
  public void setAlarm(int hrs, int mins, int secs) \{ ... \}
\}

}
\end{alltt}

\end{slide}



%%%%%%%%%%%%%%%%%%%%%%%%%%%%%%%%%%%%%%%%%%%%%%%%%%%%%%%%%%%%%%%%%%%%%%%%%%%%
\begin{slide}{Assignable and arrays}
\vspace*{-4ex}

Another implementation, using an array of 6 digits to represent
hrs:mns:secs

\begin{alltt}\code{\scriptsize public class ArrayTimer\{
  /*@ spec_public @*/ char[] currentTime; 

  //@ invariant currentTime != null;
  //@ invariant currentTime.length == 6;

  //@ assignable currentTime[*];
  public void tick() \{ ... \}

  ...
\}

}
\end{alltt}

\end{slide}


%%%%%%%%%%%%%%%%%%%%%%%%%%%%%%%%%%%%%%%%%%%%%%%%%%%%%%%%%%%%%%%%%%%%%%%%%%%%
\begin{slide}{Datagroups and arrays}
\vspace*{-4ex}

Another implementation, using an array of 6 digits to represent
hrs:mns:secs

\begin{alltt}\code{\scriptsize public class ArrayTimer\{
 {\red //@ model public JMLDatagroup time;}
  char[] currentTime;{\red //@ in time;}
 {\red //@ maps currentTime[*] \into time;}

  //@ invariant currentTime != null;
  //@ invariant currentTime.length == 6;

  //@ assignable{\red time};
  public void tick() \{ ... \}

  ...
\} }
\end{alltt} %}

\end{slide}


%%%%%%%%%%%%%%%%%%%%%%%%%%%%%%%%%%%%%%%%%%%%%%%%%%%%%%%%%%%%%%%%%%%%%%%%%%%%
\begin{slide}{Datagroups and interfaces}
\vspace*{-3ex}

Datagroups are convenient in specs for interfaces.


\begin{alltt}\code{\scriptsize public interface TimerInterface\{
  //@ model{\red instance} public JMLDatagroup time, alarm;
  //@ assignable time;
  public void tick();
  //@ assignable alarm;
  public void setAlarm(int hrs, int mins, int secs);
\}

}
\end{alltt}

Classes implementing this interface are free to choose which fields
are in the datagroups.

{\scriptsize\rm Keyword \texttt{\textbf{instance}} is needed, because fields in interfaces
are by default static fields in Java.
Interfaces in Java do not have instance fields,
but in JML they can have \textit{model} instance fields
}


\end{slide}


%%%%%%%%%%%%%%%%%%%%%%%%%%%%%%%%%%%%%%%%%%%%%%%%%%%%%%%%%%%%%%%%%%%%%%%%%%%%
\begin{slide}{The problem with assignable clauses}
\vspace*{-3ex}

Despite the abstraction possibilities offered by datagroups,
assignable clauses remain a bottlecneck both in program specification
and in program verification.

\medskip

{\scriptsize\rm Note that the proof obligation corresponding to an assignable
clause is a very complicated one, involving a quantification
over all fields not mentioned in the assignable clause}

\end{slide}



%%%%%%%%%%%%%%%%%%%%%%%%%%%%%%%%%%%%%%%%%%%%%%%%%%%%%%%%%%%%%%%%%%%%%%%%%%%%
\part{{\Large \red Aliasing (revisited)}}
%%%%%%%%%%%%%%%%%%%%%%%%%%%%%%%%%%%%%%%%%%%%%%%%%%%%%%%%%%%%%%%%%%%%%%%%%%%%

%%%%%%%%%%%%%%%%%%%%%%%%%%%%%%%%%%%%%%%%%%%%%%%%%%%%%%%%%%%%%%%%%%%%%%%%%%%%
\begin{slide}{Aliasing}
\vspace*{-3ex}

Aliasing is the root of all evil, for anyone trying trying to verify 
imperative programs.

\medskip

The \code{ArrayTimer} class just earlier is another nice example
to illustrate this.

\end{slide}


%%%%%%%%%%%%%%%%%%%%%%%%%%%%%%%%%%%%%%%%%%%%%%%%%%%%%%%%%%%%%%%%%%%%%%%%%%%%
\begin{slide}{ArrayTimer example}
\vspace*{-4ex}

Recall implementation using an array of 6 digits to represent
hrs:mns:secs

\begin{alltt}\code{\scriptsize public class ArrayTimer\{
  char[] currentTime;                    
  char[] alarmTime;

  //@ invariant currentTime != null;
  //@ invariant currentTime.length == 6;
  //@ invariant alarmTime != null;
  //@ invariant alarmTime.length == 6;

  public void tick() \{ ... \}

  public void setAlarm(...) \{ ... \}
\}

}
\end{alltt}

\end{slide}




%%%%%%%%%%%%%%%%%%%%%%%%%%%%%%%%%%%%%%%%%%%%%%%%%%%%%%%%%%%%%%%%%%%%%%%%%%%%
\begin{slide}{ArrayTimer example}
\vspace*{-4ex}


Things will go wrong if different instances of \code{ArrayTimer}
share the same \code{alarmTime} array, ie.
\begin{alltt}\code{\scriptsize
  o.alarmTime == o'.alarmTime}
\end{alltt}
for some \code{o!=o'}

\medskip

ESC/Java2 may notice this, produce a correct, but puzzling, warning.


\end{slide}
%%%%%%%%%%%%%%%%%%%%%%%%%%%%%%%%%%%%%%%%%%%%%%%%%%%%%%%%%%%%%%%%%%%%%%%%%%%%
\begin{slide}{ArrayTimer example}
\vspace*{-4ex}

To rule out such aliasing of eg. \code{alarmTime} fields:

\begin{alltt}\code{\scriptsize public class ArrayTimer\{
  char[] alarmTime;
  //@ invariant currentTime.owner == this;
  ...

  public ArrayTimer( ...)\{
    ...;
    currentTime = new char[6];
    //@ set currentTime.owner == this;
    ...
  \} 
}
\end{alltt} %}

{\scriptsize\rm (\texttt{\textbf{owner}} is a so-called ghost field, more 
about that later)}

\end{slide}




%%%%%%%%%%%%%%%%%%%%%%%%%%%%%%%%%%%%%%%%%%%%%%%%%%%%%%%%%%%%%%%%%%%%%%%%%%%%
\begin{slide}{ArrayTimer example}
\vspace*{-4ex}

Things will go wrong if an instance of \code{ArrayTimer}
aliases its \code{alarmTime}  to its \code{currentTime}, ie.
\begin{alltt}\code{\scriptsize
  o.alarmTime == o.currentTime

}
\end{alltt}

ESC/Java2 may notice this, produce a correct, but puzzling warning.

\medskip

You should add
\begin{alltt}\code{\scriptsize
  //@ invariant alarmTime != currentTime;

}
\end{alltt}
to rule out such aliasing.

\end{slide}



\ifignore 
% in part 4
%%%%%%%%%%%%%%%%%%%%%%%%%%%%%%%%%%%%%%%%%%%%%%%%%%%%%%%%%%%%%%%%%%%%%%%%%%%%
\part{{\Large \red Specification inheritance and behavioural subtyping }}
%%%%%%%%%%%%%%%%%%%%%%%%%%%%%%%%%%%%%%%%%%%%%%%%%%%%%%%%%%%%%%%%%%%%%%%%%%%%

%%%%%%%%%%%%%%%%%%%%%%%%%%%%%%%%%%%%%%%%%%%%%%%%%%%%%%%%%%%%%%%%%%%%%%%%%%%%
\begin{slide}{Behavioural subtyping}
\vspace*{-3ex}

Suppose \code{Child extends Parent}.

\begin{itemize}
\item $\!$ {\green Behavioural subtyping} =
objects from subclass \code{Child} ``behave like'' 
objects from superclass \code{Parent}
\item $\!$ {\green Principle of substitutivity} [Liskov]:\\
code will behave ``as expected'' if we provide an
\code{Child} object where a \code{Parent} object was expected. 
\end{itemize}

\end{slide}

%%%%%%%%%%%%%%%%%%%%%%%%%%%%%%%%%%%%%%%%%%%%%%%%%%%%%%%%%%%%%%%%%%%%%%%%%%%%
\begin{slide}{Behavioural subtyping}
\vspace*{-3ex}

Behavioural subtyping usually enforced by insisting that

\begin{itemize}
\item 
invariant in subclass is {\green stronger} than invariant in superclass
\item
for every method,
\begin{itemize}
\item 
precondition in subclass is {\blue weaker} (!) than precondition is superclass
\item 
postcondition in subclass is {\green stronger} than postcondition is superclass
\end{itemize}
\end{itemize}

JML achieves this using {\green specification inheritance}:
any child class {\green inherits} the specification of its parent.

\end{slide}

%%%%%%%%%%%%%%%%%%%%%%%%%%%%%%%%%%%%%%%%%%%%%%
\begin{slide}{$\!\!\!$\small Specification inheritance for invariants}
\vspace*{-3ex}

Invariants are inherited in subclasses. Eg.
\begin{alltt}
\code{\scriptsize class Parent \{
     ...
     //@ invariant invParent;
     ...  \}

  class Child extends Parent \{
     ...
     //@ invariant invChild;
     ...  \}}
\end{alltt}

the invariant for \code{Child} is \code{invChild \&\& invParent}

\end{slide}


%%%%%%%%%%%%%%%%%%%%%%%%%%%%%%%%%%%%%%%%%%%%%%
\begin{slide}{$\!\!\!\!\!$\small Specif\-ication inheritance for methods specs}
\vspace*{-4ex}
\begin{alltt}
\code{{\scriptsize class Parent \{
     //@ requires i >= 0;
     //@ ensures  \result >= i;
     int m(int i)\{ ... \}
  \}

  class Child extends Parent \{
     //@{\green also}
     //@   requires i <= 0
     //@   ensures  \result <= i;
     int m(int i)\{ ... \}
   \}}}
\end{alltt}

{\scriptsize Keyword {\green \texttt{also}} indicates there are inherited specs.}

\end{slide}

%%%%%%%%%%%%%%%%%%%%%%%%%%%%%%%%%%%%%%%%%%%%%%
\begin{slide}{$\!\!\!\!\!$\small Specif\-ication inheritance for methods specs}
\vspace*{-4ex}

Method \code{m} in \code{Child} also has to meet the spec
given in \code{Parent} class.
So the complete spec for \code{Child} is

\begin{alltt}
\code{\scriptsize class Child extends Parent \{

  /*@   requires i >= 0;
    @   ensures  \result >= i;
    @{\green also}
    @   requires i <= 0
    @   ensures  \result <= i;
    @*/
  int m(int i)\{ ... \}
 \}}
\end{alltt}

{\tiny What can result of \code{m(0)} be?}

\end{slide}

%%%%%%%%%%%%%%%%%%%%%%%%%%%%%%%%%%%%%%%%%%%%%%
\begin{slide}{$\!\!\!\!\!$\small Specification inheritance for methods specs}
\vspace*{-3ex}

This is equivalent with

\begin{alltt}
\code{\scriptsize class Child extends Parent \{

  /*@   requires i <= 0 || i >= 0;
    @   ensures  \old(i) >= 0 ==> \result >= i;
    @   ensures  \old(i) <= 0 ==> \result <= i;
    @*/
  int m(int i)\{ ... \}
 \}}
\end{alltt}

\end{slide}

\fi


%%%%%%%%%%%%%%%%%%%%%%%%%%%%%%%%%%%%%%%%%%%%%%%%%%%%%%%%%%%%%%%%%%%%%%%%%%%%
\part{{\Large \red Specification-only fields: \\ \texttt{ghost} and \texttt{model} fields}}
%%%%%%%%%%%%%%%%%%%%%%%%%%%%%%%%%%%%%%%%%%%%%%%%%%%%%%%%%%%%%%%%%%%%%%%%%%%%

%%%%%%%%%%%%%%%%%%%%%%%%%%%%%%%%%%%%%%%%%%%%%%%%%%%%%%%%%%%%%%%%%%%%%%%%%%%%
\begin{slide}{Ghost fields}
\vspace*{-3ex}

Sometimes it is convenient to introduce an
extra field, only for the purpose of specification
(aka auxiliary variable).

\medskip

A {\blue\code{ghost}} field is like a normal field,
except that it can only be used in specifications.

\medskip

A special {\blue\code{set}} command can be used
to assign a value to a \code{ghost} field.

\end{slide}

%%%%%%%%%%%%%%%%%%%%%%%%%%%%%%%%%%%%%%%%%%%%%%%%%%%%%%%%%%%%%%%%%%%%%%%%%%%%
\begin{slide}{Ghost fields - example} 
\vspace*{-3ex}

Suppose the informal spec of 
\begin{alltt}

   \code{\scriptsize{class SimpleProtocol \{

     void start() \{ ... \}
         
     void stop() \{ ... \}
    \}}}

\end{alltt} %}
says that \code{stop()}
may only be invoked after \code{start()},
and vice versa.

\medskip

This can be expressed using a {\blue ghost} field,
to represent the {\green state} of the protocol.

\end{slide}

%%%%%%%%%%%%%%%%%%%%%%%%%%%%%%%%%%%%%%%%%%%%%%%%%%%%%%%%%%%%%%%%%%%%%%%%%%%%
\begin{slide}{Ghost fields - example}
\vspace*{-4ex}

\begin{alltt} \code{\scriptsize{class SimpleProtocol \{
 {\blue //@ boolean ghost started;}

 {\green //@ requires  !started;}
 {\green //@ assignable started;}
 {\green //@ ensures    started;}
  void start() \{ 
     ...
    {\blue //@ set started = true;} \}
      
 {\green //@ requires   started;}
 {\green //@ assignable started;}
 {\green //@ ensures   !started;}
  void stop() \{ 
     ... 
    {\blue //@ set started = false;} \}
 }}
\end{alltt} %}
\end{slide}

%%%%%%%%%%%%%%%%%%%%%%%%%%%%%%%%%%%%%%%%%%%%%%%%%%%%%%%%%%%%%%%%%%%%%%%%%%%%
\begin{slide}{Ghost fields - example}
\vspace*{-4ex}

Maybe the object has some internal state that that records
if protocols is in progress, eg.

\begin{alltt}
\code{\scriptsize{class SimpleProtocol \{
 {\green //@ private ProtocolStack st;}
  ...
  void startProtocol() \{ 
     ...
     st = new ProtocolStack(...);
     ...  \}
      
  void endProtocol() \{ 
     ... 
     st = null;
     ...  \}
 }}
\end{alltt} %}
\end{slide}

%%%%%%%%%%%%%%%%%%%%%%%%%%%%%%%%%%%%%%%%%%%%%%%%%%%%%%%%%%%%%%%%%%%%%%%%%%%%
\begin{slide}{Ghost fields - example}
\vspace*{-4ex}

There may be correspondence between the {\blue ghost field} and some
{\green other field(s)}, eg.

\begin{alltt}
\code{\scriptsize{class SimpleProtocol \{
 {\green private ProtocolStack st;}
 {\blue //@ boolean ghost started;}
  //@ invariant{\blue started} <==>{\green (st !=null)};

  //@ requires  !started;
  //@ assignable started;
  //@ ensures    started;
  void startProtocol() \{ ...  \}
      
  //@ requires   started;
  //@ assignable started;
  //@ ensures   !started;
  void endProtocol() \{ ...  \} }}
\end{alltt} %}
\end{slide}

%%%%%%%%%%%%%%%%%%%%%%%%%%%%%%%%%%%%%%%%%%%%%%%%%%%%%%%%%%%%%%%%%%%%%%%%%%%%
\begin{slide}{Ghost fields - example}
\vspace*{-4ex}

We could now get rid of the ghost field, and write

\begin{alltt} \code{\scriptsize{class SimpleProtocol \{
 private{\green spec\_public} ProtocolStack st;

  //@ requires  {\green st==null};
  //@ assignable st;
  //@ ensures  {\green  st!=null};
  void startProtocol() \{ ... \}
      
  //@ requires  {\green st!=null};
  //@ assignable st;
  //@ ensures   {\green st==null};
  void endProtocol() \{ ... \} }}
\end{alltt} %}

but this is ugly and exposes implementation details.

\end{slide}

%%%%%%%%%%%%%%%%%%%%%%%%%%%%%%%%%%%%%%%%%%%%%%%%%%%%%%%%%%%%%%%%%%%%%%%%%%%%
\begin{slide}{Model fields - example}
\vspace*{-4ex}

Solution: use a {\blue model} field

\begin{alltt} \code{\scriptsize class SimpleProtocol \{
  private ProtocolStack st;  

 {\blue //@ boolean model started;} 
 {\blue //@ represents started <-- (st!=null);}

  //@ requires !started;
  //@ assigable started;
  //@ ensures   started;
  void startProtocol() \{ ... \}
      
  //@ requires  started;
  //@ assigable started;
  //@ ensures  !started;
  void endProtocol() \{ ... \} }
\end{alltt} %}
\end{slide}

%%%%%%%%%%%%%%%%%%%%%%%%%%%%%%%%%%%%%%%%%%%%%%%%%%%%%%%%%%%%%%%%%%%%%%%%%%%%
\begin{slide}{Model fields - example}
\vspace*{-4ex}

A model field also provided an associated {\blue datagroup}

\begin{alltt} \code{\scriptsize class SimpleProtocol \{
  private ProtocolStack st;  {\red //@ in started;}

 {\blue //@ boolean model started;} 
 {\blue //@ represents started <-- (st!=null);}

  //@ requires !started;
  //@ assigable started;
  //@ ensures   started;
  void startProtocol() \{ ... \}
      
  //@ requires  started;
  //@ assigable started;
  //@ ensures  !started;
  void endProtocol() \{ ... \} }
\end{alltt} %}
\end{slide}




%%%%%%%%%%%%%%%%%%%%%%%%%%%%%%%%%%%%%%%%%%%%%%%%%%%%%%%%%%%%%%%%%%%%%%%%%%%%
\begin{slide}{Model vs ghost fields}
\vspace*{-3ex}


Difference between {\green ghost} and {\blue model} is maybe confusing!

Both exist only in JML specification, and not in the code.

\begin{itemize}
\item$\!$ {\green Ghost}\\
$\bullet$ Ghost field is like a normal field. \\
$\bullet$ You can assign to it, using \code{set}, in JML annotations.
\item$\!$ {\blue Model}\\
$\bullet$ Model field is an abstract field.\\
$\bullet$ Model field is a convenient abbreviation.\\
$\bullet$ You cannot assign to it.\\
$\bullet$ Model field changes
its value whenever the representation changes.
\end{itemize}

{\scriptsize Model field is like `abstract value' for ADT (algebraic data type),
represent clause is like `representation function'.}

\end{slide}


%%%%%%%%%%%%%%%%%%%%%%%%%%%%%%%%%%%%%%%%%%%%%%%%%%%%%%%%%%%%%%%%%%%%%%%%%%%%
\part{{\Large \red Invariants}}
%%%%%%%%%%%%%%%%%%%%%%%%%%%%%%%%%%%%%%%%%%%%%%%%%%%%%%%%%%%%%%%%%%%%%%%%%%%%

%%%%%%%%%%%%%%%%%%%%%%%%%%%%%%%%%%%%%%%%%%%%%%%%%%%%%%%%%%%%%%%%%%%%%%%%%%%%
\begin{slide}{Invariants}
\vspace*{-3ex}

Invariants -- aka class invariants -- are a common \& very useful notion.

\medskip

In larger programs, the (only) interesting thing to specify are the invariants.

\medskip

However, the semantics is trickier than expected!

\medskip

Invariant is implicitly included in pre- and postconditions
of method, and in postcondition of constructors.
But there's more ...

\medskip

In fact, invariants are a hot research topic.

\end{slide}



%%%%%%%%%%%%%%%%%%%%%%%%%%%%%%%%%%%%%%%%%%%%%%%%%%%%%%%%%%%%%%%%%%%%%%%%%%%%
\begin{slide}{When should an invariant hold?}
\vspace*{-3ex}


\begin{alltt}\code{\scriptsize public class A \{
   B b;
   int i=2; //@ invariant i >= 0
   
   //@ ensures \result >=0;
   public /*@ pure @*/ int get()\{ return i; \}
    
   public void m()\{
     i--;
     ... ; // invariant possibly broken
     i++;
   \}}
\end{alltt} %}

An invariant can temporarily be broken.

\end{slide}

%%%%%%%%%%%%%%%%%%%%%%%%%%%%%%%%%%%%%%%%%%%%%%%%%%%%%%%%%%%%%%%%%%%%%%%%%%%%
\begin{slide}{When should an invariant hold?}
\vspace*{-3ex}

\begin{alltt}\code{\scriptsize public class A \{
   B b;
   int i=2; //@ invariant i >= 0

   //@ ensures \result >=0;
   public /*@ pure @*/ int get()\{ return i; \}
    
   public void m()\{
     i--;
    {\red b.m(...a)}; // invariant possibly broken
     i++;
   \}}
\end{alltt} %}

This may lead to problems if invocation of \code{b.m}
involves an invoking on the current object.

\end{slide}

%%%%%%%%%%%%%%%%%%%%%%%%%%%%%%%%%%%%%%%%%%%%%%%%%%%%%%%%%%%%%%%%%%%%%%%%%%%%
\begin{slide}{When should an invariant hold?}
\vspace*{-3ex}

Eg, suppose

\begin{alltt}\code{\scriptsize public class B \{
 ...
   public void m(A a)\{ 
    ... 
    int j = a.get(); //@ assert i>=0;
    ...
   \}}
\end{alltt} %}

The spec of \code{get()} suggests the assertion will be true.

\medskip

But if \code{get()} is invoked when \code{a}'s invariant is broken, all bets are off.

\medskip

This is known as the {\red call-back} problem.

\end{slide}


%%%%%%%%%%%%%%%%%%%%%%%%%%%%%%%%%%%%%%%%%%%%%%%%%%%%%%%%%%%%%%%%%%%%%%%%%%%%
\begin{slide}{When should an invariant hold?}
\vspace*{-3ex}

Solution to the call-back problem:
\begin{quote}
A method invariant should at all so-called {\blue visible states},
which are all beginning and end states of method invocations
\end{quote}


So there's more to invariants than just adding them to pre- and postconditons.

\medskip

NB \textit{all} invariants of \textit{all} objects should hold at 
visible states;
this clearly imposes impossible proof obligations.

\medskip

ESC/Java2 looks only at the invariants of some objects;
this is a source of unsoundness.

\medskip

Modular verification techniques for invariants are a challenge, and still 
a hot topic of research.

\end{slide}


%%%%%%%%%%%%%%%%%%%%%%%%%%%%%%%%%%%%%%%%%%%%%%%%%%%%%%%%%%%%%%%%%%%%%%%%%%%%
\begin{slide}{When should an invariant hold?}
\vspace*{-3ex}

Sometimes you do want to invoke a method at a program point
where the invariant is broken.
To do this without ESC/Java2 complaining:

\begin{itemize}
\item
A private method can be declared as \code{helper} method
\begin{alltt}\code{\scriptsize private /*@ helper @*/ void m() \{ ... \}}
\end{alltt}
Invariants do not have to hold when such a helper method is called.

\medskip

Effectively, such methods are in-lined in verifications

\item
add 
\begin{alltt}\code{\scriptsize //@ nowarn Invariant}
\end{alltt}
in the line where this method call occurs.

\smallskip

NB this is unsafe, and a possible source of unsoundness!

\end{itemize}

\end{slide}




%%%%%%%%%%%%%%%%%%%%%%%%%%%%%%%%%%%%%%%%%%%%%%%%%%%%%%%%%%%%%%%%%%%%%%%%%%%%
\begin{slide}{More problems with invariants}
\vspace*{-3ex}


\begin{alltt}\code{\scriptsize public class SortedLinkedList \{
   private int i;
   private LinkedList next;
   //@ invariant i > next.i;
   public /*@ pure @*/ int getValue()\{ return i; \}
   public /*@ pure @*/ int getNext()\{ return next; \}
   public /*@ pure @*/ int getValue()\{ return i; \}
   public void inc() \{ i++; \}
  \}}
\end{alltt}

\code{i++} won't break this object's invariant, but may break the invariant
of the object who has this object as it's \code{next}.

\end{slide}


%%%%%%%%%%%%%%%%%%%%%%%%%%%%%%%%%%%%%%%%%%%%%%%%%%%%%%%%%%%%%%%%%%%%%%%%%%%%
\begin{slide}{More problems with invariants}
\vspace*{-3ex}

The essence of the problem is that the invariant of
one object \code{o} may depend on the state of another object \code{o'}.

\medskip

When verifying the methods of \code{o'}, we have no way of knowing 
if these may break invariants given in other classes \ldots

\medskip

This is one of the sources of unsoundness of ESC/Java(2),
and most approaches to modular verification of OO programs to date.

\medskip

Only recently workable approaches for modular verification of invariants 
have been proposed, and the last word has not been said on this.

\end{slide}


\end{document}

