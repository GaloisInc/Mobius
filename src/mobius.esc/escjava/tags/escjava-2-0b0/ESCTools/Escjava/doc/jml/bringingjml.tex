%
% $Id$
%

\documentclass{acm_proc_article-sp}[10pt]

\newif\ifpdf
\ifx\pdfoutput\undefined
   \pdffalse     % no PDFLaTeX
\else
   \pdfoutput=1  % PDFLaTeX
   \pdftrue
\fi

\usepackage{times}
%\usepackage{a4wide}

\ifpdf
\usepackage[pdftex]{graphicx}
\else
\usepackage{graphicx}
\fi

%\usepackage{html}
%\usepackage{url}
\usepackage{alltt}
\usepackage{xspace}
%\usepackage{doublespace}

\ifpdf
\usepackage[pdftex,bookmarks=false,
            plainpages=false,naturalnames=true,
            colorlinks=true,pdfstartview=FitV,
            linkcolor=blue,citecolor=blue,urlcolor=blue]{hyperref}
\else
\usepackage[dvips]{hyperref}
\fi

\newcommand{\examplesize}{\footnotesize}
\newcommand{\daysince}{\texttt{days\_\-since\_\-jan1\_\-1970}}

\newtheorem{Definition}{Definition}
\newtheorem{Theorem}{Theorem}

%---------------------------------------------------------------------
% New commands, macros, etc.
%---------------------------------------------------------------------

% Global

%\newcommand{\myhref}[2]{\ifpdf\href{#1}{#2}\else\htmladdnormallinkfoot{#2}{#1}\fi}

\newcommand{\todo}{\textbf{TODO:}}

%=====================================================================

\begin{document}

% --- Author Metadata here ---
%\conferenceinfo{GCSE/SAIG '02}{2002 Pittsburgh, PA, USA}
%\setpagenumber{50}
\CopyrightYear{2003}
% Allows default copyright data (0-89791-88-6/97/05) to be over-ridden - IF NEED BE.
%\crdata{0-12345-67-8/90/01}
% --- End of Author Metadata ---

\title{Bringing JML to ESC/Java}
\numberofauthors{2}
\author{
\alignauthor Joseph R.~Kiniry\\
       \affaddr{Computing Science Department, University of Nijmegen}\\
       \affaddr{Toernooiveld 1, 6525 ED Nijmegen}\\
       \affaddr{The Netherlands}\\
       \email{kiniry@acm.org}
\alignauthor David Cok\\
       \affaddr{Eastman Kodak Company}\\
       \affaddr{R\&D Laboratories}\\
       \affaddr{1700 Dewey Avenue, Building 65}\\
       \affaddr{Rochester, New York, USA}\\
       \email{cok@frontiernet.net}
}

\maketitle

%======================================================================
\begin{abstract}
  
  Abstract.

% 1 words
  
\end{abstract}

%=====================================================================
% A category with only the three required fields
%% \category{D.1.0}{Software}
%%                 {Programming Techniques}
%%                 [General]
%% \category{D.2}{Software}
%%               {Software Engineering}
%% \category{D.3.1}{Software}
%%                 {Programming Languages}
%%                 [Formal Definitions and Theory]
%% \category{D.3.4}{Software}
%%                 {Programming Languages}
%%                 [Processors]
%%                 [preprocessors]
%% \category{F.3.1}{Theory of Computation}
%%                 {Logics and Meanings of Programs}
%%                 [Specifying and Verifying and Reasoning about Programs]
%% \category{F.4.1}{Theory of Computation}
%%                 {Mathematical Logic and Formal Languages}
%%                 [Mathematical Logic]
%% \category{F.4.3}{Theory of Computation}
%%                 {Mathematical Logic and Formal Languages}
%%                 [Formal Languages]
                
%%                 \terms{specification languages, formal methods, kind
%%                   theory, reuse, glue-code generation, automatic
%%                   programming, domain-specific languages}

% (+ 1) ~= 1

%=====================================================================
% Introduction stating precisely the problem addressed in the paper. 
%=====================================================================
\section{Introduction} % 0 words

All existing bits in this outline so far are lifted directly from the
ESC/Java User's Manual for reference.  We need to cross-check them
with the semantics of the current JML tools and update as appropriate.
Also, of course, all of these bits will need to be rewritten as
appropriate for this paper.

%=====================================================================
\section{Basic Issues}

%=====================================================================
\section{Pure Method Invocations in Assertions}

%=====================================================================
\section{Model Methods and Variables}

%=====================================================================
\section{Refinement}

%=====================================================================
\section{Routine Specification}

Except for formal parameters, identifiers used in postconditions of a
routine (and not within \\old) denote their values in the post-state.
While Java allows a routine body to include assignment to the
routine's formal parameters (thus using the parameters as local
variables), such assignments have no effect as seen by the caller,
since parameters are passed by value.  Therefore ESC/Java interprets
occurrences of formal parameters in postconditions as denoting the
original (pre-state) actual parameter values.

\subsection{Referring to a Variable's Pre-State}

A postcondition E may contain expressions of the form \\old(X).
Roughly speaking, \\old(X) means the value of X in the pre-state.  The
static type of \\old(X) is the same as the static type of X.  An
expression X used as an argument of \\old may not itself contain
applications of \\old or \\fresh.  More precise details are given below.

\subsection{Other Useful Expressions}

Within E, an expression of the form \\fresh(R) where R is a
specification expression of a reference type is true if the object
denoted by R in the post-state is allocated in the post-state
(implying that R != null in the post-state) and was not allocated in
the pre-state.  The static type of \\fresh(R) is boolean.

\subsection{Framing Pragmas}

The state components named in modifies pragmas of a routine are called
modification targets of the routine.  When checking code that calls a
routine, ESC/Java assumes that the call modifies only the routine's
modification targets (with the usual substitutions) and possibly also
any freshly allocated state, regardless of whether the call terminates
normally or abruptly. However, the current ESC/Java does not enforce
modifies pragmas when checking a routine's implementation.

If no modification targets are specified for a routine, then ESC/Java
will assume that calls to the routine modify only freshly allocated
state, if any.

The current ESC/Java does not check that the body of a routine
actually obeys the constraint expressed by the routine's modifies
pragmas.  This lack of checking is one of several potential sources of
missed warnings (unsoundness).  The potential for missed warnings is
mitigated somewhat by a fact that may have seemed surprising when we
mentioned it in the previous section (2.3.2): If a particular field
(either a static field or an instance variable) is not specified as a
target field (section 2.3.1.0) of a routine, then occurrences of that
field within arguments to \\old in the routine's postconditions are
taken to refer to post-state values.

Consider, for example a class with an integer field f and a method
incf declared as follows, with no modifies pragma:
\begin{alltt}
    //@ ensures f == \\old(f) + 1;
    void incf() {
      this.f++;
    }
\end{alltt}
    
Since f is not specified as a modification target of incf, ESC/Java
will interpret both occurrences of f in the ensures pragma as
referring to the post-state value of this.f.  Consequently ESC/Java
will be unable to show that the method establishes the specified
postcondition, and will issue a warning to that effect.

While this warning may seem surprising, the result of interpreting the
second occurrence of f as the pre-state value of this.f would be even
worse.  Under the latter interpretation ESC/Java would issue no
warnings about the body of incf, but would assume after a call
x.incf() both (1) that x.f had been incremented in accordance the
postcondition), and (2) that x.f was left unchanged in accordance with
the (unchecked) empty set of modification targets.  Since these
assumptions are mutually contradictory, the result would be equivalent
to assuming false, and ESC/Java would silently omit all checking after
the call.

As an additional guard against omission of modifies pragmas, ESC/Java
issues a caution message for any occurrence of \\old(X) in a
postcondition of a method m unless (1) the expression X mentions some
target field of m, or (2) the expression X includes an array access
and m has some modification target of the form A[I] or A[*].

Of course, the interactions of modifies and \\old described above do
not entirely make up for the fact that the current ESC/Java does no
checking of modifies pragmas.  A method declaration like
\begin{alltt}
    //@ modifies someOtherObject.f; //instead of this.f
    //@ ensures f == \\old(f) + 1;
    void incf() {
      this.f++;
    }
\end{alltt}
can still effectively disable checking of code following calls to incf.

\subsection{Specification Scope Modifiers}

It is a source of potential unsoundness for a postcondition to mention
a variable that might not be spec-accessible (section 3.3) to an
override of that method, and ESC/Java may forbid such postconditions.
In particular ESC/Java forbids postconditions of a method that mention
private variables except when the routine is static, is final, is
private, or is declared in a final class, or when the private
variables mentioned are declared spec\_public (section 2.5.0).  The
current ESC/Java doesn't forbid, for example, postconditions of public
methods from mentioning package variables, but future versions of
ESC/Java may not be so lenient.

\subsection{Specification Inheritance}

The semantics of specification inheritance in the original ESC/Java
and the JML tools differed considerably~\cite[Section
2.3]{ESCJavaUsersManual}.

In ESC/Java, when a method of a class or interface $S$ inherits or
overrides~\cite[8.4.6]{JLS2} a method $m$ from a class or interface
$T$, the method $S.m$ inherits all the preconditions, modification
targets, and postconditions of $T.m$, with the formal parameter names
of $S.m$ being substituted for those of $T.m$. This treatment of
inheritance sometimes leads to unsound checking in the presence of
multiple inheritance~\cite[Section C.0.5]{ESCJavaUsersManual}.

A method declaration that overrides another method declaration cannot
be modified with a \texttt{requires} pragma, but inherits the
overridden method's preconditions as described above.

A method declaration that overrides another method declaration cannot
be annotated with a modifies pragma, but inherits the modification
targets of the overridden method.  Note that this forbids the
overriding method from modifying additional state, but see the
description of also\_modifies below (section 2.3.8).

The also\_require pragma is a potential source of unsoundness.  Suppose
method C.m of class C overrides method I.m of interface I.  When
checking a call of the form E.m(...), where E is an expression of type
I, ESC/Java only enforces the preconditions of I.m and not any
preconditions given by also\_requires pragmas modifying the declaration
of C.m.  However, the expression E might evaluate to a value of type
C, causing the call to invoke C.m, and the correctness of C.m's
implementation may depend on preconditions given in such also\_requires
pragmas.  The reason that ESC/Java includes an also\_requires pragma,
despite its unsoundness, is that it is often essential for
preconditions of a method C.m to mention instance variables of class
C, and Java does not allow instance variables to be declared in
interfaces.  [Note, however, that ESC/Java does allow declarations of
ghost variables (section 2.6.0) in interfaces.]

%=====================================================================
\section{Remaining Incompatibilities}

%=====================================================================
% Conclusion assessing the results of the work described and its
% limitations.
%=====================================================================
\section{Conclusion} % 0 words

Conclusion.

%=====================================================================
%\section{Acknowledgments}

%======================================================================

\bibliographystyle{plain}
%\bibliography{abbrev,ads,category,complexity,conferences,hypertext,icsr,journals,knowledge,languages,linguistics,meta,metrics,misc,modeling,modeltheory,reuse,rewriting,softeng,specification,ssr,technology,theory,upcoming,upcoming_conferences,web,workshops}

%======================================================================

%\newpage
%\appendix

%======================================================================

%======================================================================
%\section{Section Name}

%======================================================================
% Fin

\end{document}

%%% Local Variables: 
%%% mode: latex
%%% TeX-master: t
%%% End: 
