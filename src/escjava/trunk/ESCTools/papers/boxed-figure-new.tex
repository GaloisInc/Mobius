%% boxed-figure-new.tex
%
% formatting boxed displays

%% some parameters
\newlength{\BFIGURESKIP}
\setlength{\BFIGURESKIP}{3mm}
\newlength{\BFIGURELINEHEIGHT}
\setlength{\BFIGURELINEHEIGHT}{0.3pt}

%% Use the following for a line at the top of a figure,
%% it is raised by \BFIGURESKIP
%% The \par at the end seems to be needed for postscript figures that follow.
\newcommand{\BTOPFIGURELINE}{%
{\rule[\BFIGURESKIP]{\columnwidth}{\BFIGURELINEHEIGHT}}\par%
}

%% A line, with not raised
\newcommand{\BFIGURELINE}{
{\rule{\columnwidth}{\BFIGURELINEHEIGHT}}
% \begin{flushleft}\rule{\columnwidth}{\BFIGURELINEHEIGHT}\end{flushleft}%
}

%% regular figures, the width of the current column of text
\newenvironment{BFIGURE}{
\begin{figure}
\BFIGURELINE
}{%
\par\vskip-\parskip
\BFIGURELINE%
\end{figure}
}

%% figures on a page by themselves
\newenvironment{BFIGUREP}{
\begin{figure}[p!]
\BFIGURELINE
}{%
\par\vskip-\parskip
\BFIGURELINE%
\end{figure}
}

%% figures at the top of the page
\newenvironment{BFIGURET}{
\begin{figure}[t!]
\BFIGURELINE
}{%
\par\vskip-\parskip
\BFIGURELINE%
\end{figure}
}

%% A line, with not raised
\newcommand{\BFULLFIGURELINE}{
{\rule{\textwidth}{\BFIGURELINEHEIGHT}}
% \begin{flushleft}\rule{\textwidth}{\BFIGURELINEHEIGHT}\end{flushleft}%
}

%% full page figures, the width of the whole page
\newenvironment{BFIGURE*}{
\begin{figure*}
\BFULLFIGURELINE
}{%
\par\vskip-\parskip
\BFULLFIGURELINE%
\end{figure*}
}

%% full page figures, on a page by themselves
\newenvironment{BFIGUREP*}{
\begin{figure*}[p!]
\BFULLFIGURELINE
}{%
\par\vskip-\parskip
\BFULLFIGURELINE%
\end{figure*}
}

%% full page figures, at the top of a page
\newenvironment{BFIGURET*}{
\begin{figure*}[t!]
\BFULLFIGURELINE
}{%
\par\vskip-\parskip
\BFULLFIGURELINE%
\end{figure*}
}

%% another way to do horizontal lines
\newcommand{\UNSPACEFORBOX}{\vspace{-\BFIGURESKIP}}
\newcommand{\HLINE}{\UNSPACEFORBOX\BFIGURELINE\UNSPACEFORBOX}

%% end of boxed-figure-new.tex
