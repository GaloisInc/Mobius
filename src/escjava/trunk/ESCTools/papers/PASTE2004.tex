% This used sigproc-sp.tex -FILE FOR V2.6SP OF ACM_PROC_ARTICLE-SP.CLS
% OCTOBER 2002
%
% ----------------------------------------------------------------------------------------------------------------
% This .tex file (and associated .cls V2.6SP) *DOES NOT* produce:
%       1) The Permission Statement
%       2) The Conference (location) Info information
%       3) The Copyright Line with ACM data
%       4) Page numbering
%
%  However, both the CopyrightYear (default to 2002) and the ACM Copyright Data
% (default to X-XXXXX-XX-X/XX/XX) can still be over-ridden by whatever the author
% inserts into the source .tex file.
% e.g.
% \CopyrightYear{2003} will cause 2003 to appear in the copyright line.
% \crdata{0-12345-67-8/90/12} will cause 0-12345-67-8/90/12 to appear in the copyright line.
%
% ---------------------------------------------------------------------------------------------------------------
% It is an example which *does* use the .bib file (from which the .bbl file
% is produced).
% REMEMBER HOWEVER: After having produced the .bbl file,
% and prior to final submission,
% you need to 'insert'  your .bbl file into your source .tex file so as to provide
% ONE 'self-contained' source file.
%
% Questions regarding SIGS should be sent to
% Adrienne Griscti ---> griscti@acm.org
%
% Questions/suggestions regarding the guidelines, .tex and .cls files, etc. to
% Gerald Murray ---> murray@acm.org
%
% For tracking purposes - this is V2.6SP - OCTOBER 2002

\documentclass{acm_proc_article-sp}

\newif\ifpdf
\ifx\pdfoutput\undefined
   \pdffalse     % no PDFLaTeX
\else
   \pdfoutput=1  % PDFLaTeX
   \pdftrue
\fi

\ifpdf
\usepackage[pdftex,bookmarks=false,
            plainpages=false,naturalnames=true,
            colorlinks=true,pdfstartview=FitV,
            linkcolor=blue,citecolor=blue,urlcolor=blue]{hyperref}
\else
\usepackage[dvips]{hyperref}
\fi

\begin{document}
%
% --- Author Metadata here ---
\conferenceinfo{PASTE}{2004 Washington, D.C. USA}
%\setpagenumber{50}
\CopyrightYear{2004} % Allows default copyright year (2002) to be over-ridden - IF NEED BE.
\crdata{TODO}  % Allows default copyright data (X-XXXXX-XX-X/XX/XX) to be over-ridden.
% --- End of Author Metadata ---

\title{Uniting ESC/Java and JML}
\subtitle{Progress in building and using ESC/Java2}
%
% You need the command \numberofauthors to handle the "boxing"
% and alignment of the authors under the title, and to add
% a section for authors number 4 through n.
%
% Up to the first three authors are aligned under the title;
% use the \alignauthor commands below to handle those names
% and affiliations. Add names, affiliations, addresses for
% additional authors as the argument to \additionalauthors;
% these will be set for you without further effort on your
% part as the last section in the body of your article BEFORE
% References or any Appendices.

\numberofauthors{2}
%
% You can go ahead and credit authors number 4+ here;
% their names will appear in a section called
% "Additional Authors" just before the Appendices
% (if there are any) or Bibliography (if there
% aren't)

% Put no more than the first THREE authors in the \author command
\author{
%
% The command \alignauthor (no curly braces needed) should
% precede each author name, affiliation/snail-mail address and
% e-mail address. Additionally, tag each line of
% affiliation/address with \affaddr, and tag the
%% e-mail address with \email.
\alignauthor David R. Cok\\
       \affaddr{457 Hillside Avenue}\\
       \affaddr{Rochester, NY}\\
       \affaddr{USA}\\
       \email{cok@frontiernet.net}
\alignauthor Joseph Kiniry\\
       \affaddr{Security of Systems Group}\\
       \affaddr{Computing Science Department}\\
       \affaddr{University of Nijmegen}\\
       \affaddr{Toernooiveld 1}\\
       \affaddr{6525 ED Nijmegen}\\
       \affaddr{The Netherlands}\\
       \email{kiniry@cs.kun.nl}
%\alignauthor Lars Th{\Large{\sf{\o}}}rv{$\ddot{\mbox{a}}$}ld\titlenote{This author is the
%one who did all the really hard work.}\\
%       \affaddr{The Th{\large{\sf{\o}}}rv{$\ddot{\mbox{a}}$}ld Group}\\
%       \affaddr{1 Th{\large{\sf{\o}}}rv{$\ddot{\mbox{a}}$}ld Circle}\\
%       \affaddr{Hekla, Iceland}\\
%      \email{larst@affiliation.org}
}
%\additionalauthors{}
\date{4 February 2004}

\newcommand{\myhref}[2]{\ifpdf\href{#1}{#2}\else\htmladdnormallinkfoot{#2}{#1}\fi}

\maketitle
\begin{abstract}
  The ESC/Java tool was a lauded advance in effective static checking
  of realistic Java programs, but has become out-of-date with respect
  to Java and the Java Modeling Language (JML).  The ESC/Java2
  project, whose progress is described in this paper, builds on the
  final release of ESC/Java from DEC/SRC in several ways.  It parses
  all of JML, thus can be used with the growing body of JML-annotated
  Java code; it has improved static checking capabilities, thus is
  becoming more complete and less unsound with each new release; and
  it has been designed, constructed, and documented in such a way as
  to improve the tool's availability to both users and researchers.
  It is intended that ESC/Java2 be used for further research in
  annotation and verification, for larger-scale case studies of
  annotation and verification, and for studies in programmer
  productivity that may result from integration with other tools that
  work with JML and Java.
\end{abstract}

\category{D.2.4}{Software Engineering}
                {Software/Program Verification}
                [Formal methods, Programming by Contract]
\category{F.3.1}{Logics and Meanings of Programs}
                {Specifying and Verifying and Reasoning about Programs}
                [assertions, invariants, logics of programs,
                pre- and post-conditions, specification techniques]

%% TODO \terms{Delphi theory}

\keywords{JML, ESC/Java, ESC/Java2, static analysis, verification, annotation languages} 

%% I think that we currently have far too many bulleted lists.  They do 
%% not read well, nor are they space efficient.  I would hope we could
%% eliminate nearly all of them. -JRK

\section{Introduction}

The ESC/Java tool developed at DEC/SRC was a pioneering tool in the
application of static program analysis and verification technology to
annotated Java programs~\cite{ESCJava}.  The tool and its built-in
prover operated automatically with reasonable performance, needing
only program annotations against which to check the program's source
code.  The annotations needed were easily read, written and understood
by those familiar with Java and were mostly consistent with the syntax
and semantics of the separate Java Modeling Language (JML)
project~\cite{Leavens-etal00,jmlpapers}.  Consequently, the original
ESC/Java (hereafter called DEC/SRC ESC/Java) was a research success
and was also successfully used by other groups for a variety of case
studies (e.g.,~\cite{Hub03,HOP04}).

Its long-term utility, however, was lessened by a number of factors.
First, the match to JML was not complete and JML continued to evolve
as research on the needs of annotations for program checking advanced.
Second, some of the deficiencies of the annotation language used by
DEC/SRC ESC/Java, while appropriate in a research mode, reduced the
overall usability of the tool.  For example, frame conditions were not
checked, but errors in frame conditions could cause the prover to
reach incorrect conclusions.  Also, the annotation language lacked the
ability to use methods in annotations, limiting the annotations to
statements only about low-level representations.  Finally, as
companies were bought and sold and research groups disbanded, there
was no continuing development of the original tool, making it
difficult to use as the Java language itself evolved and limiting the
amount of continuing support.

However, the initial positive experience prompted a vision for a
reasonably industrial-strength tool that would also be useful for
ongoing research in annotation and verification.  Thus, when the
source code for the DEC/SRC ESC/Java tool was made available for
further research and experimentation, the authors of this paper began
the ESC/Java2 project to evolve the DEC/SRC ESC/Java tool.  That
effort has the following goals:
(1) to make the source consistent with the current version of Java;
(2) to package the tool in a way that enables easy application in a
variety of environments, consistent with the licensing provisions of
the source code release;
(3) to fully parse the current version of JML;
(4) to check as much of the JML annotation language as feasible,
consistent with the original engineering goals of DEC/SRC ESC/Java
(usability at the expense of full completeness and soundness); and
(5) as a long-term goal and if appropriate, to update the related
tools that use the same code base (Calvin, Houdini, RCC) and to
integrate with other JML-based tools.  Success in the above will
enable us to test the utility of the tool to improve programmer
productivity on a broader range of significant bodies of Java source,
and to use the tool as a basis for research in unexplored aspects of
annotation and static program analysis.
  
We currently have an alpha version of ESC/Java2
\myhref{http://www.niii.kun.nl/ita/sos/projects/escframe.html}{available
  on the web} and encourage experimentation and feedback.  The source
code itself is available (and additional contributors are welcome)
subject to fairly open licensing provisions.  We fully acknowledge
that the on-going work described here builds on the very substantial
and innovative work to produce ESC/Java and the Simplify prover in the
first place and on the effort to define the Java Modeling Language.

\section{Changes to ESC/Java}

The work on ESC/Java2 required embellishments of the original DEC/SRC
ESC/Java in a number of areas.  Here we present the most significant
of these.
\subsection{Java 1.4}
The original work was performed between 1998 and 2000, and Java has
evolved since then\footnote{In fact, Java 1.5 went beta recently.  No
  work has begun on parsing or statically checking Java 1.5 code.
  Interested parties are welcome to contact the authors with regards
  to this work.}.  The principal addition required by Java 1.4 is
support for the Java {\tt assert} statement.  This is complicated
slightly by the presence of a similar \texttt{assert} statement in
JML.

\subsection{Current JML}
The Java Modeling Language is a research project in itself; hence the
JML syntax and semantics are evolving and are somewhat of a moving
target (and there is as yet no complete reference manual).  However,
the core language is reasonably stable.  The following are key
additions that have been implemented:
\setlength{\partopsep}{0in}\setlength{\parskip}{0in}\setlength{\itemsep}{0in}\setlength{\topsep}{0in}
\begin{itemize}
\setlength{\partopsep}{0in}\setlength{\parskip}{0in}\setlength{\itemsep}{0in}\setlength{\topsep}{0in}
\item parsing of all of current JML, even if the constructs are
  ignored with respect to typechecking or verification,
\item support for refinement files,
\item heavyweight annotations,
\item inheritance of annotations, particularly of \texttt{non\_null}
  modifiers, and
\item model fields, routines, and types.
\end{itemize}
In addition, the differences between DEC/SRC ESC/Java and JML noted in the JML
Reference Manual have been resolved.
% I don't think that there was ever a paper published summarizing 
% these differences. -JRK

\subsection{New verification checks}
Though all of JML is parsed, not all of it is currently checked.
Static checking of the following features has been added to that performed by
DEC/SRC ESC/Java: 
\setlength{\partopsep}{0in}\setlength{\parskip}{0in}\setlength{\itemsep}{0in}\setlength{\topsep}{0in}
\begin{itemize}
\setlength{\partopsep}{0in}\setlength{\parskip}{0in}\setlength{\itemsep}{0in}\setlength{\topsep}{0in}
\item \texttt{constraint} and \texttt{initially} clauses,
\item frame conditions, including the \texttt{in} and \texttt{maps} clauses,
\item model fields and represents clauses, and
\item annotations that include (pure) routine calls and dynamic allocations.
\end{itemize}

\subsection{Backwards incompatibilities}
The DEC/SRC ESC/Java specification language and JML arose separately.  There
was some initial but uncompleted work to unify the two.  The ESC/Java2
project intends to have the tool reflect JML as precisely as possible.
In some cases, discussion about differences resulted in changes to JML.  In
a few cases, some backwards incompatibilities in DEC/SRC ESC/Java were
introduced.  The principal such incompatibilities are these:
\setlength{\partopsep}{0in}\setlength{\parskip}{0in}\setlength{\itemsep}{0in}\setlength{\topsep}{0in}
\begin{itemize}
\setlength{\partopsep}{0in}\setlength{\parskip}{0in}\setlength{\itemsep}{0in}\setlength{\topsep}{0in}
\item The semantics of inheritance of specification clauses and of
  \texttt{non\_null} modifiers was modified to match that defined by
  JML, since the work on JML resulted in an interpretation consistent
  with behavioral subtyping.  This also changed the usage and
  semantics of \texttt{also}.
\item \texttt{modifies \char'134 everything;} is now the default
  modifies clause.
\item The syntax and semantics of \texttt{initially}, {\tt
    monitored\_by}, and \texttt{readable\_if} have changed.
\item ESC/Java2 forbids bodies of (non-model) routines to be present
  in non-Java specification files.
\end{itemize}

\section{Unimplemented Aspects of JML}

Though the core is well-supported, there are several features of JML
which are parsed and ignored, some of them experimental or not yet
endowed with a clear semantics.  The features that are currently ignored are:
\setlength{\partopsep}{0in}\setlength{\parskip}{0in}\setlength{\itemsep}{0in}\setlength{\topsep}{0in}
\begin{itemize}
\setlength{\partopsep}{0in}\setlength{\parskip}{0in}\setlength{\itemsep}{0in}\setlength{\topsep}{0in}
\item checking of access modifiers,
\item the \texttt{weakly}, \texttt{strictfp}, \texttt{volatile}, and
  \texttt{transient} modifiers,
\item the \texttt{diverges}, \texttt{hence\_by},
  \texttt{code\_contract}, \texttt{when}, and \texttt{measured\_by}
  clauses,
\item annotations within \texttt{implies\_that} and {\tt
    for\_example} sections,
\item some of the semantics associated with initialization (prior to
  construction),
\item multi-threading support beyond that in DEC/SRC ESC/\-Java,
\item serialization,
\item annotations regarding space and time consumption,
\item full support of recursive \texttt{maps} declarations,
\item model programs, 
\item verification of anonymous and block-level classes,
\item verification of set comprehension and some forms of quantified
  expressions, and
\item implementation of \texttt{modifies \char'134 everything;} within
  the body of routines.
\end{itemize}


\section{Unresolved semantic issues}

The work on ESC/Java2 has been useful in exposing and resolving
semantic issues in JML.  Since ESC/Java2 is built on a different
source code base than other JML tools, differences of interpretation
in both syntax and semantics arise on occasion.  These are generally
resolved and documented via mailing list discussions by interested
parties.  There are, however, still unresolved issues, some of which
are the subject of ongoing research:
\setlength{\partopsep}{0in}\setlength{\parskip}{0in}\setlength{\itemsep}{0in}\setlength{\topsep}{0in}
\begin{itemize}
\setlength{\partopsep}{0in}\setlength{\parskip}{0in}\setlength{\itemsep}{0in}\setlength{\topsep}{0in}
\item interpretation of the combination of purity and benevolent
  side-effects that is suitable for both static and run-time checking;
\item interpretation of exceptions in pure expressions;
\item the use of helper annotations on routines;
\item the interpretation of annotations for initialization code and
  constructors;
\item use of \texttt{measured\_by} clauses to assist in checking
  termination;
\item meaning of \texttt{code\_contract} clauses;
\item axioms and reasoning procedures to handle safe and unbounded
  arithmetic~\cite{Chalin03}.
\end{itemize}

\section{Usage experience to date}
The SoS group at the University of Nijmegen, along with other members
of the \myhref{http://www.verificard.org/}{European VerifiCard
  Project}, has used DEC/SRC ESC/Java for several projects.  For
example, Hubbers, Oostdijk, and Poll have performed verifications of
Smart Card applets using several tools, including DEC/SRC
ESC/Java~\cite{HOP04}.  Hubbers has also taken initial steps
integrating several JML-based tools~\cite{Hub03}.

These and other VerifiCard projects relied upon the specifications of
the Java Card 2.1 API written and verified by Erik Poll and Hans
Meijer~\cite{MeijerPoll01}.  This specification originally came in two
forms: one ``complete'' specification that used JML models,
heavyweight specifications, and refinements that was used with the
core JML tools, and another ``lightweight'' specification that was
meant to be used with DEC/SRC ESC/Java and other verification tools
like Jack, Krakatoa, and the Loop tool~\cite{a,b,c}.



--general opinion of work--
--transition of pov--

[I will add a one paragraph discussion here of our experiences in
the SoS group with using ESC/Java for verification of core JDK classes
and Smart Card applets. -JRK]

\section{Ongoing work}
The work on ESC/Java2 to date has resulted in a number of software
alpha releases and the work is continuing on a number of fronts.

\paragraph*{Language Issues.} Two obvious and related ongoing tasks are
the completion of additional features of JML, accompanied in some
cases by additional research to clarify the semantics and usability of
outstanding features of JML.  Usage of JML is now broad enough that
accompanying formal reference documentation would be valuable.  As
tools such as ESC/Java2 become more widely used, users will also
appreciate attention to performance, to the clarity of errors and
warnings, and to the overall user experience such tools provide.

\paragraph*{Case Studies.} The current implementation supports the
static checking of a stable core of JML.  With this initial
implementation of frame condition checking and of model fields,
represents clauses and use of routine calls in annotations, ESC/Java2
can now be used on complex and abstract specifications of larger
bodies of software.  Consequently, there is a considerable need for
good experimental usage studies that confirm that this core of JML is
useful in annotations and that the operation of ESC/Java2 (and
Simplify) on that core is correct and valuable.

\paragraph*{Verification Logic.} The logic into which Java and JML are
embedded in ESC/Java is, by design and admission of the original
authors, neither complete or fully sound.  This was the result of an
engineering judgement in favor of performance and usability.  Research
that studies expanded and larger use cases may show whether this
design decision is generally useful in practical static checking or
whether a fuller and more complicated state-based logic is required
for useful results to be obtained.

\paragraph*{User Feedback.} The purpose of using theorem provers for
static analysis, runtime assertion checking, or model checking is to
find errors and thereby improve the correctness of the resulting
software.  Thus the orientation of a tool must be towards effectively
finding {\em and interpreting} examples of incorrect behavior.  A
common complaint in using any of these tools is that it is difficult
to determine a root cause from the counterexample information provided
by the tool whether it is a failed proof or an invalid test or
execution history.  The ESC/Java project reported some work towards
appropriately pruning and interpreting counterexample information, but
there remains room for improvement.

\paragraph*{Tool Integration.} Finally, though not part of this
project, there is another set of beneficial integration work to be
done.

There are now a number of tools, generated by different research
groups, that interact with JML~\cite{Burdy-etal03}.  Among them, the
JML compiler jmlc~\cite{Cheon-Leavens02b} dynamically checks
specifications; the ESC/Java2 tool statically checks specifications;
Daikon generates specifications from execution histories (and was
integrated with the DEC/SRC version of ESC/Java~\cite{NimmerErnst01});
the JML Unit Test tool jmlunit~\cite{Cheon-Leavens02} generates test
cases from specifications.
 
A productive programmer's working environment for a large-scale
project that uses these tools would need the tools to be integrated in
a way that their outputs and inputs seamlessly inter-communicate.  A
programmer using the tools would naturally move among the various
tasks of designing, possibly using abstract specifications, writing,
testing, annotating, verifying and debugging.  Thus the tools would
need to be integrated in a way that allows efficient, incremental
behavior.

For example, once specifications generated by a Daikon-like tool are
generalized and incorporated into a method's annotations, repeated
running of Daikon will only produce results already implied by the
existing specifications.  That may be cause to ignore those subsumed
results until changes elsewhere in the program or new test data show
that the annotations are incorrect.  On the other hand, if the
specifications generated by analyzing the execution history are
implied by but narrower than the written specifications, one might
conclude that either the test data is too limited or the written
specifications (at least in this usage) are overly broad.

%% I am personally not a fan of conclusions that do not say anything 
%% additional (i.e., real implications of the paper).  Also, for a 
%% paper this short, a conclusion might just be wasted space.  If we 
%% can find something to say here that is, in fact, implicatory, they 
%% I'd be fore it. -JRK

\section{Conclusion}

\large\textbf{TODO:}\normalsize Perhaps discuss the fact that a
verifying compiler might be closer than some notable computer
scientists think~\cite{Hoare03}?  Discuss how far we think that we can
take this application of basic logics, first-order provers, and
related technologies.  Also reinforce the changing balance between
automated and interactive proving, discussed in the usage experience
section above. -JRK

%ACKNOWLEDGMENTS are optional
\section{Acknowledgments}
The authors would like to acknowledge both the work of the team that
generated ESC/Java as well as Gary Leavens and collaborators at Iowa
State University who generated JML.  In addition, Leavens and students
engaged in and helped resolve syntactic and semantic issues in JML
raised by the work on ESC/Java2.  These teams provided the twin
foundations on which the current work is built.  In addition, the many
other research groups that use and critique both JML and ESC/Java2
have provided a research environment in which the work described here
is useful.  Joseph Kiniry is supported by the NWO Pionier research
project on Program Security and Correctness.

%
% The following two commands are all you need in the
% initial runs of your .tex file to
% produce the bibliography for the citations in your paper.
\bibliographystyle{abbrv}
\bibliography{PASTE2004}  
% You must have a proper ".bib" file
%  and remember to run:
% latex bibtex latex latex
% to resolve all references
%
% ACM needs 'a single self-contained file'!
%
%APPENDICES are optional
\balancecolumns

% That's all folks!
\end{document}
