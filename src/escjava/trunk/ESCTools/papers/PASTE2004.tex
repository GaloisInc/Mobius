% This used sigproc-sp.tex -FILE FOR V2.6SP OF ACM_PROC_ARTICLE-SP.CLS
% OCTOBER 2002
%
% ----------------------------------------------------------------------------------------------------------------
% This .tex file (and associated .cls V2.6SP) *DOES NOT* produce:
%       1) The Permission Statement
%       2) The Conference (location) Info information
%       3) The Copyright Line with ACM data
%       4) Page numbering
%
%  However, both the CopyrightYear (default to 2002) and the ACM Copyright Data
% (default to X-XXXXX-XX-X/XX/XX) can still be over-ridden by whatever the author
% inserts into the source .tex file.
% e.g.
% \CopyrightYear{2003} will cause 2003 to appear in the copyright line.
% \crdata{0-12345-67-8/90/12} will cause 0-12345-67-8/90/12 to appear in the copyright line.
%
% ---------------------------------------------------------------------------------------------------------------
% It is an example which *does* use the .bib file (from which the .bbl file
% is produced).
% REMEMBER HOWEVER: After having produced the .bbl file,
% and prior to final submission,
% you need to 'insert'  your .bbl file into your source .tex file so as to provide
% ONE 'self-contained' source file.
%
% Questions regarding SIGS should be sent to
% Adrienne Griscti ---> griscti@acm.org
%
% Questions/suggestions regarding the guidelines, .tex and .cls files, etc. to
% Gerald Murray ---> murray@acm.org
%
% For tracking purposes - this is V2.6SP - OCTOBER 2002

\documentclass{acm_proc_article-sp}

\begin{document}
%
% --- Author Metadata here ---
\conferenceinfo{PASTE}{2004 Washington, D.C. USA}
%\setpagenumber{50}
\CopyrightYear{2004} % Allows default copyright year (2002) to be over-ridden - IF NEED BE.
\crdata{TODO}  % Allows default copyright data (X-XXXXX-XX-X/XX/XX) to be over-ridden.
% --- End of Author Metadata ---

\title{Uniting ESC/Java and JML ???TODO}
\subtitle{???TODO}
%
% You need the command \numberofauthors to handle the "boxing"
% and alignment of the authors under the title, and to add
% a section for authors number 4 through n.
%
% Up to the first three authors are aligned under the title;
% use the \alignauthor commands below to handle those names
% and affiliations. Add names, affiliations, addresses for
% additional authors as the argument to \additionalauthors;
% these will be set for you without further effort on your
% part as the last section in the body of your article BEFORE
% References or any Appendices.

\numberofauthors{2}
%
% You can go ahead and credit authors number 4+ here;
% their names will appear in a section called
% "Additional Authors" just before the Appendices
% (if there are any) or Bibliography (if there
% aren't)

% Put no more than the first THREE authors in the \author command
\author{
%
% The command \alignauthor (no curly braces needed) should
% precede each author name, affiliation/snail-mail address and
% e-mail address. Additionally, tag each line of
% affiliation/address with \affaddr, and tag the
%% e-mail address with \email.
\alignauthor David R. Cok\\
       \affaddr{457 Hillside Avenue}\\
       \affaddr{Rochester, NY}\\
       \affaddr{USA}\\
       \email{cok@frontiernet.net}
\alignauthor Joseph Kiniry\\
       \affaddr{TODO}\\
       \affaddr{TODO}\\
       \affaddr{TODO}\\
       \email{TODO@TODO.com}
%\alignauthor Lars Th{\Large{\sf{\o}}}rv{$\ddot{\mbox{a}}$}ld\titlenote{This author is the
%one who did all the really hard work.}\\
%       \affaddr{The Th{\large{\sf{\o}}}rv{$\ddot{\mbox{a}}$}ld Group}\\
%       \affaddr{1 Th{\large{\sf{\o}}}rv{$\ddot{\mbox{a}}$}ld Circle}\\
%       \affaddr{Hekla, Iceland}\\
%      \email{larst@affiliation.org}
}
%\additionalauthors{}
\date{4 February 2004}
\maketitle
\begin{abstract}
... abstract... ??? TODO
\end{abstract}

\category{D.2.4}{Software Engineering}{Software/Program Verification}[Formal methods, Programming by Contract]

\category{F.3.1}{Logics and Meanings of Programs}{Specifying and Verifying and Reasoning about Programs}[assertions, invariants, logics of programs,
                pre- and post-conditions, specification techniques]

%% TODO \terms{Delphi theory}

\keywords{JML, ESC/Java, TODO} % NOT required for Proceedings

\section{Introduction}
The ESC/Java tool developed at DEC/SRC [TODO-ref] was a pioneering tool
in the application of static program analysis and verification technology
to annotated Java programs.  The tool and its built-in prover operated 
automatically with reasonable performance, needing only program annotations
against which to check the program's source code.  The annotations needed were
easily read, written and understood by those familiar with Java and were mostly
consistent with the syntax and semantics of the separate Java Modeling Language 
(JML) project \cite{Leavens-etal00}\cite{jmlpapers}.  Consequently, ESC/Java was a
research success and was also successfully used by other groups on a trial 
basis.  [TODO do we need a ref here??]

Its long-term utility, however, was lessened by a number of factors.  First, 
the match to JML was not complete and JML continued to evolve as research on
the needs of annotations for program checking advanced.  Second, some of the
deficiencies of the annotation language used by ESC/Java, while appropriate 
in a research mode, reduced the overall usability of the tool.  For example,
frame conditions were not checked, but errors in frame conditions could cause
the prover to reach incorrect conclusions; the annotation language lacked 
the ability to use methods in annotations, limiting the annotations to statements
only about low-level representations.  Finally,
as companies
were bought and sold and research groups disbanded, there was no continuing
development of the original tool, making it difficult to use as the Java language
itself evolved and continuing support minimal.

However, the initial positive experience prompted a vision for a reasonably
industrial-strength tool that would also be useful for ongoing research in annotation
and verification.
Thus, when the source code for the ESC/Java tool was made available for further 
research and experimentation, the authors of this paper began the ESC/Java2 project
to evolve the ESC/Java tool.  That effort has the following goals:
\begin{enumerate}
\item to make the source consistent with the current version of Java;
\item to package the
tool in a way that enabled easy application in a variety of environments, consistent
with the licensing provisions of the source code release;
\item to fully parse the current version of JML;
\item to check as much of the JML annotation language as feasible, consistent with the
original engineering goals of ESC/Java (usability at the expense of full completeness
and soundness);
\item as a long-term goal and if appropriate,
to update the related tools that use the same code base, namely Calvin, Houdini, and RCC.
\end{enumerate}
Success in the above will enable us to
\begin{itemize}
\item test the utility of the tool on a broader range of significant bodies of Java source, and
\item use the tool as a basis for research in unexplored aspects of annotation and static 
program analysis.
\end{itemize}

We currently have alpha versions of ESC/Java2 available on the web [TODO-ref] and 
encourage experimentation and feedback.  The source code itself is available (and 
additional contributors are welcome) subject to fairly open licensing provisions.
We fully acknowledge that the on-going work described here builds on the very
substantial and innovative work to produce ESC/Java and the Simplify prover in the
first place and on the effort to define the Java Modeling Language.

\section{Changes to ESC/Java}

The work on ESC/Java2 required embellishments of the original ESC/Java in a 
number of areas.  Here we present the most significant of these.
\subsection{Java 1.4}
The original work was performed in about [TODO-date] and Java has evolved since then.
The principal addition required by Java 1.4
is support for the Java {\tt assert} statement.  This is 
complicated slightly by the presence of a similar {\tt assert} statement in JML.

\subsection{Current JML}
The Java Modeling Language is a research project in itself; hence the JML syntax
and semantics are evolving and are somewhat of a moving target (and there is as yet
no complete reference manual).  However, the core
language is reasonably stable.  The following are key additions that have been 
implemented.
\setlength{\partopsep}{0in}\setlength{\parskip}{0in}
\begin{itemize}\setlength{\itemsep}{0in}
\item parsing of all of current JML, even if the constructs are ignored
\item support for refinement files
\item heavyweight annotations
\item slight adjustments in annotation format
\item inheritance of annotations, particularly of {\tt non\_null} modifiers
\item model fields, routines, and types
\end{itemize}
In addition, the differences between ESC/Java and JML noted in early work [TODO-ref]
on JML have
been resolved.

\subsection{New verification checks}
Though all of JML is parsed, not all of it is currently checked.  Checking of the
following features has been added to that performed by ESC/Java.
\setlength{\partopsep}{0in}\setlength{\parskip}{0in}
\begin{itemize}\setlength{\itemsep}{0in}
\item {\tt constraint} and {\tt initially} clauses
\item frame conditions, including the {\tt in} and {\tt maps} clauses
\item model fields and represents clauses
\item annotations that include routine calls and dynamic allocations
\end{itemize}

\subsection{Backwards incompatibilities}
The ESC/Java specification language and JML arose separately.  There was some initial 
but uncompleted work to unify the two.
The ESC/Java2 project intends to have the tool
reflect JML as precisely as possible.  In some cases, discussion resulted in changes to
JML, but more often (but as little as possible) some backwards incompatibilities were
introduced.  The principal such incompatibilities are these:
\setlength{\partopsep}{0in}\setlength{\parskip}{0in}
\begin{itemize}\setlength{\itemsep}{0in}
\item The semantics of inheritance of specification clauses and of {\tt non\_null}
modifiers was modified to that of JML, since the work on JML resulted in an interpretation
consistent with behavioral subtyping. [TODO check the words here]  This also changed
the usage of {\tt also}.
\item The default modifies clause is now {\tt modifies \\everything;}.
\item The syntax and semantics of {\tt initially}, {\tt monitored\_by}, and {\tt readable\_if}
have changed.
\item ESC/Java2 requires semicolon termination of annotations; it also forbids bodies
of (non-model) routines to be present in non-java specification files.
\end{itemize}


\section{Aspects of JML not yet implemented}
Though the core is well-supported, there are several features of JML which are parsed and
ignored, most of them experimental or not yet endowed with a clear semantics.  
\setlength{\partopsep}{0in}\setlength{\parskip}{0in}
\begin{itemize}\setlength{\itemsep}{0in}
\item checking of access modifiers
\item the weakly, strictfp, volatile, transient modifiers
\item annotations labelled as redundant are handled just like other annotations, 
or are ignored (e.g. {\tt implies\_that} and {\tt for\_example} sections)
\item any semantics associated with initialization (prior to construction)
\item multithreading issues beyond that supported by ESC/Java
\item serialization
\item annotations regarding space and time consumption
\item full support of recursive {\tt maps} declarations
\item model programs, {\tt diverges} clauses and {\tt hence\_by} clauses
\item verification of anonymous and block-level classes
\item verification of set comprehension and some forms of quantified expressions
\item implementation of {\tt modifies \\everything;} within the body of routines
\end{itemize}


\section{Unresolved semantic issues}
The work on ESC/Java2 has been useful in exposing and resolving semantic issues in JML.
Since the ESC/Java2 is built on a different source code base than other JML tools, 
differences of interpretation in both syntax and semantics arise on occasion.  These are
generally resolved and documented via mailing list discussions by interested parties.
There are, however, still unresolved issues, some of which are the subject of ongoing
research.
\setlength{\partopsep}{0in}\setlength{\parskip}{0in}
\begin{itemize}\setlength{\itemsep}{0in}
\item interpretation of the combination of purity and benevolent side-effects that is
suitable for both static and run-time checking
\item interpretation of exceptions in pure expressions
\item the use of helper annotations on routines
\item the interpretation of annotations for initialization code and constructors
\item use of {\tt measured\_by} clauses to assist in checking termination
\item meaning of {\tt code\_contract} clauses
\item axioms and reasoning procedures to handle safe arithmetic [TODO -ref]
\end{itemize}

\section{Usage experience to date}
TODO

\section{Future work}
The work on ESC/Java2 to date has resulted in a number of software alpha releases
and the work is continuing on a number of fronts.
\begin{itemize}
\item As the previous sections show, there is a fair amount of additional software 
implementation to complete.  As is usually the case, users would 
also appreciate attention
to performance and to the clarity of errors and warnings.
\item Research is required to clarify the semantics and usability of
outstanding features of JML.  Usage of JML is now broad enough that formal 
reference documentation would also be valuable.
\item The current implementation supports the static checking of
a stable core of JML.  With this initial implementation of frame condition checking
and of model fields, represents clauses and use of routine calls in 
annotations, ESC/Java2 can now be used on complex and abstract specifications of
larger bodies of software.
Consequently, there is a considerable need for good experimental usage studies
that confirm
that this core of JML is useful in annotations
and that the operation of ESC/Java2 (and Simplify) on that core is correct and valuable.
\item The logic into which Java and JML are embedded in ESC/Java is, by design and
admission of the original authors, neither complete or fully sound.  This was the 
result of an
engineering judgement in favor of performance and usablility.  Research that studies
expanded and larger use cases may show whether this design decision 
is generally useful in
practical static checking or whether a fuller and more complicated state-based logic
is required for useful results to be obtained.
\end{itemize}

Though not part of this project, there is another set of beneficial work to be done.
There are now a number of tools \cite{Burdy-etal03}, generated by different research groups, 
that interact
with JML.  Among other tools, the jmlc\cite{Cheon-Leavens02b}
tool dynamically checks specifications; the ESC/Java2 tool statically
checks specifications; Daikon [TODO-ref] generates specifications from execution
histories; jmlunit \cite{Cheon-Leavens02} generates test cases from specifications.
To provide a productive working environment for large-scale projects that use these
tools but are not focused on their improvement or analysis, the tools need to be
better integrated.  To a fair extent that integration is present, but there are
additional opportunities.  For example, a close integration of the annotations
produced by Daikon with those used by other tools would be a worthy endeavor.   

\section{Conclusion}
TODO ???

%ACKNOWLEDGMENTS are optional
\section{Acknowledgments}
The authors would like to acknowledge both the work of the team that generated
ESC/Java as well as Gary Leavens and collaborators at Iowa State University who
generated JML.  These teams provided the twin foundations on which the current
 work is built.
In addition, the many other research groups that use and critique both 
JML and ESC/Java2 have
provided a research environment in which the work described here is useful.
Joseph Kiniry is supported by ... TODO ...

%
% The following two commands are all you need in the
% initial runs of your .tex file to
% produce the bibliography for the citations in your paper.
\bibliographystyle{abbrv}
\bibliography{PASTE2004}  
% You must have a proper ".bib" file
%  and remember to run:
% latex bibtex latex latex
% to resolve all references
%
% ACM needs 'a single self-contained file'!
%
%APPENDICES are optional
\balancecolumns

% That's all folks!
\end{document}
