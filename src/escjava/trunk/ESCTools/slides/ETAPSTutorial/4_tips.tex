%% ESC/Java2 tutorial
% David Cok% $Id$%
\documentclass[
pdf,
%ps,
%a4,
nocolorBG,
%colorBG,
slideColor,
%accumulate,
%draft,
cok,
%frames,
]{prosper}
%%%%%%%%%%%%%%%%%%%%%%%%%%%%%%%%%%%%%%%%%%%%%%%%%%%%%%%%%%%%%%%%%%%%%%%%%%%
\usepackage{alltt}
\usepackage{pstricks,pst-node,pst-text,pst-3d}
\usepackage{textcomp}
\usepackage{verbatim}
% \usepackage{colordvi}
\newcommand{\Red}[1]{{\red #1}}
\usepackage[all]{xy}

%%%%%%%%%%%%%%%%%%%%%%%%%%%%%%%%%%%%%%%%%%%%%%%%%%%%%%%%%%%%%%%%%%%%%%%%%%%%
\newrgbcolor{Yellowish}{0.90 0.85 0.650}%
\newrgbcolor{red}{1 0 0}
\newrgbcolor{purple}{0.4 0 0.7}%
\newrgbcolor{lightpurple}{0.63 0.13 0.94}%
\newrgbcolor{lime}{0.73 1 0}
\newrgbcolor{fullgreen}{0 1 0} 
\newrgbcolor{green}{0.133 0.56 0} % lichter%
\newrgbcolor{green}{0.10 0.43 0}
\newrgbcolor{knalblue}{0 0 1}
\newrgbcolor{blue}{.2 .36 .77}%
\newrgbcolor{darkblue}{0.28 0.24 0.55}
\newrgbcolor{gray}{.5 .5 .5}
%%%%%%%%%%%%%%%%%%%%%%%%%%%%%%%%%%%%%%%%%%%%%%%%%%%%%%%%%%%%%%%%%%%%%%%%%%%
\newcommand{\embf}[1]{\textit{\textbf{#1}}}
\newcommand{\rmbf}[1]{\textrm{\textbf{#1}}}
\newcommand{\ttbf}[1]{\mbox{\tt \textbf{#1}}}
\newcommand{\code}[1]{{\rm \texttt{\textbf{\small #1}}}}
\myitem{1}{\mbox{{$\bullet$}}}


\newcommand{\bsl}{\char'134}
\newcommand{\result}{\bsl result}
\newcommand{\old}{\bsl old}

\newcommand{\verbatimcode}[1]{
\begin{figure*}
\tiny
\verbatiminput{#1}
\end{figure*}
}

\newif\ifignore
\newrgbcolor{Bluish}{0.9 0.9 1.0}
\newcommand{\doos}[1]{
 \psshadowbox[fillstyle=solid,                        
              fillcolor=Bluish,                        
              framearc=0.2,                        
              framesep=2mm]{#1}} 
%%%%%%%%%%%%%%%%%%%%%%%%%%%%%%%%%%%%%%%%%%%%%%%%%%%%%%%%%%%%%%%%%%%%%%%%%%%%%
\title{\embf{\blue        {\huge Specification tips and pitfalls }}      }
\author{\embf{\Large{\red David Cok, Joe Kiniry, and Erik Poll}}
       }
\institution{\embf{Eastman Kodak Company, University College Dublin, \\ and Radboud University Nijmegen}}
\slideCaption{{\blue David Cok, Joe Kiniry \& Erik Poll - ESC/Java2 \& JML Tutorial }}
%\Logo{{\blue Erik Poll}}

\begin{document}\maketitle \boldmath
%%%%%%%%%%%%%%%%%%%%%%%%%%%%%%%%%%%%%%%%%%%%%%%%%%%%%%%%%%%%%%%%%%%%%%%%%%%%%
%%%%%%%%%%%%%%%%%%%%%%%%%%%%%%%%%%%%%%%%%%%%%%%%%%%%%%%%%%%%%%%%%%%%%%%%%%%%%
%%%%%%%%%%%%%%%%%%%%%%%%%%%%%%%%%%%%%%%%%%%%%%%%%%%%%%%%%%%%%%%%%%%%%%%%%%%%%
%%%%%%%%%%%%%%%%%%%%%%%%%%%%%%%%%%%%%%%%%%%%%%%%%%%%%%%%%%%%%%%%%%%%%%%%%%%%%
%%%%%%%%%%%%%%%%%%%%%%%%%%%%%%%%%%%%%%%%%%%%%%%%%%%%%%%%%%%%%%%%%%%%%%%%%%%%%


%%%%%%%%%%%%%%%%%%%%%%%%%%%%%%%%%%%%%%%%%%%%%%%%%%%%%%%%%%%%%%%%%%%%%%%%%%%%%

\begin{slide}{Specifications tips and pitfalls}
\begin{enumerate}
\item Inherited specifications 
\item Aliasing
\item Object invariants
\item Inconsistent assumptions
\item Exposed references
\item \old
\item How to write specs
\end{enumerate}

\end{slide}


%%%%%%%%%%%%%%%%%%%%%%%%%%%%%%%%%%%%%%%%%%%%%%%%%%%%%%%%%%%%%%%%%%%%%%%%%%%%
\part{{\Large \red \#1: Specification inheritance and behavioural subtyping }}
%%%%%%%%%%%%%%%%%%%%%%%%%%%%%%%%%%%%%%%%%%%%%%%%%%%%%%%%%%%%%%%%%%%%%%%%%%%%




%%%%%%%%%%%%%%%%%%%%%%%%%%%%%%%%%%%%%%%%%%%%%%%%%%%%%%%%%%%%%%%%%%%%%%%%%%%%
\begin{slide}{Behavioural subtyping}
\vspace*{-3ex}

Suppose \code{Child extends Parent}.

\begin{itemize}
\item $\!$ {\green Behavioural subtyping} =
objects from subclass \code{Child} ``behave like''
objects from superclass \code{Parent}
\item $\!$ {\green Principle of substitutivity} [Liskov]:\\
code will behave ``as expected'' if we provide an 
\code{Child} object where a \code{Parent} object was expected.
\end{itemize}

\end{slide}

%%%%%%%%%%%%%%%%%%%%%%%%%%%%%%%%%%%%%%%%%%%%%%%%%%%%%%%%%%%%%%%%%%%%%%%%%%%%
\begin{slide}{Behavioural subtyping}
\vspace*{-3ex}

Behavioural subtyping usually enforced by insisting that

\begin{itemize}
\item 
invariant in subclass is {\green stronger} than invariant in superclass
\item
for every method,
\begin{itemize}
\item 
precondition in subclass is {\blue weaker} (!) than precondition is superclass
\item
postcondition in subclass is {\green stronger} than postcondition is superclass
\end{itemize}
\end{itemize}

JML achieves behavioural subtyping by {\green specification inheritance}:
any child class {\green inherits} the specification of its parent.

\end{slide}


%%%%%%%%%%%%%%%%%%%%%%%%%%%%%%%%%%%%%%%%%%%%%%
\begin{slide}{$\!\!\!$\small Specification inheritance for invariants}
\vspace*{-3ex}

Invariants are inherited in subclasses. Eg.
\begin{alltt}
\code{\scriptsize class Parent \{
     ...
     //@ invariant invParent;
     ...  \}

  class Child extends Parent \{
     ...
     //@ invariant invChild;
     ...  \}}
\end{alltt}

the invariant for \code{Child} is \code{invChild \&\& invParent}

\end{slide}

%%%%%%%%%%%%%%%%%%%%%%%%%%%%%%%%%%%%%%%%%%%%%%
\begin{slide}{$\!\!\!\!\!$\small Specif\-ication inheritance for method specs}
\vspace*{-4ex}
\begin{alltt}
\code{{\scriptsize class Parent \{
     //@ requires i >= 0;
     //@ ensures  \result >= i;
     int m(int i)\{ ... \}
  \}

  class Child extends Parent \{
     //@{\green also}
     //@   requires i <= 0
     //@   ensures  \result <= i;
     int m(int i)\{ ... \}
   \}}}
\end{alltt}

{\scriptsize Keyword {\green \texttt{also}} indicates there are inherited specs.}

\end{slide}


%%%%%%%%%%%%%%%%%%%%%%%%%%%%%%%%%%%%%%%%%%%%%%
\begin{slide}{$\!\!\!\!\!$\small Specif\-ication inheritance for method specs}
\vspace*{-4ex}

Method \code{m} in \code{Child} also has to meet the spec
given in \code{Parent} class.
So the complete spec for \code{Child} is

\begin{alltt}
\code{\scriptsize class Child extends Parent \{

  /*@   requires i >= 0;
    @   ensures  \result >= i;
    @{\green also}
    @   requires i <= 0
    @   ensures  \result <= i;
    @*/
  int m(int i)\{ ... \}
 \}}
\end{alltt}

{\tiny What can result of \code{m(0)} be?}

\end{slide}

%%%%%%%%%%%%%%%%%%%%%%%%%%%%%%%%%%%%%%%%%%%%%%
\begin{slide}{$\!\!\!\!\!$\small Specification inheritance for method specs}
\vspace*{-3ex}

This spec for \code{Child} is equivalent with

\begin{alltt}
\code{\scriptsize class Child extends Parent \{

  /*@   requires i <= 0 || i >= 0;
    @   ensures  \old(i) >= 0 ==> \result >= i;
    @   ensures  \old(i) <= 0 ==> \result <= i;
    @*/
  int m(int i)\{ ... \}
 \}}
\end{alltt}

\end{slide}

\ifignore
%%%%%%%%%%%%%%%%%%%%%%%%%%%%%%%%%%%%%%%%%%%%%%%%%%%%%%%%%%%%%%%%%%%%%%%%%%%%%

\begin{slide}{Inherited specifications}
\vspace*{-7ex}
For example, in the code below
\begin{itemize}
\item Parent.m(i) satisfies $i \geq 0 \Rightarrow \result \geq i$
\item Child.m(i) satisfies \hspace{.5ex} $i \geq 0 \Rightarrow \result \geq i$ 
\\\textit{and} \hspace{13.5ex}  $i \leq 0 \Rightarrow \result \leq i$
\\so Child.m(0) must be 0.
\item[]
\end{itemize} 
{\tiny \begin{verbatim}class Parent {
    //@ requires i >= 0;     
    //@ ensures  \result >= i;     
   int m(int i);  
}  
class Child extends Parent {     
    //@ also     
    //@   requires i <= 0     
    //@   ensures  \result <= i;     
int m(int i); 
}
\end{verbatim}
}
\vspace*{-20ex}
\begin{picture}(100,100)(0,0)
\thicklines
\red
\put(51,35){\oval(25,8)}
\put(102,-10){Indicates there are inherited specs}
\put(100,-3){\vector(-1,1){35}}
\end{picture}

\vspace*{-20ex}
\begin{picture}(100,100)(0,0)
\thicklines
\green
\put(200,30){\shortstack[l]{
Note: In \\
Parent p = new Parent(); \\
Parent pc = new Child(); \\
Child c = new Child();\\
p.m(i); // i must be >= 0\\
pc.m(i); // i must be >= 0\\
c.m(i); // i must be >= 0 or <= 0
}}
\end{picture}

\end{slide}
\fi

%%%%%%%%%%%%%%%%%%%%%%%%%%%%%%%%%%%%%%%%%%%%%%%%%%%%%%%%%%%%%%%%%%%%%%%%%%%%%

\begin{slide}{Inherited specifications: trick}
\vspace*{-4ex}
Another example: two Objects that are == are always also {\knalblue equals}.  But the converse is not
necessarily true.  But it is true for objects whose dynamic type is Object.

\vspace*{3ex}

\begin{verbatim}
public class Object {
  //@ ensures (this == o) ==> \result;
  /*@ ensures \typeof(this) == \type(Object) 
              ==> (\result == (this==o));
  */
  public boolean equals(Object o);
}
\end{verbatim}
\vspace*{-10ex}
\begin{picture}(100,100)(0,0)
\thicklines
\red
\put(0,0){True for all Objects}
\put(58,10){\vector(1,3){32}}

\put(150,10){Not necessarily true for subtypes}
\put(200,20){\vector(-1,3){20}}
\end{picture}

\end{slide}

%%%%%%%%%%%%%%%%%%%%%%%%%%%%%%%%%%%%%%%%%%%%%%%%%%%%%%%%%%%%%%%%%%%%%%%%%%%%%

\begin{slide}{Inherited specifications}
\vspace*{-4ex}

So
\begin{itemize}
\item Base class specifications apply to subclasses
\begin{itemize}
\item that is, ESC/Java2 enforces \textit{behavioral subtyping}
\item Specs from implemented \textit{interfaces} also must hold for implementing classes
\end{itemize}
\item Be thoughtful about how strict the base class specs should be
\item Guard them with {\blue \char'134 typeof(this) == \char'134 type(...) } if need be
\item Restrictions on exceptions such as {\blue normal\_behavior} or {\blue signals (E e) false; }
will apply to derived classes as well.
\end{itemize}
\end{slide}


%%%%%%%%%%%%%%%%%%%%%%%%%%%%%%%%%%%%%%%%%%%%%%%%%%%%%%%%%%%%%%%%%%%%%%%%%%%%%

\ifignore

% treated - in more detail - in 5_jml_more

\begin{slide}{\#2: Specifying exceptions}
\vspace*{-8ex}
\begin{itemize}
\item A {\blue signals} clause such as \\
{\blue signals (FileNotFoundException e) true;}\\
states what must be true if an exception of the stated type is thrown.
\item It does not say what other exception types may or may not be thrown.
\item To forbid a particular exception, omit it from the Java throws clause or write \\
{\blue signals (FileNotFoundException e) false;}\\
\item To limit the set of allowed exceptions, use a postcondition such as 
\begin{verbatim}
/*@ signals (Exception e) e instanceof E1 
                       || e instanceof E2 
                       || ... ;
 */
\end{verbatim}
\item To forbid all exceptions use a {\knalblue normal\_behavior} or 
{\blue signals (Exception e) false;} specification - be careful not to overly restrict derived classes
\end{itemize}

\end{slide}

\fi
%%%%%%%%%%%%%%%%%%%%%%%%%%%%%%%%%%%%%%%%%%%%%%%%%%%%%%%%%%%%%%%%%%%%%%%%%%%%%

%%%%%%%%%%%%%%%%%%%%%%%%%%%%%%%%%%%%%%%%%%%%%%%%%%%%%%%%%%%%%%%%%%%%%%%%%%%%
\part{{\Large \red \#2: Aliasing}}
%%%%%%%%%%%%%%%%%%%%%%%%%%%%%%%%%%%%%%%%%%%%%%%%%%%%%%%%%%%%%%%%%%%%%%%%%%%%


\begin{slide}{Aliasing}
\vspace*{-9ex}
A common but non-obvious problem that causes violated invariants is aliasing.
{\tiny
\begin{verbatim}
public class Alias {
  /*@ non_null */ int[] a = new int[10];
  boolean noneg = true;

  /*@ invariant noneg ==> 
                 (\forall int i; 0<=i && i < a.length;  a[i]>=0); */

  //@ requires 0<=i && i < a.length;
  public void insert(int i, int v) {
    a[i] = v;
    if (v < 0) noneg = false;
  }
}
\end{verbatim}  
}
produces
{\tiny
\begin{verbatim}
Alias.java:12: Warning: Possible violation of object invariant (Invariant)
  }
  ^
Associated declaration is "Alias.java", line 5, col 6:
  /*@ invariant (\forall int i; 0<=i && i < a.length; 
\end{verbatim}
}

\end{slide}
\begin{slide}{Aliasing}
\vspace*{-6ex}
A full counterexample context ({\blue -counterexample} option) produces, among lots of other information:
\begin{verbatim}

    brokenObj%0 != this
    (brokenObj%0).(a@pre:2.24) == tmp0!a:10.4
    this.(a@pre:2.24) == tmp0!a:10.4

\end{verbatim}
that is, {\blue this} and some different object ({\blue brokenObj}) share the \\ same  {\blue a} object. 


\begin{picture}(150,150)(0,0)
\red
\put(38,86){\framebox(55,27){}}
\put(113,86){\framebox(55,27){}}
\put(40,90){\shortstack[l]{int noneg \\ int[] a}}
\put(115,90){\shortstack[l]{int noneg \\ int[] a}}
\put(175,60){\framebox(75,7){ }}
\knalblue
\put(38,118){this}
\put(113,118){brokenObj}
\put(175,72){\textit{an} int[] \textit{object}}
\green
\put(75,92){\vector(4,-1){100}}
\put(150,92){\vector(1,-1){25}}
\put(75,92){\circle*{5}}
\put(150,92){\circle*{5}}
\end{picture}
\end{slide}
\begin{slide}{Aliasing}
\vspace*{-6ex}
To fix this, declare that {\blue a} is owned only by its parent object:\\
( {\blue owner} is a ghost field of java.lang.Object )
\tiny
\begin{verbatim}
public class Alias {
  /*@ non_null */ int[] a = new int[10];
  boolean noneg = true;

  /*@ invariant (\forall int i; 0<=i && i < a.length; 
                       noneg ==> (a[i]>=0)); */
  //@ invariant a.owner == this;

  //@ requires 0<=i && i < a.length;
  public void insert(int i, int v) {
    a[i] = v;
    if (v < 0) noneg = false;
  }

  public Alias() {
    //@ set a.owner = this;
  }
}
\end{verbatim}
\vspace*{-40ex}
\begin{picture}(100,100)(0,0)
\thicklines
\red
\put(80,60){\oval(150,10)}
\put(75,-40){\oval(110,10)}
\end{picture}

\begin{picture}(150,150)(-150,-80)
\red
\put(38,86){\framebox(45,25){\shortstack[l]{int noneg \\ int[] a}}}
\put(113,86){\framebox(45,25){\shortstack[l]{int noneg \\ int[] a}}}
\put(170,42){\framebox(45,25){\shortstack[l]{int[] ...... \\ owner}}}
\put(100,7){\framebox(45,25){\shortstack[l]{int[] ...... \\ owner}}}
\knalblue
\put(38,115){this}
\put(113,115){brokenObj}
\put(170,70){\textit{an} int[] \textit{object}}
\put(100,35){\textit{an} int[] \textit{object}}
\green
\put(70,92){\vector(1,-2){30}}
\put(145,92){\vector(1,-1){25}}
\put(70,92){\circle*{5}}
\put(145,92){\circle*{5}}
\put(208,50){\circle*{5}}
\put(138,15){\circle*{5}}
\put(208,50){\vector(-1,1){50}}
\put(138,15){\vector(-2,3){55}}
\end{picture}

\end{slide}

\begin{slide}{Aliasing}
\vspace*{-9ex}
Another example.  This one fails on the postcondition.
\tiny
\begin{verbatim}
public class Alias2 {
  /*@ non_null */ Inner n = new Inner();
  /*@ non_null */ Inner nn = new Inner();
  //@ invariant n.owner == this;
  //@ invariant nn.owner == this;

  //@ ensures n.i == \old(n.i + 1);
  public void add() {
    n.i++;
    nn.i++;
  }

  Alias2();
}

class Inner { 
  public int i; 
  //@ ensures i == 0;
  Inner();
}
\end{verbatim}
\vspace*{-40ex}
\begin{picture}(100,100)(0,0)
\thicklines
\red
\put(90,82){\oval(170,10)}
\end{picture}

\end{slide}

\begin{slide}{Aliasing}
\begin{itemize}
\item The counterexample context shows
\begin{verbatim}

    this.(nn:3.24) == tmp0!n:10.4
    tmp2!nn:11.4 == tmp0!n:10.4

\end{verbatim}
\item These hint that {\blue n} and {\blue nn} are references to the same object.

\item If we add the invariant {\blue //@ invariant n != nn; } to forbid aliasing between these two fields, then all is well.
\end{itemize}
\end{slide}

\begin{slide}{Aliasing}
\begin{itemize}

\item Aliasing is a serious difficulty in verification
\item Handling aliasing is an active area of research, related to handling frame conditions
\item It is all about knowing what is modified and what is not
\item These {\blue owner} fields or the equivalent create a form of encapsulation that can be 
checked by ESC/Java to control what might be modified by a given operation
\item There is a proposal to add so-called {\blue universes} to JML
to provide a more advanced form of alias control.
\end{itemize}
\end{slide}

%%%%%%%%%%%%%%%%%%%%%%%%%%%%%%%%%%%%%%%%%%%%%%%%%%%%%%%%%%%%%%%%%%%%%%%%%%%%%

\begin{slide}{\#3: Write object invariants}
\vspace*{-6ex}
\begin{itemize}
\item Be sure that class invariants are about the object at hand.  
\item Statements about all objects of
a class may indeed be true, but they are difficult to prove, especially for automated provers.

\item For example, if a predicate P is supposed to hold for objects of type T, then do {\red not} write
\begin{verbatim}
//@ invariant (\forall T t; P(t));
\end{verbatim}

\item Instead, write
\begin{verbatim}
//@ invariant P(this);
\end{verbatim}

\item The latter will make a more provable postcondition at the end of a constructor.
\end{itemize}

\end{slide}

%%%%%%%%%%%%%%%%%%%%%%%%%%%%%%%%%%%%%%%%%%%%%%%%%%%%%%%%%%%%%%%%%%%%%%%%%%%%%

\begin{slide}{\#4: Inconsistent assumptions}
\vspace*{-3ex}
If you have inconsistent specifications you can prove anything:

\verbatimcode{examples/Inconsistent.java}
\end{slide}

\begin{slide}{\#4: Inconsistent assumptions}
\vspace*{-3ex}
Another example:

\verbatimcode{examples/Inconsistent2.java}

We hope to put in checks for this someday!

\end{slide}


%%%%%%%%%%%%%%%%%%%%%%%%%%%%%%%%%%%%%%%%%%%%%%%%%%%%%%%%%%%%%%%%%%%%%%%%%%%%%

\begin{slide}{\#5: Exposed references}
\vspace*{-9ex}
Problems can arise when a reference to an internal object is exported from a class:
\begin{figure*}
\tiny
\verbatiminput{examples/Exposed.java}
\end{figure*}
\vspace*{-3ex}
\begin{itemize}
\item ESC/Java does not check that \textit{every} allocated object still satisfies its invariants.
\item Similar hidden problems can result if public fields are modified directly.
\end{itemize}

\begin{picture}(100,100)(0,0)
\thicklines
\red
\put(165,195){\oval(270,10)}
\end{picture}

\end{slide}

%%%%%%%%%%%%%%%%%%%%%%%%%%%%%%%%%%%%%%%%%%%%%%%%%%%%%%%%%%%%%%%%%%%%%%%%%%%%
\part{{\Large \red \#6: \old }}
%%%%%%%%%%%%%%%%%%%%%%%%%%%%%%%%%%%%%%%%%%%%%%%%%%%%%%%%%%%%%%%%%%%%%%%%%%%%


%%%%%%%%%%%%%%%%%%%%%%%%%%%%%%%%%%%%%%%%%%%%%%%%%%%%%%%%%%%%%%%%%%%%%%%%%%%%%

\begin{slide}{\old}
\vspace*{-6ex}
{\knalblue \old} is used to indicate evaluation in the pre-state in a postcondition expression.\\
\vspace{1ex}
Consider specifying
{\tiny
\begin{verbatim} 
public static native void arraycopy(Object[] src, int srcPos,
                                    Object[] dest, int destPos, int length);
\end{verbatim}
}
Try:
{\tiny
\begin{verbatim} 
ensures (\forall int i; 0<=i && i<length; dest[destPos+i] == src[srcPos+i]);
\end{verbatim}
}


\end{slide}

%%%%%%%%%%%%%%%%%%%%%%%%%%%%%%%%%%%%%%%%%%%%%%%%%%%%%%%%%%%%%%%%%%%%%%%%%%%%%

\begin{slide}{\old}
\vspace*{-6ex}
{\knalblue \old} is used to indicate evaluation in the pre-state in a postcondition expression.\\
\vspace{1ex}
Consider specifying
{\tiny
\begin{verbatim} 
public static native void arraycopy(Object[] src, int srcPos,
                                    Object[] dest, int destPos, int length);
\end{verbatim}
}
Try:
{\tiny
\begin{verbatim} 
ensures (\forall int i; 0<=i && i<length; dest[destPos+i] == src[srcPos+i]);
\end{verbatim}
}

{\red
Wrong!}\\

\end{slide}

%%%%%%%%%%%%%%%%%%%%%%%%%%%%%%%%%%%%%%%%%%%%%%%%%%%%%%%%%%%%%%%%%%%%%%%%%%%%%

\begin{slide}{\old}
\vspace*{-6ex}
{\knalblue \old} is used to indicate evaluation in the pre-state in a postcondition expression.\\
\vspace{1ex}
Consider specifying
{\tiny
\begin{verbatim} 
public static native void arraycopy(Object[] src, int srcPos,
                                    Object[] dest, int destPos, int length);
\end{verbatim}
}
Try:
{\tiny
\begin{verbatim} 
ensures (\forall int i; 0<=i && i<length; dest[destPos+i] == src[srcPos+i]);
\end{verbatim}
}

{\red
Wrong!}\\
\vspace{1ex}
Besides exceptions and invalid arguments, don't forget aliasing - {\blue dest} and {\blue src} may be the same array:
{\tiny
\begin{verbatim} 
ensures (\forall int i; 0<=i && i<length; 
                            dest[destPos+i] == \old(src[srcPos+i]);
\end{verbatim}
}

\end{slide}

%%%%%%%%%%%%%%%%%%%%%%%%%%%%%%%%%%%%%%%%%%%%%%%%%%%%%%%%%%%%%%%%%%%%%%%%%%%%%

\begin{slide}{\old}
\vspace*{-6ex}
{\knalblue \old} is used to indicate evaluation in the pre-state in a postcondition expression.\\
\vspace{1ex}
Consider specifying
{\tiny
\begin{verbatim} 
public static native void arraycopy(Object[] src, int srcPos,
                                    Object[] dest, int destPos, int length);
\end{verbatim}
}
Try:
{\tiny
\begin{verbatim} 
ensures (\forall int i; 0<=i && i<length; dest[destPos+i] == src[srcPos+i]);
\end{verbatim}
}

{\red
Wrong!}\\
\vspace{1ex}
Besides exceptions and invalid arguments, don't forget aliasing - {\blue dest} and {\blue src} may be the same array:
{\tiny
\begin{verbatim} 
ensures (\forall int i; 0<=i && i<length; 
                            dest[destPos+i] == \old(src[srcPos+i]);
\end{verbatim}
}
And don't forget the other elements:
{\tiny
\begin{verbatim} 
ensures (\forall int i; (0<=i && i<destPos) || 
                              (destPos+length <= i && i < destPos.length); 
                            dest[i] == \old(dest[i]);
\end{verbatim}
}

\end{slide}

%%%%%%%%%%%%%%%%%%%%%%%%%%%%%%%%%%%%%%%%%%%%%%%%%%%%%%%%%%%%%%%%%%%%%%%%%%%%%

\begin{slide}{\old}
\vspace*{-6ex}

In postcondition 
{\tiny
\begin{verbatim} 
ensures (\forall int i; 0<=i && i<length; 
                            dest[destPos+i] == \old(src[srcPos+i]);
public static native void arraycopy(Object[] src, int srcPos,
                                    Object[] dest, int destPos, int length);
\end{verbatim}
}
shouldn't we write \code{\old(length)} instead of \code{length}?

\end{slide}

%%%%%%%%%%%%%%%%%%%%%%%%%%%%%%%%%%%%%%%%%%%%%%%%%%%%%%%%%%%%%%%%%%%%%%%%%%%%%

\begin{slide}{\old}
\vspace*{-6ex}

In postcondition 
{\tiny
\begin{verbatim} 
ensures (\forall int i; 0<=i && i<length; 
                            dest[destPos+i] == \old(src[srcPos+i]);
public static native void arraycopy(Object[] src, int srcPos,
                                    Object[] dest, int destPos, int length);
\end{verbatim}
}
shouldn't we write \code{\old(length)} instead of \code{length}?

And \code{\old(dest)[...]} instead of \code{dest[...]}?

\end{slide}

%%%%%%%%%%%%%%%%%%%%%%%%%%%%%%%%%%%%%%%%%%%%%%%%%%%%%%%%%%%%%%%%%%%%%%%%%%%%%

\begin{slide}{\old}
\vspace*{-6ex}

In postcondition 
{\tiny
\begin{verbatim} 
ensures (\forall int i; 0<=i && i<length; 
                            dest[destPos+i] == \old(src[srcPos+i]);
public static native void arraycopy(Object[] src, int srcPos,
                                    Object[] dest, int destPos, int length);
\end{verbatim}
}
shouldn't we write \code{\old(length)} instead of \code{length}?

And \code{\old(dest)[...]} instead of \code{dest[destPos+i]}?

\smallskip

Strictly speaking: yes. But because this is so easy to get forget,
{\blue any mention of an argument \code{x} in postcondition means \code{\old(x)}.}

\end{slide}

%%%%%%%%%%%%%%%%%%%%%%%%%%%%%%%%%%%%%%%%%%%%%%%%%%%%%%%%%%%%%%%%%%%%%%%%%%%%%

\begin{slide}{\old}
\vspace*{-6ex}

In postcondition 
{\tiny
\begin{verbatim} 
ensures (\forall int i; 0<=i && i<length; 
                            dest[destPos+i] == \old(src[srcPos+i]);
public static native void arraycopy(Object[] src, int srcPos,
                                    Object[] dest, int destPos, int length);
\end{verbatim}
}
shouldn't we write \code{\old(length)} instead of \code{length}?

And \code{\old(dest)[...]} instead of \code{dest[destPos+i]}?

\smallskip

Strictly speaking: yes. But because this is so easy to get forget,
{\blue any mention of an argument \code{x} in postcondition means \code{\old(x)}.}

\medskip

{\scriptsize This means it's impossible to refer to the new value of 
\code{length} in postcondition of \code{arraycopy}. But this value is
unobservable for clients anyway.}

\end{slide}

%%%%%%%%%%%%%%%%%%%%%%%%%%%%%%%%%%%%%%%%%%%%%%%%%%%%%%%%%%%%%%%%%%%%%%%%%%%%%

%%%%%%%%%%%%%%%%%%%%%%%%%%%%%%%%%%%%%%%%%%%%%%%%%%%%%%%%%%%%%%%%%%%%%%%%%%%%
\part{{\Large \red \#7: How to write specs }}
%%%%%%%%%%%%%%%%%%%%%%%%%%%%%%%%%%%%%%%%%%%%%%%%%%%%%%%%%%%%%%%%%%%%%%%%%%%%


%%%%%%%%%%%%%%%%%%%%%%%%%%%%%%%%%%%%%%%%%%%%%%%%%%%%%%%%%%%%%%%%%%%%%%%%%%%%%

\overlays{9}{
\begin{slide}{Getting started}
\vspace*{-6ex}
\begin{itemstep}
\item Start with foundation and library routines
\item For each field: is there an invariant for this field?
\item For each reference field: should it be {\knalblue non\_null}?
\item For each reference field: should an {\knalblue owner} field be set for it?
\item For each method: should it be {\knalblue pure}?  Should the arguments or the result be {\knalblue non\_null}?
\item For each class: what {\knalblue invariant} expresses the self-consistency of the internal data?
\item Add {\knalblue pre-} and {\knalblue post-conditions} to limit the inputs and outputs of each method.
\item Add possible unchecked {\knalblue exceptions} to throws clauses.
\item Start with simple specifications; proceed to complex ones as they have value.
\end{itemstep}

\end{slide}}

\begin{slide}{Getting started}
\begin{itemize}
\item Separate conjunctions to get information about which conjunct is violated. Use
{\scriptsize \begin{verbatim}
requires A;
requires B;
\end{verbatim}
}
not
{\scriptsize \begin{verbatim}
requires A && B;
\end{verbatim}
}

\item Use {\knalblue assert} statements to find out what is going wrong.

\item Use {\knalblue assume} statements {\red \textit{that you KNOW are correct}} to help the prover along.
\end{itemize}

\end{slide}

%%%%%%%%%%%%%%%%%%%%%%%%%%%%%%%%%%%%%%%%%%%%%%%%%%%%%%%%%%%%%%%%%%%%%%%%%%%%%
\begin{slide}{\textit{Finally}}
\begin{itemize}

\item Specification is {\green tricky} - getting it right is hard, even with tools

\item {\green Try it} - a substantial research gap is experience on industrial-scale sets of code

\item {\green Communicate} - we are willing to offer advice

\item {\green Share} your experience - tools will get better and we will all learn better techniques for 
successful specification (use JML and ESC/Java mailing lists)


\end{itemize}

\end{slide}
%%%%%%%%%%%%%%%%%%%%%%%%%%%%%%%%%%%%%%%%%%%%%%%%%%%%%%%%%%%%%%%%%%%%%%%%%%%%%
\end{document}
