%% ESC/Java2 tutorial
% David Cok% $Id$%
\documentclass[
pdf,
%ps,
%a4,
nocolorBG,
%colorBG,
slideColor,
%accumulate,
%draft,
cok,
%frames,
]{prosper}
%%%%%%%%%%%%%%%%%%%%%%%%%%%%%%%%%%%%%%%%%%%%%%%%%%%%%%%%%%%%%%%%%%%%%%%%%%%
\usepackage{alltt}
\usepackage{pstricks,pst-node,pst-text,pst-3d}
\usepackage{textcomp}
\usepackage{verbatim}
% \usepackage{colordvi}
\newcommand{\Red}[1]{{\red #1}}
\usepackage[all]{xy}

%%%%%%%%%%%%%%%%%%%%%%%%%%%%%%%%%%%%%%%%%%%%%%%%%%%%%%%%%%%%%%%%%%%%%%%%%%%%
\newrgbcolor{Yellowish}{0.90 0.85 0.650}%
\newrgbcolor{red}{1 0 0}
\newrgbcolor{purple}{0.4 0 0.7}%
\newrgbcolor{lightpurple}{0.63 0.13 0.94}%
\newrgbcolor{lime}{0.73 1 0}
\newrgbcolor{fullgreen}{0 1 0} 
\newrgbcolor{green}{0.133 0.56 0} % lichter%
\newrgbcolor{green}{0.10 0.43 0}
\newrgbcolor{knalblue}{0 0 1}
\newrgbcolor{blue}{.2 .36 .77}%
\newrgbcolor{darkblue}{0.28 0.24 0.55}
\newrgbcolor{gray}{.5 .5 .5}
%%%%%%%%%%%%%%%%%%%%%%%%%%%%%%%%%%%%%%%%%%%%%%%%%%%%%%%%%%%%%%%%%%%%%%%%%%%
\newcommand{\embf}[1]{\textit{\textbf{#1}}}
\newcommand{\rmbf}[1]{\textrm{\textbf{#1}}}
\newcommand{\ttbf}[1]{\mbox{\tt \textbf{#1}}}
\newcommand{\code}[1]{{\rm \texttt{\textbf{\scriptsize #1}}}}
\myitem{1}{\mbox{{$\bullet$}}}

\newcommand{\vooralle}{\(\backslash\)forall}
\newcommand{\eris}{\(\backslash\)exists}
\newcommand{\bsl}{\char'134}
\newcommand{\result}{\bsl result}
\newcommand{\old}{\bsl old}

\newcommand{\verbatimcode}[1]{
\begin{figure*}
\tiny
\verbatiminput{#1}
\end{figure*}
}

\newif\ifignore
\newrgbcolor{Bluish}{0.9 0.9 1.0}
\newcommand{\doos}[1]{
 \psshadowbox[fillstyle=solid,                        
              fillcolor=Bluish,                        
              framearc=0.2,                        
              framesep=2mm]{#1}} 
%%%%%%%%%%%%%%%%%%%%%%%%%%%%%%%%%%%%%%%%%%%%%%%%%%%%%%%%%%%%%%%%%%%%%%%%%%%%%
\title{\embf{\blue        {\huge ESC/Java2       \\ \medskip       {\Large Use and Features }}      }}
\author{\embf{\Large{\red David Cok, Joe Kiniry, Erik Poll}}       }
\institution{\embf{Eastman Kodak Company, University College Dublin, \\ and Radboud University Nijmegen}}
\slideCaption{{\blue David Cok, Joe Kiniry \& Erik Poll - ESC/Java2 \& JML Tutorial}}
%%%%%%%%%%%%%%%%%%%%%%%%%%%%%%%%%%%%%%%%%%%%%%%%%%%%%%%%%%%%%%%%%%%%%%%%%%%%%
\begin{document}\maketitle \boldmath
%%%%%%%%%%%%%%%%%%%%%%%%%%%%%%%%%%%%%%%%%%%%%%%%%%%%%%%%%%%%%%%%%%%%%%%%%%%%%
%%%%%%%%%%%%%%%%%%%%%%%%%%%%%%%%%%%%%%%%%%%%%%%%%%%%%%%%%%%%%%%%%%%%%%%%%%%%%
%%%%%%%%%%%%%%%%%%%%%%%%%%%%%%%%%%%%%%%%%%%%%%%%%%%%%%%%%%%%%%%%%%%%%%%%%%%%%
%%%%%%%%%%%%%%%%%%%%%%%%%%%%%%%%%%%%%%%%%%%%%%%%%%%%%%%%%%%%%%%%%%%%%%%%%%%%%
%%%%%%%%%%%%%%%%%%%%%%%%%%%%%%%%%%%%%%%%%%%%%%%%%%%%%%%%%%%%%%%%%%%%%%%%%%%%%

\begin{slide}{ }

\vspace*{0ex}
\begin{center}{\large \red The ESC/Java2 tool}\end{center}
\end{slide}

%%%%%%%%%%%%%%%%%%%%%%%%%%%%%%%%%%%%%%%%%%%%%%%%%%%%%%%%%%%%%%%%%%%%%%%%%%%%%

\begin{slide}{Structure of ESC/Java2}
\vspace*{-6ex}
\begin{itemize}
\item[] ESC/Java2 consists of a 
\begin{itemize}
\item parsing phase (syntax checks),
\item typechecking phase (type and usage checks),
\item static checking phase (reasoning to find potential bugs) - runs a behind-the-scenes prover called Simplify
\end{itemize}
\item[] Parsing and typechecking produce {\red cautions} or {\red errors}.
\item[] Static checking produces {\red warnings}.
\item[] \textit{The focus of ESC/Java2 is on static checking, but reports of bugs, unreported errors, confusing messages, documentation or behavior, and even just email about your application and degree of success are {\green Very Welcome}.  [and Caution: this is still an {\red alpha} release]}
\end{itemize}
\end{slide}

%%%%%%%%%%%%%%%%%%%%%%%%%%%%%%%%%%%%%%%%%%%%%%%%%%%%%%%%%%%%%%%%%%%%%%%%%%%%%

\begin{slide}{Running ESC/Java2}
\vspace*{-6ex}
\begin{itemize}
\item Download the binary distribution from {\green http://secure.ucd.ie/products/opensource/ESCJava2}
\item Untar the distribution and follow the instructions in {\green README.release} about setting environment variables.
\item Run the tool by doing one of the following:
\begin{itemize}
\item Run a script in the release: {\green escjava2} or {\green escj.bat}
\item Run the tool directly with {\green java -cp esctools2.jar escjava.Main}, but then you need to be sure to provide values for the {\blue -simplify} and {\blue -specs}  options.
\item Run a GUI version of the tool by double-clicking the release version of {\green esctools2.jar}
\item Run a GUI version of the tool by executing it with {\green java -jar esctools2.jar} (in which case you can add options).
\end{itemize}
\end{itemize}
\end{slide}

%%%%%%%%%%%%%%%%%%%%%%%%%%%%%%%%%%%%%%%%%%%%%%%%%%%%%%%%%%%%%%%%%%%%%%%%%%%%%

\begin{slide}{Supported platforms}
\vspace*{-6ex}
ESC/Java2 is supported on
\begin{itemize}
\item Linux
\item MacOSX
\item Cygwin on Windows
\item Windows (but there are some environment issues still to be resolved)
\item Solaris (in principle - we are not testing there)
\end{itemize}
Note that the tool itself is relatively portable Java, but the underlying prover is a Modula-3 application that must be compiled and supplied for each platform.

Help with platform-dependence issues is welcome.
\end{slide}

%%%%%%%%%%%%%%%%%%%%%%%%%%%%%%%%%%%%%%%%%%%%%%%%%%%%%%%%%%%%%%%%%%%%%%%%%%%%%

\begin{slide}{Environment}
\vspace*{-6ex}
The application relies on the environment having 
\begin{itemize}
\item a Simplify executable (such as Simplify-1.5.4.macosx) for your platform, typically in the same directory as the application's jar file;
\item the {\blue SIMPLIFY} environment variable set to the name of the executable for this platform;
\item a set of specifications for Java system files - by default these are bundled into the application jar file, but they are also in {\blue jmlspecs.jar}.
\item The scripts prefer that the variable {\blue ESCTOOLS\_RELEASE} be set to the directory containing the release.
\end{itemize}
\end{slide}

%%%%%%%%%%%%%%%%%%%%%%%%%%%%%%%%%%%%%%%%%%%%%%%%%%%%%%%%%%%%%%%%%%%%%%%%%%%%%

\begin{slide}{Command-line options}
\vspace*{-9ex}
The items on the command-line are either options and their arguments or input entries.
Some commonly used options (see the documentation for more):
\tiny
\begin{itemize}
\item {\knalblue -help} - prints a usage message
\item {\knalblue -quiet} - turns off informational messages (e.g. progress messages)
\item {\knalblue -nowarn} - turns off a warning
\item {\knalblue -classpath} - sets the path to find referenced classes [best if it contains `.']
\item {\knalblue -specs} - sets the path to library specification files
\item {\knalblue -simplify} - provides the path to the simplify executable
\item {\knalblue -f} - the argument is a file containing command-line arguments
\item {\knalblue -nocheck} - parse and typecheck but no verification
\item {\knalblue -routine} - restricts checking to a single routine
\item {\knalblue -eajava}, {\knalblue -eajml} - enables checking of Java assertions
\item {\knalblue -counterexample} - gives detailed information about a warning
\end{itemize}
\end{slide}

%%%%%%%%%%%%%%%%%%%%%%%%%%%%%%%%%%%%%%%%%%%%%%%%%%%%%%%%%%%%%%%%%%%%%%%%%%%%%

\begin{slide}{Input entries}
\vspace*{-6ex}
The input entries on the command-line are those classes that are actually checked.
Many other classes may be referenced for class definitions or specifications - these are found on the classpath (or sourcepath or specspath).

\begin{itemize}
\item {\knalblue file names} - of java or specification files (relative to the current directory)
\item {\knalblue directories} - processes all java or specification files (relative to the current directory)
\item {\knalblue package} - (fully qualified name) - found on the classpath
\item {\knalblue class} - (fully qualified name) - found on the classpath
\item {\knalblue list} - (prefaced by {\green -list}) - a file containing input entries
\end{itemize}
\end{slide}

%%%%%%%%%%%%%%%%%%%%%%%%%%%%%%%%%%%%%%%%%%%%%%%%%%%%%%%%%%%%%%%%%%%%%%%%%%%%%

\begin{slide}{Specification files}
\vspace*{-6ex}
\begin{itemize}
\item Specifications may be added directly to .java files
\item Specifications may alternatively be added to specification files.
\begin{itemize}
\item No method bodies
\item No field initializers
\item Recommended suffix: {\blue .refines-java}
\item Recommend a {\blue refines} annotation (see documentation)
\item Must also be on the classpath
\end{itemize}
\end{itemize}
\end{slide}

%%%%%%%%%%%%%%%%%%%%%%%%%%%%%%%%%%%%%%%%%%%%%%%%%%%%%%%%%%%%%%%%%%%%%%%%%%%%%

\begin{slide}{Specification file example}
\vspace*{-6ex}
\tiny
\begin{verbatim}
package java.lang;
import java.lang.reflect.*;
import java.io.InputStream;

public final class Class implements java.io.Serializable {

    private Class();

    /*@ also public normal_behavior
      @   ensures \result != null && !\result.equals("")
      @        && (* \result is the name of this class object *);
      @*/
    public /*@ pure @*/ String toString();

    ....


\end{verbatim}
\end{slide}

%%%%%%%%%%%%%%%%%%%%%%%%%%%%%%%%%%%%%%%%%%%%%%%%%%%%%%%%%%%%%%%%%%%%%%%%%%%%%

\begin{slide}{Bag demo}

\end{slide}

%%%%%%%%%%%%%%%%%%%%%%%%%%%%%%%%%%%%%%%%%%%%%%%%%%%%%%%%%%%%%%%%%%%%%%%%%%%%
\begin{slide}{modular reasoning}
\vspace*{-6ex}

ESC/Java2 reasons about every method individually.
So in

\begin{alltt}\code{ class A\{
  byte[] b;
  public void n() \{ b = new byte[20]; \}
  public void m() \{ n();
                    b[0] = 2;
                    ...       \}
}
\end{alltt} %}

ESC/Java2 warns that \texttt{b[0]} may be a null dereference here,
even though you can see that it won't be.
\end{slide}

\begin{slide}{modular reasoning}
\vspace*{-6ex}
To stop ESC/Java2 complaining: add a postcondition
\begin{alltt}\code{ class A\{
  byte[] b;
 {\green //@ ensures b != null && b.length = 20;}
  public void n() \{ b = new byte[20]; \}
  public void m() \{ n();
                    b[0] = 2;
                    ...       \} 
}
\end{alltt} %}
So: property of method that is relied on has to be made explicit.

Also: subclasses that override methods have to preserve these.

\end{slide}

%%%%%%%%%%%%%%%%%%%%%%%%%%%%%%%%%%%%%%%%%%%%%%%%%%%%%%%%%%%%%%%%%%%%%%%%%%%%
\begin{slide}{modular reasoning}
\vspace*{-6ex}

Similarly, ESC/Java will complain about \texttt{b[0] = 2} in

\begin{alltt}\code{ class A\{
  byte[] b;
  public void A() \{ b = new byte[20]; \}
  public void m() \{ b[0] = 2;
                    ...  \}

}
\end{alltt} %}
Maybe you can see that this is a spurious warning, though this will be
harder than in the previous example: you'll have to inspect {\em all}
constructors and {\em all} methods.

\end{slide}

%%%%%%%%%%%%%%%%%%%%%%%%%%%%%%%%%%%%%%%%%%%%%%%%%%%%%%%%%%%%%%%%%%%%%%%%%%%%
\begin{slide}{modular reasoning}
\vspace*{-6ex}

To stop ESC/Java2 complaining here: add an invariant

\begin{alltt}\code{ class A\{
  byte[] b;
  {\green //@ invariant b != null && b.length == 20;}
  {\green     // or weaker property for b.length ?}
  public void A() \{ b = new byte[20]; \}
  public void m() \{ b[0] = 2;
                    ...  \}
}
\end{alltt} %}

So again: properties you rely on have to be made explicit.

\medskip

And again: subclasses have to preserve these properties.

\end{slide}

%%%%%%%%%%%%%%%%%%%%%%%%%%%%%%%%%%%%%%%%%%%%%%%%%%%%%%%%%%%%%%%%%%%%%%%%%%%%
\begin{slide}{assume}
\vspace*{-6ex}

Alternative to stop ESC/Java2 complaining: add an assumption:
\begin{alltt}\code{ 
    ...
    //@{\blue assume} b != null && b.length > 0;
    b[0] = 2;
    ...  
}
\end{alltt}

Especially useful during development, when you're still trying to
discover hidden assumptions, or when ESC/Java2's reasoning power is
too weak.

\medskip

(\texttt{requires} can be understood as a form of \texttt{assume}.)

\end{slide}

%%%%%%%%%%%%%%%%%%%%%%%%%%%%%%%%%%%%%%%%%%%%%%%%%%%%%%%%%%%%%%%%%%%%%%%%%%%%
\begin{slide}{need for assignable clauses}
\vspace*{-6ex}


\begin{alltt}\code{ class A\{
  byte[] b;
  ...
  public void m() \{ ... 
                    b = new byte[3];
                    //@ assert b[0] == 0; // ok!
                    o.n(...);
                    //@ assert b[0] == 0; // ok?
                    ... 
  \}

}
\end{alltt} %}
What does ESC/Java need to know about \texttt{o.n} to check the second assert ?

\end{slide}


\begin{slide}{need for assignable clauses}
\vspace*{-6ex}

\begin{alltt}\code{ class A\{
  byte[] b;
  ...
  public void m() \{ ... 
                    b = new byte[3];
                    //@ assert b[0] == 0; // ok!
                    o.n(b);
                    //@ assert b[0] == 0; // ok?
                    ... 
  \}

}
\end{alltt} %}

A detailed spec for \texttt{o.n} might give a postcondition saying
that \texttt{b[0]} is still 0.

\end{slide}


\begin{slide}{need for assignable clauses}
\vspace*{-6ex}

\begin{alltt}\code{ class A\{
  byte[] b;
  ...
  public void m() \{ ... 
                    b = new byte[3];
                    //@ assert b[0] == 0; // ok!
                    o.n();
                    //@ assert b[0] == 0; // ok?
                    ... 
  \}

}
\end{alltt} %}

If the postcondition of \texttt{o.n} doesn't tell us \texttt{b} won't be not null
-- and can't be expected to -- we need the \texttt{assignable} clause to
tell us that \texttt{o.n} won't affect \texttt{b[0]}.

\medskip

{\scriptsize Declaring \texttt{\textbf{o.n}} as pure would solve the problem.}


\end{slide}






%%%%%%%%%%%%%%%%%%%%%%%%%%%%%%%%%%%%%%%%%%%%%%%%%%%%%%%%%%%%%%%%%%%%%%%%%%%%%
\begin{slide}{ESC/Java is not complete}
\vspace*{-6ex}

ESC/Java may produce warnings about correct programs.

\begin{alltt}
\texttt{\textbf{\scriptsize
 {\blue /*@ requires 0 < n;
    @ ensures \result ==
    @              (\eris int x,y,z;
    @                pow(x,n)+pow(y,n) == pow(z,n));
    @*/}
  public static boolean{\green fermat}(double n) \{
    return (n==2);
  \}
}}
\end{alltt}

Warning: {\em postcondition possibly not satisfied}

{\scriptsize (Typically, the theorem prover times out in complicated cases.)}

\end{slide}




%%%%%%%%%%%%%%%%%%%%%%%%%%%%%%%%%%%%%%%%%%%%%%%%%%%%%%%%%%%%%%%%%%%%%%%%%%%%%
\begin{slide}{ESC/Java is not sound}
\vspace*{-6ex}

ESC/Java may fail to produce warning about incorrect program.

\begin{alltt}
\texttt{\textbf{\footnotesize  public class Positive\{
      private int n = 1; {\blue //@ invariant n > 0;}

      public void increase()\{ n++; \}
  \}
}}
\end{alltt}

ESC/Java(2) produces no warning, but \code{increase} may break the invariant, namely if \code{n} is $2^{32}-1$.

\medskip

This can be fixed by improved model of Java arithmetic,
but this does come at a price (both in specs and in code).

\end{slide}



%%%%%%%%%%%%%%%%%%%%%%%%%%%%%%%%%%%%%%%%%%%%%%%%%%%%%%%%%%%%%%%%%%%%%%%%%%%%%
\begin{slide}{ESC/Java is not sound}
\vspace*{-6ex}

More fundamental problem: {\green sound modular verification for OO programs
with invariants}.
\begin{alltt}\texttt{\textbf{\footnotesize public class{\green A}\{               public class{\green B}\{
  B{\green b};                         
  int{\green x};                       int{\green y};
  //@{\blue invariant x <= b.y;}        
  void{\green decr\_x}()\{                void{\green decr\_y}()\{          
    x--;  \}                          y--; \}
 \}                            \}
}}
\end{alltt}

How can we know that invoking \code{decr\_y} on some \code{B} won't break
the invariant of some \code{A}, or some object whose invariant
depends on a \code{B} object.



\end{slide}

%%%%%%%%%%%%%%%%%%%%%%%%%%%%%%%%%%%%%%%%%%%%%%%%%%%%%%%%%%%%%%%%%%%%%%%%%%%%%
\begin{slide}{ESC/Java is not sound}
\vspace*{-7ex}

\begin{alltt}\texttt{\textbf{\footnotesize public class{\green A}\{               public class{\green B}\{
  B{\green b};                         
  int{\green x};                       int{\green y};
  //@{\blue invariant x <= b.y;}        
  void{\green decr\_x}()\{x++;\}           void{\green incr\_y}()\{\Red{y++;}\}    
 \}                            \}

 public class{\green D}\{
  B{\green b};
  void{\green decr\_y}()\{          
    b.y--;  \}                          
 \}                            }}
\end{alltt}

How can \code{D} know it might be breaking \code{A}'s invariant?

\end{slide}


%%%%%%%%%%%%%%%%%%%%%%%%%%%%%%%%%%%%%%%%%%%%%%%%%%%%%%%%%%%%%%%%%%%%%%%%%%%%%
\begin{slide}{Modularity problem}
\vspace*{-6ex}

{\blue Modular verification for (open) OO programs
with invariants} is a big \& fundamental problem.
Most verification tools fail here.
Root causes:
\begin{enumerate}
\item invariants talking about another object's fields
\item object modifying another object's field
\item possibility of {\green aliasing}
\end{enumerate}

{\scriptsize NB 1 \& 2 are unavoidable, eg. think of an object
modifying  -- or its invariant mentioning -- the contents of an 
array field}

\medskip

Alias control and ownership might provide solutions,
eg.\ universes by Peter M\"uller \& co or 
explicit pack/unpack operations by Rustan Leino \& co.

\end{slide}



\end{document}

\end{document}
