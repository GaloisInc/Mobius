%
% ESC/Java2 tutorial
% David Cok
% $Id$
%

\documentclass[
pdf,
%ps,
%a4,
nocolorBG,
%colorBG,
slideColor,
%accumulate,
%draft,
cok,
%frames,
]{prosper}


%%%%%%%%%%%%%%%%%%%%%%%%%%%%%%%%%%%%%%%%%%%%%%%%%%%%%%%%%%%%%%%%%%%%%%%%%%%
\usepackage{alltt}
\usepackage{pstricks,pst-node,pst-text,pst-3d}
\usepackage{textcomp}
\usepackage{verbatim}
%\usepackage{colordvi}
\newcommand{\Red}[1]{{\red #1}}
\usepackage[all]{xy}


%%%%%%%%%%%%%%%%%%%%%%%%%%%%%%%%%%%%%%%%%%%%%%%%%%%%%%%%%%%%%%%%%%%%%%%%%%%

%\newrgbcolor{Yellowish}{0.90 0.85 0.650}

%\newrgbcolor{red}{1 0 0}
\newrgbcolor{purple}{0.4 0 0.7}
%\newrgbcolor{lightpurple}{0.63 0.13 0.94}

%\newrgbcolor{lime}{0.73 1 0}
\newrgbcolor{fullgreen}{0 1 0} 
\newrgbcolor{green}{0.133 0.56 0} % lichter
%\newrgbcolor{green}{0.10 0.43 0}

\newrgbcolor{knalblue}{0 0 1}
\newrgbcolor{blue}{.2 .36 .77}
%\newrgbcolor{darkblue}{0.28 0.24 0.55}

\newrgbcolor{gray}{.5 .5 .5}

%%%%%%%%%%%%%%%%%%%%%%%%%%%%%%%%%%%%%%%%%%%%%%%%%%%%%%%%%%%%%%%%%%%%%%%%%%%

\newcommand{\embf}[1]{\textit{\textbf{#1}}}
\newcommand{\rmbf}[1]{\textrm{\textbf{#1}}}
\newcommand{\ttbf}[1]{\mbox{\tt \textbf{#1}}}
\newcommand{\code}[1]{{\rm \texttt{\textbf{\small #1}}}}



\myitem{1}{\mbox{{$\bullet$}}}

\newcommand{\bsl}{\char'134}
\newcommand{\result}{\bsl result}
\newcommand{\old}{\bsl old}


\newcommand{\verbatimcode}[1]{
\begin{figure*}
\tiny
\verbatiminput{#1}
\end{figure*}
}

\newif\ifignore

\newrgbcolor{Bluish}{0.9 0.9 1.0}
\newcommand{\doos}[1]{\psshadowbox[fillstyle=solid,
                        fillcolor=Bluish,
                        framearc=0.2,
                        framesep=2mm]
                        {#1}}

%%%%%%%%%%%%%%%%%%%%%%%%%%%%%%%%%%%%%%%%%%%%%%%%%%%%%%%%%%%%%%%%%%%%%%%%%%%%%

\title{\embf{\blue 
       {\huge ESC/Java2
       \\ \medskip
       {\Large Use and Features }}
      }}
\author{\embf{\Large{\red David Cok, Erik Poll, Joe Kiniry}}
       }
\institution{\embf{{\large Eastman Kodak Company; University of Nijmegen}\\ June 2004}}

\slideCaption{{\blue Cok, Poll, Kiniry - ESC/Java2 Tutorial - June 2004}}

%%%%%%%%%%%%%%%%%%%%%%%%%%%%%%%%%%%%%%%%%%%%%%%%%%%%%%%%%%%%%%%%%%%%%%%%%%%%%

\begin{document}

\maketitle 

\boldmath


%%%%%%%%%%%%%%%%%%%%%%%%%%%%%%%%%%%%%%%%%%%%%%%%%%%%%%%%%%%%%%%%%%%%%%%%%%%%%
%%%%%%%%%%%%%%%%%%%%%%%%%%%%%%%%%%%%%%%%%%%%%%%%%%%%%%%%%%%%%%%%%%%%%%%%%%%%%
%%%%%%%%%%%%%%%%%%%%%%%%%%%%%%%%%%%%%%%%%%%%%%%%%%%%%%%%%%%%%%%%%%%%%%%%%%%%%
%%%%%%%%%%%%%%%%%%%%%%%%%%%%%%%%%%%%%%%%%%%%%%%%%%%%%%%%%%%%%%%%%%%%%%%%%%%%%
%%%%%%%%%%%%%%%%%%%%%%%%%%%%%%%%%%%%%%%%%%%%%%%%%%%%%%%%%%%%%%%%%%%%%%%%%%%%%

\begin{slide}{ }

\vspace*{0ex}
\begin{center}{\large \red The ESC/Java2 tool}\end{center}
\end{slide}


%%%%%%%%%%%%%%%%%%%%%%%%%%%%%%%%%%%%%%%%%%%%%%%%%%%%%%%%%%%%%%%%%%%%%%%%%%%%%


\begin{slide}{Structure of ESC/Java2}
\vspace*{-6ex}
\begin{itemize}
\item[] ESC/Java2 consists of a 
\begin{itemize}
\item parsing phase (syntax checks),
\item typechecking phase (type and usage checks),
\item static checking phase (reasoning to find potential bugs) - runs a behind-the-scenes prover called Simplify
\end{itemize}
\item[] Parsing and typechecking produce {\red cautions} or {\red errors}.
\item[] Static checking produces {\red warnings}.
\item[] {\it The focus of ESC/Java2 is on static checking, but reports of bugs, unreported errors, confusing messages, documentation or behavior, and even just email about your application and degree of success are {\green Very Welcome}.  [and Caution: this is still an {\red alpha} release]}
\end{itemize}

\end{slide}

%%%%%%%%%%%%%%%%%%%%%%%%%%%%%%%%%%%%%%%%%%%%%%%%%%%%%%%%%%%%%%%%%%%%%%%%%%%%%


\begin{slide}{Running ESC/Java2}
\vspace*{-6ex}
\begin{itemize}
\item Download the binary distribution from {\green http://www.cs.kun.nl/sos/research/escjava}
\item Untar the distribution and follow the instructions in {\green README.release} about setting environment variables.
\item Run the tool by doing one of the following:
\begin{itemize}
\item Run a script in the release: {\green escjava2} or {\green escj.bat}
\item Run the tool directly with {\green java -cp esctools2.jar escjava.Main}, but then you need to be sure to provide values for the {\blue -simplify} and {\blue -specs}  options.
\item Run a GUI version of the tool by double-clicking the release version of {\green esctools2.jar}
\item Run a GUI version of the tool by executing it with {\green java -jar esctools2.jar} (in which case you can add options).
\end{itemize}
\end{itemize}
\end{slide}

%%%%%%%%%%%%%%%%%%%%%%%%%%%%%%%%%%%%%%%%%%%%%%%%%%%%%%%%%%%%%%%%%%%%%%%%%%%%%


\begin{slide}{Supported platforms}
\vspace*{-6ex}
ESC/Java2 is supported on
\begin{itemize}
\item Linux
\item MacOSX
\item Cygwin on Windows
\item Windows (but there are some environment issues still to be resolved)
\item Solaris (in principle - we are not testing there)
\end{itemize}
Note that the tool itself is relatively portable Java, but the underlying prover is a Modula-3 application that must be compiled and supplied for each platform.

Help with platform-dependence issues is welcome.
\end{slide}

%%%%%%%%%%%%%%%%%%%%%%%%%%%%%%%%%%%%%%%%%%%%%%%%%%%%%%%%%%%%%%%%%%%%%%%%%%%%%


\begin{slide}{Environment}
\vspace*{-6ex}
The application relies on the environment having 
\begin{itemize}
\item a Simplify executable (such as Simplify-1.5.4.macosx) for your platform, typically in the same directory as the application's jar file;
\item the {\blue SIMPLIFY} environment variable set to the name of the executable for this platform;
\item a set of specifications for Java system files - by default these are bundled into the application jar file, but they are also in {\blue jmlspecs.jar}.
\item The scripts prefer that the variable {\blue ESCTOOLS\_RELEASE} be set to the directory containing the release.
\end{itemize}
\end{slide}

%%%%%%%%%%%%%%%%%%%%%%%%%%%%%%%%%%%%%%%%%%%%%%%%%%%%%%%%%%%%%%%%%%%%%%%%%%%%%


\begin{slide}{Command-line options}
\vspace*{-9ex}
The items on the command-line are either options and their arguments or input entries.
Some commonly used options (see the documentation for more):
\tiny
\begin{itemize}
\item {\knalblue -help} - prints a usage message
\item {\knalblue -quiet} - turns off informational messages (e.g. progress messages)
\item {\knalblue -nowarn} - turns off a warning
\item {\knalblue -classpath} - sets the path to find referenced classes [best if it contains `.']
\item {\knalblue -specs} - sets the path to library specification files
\item {\knalblue -simplify} - provides the path to the simplify executable
\item {\knalblue -f} - the argument is a file containing command-line arguments
\item {\knalblue -nocheck} - parse and typecheck but no verification
\item {\knalblue -routine} - restricts checking to a single routine
\item {\knalblue -eajava}, {\knalblue -eajml} - enables checking of Java assertions
\item {\knalblue -counterexample} - gives detailed information about a warning
\end{itemize}
\end{slide}

%%%%%%%%%%%%%%%%%%%%%%%%%%%%%%%%%%%%%%%%%%%%%%%%%%%%%%%%%%%%%%%%%%%%%%%%%%%%%


\begin{slide}{Input entries}
\vspace*{-6ex}
The input entries on the command-line are those classes that are actually checked.
Many other classes may be referenced for class definitions or specifications - these are found on the classpath (or sourcepath or specspath).

\begin{itemize}
\item {\knalblue file names} - of java or specification files (relative to the current directory)
\item {\knalblue directories} - processes all java or specification files (relative to the current directory)
\item {\knalblue package} - (fully qualified name) - found on the classpath
\item {\knalblue class} - (fully qualified name) - found on the classpath
\item {\knalblue list} - (prefaced by {\green -list}) - a file containing input entries
\end{itemize}
\end{slide}

%%%%%%%%%%%%%%%%%%%%%%%%%%%%%%%%%%%%%%%%%%%%%%%%%%%%%%%%%%%%%%%%%%%%%%%%%%%%%

\begin{slide}{Specification files}
\vspace*{-6ex}
\begin{itemize}
\item Specifications may be added directly to .java files
\item Specifications may alternatively be added to specification files.
\begin{itemize}
\item No method bodies
\item No field initializers
\item Recommended suffix: {\blue .refines-java}
\item Recommend a {\blue refines} annotation (see documentation)
\item Must also be on the classpath
\end{itemize}
\end{itemize}
\end{slide}

%%%%%%%%%%%%%%%%%%%%%%%%%%%%%%%%%%%%%%%%%%%%%%%%%%%%%%%%%%%%%%%%%%%%%%%%%%%%%

\begin{slide}{Specification file example}
\vspace*{-6ex}
\tiny
\begin{verbatim}
package java.lang;
import java.lang.reflect.*;
import java.io.InputStream;

public final class Class implements java.io.Serializable {

    private Class();

    /*@ also public normal_behavior
      @   ensures \result != null && !\result.equals("")
      @        && (* \result is the name of this class object *);
      @*/
    public /*@ pure @*/ String toString();

    ....


\end{verbatim}
\end{slide}

%%%%%%%%%%%%%%%%%%%%%%%%%%%%%%%%%%%%%%%%%%%%%%%%%%%%%%%%%%%%%%%%%%%%%%%%%%%%%

\begin{slide}{}

\vspace*{0ex}
\begin{center}{\large \red ESC/Java2's static checking warnings}\end{center}
\end{slide}


%%%%%%%%%%%%%%%%%%%%%%%%%%%%%%%%%%%%%%%%%%%%%%%%%%%%%%%%%%%%%%%%%%%%%%%%%%%%%


\begin{slide}{Types of ESC/Java2 warnings}
\vspace*{-6ex}

ESC/Java2 warnings fall into various categories:
\begin{itemize}
\item[] - warnings about possible {\knalblue runtime exceptions}:
(Cast, Null, NegSize, IndexTooBig, IndexNegative, ZeroDiv, ArrayStore)
\item[]
\begin{itemize}
\item These are the most common runtime exceptions caused by coding problems (that is, not by 
explicitly throwing an exception)
\item They do not include nearly all of the possible runtime exceptions
\item Most of the others are explicitly thrown by various library methods
\end{itemize}
\end{itemize}
\end{slide}


%%%%%%%%%%%%%%%%%%%%%%%%%%%%%%%%%%%%%%%%%%%%%%%%%%%%%%%%%%%%%%%%%%%%%%%%%%%%%
\begin{slide}{Cast Warning}
\vspace*{-9ex}

The Cast warning occurs when ESC/Java2 cannot verify that a ClassCastException will not be thrown:
\begin{figure*}
\tiny
\verbatiminput{examples/CastWarning.java}
\end{figure*}
results in
{\tiny
\begin{verbatim}
------------------------------------------------------------------------
CastWarning.java:3: Warning: Possible type cast error (Cast)
    String s = (String)o;
               ^
------------------------------------------------------------------------
\end{verbatim}
}
But this is OK:
\begin{figure*}
\tiny
\verbatiminput{examples/CastWarningOK.java}
\end{figure*}

\end{slide}

\begin{slide}{Cast Warning}
So is this:
\begin{figure*}
\tiny
\verbatiminput{examples/CastWarningOK2.java}
\end{figure*}

\end{slide}

%%%%%%%%%%%%%%%%%%%%%%%%%%%%%%%%%%%%%%%%%%%%%%%%%%%%%%%%%%%%%%%%%%%%%%%%%%%%%
\begin{slide}{Null Warning}
\vspace*{-9ex}

The Null warning occurs when ESC/Java2 cannot verify that a NullPointerException will not be thrown:
{\tiny
\begin{figure*}
\verbatiminput{examples/NullWarning.java}
\end{figure*}
}
results in
{\tiny
\begin{verbatim}
------------------------------------------------------------------------
NullWarning.java:3: Warning: Possible null dereference (Null)
    int i = o.hashCode();
             ^
------------------------------------------------------------------------
\end{verbatim}
}

But this is OK:
{\tiny
\begin{figure*}
\verbatiminput{examples/NullWarningOK.java}
\end{figure*}
}


\end{slide}
%%%%%%%%%%%%%%%%%%%%%%%%%%%%%%%%%%%%%%%%%%%%%%%%%%%%%%%%%%%%%%%%%%%%%%%%%%%%%
\begin{slide}{ArrayStore Warning}
\tiny

\vspace*{-12ex}

The ArrayStore warning occurs when ESC/Java2 cannot verify that the assignment of an object 
to an array element will not result in an ArrayStoreException:
\vspace*{-1ex}
{\tiny
\begin{figure*}
\verbatiminput{examples/ArrayStoreWarning.java}
\end{figure*}
}
\vspace*{-2ex}
results in
\vspace*{-1ex}
{\tiny
\begin{verbatim}
------------------------------------------------------------------------
ArrayStoreWarning.java:4: Warning: Type of right-hand side possibly not 
a subtype of array element type (ArrayStore)
    s[0] = o;
         ^
------------------------------------------------------------------------
\end{verbatim}
}
\vspace*{-1ex}
But this is OK:
\vspace*{-1ex}
{\tiny
\begin{figure*}
\verbatiminput{examples/ArrayStoreWarningOK.java}
\end{figure*}
}
\end{slide}
%%%%%%%%%%%%%%%%%%%%%%%%%%%%%%%%%%%%%%%%%%%%%%%%%%%%%%%%%%%%%%%%%%%%%%%%%%%%%
\begin{slide}{ZeroDiv, index Warnings}
\vspace*{-6ex}

\begin{itemize}
\item {\knalblue ZeroDiv} - issued when a denominator (integer division) may be 0
\item {\knalblue NegSize} - issued when the array size in an array allocation expression may be negative
\item {\knalblue IndexNegative} - issued when an array index may be negative
\item {\knalblue IndexTooBig} - issued when an array index may be greater than or equal to the array length
\end{itemize}
{\tiny
\begin{figure*}
\verbatiminput{examples/Index.java}
\end{figure*}
}
\end{slide}

%%%%%%%%%%%%%%%%%%%%%%%%%%%%%%%%%%%%%%%%%%%%%%%%%%%%%%%%%%%%%%%%%%%%%%%%%%%%%

\begin{slide}{Types of ESC/Java2 warnings}
\vspace*{-6ex}

ESC/Java2 warnings fall into various categories:
\begin{itemize}
\item[] - {\gray warnings about possible { runtime exceptions}:
(Cast, Null, NegSize, IndexTooBig, IndexNegative, ZeroDiv, ArrayStore)}

\item[] - warnings about possible method {\knalblue specification violations}:
(Precondition, Postcondition, Modifies)

\item[]
\begin{itemize}
\item These are all caused by violations of explicit user-written method specifications
\end{itemize}

\end{itemize}
\end{slide}
%%%%%%%%%%%%%%%%%%%%%%%%%%%%%%%%%%%%%%%%%%%%%%%%%%%%%%%%%%%%%%%%%%%%%%%%%%%%%
\begin{slide}{Pre, Post warnings}
\vspace*{-6ex}
These warnings occur in response to user-written preconditions (requires), postconditions (ensures, signals), or assert statements.


\begin{figure*}
\tiny
\verbatiminput{examples/PrePost.java}
\end{figure*}
\end{slide}

%%%%%%%%%%%%%%%%%%%%%%%%%%%%%%%%%%%%%%%%%%%%%%%%%%%%%%%%%%%%%%%%%%%%%%%%%%%%%
\begin{slide}{Frame conditions}

\vspace*{-8ex}

\begin{itemize}
\item To reason (modularly) about a call of a method, one must know what that method might modify: this is specified by 
\begin{itemize}
\item {\knalblue assignable} clauses
{\tiny

\begin{verbatim}
//@ assignable x, o.x, this.*, o.*, a[*], a[3], a[4..5];
\end{verbatim}
}
\item {\knalblue modifies} clauses (a synonym)
\item {\knalblue pure} modifier
{\tiny

\begin{verbatim}
//@ pure
public int getX() { return x; }
\end{verbatim}
}

\end{itemize}
\item Assignable clauses state what fields may be assigned within a method
- this is the set of what might be modified
\item The default assignable clause is \\ {\knalblue assignable \bsl everything;} (but it is better to be explicit because {\knalblue \bsl everything;} is not fully implemented and ESC/Java2
can reason better with more explicit frame conditions)
\item A {\knalblue pure} method is  {\knalblue assignable \bsl nothing;}
\end{itemize}
\end{slide}

\begin{slide}{Frame conditions}

\vspace*{-6ex}

\begin{itemize}
\item A {\knalblue Modifies} warning indicates an attempt to assign to an object field that is not in a modifies clause
\item Note: Some violations of modifies clauses can be detected at typecheck time.
\item Note also:  Handling of frame conditions is an active area of research.
\end{itemize}
\end{slide}


%%%%%%%%%%%%%%%%%%%%%%%%%%%%%%%%%%%%%%%%%%%%%%%%%%%%%%%%%%%%%%%%%%%%%%%%%%%%%
\begin{slide}{Modifies warnings}
\vspace*{-9ex}
For example, in 

\begin{figure*}
\tiny
\verbatiminput{examples/ModifiesWarning.java}
\end{figure*}
we don't know if {\blue o} equals {\blue this}; since only {\blue this.i} may be assigned,
 ESC/Java2 produces
\tiny
\begin{verbatim}
------------------------------------------------------------------------
ModifiesWarning.java:7: Warning: Possible violation of modifies clause (Modifies)
    o.i = 2; // Modifies warning
    ^
Associated declaration is "ModifiesWarning.java", line 4, col 6:
  //@ assignable i;
      ^
------------------------------------------------------------------------
\end{verbatim}

\end{slide}

%%%%%%%%%%%%%%%%%%%%%%%%%%%%%%%%%%%%%%%%%%%%%%%%%%%%%%%%%%%%%%%%%%%%%%%%%%%%%

\begin{slide}{Types of ESC/Java2 warnings}
\vspace*{-6ex}

ESC/Java2 warnings fall into various categories:
\begin{itemize}
\item[] - {\gray warnings about possible { runtime exceptions}:
(Cast, Null, NegSize, IndexTooBig, IndexNegative, ZeroDiv, ArrayStore)}

\item[] - {\gray warnings about possible { specification violations}:
(Precondition, Postcondition, Modifies)}

\item[] - {\knalblue non null} violations (NonNull, NonNullInit)

\item[]
\begin{itemize}
\item These warnings relate to explicit {\blue non\_null} field or parameter specifications
\end{itemize}

\end{itemize}
\end{slide}
%%%%%%%%%%%%%%%%%%%%%%%%%%%%%%%%%%%%%%%%%%%%%%%%%%%%%%%%%%%%%%%%%%%%%%%%%%%%%
\begin{slide}{NonNullInit warning}
\vspace*{-6ex}
Class fields declared {\knalblue non\_null} must be initialized to values that are not null in each constructor, else a NonNullInit warning is produced.
\begin{figure*}
\tiny

\verbatiminput{examples/NonNullInit.java}
\end{figure*}
produces
\tiny

\begin{verbatim}
------------------------------------------------------------------------
NonNullInit.java:4: Warning: Field declared non_null possibly 
not initialized (NonNullInit)
  public NonNullInit() { }
                         ^
Associated declaration is "NonNullInit.java", line 2, col 6:
  /*@ non_null */ Object o;
      ^
------------------------------------------------------------------------
\end{verbatim}
\end{slide}

%%%%%%%%%%%%%%%%%%%%%%%%%%%%%%%%%%%%%%%%%%%%%%%%%%%%%%%%%%%%%%%%%%%%%%%%%%%%%
\begin{slide}{NonNull warning}
\vspace*{-6ex}
A NonNull warning is produced whenever an assignment is made to a field or variable that has been declared non\_null but ESC/Java2 cannot determine that the right-hand-side value is not null.
\begin{figure*}
\tiny

\verbatiminput{examples/NonNull.java}
\end{figure*}
produces
\tiny

\begin{verbatim}
------------------------------------------------------------------------
NonNull.java:4: Warning: Possible assignment of null to variable 
declared non_null (NonNull)
  public void m(Object oo) { o = oo; } // NonNull warning
                               ^
Associated declaration is "NonNull.java", line 2, col 6:
  /*@ non_null */ Object o;
      ^
------------------------------------------------------------------------
\end{verbatim}
\end{slide}

%%%%%%%%%%%%%%%%%%%%%%%%%%%%%%%%%%%%%%%%%%%%%%%%%%%%%%%%%%%%%%%%%%%%%%%%%%%%%
\begin{slide}{NonNull warning}
\vspace*{-9ex}
But this is OK
\begin{figure*}
\tiny

\verbatiminput{examples/NonNullOK.java}
\end{figure*}
So is this
\begin{figure*}
\tiny

\verbatiminput{examples/NonNullOK2.java}
\end{figure*}
So is this
\begin{figure*}
\tiny

\verbatiminput{examples/NonNullOK3.java}
\end{figure*}

\begin{picture}(150,150)(-200,-220)
\green
\put(0,0){\shortstack[l]{
{\knalblue non\_null} can be applied to\\
- a field\\
- a formal parameter\\
- a return value\\
- a local variable\\
- ghost and model variables
}}
\end{picture}
\end{slide}

%%%%%%%%%%%%%%%%%%%%%%%%%%%%%%%%%%%%%%%%%%%%%%%%%%%%%%%%%%%%%%%%%%%%%%%%%%%%%

\begin{slide}{Types of ESC/Java2 warnings}
\vspace*{-6ex}

ESC/Java2 warnings fall into various categories:
\begin{itemize}
\item[] - {\gray warnings about possible { runtime exceptions}:
(Cast, Null, NegSize, IndexTooBig, IndexNegative, ZeroDiv, ArrayStore)}

\item[] - {\gray warnings about possible method { specification violations}:
(Precondition, Postcondition, Modifies)}

\item[] - {\gray non null violations (NonNull, NonNullInit)}

\item[] - {\knalblue loop} and {\knalblue flow} specifications (Assert, Reachable, LoopInv, DecreasesBound)

\item[]
\begin{itemize}
\item These are caused by violations of specifications in a routine body
\end{itemize}

\end{itemize}
\end{slide}
%%%%%%%%%%%%%%%%%%%%%%%%%%%%%%%%%%%%%%%%%%%%%%%%%%%%%%%%%%%%%%%%%%%%%%%%%%%%%
\begin{slide}{Body assertions}
\scriptsize
\vspace*{-10ex}

\begin{itemize}
\item {\knalblue Assert}: warning occurs when an {\blue assert} annotation may not be satisfied
\item {\knalblue Reachable}: not in JML, only in ESC/Java2; occurs with the
{\blue //@ unreachable;}
annotation, which is equivalent to\\ {\blue //@ assert false; }
\end{itemize}
\tiny
Example:
\begin{figure*}
\verbatiminput{examples/AssertWarning.java}
\end{figure*}
\end{slide}

%%%%%%%%%%%%%%%%%%%%%%%%%%%%%%%%%%%%%%%%%%%%%%%%%%%%%%%%%%%%%%%%%%%%%%%%%%%%%
\begin{slide}{Loop assertions}
\scriptsize
\vspace*{-10ex}

\begin{itemize}
\item A {\knalblue loop\_invariant} assertion just before a loop asserts a predicate that is true prior to each iteration and at the termination of the loop (or a {\knalblue LoopInv} warning is issued).
\item A {\knalblue decreases} assertion just before a loop asserts a (int) quantity that is non-negative and decreases with each iteration (or a {\knalblue DecreasesBound} warning is issued).
\item {\red Caution: Loops are checked by unrolling a few times}.

\end{itemize}

{
\tiny
Example:
\begin{figure*}
\verbatiminput{examples/LoopInvWarning.java}
\end{figure*}
}

\vspace*{-40ex}
\begin{picture}(100,100)(-82,75)
\thicklines
\red
\put(58,42){\oval(8,8)}
\put(152,53){\oval(8,8)}
\put(82,5){In the scope of the loop variable}
\put(85,15){\vector(-1,1){23}}
\put(97,15){\vector(3,2){52}}
\end{picture}

\end{slide}

%%%%%%%%%%%%%%%%%%%%%%%%%%%%%%%%%%%%%%%%%%%%%%%%%%%%%%%%%%%%%%%%%%%%%%%%%%%%%

\begin{slide}{Types of ESC/Java2 warnings}
\vspace*{-6ex}

ESC/Java2 warnings fall into various categories:
\begin{itemize}
\item[] - {\gray warnings about possible { runtime exceptions}:
(Cast, Null, NegSize, IndexTooBig, IndexNegative, ZeroDiv, ArrayStore)}


\item[] - {\gray warnings about possible method { specification violations}:
(Precondition, Postcondition, Modifies)}

\item[] - {\gray non null violations (NonNull, NonNullInit)}

\item[] - {\gray loop and flow specifications (Assert, Reachable, LoopInv, DecreasesBound)}

\item[] - { warnings about possible class specification violations}:
(Invariant, Constraint, Initially)

\end{itemize}
\end{slide}
%%%%%%%%%%%%%%%%%%%%%%%%%%%%%%%%%%%%%%%%%%%%%%%%%%%%%%%%%%%%%%%%%%%%%%%%%%%%%
\begin{slide}{class invariant warnings}


\vspace*{-6ex}
Invariant and constraint clauses generate additional postconditions for every method.
If they do not hold, appropriate warnings are generated:

\begin{figure*}
\tiny
\verbatiminput{examples/Invariant.java}
\end{figure*}
\end{slide}

%%%%%%%%%%%%%%%%%%%%%%%%%%%%%%%%%%%%%%%%%%%%%%%%%%%%%%%%%%%%%%%%%%%%%%%%%%%%%
\begin{slide}{Initially warning}
\vspace*{-6ex}
An initially clause is a postcondition for every constructor:

\begin{figure*}
\tiny

\verbatiminput{examples/Initially.java}
\end{figure*}
produces
\tiny

\begin{verbatim}
------------------------------------------------------------------------
Initially.java:5: Warning: Possible violation of initially condition 
at constructor exit (Initially)
  public Initially() { } // does not set i - Initially warning
                       ^
Associated declaration is "Initially.java", line 3, col 20:
  public int i; //@ initially i == 1;
                    ^
------------------------------------------------------------------------
\end{verbatim}
\end{slide}

%%%%%%%%%%%%%%%%%%%%%%%%%%%%%%%%%%%%%%%%%%%%%%%%%%%%%%%%%%%%%%%%%%%%%%%%%%%%%

\begin{slide}{Types of ESC/Java2 warnings}
\vspace*{-6ex}

ESC/Java2 warnings fall into various categories:
\begin{itemize}
\item[] - {\gray warnings about possible { runtime exceptions}:
(Cast, Null, NegSize, IndexTooBig, IndexNegative, ZeroDiv, ArrayStore)}

\item[] - {\gray warnings about possible method { specification violations}:
(Precondition, Postcondition, Modifies)}

\item[] - {\gray non null violations (NonNull, NonNullInit)}

\item[] - {\gray loop and flow specifications (Assert, Reachable, LoopInv, DecreasesBound)}

\item[] - { \gray warnings about possible class specification violations:
(Invariant, Constraint, Initially)}


\item[] - {exception problems (Exception)}
\item[]
\begin{itemize}
\item These warnings are caused by undeclared exceptions
\end{itemize}

\end{itemize}
\end{slide}

%%%%%%%%%%%%%%%%%%%%%%%%%%%%%%%%%%%%%%%%%%%%%%%%%%%%%%%%%%%%%%%%%%%%%%%%%%%%%
\begin{slide}{Exceptions - Errors}

\vspace*{-6ex}
\begin{itemize}
\item Java {\knalblue Errors} (e.g. OutOfMemoryError) can be thrown at any time
\begin{itemize}
\item No declarations are needed in throws clauses
\item No semantics are implied by JML
\item No checking is performed by ESC/Java2
\end{itemize}
\end{itemize}

\begin{picture}(250,120)(0,0)
\thicklines
\red
\put(90,90){Throwable}
\put(118,88){\vector(-4,-1){75}}
\put(20,60){Error}
\put(118,88){\vector(4,-1){75}}
\put(180,60){Exception}
\put(206,57){\vector(-3,-1){53}}
\put(206,57){\vector(3,-1){53}}
\put(100,30){\it checked exceptions}
\put(220,30){RuntimeException}
\put(270,28){\vector(0,-1){20}}
\put(215,0){\it unchecked exceptions}
\end{picture}

\end{slide}

%%%%%%%%%%%%%%%%%%%%%%%%%%%%%%%%%%%%%%%%%%%%%%%%%%%%%%%%%%%%%%%%%%%%%%%%%%%%%
\begin{slide}{Checked Exceptions}
\vspace*{-6ex}
\begin{itemize}
\item Java {\knalblue checked} exceptions (e.g. FileNotFoundException) are Exceptions that are not RuntimeExceptions:
\begin{itemize}
\item Declarations of exceptions mentioned in the body are required in throws clauses
\item ESC/Java2 checks during typechecking that throws declarations are correct (as a Java compiler does)
\item Typically specified in signals clauses in JML
\item ESC/Java2 checks via reasoning that signals conditions hold
\item Default specification is that {\it declared} exceptions may occur:  signals (Exception) true;
\item ESC/Java2 presumes that checked exceptions not declared in a throws clause will not occur.
\end{itemize}
\end{itemize}
\end{slide}

%%%%%%%%%%%%%%%%%%%%%%%%%%%%%%%%%%%%%%%%%%%%%%%%%%%%%%%%%%%%%%%%%%%%%%%%%%%%%
\begin{slide}{Unchecked Exceptions}

\vspace*{-6ex}
\begin{itemize}
\item Java {\knalblue unchecked} exceptions (e.g. NoSuchElementException) are RuntimeExceptions:
\begin{itemize}
\item Java does not require these to be declared in throws clauses
\item ESC/Java2 is stricter than Java - it will issue an Exception warning if an unchecked exception might be {\it explicitly} thrown but is not declared in a throws declaration 
\item Caution: currently ESC/Java2 will assume that an undeclared unchecked exception will not be thrown, even if it is specified in a signals clause - 
\begin{itemize}
\item[] Declare all unchecked exceptions that might be thrown! 
\item[] (e.g. especially when there is no implementation to check).
\end{itemize}
\end{itemize}

\end{itemize}
\end{slide}


%%%%%%%%%%%%%%%%%%%%%%%%%%%%%%%%%%%%%%%%%%%%%%%%%%%%%%%%%%%%%%%%%%%%%%%%%%%%%

\begin{slide}{Exception warning}
\vspace*{-12ex}
So this
\begin{figure*}
\tiny

\verbatiminput{examples/Ex.java}
\end{figure*}
produces
{\tiny

\begin{verbatim}
------------------------------------------------------------------------
Ex.java:4: Warning: Possible unexpected exception (Exception)
  }
  ^
Execution trace information:
    Executed then branch in "Ex.java", line 3, col 32.
    Executed throw in "Ex.java", line 3, col 32.

------------------------------------------------------------------------
\end{verbatim}
}
Turn off this warning by
\begin{itemize}
\item declaring the exception in a throws clause
\item using {\knalblue //@ nowarn Exception; } on the offending line
\item using a {\knalblue -nowarn Exception} command-line option
\end{itemize}
\end{slide}


%%%%%%%%%%%%%%%%%%%%%%%%%%%%%%%%%%%%%%%%%%%%%%%%%%%%%%%%%%%%%%%%%%%%%%%%%%%%%

\begin{slide}{Types of ESC/Java2 warnings}
\vspace*{-9ex}

ESC/Java2 warnings fall into various categories:
\begin{itemize}

\item[] - {\gray warnings about possible { runtime exceptions}:
(Cast, Null, NegSize, IndexTooBig, IndexNegative, ZeroDiv, ArrayStore)}

\item[] - {\gray warnings about possible method { specification violations}:
(Precondition, Postcondition, Modifies)}

\item[] - {\gray non null violations (NonNull, NonNullInit)}

\item[] - {\gray loop and flow specifications (Assert, Reachable, LoopInv, DecreasesBound)}

\item[] - { \gray warnings about possible class specification violations:
(Invariant, Constraint, Initially)}


\item[] - {\gray exception problems (Exception)}

\item[] - {\knalblue multithreading} (Race, RaceAllNull, Deadlock)
\item[]
\begin{itemize}
\item These warnings are caused by potential problems with monitors
\item Multithreading problems caused by the absence of any synchronization are not detected.
\end{itemize}

\end{itemize}
\end{slide}
%%%%%%%%%%%%%%%%%%%%%%%%%%%%%%%%%%%%%%%%%%%%%%%%%%%%%%%%%%%%%%%%%%%%%%%%%%%%%
\begin{slide}{Race conditions}

\vspace*{-6ex}

\begin{itemize}
\item Java defines monitors associated with any object and allows critical sections to be guarded by synchronization statements.
\item ESC/Java permits fields to be declared as {\blue monitored} by one or more objects.
\item To read a monitored field, at least one monitor must be held (or a Race warning is issued).
\item To write a monitored field, all non-null monitors must be held (or a Race warning is issued).
\item To write a monitored field, at least one of its monitors must be non-null (or a RaceAllNull warning is issued).
\end{itemize}
\end{slide}

%%%%%%%%%%%%%%%%%%%%%%%%%%%%%%%%%%%%%%%%%%%%%%%%%%%%%%%%%%%%%%%%%%%%%%%%%%%%%

\begin{slide}{Race warnings}
\vspace*{-6ex}
For example, 

\begin{figure*}
\tiny
\verbatiminput{examples/RaceWarning.java}
\end{figure*}
produces
\tiny
\begin{verbatim}
------------------------------------------------------------------------
RaceWarning.java:6: Warning: Possible race condition (Race)
    i = 0; // should have a synchronization guard
      ^
Associated declaration is "RaceWarning.java", line 2, col 6:
  //@ monitored
      ^
------------------------------------------------------------------------
\end{verbatim}
\end{slide}

%%%%%%%%%%%%%%%%%%%%%%%%%%%%%%%%%%%%%%%%%%%%%%%%%%%%%%%%%%%%%%%%%%%%%%%%%%%%%
\begin{slide}{Deadlocks}

\vspace*{-6ex}

\begin{itemize}
\item Deadlocks occur when each thread of a group of threads needs monitors held by another thread in the group.
\item One way to avoid this is to always acquire monitors in a specific order.
\item This requires
\begin{itemize}
\item the user state a (partial) order for monitors (typically using an axiom)
\item that it be clear before acquiring a monitor that the thread does not hold any `larger' monitors (typically a precondition)
\end{itemize}
\item Checking for Deadlock warnings is off by default but can be turned on with {\knalblue -warn Deadlock}.
\end{itemize}
\end{slide}

%%%%%%%%%%%%%%%%%%%%%%%%%%%%%%%%%%%%%%%%%%%%%%%%%%%%%%%%%%%%%%%%%%%%%%%%%%%%%

\begin{slide}{Deadlock warnings}
\vspace*{-6ex}
For example:

\begin{figure*}
\tiny
\verbatiminput{examples/DeadlockWarning.java}
\end{figure*}

\begin{picture}(100,100)(0,0)
\thicklines
\red
\put(120,157){\oval(200,10)}
\end{picture}

\end{slide}

%%%%%%%%%%%%%%%%%%%%%%%%%%%%%%%%%%%%%%%%%%%%%%%%%%%%%%%%%%%%%%%%%%%%%%%%%%%%%

\begin{slide}{Types of ESC/Java2 warnings}
\vspace*{-6ex}

ESC/Java2 warnings fall into various categories:
\begin{itemize}

\item[] - {\gray warnings about possible { runtime exceptions}:
(Cast, Null, NegSize, IndexTooBig, IndexNegative, ZeroDiv, ArrayStore)}

\item[] - {\gray warnings about possible method { specification violations}:
(Precondition, Postcondition, Modifies)}

\item[] - {\gray non null violations (NonNull, NonNullInit)}

\item[] - {\gray loop and flow specifications (Assert, Reachable, LoopInv, DecreasesBound)}

\item[] - { \gray warnings about possible class specification violations:
(Invariant, Constraint, Initially)}


\item[] - {\gray exception problems (Exception)}

\item[] - {\gray multithreading (Race, RaceAllNull, Deadlock)}

\item[] - a few others (OwnerNull, Uninit, Unreadable, Writable)

\end{itemize}
\end{slide}
%%%%%%%%%%%%%%%%%%%%%%%%%%%%%%%%%%%%%%%%%%%%%%%%%%%%%%%%%%%%%%%%%%%%%%%%%%%%%

\begin{slide}{Other warnings}

\vspace*{-6ex}
\begin{itemize}
\item {\knalblue Uninit}: used with the {\blue uninitialized} annotation
\item {\knalblue OwnerNull}: see the ESC/Java User Manual for a description
\item {\knalblue Unreadable}: occurs with the {\blue readable\_if} annotation on shared variables.
[ JML's change of syntax from {\blue readable\_if} to {\blue readable} is not complete in ESC/Java2.]
\item {\knalblue Writable}: occurs with the {\blue writable\_if} annotation on shared variables.
[ JML's change of syntax from {\blue writable\_if} to {\blue writable} is not complete in ESC/Java2.]
\end{itemize}
\end{slide}

%%%%%%%%%%%%%%%%%%%%%%%%%%%%%%%%%%%%%%%%%%%%%%%%%%%%%%%%%%%%%%%%%%%%%%%%%%%%%
\begin{slide}{trace information}
\vspace*{-9ex}
For complicated bodies, the warning messages give some information about which
if-then-else branches caused the warning:
\begin{figure*}
\tiny

\verbatiminput{examples/Trace.java}
\end{figure*}
\vspace*{-2ex}
produces
\tiny

\begin{verbatim}
Trace.java:8: Warning: Postcondition possibly not established (Post)
  }
  ^
Associated declaration is "Trace.java", line 2, col 6:
  //@ ensures \result > 0;
      ^
Execution trace information:
    Executed else branch in "Trace.java", line 4, col 4.
    Executed then branch in "Trace.java", line 5, col 16.
    Executed return in "Trace.java", line 5, col 16.
\end{verbatim}
\end{slide}
%%%%%%%%%%%%%%%%%%%%%%%%%%%%%%%%%%%%%%%%%%%%%%%%%%%%%%%%%%%%%%%%%%%%%%%%%%%%%

\begin{slide}{Counterexamples}
\vspace*{-6ex}
\begin{itemize}
\item Sometimes when a specification is found to be invalid, ESC/Java2 will produce a 
{\it counterexample context}.
\item A full context will be produced with the {\knalblue -counterexample} option
\item These are difficult to read, but can give information about the reason for failure.
\item They state formulae that the prover believes to be true; if there is something you think should
not be true, that is a hint about the problem.
\item[]
\item Note also: Typically only one warning will be issued in a given run.
\end{itemize}
\end{slide}

%%%%%%%%%%%%%%%%%%%%%%%%%%%%%%%%%%%%%%%%%%%%%%%%%%%%%%%%%%%%%%%%%%%%%%%%%%%%%

\begin{slide}{}

\vspace*{0ex}
\begin{center}{\large \red Specification tips and pitfalls}\end{center}
\end{slide}

%%%%%%%%%%%%%%%%%%%%%%%%%%%%%%%%%%%%%%%%%%%%%%%%%%%%%%%%%%%%%%%%%%%%%%%%%%%%%

\begin{slide}{\#1: Inherited specifications}
\begin{itemize}
\item Base class specifications apply to derived classes
\begin{itemize}
\item that is, ESC/Java2 enforces {\it behavioral subtyping}
\item Specs from implemented interfaces also must hold for implementing classes
\end{itemize}
\item Be thoughtful about how strict the base class specs should be
\item Guard them with {\blue \char'134 typeof(this) == \char'134 type(...) } if need be
\item Restrictions on exceptions such as {\blue normal\_behavior} or {\blue signals (E e) false; }
will apply to derived classes as well.
\end{itemize}
\end{slide}


%%%%%%%%%%%%%%%%%%%%%%%%%%%%%%%%%%%%%%%%%%%%%%%%%%%%%%%%%%%%%%%%%%%%%%%%%%%%%

\begin{slide}{\#1: Inherited specifications}
\vspace*{-7ex}
For example, in the code below
\begin{itemize}
\item Parent.m(i) satisfies $i \geq 0 \Rightarrow \result \geq i$
\item Child.m(i) satisfies \hspace{.5ex} $i \geq 0 \Rightarrow \result \geq i$ 
\\{\it and} \hspace{14.5ex}  $i \leq 0 \Rightarrow \result \leq i$
\\so Child.m(0) must be 0.
\item[]
\end{itemize}
{\tiny
\begin{verbatim}
  class Parent {
     //@ requires i >= 0;
     //@ ensures  \result >= i;
     int m(int i);
  }

  class Child extends Parent {
     //@ also
     //@   requires i <= 0
     //@   ensures  \result <= i;
     int m(int i);
 }
\end{verbatim}
}
\vspace*{-20ex}
\begin{picture}(100,100)(0,0)
\thicklines
\red
\put(51,35){\oval(25,8)}
\put(102,-10){Indicates there are inherited specs}
\put(100,-3){\vector(-1,1){35}}
\end{picture}

\vspace*{-20ex}
\begin{picture}(100,100)(0,0)
\thicklines
\green
\put(200,30){\shortstack[l]{
Note: In \\
Parent p = new Parent(); \\
Parent pc = new Child(); \\
Child c = new Child();\\
p.m(i); // i must be >= 0\\
pc.m(i); // i must be >= 0\\
c.m(i); // i must be >= 0 or <= 0
}}
\end{picture}

\end{slide}

\begin{slide}{\#1: Inherited specifications}
Another example: two Objects that are == are always also {\knalblue equals}.  But the converse is not
necessarily true.  But it is true for objects whose dynamic type is Object.

\vspace*{3ex}

\begin{verbatim}
public class Object {
  //@ ensures (this == o) ==> \result;
  /*@ ensures \typeof(this) == \type(Object) 
              ==> (\result == (this==o));
  */
  public boolean equals(Object o);
}
\end{verbatim}
\vspace*{-10ex}
\begin{picture}(100,100)(0,0)
\thicklines
\red
\put(0,0){True for all Objects}
\put(58,10){\vector(1,3){32}}

\put(150,10){Not necessarily true for subtypes}
\put(200,20){\vector(-1,3){20}}
\end{picture}

\end{slide}


%%%%%%%%%%%%%%%%%%%%%%%%%%%%%%%%%%%%%%%%%%%%%%%%%%%%%%%%%%%%%%%%%%%%%%%%%%%%%

\begin{slide}{\#2: Specifying exceptions}
\vspace*{-8ex}
\begin{itemize}
\item A {\blue signals} clause such as \\
{\blue signals (FileNotFoundException e) true;}\\
states what must be true if an exception of the stated type is thrown.
\item It does not say what other exception types may or may not be thrown.
\item To forbid a particular exception, omit it from the Java throws clause or write \\
{\blue signals (FileNotFoundException e) false;}\\
\item To limit the set of allowed exceptions, use a postcondition such as 
\begin{verbatim}
/*@ signals (Exception e) e instanceof E1 
                       || e instanceof E2 
                       || ... ;
 */
\end{verbatim}
\item To forbid all exceptions use a {\knalblue normal\_behavior} or 
{\blue signals (Exception e) false;} specification - be careful not to overly restrict derived classes
\end{itemize}

\end{slide}


%%%%%%%%%%%%%%%%%%%%%%%%%%%%%%%%%%%%%%%%%%%%%%%%%%%%%%%%%%%%%%%%%%%%%%%%%%%%%

\begin{slide}{\#3: Aliasing}
\vspace*{-9ex}
A common but non-obvious problem that causes violated invariants is aliasing.
{\tiny
\begin{verbatim}
public class Alias {
  /*@ non_null */ int[] a = new int[10];
  boolean noneg = true;

  /*@ invariant noneg ==> 
                 (\forall int i; 0<=i && i < a.length;  a[i]>=0); */

  //@ requires 0<=i && i < a.length;
  public void insert(int i, int v) {
    a[i] = v;
    if (v < 0) noneg = false;
  }
}
\end{verbatim}  
}
produces
{\tiny
\begin{verbatim}
Alias.java:12: Warning: Possible violation of object invariant (Invariant)
  }
  ^
Associated declaration is "Alias.java", line 5, col 6:
  /*@ invariant (\forall int i; 0<=i && i < a.length; 
\end{verbatim}
}

\end{slide}
\begin{slide}{\#3: Aliasing}
\vspace*{-6ex}
A full counterexample context ({\blue -counterexample} option) produces, among lots of other information:
\begin{verbatim}

    brokenObj%0 != this
    (brokenObj%0).(a@pre:2.24) == tmp0!a:10.4
    this.(a@pre:2.24) == tmp0!a:10.4

\end{verbatim}
that is, {\blue this} and some different object ({\blue brokenObj}) share the \\ same  {\blue a} object. 


\begin{picture}(150,150)(0,0)
\red
\put(38,86){\framebox(55,27){}}
\put(113,86){\framebox(55,27){}}
\put(40,90){\shortstack[l]{int noneg \\ int[] a}}
\put(115,90){\shortstack[l]{int noneg \\ int[] a}}
\put(175,60){\framebox(75,7){ }}
\knalblue
\put(38,118){this}
\put(113,118){brokenObj}
\put(175,72){{\it an} int[] {\it object}}
\green
\put(75,92){\vector(4,-1){100}}
\put(150,92){\vector(1,-1){25}}
\put(75,92){\circle*{5}}
\put(150,92){\circle*{5}}
\end{picture}
\end{slide}
\begin{slide}{\#3: Aliasing}
\vspace*{-6ex}
To fix this, declare that {\blue a} is owned only by its parent object:\\
( {\blue owner} is a ghost field of java.lang.Object )
\tiny
\begin{verbatim}
public class Alias {
  /*@ non_null */ int[] a = new int[10];
  boolean noneg = true;

  /*@ invariant (\forall int i; 0<=i && i < a.length; 
                       noneg ==> (a[i]>=0)); */
  //@ invariant a.owner == this;

  //@ requires 0<=i && i < a.length;
  public void insert(int i, int v) {
    a[i] = v;
    if (v < 0) noneg = false;
  }

  public Alias() {
    //@ set a.owner = this;
  }
}
\end{verbatim}
\vspace*{-40ex}
\begin{picture}(100,100)(0,0)
\thicklines
\red
\put(80,60){\oval(150,10)}
\put(75,-40){\oval(110,10)}
\end{picture}

\begin{picture}(150,150)(-150,-80)
\red
\put(38,86){\framebox(45,25){\shortstack[l]{int noneg \\ int[] a}}}
\put(113,86){\framebox(45,25){\shortstack[l]{int noneg \\ int[] a}}}
\put(170,42){\framebox(45,25){\shortstack[l]{int[] ...... \\ owner}}}
\put(100,7){\framebox(45,25){\shortstack[l]{int[] ...... \\ owner}}}
\knalblue
\put(38,115){this}
\put(113,115){brokenObj}
\put(170,70){{\it an} int[] {\it object}}
\put(100,35){{\it an} int[] {\it object}}
\green
\put(70,92){\vector(1,-2){30}}
\put(145,92){\vector(1,-1){25}}
\put(70,92){\circle*{5}}
\put(145,92){\circle*{5}}
\put(208,50){\circle*{5}}
\put(138,15){\circle*{5}}
\put(208,50){\vector(-1,1){50}}
\put(138,15){\vector(-2,3){55}}
\end{picture}

\end{slide}

\begin{slide}{\#3: Aliasing}
\vspace*{-9ex}
Another example.  This one fails on the postcondition.
\tiny
\begin{verbatim}
public class Alias2 {
  /*@ non_null */ Inner n = new Inner();
  /*@ non_null */ Inner nn = new Inner();
  //@ invariant n.owner == this;
  //@ invariant nn.owner == this;

  //@ ensures n.i == \old(n.i + 1);
  public void add() {
    n.i++;
    nn.i++;
  }

  Alias2();
}

class Inner { 
  public int i; 
  //@ ensures i == 0;
  Inner();
}
\end{verbatim}
\vspace*{-40ex}
\begin{picture}(100,100)(0,0)
\thicklines
\red
\put(90,82){\oval(170,10)}
\end{picture}

\end{slide}

\begin{slide}{\#3: Aliasing}
\begin{itemize}
\item The counterexample context shows
\begin{verbatim}

    this.(nn:3.24) == tmp0!n:10.4
    tmp2!nn:11.4 == tmp0!n:10.4

\end{verbatim}
\item These hint that {\blue n} and {\blue nn} are references to the same object.

\item If we add the invariant {\blue //@ invariant n != nn; } to forbid aliasing between these two fields, then all is well.
\end{itemize}
\end{slide}

\begin{slide}{\#3: Aliasing}
\begin{itemize}

\item Aliasing is a serious difficulty in verification
\item Handling aliasing is an active area of research, related to handling frame conditions
\item It is all about knowing what is modified and what is not
\item These {\blue owner} fields or the equivalent create a form of encapsulation that can be 
checked by ESC/Java to control what might be modified by a given operation
\end{itemize}
\end{slide}


%%%%%%%%%%%%%%%%%%%%%%%%%%%%%%%%%%%%%%%%%%%%%%%%%%%%%%%%%%%%%%%%%%%%%%%%%%%%%

\begin{slide}{\#4: Write object invariants}
\vspace*{-6ex}
\begin{itemize}
\item Be sure that class invariants are about the object at hand.  
\item Statements about all objects of
a class may indeed be true, but they are difficult to prove, especially for automated provers.

\item For example, if a predicate P is supposed to hold for objects of type T, then do {\red not} write
\begin{verbatim}
//@ invariant (\forall T t; P(t));
\end{verbatim}

\item Instead, write
\begin{verbatim}
//@ invariant P(this);
\end{verbatim}

\item The latter will make a more provable postcondition at the end of a constructor.
\end{itemize}

\end{slide}

%%%%%%%%%%%%%%%%%%%%%%%%%%%%%%%%%%%%%%%%%%%%%%%%%%%%%%%%%%%%%%%%%%%%%%%%%%%%%

\begin{slide}{\#5: Inconsistent assumptions}
\vspace*{-3ex}
If you have inconsistent specifications you can prove anything:

\verbatimcode{examples/Inconsistent.java}
\end{slide}

\begin{slide}{\#5: Inconsistent assumptions}
\vspace*{-3ex}
Another example:

\verbatimcode{examples/Inconsistent2.java}

We hope to put in checks for this someday!

\end{slide}




%%%%%%%%%%%%%%%%%%%%%%%%%%%%%%%%%%%%%%%%%%%%%%%%%%%%%%%%%%%%%%%%%%%%%%%%%%%%%

\begin{slide}{\#6: Exposed references}
\vspace*{-9ex}
Problems can arise when a reference to an internal object is exported from a class:
\begin{figure*}
\tiny
\verbatiminput{examples/Exposed.java}
\end{figure*}
\vspace*{-3ex}
\begin{itemize}
\item ESC/Java does not check that {\it every} allocated object still satisfies its invariants.
\item Similar hidden problems can result if public fields are modified directly.
\end{itemize}

\begin{picture}(100,100)(0,0)
\thicklines
\red
\put(165,195){\oval(270,10)}
\end{picture}

\end{slide}


%%%%%%%%%%%%%%%%%%%%%%%%%%%%%%%%%%%%%%%%%%%%%%%%%%%%%%%%%%%%%%%%%%%%%%%%%%%%%

\begin{slide}{\#7: \old}
\vspace*{-6ex}
{\knalblue \old} is used to indicate evaluation in the pre-state in a postcondition expression.\\
\vspace{1ex}
Consider specifying
{\tiny
\begin{verbatim} 
public static native void arraycopy(Object[] src, int srcPos,
                                    Object[] dest, int destPos, int length);
\end{verbatim}
}
Try:
{\tiny
\begin{verbatim} 
ensures (\forall int i; 0<=i && i<length; dest[destPos+i] == src[srcPos+i]);
\end{verbatim}
}


\end{slide}


%%%%%%%%%%%%%%%%%%%%%%%%%%%%%%%%%%%%%%%%%%%%%%%%%%%%%%%%%%%%%%%%%%%%%%%%%%%%%

\begin{slide}{\#7: \old}
\vspace*{-6ex}
{\knalblue \old} is used to indicate evaluation in the pre-state in a postcondition expression.\\
\vspace{1ex}
Consider specifying
{\tiny
\begin{verbatim} 
public static native void arraycopy(Object[] src, int srcPos,
                                    Object[] dest, int destPos, int length);
\end{verbatim}
}
Try:
{\tiny
\begin{verbatim} 
ensures (\forall int i; 0<=i && i<length; dest[destPos+i] == src[srcPos+i]);
\end{verbatim}
}

{\red
Wrong!}\\

\end{slide}


%%%%%%%%%%%%%%%%%%%%%%%%%%%%%%%%%%%%%%%%%%%%%%%%%%%%%%%%%%%%%%%%%%%%%%%%%%%%%

\begin{slide}{\#7: \old}
\vspace*{-6ex}
{\knalblue \old} is used to indicate evaluation in the pre-state in a postcondition expression.\\
\vspace{1ex}
Consider specifying
{\tiny
\begin{verbatim} 
public static native void arraycopy(Object[] src, int srcPos,
                                    Object[] dest, int destPos, int length);
\end{verbatim}
}
Try:
{\tiny
\begin{verbatim} 
ensures (\forall int i; 0<=i && i<length; dest[destPos+i] == src[srcPos+i]);
\end{verbatim}
}

{\red
Wrong!}\\
\vspace{1ex}
Besides exceptions and invalid arguments, don't forget aliasing - {\blue dest} and {\blue src} may be the same array:
{\tiny
\begin{verbatim} 
ensures (\forall int i; 0<=i && i<length; 
                            dest[destPos+i] == \old(src[srcPos+i]);
\end{verbatim}
}

\end{slide}


%%%%%%%%%%%%%%%%%%%%%%%%%%%%%%%%%%%%%%%%%%%%%%%%%%%%%%%%%%%%%%%%%%%%%%%%%%%%%

\begin{slide}{\#7: \old}
\vspace*{-6ex}
{\knalblue \old} is used to indicate evaluation in the pre-state in a postcondition expression.\\
\vspace{1ex}
Consider specifying
{\tiny
\begin{verbatim} 
public static native void arraycopy(Object[] src, int srcPos,
                                    Object[] dest, int destPos, int length);
\end{verbatim}
}
Try:
{\tiny
\begin{verbatim} 
ensures (\forall int i; 0<=i && i<length; dest[destPos+i] == src[srcPos+i]);
\end{verbatim}
}

{\red
Wrong!}\\
\vspace{1ex}
Besides exceptions and invalid arguments, don't forget aliasing - {\blue dest} and {\blue src} may be the same array:
{\tiny
\begin{verbatim} 
ensures (\forall int i; 0<=i && i<length; 
                            dest[destPos+i] == \old(src[srcPos+i]);
\end{verbatim}
}
And don't forget the other elements:
{\tiny
\begin{verbatim} 
ensures (\forall int i; (0<=i && i<destPos) || 
                              (destPos+length <= i && i < destPos.length); 
                            dest[i] == \old(dest[i]);
\end{verbatim}
}



\end{slide}


%%%%%%%%%%%%%%%%%%%%%%%%%%%%%%%%%%%%%%%%%%%%%%%%%%%%%%%%%%%%%%%%%%%%%%%%%%%%%

\begin{slide}{Getting started}
\vspace*{-6ex}
\begin{itemize}
\item Start with foundation and library routines
\item For each field: should it be {\knalblue non\_null}?
\item For each reference field: should an {\knalblue owner} field be set for it?
\item For each method: should it be {\knalblue pure}?  Should the arguments or the result be {\knalblue non\_null}?
\item For each class: what {\knalblue invariant} expresses the self-consistency of the internal data?
\item Add {\knalblue pre-} and {\knalblue post-conditions} to limit the inputs and outputs of each method.
\item Add possible unchecked {\knalblue exceptions} to throws clauses.
\item Start with simple specifications; proceed to complex ones as they have value.
\end{itemize}

\end{slide}

\begin{slide}{Getting started}
\begin{itemize}
\item Separate conjunctions to get information about which conjunct is violated. Use
{\tiny \begin{verbatim}
requires A;
requires B;
\end{verbatim}
}
not
{\tiny \begin{verbatim}
requires A && B;
\end{verbatim}
}

\item Use {\knalblue assert} statements to find out what is going wrong.  Use {\knalblue assume} statements {\red \it that you KNOW are correct} to help the prover along.
\end{itemize}

\end{slide}



%%%%%%%%%%%%%%%%%%%%%%%%%%%%%%%%%%%%%%%%%%%%%%%%%%%%%%%%%%%%%%%%%%%%%%%%%%%%%

\begin{slide}{\it Finally}
\begin{itemize}

\item Specification is {\green tricky} - getting it right is hard, even with tools

\item {\green Try it} - a substantial research gap is experience on industrial-scale sets of code

\item {\green Communicate} - we are willing to offer advice

\item {\green Share} your experience - tools will get better and we will all learn better techniques for 
successful specification (use JML and ESC/Java mailing lists)


\end{itemize}

\end{slide}

%%%%%%%%%%%%%%%%%%%%%%%%%%%%%%%%%%%%%%%%%%%%%%%%%%%%%%%%%%%%%%%%%%%%%%%%%%%%%

\end{document}
